\PassOptionsToPackage{unicode=true}{hyperref} % options for packages loaded elsewhere
\PassOptionsToPackage{hyphens}{url}
%
\documentclass[oneside,11pt,spanish,]{extbook} % cjns1989 - 27112019 - added the oneside option: so that the text jumps left & right when reading on a tablet/ereader
\usepackage{lmodern}
\usepackage{amssymb,amsmath}
\usepackage{ifxetex,ifluatex}
\usepackage{fixltx2e} % provides \textsubscript
\ifnum 0\ifxetex 1\fi\ifluatex 1\fi=0 % if pdftex
  \usepackage[T1]{fontenc}
  \usepackage[utf8]{inputenc}
  \usepackage{textcomp} % provides euro and other symbols
\else % if luatex or xelatex
  \usepackage{unicode-math}
  \defaultfontfeatures{Ligatures=TeX,Scale=MatchLowercase}
%   \setmainfont[]{EBGaramond-Regular}
    \setmainfont[Numbers={OldStyle,Proportional}]{EBGaramond-Regular}      % cjns1989 - 20191129 - old style numbers 
\fi
% use upquote if available, for straight quotes in verbatim environments
\IfFileExists{upquote.sty}{\usepackage{upquote}}{}
% use microtype if available
\IfFileExists{microtype.sty}{%
\usepackage[]{microtype}
\UseMicrotypeSet[protrusion]{basicmath} % disable protrusion for tt fonts
}{}
\usepackage{hyperref}
\hypersetup{
            pdftitle={CARLOS VI EN LA RÁPITA},
            pdfauthor={Benito Pérez Galdós},
            pdfborder={0 0 0},
            breaklinks=true}
\urlstyle{same}  % don't use monospace font for urls
\usepackage[papersize={4.80 in, 6.40  in},left=.5 in,right=.5 in]{geometry}
\setlength{\emergencystretch}{3em}  % prevent overfull lines
\providecommand{\tightlist}{%
  \setlength{\itemsep}{0pt}\setlength{\parskip}{0pt}}
\setcounter{secnumdepth}{0}

% set default figure placement to htbp
\makeatletter
\def\fps@figure{htbp}
\makeatother

\usepackage{ragged2e}
\usepackage{epigraph}
\renewcommand{\textflush}{flushepinormal}

\usepackage{indentfirst}

\usepackage{fancyhdr}
\pagestyle{fancy}
\fancyhf{}
\fancyhead[R]{\thepage}
\renewcommand{\headrulewidth}{0pt}
\usepackage{quoting}
\usepackage{ragged2e}

\newlength\mylen
\settowidth\mylen{...................}

\usepackage{stackengine}
\usepackage{graphicx}
\def\asterism{\par\vspace{1em}{\centering\scalebox{.9}{%
  \stackon[-0.6pt]{\bfseries*~*}{\bfseries*}}\par}\vspace{.8em}\par}

 \usepackage{titlesec}
 \titleformat{\chapter}[display]
  {\normalfont\bfseries\filcenter}{}{0pt}{\Large}
 \titleformat{\section}[display]
  {\normalfont\bfseries\filcenter}{}{0pt}{\Large}
 \titleformat{\subsection}[display]
  {\normalfont\bfseries\filcenter}{}{0pt}{\Large}

\setcounter{secnumdepth}{1}
\ifnum 0\ifxetex 1\fi\ifluatex 1\fi=0 % if pdftex
  \usepackage[shorthands=off,main=spanish]{babel}
\else
  % load polyglossia as late as possible as it *could* call bidi if RTL lang (e.g. Hebrew or Arabic)
%   \usepackage{polyglossia}
%   \setmainlanguage[]{spanish}
%   \usepackage[french]{babel} % cjns1989 - 1.43 version of polyglossia on this system does not allow disabling the autospacing feature
\fi

\title{CARLOS VI EN LA RÁPITA}
\author{Benito Pérez Galdós}
\date{}

\begin{document}
\maketitle

\hypertarget{i}{%
\chapter{I}\label{i}}

\begin{flushright}
\textbf{Tetuán, mes de Adar, año 5620.}
\end{flushright}

¡Vive Dios, que no sé ya cómo me llamo! \emph{Yahia} dicen los del
\emph{Mellah} al verme; Alarcón me saluda con apodos burlescos,
\emph{Profetángano}, \emph{Don Bíblico}; para algunos moros maleantes
soy \emph{Djinn}, que quiere decir \emph{diablillo}, \emph{geniecillo};
y mi venerable amigo el castrense don \emph{Toro Godo} me ha puesto el
remoquete de \emph{Confusio} (con \emph{ese}). Cuando me recojo en mí, y
examino y desdoblo mi personalidad, ahora tan envuelta sobre sí propia,
vengo a reconocer que soy aquel Juan que vino de España con el Ejército
de O'Donnell, trayendo consigo poco más de lo puesto, un humilde y no
manchado apellido, que creo era Santiuste, y una condición que tengo por
sencilla y mansa, la cual, dividida en cuartos, me da tres partes de
galán enamoradizo y un cuartillo de poeta. Tal soy, tal fui. Quiero
reconstruir mi ser sintético, y fundar en él la nueva conciencia que
necesito al cabo de tantos trastornos, en ésta mi africana vida tan
atropellada y exuberante.

Si apenas sé cómo me llamo, tampoco me doy clara cuenta de la religión
que profeso, pues las tres que aquí tenemos, confunden en los espacios
de mi espíritu sus viejos dogmas y sus ritos pintorescos. Y ved aquí que
yo, el hombre de las grandes confusiones, el panteólogo desmemoriado
que, al descuidar la fijeza de su nombre, borra con igual descuido los
nombres de las cosas, me meto a refundir en una sola creencia las tres
que aquí los humanos practican, divididos en castas, familias o rebaños,
con sus marcas correspondientes. Adviértase que la síntesis religiosa es
para mi uso particular y exclusivo goce, sin ningún prurito de
apostolado ni cosa que lo valga. Las tres me mandan que ame a Dios sobre
todas las cosas, y al prójimo como a mí mismo, y que perdone las
ofensas; las tres me señalan la vida perdurable como fin sin fin de
nuestro ser, y me ofrecen recompensa o castigo conforme al valor moral
de mis acciones, mientras me tiene Dios estacado en la sociedad humana,
paciendo en las no siempre fértiles praderas de la vida fisiológica.

Ninguna creencia monoteísta me manda matar ni robar; pero veo que todas
violan el precepto en las guerras y trapisondas, mayormente si éstas son
traídas por el furor pietista de los pastores que nos guían en este
mundo, y en los caminos para llegar felizmente al otro. Yo ni mato ni
robo, y considero la guerra como el pecado mortal de las naciones. En el
tratado del amor de mujer manifiestan las tres hermanas\ldots{} (que así
las llamo por no encontrar nombre adecuado con que designar su indudable
parentesco)\ldots{} manifiestan, digo, divergencias mayores que en otros
delicados puntos. Cuál dice que nos casemos con una sola; cuál, que con
cuatro; y alguna se nos muestra tan adusta y regañona en lo concerniente
al trato mujeril, que, si obedeciéramos con rigor inflexible sus crueles
prohibiciones, dentro de un par de siglos no habría ya mundo para
contarlo. Pastores y rebaño infringen con tácito acuerdo la inhumana ley
que proscribe toda alegría, y así, con el prohibir y el infringir bien
alternados, con este ten con ten, como dijo el otro, rebaño y pastores
van tirando hasta el fin de los siglos.

En verdad os digo que no me ha costado grandes quebraderos de cabeza
encontrar la idea fundente de los distintos criterios con que éste y el
otro Decálogo tratan de regular la máquina de nuestras pasiones. Yo
cumplo, yo infrinjo conforme a supremos dictados de humanidad viviente y
creadora, y al punto me sale la ley de indulto que acalla mi conciencia,
reconciliándome con las soberanas leyes\ldots{} Espero que este relato
de mi vida en tierras africanas me dará nuevas ocasiones de explanar con
detenimiento materias tan sutiles, y ahora, puesto a infringir,
quebranto el método natural de toda narración, y divago a mi antojo,
volando de idea en idea y de impresión en impresión.

Sabed que algunos días me levanto y me acuesto con la firme creencia de
que vivo en el más bárbaro país del mundo; sabed que no pocas noches me
acuesto y me levanto con la idea de que he venido a caer en un país
donde debemos aprender la civilización antes que enseñarla. El caviloso
examen de estas contradictorias opiniones mías a veces me ocupa mañanas
y tardes, sin que de mi tenaz raciocinio salga el término discreto en
que pueda fundar la verdad. Me interrogo y no sé qué contestarme. «¿Por
qué ha de ser signo de incultura el anónimo de estas calles, plazoletas,
encrucijadas y pasadizos? ¿Qué va ganando Tetuán con el furor bautismal
de los españoles, que no paran estos días de clavar rótulos en todas las
vías urbanas, trayéndonos acá la enfadosa titulación de las calles
europeas? ¿Son los tetuaníes mejores de lo que eran porque se llame
\emph{calle del Rey} lo que antes llamábamos, sin letrero alguno,
\emph{Kaisería}; \emph{calle de Cantabria} la extensa vía de
\emph{Trankats}, y de \emph{Chiclana} la famosa \emph{El Haddadin?»} Los
vencedores estampan en el cuerpo de la ciudad conquistada la marca de su
prepotencia; en él practican una especie de tatuaje con los nombres de
todas las unidades de su ejército y los de famosos territorios y pueblos
de España. \emph{Ojos de Manantiales} ha venido a ser un diccionario de
la guerra y de la paz. Los tetuaníes hojean el indigesto infolio sin
entender una sola letra; saben que están vencidos; sienten la mano del
dominador; pero miran con desprecio las muestras de su escritura y
lenguaje que el español va pintando en las paredes. Yo digo: «Bautizando
calles, nada conseguiréis. En las poblaciones marroquíes no habría
calles si no fuera indispensable un poco de suelo común para ir de un
edificio a otro. Dejaos de callejear, y buscad la vía por donde
penetréis en los corazones.»

Ayer comí con Alarcón y Rinaldi en la Judería, donde reside el primero.
Ambos se burlaron de mi ropa moruna, invitándome a reponer en mi persona
las decorosas prendas del vestir europeo. No me mordí la lengua para
defender mi vestido y prestancia, y despotriqué furiosamente contra el
odioso pantalón, incómodo y deshonesto, contra las chaquetas y levitas
de lúgubres colores, contra los acartonados cuellos de las camisas y las
ridículas corbatas que nos oprimen el pescuezo. «Cuando me acuerdo---les
dije,---del sombrero de copa, y de que yo he llevado ese absurdo
chapitel sobre mi cráneo, viendo en derredor mío, día y noche,
innumerables seres humanos afeados de igual manera, creo haber
despertado de angustiosa pesadilla, en la cual soñaba yo, y medio Madrid
conmigo, que éramos tubos de latón, y que por la cabeza despedíamos todo
el humo de las vanidades humanas.» Ya empiezo a dudar de que tales
sombreros hayan existido y de que yo me los haya puesto; ya veo
representada en ellos toda la impertinencia meticulosa y refistolera de
lo que llamamos \emph{Administración Pública}, la oquedad del
\emph{Organismo Burocrático}, nuevo poder erizado de fórmulas, de
ataduras, de pinchos, y que al exterior trata de hacerse imponente con
su empaque en cierto modo sacerdotal. Casullas me parecen las negras
levitas, y mitras los sombreros de copa. Vistos desde aquí los señores
de mi tierra y los primates de la política, me inspiran miedo
supersticioso. Su saludo, quitándose el tubo y volviendo a ponérselo
sobre la cabeza, en casi todos calva, me hace el efecto de un signo
hierático, como el gesto de aquellos figurones que decoran los
monumentos egipcios o babilónicos.

De estas extravagancias mías se ríen Alarcón y Rinaldi, y el \emph{moro
de Guadix} me contesta con otras más graciosas y peregrinas, acabando
por darme la razón y renegar conmigo de algunos usos europeos.
Alegrábamos nuestra comida con burlas y chascarrillos, poniendo en
caricatura el habla dengosa de las hebreas que nos servían, hijas de
Abraham Mendes, en cuya casa, que no es de las peores del \emph{Mellah},
tiene Alarcón su alojamiento. Este Abraham es hermano de \emph{Jakub
Mendes}, y como él, tratante en piedras y metales preciosos. A dos pasos
de allí, en la calle que ahora lleva el rótulo de \emph{Numancia}, tengo
yo modestísimo albergue que me proporcionó \emph{Simi}, pared por medio
con su casa, y que amueblamos con prestados trebejos, tapicería y
cerámica. Luce nuestro ajuar más de lo que debiera por el buen gusto con
que todo lo apaña y adereza \emph{Yohar}, cuidando de que en cada objeto
se vean de cara las partes libres de manchas, deterioros o desgarrones,
y de que queden en la obscuridad las estragadas por el uso y el tiempo.
Tal es el arte de mi compañera, que nuestra casa, en la cual estamos
como en un estuche por su extremada pequeñez, parece bonita sin serlo
realmente, y hasta nos da la ilusión de holgura en su exigüidad
molestísima. Influye no poco en esto nuestra imaginación, que desde los
días del rapto no cesa de construir en derredor de nuestra pobreza un
mundo risueño y grato: gracias a ella, lo duro se nos vuelve blando,
ancho lo angosto, y cuando yo, poniéndome en pie con descuido, sin
acordarme de la corta altura de la estancia, doy con la cabeza en el
techo, las estrellas que veo son los luminosos ojos de
\emph{Yohar}\ldots{} La imaginación nos calienta el comistraje que frío
recibimos de las manos de \emph{Mazaltob}, y nos disminuye
considerablemente el número de pulgas y de otras perversas alimañas que
de la casa de \emph{Simi} vienen a la nuestra, en busca del pasto
abundante que les ofrecen los cuerpos jóvenes\ldots{}

Otra vez divago, lector mío: no puedo sujetar mi versátil pensamiento,
que se me tuerce y ladea cuando más en derechura quiero llevarlo\ldots{}
Recojamos y anudemos la hebra interrumpida. Digo, pues, que Alarcón y
Rinaldi, después que almorzamos, me llevaron a dar un paseo por la
ciudad, y al cabo de unas vueltas perezosas por las calles próximas al
\emph{Zoco} fuimos a parar al \emph{Fondac}, que es como decir parador,
lugar de reposo y transacciones comerciales, que los españoles han
transformado llevando a él la cháchara morosa de los casinos de allende.
Oficiales de distintas armas tomaban café bajo el emparrado sin hoja que
entre las dos crujías del local forma un techo completamente ilusorio.
Con unos y otros charlamos, hasta que, secos nuestros gaznates, hubimos
de humedecerlos con las infernales bebidas europeas que allí vendía un
travieso argelino, de cuyo nombre no me acuerdo. Se hablaba del delirio
patriótico con que acogían todas las ciudades de España los recientes
triunfos; de los planes de O'Donnell; de los rumores de próxima paz; se
traslucía en todos el deseo de que ésta llegara pronto, pues ya era hora
de consolidar las glorias en el descanso; algunos dedicaban palabras
medrosas a los estragos del cólera morbo, dentro y fuera de la ciudad,
llevando cuenta de los casos que por la celeridad de la muerte infundían
mayor lástima y terror.

En estas conversaciones nos entreteníamos, cuando me sobrecogió la
presencia de dos sujetos que aparecieron por el \emph{foro} del
\emph{Fondac}, y así lo expreso, porque siempre vi en aquel patinillo
disposición semejante a la de un escenario: paredes a izquierda y
derecha con puertas practicables; foro de tenduchas arrimadas a una
pared con angostos ajimeces; bambalinas de emparrado\ldots{} De una de
las tiendas del fondo, o de la portezuela mal escondida en la rinconada,
no estoy bien seguro, salieron los dos hombres en quienes mis ojos y mi
atención se clavaron: el uno moro de buen porte, viejo barbudo el otro y
de traza judaica. Pasaron cerca de mí, y ya en los bordes de lo que
podríamos llamar proscenio, detuviéronse para mirarme. En el moro noté
lástima cariñosa; en el hebreo, desdén, odio, rabia: su boca me habría
mordido si pudiera, y sus ojos, fulgurantes bajo las cejas blancas de
cerdosos pelos, me lanzaban miradas que me habrían deshecho si fuesen
rayos\ldots{} Eran mi fanático suegro \emph{Simuel Riomesta} y mi
gallardo amigo \emph{El Nasiry}.

\hypertarget{ii}{%
\chapter{II}\label{ii}}

\emph{Segunda semana de Adar}.---Se alejaron hablando de mí, bien lo
conocía yo, y a mayor distancia volvieron a detenerse y a mirarme.
\emph{Riomesta} unió al rencoroso mirar un gesto de amenaza, extendiendo
el rígido brazo hacia mi humilde persona. Desaparecieron, dejando en mí
una sensación de ansiedad expectante. Toda la tarde, antes y después de
abandonar a mis amigos, estuve muy metido en cavilaciones. Asaltaban
sucesivamente mi espíritu presagios de distintas calamidades, y mi
excitada memoria reproducía con maligna insistencia hechos observados en
mi propia casa dos y tres días antes. No he dicho aún, por no tener
ocasión de ello, que mis vecinas me habían informado de las visitas que
a \emph{Yohar} hizo \emph{Riomesta} algunas tardes, hallándome yo
ausente. Ignoraban lo que hija y padre habían hablado, por ser el
camaranchón inaccesible a la curiosidad de ojos y oídos; pero veían
salir al viejo bufando, con temblor de la mandíbula inferior y de su
barba hirsuta. Luego encontraban a la blanca mujer deshecha en
lloriqueos, y algún día viéronla rasgar con fiero impulso un pañuelo de
fina seda con que su seno cubría. Interrogada por mí sobre el
particular, \emph{Yohar} me contó que su padre la reprendía y amenazaba,
negándole todo auxilio de dinero mientras viviese conmigo\ldots{} Verdad
parecía esto; mas no era, según mi entender, la verdad completa. Algo
más había, sin duda, que en el pensamiento de mi amada quedaba como en
expectación medrosa, no sin que lo dejasen transparentar sus ojos
dormilones y aun la tersa blancura de su frente.

Debo decir que no ha desmentido \emph{Yohar} ni un solo día la
inclinación amorosa que la trajo a mi lado, ni ha dejado de ser tierna,
dulce, firme y encendida en su afecto. Sólo para mí vive, como yo para
ella, y en sus cálculos de futura existencia habla como si nuestros
destinos fuesen inseparables, y nuestras almas no supieran romper su
armonía venturosa. En los azarosos días, antes y después de la ocupación
de Tetuán por los españoles, el ánimo de \emph{Yohar} era de una
igualdad encantadora; ninguna privación ni molestia lo abatían; ningún
contratiempo apagaba en sus labios la franca sonrisa con que iluminaba
mi existencia y la suya\ldots{} Instalados en la casuca del
\emph{Mellah}, porque nuestro menguado peculio no nos consentía mejor
vivienda, nos avenimos a la estrechez, y extremando la conformidad,
llegamos a encontrar delicioso aquel escondrijo y hasta muy favorable a
la salud. Burlándonos de las molestias, concluíamos por soportarlas y
aun por creerlas buenas: la sal de las bromas y la dulzura del amor,
alternadas en el tiempo sin espacio de hastío entre una y otra, nos
sazonaban la vida en tal manera, que no ambicionábamos vida mejor.

Cuando nos faltaba qué comer, porque \emph{Simi} no había logrado vender
el puñadito de aljófar que a nuestro sustento destinábamos cada semana,
\emph{Yohar} distraía y engañaba nuestra inanición con humoradas
donosas. Algunas mañanas, en los ratos que mediaban entre un despertar
alegre y un desayuno de inaudita frugalidad, hacía volatines sobre las
enjalmas y tapices del camastro, y elevando sus extremidades inferiores
de inmaculada blancura, daba pataditas en el techo; o bien se deslizaba
por un hueco alto del tabique medianero entre la alcoba-sala y el
comedor-cocina, no más grandes que un confesonario de mi tierra,
realizando el prodigio de adelgazar su cuerpo hasta lo increíble, y de
imitar las ondulaciones de la culebra. Y alguna vez, cuando se me pegan
las sábanas, suele despertarme armando en la próxima cocina un pavoroso
ruido de platos vacíos, imitando el que hacen los duendes o diablillos
que invaden las viviendas abandonadas. Me maravilla la destreza de manos
de \emph{Yohar}, que mezcla con estos ruidos el de una pandereta y
furibundos toques de almirez.

Un sábado, bien lo recuerdo, cuando comíamos la excelente \emph{adafina}
con que nos obsequió \emph{Mazaltob}, tuvo mi \emph{Yohar} el mal
acuerdo de reiterar tardíamente sus primeras instancias para que yo
abrazase su ley. Con negativa tan terminante había yo rechazado sus
proposiciones en los días que bien puedo llamar nupciales, que no creí
volviese a mentar semejante asunto. Y no sólo habíamos convenido en que
yo no cambiara de religión, sino que ella se mostró cautivada del
Cristianismo y deseosa de abrazarlo, para que nuestra común fe bendijera
el himeneo de nuestras almas. Había yo empezado a instruirla en los
misterios dogmáticos de mi fe, así como en la dulce moral de Cristo, y
veía con gozo su adaptación fácil a los nuevos ritos, y el calor y
entusiasmo con que recibía mis lecciones. ¿Por qué de la noche a la
mañana dejaba entrever repugnancias de su abjuración, y me proponía que
fuese yo el que diera el atrevido paso para llegar a la igualdad o
armonía de nuestras creencias?

Pasados unos días, en plena festividad de \emph{Purim}, creí haber
convencido a \emph{Yohar}. Derramó tiernas lágrimas; su viva imaginación
me siguió por los espacios del idealismo cristiano, y cuando estaba
conmigo en la zona más alta, cayó de improviso, expresando así la
sincera verdad de sus deseos: «Oye tú, mi \emph{Yahia}: ¿no percatas que
ha de enfurecierse el Dío cuando vea que troco mi ley y me jago
cristianica? Dejarme has como so, y tú lo mesmo con tu Jesuscristo. Onde
por ello diremos a casarnos a Gilbartal, y allí moraremos, tú mercador,
yo señora polida y esponjada de ropa\ldots{} A casarnos por lo inglés,
\emph{Yahia}, y a ser ricos con cuenta grande de \emph{doblas, doros y
fluses.»}

Ya me había manifestado \emph{Yohar}, con vaga ensoñación de grandezas,
sus deseos de vida europea, conservando la fe judaica. No se borraba de
su memoria el recuerdo de unas señoras hebreas de Gibraltar que poco
antes de la guerra recalaron en Tetuán, deslumbrando con su riqueza y
lujo. Vestían trajes europeos de formas extravagantes y de colores
vivos; cargaditas iban de alhajas; derrochaban la plata menuda, y aun el
oro, en el auxilio de los judíos indigentes. Fueron por muchos días
admiración y comidilla de todo el vecindario del \emph{Mellah}\ldots{}
Un barquito muy cuco, propiedad de un inglés millonario, las había
traído de Tánger al Río Martín, y en este punto se reembarcaron para
recorrer toda la costa septentrional del continente hasta Damieta o
Alejandría. Dejaron tras sí una estela luminosa en el pensamiento de las
hebreas pobres, y en las ricas un dejo de admiración que fácilmente en
envidia se trocaba. Mi \emph{Yohar}, según pude entender, no era la
menos dañada en su espíritu por aquellas fugaces visiones de opulencia y
de lo que ella creía la suma elegancia. Desviada de tales pensamientos
por el arrebato amoroso, a ellos volvía, con la remisión de aquella
dulce fiebre, y trataba de conciliar el querer y el presumir, forjándose
una ilusión de vida en que la comodidad y riquezas se fundiesen con el
amor del pobre \emph{Yahia}.

No hay que decir que yo, con mis sutilezas retóricas, traté de apartar a
la blanquísima hembra de aquellas manías. Discutíamos, y al parecer mis
pensamientos vencían y dispersaban los suyos, sin que por esto pudiera
declararme vencedor. Creía yo haber tomado la plaza, y ésta me mostraba
al siguiente día sus muros inexpugnables; que las mujeres dejan tomar al
hombre la fortaleza de su espíritu, y al instante de nuevo la levantan
con los mismos caprichos y tenaces deseos. Yo le argüía con lógica
incontestable; demostrábale que, abandonados de su padre \emph{Simuel},
no teníamos esperanza de riqueza ni aun de bienestar mediocre; que
nuestra salida del atolladero era un pasar modestísimo, trabajando los
dos en cualquier oficio, o en un menudo comercio. Conciliáramos ante
todo nuestras conciencias, dando solución práctica al intríngulis
religioso, y después podíamos allegar en Europa el pan de cada día,
seguros de que la protección de Dios no había de faltarnos. Sobre estas
ideas pasaba ella volando con las irisadas alas de su vana superstición.
Confiaba loca y ciegamente en la suerte, que los judíos llaman
\emph{mazzal}; creía en el súbito hallazgo de tesoros, en la emergencia
de un cúmulo de circunstancias u ocasiones providenciales para
enriquecernos de la mañana a la noche, en la teatral aparición de genios
o diablillos que caían del cielo o brotaban de la tierra para ofrecernos
con su protección todos los bienes del mundo. Ferviente devota de la
suerte, terminaba nuestras disputas con el expresivo refrán hebreo:
\emph{Daca un cuajito de maizal y tírame a las fondinas de la mar}.

Fácil es comprender, por lo dicho, que el problema vital me inquietaba
cada día más, y que pensaba seriamente en plantar los jalones de nuestra
existencia definitiva. Los recursos para subsistir, representados por
puñados de aljófar que cada día iban mermando, pronto se extinguirían.
La vida en Tetuán se hacía imposible: era forzoso pasar el Estrecho y
establecernos en tierra europea, donde hallaríamos fácilmente cualquier
arbitrio para ganar el sustento. Lo más próximo, lo más hospitalario,
era sin duda el Peñón, aquel pedazo de tierra híbrida y cosmopolita que
aún tiene algo de España, algo más de Inglaterra, y mucho de los vecinos
países africanos. En aquel solar anclado en el Océano, viven en santa
paz la libertad, el comercio, el contrabando, y en busca del bienestar
andan allí de la mano todas las religiones.

A Gibraltar, pues, dirigí mis propósitos, discurriendo la granjería en
que más fácilmente podíamos \emph{Yohar} y yo ejercitarnos. Pensé que el
comercio de fruta no tiene hoy la extensión debida, por la indolencia de
estos pobres berberiscos, y me sentí con ánimos para darle mayor vuelo.
La campiña de Tetuán es pródiga en rechazamos frutas, aun en aquellas
partes de la tierra más descuidadas de la mano del hombre. Las naranjas
de \emph{Quitan}, dulces y finas, han aprendido ya el camino del mercado
de Gibraltar; no así los exquisitos y olorosos melocotones de
\emph{Hal-lila}, que por criarse a mayor distancia de Río Martín, no
aciertan a salir en busca del dinero. ¿Por qué no he de ser yo quien
abra una vía fácil a tan rico producto, agregando a él las peras
\emph{meski} o moscateles, que por su extrema delicadeza no se avienen
con la lentitud del transporte, y las uvas de \emph{Dar Murcia}, que,
según dicen, en ninguna región de Europa tienen semejante?

Pensando en esto, mi fantasía me lleva más allá de los límites de
ambición de un humilde mercader, y con los ensueños comerciales empalmo
los agrícolas, imaginando que el cultivo del algodón en parte del valle
de Tetuán y en los términos de \emph{Beni Said} y \emph{Beni Madán}
crearía incalculable riqueza\ldots{} ¿Verdad que me parezco a los
políticos proyectómanos de mi patria, que amenizan los ocios de la
oficina engrosando ilusiones, fabricando porvenires, o construyendo
emporios con materiales de cifras mentirosas, y amañadas premisas de
aptitudes falsas o de fertilidades de fantasía\ldots? No: déjeme yo de
algodones y monsergas, y aténgame al modesto trajín de comprar fruta por
poco precio para venderla como pueda, engañando al infeliz consumidor
que me caiga por delante.

Combatía yo la testarudez y las limitadas nociones de \emph{Yohar} con
medios persuasivos de indudable eficacia: eran éstos la rica ideación
europea, el lenguaje castellano usado por mí con gallardía teórica, y
variedad abundante de vocablos y locuciones. El hablar mío la subyugaba,
y sus ideas rutinarias, expuestas con dicción tosca, mísera, como un
instrumento roto y destemplado, eran reducidas a polvo por mis ideas.
Fáciles triunfos alcanzaba yo diariamente en nuestras disputas; mas
llegó un punto en el cual mi argumentación para ella rica y fascinadora,
mi lenguaje armonioso, mi dicción pura, que en sus oídos sonaba como
arte lírico de cadencias musicales, no causaban efecto sensible, y eran
como los ruidos de la lluvia o del viento. Convencido yo de que nuestra
situación no tenía salida venturosa, y de que habíamos de sucumbir si no
luchábamos bravamente por la existencia, traté de inculcarle la idea
cristiana de la conformidad con las adversidades, de la tribulación como
fundamento de la verdadera alegría y de la paz del alma. Si la pobreza y
el trabajo eran nuestra única solución, debíamos afrontar el infortunio
con ánimo sereno, y hacer de él el amigo y el tutor de nuestras almas.
Evocando todo lo que yo había leído en libros místicos y ascéticos, hice
la apología de la pobreza; demostré a \emph{Yohar} que admitida y
agasajada en nuestros corazones la certidumbre del no poseer, hallamos
en ella un bien positivo que fácilmente se trueca en la mayor riqueza;
acabé por asegurarle que la suma carencia es al fin la suma posesión de
todos los bienes, y que de la tristeza y del abandono surge, como el día
de la noche, el mayor regocijo de las almas bien templadas. Todo esto
dije y argumenté, desplegando las facultades que me ha dado Dios; pero
mi opulenta retórica, mi verbo armonioso con líricos arrebatos, no
hicieron en ella más impresión que si le hablara en lengua chinesca.

¡Aceptar la pobreza, más aún, amarla, y alegrarse de ser pobre! Esto no
entraba en el cerebro de \emph{Yohar} ni con escoplo y martillo\ldots{}
Vi que la penetración de mis ideas era estorbada por una capa de
egoísmos atávicos, obra lenta y formidable de la especie,
reproduciéndose en moldes iguales al través de cien generaciones. Por
primera vez, \emph{Yohar} se reía de mis bellos discursos, holgándose de
no sacar de mi poética prosa ninguna substancia. Suspendí al cabo mis
sermones, dándome a pensar con qué ligaduras podría sujetar a la Perla
si nuestros destinos nos llevaban efectivamente a vida rigurosa y
austera\ldots{} Mas no tuve tiempo de coordinar nuevos planes, porque
Dios precipitó sobre mí sucesos sorprendentes y desgraciados, que
pusieron en dispersión mis ideas, y aplastaron, literalmente, mi
voluntad.

De esto escribiré otro día\ldots{} Lo que es hoy, fatiga y tristeza
paralizan mi mano cuando intento coger la pluma.

\hypertarget{iii}{%
\chapter{III}\label{iii}}

\textbf{Tetuán}, \emph{mes de Adar}.---Pienso que esto que escribo no
tendrá lectores\ldots{} Mi amigo ilustrísimo, el marqués de Beramendi,
me ha dicho \emph{mutatis mutandis}: «Desengañado Juan, si no quieres
referir cosas de guerra, refiere cosas de paz; si te repugnan los
asuntos públicos, ya sean militares, ya políticos, cuéntame los tuyos,
que en muchos casos las historias de hombres aislados y sueltos cautivan
más que las de tribus o naciones. Con sinceridad lo digo: las aventuras
de cualquier español voluntarioso, enamorado y poco sufrido, me saben a
historia general más que las acartonadas narraciones de batallas, o de
tumultos populares que alteran la tranquilidad de la Puerta del Sol y
calles adyacentes.» Esto me dijo en la última carta que de él
recibí\ldots{} ¿Cuándo? Paréceme que ha pasado un siglo\ldots{} En
derredor de mi memoria revolotean como palomitas mis recuerdos,
queriendo volver al palomar abandonado\ldots{} Pienso que llegó a mis
manos la última carta del Marqués cuando acampábamos junto a la Aduana
del Río Martín\ldots{} Pasaron días y días sin que me entrasen ganas de
seguir la senda literaria que mi amigo me marcaba, hasta que una mañana,
sin saber de dónde venía tal impulso de mi movediza voluntad, me sentí
historiador de mí mismo, y agarré el primer cálamo que en las judías
estancias del \emph{Mellah} encontré.

Escribo sin saber a dónde irán a parar estas crónicas. Ignoro si serán
leídas por muchos, o tan sólo por el desocupado Beramendi, que como
hombre rico se permite curiosidades superfluas y entretenimientos sin
ningún fin práctico. Sé que guarda papeles mil, escritos por hombres o
mujeres extravagantes; que reúne cartas amorosas, sin excluir las más
ridículas, y que a todo amigo que sale de viaje le pide una relación
sincera de cuanto ve y padece en galeras y paradores. Hace colección de
confidencias de locos o criminales, ya sean escritas para la familia, ya
con el fin de solicitar una publicidad que difícilmente encuentran. Pues
allá te van también mis confidencias, ¡oh, Pepito ilustre!, sin que sea
mi ánimo darte en ellas un modelo de discreción, ni tampoco enseñanza
para los que gusten de aprender en las vidas ajenas el régimen de la
propia. Serán mis escritos, como yo, desordenados, ahora discretos,
ahora desvanecidos en estrafalarios ensueños o en caprichosas
divagaciones. A falta de método, hallarás en ellos sinceridad, y el
prurito constante de no recatar de la publicidad, si por acaso la
hubiere, los pensamientos más recónditos.

Sigo contando. Invitáronme aquel día Rinaldi, Alarcón y el pintor
francés Iriarte a visitar al General en Jefe en su campamento. O'Donnell
había cambiado la blanda ociosidad del palacio de Ersini, en el centro
más laberíntico de Tetuán, por la estrechez de una tienda, rodeado de
sus tropas, que aún sueñan con mayores triunfos. Acampa el Caudillo
fuera del pueblo, en la primera vega que se encuentra conforme salimos
por \emph{Bab el-aokla}, ahora \emph{Puerta de la Reina}. Otro
campamento hay por la parte del Oeste, camino de \emph{Bu-Sfiha} y en él
están Prim y Zabala, el cual, restablecido de su dolencia, ha vuelto a
campaña. Aunque extremaron sus halagos para llevarme consigo, no quise
bajar a los campamentos. Díjome Alarcón que aquel día se celebraba la
primera conferencia para tratar de la paz, y que habían venido unos
morazos muy elegantes con poderes del Emperador. Ni con el incentivo de
ver moros bonitos lograron seducirme. Les acompañé hasta la salida de la
ciudad, y me volví a la \emph{Kaisería}, donde también yo tenía mis
paces que ajustar, o sea un tratado de alianza comercial con dos
argelinos que traficaban en Gibraltar y Marsella, hombres de gran
diligencia y despejo, a quienes conocí antes de la ocupación, y me
habían mostrado simpatía y confianza.

Ofrecieron incorporarme a sus negocios, tomando de mí, no capital que no
poseo, sino el trabajo asiduo, la fidelidad y mi conocimiento de la
lengua española, dándome una participación por de pronto exigua, pero
que luego iría creciendo, creciendo\ldots{} ¡Dios me valga!\ldots{} el
\emph{mazzal} soñado por mi Perla no era un espejismo nebuloso, sino una
realidad que a la mano se nos venía, cosa tangible, sonante y sabrosa.
«¡Oh \emph{Yohar}---pensé,---no verás el rostro descarnado de la
pobreza!\ldots» Pues ello era que mis amigos \emph{Djar y Ben Sulim} se
proponían extender sus negocios a Málaga y Cádiz, y desde aquí penetrar
hasta el corazón de Andalucía, que es Sevilla la grande, la graciosa,
\emph{orgullo y regocijo del Padre Eterno}.

Imaginad mi júbilo cuando los argelinos me propusieron tomarme, no diré
por socio, sino por auxiliar de las granjerías que iban a emprender en
España. Introducirían directamente los magníficos tafiletes, dátiles,
miel, madera de alerce y otros artículos. Necesitaban una cabeza
española que les guiara en los senderos de la vida peninsular, y como
tenían de mi entendimiento una opinión harto favorable, por lo que
habían oído a \emph{El Nasiry}, creyeron haber encontrado el hombre de
aptitudes para el caso. A las ideas que iban ellos expresando, me
anticipaba yo saltando por encima de sus razones y sugiriéndoles nuevas
ideas de ignorados negocios pingües que en España podrían realizar, y
encareciéndoles la sutileza y probidad con que yo les ayudaría en la
multiplicación de sus ganancias. Por de pronto, yo multiplicaba mis
ilusiones y las hinchaba desmedidamente, dejando correr mi fantasía con
ímpetu semejante al de la famosa lechera. Ya era yo comerciante. Me
estrenaba como dependiente; pronto sería socio; establecido después por
mi cuenta con capital propio, en pocos años me vería bien acomodado,
pudiente, rico\ldots{} ¡Como hay Dios, que así había de ser!

Loco salí de la tienda de los argelinos, y todos los caminos parecíanme
largos para volver a mi tugurio, ansioso de contarle a \emph{Yohar}
tales bienandanzas. Ya veíamos venir el suspirado \emph{mazzal}\ldots{}
ya se disipaban los temores de pobreza vil\ldots{} ya teníamos abierto
un camino de bienestar, si estrechito en su primer trozo, luego ancho y
florido\ldots{} ¡Y qué asustada y cuidadosa estaría la pobrecita
\emph{Perla} esperándome, pues aquel día, por mis dilatadas
conversaciones con los de Argel, regresaba yo al nido dos horas más
tarde de lo regular!\ldots{} Pero su inquietud tendría remedio
instantáneo en el alegrón que yo le llevaba. Ya me imaginaba yo su
júbilo y los extremos que haría para manifestarlo, pues es mujer que
nunca pone discretos límites a la expresión de sus sentimientos. De
seguro se lanzaría con ardor al juego de volatines y atletismo, haciendo
alarde de su extraordinaria fuerza y agilidad; daría vueltas de carnero
en nuestro camastro; remontaría sus remos inferiores pisoteando el
techo, quizás abriendo en él un boquete; andaría con las palmas de las
manos; imitaría a la serpiente y al cocodrilo, sin olvidar el furioso
estruendo de platos y almirez para sorprender y aterrorizar a la
vecindad\ldots{} Todo esto pensaba yo corriendo hacia mi vivienda, y en
mitad del \emph{Zoco} me encontré a \emph{Esdras} el borriquero, que del
\emph{Mellah} salía. Lo mismo fue verme, que tirarse del asno y acudir a
mí con solícita premura.

\emph{«Goi}---me dijo:---sé que a tu tierra te tornas\ldots{} Yo te
ruego dejarme dir contigo\ldots{} por si allá topo más mejor fortuna.
Español bueno aquí\ldots{} allá buen gentío español. Aflójame tu
voluntad, \emph{goi}, y llévame\ldots»

---¿Sabes ya que me dedicaré al comercio, que iré a Gibraltar, a
España?---dije, sorprendido de que aquel desdichado conociera el nuevo
camino que la suerte me abría.

---Lenguas todas del \emph{Mellah} cuentan que te vas y no güelves, ca
en el Marroco no tienes vivires apañados.

---Cierto es, \emph{Esdras}, que aquí no hallamos buen vivir, y debemos
ausentarnos.

Díjome entonces que él se sentía mercachifle, y que la mala suerte le
condenaba a ganarse la vida con su borrico en tan mísero estado\ldots{}
En España, trabajando conmigo en la compra y venta de ropa vieja, que él
sabía remendar y poner como nueva, ganaríamos \emph{mucha cuenta de
plata}. Mi alegría me hizo benévolo, inclinándome a la protección de los
desvalidos: le prometí hacer en su provecho cuanto pudiera, y no le
entretuve más tiempo, porque la impaciencia me abrasaba.

Pocos pasos me separaban ya de mi nido. A él corrí desalado\ldots{} Al
entrar en la sucia calle que se decora con el épico nombre de
\emph{Numancia}, vi frente a la puerta de \emph{Simi}, que era mi
puerta, un grupo de judías, las cuales, en cuanto me vieron llegar, se
encararon conmigo saludándome con una exclamación lúgubre, que me dejó
helado. «¡Guay de ti, \emph{Yahia!} ¡El Dío se apiade del coitadico
\emph{Yahia!»} Así gritaban, manoteando en forma semejante a los
aspavientos de duelo que hacen aquí las mujeres ante los difuntos. Pensé
que un gran infortunio había ocurrido durante mi ausencia, y en mi
interrogación ansiosa no acerté a pronunciar más que el nombre de
\emph{Yohar}. Antes de responderme concretamente, repitieron su clamor
doloroso: «¡Ay, mi corazón, mi corazón!\ldots{} ¡Ay, mi cordojo grande!
¡Ay, qué extremación de desdicha!» Angustiado y loco, no sabía yo qué
decir. Sin duda, mi \emph{Yohar} había muerto. ¿Dónde estaba?\ldots{}
Corrí a besar su cadáver\ldots{} «No te endolores más de cuenta,
Yahia---me dijo \emph{Mazaltob} poniéndome en el pecho las palmas de sus
manos.---Sábete que \emph{Yohar} no es muerta, sino ida\ldots» «Ida es
de tu casa esa perra,» gritó \emph{Simi} ronca de ira.

¡Ay de mí! Entre todas me cogieron y me llevaron al patinillo de
\emph{Mazaltob}. Más muerto que vivo estaba yo, y no podía valerme.
Comprendí el funesto caso; la verdad penetró en mí con lívida claridad.
«¿Pero es cierto que \emph{Yohar} se ha ido de mí?\ldots{} ¿que mi
\emph{Perla} me abandona?»

---Cierto es como la luz de Adonai---replicó la hechicera.---Asosiégate,
\emph{goi}, y aflójate de rabia, que agora es ocasión de que te
apersones con virtud que ella no tiene. Tú sodes bueno y barragán; ella,
una puerca \emph{fidionda}.

La hermana mayor de \emph{Simi}, llamada \emph{Hanna}, vendedora de ropa
vieja, me trajo un pañuelo grande, de frágil tela llena de zurcidos, y
con gravedad sacerdotal me dijo: «Coge este lienzo que para nada vale
ya, y rásgalo con fuerza para desafogar tu ira. Con los pedazos te
lavarás el rostril de las glárimas que derrames, y así quedarte has
sosegadico de tus entrañas.» Obedecí a la hebrea en lo de rasgar la
tela, lo que hice de un tirón con verdadera furia. Luego les pedí
explicaciones. «Contadme, referidme todo. ¿Se ha ido por su propia
voluntad, o vino su padre a llevársela por fuerza?»

En vez de referirme sucintamente lo sucedido, \emph{Simi} rompió en
maldiciones contra \emph{Yohar}. «Le venga el mal de la cabra, cuerno,
sarna y barbas.» Y la feroz \emph{Hanna}, rasgando por su cuenta otro
lienzo grande, que no era más que un pingajo corcusido, gritó: «¡Hija de
la \emph{baranid-dah} enconada!» Esta maldición es de tan feo sentido
que no puedo traducirla. Comprendiendo \emph{Mazaltob} antes que las
otras mi situación de ansiosa incertidumbre, inició la referencia clara
de los hechos: «Vinieron por ella su padre \emph{Riomesta} y \emph{El
Nasiry}. Tirándola del brazo se la llevaron. Ella hizo semblanza de
desgana y salió lloricosa\ldots» «Mas era compostura de mentira---dijo
\emph{Simi},---que yo le caté los ojos bien secos cuando jacía que
ploraba, y sus ahijidos eran someros de la boca, y no le salían del
jondo.» Y \emph{Hanna} prosiguió: «Ya lo tenían amasado el padre y la
hija en el forno de sus codicias\ldots{} Ya estaba tratado, de días
luengos atrás, casarla con un \emph{sephardim} de Constantinopla, que
tiene casa en Gilbratal, \emph{Natham Papo Acevedo}, de mucha fazenda y
compra-venta de fierro.»

Y he aquí que \emph{Mazaltob} me trajo té caliente aromatizado con
\emph{nana}, y que los primeros buches de la tónica bebida calmaron un
tanto mis irritados nervios\ldots{} Siguió la hechicera ilustrando con
interesantes pormenores la historia que había empezado \emph{Hanna}:
«Hoy tiene \emph{Riomesta} en su casa \emph{envita}; él mismo fue esta
mañana al matadero a degollar un pato graso; aluego compró en la tienda
de \emph{Saddi} un cazolito de pimento y otro de aceitunas curadas;
aína, entre \emph{Simuel} y la criada \emph{Mesooda} pusieron a asar el
pato\ldots{} Ha días que \emph{Mesooda} jace jaleas muchas, y dolces,
pastas riales, y almibres ricos de todo dulzor\ldots{} Oyí que ponen
otrosí un grande pez que trujo de Río Martín el borriquero Esdras, y lo
asarán en cazolón con manteca, citrón y especias de olor\ldots{} Pondrán
aguardente y licor fino de rosa\ldots{} en canecos de vidro\ldots{} Todo
esto será para envitar al novio \emph{Papo Acevedo}, que llegó
anoche\ldots{} Da \emph{Riomesta} a su hija dote valoroso, sacos muchos
de doblones y plata en un cofre holgón\ldots»

Y \emph{Hanna}, con voz de sibila, prosiguió: «Farán la boda en el mes
de \emph{Siwan}, pasada la vegilia de \emph{Schabuot}. Haberán gallinas
muchas, licores finos de la Francia, olivas gordas del \emph{Andalus},
seis carneros fritos para sesenta envitados, tortas blancas y pretas, y
una corambre de vino. Será boda roidosa con vigolines y vigüelas, mósica
de dulzor y alborotos\ldots{} pues ainda tocarán tambora y
almireces\ldots»

Y otra de aquellas bíblicas tarascas, llamada \emph{Reina}, gorda y
crasa, ceñido el rostro con dos lienzos blancos, el uno haciendo
barbuquejo, el otro turbante, clamó con voz semejante a la de las
plañideras que se alquilan para los funerales: \emph{«Guau,
guau}\ldots{} ¿Qué es de ti, mancebo adolorado? ¿Perdiste tu coima?
Tómate agora buen caldo, y quédate riyendo de ella; no la endereces
llanto ni te asofoques de lamentación, que ya ella no es blanca, sino
preta, preta de su maldad. Quítate del corazón el celo, y no te membres
del melindre con ella, que es una perra \emph{niscaliá. Guau, guau}.
Fuese con otro; déjala, y no te deplores. Blancura de leche no tiene ya,
sino sombra de noche escura\ldots{} Agora la ves desmayada con
\emph{Papo Acevedo}. Ríyete, y gózate de verte liberado y desenvolvido
de esa puerca.»

Y dijo \emph{Hanna} la ropavejera: «No invidies a \emph{Natham Papo},
que él no tendrá ventura con \emph{Yohar}, sino potra y quebradura, y tú
serás gozón y bonito barragán de otras más garridas.»

Y dijo \emph{Simi}: «Beberás leche de camella, que es de virtú, y te
zajumarás con olores y jumos de \emph{nana}, y con esto y con el semah,
que yo te colgaré del pecho, se te ha de quitar la secura de tu meollo,
y el celo de \emph{Yohar}, que es tu mal, mal de hombre mujerado, y la
fiel se te golverá miel.»

No puedo negar que las vociferaciones de aquellas estantiguas calmaban
mi pena y me abrían horizontes de consuelo; extraño fenómeno, que no he
podido explicarme. Por último, la hechicera \emph{Mazaltob}, que en
cierto modo solía poner en su conducta y en su lenguaje unas briznas de
filosofía práctica, me acarició y popó con maternal dulzura diciéndome:
«No te apenes, hijo, y repárate de ese cordojo. Ya me has uyido mil
veces que si \emph{Moseh morió, Adonai quedó.»} Con esto quería
significar que debemos mirar serenos el paso de las desdichas
temporales, fijando los ojos del alma en lo inmutable y eterno.

\hypertarget{iv}{%
\chapter{IV}\label{iv}}

Viéndome más sereno, me obsequió \emph{Simi} con pipas de calabaza y
sandía tostadas, golosina que entretiene la voluntad y disipa los
pensamientos rencorosos. No obstante mi aparente conformidad con el
Destino, la procesión de mis agravios iba por dentro, y no podía
resignarme a la traición de \emph{Yohar} sin decir a ésta cuatro
verdades más o menos frescas, y sin coger por mi cuenta al
\emph{sephardim} que me robaba la mujer, y obsequiarle con una pateadura
en el \emph{Mellah} o donde quiera que le encontrase. Como español y
como cristiano, no podía evadir el precepto de honor que a una venganza
donosa y pública me obligaba, y habría dejado en mal lugar a mi
nacionalidad y a mi fe (aunque esto parezca mentira), si al cumplimiento
de tan sagrado compromiso no me aprestase sin perder horas ni minutos.
Cuando este propósito manifesté a las judías que me rodeaban, advertí en
ellas más sorpresa que terror. No comprendían mi acción vengadora ni los
sentimientos en que tenía su origen. Alguna me incitó a la paciencia, y
en otras noté una vaga admiración de mi audacia \emph{barragana}, en el
sentido de arranque temerario y caballeresco. Cuando les dije que
\emph{Natham Papo} y yo nos pelearíamos hasta que uno de los dos quedase
tendido en medio de la calle, se asustaron. \emph{Hanna} se apresuró a
rasgar otro indecente trapo inservible, y \emph{Mazaltob}, con acento de
prudencia, me agarró del brazo diciéndome: «Tente, goi, tente con
justedad, y cata que \emph{Papo Acevedo} está abrigado debajo de la
bandera cónsula de la Ingalaterra. Serás cogido y aína llevarás
condenación de azotes.» De la escandalosa chillería de aquellas pécoras
no hice ya maldito caso, y me zafé de sus garras, echando a correr fuera
de la casa y por la calle adelante, sin cuidarme de las mujeres sucias y
chiquillos tiñosos que a mi paso repetían el fúnebre \emph{guau},
\emph{guau}.

Tomé la vuelta de calles excéntricas para dirigirme a la parte del
\emph{Mellah} llamada \emph{Meca}, donde está la casa de mis enemigos,
decidido a meterme en ella y coger por los cabezones al \emph{sephardim
Papo} si, por desgracia suya, allí le encontraba. Ya distaba veinte
pasos de la morada de \emph{Riomesta}, cuando vi que de ella salía mi
sabio amigo el rabino \emph{Baruc Nehama}, llenando la calle con su
procerosa estatura y la opulencia de sus barbas patriarcales. Lo mismo
fue verme, que venir hacia mí con los brazos abiertos, y no esperó a
tenerme cogido para echar así la voz tonante: «¿A dó vas, mancebo
voluntarioso? Por el aire que trais y el brillar de tus ojos, me parece
que vienes con ira\ldots{} De aquí no pases, ni te pongas injurioso, que
no has razón para ello.» Contestele que razón me sobraba, y que quería
demostrar que no se juega con un caballero castellano. Pero a mis
atropelladas voces contestó con estas otras de grandísima sensatez:
«Bien sé que eres caballero, y que entre tus antespasados cuentas al
señor Cid, y a otros Cides, como verbigracia el mío señor don Gonzalvo
de Córdoba; pero eso no es al caso, pues nadie ha puesto borrón en tu
caballería\ldots{} A \emph{Yohar} te llevaste contra la ley nuestra y la
tuya, y es de justedad que pierdas lo que allegaste con
latronicio\ldots{} No pienses en traer acá duelos con Papo, que es
hombre de cuenta; y si en la calle te topas con él, él te deseará la
paz, como si topara un buen amigo. Generancio tras generancio, Papo
viene de tu tierra y es judeo-español, de los Acevedos de Plasencia, con
quienes tuvo parentesco el que llamáis don Cristóforo Colón, primer
catador de vuestras Américas de cacia Poniente\ldots{} Ten cordura, ten
agudeza, hijo\ldots{} Yo digo que bien puede agradecer \emph{Yohar} al
\emph{sephardim} que la haiga cogido encariciada de manos de otro. En
ello mostra \emph{Papo} ser varón coronado de virtudes.»

Como yo soltase, al oír esto, una risa burlesca, se incomodó el hombre,
y creyéndose en la tribuna de la Sinagoga, clamó con voces predicantes:
«Con \emph{Yohar} culpaste, desvergonzaste y ficiste fealdad\ldots{}
¡Guai, gente pecadora, pueblo pesado de delictos, semen de
malinidades!\ldots» Estos sacrosantos desatinos agotaron mi paciencia y
me encendieron la sangre. Faltaba, según hoy lo entiendo, menos de un
segundo para que yo le tirase de las barbas al espantajo rabínico. Ello
había de ser entre vituperio y caricia, por consideración a su edad
avanzada; mas no fue de ningún modo, porque en el primer momento de mi
intención, vi que de la casa de \emph{Riomesta} salía un moro elegante:
era \emph{El Nasiry}, hijo de Ansúrez. Quedó el rabino suspenso en sus
declamaciones, yo contenido en mi cólera, y me alegré de no haberle
sacudido la enmarañada zalea de sus barbas. Con respeto, dando
cabezadas, miró \emph{Baruc} al moro, mientras éste decía: «Juan, se
acabaron las bromas. No estamos aquí en España.»

---En España estamos, \emph{El Nasiry}---repliqué yo; y \emph{Baruc} se
dejó decir:---Donnell y Prim han venido a conquistar el suelo del
Maroco, no sus mujeres.

Al hablar así, miraba risueño al moro, solicitando su aquiescencia; pero
mi paisano, con señoril gravedad, no dejó traslucir ningún sentimiento
en su rostro hispano-árabe. Atenazándome el brazo con su fuerte garra,
me ordenó que le siguiese, y el rabino tomó la dirección de su casa, en
la calle próxima, despidiéndose con esta exhortación: «Hazle entrar en
judicio, \emph{El Nasiry}, y que no quite la paz a fijos buenos de
Israel.» Desapareció por una callejuela. Y he aquí que el hijo de
Ansúrez, llevándome por otra, me hablaba con su habitual donosura. «En
tu casa te vestirás con \emph{yoka}, ceñidor y bonete judío, y vendrás
conmigo a donde yo quiera llevarte\ldots{} Y esto sin replicar ni oponer
la menor resistencia, pues si no me obedeces, no serás mi amigo español,
sino un perro vagabundo.» Yo callaba. Por fin, oídas dos, tres veces,
sus recriminaciones, me sentí dominado, sin ninguna fuerza para oponerme
a la despótica voluntad del caballero español y agareno. No diré que
fui, sino que mi tirano me llevó a la que había sido mi casa: allí
\emph{Mazaltob} y \emph{Simi} me proveyeron de la \emph{yoka}, ceñidor y
bonete. Vestido de hebreo, dejeme conducir por \emph{El Nasiry}, que sin
decirme nada me metió en su casa, donde vi aprestos de viaje, mulas bien
enjaezadas, fardos, tienda de campaña\ldots{} No necesité más
explicaciones para comprender que mi amigo partía de Tetuán, y que
consigo quería llevarme de grado o por fuerza. No sé qué sentimientos
embargaban mi alma\ldots{} Mi aflicción por la forzada ausencia quería
buscar consuelo y descanso en la ausencia misma. No sé lo que aquello
era.

Pedí permiso a mi tirano para escribir mis tristes sensaciones de aquel
día; diómelo; tracé con mano rápida y temblorosa esta parte del diario
de mis aventuras; tomé algún alimento, y cual manso cordero me entregué
al que se había hecho mi pastor. Poco antes de partir me habló éste con
severidad, diciéndome que había dado fianza de que yo partía de Tetuán
con propósito firme de jamás volver, y que esperaba de mi honradez que
así lo jurase y cumpliese. Agregó que para responder de mi ausencia
había exigido que me fuesen sufragados los gastos de mi regreso a
España; y al efecto, a mi disposición tenía un remedión de plata y oro,
facilitado por mitad, con gallarda esplendidez, por \emph{Riomesta} y
\emph{Papo Acevedo}. Al oír esto estallé en indignación. ¡Recibir dinero
de judíos por compra-venta del amor de \emph{Yohar!} ¿Eran ellos la
Sinagoga y yo el Iscariote? ¿Olvidaba \emph{El Nasiry} la secular
condición de su raza hasta el punto de creer que un español puede
pisotear la ley de honor, vendiendo por treinta o tres mil dineros a la
mujer que ama? ¡Vileza inconcebible en todo cristiano, y singularmente
en el que ha nacido en la tierra clásica de la dignidad y el decoro!
¡Antes me cortaría la mano que recibir en ella los ochavos viles del
avaro \emph{Riomesta}, del \emph{Papo} cínico, que quiere tapujar con un
puñadito de oro lo que fue mi felicidad y es ahora su oprobio!\ldots{}
Todo esto y algo más dije, derrochando sin tasa las exclamaciones de
enfático orgullo que dan riqueza y sonoridad tonante a nuestra lengua.
Oyome \emph{El Nasiry} con serenidad más musulmana que ibérica, y
comentó mi furia tan sólo con la irónica sonrisa que mantuvo en sus
labios mientras duraron mis roncas protestas en nombre del honor.

«Muy bien, Juanito---me dijo, cuando sofocado yo del esfuerzo verbal
aguardaba su respuesta.---Ya me tenía yo tragado que saldrían a relucir
los Cides y Quijotes\ldots{} Muy señores míos. ¿Cómo va de salud? ¿Y en
casa, todos buenos?\ldots{} Pues en esta tierra, para que te vayas
enterando, poco tienen que hacer los Quijotes y Cides. Y ya que los has
traído contigo, vuélvanse contigo a España\ldots{} Sabrás, hijo mío, que
el honor y la caballería consisten aquí en vivir como se pueda,
guardando la religión y cumpliendo todos los deberes\ldots{} En la
España de la parte acá del mar, no da de comer el honor, ni al dinero se
le mira con mal ojo, venga de donde viniere\ldots{} Te veo muy tonto con
los ascos que haces a la plata de \emph{Riomesta} y de \emph{Natham
Papo}, y nada más hablaremos de ello por ahora. En el camino se hablará.
Hoy te dejo en tu vana jactancia\ldots{} No nos detengamos, hijo mío, y
aprovechemos lo que resta de día para salir de Tetuán. El camino es
largo y dará tiempo a tus reflexiones\ldots{} En marcha.

Montamos en sendas mulas bien aparejadas, formando con los servidores y
arrieros de \emph{El Nasiry} una lucida caravana, y antes de que
arrancáramos, vi que \emph{Mazaltob}, \emph{Simi} y otras judías
faranduleras que me tienen ley, se agrupaban en la esquina del palacio
del Gobernador, y desde allí, temerosas de aproximarse, me despedían con
expresivas garatusas. La presencia de aquellas mujeres, ni santas ni
limpias, me afectó y entristeció sobremanera por las remembranzas que
traían a mi corazón y a mi mente. Mirada cariñosa dejé volar hacia
ellas, y la emoción me obligó a volver el rostro, hasta que me fue
preciso atender a los primeros pasos de mi mula\ldots{} En la extensa
calle que hoy llaman de \emph{Cantabria}, hubo de pararse nuestra
caravana por un entorpecimiento de cargas de leña que zafios montañeses
no acertaban a retirar a uno y otro lado de la vía pública. Mientras
ésta se despejaba, vi pasar un grupo de oficiales, del cual se destacó
mi bondadoso amigo el castrense \emph{don Toro} para venir a saludarme.
Hablamos un ratito; díjele que abandonaba con tristeza la dulce Tetuán
para internarme en el Imperio, y él me compadeció, despidiéndome con
estas palabras: «El Señor vaya contigo, buen \emph{Confusio} (con
\emph{ese}), y te limpie de las confusiones que \emph{Allah} y
\emph{Adonai} han embutido en tu cabeza\ldots{} ¿Qué dices?, ¿que acaso
vuelvas a España? Allí te quiero ver, \emph{Confusio} amigo\ldots{} La
Virgen te acompañe.»

Salimos por la \emph{Puerta de Fez}\ldots{} Adiós, Tetuán, blanca
paloma, virginal doncella que fuiste, antes que el español te cogiera y
manoseara; adiós, \emph{Ojos de Manantiales}, manantial de vida para mí,
pues las amarguras y alegrías, las dulces emociones y acerbas penas que
en ti he sentido, fueron acrecimiento extraordinario de mi sensibilidad,
copiosa reproducción de mis ideas, con lo que parecen multiplicados mis
días y soberanamente hinchada de sucesos mi existencia, como río en que
entran aguas muchas. Adiós, tierra de maldición y de bendición, más, al
fin, de lo segundo que de lo primero, pues bendición es el exceso de
vida en tiempo corto, el ver largo, aprender hondo, y llenar nuestras
trojes con abundantes cosechas de experiencia. Bendito es todo lugar
que, por mucho que se viva, no puede ser olvidado. Hermosa eres, Tetuán,
por el misterio de tus calles, la poesía de tus contornos, por la serena
confianza de las tres religiones que en tu regazo duermen, más hermosa
aún como nido de amores, como alivio y orgullo del hombre enamorado.
Adiós, en fin, dulce \emph{Yohar}, estatua de la blancura, monumento de
ternura, vaso de miel que en su hondura esconde la traición. Yo pido a
mi Jesucristo que te dé la paz, si tu Adonai no quiere dártela.

\hypertarget{v}{%
\chapter{V}\label{v}}

\textbf{Samsa}, \emph{mes de Nissan}.---Feliz ha sido la primera etapa
de nuestro viaje. De Tetuán a esta risueña y patriarcal aldea hemos
venido \emph{El Nasiry} y yo silenciosos, cada cual entretenido en
arrullar sus pensamientos, para que se duerman al compás del andar
cuidadoso de las mulas. En verdad, no he visto mulitas más discretas en
el paso que las de esta tierra; su mansedumbre y la suavidad de sus
movimientos superan a los encomios que todo europeo les tributa. Diríase
que sienten interés fraternal por el ser humano que oprime sus lomos, y
que es para ellas punto de honor llevarlo sano y salvo al término de su
viaje. No quitan los ojos del terreno, como si éste fuera un libro en
que van leyendo el orden y señalamiento de los puntos en que han de
asentar sus cascos duros, dotados de cierta delicadeza pulsátil.

Pues, señor, aún no me ha dicho \emph{El Nasiry} a dónde me lleva. Sólo
sé que la razón de hacer escala en este pueblo es recoger al hijo de un
grande amigo suyo, llamado \emph{Mohammed Requena}, para llevarle con
nosotros. Es este \emph{Requena} un moro de casta granadina, anciano,
rico, bondadoso y de sutil ingenio. El exquisito trato de tan noble
señor serena mi turbado espíritu\ldots{}

Aún no sé cuándo saldremos: el adolescente por quien hemos venido está
enfermo de tenaces calenturas. Titubea \emph{El Nasiry}, solicitado, por
una parte, de su impaciencia, por otra del amor al \emph{Requena}.
Quiere partir pronto a donde le llaman apremiantes intereses, y le
aflige marcharse sin el chico. Han pasado tres días de incertidumbre, de
aplazamientos, de esperanzas no realizadas. Por fin, entiendo que nos
vamos\ldots{} Aún intenta el viejo \emph{Requena} detener algunos días a
su amigo, encareciéndole lo peligroso del tránsito por el valle que
ocupan las tropas de O'Donnell. Una batalla no muy sonada se dio estos
días en \emph{Samsa}\ldots{} Frustradas las primeras negociaciones de
paz, el cañón atronará pronto estos amenos valles. No debemos partir,
según el viejo, mientras no pase la chamusquina. Pero \emph{El Nasiry}
tiene prisa, y confía en llegar al desfiladero del \emph{Fondac} antes
que estalle la tormenta humana, más terrible y asoladora que la de los
cielos.

Partimos al fin. No diré que me alegro, porque la hospitalidad
espléndida que aquí me dan y el trato bondadoso de \emph{Requena} han
sido para mí como un ambiente tibio y sedante, en el cual se marchitan
los sentimientos exaltados, dejando florecer tan sólo la plácida amistad
y la gratitud\ldots{} En esta casa no hay mujeres\ldots{} quiero decir,
no hay más que tres esclavas, largas de edad y cortas de
hermosura\ldots{} ¡Descanso del espíritu; descanso de la idealidad, de
aquel irritable genio, que, como el de la poesía, no enciende las llamas
de su inspiración sino ante la belleza y la juventud!\ldots{} Adiós, paz
nemorosa de \emph{Samsa}; adiós, aldea linda y quieta, de rumorosas
aguas, de frescos naranjales\ldots{} Bendiga Dios las apiñadas flores de
tus almendros, perales y manzanos, para que críen abundante y dulce
fruto\ldots{} Adiós, viejas apacibles, medicina de los delirios de
amor\ldots{} abur, abur\ldots{}

\textbf{Stchaidi}, \emph{últimos días de Nissan}.---Gracias a Dios que
encuentro lugar para escribir con relativo sosiego, y un cierto acomodo
que tiene lejano parentesco con la comodidad. Fatigas y sustos enormes
he pasado; impulsos de huracán me han traído hasta aquí; quebrantado
está mi cuerpo de los golpes y vaivenes; quebrantado mi espíritu de las
terribles emociones\ldots{} Reanudo mi verídico relato diciendo que
salimos de \emph{Samsa} al anochecer, y que serían las diez de la noche
cuando los delanteros de nuestra caravana se pararon, y dieron a nuestro
amo esta voz de alarma: «Señor, no podemos seguir. Están aquí.» Los que
allí estaban eran los españoles: se les conocía por el rugido seco de
las interjecciones castellanas.

Celebraron consejo los guías y \emph{El Nasiry}. Como voy entendiendo el
árabe, pude fácilmente hacerme cargo de lo que decían. No podíamos
encaminarnos al puente sin meternos entre las tropas españolas; habíamos
de ir en busca del vado de \emph{Bu Sfiha}, donde el paso es difícil,
por venir los ríos muy crecidos a causa del deshielo\ldots{} Oídas las
diferentes opiniones, decidió el amo que pusiéramos \emph{pecho al
agua}, pues no había otro remedio, si no preferíamos volvernos a Tetuán
y esperar a que pasase el nublado de guerra. Apechugamos, pues, con el
vado, y ello fue a media noche, con ceguera de nuestros ojos, que a eso
equivalía la obscuridad y temerosa hinchazón de las aguas; paso tan
comprometido como el que intentó Faraón en el Mar Rojo persiguiendo a
los israelitas, con la diferencia de que no nos ahogamos por milagro de
Dios. A mi mula y a mí nos faltó poco para ser arrastrados por la onda;
pero al fin salimos de aquel apuro tomando suelo en la otra orilla. El
pobre animal mostraba con pataditas el contento de verse salvo de su
naufragio.

Pero la desgracia no se cansaba de perseguirnos: en la orilla de
salvación nos salieron al encuentro soldados de Isabel II que nos dieron
el \emph{quién vive}, y nos obligaron a tomar mayor vuelta para
continuar hacia Poniente. \emph{El Nasiry} bufaba de cólera tanto como
yo tiritaba del frío y la mojadura. Pero había llegado la hora de la
paciencia y de la conformidad con el Destino. Siguiendo por el camino
curvo que al pie del monte \emph{Beniber} nos conducía, por donde
pensábamos hallar paso franco hacia el \emph{Fondac}, anduvimos despacio
todo el resto de la noche. Un mendigo desarrapado y viejo que se nos
agregó, nos dijo que el \emph{sbañul} tenía toda su tropa al otro lado
del agua. En Lausie estaba \emph{El Chaue} (entendí Echagüe);
\emph{Z'baalah} \emph{(Zabala)} y Turón en el puente de \emph{Bu-Sfiha};
\emph{Chej El Dónel} y Prim en el monte de \emph{Uadrás}, y en
\emph{Benider} se había plantado \emph{Muley El Abbás} con su ejército
moro, el cual era tan fuerte y aguerrido que allí los infieles
fenecerían de una vez, sin que viviera uno solo para contarlo. ¿Y qué
musulmán creyente podía dudar que ahora la venganza del Mogreb quedaría
consumada, Tetuán redimida de su cautiverio, y los españoles lanzados al
mar para que a nado o como pudiesen se fueran a su terruño?

Sorprendionos el día junto a las avanzadas del ejército marroquí.
Alegrose mi amo de verse próximo a su amigo \emph{Muley El Abbás}, que
sin duda no nos pondría obstáculos para seguir nuestro camino.
Descansamos; fraternizó \emph{El Nasiry}, con aquella gente de variadas
castas, y como yo, por mi traza judaica, era mirado con antipatía y
recelo, mi protector y compatriota el hijo de Ansúrez hubo de decir que
era yo su esclavo. No de otra manera podía designar la especial
servidumbre a que están sometidos los hebreos de las comarcas interiores
del Imperio. Para que estos desgraciados puedan burlar la muerte que a
cada instante les amenaza, cada familia o individuo se pone al amparo de
un señor musulmán, el cual, a cambio de la \emph{guería} o capitación y
de bajos servicios, es protegido con la eficacia suficiente para que
nadie se meta con él. A los que en tal servidumbre viven se les llama
\emph{demmi}, que significa \emph{individuo de un pueblo sometido}, y no
se les da nombre alguno. A cada cual se le conoce por el \emph{judío de
Fulano}. Conforme a las instrucciones de \emph{El Nasiry}, yo fui
\emph{su judío} desde que llegamos al campamento, y para desempeñar muy
al vivo mi papel, me ocupaba en los menesteres más humildes: limpiar las
mulas y darles pienso, fregar los platos, encender la lumbre para hacer
nuestra comida\ldots{} \emph{Ibrahim} y los demás servidores del señor,
aleccionados por éste, me trataban como a un perro; farsa que si por un
lado me molestaba, por otro a gratitud me movía, pues con ella tenía
bien garantizada mi pobre existencia.

Dejándonos en la tienda que un \emph{Kaid de Anyera} nos proporcionó,
\emph{El Nasiry} fue a visitar a \emph{Muley El Abbás}; mas hubo de
volverse sin llegar a la tienda del Príncipe, porque a mitad del camino
le cerraron el paso los movimientos del tropel marroquí. El espantoso
ruido de fusilería nos dijo que había comenzado una fiera batalla. Desde
donde estaba yo, no se veía más que el cortinón de polvo extendido en
los aires, tras un primer término de hombres a caballo que aguardaban
como en reserva. Los gritos de los moros, que comúnmente no saben
combatir sin lanzar a los aires chillería discorde, daban a mis oídos
una descripción vaga de los accidentes de la pelea. El alejarse y el
volver de la onda sonora parecía como el alternado sube y baja de los
favores de la fortuna entre moros y cristianos. Sonaba de un modo el
rumor de los graznidos cuando el Islam avanzaba, y de otro cuando
retrocedía. Hostigado de la curiosidad, avancé entre la muchedumbre de
caballos para echar un lejano vistazo a la refriega, pero a los pocos
pasos retrocedí asustado de mí mismo. Caí en la cuenta de que la mayor
falsedad del papel que yo representaba era mostrar interés por cosas tan
opuestas a la esclavitud como son la guerra, el heroísmo, y cuanto se
relaciona con los aspectos nobles de la vida. Un \emph{demmi} o judío
esclavo debía ser o parecer completamente idiota, cerrado de
inteligencia, grosero y bajuno de sentimientos, so pena de que
descargaran sobre él todas las iras del árabe orgulloso. Volvime a donde
estaba, y en mi rostro puse la estúpida indiferencia de un animal a
quien nada interesan ni nada dicen las grandezas humanas.

Pero transcurrido algún tiempo (no puedo precisar su medida), en aquella
expectación del que escucha y no ve una próxima tragedia, no me valió mi
fingida humildad, y a punto estuve de que me saliera muy cara la
imperfecta comedia de mi esclavitud. Llegaron los primeros heridos
retirados de la acción, unos por su pie, otros traídos en volandas, y al
ver yo que arrojaban en tierra un mísero cuerpo agujereado de balazos
por donde se le iba la sangre; al ver que aún tenía vida y que clamaba
por conservarla pidiendo con desgarradores ayes auxilio y caridad, sentí
que mi corazón cristiano hacia él se iba como las mariposas a la luz.
Nunca lo hubiera hecho. Aún no había yo puesto mis manos sobre aquel
muerto vivo, cuando el empujón de un brazo vigoroso me tiró hacia atrás;
caí de manera poco noble, de espaldas, las cuatro extremidades en alto,
y no bien toqué el duro suelo, vinieron sobre mí sin fin de patas moras,
con babuchas o sin ellas, que me pisotearon y magullaron sin que yo
pudiera valerme. Armas no tenía yo, que si las tuviera ¡vive Dios!, no
me habrían pisado aquellos brutos sin que alguno me lo pagara con su
vida. La mía estuvo en un tris, y mi dignidad fue más que ultrajada con
tantas coces. Ya vi algún yatagán que venía contra mis entrañas y que el
buen Ibrahim apartó con mano diligente\ldots{} ¡Horrible condición la
del judío esclavo en estas tierras, donde ni aun la dulce compasión se
le consiente! Un perro puede aquí amar al hombre, y un esclavo no.

Arrimado a las mulas, como a seres benignos, me hallaba yo, reponiéndome
del quebranto de mis pobres huesos, cuando volvió \emph{El Nasiry}. En
un aparte breve quise contarle mi desgraciado suceso; pero antes que yo
entrase en materia, llevó la conversación a más grave asunto. Díjome
que, en su paseo de vuelta, pudo apreciar que sobre los españoles
llevaban ventaja los moros. Habían éstos entrado en la pelea con brío
extraordinario, alentados por los árabes de \emph{Hiaina}, que aquel día
llegaron con guerrero entusiasmo, y dando el ejemplo de bravura, en todo
el ejército encendieron el furor de la guerra santa. Añadió que desde el
principio de la campaña no habían combatido los marroquíes con tanta
fiereza militar y religiosa. Creyérase que el Profeta mismo había
descendido a las filas desde la región celestial en que mora. Esto me
dijo en lugar donde nadie podía escucharnos, y en él noté una extraña
inquietud y desconcierto del ánimo por la inaudita novedad del
vencimiento de los españoles. Poco después le vi en un grupo de
berberiscos, congratulándose de lo bien que iba la batalla, y dando las
gracias a Mahoma y Allah por la ya segura victoria. Admiré la soberana
perfección de su fingimiento, y de él tomé modelo para instruirme y
doctorarme en el estudio de mi figurada ignominia.

Mediodía era ya cuando el repecho donde estábamos se aclaró de gente,
señal de que los moros ganaban terreno, metiéndose en las posiciones
españolas del llano de \emph{Bu-Sfiha}, llamado por nosotros
\emph{Buceja}. Cierto era que los perros del Islam iban ganando. He aquí
que yo, apóstol humanitario y nada belicoso, sentía ganas de correr
hacia los míos y ayudarles a dar a estos brutos una paliza tal que
fueran todos a contarlo al paraíso de Mahoma. ¡Qué inmensa dicha poder
cobrarles con furibundos pinchazos en el vientre la tremenda pateadura
con que me habían ablandado los huesos!\ldots{} En esto llegó una turba
de los de \emph{Hiaina}, graznando con feroz alegría. Algo pude
comprender de la jerga que hablaban: «Los españoles eran
arrollados\ldots{} Casi no quedaba ya ninguno de aquellos gigantes que
llaman \emph{catalonios}\ldots{} El campo estaba \emph{alfombrado} de
cuerpos cristianos\ldots{} A Prim, que había salido echando bravatas, le
habían abierto en canal dos veces. De otros generales se supo que eran
ya cadáveres, y \emph{Chej El Dónel} tenía rota la cabeza\ldots»

Venían aquellos bárbaros en busca de agua, locos, abrasados por la
sed\ldots{} A una señal de \emph{Ibrahim}, acudimos él y yo con cántaros
llenos que en nuestra tienda teníamos. Yo di de beber al que con más
furia ladraba; después a otro y a otro, todos feísimos, negros y de
espantable catadura. ¿Creéis que me agradecieron el socorro que les di?
No, por Dios: uno de ellos, portador de una vara que parecía de acero
por lo dura y flexible, me apaleó con ella fieramente, y antes de que
acabara, los demás no hallaban mejor modo de expresar su alegría que
abofeteándome con saña y burla. Me obligaba mi esclavitud a poner en
práctica la horrenda humildad ordenada por Jesucristo, que es ofrecer la
mejilla izquierda después de bien aderezada de sopapos la derecha. Yo,
con perdón de nuestro Redentor, no pude hacerlo, y ya tenía cogido por
el pescuezo al verdugo de mi rostro para vengarme de él como pudiese,
cuando un grito de \emph{El Nasiry} me contuvo, y me aseguró con su
afectada cólera la vida que yo ciegamente comprometía. Separándome con
fuerte brazo del lugar de mi perdición, me dijo: «Quítate allá,
\emph{demmi}\ldots{} Tú das de beber a las mulas, no a los hombres de
Dios.»

\hypertarget{vi}{%
\chapter{VI}\label{vi}}

Y he aquí que, pasado aquel sofoco, nos cogió el amo a \emph{Ibrahim} y
a mí en la soledad interior de la tienda, para darnos esta orden
apremiante: «Se confirma que los moros van ganando las posiciones de los
cristianos, pues a cada instante se apartan más de aquí y se corren
hacia \emph{Bu-Sfiha}. Aprovechemos la clara y el despejo del terreno
por esta parte para seguir nuestra marcha. Recoged todo, enjaezad las
mulas, y echemos a correr sin decir nada por las veredas más altas, a
ver si Allah, o el Zancarrón, o el mismo \emph{Eblis} nos permiten
llegar al paso del \emph{Fondac} antes que cierre la noche.»

Tal como lo dijo lo hicimos, y a espaldas de las envalentonadas hordas
de \emph{Muley El Abbás} nos deslizamos por atajos próximos, sin que en
nuestra salida pararan mientes los guerreros que allí quedaban. Tomamos
desde la partida un vivo trote, huyendo de la cruel matanza; mas por
alejarnos rápidamente no perdían nuestros oídos la sensación del inmenso
ruido de la batalla, acrecido al avanzar de la tarde con el pavoroso
estruendo de la artillería española. Mostrábase el cielo poco benigno
con los combatientes, porque al frío seco que desde el amanecer soplaba,
sucedió por la tarde lluvia pertinaz, a intervalos arreciada con
tremendos chaparrones. Cuando nuestras valientes cabalgaduras atacaban
la cuesta que sube a la divisoria de \emph{Djibel Hiamar}, corrían por
aquellos vericuetos las aguas con torrencial sonido, arrastrando
piedras. El camino no merece tal nombre: no es más que un sendero del
cual han sido artífices los cascos de las caballerías. Son aquí más
ingenieros los animales que los hombres.

Momentos hubo en que la ascensión por la pendiente \emph{Aaba-El-Fondak}
era penosa, con su tantico de peligro. En cualquier país que no fuera
Marruecos los caminantes habrían retrocedido, aplazando su viaje para
mejor ocasión. Aquí no se asustan de nada que sea incomodidad, y
aborrecen las carreteras de piso igual y sólido. ¡Y pensar que en
nuestra España ha ocurrido lo mismo casi hasta nuestros días! Por
vericuetos inaccesibles como los que yo he pasado al subir de Tetuán al
\emph{Fondac}, hacían sus grandes viajatas los españoles de generaciones
no lejanas; así caminaban los mercaderes con sus acopios; así las
hermandades y cofradías que transportaban reliquias o cuerpos de santos
incorruptos; así los grandes reyes Isabel y Fernando, en solemne visita
de sus estados, y así las comitivas de princesas que venían a casarse
con algún Felipe o con algún Carlos de los que nos depararon las casas
de Austria y de Borbón. Con estos recuerdos, yo me hacía la cuenta de
que atravesaba las cordilleras de mi enriscada España, en alguna
expedición política o comercial, entre Castillas\ldots{}

Tan ceñudo se puso el cielo a media tarde, y tales cantarazos de lluvia
descargó, que la impedimenta que llevábamos, cuatro acémilas con cofres
de ropa, sacas de víveres y material de tienda de campaña, quedaron
rezagadas por no poder vencer la pendiente con la pesadumbre de sus
cargas. En lugar áspero donde la montaña nos deparó una oquedad rocosa,
buen amparo contra el furioso aguacero, dispuso \emph{El Nasiry} que
hiciéramos alto, lo que las mulas y yo agradecimos sobremanera. Allí nos
paramos y acogimos, no sólo por resguardar nuestros rostros de los
furibundos latigazos de la lluvia, sino por dar tiempo a que pudieran
las retrasadas acémilas rebasar la pendiente y agregarse al cuerpo de la
caravana. Tan inquieto estaba nuestro amo, que daba miedo ver su cara,
el fruncimiento de sus cejas, y aquel mover y apretar de mandíbulas,
cual si mascando estuviera una cosa muy amarga. En verdad, maldita
gracia tenía que se nos perdieran una o más cargas del convoy con lo que
llevábamos para nuestro sustento, amén del dinero y materia comercial de
algún valor.

Por fin, a la media hora de angustiosa espera, vimos llegar a uno de los
jayanes con la mula que conducía, chorreando agua los dos. Lo primero
que dijo fue que otra carga venía detrás, a corta distancia, y que las
dos restantes quedaban en los repechos más bajos aguardando a que
cediera el temporal. Echó de su boca \emph{El Nasiry} sin fin de
maldiciones en lengua arábiga, y alguna en español neto de las más
trepidantes; y cuando yo me permitía consolarle del contratiempo con
vulgares razones, como la confianza en la Providencia y otras del orden
anodino, el arriero soltó esta grave noticia que a todos nos dejó
suspensos: «Señor, sabrás que la ventaja de los moros se ha trocado en
derrota y palos. El cañoneo de los españoles ha traído a éstos la
ganancia de la lid, y ahora, con permiso y ayuda de los malditos
diablos, están barriendo como con escobas el campo que habían
conquistado los creyentes.» Puso \emph{El Nasiry} al oír esto la cara de
compunción hipócrita que tiene para estos casos, y exclamó mirando al
cielo tempestuoso: «Cúmplase la voluntad de Allah\ldots{} Suframos,
¡ay!, el castigo que merecemos por nuestros pecados y la flojedad de
nuestra fe. ¡Loor siempre al Clemente y Misericordioso!» Yo me puse
también la máscara de una grande aflicción y dije \emph{amén},
reconociendo así que por nuestros pecadillos consentía el Sumo Dios la
tremenda paliza que los cristianos administraban a estos zopencos.

Trajo el segundo arriero la noticia de que se había iniciado la retirada
de las tropas moras, corriendo hacia la montaña. El cañón español no
cesaba de aventar las tribus del Mogreb. Era un espectáculo de horrible
desolación\ldots{} así como la fin del mundo\ldots{} Había resucitado
Prim, saliendo de un montón de muertos, y con una quijada de caballo
mataba cuantos moros cogía por delante. Los demonios hacían visajes
horribles combatiendo en las filas cristianas, y Mahoma chillaba en los
aires, con \emph{tronío} y \emph{llorío} que era como la ira de Dios en
medio de las nubes\ldots{} Nuevas exhortaciones de \emph{El Nasiry} a la
conformidad y paciencia. Ya podíamos ver bien claras las resultas de
tanto pecar y de habernos descuidado en la oración y enfriado en la
creencia. \emph{Amén}, \emph{amén}\ldots{} En el sitio donde estábamos,
que era como caverna de poca hondura, llegaba a nuestras orejas con
intervalos el fragor de la artillería cristiana, según las idas y
venidas del viento. Después de traer el espanto a nuestros oídos, lo
alargaba para otra región, llevando a oídos distantes la misma sensación
pavorosa. Dijo el segundo arriero que los moros en retirada avanzaban
subiendo. Era una ola de mil colores mojados, un rebaño de miles de
patas que huía del llano al monte, entre fango, bajo cortinas de agua, y
acosado por el fuego.

Sabido esto por mi amo, fue más viva la expresión de su inquietud: le
vimos atormentado por cruel duda; tan pronto tomaba una resolución, como
de ella se arrepentía. Por fin, se arrancó a decirnos: «Aunque perdamos
las dos acémilas que se han quedado atrás, debemos seguir hacia el
\emph{Fondac}\ldots{} con toda la prisa que se pueda\ldots{} y allí, si
la ola que viene tras de nosotros, y que hasta el \emph{Fondac} no ha de
parar, nos permite algún descanso, lo tomaremos. Si no, adelante
siempre, y Allah nos guíe y nos socorra. En marcha todo el mundo.»

¡En marcha, huyendo de la ola y tomándole la delantera cuanto fuese
posible! La parte del camino que nos faltaba para coger la divisoria del
riscoso \emph{Djibel Habib}, era la más fatigosa y endiablada. Entramos
por un desfiladero angosto y torcido en innumerables vueltas y dobleces,
siempre subiendo; a nuestra derecha, montes altísimos de donde se
desgajaban torrentes de agua arrastrando piedras; a nuestra izquierda,
vertiente de barrancadas que acaban en invisibles abismos\ldots{} Iban
las cabalgaduras una tras otra, pisando con singular cautela el suelo
pedregoso y húmedo. Admiré en la mía el pasmoso instinto con que
sorteaba las pendientes resbaladizas. A veces posaba su casco tan
delicadamente como si bailara un minueto con las más remilgadas
etiquetas que ilustran el arte de mover los pies. ¡Apreciable persona
cuadrúpeda, o animal apersonado, manso, discreto, cumplidor exacto del
más penoso deber, sin otra recompensa que un poco de cebada! Ya era
noche obscura cuando franqueábamos la divisoria; llegamos a un punto en
que los abismos que antes veíamos por la izquierda abrían sus negras
bocas por la derecha. Cesó la lluvia, y el viento helado campaba por sus
respetos en los caballetes del monte. Íbamos ya cuesta abajo. Las mulas,
inducidas a mayor cuidado por la obscuridad, andaban con más lentitud,
tanteando el suelo\ldots{} Por fin, al cuarto de hora de descenso, vimos
a la izquierda un cuadrado regular, construcción chata que blanqueaba en
las tinieblas. Era el \emph{Fondac}\ldots{}

Era el indecente y destartalado parador, en que el \emph{Majzen}, o
Gobierno central, atiende al descanso y refacción de viajantes y
caballerías. La estructura y disposición del edificio me recordó los
corrales que dan abrigo a los rebaños de toros o de ovejas en las
sierras y descampados de nuestra Península. Cuatro paredes en
rectángulo, no muy altas; en la del frente una puerta; en el centro un
patio claustrado de tejavanas; a los lados de la puerta dos estancias
donde vive el administrador, funcionario del Estado; basura, montones de
paja, obscuridad de noche, frío y polvo siempre, componen el
desmantelado edificio. Concluyen de arreglarlo y le dan la última mano
de pintoresca barbarie las turbas que por horas o por días lo habitan.
Cuando nos apeamos frente a la puerta, vi que en el fondo del corral
pestañeaba la luz de un candilejo; la luz se fue acercando, trayendo
detrás de sí a un árabe caduco y medio cegato que saludó a \emph{El
Nasiry} como a un antiguo conocimiento. Al entrar, vimos sombrajos de
caballerías y algunos bultos de moros tumbados en el suelo.

Ordenó mi amo que se diese un pienso a nuestros animales sin quitarles
las monturas, pues habíamos de partir al instante; pidió al guardián
café caliente, entramos en uno de los cuartuchos laterales, amueblados
exclusivamente con paja, para que cada cual, según los modos o
costumbres de su indolencia, se tumbase y estirase. Tan inquieto y
abrasado en zozobras estaba mi amo, que cuando el vejete nos trajo el
café, servido en vasos humeantes, no se cuidó de catarlo. Yo sí lo hice
porque me sentía transido y desmayado. \emph{El Nasiry}, según me dijo,
apartar no podía de su mente la idea y la imagen de aquella ola del
Mogreb derrotado y huido. Hacia donde estábamos vendría la ola, pues no
había más camino de fuga que el que seguíamos, ni en dicho camino más
reposo que el maldito \emph{Fondac}.

De improviso, estando él y yo en estos pensamientos y melancolías, oímos
ruido al exterior, que no era del viento, sino de caballerías
galopantes, y de voces al parecer humanas o de diablos que hablaran a
estilo de los hombres. No pudo contenerse \emph{El Nasiry} y salió,
salimos a la puerta. Lo que llegaba era la ola, sus primeras espumas
salpicantes. Dos moros se apearon: venían manchados de sangre y lodo,
pintadas en el rostro la ira, la ansiedad, la desesperación; sus
caballos negros blanqueaban del sudor, y apenas podían valerse ya, mal
sostenidos por sus remos temblorosos. Apenas se apearon los dos primeros
jinetes de la ola, vimos llegar a otros dos, y como al medio minuto,
seis más en caballos derrengados ya del furioso correr, los vientres
heridos y rasgados por las espuelas\ldots{} Quisimos volver a nuestro
albergue y asegurarnos contra la invasión; mas la curiosidad de ver la
ola engrandeciéndose a medida que avanzaba, nos detuvo en la puerta. Los
primeros que a pie llegaron fueron tres, con resoplido de peatones que
ganan el premio en la carrera; tras ellos aparecieron cuatro; luego, de
golpe, como unos veinte, seguidos de tres a caballo: uno de estos
jinetes venía mal herido y medio muerto. Antes de que lo bajaran del
caballo, se cayó él como un fardo, y al rebotar en el suelo, dio señal
de agónica vida en voces roncas\ldots{} Aterrados entramos mi amo y yo
en el corral, y al punto nos obligó a salir de nuevo un gran vocerío,
clamor inmenso, como si todos los gemidos del dolor humano se tradujeran
al lenguaje de la mar brava revolcándose en la playa pedregosa. Era la
plenitud del ejército en dispersión, que a lo alto del monte llegaba ya
con el imponente hervir de su cólera despechada, y la espuma de las
maldiciones que escupía contra la tierra y el cielo.

\hypertarget{vii}{%
\chapter{VII}\label{vii}}

Ya no había salvación; nos ahogamos en la onda de salvaje humanidad,
empujada del pánico, del hambre y de toda suerte de locura\ldots{} Ya no
podíamos andar por dentro ni por fuera del inmundo corralón, sino con
esfuerzo y braceo de nadadores, abriendo hueco entre la carne sudorosa.
El aliento de la masa humana nos asfixiaba; el rumor de cólera y rabia
nos enloquecía. Ya mi amo y yo, forcejeando en el interior, no
encontrábamos a los criados moros, ni las caballerías, ni el café que
habíamos dejado a medio tomar; ya íbamos y veníamos llevados de la onda;
ya, por los gritos que proferían tantas bocas feroces de blancos
dientes, y por la expresión terrorífica de tantos rostros negros y
blancos, bruñidos del sudor, llegábamos a creer que también nosotros
veníamos huidos del combate, y que traíamos en nuestras almas la furiosa
rabia de la derrota.

Quiso \emph{El Nasiry} congraciarse con los que más cerca teníamos en
aquel penoso braceo en medio de la onda, y algo les dijo de la batalla y
de lo mal que se había portado Allah con sus fieles creyentes. Los que
le oían respondieron con voces famélicas más que patrióticas: tenían
hambre, y querían repararse con algún alimento hasta que pudieran llegar
a sus casas en remotos aduares. Otros vociferaron contra O'Donnell y
Prim, renovando la ridícula leyenda del pacto entre españoles y
demonios. Ya tenían los moros sometidos a los cristianos; ya el campo de
éstos era \emph{una alfombra} de cadáveres, cuando se desgajó el cielo
vomitando diablos; resucitaron los cristianos muertos, y el Mogreb
vencedor fue vencido por máquina sobrenatural\ldots{} En la fuga, los
heridos que traían fueron abandonados en el monte, donde los cuervos se
encargarían de comérselos tranquilamente. ¡Felices los muertos porque
subirían al paraíso de frescas aguas cristalinas!

Logramos al fin topar con \emph{Ibrahim}. Éste nos dijo que antes que él
pudiera evitarlo le habían quitado y abierto el fardo de una de las
acémilas, el cual, como era cosa de condumio, pasó en un santiamén a las
bocas voraces y a los estómagos hambrientos. No se incomodó \emph{El
Nasiry} al oírlo; antes bien mostrose conforme con el despojo,
asegurando que a su intención caritativa se habían anticipado los
ladrones\ldots{} En tan apretada situación estábamos, sin poder entrar
ni salir, ni recoger lo nuestro, ni escaparnos de tanta confusión y
laberinto, cuando llegaron a nuestros oídos voces muy distintas de las
desesperadas voces de la onda. Al mismo tiempo se arremolinaron los que
llenaban el ancho corral, abrieron paso, y pude ver a un negro
\emph{bokari} que látigo en mano apartaba a un lado y otro la bárbara
plebe, sacudiendo sin compasión sobre los estrujados cuerpos. Tuve la
desgracia de que el látigo de aquel sayón me cogiera de lado a lado la
cara, haciendo saltar de mi cabeza el bonetillo que la cubría. Lastimado
de tal injuria, oí decir claramente al zurrador que diéramos paso y
fácil entrada en el corralón al poderoso señor tal y tal, que venía de
parte del Sultán para tratar guerra y paces.

Abriose al fin en la masa cavidad suficiente para que entrase un morazo
montado en mula de tan alto aparejo, que el hombre parecía cabalgante en
una torre. Tras él entraron cuatro más, caballeros en airosos corceles,
y le seguía una escolta que en su mayor parte hubo de quedarse fuera.
Con tal cuña, ya estábamos los de dentro en punto de ahogarnos de
verdad. La suerte fue que el del zurriago, antes que su altísimo señor
se apease, trató de despejar el local gritando: «fuera, fuera, canalla:
dejad hueco al señor\ldots» También a mí me tocó buena parte de esta
nueva zurribanda. En fin, salió la chusma del corral, a borbotones o
chorros, como el agua de sucio estanque al cual se abren las compuertas,
y desde este punto ya respiramos y nos esponjamos, y yo pude hacerme
cargo, por el escozor de mi piel, de los desastrosos efectos del látigo.

Pero como es invariable ley humana que vengan siempre enlazadas y
cogidas del brazo las bienandanzas y las desdichas, sucedió en aquel
caso que tras el peligro de ahogarnos en la ola de los vencidos, vino la
suerte y buena coyuntura de que mi amo \emph{El Nasiry} y aquel pomposo
sujeto, emisario del Sultán, fuesen amigos. No hay que decir cuánto me
alegré de verles saludarse y hacerse graciosas zalemas, celebrando su
encuentro. Entraron luego los dos en el primer aposento donde estuvimos,
y recostáronse en la paja muelle, único diván y revolcadero de personas
que allí existía. Quedeme yo en el corral, entre caballos y mulas, y
hasta la madrugada, cuando ya salíamos de aquel infierno del
\emph{Fondac}, no pude saber quién era el caballero del blanco alquicel
tan bien escoltado de moros elegantes.

Dos o tres veces me recitó \emph{El Nasiry} el rosario de los nombres de
aquel señor, los que apunto cuidadosamente para que ninguno se me escape
de la hebra en que van engarzados. Llámase el \emph{Kaid Abu Abdal-lah},
\emph{Mohammed Sen Dris Ben Hammam El Ferrari}. Según cuenta, Su
Majestad el \emph{Sultán Sidi Mohammed Ben Adderrahman}, viendo el mal
cariz que tomaba la guerra, le llamó, y dándole sesenta mil ducados con
que remediar al ejército, ordenole que al campamento se trasladase, y
examinara el estado de ánimo y disciplina de las tropas, para ver si
convenía proseguir la campaña o rematarla de plano con las más
ventajosas paces que se pudieran obtener. Iba, pues, \emph{El Ferrari} a
tomar el pulso al enfermo, y por cierto que le encontraba dando las
boqueadas, menos necesitado de medicinas que de los últimos Sacramentos.
Sin duda el buen señor se haría cargo, por la desolación que allí veía y
por lo que debió de contarle mi amo, de la soberbia tunda que aquella
misma tarde había sufrido \emph{El Mogreb}, y de la necesidad de acudir
pronto al descanso de la paz, que el marroquí desea, y al español no le
vendrá mal.

La oportuna llegada de aquel fantasmón fue venturosa para nuestra
caravana, porque, despejado el patio, pudo mi amo recoger lo que quedaba
de lo suyo y disponer que partiéramos inmediatamente. Esperanzas no
teníamos ya de que pareciesen las dos acémilas que nos arrebató la ola
en medio de la cuesta. La que desvalijada fue en el \emph{Fondac} quedó
en menos de un tercio de las vituallas que transportaba. Sólo permanecía
completa la que llevaba el material de la tienda, ropa y algo de plata.
Con pérdidas tantas, ya podía dar gracias a Dios nuestro amo \emph{El
Nasiry} por haber salvado las vidas de todos en aquel terrestre
naufragio. Reunidos los sirvientes para la marcha, aún tuvimos que
aguantar casi a obscuras dos chubascos más sobre los ya sufridos. El uno
fue la plática larguísima del señor moro \emph{El Ferrari}, uno de los
hombres más habladores que he visto en mi vida. Por su caudal oratorio,
le creímos enviado de Mahoma para implantar en el \emph{Mogreb} el
sistema parlamentario. El otro chaparrón nos lo proporcionó un
\emph{Kaid} de Fez, que vino en las últimas aguas de la ola y que
resultó, como el otro, amigo de mi amo. Traía toda la rabia y
resquemores de la derrota; pero también una honrada sinceridad digna de
las mayores alabanzas. Hartándose del café rico con que obsequió a todos
\emph{El Ferrari}, dijo que los españoles habían hecho un esfuerzo
grande para vencer, y que estaban cansados; pero que no había medio de
luchar con ellos mientras \emph{El Mogreb} no tuviese una mediana
organización militar, y trenes de Artillería con personal entendido que
la manejara y sirviera, así en el llano como en los pasos de montaña.

Urgía, pues, según \emph{Ben Hair}, que así llamaban al de Fez, negociar
una paz decente, para que volvieran los cristianos a su casa, y
recogidos los moros en su solar, pensaran luego en adestrarse y
prevenirse por si aquéllos volvían con nuevas pretensiones de conquista.
Tal como hoy están las cosas, no puede el moro resistir las embestidas
del cristiano, pues si perversa es la religión de éste como inspirada
del Infierno, tiene en cambio artillería magnífica con la cual se
remedia de la desventaja de su religión. La musulmana, que es única
religión verdadera, no excluye los cañones, ni se opone al uso y buen
gobierno de estas terribles máquinas; que bien claro nos dice el Profeta
en su santo libro: «Sé ferviente en la oración, y Allah pondrá en tus
manos el rayo con que podrás aniquilar al incrédulo.» Con la voz
\emph{rayo} significó Mahoma piezas de grueso y mediano calibre de los
mejores sistemas que los mismos incrédulos inventan y perfeccionan para
guerrear unos con otros\ldots{} Dichas estas cosas atinadas, tan del
gusto de todo buen musulmán, nos dio cuenta minuciosa de la batalla,
refiriendo los designios, los movimientos, las astucias y ardides de
ambos combatientes, historia que no reproduzco porque no me tachen de
prolijo y fastidioso. Nada olvidó \emph{Ben Hair} de la pericia de Ros,
Echagüe y Zabala, de la bravura temeraria de Prim, del tino y dirección
admirable de O'Donnell. Reconocía las grandes dotes de sus enemigos, y
los encomiaba sin quitar a los suyos su parte de heroísmo y de
conocimiento, con lo que nos hicimos cargo los oyentes de la belicosa
acción a que los moros dan el nombre de \emph{Bu-Sfiha}, y los españoles
el de \emph{Uad Ras}, o más propiamente \emph{Uadrás}.

Contaré ahora las obscuras tragedias mías y mis personales batallas, que
no serían conocidas de ningún cristiano si yo no las escribiese aquí
para desahogo mío y recreo del bonísimo Beramendi. Sabed, oh lectores
fingidos y sin razón inventados por mi pluma, sabed que, dispuesta la
partida, me ordenó mi amo, en la puerta misma del \emph{Fondac}, que
diese de beber a las mulas. Obedecí; llevé mis bestias al costado
exterior del edificio, por el Este, donde están el pozo y abrevadero, y
cuando quise sacar agua, vi dos espingardas arrimadas al brocal, y sobre
él un espadón unido al tahalí. Con todo respeto cogí las armas para
colocarlas en otro lado\ldots{} ¡Cristo Padre! Nunca tal hubiera hecho.
Aún no había puesto mi mano pecadora en aquellos instrumentos que sin
duda eran sagrados, cuando una fiera con trazas de hombre saltó de en
medio de la obscuridad, como tigre que acecha en el matorral, y dándome
un fuerte manotazo, al que acompañaron las voces de \emph{ladrón},
\emph{perro} y no sé qué más, me derribó al suelo. Apenas caído,
salieron no sé si tres o cuatro bestias humanas, y me levantaron en vilo
sin que yo pudiera defenderme ni desasirme de tantas brutales manos que
me cogían\ldots{} Reuniéronse al instante muchos más, en número que a mí
me pareció legión de demonios, y con griterío infernal, en habla
riffeña, me pasearon en alto, éste me coge, éste me suelta, de todos
golpeado, zarandeado y escarnecido\ldots{} A mis voces acudieron
\emph{Ibrahim} y otro de los servidores de \emph{El Nasiry}; mas nada
podían dos hombres piadosos contra quince o veinte desalmados, que sólo
tenían de humanidad el habla y la figura, y aun sobre éstas habría mucho
que decir\ldots{}

Pues nada menos querían aquellos monstruos que tirarme a una cisterna
que a poca distancia del pozo abre su siniestra cavidad entre rocas. Yo
no sabía que existiera aquel abismo hasta el momento en que, suspendido
sobre él por las manos de mis verdugos, vi su temerosa hondura, y en el
fondo un espejo de agua inmóvil, que reproducía el cielo, y en él la
media cara de la luna que aquella noche entre celaje y celaje nos
alumbraba. Fue un instante no más, dos segundos o tres de terror y
angustia indefinibles. No caí al hondo, donde habría perecido, porque mi
desesperación se agarraba con ferocidad a los cuellos, a los brazos de
los mismos que querían arrojarme, porque hice presa con los dientes en
alguna oreja, en algún trapo de turbante, y porque, al fin, mi noble amo
acudió a mi vocerío angustioso y al veloz llamamiento de \emph{Ibrahim}.
Salvado fui de milagro, y esto lo debí a los astros del cielo más que a
\emph{El Nasiry} y a \emph{El Ferrari}, que resultaron, por lo que voy a
decir, instrumento providencial del prodigio de mi salvación.

Pues sucedió que mi amo y el noble mensajero del Sultán habían salido a
la puerta a percatarse del firmamento, del cariz de la luna, de la
dirección del rabo de la \emph{Osa}, que los árabes llaman \emph{Aldebb
al Akbar}, de las alturas a que estaban sobre el horizonte otros grupos
de estrellas, de la situación de \emph{Júpiter o Marte} (no sé cuál) con
respecto a las figuras zodiacales. Era \emph{El Ferrari}, según supe
después, muy experto en la astronomía empírica, y no pasaba noche sin
que examinara los espacios siderales, no sólo por gusto de la
contemplación de lo infinito, sino por atisbar los signos que relacionan
el cielo y sus aspectos con los destinos humanos. Estaba, pues, \emph{El
Ferrari} dando a mi amo lección astronómica o astrológica, ayudado de un
palo con que iba señalando cada familia estelar, y su sagaz conocimiento
marcaba las señales anunciadoras de la paz entre España y el Mogreb,
cuando llegaron a los dos señores mis gritos angustiosos y las voces de
\emph{Ibrahim}. Corrió primero \emph{El Nasiry} a donde yo clamaba,
pendiente sobre el abismo, mi vida separada de mi muerte por el espacio
de un segundo, y quitándole a su amigo el palo con que a las estrellas
apuntaba, con él dio en las espaldas de mis verdugos, echándoles por
delante furibundas imprecaciones. A esto debí la vida\ldots{} Y yo
pregunto ahora: «¿qué hubiera sido del pobre Juan, si en el momento de
salir yo con las mulas para darles de beber, no hubieran salido también
los señores al campo raso, para escudriñar con miras mágicas los
espléndidos signos del firmamento?» Por eso he dicho que las estrellas
me salvaron\ldots{} Algo tiene la magia cuando me vi obligado a
bendecirla. ¿Cómo no, si de ella estuvieron pendientes mi vida y la paz
del Mogreb?

\hypertarget{viii}{%
\chapter{VIII}\label{viii}}

Y que no tardé poco, ¡Dios me valga!, en reponerme de aquel espanto. No
se vuelve de los bordes de la muerte sin que quede nuestra ánima
suspensa y aterrada por algún tiempo. Miraba la media cara de la luna en
el cielo, jugando al escondite entre celajes, y su claridad me daba
horror, como cuando la vi en el fondo de la cisterna, llamándome a
terrible agonía en las dormidas aguas. Diéronme a beber café, que me
reparó con su calor el estómago y todo el organismo. Vi con gratitud el
rostro amigable de \emph{El Nasiry}, a la luz del candil que nos
alumbraba en la estancia guarnecida de pajas hediondas; vi también el de
\emph{El Ferrari}, advirtiendo entonces que el buen señor es tuerto, y
maravillándome de que con un ojo solo pueda escudriñar los espacios
celestiales, y leer en ellos los obscuros enigmas de la Humanidad.

Pero nada me dio tanto gusto como ver y oír que ambos señores se
despedían uno del otro, señal de que partíamos de aquel \emph{Fondac}
que, si no era ya mi Infierno, había sido mi Purgatorio, del cual salía
mi alma bien purgada y limpia de cuantos pecados en la blanca Tetuán
cometí. Siglos se me hacían los minutos que aún tardamos en apartarnos
del horrible parador. Mentira me pareció que perdía de vista la casa
inmunda, el pozo, la horrible cisterna y sus aguas dormidas, que fueron
espejo en que me asomé y vi la eternidad. Adiós, \emph{Fondac}
lúgubre\ldots{} ¡Que no me muera yo sin recibir la noticia de que te ha
reducido a escombros un terremoto, o a cenizas un rayo del
cielo!\ldots{} Tan batanado, tan dolorido estaba mi cuerpo del diluvio
de golpes y porrazos, tan agobiado de ansiedad y terror mi espíritu, que
difícilmente podía tenerme en la silla. Desde Samsa no había dormido, ni
en mi cuerpo había entrado más alimento que algunos sorbos de
café\ldots{} A cada instante encontrábamos grupos de moros que
regresaban a sus aldeas después de la batalla, unos con la espingarda al
hombro, otros inermes, todos andrajosos y escuálidos, con la tristeza
pintada en el rostro. Al pasar junto a ellos, creía yo que me miraban
con ira, como queriendo repetir en mí los pasados ultrajes. Yo dije a
\emph{El Nasiry}: «Menos temo en esta montuosa soledad a los chacales y
hienas que a los hombres. Lleguemos pronto a donde yo encuentre descanso
y paz.» Mi buen amo me tranquilizó con dulces palabras.

El alba sonrosada nos aclaró el camino a la hora y media de salir del
\emph{Fondac}. Bajamos por despeñada cuesta; dejamos a la izquierda los
caminos de \emph{Arsila} y \emph{Alkazar-Kebir}. El paso descendente de
la mula, como tanteo de peldaños de desigual altura, me molestaba lo
indecible, desguazándome todo el esqueleto\ldots{} Vadeamos multitud de
arroyos que bajaban turbulentos, batiendo en la carrera sus aguas
achocolatadas. Y aquel paso entre pedregales no tenía fin. Ansiaba yo
llegar al llano, que veía bajo las pisadas de mi mula; pero el llano no
quería dejarse pisar, y burlaba la ansiedad de mis ojos hundiéndose más
a cada paso que dábamos hacia él\ldots{} Por fin, dormitando yo sobre la
mula, llegamos a un lugar donde se hizo alto. Allí descansé un poco; me
revolqué en el suelo, como los burros cuando se les libra de la albarda;
comimos algo, y otra vez al tormento de la montura y del andar
cadencioso. Llano adelante, vimos los montes que arriba se quedaban
arrogantísimos con turbante de nubarrones. Contemplándolos tan hermosos,
les eché mi despedida con la firme intención de no volver a pasar por
ellos. Nada digno de contarse me aconteció en el resto del día, hasta
que llegamos a esta aldea situada en medio de un extenso prado, donde se
resolvió pasar la noche y reposar las molidas osamentas.

En \emph{Stchaidi}, donde escribo, hallamos un amigo y cliente de
\emph{El Nasiry}, que no nos permitió armar la tienda, ganoso de
aposentarnos en sus admirables chozas con techo de paja, las cuales eran
mejores que algunas casas de Tetuán. Debió de decirle mi amo quién era
yo y la razón del tapujo hebreo que llevaba, porque \emph{Bu S'liman},
que tal es el nombre de aquel simpático y amable moro, me aposentó en un
cuarto muy bueno, a flor de tierra sí, pero desahogado y limpio. La
puerta era tan chica, que tenía yo que entrar a gatas. En un costado de
la estancia me armaron cama blanda en horizontal nicho guarnecido de
azulejos, y para mayor sorpresa mía pusiéronme una mesilla de ocho patas
con utensilios de escribir, lo cual significaba que me tomaron por poeta
o literato. No fue superior, pero sí abundante, la comida que nos
sirvieron en otra choza más grande que la mía, rodeada de higueras,
tártagos y mimosas. Reparé yo mi estómago, y luego me metí en el nicho,
de donde por mi gusto no hubiera salido en tres días. Dormí menos de lo
que me pedía el cuerpo; pero como \emph{El Nasiry} resolvió prolongar la
estadía para tratar con \emph{Bu S'liman} y otros moros de un negocio de
ganado, tuve tiempo de escribir dos o tres horas, y de coger después el
sueño, empalmando sabrosamente la segunda tarde con la segunda mañana.
¡Ay, qué contentas quedaron mis carnes, y con qué devoción dieron
gracias a Dios mis huesos atormentados!

\textbf{Tánger}, \emph{fines de marzo}.---Aquel \emph{Bu S'liman} que
nos hospedó en \emph{Stchaidi}, es alto, rubio, entrado en los sesenta
años, saludable y fuerte, con sólo un achaque de la vista que le obliga
al uso de antiparras de vidrios obscuros y gordos montados en cuerno.
Dos chicos que nos servían de comer mostraban también en sus ojos la
pitaña, mal endémico sin duda en aquel terreno pantanoso. Mujeres vi a
lo lejos en chozas distantes, gordas, con tapujo de tela grosera y
blanca, dejando ver las piernas coloradas de ancho tobillo. Unas lavaban
ropa, y otras la tendían en retamas\ldots{} No sé por qué me pareció
renegado el tal \emph{Bu S'liman}. No hablaba o hablar no quería lengua
española; pero bien pude apreciar que la entiende. Al despedirnos, me
hizo no pocas reverencias, singular contraste con las vejaciones que en
las etapas anteriores del viaje sufrí\ldots{} Tal vez el muy guasón de
\emph{El Nasiry} le ha dicho que soy algún rico personaje español
disfrazado, o cercano pariente de Isabel II, que vengo a tomar nota de
las grandes riquezas naturales del Imperio y de la suave condición de
sus habitantes.

Con el descanso del cuerpo volvieron a mi ser la perdida inteligencia y
la perdida fluidez del discurso. Así, cuando caminábamos hacia Tánger,
por las lomas de suelo arenoso, entablamos mi amo y yo conversación
amena, que de uno en otro tema nos hacía olvidar sabrosamente la tediosa
longitud de la marcha. Tuve yo especial gusto en hacer recuerdo y
enumeración detallada de los ultrajes que recibí en el campamento y en
el \emph{Fondac}, pintando con vivos colores el gran peligro en que vi
mi existencia.

---En rigor, no debí yo acudir a salvarte---dijo \emph{El Nasiry},
socarrón,---porque hallándote tan desesperado por la infidelidad de la
blanca \emph{Yohar}, más me hubieras agradecido el dejarte morir que el
defenderte la vida. Los despechados de amor suelen en España curarse de
su pena con un tiro en la sien o puñalada en el corazón, y otros hay que
a la guerra van a que los maten. No debes, pues, alegrarte de tu
salvación, sino sentirla. Mejor estarías ahora en el fondo de la
cisterna del \emph{Fondac}.

---No, no, amigo \emph{Nasiry}, que aunque la traición de \emph{Yohar}
me destrozó el alma, y quedé muy afligido y dado a los demonios, no era
tanto que apreciase mi vida en menos que el amor de la judía blanca.
Necedad habría sido dejarme ir al Infierno o al Purgatorio, mientras
\emph{Papo Acevedo} se quedaba riéndose de mí\ldots{} En el
\emph{Fondac}, entretuve mi mente algunos ratos con la blancura y
suavidad de la piel de \emph{Yohar}; pero si mil cosas dulces y amargas
pensé de ella, no me pasó por las mientes ni por el corazón el deseo de
volver a tomarla si el maldito \emph{Papo} quisiera devolvérmela.

---Naturalmente---replicó mi amo y amigo;---que la caballerosidad y el
honor, en los cuales veo yo como alambres o palitroques que componen la
armadura de tu persona para mantenerla tiesa; el honor, digo, y la
fanfarrona caballerosidad, no harían pocos remilgos si tú volvieras a
tomar a la blanca paloma después de papujada por su segundo dueño el
\emph{sephardim}. ¡Buenos se pondrían tus antepasados si faltaras así al
decoro y te pasaras por debajo de la pata los timbres gloriosos!\ldots{}

---Abre los oídos, \emph{El Nasiry}---dije yo,---para que me oigas bien
lo que quiero contarte. Déjame que sea franco y que me vuelva atrás de
lo que aquella tarde desembuché tocante al honor y la caballería. No
tengo inconveniente en asegurarte que los vejámenes y atropellos que he
sufrido me han hecho bajar la cresta de mi orgullo. Bien claramente veo
que no somos nada, y que no existen otros males verdaderos más que el
perder la vida, ser matado en plena juventud. Y si quieres que llegue a
los extremos de la sinceridad, abre más los oídos y entérate de que
cuanto te dije para rechazar los dineros de \emph{Riomesta} y de
\emph{Papo}, debes tenerlo por no salido de mis labios\ldots{} Pues
siendo yo pobre como las ratas, y viéndome sin mujer y sin ningún medio
de ganar la vida, ¿qué menos puedo hacer que tomar lo que me den,
agradeciéndolo, si no a ellos, a ti, que has sido el promovedor de este
donativo? Dame, pues, el socorro que para mi huida previnieron aquellos
hermanos de Judas Iscariote.

---Eso sí que no haré---replicó \emph{El Nasiry}, extremando su guasa
hasta los mayores disimulos,---porque me lastimaste echando sobre mí,
con palabras amañadas, la nota de entrometido y tercero; lo que me llegó
tanto al alma, que ni te perdono tu lenguaje insolente, ni te doy los
dineros, que ahora quedan para mí. Ya me advirtieron \emph{Papo} y
\emph{Riomesta}, al entregarme la bolsita, que si en ti notaba
repugnancia de coger dinero de judíos, me quedase yo con la bolsa y te
abandonara con desprecio a tu pobreza enfatuada.

---Pues yo te juro, \emph{El Nasiry}, por la salvación de mi alma, que
no siento ya la menor repugnancia de tomar esos ochavos de plata y oro,
ni creo que se ha de manchar mi mano al cogerlos.

---No, no, que ahora me salen a mí el honor y la caballería de un rincón
donde los tiene guardados mi alma española, y aunque salen con algo de
polilla y olor de cosa descompuesta, traen bastante poder para decirte
que te fastidies por haberme ofendido\ldots{} Tan caballero soy como tú,
y poco va de Marruecos a España.

---Tú harás lo que quieras, \emph{El Nasiry}---le dije poniéndome al
tono de su marrullería.---La gratitud me hace tu esclavo. Si es tu gusto
guardarte la bolsita que \emph{Riomesta} y \emph{Papo} te dieron para
mí, hazlo en buen hora. Pero si acaso mudaras de voluntad y se te
metiera entre ceja y ceja que yo tome la bolsa, venciendo para ello mi
repugnancia, aquí me tienes dispuesto a satisfacer tus deseos,
encerrando bajo siete llaves los escrúpulos que te lastimaron. Así lo
juro, y te lo firmaré con mi sangre si fuere menester. En nuestra tierra
dicen: \emph{cuando pasan rábanos, comprarlos}\ldots{} ¿Has olvidado
este refrán?

---De sabiduría tomada de los rábanos, sólo recuerdo aquella que dice
que no debemos tomarlos por las hojas.

Interrumpió nuestro coloquio la vista de Tánger que de improviso a
nuestros ojos hubo de presentarse en una vuelta del camino. Quedé yo
suspenso ante la ciudad mora, toda blanca, recostada en una colina
verde; pero mucho más me sorprendió y recreó la imponente faja de mar
azul que vi súbitamente surgir entre el cielo y la tierra. Era el
\emph{Estrecho}, que en aquel momento me pareció el \emph{Ancho}, por
creer yo que había más agua de lo regular entre los dos continentes, y
que debían estar menos separados \emph{Mogreb El Andalus} y
\emph{Mogreb-El-Aksá}. El aire diáfano aproximaba los contornos
distantes. Señalando la costa frontera, \emph{El Nasiry} me dijo: «Allí
tienes la tierra de la caballería y del honor. ¿Ves aquel caserío que
blanquea en la orilla del mar? Es Tarifa, donde Guzmán llamado el
Bueno\ldots{} ya sabes\ldots{} Corre la vista hacia la izquierda, y
verás blanquear otro pueblo. Es Conil\ldots{} más acá verás un
cabo\ldots{} Es Trafalgar, donde los ingleses\ldots{} ya sabes\ldots»

¡Hermoso espectáculo!\ldots{} ¡Confusión grande de los ojos y de la
mente!\ldots{} ¡En tan corto espacio, cuánta Historia!

\hypertarget{ix}{%
\chapter{IX}\label{ix}}

\textbf{Tánger}, \emph{Abril (rija de nuevo el almanaque
cristiano)}.---Ya estoy en la ciudad marroquí del Estrecho, la más
arrimada a la civilización europea, aunque sólo reciba de ella
sensaciones de vista y olor que no llegan al alma\ldots{} Pero dejo esto
para mejor ocasión, que en la presente debo contar mi llegada, mi
instalación en la morada de \emph{El Nasiry}. Quedaron las caballerías
en una casa próxima a la puerta por donde entramos, y el hijo de
Ansúrez, seguido tan sólo de \emph{Ibrahim} y un servidor, se dirigió a
su vivienda por empinadas calles que conducen al alto en que está la
Alcazaba. Nos apeamos junto a una puerta humilde; hízose cargo de las
mulas \emph{Ibrahim}, y el amo y yo entramos a un patio ni grande ni
bonito, sin adorno de tracería ni frescura de plantas. Salió a
recibirnos la esclava \emph{Maimuna} y con ella se internó \emph{El
Nasiry}, ordenándome que en aquel patio le esperase. Un ratito estuve
allí solo y aburrido, hasta que vi venir al amo, que, llevándome al
portal y metiéndose conmigo en un desmantelado aposento donde no había
cama, ni sillas, ni mueble alguno, ni más descanso que el de un poyo con
el revoco desconchado, me habló de este modo: «Esta casa, alquilada para
poco tiempo, no tiene comodidad para mi familia ni para mis huéspedes.
La dejaremos en cuanto la paz nos permita volver a mi casa de Tetuán. Mi
hospitalidad, como ves, es bastante mezquina; pero confórmate hijo, pues
no hay otra cosa. Adecentaremos este cuartucho con alfombras y tapices,
y el poyo, guarnecido de buenas mantas, te servirá de lecho. Se te
pondrá una mesilla o cualquier trebejo donde puedas escribir; se te
proveerá de tintero y papelorio\ldots{} Comerás conmigo alguna vez en
aquella estancia que ves al otro lado del patio, con puerta labrada de
alfarjía, o comerás aquí solito, servido por \emph{Maimuna}. Baño
tendrás también cuando lo pidas\ldots{} Dispensa, hijo, que no sea más
espléndido; pero ya ves\ldots{} soy aquí ave de paso, y no he podido
encontrar mejor nido.»

Haciendo gala de la humildad y gratitud que me correspondían, le dije
que su hospitalidad, con sólo ofrecerme un techo y un pedazo de pan, era
mucho más de lo que yo merezco. Y él entonces, sentándose en el poyo
junto a mí, me soltó lo más interesante y pertinente del sermón que
preparado traía. Aquí lo copio: «No porque mi hospitalidad sea mísera,
impropia de mi posición, dejaré de suplicarte que correspondas tú al
amparo que te ofrezco. No estará bien que, dándote yo asilo, saques tú
ahora las mañas españolas y cristianas, burlando la confianza que pongo
en ti. Olvida que eres \emph{de la otra banda}, de que yo también lo
fui, y dame palabra de respetar los hábitos morunos, que yo guardo y
reverencio desde que los adopté con libre voluntad. Te lo diré más
claro, Juan: aquí hay mujeres; yo tengo mis mujeres, y los moros las
guardamos del apetito y de la vista de los extraños. Ya sabes que esto
es así, y no me pondrás en el caso de enseñártelo de otro modo.
Recogidas están las hembras en la parte de la casa que se las destina, y
allá viven solas, sin más salida y desahogo que la azotea, en donde por
las tardes se solazan. Estando tú aquí, las obligaré a mayor escondite,
prohibiéndolas que asomen las narices a este patio, y aun que curioseen
en las celosías altas que desde aquí ves, y por cuyos huequecillos
puedes ser visto. Si a ellas las guardo, a ti con mayor rigor te
amonesto para que en ninguna manera traspases la puerta por donde entrar
me viste; tampoco esta otra del ángulo derecho, donde hay una
escalerilla que sube al piso alto. Mucho cuidado, Juan. Cada país tiene
sus dogmas, y yo, al acomodarme a la vida mora, he abrazado esta
religión de las costumbres, y antes me dejaré morir que faltar a ella o
consentir las faltas de los demás en mi propia casa. Tenlo
entendido\ldots{} y no te digo más.»

---¡Oh!, \emph{El Nasiry}---exclamé con dignidad.---¿Cabe en ti la
sospecha de que yo cometa acción tan vil? ¡Burlar yo tu hospitalidad!
¡Abusar de tu confianza! ¿Por quién me tomas, \emph{El Nasiry}, o
Gonzalo Ansúrez, para hablar en cristiano?

---¡Ah!\ldots{} No dudo, no dudo de tu honradez\ldots{} Pero\ldots{} por
si acaso, Juan, por si acaso, te hago las advertencias que has oído,
pues nadie hay en el mundo que esté libre de una mala tentación\ldots{}
Desconfiados somos los que profesamos la fe de Allah, ley de pura
desconfianza\ldots{} y cartuchera en el cañón.

---Como toda ley que gobierna el alma: prohibiciones y más
prohibiciones, lo que pone a los fieles en el trance de infringir alguna
vez que otra\ldots{} Pero éste es un caso de honor, de amistad y de
compañerismo. Ten de mí la seguridad que tendrías de un hermano.

---Sí que la tengo; pero me pongo en guardia, y así es mayor mi
seguridad. No olvido, Juan, que tus amigos españoles te llaman
\emph{Confusio}, con lo que indican que está en tu naturaleza el
confundir las cosas, sin que sepas remediarlo\ldots{} Puede suceder que
un día te levantes con los sentidos trastornados, y sin darte cuenta
confundas lo cristiano con lo moro\ldots{} y recaigas en la gran
confusión española, que es respetar lo ajeno si se trata de dinero o
alhajas, y no respetarlo si se llama mujer. Para el español no hay ley
de tuyo y mío cuando se encapricha por una hembra suelta o atada, con
dueño o sin él\ldots{} Podías tú, con muchísima honradez, irte del
seguro, y por eso te aviso que estoy a la mira\ldots{} Y punto final,
que para los dos basta con lo dicho.

Reiteradas mis protestas de fidelidad, volvió mi amo a sus quehaceres en
el interior de la casa, y yo, tendiéndome en el poyo sobre la blandura
de tapiz y mantas que me trajo la diligente \emph{Maimuna}, me entregué
al descanso con la quietud y descuido de quien tiene asegurada la
pitanza y un techo. Al siguiente día, diome la esclava el café y pan que
necesitaba para mi desayuno, y luego vino \emph{Ibrahim} con un traje
español que para mí había comprado a un ropavejero judío. Grima sentí al
ver el odioso pantalón, un levitín de paño y un chaleco rameado, que me
parecieron prendas de malísimo corte, en mediano uso todavía, no mal
apañadas de zurcidos y arreglos. Trabajo me costó meter mi cuerpo en
aquellos andrajos de la civilización, tan diferentes de los airosos
trajes árabe y hebreo a que se habían hecho mi rostro y mis carnes; pero
al fin me vestí a la europea, que tal era el deseo de mi protector. En
la cabeza, no disponiendo aún de sombrero adecuado, me puse un fez, y di
con mi cuerpo en la calle, ansioso ya de ver la ciudad a que me habían
traído mis africanas aventuras.

Si gana Tetuán a Tánger por el misterioso laberinto de sus calles y por
la grandeza y frescura de los montes y vegas que la circundan, ventaja
lleva este pueblo al otro por la majestad del mar, en cuya orilla está
edificado, y por la diligencia de tanto comercio y del entrar y salir de
mercancías. Incansable y curioso recorrí toda la población, dominándola
de un extremo a otro. Vi el \emph{Zoco grande}, concurrido de tantos
mercaderes y de la pobretería pintoresca de \emph{derviches}, juglares,
mendigos y fascinadores de serpientes; admiré el \emph{Marchan} con
lindas casas europeas; descendí por la calle principal al \emph{Zoco
chico}, hervidero de judíos, de españoles y de otros europeos que han
traído las modas haraganas de cafés y cantinas; seguí hasta el puerto,
donde vi los cárabos y faluchos que hacen la navegación del Estrecho, y
algún vapor de Marsella o Gibraltar; vi la Aduana opulenta con
tantísimos ganapanes afanados en el mete y saca de fardos y cajones;
salime luego por la puerta que da paso a la playa; corrí por las arenas
de ésta, viendo la cáfila interminable de moros campesinos que llegan
diariamente al mercado seguidos del burro y la familia, con cargas
míseras de carbón o de leña, y por allí anduve largo rato considerando
cuán intensa y lacerante es la pobreza de este pueblo marroquí, y qué
poco alivio recibe de la civilización europea, por la castiza
inflexibilidad y resistencia del carácter berberisco. La valla de su
religión le separará siempre del resto del mundo, aun cuando todo el
mundo viniese a ocupar su suelo. Así vemos que, con raras excepciones,
pobreza y barbarie se mantienen aquí tan dueñas de la vida como en los
pueblos y aduares de tierra adentro, al pie del huraño Atlas.

De vuelta a mi morada humilde, invitado fui a la mesa de mi protector,
que en realidad no era mesa, sino una baja tarima, junto a la cual él y
yo en el santo suelo nos sentamos, a usanza mora, entre cojines blandos.
Nos servían \emph{Ibrahim} y \emph{Maimuna}, tomando los platos de unas
manos blancas o morenas, que detrás de una cortina se parecían, con el
espacio y tiempo precisos para dar y coger loza, y sin que más allá de
las manos pudiéramos distinguir ningún pedacito de brazo ni menos de
rostro. Comimos estofado de carnero con tanto aroma de especias, que más
regalaba el olfato que el gusto; unas como albóndigas ensartadas en
palitos, todo ello con el indispensable \emph{kusk-sú} o
\emph{alcuzcuz}, que amañábamos en pelotillas. Siguieron los pasteles
dulces llamados \emph{el macrod}, y otras especies de almíbares o
mermeladas empalagosas. Hacían los dedos de tenedores y cucharas,
suciedad que pronto se remediaba con el lavar de manos en perfumosas
aguas. Luego nos dieron té, más moro que chinesco, con hojitas de
yerbabuena flotantes en la infusión abrasadora. Trajo \emph{Ibrahim} las
pipas o fumaderas, que yo acepté porque no eran del maldito \emph{Kif},
sino de buen tabaco de Gibraltar; y en esto se reclinó \emph{El Nasiry}
sobre el cojín que tenía por el lado derecho, y fumando y sonriendo, con
un tonillo agridulce y socarronas pausas, me dijo:

---Mientras comíamos, observaba yo que tu curiosidad no tenía descanso.
Te traían sin sosiego las manos que veías en aquella puerta soltando y
cogiendo platos\ldots{} Por ser esta casa tan menguada, que en ella
falta espacio para todo, has visto esas manos; que si la casa fuera como
la de Tetuán, ni sombra de tales manos verías\ldots{} Y vete curando de
esas mañas fisgoneras, buen \emph{Confusio}, pues nada absolutamente has
de ver, y cuanto menos mires, más tranquilo estarás.

---Mi curiosidad por las manos que se aparecían y ocultaban---le
respondí,---no tiene nada de maliciosa. Tú me has contado que posees
tres mujeres, y que la preferida, la verdadera esposa, se llama
\emph{Puerta de Dios}.

---Así es: en árabe su nombre es \emph{Bab-el-lah}.

---Sin la pretensión de ver a tu esposa, pues sólo el pretenderlo sería
impertinencia grave, yo te digo que deseo tu felicidad y la de esa
señora, como la de tus esclavas, que son también tus mujeres\ldots{}

---La una es \emph{Quentza}, la otra \emph{Erhimo}. Ni a estas dos, ni a
\emph{Bab-el-lah}, has de verlas por mucho que aguces el filo de tu
curiosidad. Yo te hablé de ellas porque con alguien había de desahogar
mi alma en los días de ausencia. Yo amo a mi familia, y mis mujeres y
mis hijos me absorben todos los pensamientos cuando estoy lejos de casa.

Después de una pausa en que los dos mirábamos silenciosos los giros del
humo de nuestras pipas, mi protector y amigo me dijo que si nunca podría
yo ver a sus mujeres, no tenía inconveniente en mostrarme sus hijos.
Poco después aparecieron, traídos por \emph{Maimuna}, un niño como de
cinco años y una niña como de siete, tan lucidos y graciosos que quedé
absorto contemplándolos. A uno y otra acaricié, extremando mis afectos
en la niña, llamada \emph{Luz-il-lah}, y recreándome en el gran parecido
que le encontré con su hermosa tía. La pureza de facciones, el divino
conjunto del rostro, la proporción y medida de todas las partes del
cuerpo, igualmente se mostraban en la hija y en la nieta de Ansúrez. El
niño era también muy lindo, de color moreno aceitunado, esbelto de
talle, los ojos ávidos de penetración, con un brillo que me recordaba
los de Vicentito Halconero. A la chiquilla le caía tan bien el trajecito
de mora, que no podía yo imaginarla vestida de otra manera. Llevaba un
caftán finísimo listado de amarillo, faja colorada, aros de oro en las
orejitas, y en la cabeza un bonete de terciopelo rojo; los pies desnudos
en babuchas con la punta encorvada. \emph{Alí Ben Sur} llevaba la menor
cantidad de ropa, luciendo así su varonil gallardía. No me hartaba de
besarlos, y hablé con ellos todo lo que pude, valiéndome del poquito
árabe que yo sé y del corto número de voces españolas que ellos conocen.
Su padre, alelado de orgullo, y cayéndosele la baba, repetía la cariñosa
queja de todos los padres, así moros como cristianos: «Ah, son muy
malos\ldots{} No se les puede sujetar\ldots{} Todo el día están
alborotando\ldots{} Me vuelven loco\ldots»

Contemplando con mayor arrobamiento el rostro de la encantadora niña,
dije a \emph{El Nasiry}: «En tu \emph{Luz-il-lah} veo todos los rasgos
de la noble raza de Ansúrez. Veo algo más: otra raza escogida, superior.
O mucho me engaño, o la madre de esta niña es una mujer espléndida,
hermosísima.» Y \emph{El Nasiry}, poniendo los ojos en blanco para dar
toda la expresión posible al encomio, me respondió: «Tan hermosa es,
Juan, que no parece criatura mortal, sino ángel del Cielo. No hay en
ninguna lengua palabras con qué describir y cantar tanta belleza. Como
poeta que eres, podrás imaginarla; verla nunca podrás.» Y dicho esto,
con musulmán gesto ordenó a los niños que se retirasen.

\hypertarget{x}{%
\chapter{X}\label{x}}

Bien sea porque las prohibiciones reiteradas de \emph{El Nasiry} me
movieran a mayor deseo de lo prohibido, bien porque la holganza diera
más espacio a mi curiosidad, ello es que yo quería violar el secreto de
aquel oculto mujerío, no por quitarle nada a mi protector y amigo, ni
por meterme a seductor de moras, sino por verlas, nada más que por
verlas, y dar a mis ojos el sabroso espectáculo de tan interesante
aspecto del vivir musulmán. Singularmente aguijaba mi curiosidad aquella
\emph{Puerta de Dios}, belleza única y soberana, al decir de su dueño,
la cual no tenía semejante más que entre los ángeles y serafines. Ánimas
benditas, ¿cómo sería aquella \emph{Bab-el-lah}? ¿No me depararía Dios
la ventura de ver y apreciar una de sus creaciones más admirables?
Bastaríame con una rápida visión de tan sobrehumana belleza, la cual por
su perfecta y divina forma no habría de despertar en mí ni el más leve
destello de lo que llamaba don Quijote \emph{incitativo melindre}.

En estas ideas y deseos estuve todo el día siguiente al del comistraje
con \emph{El Nasiry}. Hallándome fastidiado en mi ratonera y habiendo
escrito ya todo lo que tenía que escribir, salí a pasearme al patio con
más gusto del que me daban los paseos y vueltas por la ciudad, donde
poco había que me cautivase, pues todo lo tenía bien visto y examinado.
Ocasión es ésta de decir que de mi fastidio era responsable mi
protector, pues en Tánger me retenía sin otra razón que no haber llegado
el vapor que debía llevarme a Cádiz. Otros vapores anclaban en el
puerto; pero iban a Gibraltar o a Marsella, y El Nasiry no quería
embarcarme sino para puerto español. Y entre tanto que así me tenía
prisionero sin ofrecerme ningún solaz en su casa ni fuera de ella, no me
daba los dineros de \emph{Papo} y \emph{Riomesta}, que sin duda me
habrían servido para que yo buscase algún regocijo en la ciudad. «No te
doy la bolsa---me decía,---porque la vaciarás estúpidamente aquí, y no
hemos de pedir al marido y al padre de \emph{Yohar} que te la llenen
otra vez.» Sin blanca me aburría en mis paseos por Tánger. Nunca me
llevaba \emph{El Nasiry} a la carrera de sus negocios, ni tenía yo
ningún amigo judío ni cristiano de quien acompañarme.

Pues como decía, salí a pasearme al patio silencioso, sin que ninguna
distracción encontrase en mi ir y venir de animal enjaulado. Mas no
sucedió lo mismo a la tarde siguiente, porque me dio por mirar a las
altas celosías, y la realidad o mi deseo me hicieron ver sombras o
bultos que tras los huequecillos se movían. Mi fantasía loca fingió en
algún instante que asomaban por el enrejado los fulgurantes ojos de
\emph{Bab-el-lah}; mas en realidad nada que a humanos ojos se pareciese
distinguieron los míos. Lo que sí puedo asegurar es que, más avanzada la
tarde, oí cuchicheos, voces arábigas que desmenuzadas e incomprensibles
salían por aquel tamiz. No quise yo ser menos que las escondidas moras,
y a sus risas correspondí con lo que me pareció más propio:
exclamaciones de sorpresa, posturas airosas, miradas interrogativas que
partirían los corazones más duros\ldots{} Pero aquel inocente juego tuvo
pronto su fin: oí la voz bronca de \emph{Maimuna}, y poco después, tras
de las celosías no había cuchicheos ni sombrajos; las mujeres con su
guardiana y los chiquillos se habían subido a la azotea. Quedé yo
consolado de mi fastidio y con esperanzas de nuevos sucesos que abrieran
camino para una sabrosa aventura.

Por la noche, después de cenar con \emph{El Nasiry}, que nada me dijo de
mis telégrafos con las moritas, señal de que nada supo, me acosté mecido
por mi imaginación en vagorosas ilusiones, y soñé que en mí se
reproducía la historia del Cautivo contada por Cervantes en el
\emph{Quijote}. En el patio de mi hospedaje vi el \emph{baño} de Argel,
donde me tenía prisionero el bárbaro renegado \emph{Azan bajá}, y por
las celosías vi asomar la caña con que la misteriosa \emph{Lela Mariem}
me manifestaba ser yo el preferido entre los demás cautivos; vi los
movimientos y signos de la caña, y ésta, por fin, me entregaba un papel
con amorosos requerimientos escritos en lengua arábiga\ldots{} El día me
despabiló y encendió más en mis románticos deseos, y cuando me lancé a
mi vagar vertiginoso por las calles, pensaba en la posibilidad de una
aventura gallarda, y me decía: «De fijo, lo primero que ha de
preguntarme Beramendi será si he logrado penetrar en un harem y ser
dueño de sus poéticos arcanos. Lo menos que pensará el buen señor es que
he logrado quebrantar la misteriosa clausura, sobornando eunucos o
cortándoles la cabeza, y que en un dos por tres he arrebatado lindamente
a la odalisca más hermosa para traerla a mi amor, primero, después a la
fe de Cristo nuestro Redentor\ldots{} Muy desairado será para mí
desengañar al Marqués, y declararle que ni he visto harenes más que por
el forro, ni he violentado sus puertas, ni menos he sacado ninguna
odalisca como no sea en sueños.»

Llegó por fin la tarde, que era la más propicia ocasión de mis
travesuras, porque siempre, de tres a siete, estaba ausente \emph{El
Nasiry}. Antes de salir al patio, me puse en acecho de los más
significantes rumores que vinieran de las celosías; algunos oí, que me
parecieron de animada conversación; salí de mi camarín, anduve con
cautela por el patio, miré a lo alto, no sin esperanza de ver asomar la
caña de \emph{Lela Mariem}, y de pronto hirió mis oídos un grande
estrépito de pasos, golpes, carreras, chillidos de mujeres y llanto de
chiquillos. ¡Terrible trapatiesta se armaba en el harem! Sin duda las
moritas se tiraban de los pelos, o se azotaban con furia las sonrosadas
carnes. ¡Qué ocasión más bonita para subir a ponerlas en paz! Tanto
arreció el tumulto que me alarmé de veras, llegando a creer que había
fuego en las habitaciones altas\ldots{} Sí: fuego debía de ser\ldots{}
¡fuego chillaban las espantadas voces! Movido de un sentimiento
humanitario, sin pensar más que en la salvación de mis semejantes, y
libre mi espíritu de aquel melindre del serrallo y sus odaliscas, corrí
a la escalerilla del rincón, cuyo ingreso está defendido por una puerta;
empujé ésta sin acordarme de la prohibición de \emph{El Nasiry}; entré,
subí, salté los primeros peldaños, y aún no había llegado a la mitad de
la empinada escalera, de un tramo solo, fatigoso y largo, cuando bajó
con veloz descenso, a trompicones, la esclava \emph{Maimuna}, viniendo a
chocar contra mí. Si no la sujetaran mis brazos cuidando de guardar mi
equilibrio, habríamos bajado los dos de cabezas.

En el brevísimo instante de mi violento abrazo con \emph{Maimuna}, vi en
lo más alto de la escalera una mujer de gigantesca estatura, negra como
el ébano, de hocico largo y labios bozales. Apenas pude apreciar en su
poca ropa una tela listada de rojo y blanco; en su cabeza la pincelada
chillona de un pañuelo encarnado; en otra parte brillo de aretes, de
ajorcas, de no sé qué áureos metales; vi sus largas piernas desnudas; vi
el bulto enorme de sus pechos\ldots{} y viendo esto y algo más con
brevedad de relámpago, oí la voz de la negra giganta increpando a
\emph{Maimuna}, y oí también la réplica de ésta. Me bastó el poco árabe
que sé para entender el diálogo airado entre la mujer de hocico de mona
y la vieja esclava. Recriminaba la de arriba con el dejo de quien ha
pegado antes de recriminar. Parecía decir: «Puedo más que tú, bribona,
ya lo has visto, y te deshago de un puñetazo. Atrévete conmigo\ldots{}
¿Crees que me dejo pegar como estas pobres tontas?» Y dijo la esclava:
«Guárdate, \emph{Bab-el-lah}, que ya sabrá \emph{El Nasiry} tus
maldades\ldots{} Guárdate, \emph{Bab-el-lah.»}

No me dio tiempo la vieja para pensar ni decir cosa alguna, porque, como
digo, bajamos los dos hechos una pelota. Aún pude ver, en un segundo
relámpago más breve que el primero, otro bulto de mora que rápidamente
pasó tras de la negra. Distinguí sólo un caftán listado de verde y
blanco\ldots{} No vi si era blanca o morena la que lo llevaba; oí
sollozos, una retahíla de denuestos contra \emph{Maimuna}\ldots{} Ésta
me empujó, apenas llegamos al fin de la escalera, y desfogando en mí su
cólera, lanzome al patio diciendo: «¿Tú, \emph{Yahia,} qué tienes que
ver en esto? Guárdate si \emph{El Nasiry} sabe que has visto\ldots{} ¡Si
sabe\ldots{} pobre ratón \emph{Yahia!} Escóndete, vete a la calle.»
Antes que yo pudiera responderle, corrió al comedor bajo, de donde salió
al punto con un manojo de llaves. Cerró la puerta de la escalerilla y se
fue hacia el segundo patio, gruñendo y echando maldiciones. Su cara era
un muestrario de arañazos.

Por muchas razones estaba yo turbado y lelo; pero mi mayor confusión
provenía del descubrimiento y hallazgo de la divina \emph{Bab-el-lah},
la cual no podía ser otra que la ferocísima negra que yo había visto,
verdadera mula en dos pies. Dos veces habíala llamado por su nombre la
esclava. No podía dudar que era ella, la predilecta de \emph{El Nasiry},
ni que en éste debía yo ver el primero y más salado guasón del mundo.
Imposible que aquella giganta jimiosa fuera madre de la linda criatura
\emph{Luz-il-lah}, quien sin duda nació de otra predilecta anterior, o
de una esclava blanca\ldots{} Pensado esto, me puse a reconstruir
lógicamente el gran alboroto mujeril en cuyo final intervine a tontas y
a locas, y de mi mental trabajo resultó esta hipótesis razonable.
\emph{Maimuna} es \emph{jarifa}: llaman así a las esclavas viejas de
probada lealtad, a quienes el moro confía el gobierno y disciplina de
sus mujeres, ya sean esposas, ya esclavas. Tiene la \emph{jarifa}
autoridad para dirigirlas en sus ocupaciones, que más bien son
pasatiempos, en sus lavatorios y afeites; tiene poder para obligarlas a
guardar la debida concordia, para castigarlas si riñen. Sin duda,
\emph{Maimuna} desempeñaba estas funciones asistida de un vergajo con
que vapuleaba las carnes blandas de las odaliscas sin hacerles gran
daño; seguramente las tres mujeres de \emph{El Nasiry} no vivían en
completa paz\ldots{} Imaginaba yo que la horrenda \emph{Puerta de Dios}
formaba sola un bando poderoso contra las otras, para mí de estampa
desconocida, y que comúnmente se ponía de parte de \emph{Maimuna} cuando
ésta tiraba de vergajo. Pero también discurrí que como negra y favorita,
podía proceder en sentido contrario, favoreciendo a las débiles contra
la bárbara tiranía de la \emph{jarifa}, no menos cruel que un cómitre de
galera.

No necesito decir que desde que tal pensé, me interesaron vivamente las
otras dos mancebas que imaginaba tiernas, blancas y graciosas,
verdaderas flores de serrallo. Por mi desgracia, yo no podría ofrecerles
mi protección, ni aun siquiera verlas, pues el lance de aquella tarde
apretaría más el encierro y sujeción de las pobres muchachas. Temblando
estaba yo de que \emph{El Nasiry} se diese por enterado de mi intento de
subir al harem. Ya tenía yo preparada mi disculpa razonable: «Pensé que
ardía la casa\ldots{} ¿Cómo no acudir a sofocar el incendio?\ldots» Pero
mis temores se disiparon aquella noche frente al amo, que nada me dijo,
señal de que no le habían contado el lance. Esto me alentó en mis
románticos ensueños. Por la noche me escarbaban el corazón no sé qué
punzaditas que traduje en esperanzas, y éstas se aproximaron enormemente
a la realidad en la tarde siguiente, cuando, hallándome en mis soledades
del patio, vi que por los huecos de la celosía asomaban tres blancos
dedos, a punto que rasgaba los aires un siseo dulcísimo, como caricias
que en mi oído hiciera la voz de los ángeles\ldots{} La sorpresa y
emoción me dejaron inmóvil y mudo.

No eran blancos, como he dicho, sino amarillos, los dedos que en la
celosía me hablaban un lenguaje enloquecedor; la natural blancura
desaparece bajo el tinte que se dan las moras en manos y pies con una
hierba llamada \emph{el henna}\ldots{} Púseme bajo la celosía, esperando
alguna voz que me aclarase el obscuro lenguaje de los amarillos dedos, y
vi que éstos se doblaban en la dirección de la puerta de la escalerilla.
Corrí hacia la puerta\ldots{} tuve buen cuidado de no dar golpes en ella
ni hacer el menor ruido, pues lo que hubiera de pasar, forzosamente
requería silencio absoluto. Apliqué mi oído a la cerradura y a las
maderas; esperé largo rato. Ligera sacudida estremeció la puerta\ldots{}
luego sentí\ldots{} no diré una voz, sino aliento que por el agujero de
la llave salía gozoso en busca de mis oídos. No sentí lo que aquel
aliento decía. Con audacia donjuanesca me lancé a iniciar el coloquio de
intrigante amor: «¿Eres tú, hermosa \emph{Quentza?»} pregunté con
susurro. Y de dentro vino como un suspiro la respuesta: «No: soy
\emph{Erhimo.»}

\hypertarget{xi}{%
\chapter{XI}\label{xi}}

No sé qué habría dado yo en aquel instante por poseer el árabe, para
expresar de corrido y sin ningún tropiezo todo lo que se me ocurría.
Pero por mis pecados, ni yo era capaz de sostener conversación tan
importante con secreteo al través de una puerta, ni de lo que decía
\emph{Erhimo} llegaba a mi entendimiento más que alguna que otra frase
suelta: «\emph{Bab-el-lah mala}, \emph{Maimuna} mala\ldots{} yo mucho
padecer.» No era esto poco. Como pude, evocando todo mi saber arábigo,
logré decirle que abriese la puerta, y desde dentro vino una retahíla de
la cual pude entresacar estas palabras: \emph{llave\ldots{} dormida
Maimuna}\ldots{} \emph{miedo}\ldots{} \emph{Bab-el-lah}
\emph{despierta}\ldots{} Yo traduje que aunque la esclava dormía, no
osaba quitarle la llave, porque la negra, que es muy mala, estaba
despierta\ldots{} Propuse yo entonces que abriera por la noche. «De
noche no\ldots{} Miedo\ldots{} \emph{El Nasiry}\ldots» fue su
respuesta\ldots{} En efecto: buena la armábamos si el amo nos
sorprendía\ldots{} «Mañana» dijo ella claramente, y yo repetí:
«\emph{mañana}.» Quería yo hablar a todo trance, y no pudiendo decir lo
que debía, conforme a las circunstancias y al desarrollo lógico del
diálogo, me lancé a la descarada emisión de lo que sabía, viniera o no a
cuento.

Con esta idea, traje a mi feliz memoria un \emph{Prontuario de la
conversación hispano-árabe}, donde adquirí mis primeros conocimientos de
esta hermosa lengua, y escogiendo ante todo una sarta de adjetivos y
nombres usuales que en Tetuán me aprendí de memoria, y aplicándolos a mi
interlocutora invisible, los fui metiendo con voz melosa por el agujero
de la llave. Véanse estos ejemplos: «Eres \emph{dulce}, \emph{Erhimo},
como la \emph{miel}, \emph{gallarda} como la \emph{palmera}, \emph{azul}
como el \emph{cielo}; eres \emph{rosa} y \emph{clavellina}; eres
\emph{jardín de delicias}, y no hay \emph{estrella} como tus
\emph{ojos.»} Luego, sin darme reposo, enjareté las cláusulas lisonjeras
y amables que sabía: «A tu lado vuelan los instantes\ldots» «Me alegro
mucho de que estés buena con toda tu familia\ldots» «¡Qué hermoso día
hace!\ldots» «Vámonos de paseo\ldots» ¡Y era de noche!

No me salió mal la prueba de mi \emph{Prontuario}, porque \emph{Erhimo},
tomando por espontánea la frase última, me dijo con sollozo: «Yo pasear
no\ldots{} soy esclava\ldots» y luego siguió con una larga relación en
que pude pescar palabras sueltas como: «\emph{El Nasiry}\ldots{}
Allah\ldots{} veneno\ldots{} zapatero\ldots{} dinero\ldots{} dolor de
muelas\ldots{} libertad\ldots{} jumento\ldots{} ojo\ldots» Nada en
limpio saqué de tal galimatías; mas por no estar callado ni parecer que
no entendía, solté esta frase, que era de las más fijas en mi memoria:
«¿Estás segura de lo que dices?» Ella entonces habló de nuevo con más
calor y viveza, como repitiendo y ampliando sus anteriores razones. Yo
le solté otros conceptos de mi \emph{Prontuario}: «Me sorprende el
saberlo\ldots{} ¡Cuánto me afligen tus desgracias!»

En resolución, el jugo que yo sacaba de nuestro coloquio era que
\emph{Erhimo} me pedía que la libertase, y naturalmente yo le daba a
entender que no deseaba otra cosa. Firme en mi idea, le dije: «No
ambiciono más que tu felicidad\ldots{} Sólo vivo para ti.» Bien clara
llegó a mi intelecto la expresión de su gratitud: «¿Cómo pagarte tan
gran beneficio?» ¡Al fin nos entendíamos! Ya me fueron fáciles las
preguntas: «¿Cuándo, gacela\ldots? ¿Estarás dispuesta, ensueño de los
ángeles? ¿Dónde te espero?» Y ella me soltó nueva tarabilla con más
presteza que antes. Por mucha atención y cuidado que puse, no cogí más
que estos vocablos desengarzados del rosario de su charla: «Ojo\ldots{}
zapatero\ldots{} adiós\ldots{} libertad\ldots{} buen
\emph{Confusio}\ldots{} agradecimiento\ldots{} veneno\ldots{}
\emph{Maimuna}\ldots{} carta\ldots{} puerta\ldots{} salida\ldots{}
noche\ldots» Y otra vez repitió, hasta tres veces: «Carta, noche,
puerta.» No podía ser más claro: me escribiría una carta, la cual
asomaría por debajo de la puerta, cuando la sosegada noche derramara su
obscuridad en el patio. Dio suaves golpecitos en la madera, los cuales
sentí como blanda caricia en mi corazón enamorado, y dijo hasta cinco
veces \emph{adiós}\ldots{} Oí el dulce pisar de sus chancletas,
retirándose escalones arriba.

Quedé yo embelesado y atónito del júbilo que me causaron la ilusión de
amor y mi singular charla equívoca con \emph{Erhimo}, dulcísimo
coloquio, aun sin saber yo fijamente lo que habíamos dicho y tratado.
Pero de la confusión del lenguaje sobresalía un hecho; y era que la
mora, prendada de mi donosura, que contemplado había desde las altas
rejas, quería que yo la sacase de su esclavitud, y conmigo la llevase a
la civilización y a la Cristiandad. Esto me vanagloriaba, me volvía
loco, y mis escrúpulos por traicionar la hospitalidad de \emph{El
Nasiry} se disiparon con la idea de que sacaba un alma de las tinieblas
a la luz\ldots{} Tan encendida estaba mi mente con mi cercano triunfo de
enamorado y de catequista, que salí de la casa y me lancé al enredo de
las calles morunas, para derramar en ellas mi alegría, mi ilusión, mi
éxtasis\ldots{} Molinillo era mi pensamiento imaginando con giro febril
la hermosura de \emph{Erhimo}. ¡Qué ojos obscuros, entornados,
flechantes al resguardo de las grandes pestañas, decidores de mil
secretos del amor de los ángeles y del de los humanos!\ldots{} ¡qué
risueña y regalada boquita!\ldots{} ¡qué cabellos sedosos, negros,
destinados a mayor encanto cuando los humedeciera el agua del
bautismo!\ldots{} ¡qué talle flexible y pegadizo, imitador de la
serpiente en sus ondulaciones, y qué cuerpo, en fin, imitador de la
gacela en su agilidad voladora! ¡Vaya unos andares y un revuelo de hurí,
como las que cantan y retozan en el paraíso musulmán!\ldots{} Pero no:
¡atrás Mahoma y sus ritos mentirosos! Reunía yo en mi pensamiento las
dos esencias de amor y religión, y quería ser en una pieza el galán
dichoso amado por \emph{Erhimo}, y el sacerdote que vertiera en su
cabeza el agua salvadora. ¡Doble triunfo y alegría dos veces inefable!

Llegada la noche, me metí en casa, donde tuve la suerte de cenar solo.
Francamente, en tal noche me habrían sido penosas la presencia y mirada
de \emph{El Nasiry}. Entre la moral mahometana y la mía española no
había concordia ni avenencia. Con sólo pasar de una raza a otra, el mal
se trocaba en bien y el pecado en virtud. Mejor era que no habláramos.
Los hechos hablarían\ldots{} Pues señor: en cuanto quedó anegada en
sombras la casa, cerrada la puerta, \emph{Ibrahim} recogido a lo hondo
del segundo patio, y todo en silencio, ya no pensé más que en vigilar la
puerta por cuyo hueco inferior, Oriente rastrero de mi dicha, había de
aparecer el sol de la anunciada carta\ldots{} Pasaron horas de febril
expectación. Mi ansiedad era juguete del tiempo, y éste un envidioso de
las delicias de mi aventura. Como no tengo reloj, ni hay en aquel
maldito pueblo torres de iglesia que con campanadas marquen las horas,
no podía yo precisar el tiempo transcurrido: sólo sabía que los minutos
remedaban la longitud de los años. Acabadita mi auscultación de la
puerta, esperando en ella rumor de pasos o siseo, volvía yo a lo
mismo\ldots{} Poco tiempo estaba lejos de las maderas que eran la
síntesis de todo el Universo. Creía que si me alejaba por dos o tres
segundos, haría esperar a \emph{Erhimo}\ldots{} Por fin, a una hora que
sin duda era de las correspondientes a la madrugada, saltaron a mi oído
los anhelados rumores. Fue susurro no más del aliento de la odalisca,
que me dijo: «\emph{Confusio}, toma la carta.» Sentí el roce del papel
pasando de dentro afuera. Al mismo tiempo, la mora, adelgazando más su
voz, me echó por el agujero de la llave un \emph{adiós} seguido de
expresiones medrosas, que traduje libremente de este modo: «No puedo
estar aquí, buen \emph{Confusio:} el menor ruido sería mi perdición. Lee
la carta y haz lo que te digo\ldots» Se retiró escalera arriba. Oí un
paso blando de pie desnudo.

La desesperación que me acometió al volver a mi cuarto, no la
comprenderás, ¡oh, lector mío!, si no te digo que me encontré sin luz y
sin fósforos, por habérseme olvidado decir a \emph{Ibrahim} que me
dejase bujía y con qué encenderla. Forzosamente había de esperar a que
la luz solar me alumbrase la lectura del divino mensaje, el cual era un
papel escrito por todo un lado y la mitad de otro, doblado y sin cierre
ni sobre. Me llené de paciencia, me tumbé vestido y dormí algunos ratos,
sin soltar de mi mano el papel, que aún emboscaba en la obscuridad sus
misteriosos caracteres. Despierto con la claridad matinal, advertí que
la carta se componía de confusos garabatos escritos con tinta roja.
¡Nueva desesperación! Arábigos eran los caracteres, pero trazados por
mano tan inexperta, que su interpretación habría sido un problema para
cualquier práctico, para mí no digamos\ldots{} No acertaré a expresar
cuánto me estorbaba lo negro, diré mejor, lo rojo de aquellos trazos.
Repasados tres o cuatro veces los torcidos renglones, creí descifrar
estas voces: «burro, ojo, zapatero, libertad, etc\ldots» En lo escrito,
lo mismo que en el habla de la bella \emph{Erhimo}, no pescaba yo más
que algunos vocablos de los muchos que en aquel confuso mar nadaban,
cual minúsculos, inquietos pececillos.

Pero yo buscaría un buen entendedor que lo tradujese y desentrañase,
aunque los garfios, rabillos y puntos trazados por la mora fuesen obra
del mismo diablo. Entretuve dos horas largas de la mañana en escribir
todo lo pasado de mi aventura, mientras llegaba la parte de ella
escondida aún en los senos del tiempo, y que sin duda habría de ser la
más interesante. Terminando estaba ya mi trabajo del día, cuando me
quitó la luz de la ventana una sombra que en ella se interpuso. Era
\emph{El Nasiry}, que me saludó en esta forma: «Allah sea contigo,
amable \emph{Confusio}. ¿Estás escribiendo? Pues acaba pronto, hijo, que
hoy tenemos mucho que hablar\ldots{} y que hacer.» Concluyo, pues así me
lo manda el amo, diciendo que en este instante entra \emph{El Nasiry} en
mi aposento, y que en su rostro y ademán creo notar una cierta gravedad
en él desusada, y ante la cual se pone en guardia mi espíritu, armándose
de todas sus facultades agresoras y defensivas. Aunque al pronto su
vista me causó algún temblor, luego me fortalecí. Ya no tiemblo;
espero\ldots{}

Adiós, amigos. Hasta otra, que será donde Dios quiera, o en el amenísimo
Valle de Josafat.

\textbf{Cádiz}, \emph{Marzo}.---¿Pensáis que he venido acá con la ideal
\emph{Erhimo}? ¿Pensáis que me ha lanzado \emph{El Nasiry}, tirándome
como pelota de un lado a otro del Estrecho?\ldots{} Esperad un poco;
dejadme tomar el hilo de mi relato en el punto mismo en que el renegado
Ansúrez me obligó a romperlo. Entró, como dije, y viéndome limpiar mis
plumas, que por algún tiempo habrían de estar ociosas, me soltó este
jicarazo: «Recoge tu equipaje y dispón tu persona, que ha llegado la
hora de embarcarte. Llamo equipaje a tu ropa interior, lavada o por
lavar, que puedes envolver en un pañuelo grande; a lo que traes sobre tu
cuerpo, y a los papeles que has escrito, todo lo cual en corto tiempo
puede ser prevenido. ¡Feliz el hombre que viaja con tanto alivio de
bagaje como los pájaros!»

---¿Pero ha llegado el vapor?---exclamé no hallando mejor disimulo de mi
perplejidad.---El vapor no ha llegado, \emph{El Nasiry}.

---Ha llegado anoche, y partirá hoy a las doce, a menos que tú lo eches
a pique llenándolo de malos pensamientos---afirmó el renegado con
firmeza, que me desconcertó más de lo que yo estaba.

---¡A las doce! Pues aún falta mucho tiempo.

Y él, con autoridad incisiva que no dejaba lugar a protestas, me ordenó
que hiciera mi menguado envoltorio, y le siguiese sin vacilación ni
excusas. Y como para suavizar la aspereza de su despotismo, sacó la
bolsa judaica, y la sopesó haciendo sonar las monedillas. No puedo negar
que el metálico ruido desarmó un tanto mi resistencia. Perezoso, fui
recogiendo y empaquetando mis cosas, mientras el renegado añadía razones
que me movieron más a obedecerle. «Sabrás---me dijo---que tengo prisa
por embarcarte, porque esta tarde he de partir para Tetuán, ya de
arrancada con toda mi familia.»

---¿Ya?\ldots{} ¿A Tetuán? ¿Pues qué\ldots{} hay ya paces entre España y
Marruecos?

---Paz venturosa firmaron ayer O'Donnell y \emph{Muley El Abbás}. Todo
Tánger lo sabe, menos tú, que no vives en la realidad, sino en el mundo
de los ensueños tontos y falaces\ldots{} Es raro que el hombre que se
llamó Predicante de la Paz, no se alegre ahora de verla declarada y
ajustada por dos pueblos hermanos\ldots{} hermanos digo, y no es para
que te asustes y pongas esa cara de idiota\ldots{} ¿Qué piensas? ¿Ahora
sales con que quieres guerra, y que sigan rompiéndose el bautismo y la
circuncisión Marruecos y España?

---No, no: guerra no quiero, sino paz. La paz es mi elemento\ldots{} En
la paz desarrolla mi espíritu sus\ldots{} no sé cómo decirlo\ldots{} sus
ideales doctrinas\ldots{} Estoy contento de que no haya más guerra.
Cuéntame\ldots{} Pero no\ldots{} Antes dime\ldots{} dime por qué te vas
a Tetuán tan de improviso, con toda tu reata de chiquillos y mujeres.

---Hijo mío, estoy en el aprieto de llegar pronto a Tetuán, y un día más
que tarde podría traerme desdicha grande. No cabe más dilación, ahora
que la paz me abre el camino de mi casa\ldots{} Pues sabrás, pobre
\emph{Confusio}, que tengo enferma gravemente a una de mis esclavas, la
más cariñosa, buena y apacible. Meses ha fue aquejada de un humorcillo
que primero se le manifestó en el oído, luego en el cuello. Este achaque
menoscabó grandemente su hermosura, por causa del sarpullido y del olor
nada grato. Terribles dolores en dientes y muelas le quitaban el sueño,
y de resultas de ello, la magnífica dentadura, que era como ringlera de
perlas, quedó deslucida por caérsele algunas piezas de las más visibles.
Lo que ha sufrido la pobre no puedes imaginártelo\ldots{} Apareció luego
el humorcillo en las piernas, con lo que se deslució aquel cuerpo de
estatua, aquella piel que superaba en tersura y suavidad, puedes
creérmelo, al más fino raso y al terciopelo más pulido. Con ungüentos
preparados de las curanderas que aquí tenemos, se logró atajar el
humorcillo en partes del cuerpo bajo y alto, donde más se estragaba y
descomponía la belleza. Pero de pronto, cátate que aparece el maleficio
en el ojo izquierdo, cebándose en uno de aquellos dos soles de su cara,
que sólo con el del cielo podrían ser comparados, ¡ay!\ldots{} En parte
tan delicada, nada han podido los remedios de acá, y ya la tengo, si no
irremediablemente tuerta, a punto de serlo para toda su vida, que es la
mayor desolación que podrías imaginar en el vergel de aquel rostro de
hurí.

Oí esta relación entre espantado y receloso, dudando si admitirla como
verdadera, o si debía diputar a \emph{El Nasiry} por el más redomado
guasón de todo el orbe cristiano y mahometano.

\hypertarget{xii}{%
\chapter{XII}\label{xii}}

«Ya comprendo---le dije---tu impaciencia por llegar a Tetuán. Allí
tienes a los médicos del Ejército español, entre los cuales los hay de
muchísima ciencia, y de mano segura contra las peores enfermedades.»

---Has adivinado mi intención. A eso voy. Me han dicho que entre tales
Físicos hay uno que de este mal del humor, y de otros más hondos e
invisibles, entiende como nadie\ldots{} Porque aún no sabes que el mayor
mal de mi esclava no es el achaque del ojo, ni la piel afeada, ni el que
haya huido de su boca aquel aliento de rosas y clavellinas; no es eso lo
peor, \emph{Confusio} amigo, sino que con el mucho padecer, y el no
dormir y el condolerse de su hermosura perdida, se le ha escapado de la
cabeza el juicio que antes tuvo y que por ningún medio podemos
devolverle. Desde que llegamos aquí, ha dado en la más extraña manía que
cabe en cerebro de mujer, y es pensar y decir que no la queremos, que la
atormentamos, que el parche que le ponemos en el ojo está envenenado
para que se quede tuerta más pronto, y, por fin, ha caído en la
disparatada locura de pedir que la devuelva yo a su primer dueño, un
amigo mío de Fez, llamado \emph{El Jarráz} \emph{(el zapatero)}, porque
lo fue su padre. Este buen amigo me la vendió por poco dinero\ldots{}
mejor será decir que me la cambió por un burro, o que fue un excelente
burro garañón el precio de la bella morita\ldots{} No se contenta
\emph{Erhimo} con clamar por el \emph{zapatero}, sino que se pasa el día
gritando, y se quiere matar; a toda persona que ve en este patio, aunque
sea desconocida, la llama, y como puede le cuenta su desgracia, le
manifiesta sus ganas de ser restituida al que me la vendió, y le pide
auxilio para tan grande locura o desatino, pues el \emph{zapatero} se ha
muerto, y aunque viviese no la cuidaría con el esmero y paternal cariño
que yo pongo en ella\ldots{} Créeme, \emph{Confusio}; estoy
afligidísimo: yo miro a mis mujeres, no como esclavas a estilo moro,
sino como a hijas de Dios, mis iguales en la dignidad y el amor, y esto,
yo te lo juro, es lo que más fijo se me ha quedado en el alma de todo el
cristianismo, que abandoné cuando de aquella tierra me vine, y cambié de
ropa, de habla y de conciencia.

Dijo esto con sinceridad patética, o con un arte superior que fingía
soberanamente la verdad; y en la duda de si debía creerle o no, me
decidí por lo primero, rindiéndome a sus designios. Esto era, en mi
humildísima posición, más cuerdo y más fácil que no plantarme contra él
en terreno tan inseguro como el de un loco ensueño de aventura
novelesca. Admití resueltamente lo que me dijo mi protector, y con
gallardo arranque le mostré la carta de \emph{Erhimo}, diciéndole:
«Hazme el favor de descifrarme estos garabatos infernales que en el
patio me encontré anoche.» Y él, echándose a reír, una vez cogido el
papel, me contestó: «No necesito descifrarlos, porque ya sé lo que aquí
se ha escrito. La pobre enferma no sabe escribir; pero \emph{Quentza} sí
sabe, que estuvo en la esclavitud de un maestro que fue el primer
gramático y el más nombrado pendolista de Fez. \emph{Erhimo} pidió a su
compañera que le escribiese la carta; la otra no quería, por ser cosa
vedada entre mujeres el toma y daca de cartitas con los de fuera. Pero
yo le dije a \emph{Quentza}: hazle el gusto y escríbele lo que te dicte,
para que con la negativa no se le encienda más el odio que por su grave
demencia nos ha tomado. Anoche se escondieron en la estancia para
escribir: \emph{Quentza} me lo ha contado. \emph{Bab-el-lah}, que es
toda prudencia y bondad, opinó también que no contrariáramos a la
infeliz \emph{Erhimo}, y de ella ha partido la idea de irnos pronto a
Tetuán en busca del médico sabio que me ha de curar, si Allah lo
permite, a esta prenda del alma.»

Antes de acabar de decirlo, \emph{El Nasiry} rompió el papel en
pedacitos, lo que yo vi como si desgarrara las hojas de un poema, no tan
bello por lo ya escrito, como por lo que aún estaba por escribir.
Arrojados al patio los fragmentos del papel, un vientecillo que entró
por el portal dispersó con el mismo soplo juguetón las estrofas que yo
compuse y las que aún estaban en la mente divina de la Musa.

Cogiome del brazo el hijo de Ansúrez, y me dejé llevar a la calle
tranquilamente. \emph{Ibrahim} fue delante con el encargo de comprar una
maleta de mano en que llevar con más decoro mi ropita. Digo que iba yo
tranquilo, pero no alegre, sino con tristeza mezclada de resignación;
que no pudo quedar mi espíritu en mejor estado después de arrancarle de
un tirón las alas con que quería largarse a dar una vuelta por los
espacios de la poesía, lindantes con lo infinito\ldots{} Pero bien sabía
yo que nada nos alivia de los propios cuidados como el poner interés y
conversación en los cuidados públicos; y con esta idea, calles abajo,
pregunté a \emph{El Nasiry} cómo y cuándo y en qué condiciones se había
hecho la paz.

---Pues la primera condición de la paz es que los españoles se volverán
a su casa, donde, si quieren guerra, pueden ejercitarse en la civil todo
lo que gusten.

---Pero no se irá España de Marruecos sin llevarse algo, que alforjas ha
traído, ¡vive Dios!, y gran mengua sería llevarlas vacías.

---No se lleva nada\ldots{} Digo, sí: le dan un poquito de terreno
pegado a Ceuta. Esta plaza es hoy para España una chuleta que no tiene
más que el hueso. Necesario será pegar al hueso un poco de carne\ldots{}
También se lleva\ldots{} digo, se llevará, una linda playa del mar
Océano, excelente para recoger conchitas y para la pesca de truchas de
agua salada\ldots{}

---Poco ganaría con esto, si no se llevara también a \emph{Ojitos de
Manantiales}.

---¡Ay!, no: \emph{Ojitos} aquí se queda, rescatada por Marruecos, que
compra su libertad con veinte millones de duros.

---¡Jesús, cuánto dinero!\ldots{} ¿Pero cómo se va el español, si ya
tiene a Tetuán por suya, y ha rotulado en lengua castellana todas las
calles?

---Borraremos los rótulos después de entregar los veinte
millones\ldots{} También daremos a España un tratado de comercio.

---Poco es lo que sacamos de esta guerra, costosa en dinero y más
costosa de sangre.

---Poco no, porque España ha conseguido lo que se proponía, que no era
conquistar territorios, sino hacer una demostración de su poder militar.
Todo el mundo ha podido ver que tenéis un gran Ejército pequeño.

---Gran desatino has dicho, \emph{El Nasiry}, aplicando a un objeto
calificaciones de sentido contrario. Si nuestro Ejército es pequeño,
¿cómo puede ser grande?

---Grandeza y pequeñez no aplico juntas, sino cada cualidad por distinto
lado. Es grande vuestro Ejército, porque tiene generales entendidos que
lo manden; tiene oficiales que conocen y practican con devoción
religiosa los dogmas de valor, deber y disciplina; soldados tiene que
son heroicos con inocencia y naturalidad, borregos para el amor de la
patria, leones para su defensa; tiene, en fin, armas y pertrechos de
superior calidad, todo bien discurrido y dispuesto por manos sabias y
militares. Pero si por esto es grande, pequeño es por la cifra de sus
hombres, la cual no le bastará contra cualquiera otro de los Reinos
ambiciosos que hay en esos mundos, del Estrecho para allá.

Esto dijo \emph{El Nasiry}, y sus ideas reproduzco vistiéndolas con un
poco de ornato retórico. Luego siguió: «No digamos que se llevará España
las alforjas sin más carga que el dinero. Se lleva también buen surtido
de honor y caballería, cosas que entiendo yo van escaseando allá por el
desmedido uso que de ellas se ha hecho. Lleva también el mayor acopio
posible de militar autoridad, con que el buen O'Donnell pueda espantar y
hacer el coco a los políticos que le estorban, o no le dejan hacer su
gusto en el gobierno de una nación revuelta, engañada y desengañada de
tantas coplas de libertad, constitución, y viva la Pepa\ldots{} No, no
deben irse descontentos los españoles con este botín, y de añadidura
veinte millones, admitido que se los paguemos, aunque sea en chapas de
cobre, más parecidas a cabezas de clavos viejos que a monedas de
cristianos\ldots»

En esta conversación amena recorrimos las torcidas calles hasta llegar
al puerto. Nos metimos en la Aduana, de cuyo administrador y ministriles
era amigo mi protector, y al cabo de otro rato invertido en saludos
cortos y coloquios luengos acerca de la paz, llegó \emph{Ibrahim} con mi
maletita y el billete de mi pasaje en el vapor. Aún no había prisa para
embarcarme. Llevome \emph{El Nasiry} a un rincón solitario, donde nos
brindaban cómodo asiento unos sacos de trigo, y sentados ambos, mi amigo
sacó la encarnada bolsa de \emph{Riomesta} y \emph{Papo}, le dio unos
toquecitos para que sonara el metal, y poniéndola al fin en mi mano
\emph{¡alleluia!}, me dijo: «Aquí tienes los cien duros que los
\emph{sinagogos} te dieron por el desempeño de la blanca \emph{Yohar}.
No es eso sólo lo que llevas; pues tu amigo \emph{El Nasiry} te da otros
cien \emph{borques}, que encontrarás también en la bolsa, descontado tan
sólo el precio del billete del vapor. No irás descontento con tus ciento
noventa y cinco duros. Otros han hecho más que tú en África, y se llevan
menos. Créeme que embarcando contigo un par de moras o una docena de
judías, irías más pobre que vas.»

Cogiendo en mis manos la bolsita (mentira me pareció), eché de mi boca
cuantas palabras y conceptos me parecieron pertinentes para expresar la
gratitud, sin cuidarme de adornarlas, pues no era menester, con ningún
artificio. Claramente vi ya en Gonzalo Ansúrez un buen amigo, cuyos
sentimientos cristianos y generosos en aquel caso se mostraban. No me
pidió cuenta de mis diabluras en el patio, que sin duda conocía, ni me
riñó por haber intentado sonsacarle a la doliente \emph{Erhimo}. Fue
liberal, fue magnánimo, y para que veáis cuánto me estimaba y en qué
opinión tan alta me tenía, copio lo que momentos antes de mi partida me
dijo, y lo que me aconsejó y recomendó con paternal solicitud. Fue de
este modo: «Bien claro ves, \emph{Confusio} amigo, que te has hecho
lugar en mi corazón, a pesar de tus ligerezas y del poco brío con que
atiendes a refrenar tus liviandades. Careces de voluntad firme para
poner tus acciones en la regla debida, y dejándote llevar de la
imaginación loca, faltas a la amistad y al honor. A pesar de esto, yo te
estimo por tu ingenio, y por tu buen corazón te perdono tus travesuras.
Vuelves ahora a España, donde has de vivir, o de un empleo, que ha
venido a ser el arbitrio de los más, o de tu trabajo, que será el mejor
arbitrio. Dime, pues, a qué piensas dedicarte, porque si es tu ánimo
agostar tu inteligencia en una oficina, valdría más que aquí te quedaras
para toda la vida. En caso de que pienses consagrarte a una carrera
noble, profesión u oficio liberal, dime cuál es, para que yo te aconseje
según el entender mío, que, aunque te parezca corto, es largo de agudeza
y de esa gramática que llamáis parda.»

---Pues sabrás---le respondí---que mis gustos y todo mi ser me llaman a
las ocupaciones espirituales, y me alejan de lo material y positivo. No
sé si me entenderás\ldots{} Soy enemigo de la violencia: no hay que
hablarme, pues, de que sea yo militar. Detesto los enredos curiales y la
prestidigitación leguleya: nunca seré abogado ni escribano, ni juez. La
Medicina y Farmacia no entran en mí, creyente en la Naturaleza, que así
trae los males como los quita. Artes de ingeniero no me seducen, porque
ellas tienen su fundamento en las Matemáticas, que no he podido entender
nunca. Marina me repugna, porque nada me causa tanto pavor como el
oleaje de las aguas y el vaivén de los barcos. Comercio no entra en mí,
porque se basa en los números, y en un calcular frío de ganancias y
pérdidas que no se aviene a mi entendimiento. A mercader quise meterme
cuando discurría los medios de mantener el lujo de \emph{Yohar}; pero
ello fue un comercio de pura fantasía y de navegación aérea, que me
habría lanzado al abismo. \emph{Papo Acevedo} entiende de comercio más
que yo: por eso se llevó a \emph{Yohar}\ldots{} Pues no me queda más que
una carrera, oficio y profesión noble que colme mis anhelos entre todas
las que conozco: ¿no adivinas cuál es? ¿No entiendes que, o no seré
nunca nada, o seré hombre de religión que lleve las almas al bien, los
corazones a la virtud; no ves, en fin, que he de ser sacerdote si quiero
ser algo?

---Por un lado---me contestó \emph{El Nasiry} poniéndose la máscara
guasona,---veo tu aptitud para esa carrera; por otro, veo todo lo
contrario. Si los curas no estuvieran en el mundo más que para predicar,
serías tú el primero de todos. Pero si están para dar ejemplo, que es el
sermón mudo de mayor eficacia, me parece, querido \emph{Confusio}, que
no sirves, no sirves\ldots{}

---Ya te haré comprender que sirvo. Por de pronto, sábete que a mí me
han dicho lo que Castelar: «Hazte cura y arrastrarás a las muchedumbres
para llevarlas a donde quieras\ldots» Me siento predicador, \emph{El
Nasiry}; reconozco en mí la virtud convincente y avasalladora que ha
sido la fuerza de todo apostolado\ldots{} Me siento también confesor,
templador de almas, con el arte psicológico para dar a las conciencias
su tranquilidad, y restablecer la moral perturbada\ldots{} Conozco los
dogmas; sé explanar los misterios; entiendo los ritos y sé apreciar su
belleza; soy teólogo, soy litúrgico, soy también algo canonista. ¿Qué me
falta?

---Pues te falta\ldots{}

---A eso voy. Déjame hablar. Al decir que algo me falta, debiste decir
que algo me sobra.

---Eso, eso.

---No estás en lo razonable con la sobra ni con la falta, pues lo que tú
crees sobrante, no es tal, sino que está muy en su lugar. Te diré que no
sólo creo compatible el sacerdocio con el cariño de mujer, sino que lo
creo necesario, indispensable. Ahí está el \emph{quid}, amigo
\emph{Nasiry}\ldots{} Ni el celibato ni el uso constante de la negra
sotana, manteo y teja, dan al sacerdote mayor dignidad y veneración más
alta. Al contrario, toda esa negrura de fuera y de dentro, le aleja de
los corazones\ldots{} de lo que resulta que lo sobrante, según tú, no
sobra, sino que está en su punto, como te dije, y que es locura enmendar
la plana a la santa Naturaleza.

---Bien, hijo mío, bien\ldots{} No dudo que seas religioso y gran
predicador; pero dudo que puedas reformar lo que por designio de la
Iglesia o del mismo Dios, según decís, es como es; y así lo has
encontrado, \emph{Confusio}, y así lo tendrás que dejar.

---Yo no reformo a nadie; a mí me reformaré si puedo, o me dejaré como
estoy.

Algo más iba a decir; pero un tremendo silbido que venía del vapor puso
fin a mi conversación con \emph{El Nasiry} y a mi vida africana. Los dos
nos levantamos, y con igual emoción nos dimos los brazos. Sacó después
de su pecho mi amigo un voluminoso pliego, que me confió, encargándome
que a su padre lo entregara. Contenía carta para éste y para otras
personas de su nunca olvidada familia. Le prometí ponerlo, en manos del
propio Jerónimo Ansúrez\ldots{} Repetimos nuestros afectos, en él y en
mí salidos del corazón, y prometiéndole yo escribirle mis andanzas en
tierra española, asegurándome él que siempre me recordaría con gozo, nos
separamos, y fui llevado a la lancha por el procedimiento de embarque
más peregrino y chusco que han visto humanos ojos. Un fornido moro me
cogió en vilo, y metiéndose en el agua hasta llegar a donde flotaba el
bote, allí me dejó sin la más leve mojadura\ldots{} Otros pasajeros,
antes y después de mí, entraron del mismo modo en el reino de
Neptuno\ldots{} Vi a \emph{El Nasiry} y a \emph{Ibrahim} que desde
tierra me saludaban. Adiós, simpático amigo, compañero fiel; adiós
Tánger; adiós Mogreb, desvanecimiento de ilusiones\ldots{} Aquí va la
pobre hoja desprendida del árbol de la poesía\ldots{} África me
suelta\ldots{} Europa me toma.

\hypertarget{xiii}{%
\chapter{XIII}\label{xiii}}

\textbf{Madrid}, \emph{Marzo}.---Dejadme que omita las desabridas
incidencias de los dos días que pasé en Cádiz, donde ya no encontré ni
familia ni amigos, que a tal soledad me ha traído el rigor de ausencias
y muertes; ni el cansado viaje que emprendí en ferrocarril para seguirlo
luego en perezosa diligencia hasta más acá de la Argamasilla y tierras
quijotiles, donde vuelve a remolcarnos la negra máquina, y nos trae a la
comarca polvorosa en que se asientan los dos grandes pueblos de Getafe y
Madrid. Omito también el contaros cuán melancólico fue mi dilatado
viaje, con equipo corto y carga excesiva de añoranzas. En el traqueteo
de coches arrastrados de caballos o de veloz locomotora, los recuerdos
agobiaban mi mente, o en ella se sucedían por turno, cuando no entraban
en tropel, fatigándome con la intensa reproducción de la realidad. ¡Oh
dulce \emph{Yohar} blanquísima, oh soñada y nunca vista \emph{Erhimo},
oh misterios del África musulmana y judía, oh tormentos, injurias y
riesgos de morir! Todo se renovó en mi mente, así como la gallarda
amistad de \emph{El Nasiry}, espejo de caballeros renegados.

La despoetización, el desplome ruinoso de mis ilusorias aventuras,
entristeció soberanamente mi ánimo; pero éste no quería rendirse, y como
caballo de raza trataba de enderezarse después de su resbalón y caída.
Digo esto porque a mitad del camino, sobre las desvanecidas imágenes de
\emph{Erhimo} no vista y de Yohar inconstante, empezó a destacarse y
tomar cuerpo mental la imagen de Lucila, ilusión que, disipada en
África, en Europa iba recobrando su brillo. A medida que yo avanzaba por
estas tierras pardas, se me presentaba más clara y hermosa, dentro del
magín, la figura y persona de la ideal mujer, viuda de Halconero y madre
del interesante niño Vicente. Era esto como si lo cierto recobrara el
puesto que le había quitado lo dudoso y fugaz.

Y recuerdo que al pasar por la nobilísima villa de Tembleque, y por el
no menos ilustre lugar de Quero, que rodean saladas lagunas, mi mente y
mis sentidos apreciaron toda la majestad de la hija de Ansúrez, su
exquisita belleza, el hechizo de su voz, las soberanas virtudes que
subliman su persona\ldots{} Y ya en el paso entre Valdemoro y Pinto,
lugares famosos por sus alborozantes vinos, iba mi pensamiento tan
recalentado en la mental contemplación de la sin par señora, que ya se
me hacían siglos los minutos que tardara en rendirle toda mi
voluntad\ldots{} Llegué por fin a Madrid, vencido el cansancio por la
ilusión risueña de reanudar mis amistades, y de reparar el olvido de
tantas cosas y personas agradables o bellas. Desde la estación a mi
casa, que era mi hospedaje antiguo en la calle de Milaneses, hirió mi
vista el repugnante espectáculo de los sombreros de copa, lo que me
acibaró el gusto de la llegada. Vi tantos y tan feos, que jamás cosa
alguna del mundo me hirió la retina con mayor desagrado. Los hombres que
aquel ridículo armatoste cargaban, pareciéronme agobiados de tristeza;
las mujeres, enjauladas de medio cuerpo abajo en los miriñaques, se me
figuraron muñecas fúnebres\ldots{} Anochecía; los faroleros encendían el
gas, y a la claridad amarilla, personas y tiendas, las altas casas y el
empedrado suelo, los coches y su desapacible ruido sobre las piedras o
adoquines, llenaban mi alma de antipatía\ldots{} Completaron mi enojo
los carteles pegados en las esquinas, los aguadores y los corchetes, los
vendedores de romances y los ciegos siniestros que piden con la terrible
amenaza de un violín o guitarra.

En mi casa entré con mi pobre y flaca maleta. Creyó la patrona que yo le
traía unas babuchas bordadas de oro. No fue mal chasco el que se llevó,
viendo que sólo la obsequié con un saquito de hierbas olorosas (recuerdo
amigable introducido en mi maleta por el buen \emph{Ibrahim}); mas no
quiso tomarlas hasta que se las metí por los ojos, encareciéndolas como
prodigiosa droga medicinal y cosmética, de grandísima virtud para el
disimulo de la vejez y prolongación de la vida. Pedí cena y cama; dormí,
que buena falta me hacía, y mis primeros propósitos al siguiente día
fueron presentarme al marqués de Beramendi, y procurarme ropa más airosa
y flamante con que visitar a los Ansúrez. Ya eran las diez cuando
llamaba yo a la puerta de mi Mecenas. Tales burlas de mi facha hizo mi
noble amigo, que me avergonzó. Más me habría valido regresar a Madrid
con el trajecito moro que me arregló \emph{Mazaltob} y que dejé en mi
tugurio del \emph{Mellah} (calle de Numancia).

Pero, en fin, ello es que, aparte del cómico efecto de mi traje,
adquirido en el Rastro tangerino, Beramendi me recibió con grande
agasajo y afabilidad, y en las dos horas que permanecí en su casa, no se
hartaba de oír las explicaciones que a sus preguntas sobre la vida
africana le daba yo, tan incansable en el discurso como él en su
curiosidad. Díjome que la historia personal que en Tetuán empecé a
escribirle, le encantaba; elogió benévolo la relación de mis desventuras
al ser abandonado de la blanca judía, y se regocijó de mi salida con
\emph{El Nasiry}, y del incidente de la bolsa, que primero rechacé
puntilloso y luego admití agradecido. Interesantes halló los lances
apurados del \emph{Fondac}, que a punto estuvieron de ser tragedia; y al
recibir de mi mano lo escrito en Tánger, por no haber correo que antes
de mi propia repatriación lo trajese, prometió leerlo aquella misma
noche. Más que la Historia seca de los públicos acontecimientos, le
cautivan las referencias de andanzas particulares, y en ellas ve el
colorido de la Historia general, la cual, sin este matiz de sangre, de
fuego anímico, no es más que un trazo negro que así fatiga la vista como
la memoria.

Pero lo que de su charlar festivo y cariñoso me cautivó más fue que me
anunciase el propósito de enviarme a una segunda expedición informatoria
y descriptiva, por su cuenta y riesgo, obligándome yo a escribirle
cuanto me ocurriese y darle noticia de cosas o personas determinadas,
para lo cual llevaría un guión de las materias que serían objeto de mis
pesquisas. No comprendí yo la índole de la misión que mi amigo quería
confiarme; y como le preguntase con cierta inquietud y repugnancia si
era cosa de guerra, díjome que era más bien cosa de paz, o más claro, de
diplomacia. No satisfizo por el pronto mi curiosidad, limitándose a
decirme que sólo me concedía dos días de descanso, y que me preparase
para partir por los caminos y lugares que se me designaran. Estas
órdenes de ausencia pronta me contrariaron un poco, pues yo deseaba
quedarme en Madrid algún tiempo, y así lo manifesté a mi amigo. Tenía
que ver a los Ansúrez, para quienes traigo un pliego de \emph{El
Nasiry}; érame preciso, por imperiosa necesidad de mi espíritu, visitar
a Lucila, reanudar con ella un melindre de amor interrumpido por mi
viaje a Marruecos, o mejor dicho, consolidar una inteligencia de
corazones, que sólo se había manifestado con vagos efluvios traídos y
llevados de rostro en rostro por el mirar, y de alma en alma por
palabritas eutrapélicas. Al oír esto, soltó la risa el Marqués con no
menos burla de mí que al mofarse de mi ropa, y añadió que de la cabeza
me arrancase aquella ilusión, pues ya Lucila había perdido todo su
encanto y despojádose de toda poesía.

«Pues qué---pregunté yo con ansiedad no disimulada,---¿se le ha caído el
pelo, le lloran los ojos, ha perdido los dientes, o padece algún achaque
por donde le haya venido mal olor de boca?»

«No es nada de eso---me respondió mi Mecenas,---que de su hermosura no
hay nada que decir: se conserva tan guapota y sugestiva como cuando Dios
le hizo el favor de enviudarla; pero si no le ha salido grano maligno en
el rostro, le ha salido un novio respetable y antipático, con el cual ha
hecho trato honesto de casarse en cuanto pase el plazo que marca la
sociedad al dolor de las viudas.» Y yo al oír esto, exclamé «¡Jesús!» no
pudiendo decir más, porque mi estupor y disgusto no me daban voces para
expresar de momento lo que sentí. Era ya sistemática perrería de mi
Destino que ninguna ilusión se me lograse, y que todos mis castillos de
amor cayesen por el suelo. ¡Y en aquel castillo lucilesco confiaba yo
para guarecerme de las inclemencias de mi juventud, como definitivo y
sólido refugio para lo restante de mis días!

«Consuélate, buen \emph{Confusio}---me dijo mi patrono,---que aún eres
joven y hallarás el refugio que deseas y mereces. Ya no es Lucila la
gallarda representación del sentimiento heroico y popular; ya la
maléfica influencia de un pretendiente empalagoso ha trastornado aquel
espíritu, ha demolido lo más bello que en él había para levantar un
vulgarísimo edificio\ldots{} ¿de qué dirás?»

---¿De qué? Dígamelo pronto, por Cristo.

---Pues ahora no le da por las glorias militares\ldots{} Todo eso pasó
sin dejar rastro\ldots{} Ahora, pásmate\ldots{} le da por lo
administrativo. Vencedor nuestro Ejército en África y dueño de Tetuán,
el fuego de la leyenda es ya ceniza de la Historia. ¿No sabes que ha
venido de fuera una moda horrible, una tromba, un huracán, una cosa
pedestre y asoladora que se llama \emph{Economía Política}? ¿No sabes
que ahora el buen tono está en ser uno \emph{economista}, y en predicar
el fárrago de las ideas \emph{económicas}? Pues este virus, como diría
mi señor suegro, ha dañado el alma candorosa y esencialmente hispana de
aquella ideal mujer. Una frasecilla que ahora está de moda, y que tiene
su lugar en todo cerebro baldío, ha sido el hielo que ha esterilizado
aquella soberana inteligencia. ¿No adivinas cuál es la mortífera frase?
Pues es ésta: \emph{Menos política y más administración}\ldots{} ¡Ya ves
qué desastre! Sin duda el entendimiento de Lucila habría permanecido
refractario a tales tonterías, si no hubiera caído en la flaqueza de ese
noviazgo. El corruptor de la celtíbera es un hombre de más de cuarenta
años, llamado don Ángel Cordero, viudo también, dueño y cultivador de
tierras en Aldea del Fresno y Cadalso de los Vidrios, y tan ferviente
devoto de la \emph{Economía Política}, que a comprar volúmenes de esta
ciencia del Limbo dedica buena parte de sus rentas. Ha leído cuanto
españoles y franceses escribieron de la monserga económica, y
trastornado con tal pestilencia, como Don Quijote con la de los libros
caballerescos, no ha parado hasta inficionar a Lucila.

---No obstante, señor Marqués---dije yo, viendo en las razones de mi
amigo, más que un discreto pensar, una sutil aberración
humorística,---yo veré a Lucila, yo me informaré del estado de su
ánimo\ldots{}

---¡Si no podrás verla! Hace un mes que reside en la Villa del Prado. ¿Y
allí qué hace? Pues quemar sus lindas pestañas llevando con minuciosa
exactitud las cuentas de trigo, cebada y paja, de jornales, de cuanto
constituye el toma y daca de una gran propiedad rústica. El bruto del
novio, el desaborido economista, está también por allá, en un predio y
caserío lindantes con los de Halconero, y es quien la instruye en todas
esas cábalas; y para acabar de volverla loca, le ha enseñado la
diabólica máquina de contar que llaman \emph{Partida doble}.

---¿Y Vicentito?\ldots---dije yo asiéndome a un afecto que sin duda no
me será robado por la intrusa Administración.

---Te recomiendo que dejes a un lado niños que no sean tuyos, y que no
fundes tus cálculos en nada concerniente a la infancia, pues ya sabes lo
que resulta de acostarse con ella. Reconoce, amigo
\emph{Confusio}\ldots{} y bien sabe Dios con cuánto gusto te doy este
apodo que te colgó el castrense; reconoce que la dama celtíbera y su
niño han perdido aquel encanto y seducción de otros días. No pienses más
en ellos\ldots{} y lánzate solo a los campos de la vida, que aún te
reservan sus tesoros.

---Francamente, señor Marqués---indiqué con cierta cortedad,---de lo que
usted me cuenta, lo que peor y más lamentable me parece es el novio que
le ha salido a esa linda mujer. Pero las aficiones de ella al orden de
cuentas y a mirar por los intereses suyos y de sus hijos, no me
desagradan\ldots{} Al contrario\ldots{} ¿Querrá usted creer que cuando
venía yo dando tumbos por esa Mancha, sin apartar de Lucila mi
pensamiento; cuando yo acariciaba en mi alma el amor de ella como reposo
y cristalización de mi vida, me sentía también un poquito
administrativo? Como que la administración es el descanso, es la paz, es
el reparo que pone la prosaica Aritmética a las demasías del Heroísmo.

---¡Tú administrativo! No, \emph{Confusio}, no me harás creer tal
disparate. Comprendo al enamorado, que en un rapto de demencia, apechuga
con la \emph{Partida doble}, si ve que la mujer de sus sueños anda entre
números. Pero tú no harás eso; tú eres \emph{Confusio}, y tu misión es
vivir, ver tierras, pueblos, y humanidad próxima y lejana; probar todas
las pasiones, sufrir todos los infortunios y gustar alegrías inefables.
Tu misión es ésta, \emph{Confusio} amigo, y por ser tuya esta misión y
no mía, te envidio, quisiera ser como tú, pobre, aventurero, hijo de tus
obras, soberanamente libre.

\hypertarget{xiv}{%
\chapter{XIV}\label{xiv}}

No necesitó el buen Fajardo extremar los recursos de su mágico talento
para que yo me sometiese a cuanto de mí deseaba, sin meterme a discutir
sus designios ni a indagar las causas que movían su conducta. Ofrecile
desempeñar cuantas misiones diplomáticas o de cualquier género quisiera
confiarme, y sólo puse la objeción del corto tiempo que para mi descanso
en Madrid me concedía; alegué, en apoyo de este deseo, la necesidad de
ver a Jerónimo Ansúrez, para quien el renegado me dio un pliego que
debía yo entregar en propia mano.

«No está en Madrid Jerónimo---me dijo Beramendi,---ni le verás aquí
mientras su hija permanezca en la Villa del Prado engolfada en sus
cuentas. Yo sé de qué tratan las cartas de Gonzalo, que traes para su
padre y su hermana, y a decírtelo voy, para que veas que no me oculta el
celtíbero ningún secreto de su familia. Uno de los hijos de Jerónimo,
llamado Gil, \emph{Egidius}, según el sagaz investigador \emph{Maese
Ventura Miedes}, ha salido aficionado a la vida bandolera. En tierras de
la baja Cataluña y del Maestrazgo ha dado no poco que hacer a la Guardia
Civil, asaltando masías o acechando caminantes desprevenidos, ya solo,
ya en cuadrilla con otros vagabundos y ladrones. Afortunado en algunas
de estas malandanzas, fue desgraciado en otras, viéndose tan perdido,
que de la libertad de sus atrevimientos vino a parar a la cárcel, y de
aquí al presidio de Tarragona, de donde le habría sacado el verdugo si
él con artificios increíbles no se escapara para volver a su vida
criminal en los montes de Gandesa. Después se ha sabido que, valido Gil
de disfraces ingeniosos, anda por los pueblos de las bocas del Ebro,
engañando a las gentes sencillas con un comercio que al menor tropiezo
puede llevarle otra vez al presidio. En estas barrabasadas de Gil o
\emph{Egidius}, ve Jerónimo la deshonra de su familia, al fin rescatada
de la miseria y del oprobio por la unión de Lucila con Halconero; y no
pudiendo persuadir a ese pillastre a cambiar de vida, ha escrito del
particular a su hijo Gonzalo para que vea si con halagos podrá éste
inclinarle a que se vaya con él a tierras de moros, donde ha de ser más
fácil que aquí someterle y llevarle a una buena conducta. Más que ver a
Gil en un patíbulo, quiere Jerónimo verle moro y circunciso. De esto han
tratado en largas epístolas el celtíbero y el renegado, y en el pliego
que tú traes vendrá seguramente el plan de Gonzalo para llevarle con
astucias o promesas al delicioso país berberisco, donde por los duros
medios mahometanos será domado ese tunante\ldots{} Puedes dejarme el
pliego, que será puesto en manos de Ansúrez en cuanto aporte por acá, y
vete sin cuidado, que yo quedo en Madrid encargado de este negocio.»

---Bueno, señor---le dije accediendo a cuanto me proponía.---En sus
manos pongo el pliego de Gonzalo Ansúrez\ldots{} Haga usted lo que
quiera con los papeles, que yo me desentiendo absolutamente de estos
particulares.

---Vengan los papeles\ldots{} y ahora\ldots{} fíjate bien en lo que te
digo. Es muy variada y compleja la familia de los Ansúrez. Por los
lugares que has de visitar cuando salgas a la comisión que te encargo,
anda ese tuno de Gil o \emph{Egidius}. Si con él te encuentras, ten
mucho cuidado, Juan, que podrá engañarte y meterte en un gran enredo que
dé contigo y con él en la cárcel. Ya sabes que todos los individuos de
esa familia, de ese índice histórico, de ese resumen étnico, son de una
agudeza formidable. El ingenio y la simpatía personal los asisten, así
para el mal como para el bien. Guárdate de ese Ansúrez andariego, que
es, entre ellos, el verdaderamente peligroso. Y por hoy, nada más te
digo sino que descanses, y vuelvas mañana bien preparado del
entendimiento y de los oídos.

Puntual acudí a la mañana siguiente, ya mejoradito de ropa, que adquirí
a bajo precio en un bazar de elegancias económicas, y las primeras
palabras del Marqués fueron para felicitarme graciosamente por mis
aventuras en la casa de \emph{El Nasiry}, que acababa de leer en las
cartas que yo mismo he traído. Mucho le ha regocijado mi tentativa de
asaltar el harem y de llevarme a \emph{Erhimo}, así como la solución
discreta que el agudísimo renegado supo dar a mi travesura. En cuanto a
la apreciación del hecho, los puntos de vista del Marqués pareciéronme
harto ligeros. Sostiene que lo de los malos humores de \emph{Erhimo}, y
lo de su ojo tuerto, su mal olor de boca y sus accesos de locura, no
fueron más que un sutil artificio de \emph{El Nasiry} para
desilusionarme y resolver pacífica y donosamente una cuestión tan grave.
En ello se reveló el hombre de extraordinaria marrullería y de artes de
gobierno, pues si hubiera yo conseguido mi objeto, sabe Dios cuáles
habrían sido las consecuencias. Probablemente habrían acabado en Tánger
mis pobres días.

Según Beramendi, la mora, de quien no pude ver más que los dedos
amarillos, era realmente el prodigio de hermosura sólo comparable a los
ángeles del paraíso mahometano. Cansada la odalisca de su esclavitud, me
había elegido a mí por su caballero libertador\ldots{} Al decir
\emph{ojo}, no quiso expresar que estuviese tuerta, sino recomendarme
que anduviera yo muy listo y con mucho \emph{ojo} y donaire para
libertarla. Los árabes emplean figuras en sus más usuales formas de
lenguaje\ldots{} Y con la voz \emph{jumento} quiso decir que tuviera yo
preparado este humilde animal para que la salida de la prófuga no fuera
notada\ldots{} Y me ordenaba que tomase yo las trazas de \emph{zapatero}
remendón con el mismo objeto de fingir insignificancia y modestia. Sin
duda, \emph{El Nasiry} supo el contenido de la carta por delación de
\emph{Quentza}, y tramó el engaño con que me había desarmado del
caballeresco empaque de mi aventura.

Aunque no acabaron de convencerme las razones y crítica del Marqués,
sentí renacer en mí la penita de mi desengaño amoroso. Pero mi ilustre
amigo acudió a consolarme, sosteniendo que debo estar muy agradecido a
\emph{El Nasiry} por su conducta discreta y humana. Habíase mostrado
magnánimo y paternal, evitándome un conflicto de solución violenta, y
quizás trágica\ldots{} Naturalmente, admití el consuelo reparador, y lo
pasado, pasado. El presente continuaba ofreciéndose a mis ojos rodeado
de tinieblas y misterio. Digo est, porque antes que termináramos el
Marqués y yo la conversación que copio, entró un tal \emph{Sebo}, ex
polizonte y servidor clandestino de mi noble amigo en sus recónditas
excursiones por el subsuelo político. Traía el tal una maleta casi nueva
o a medio uso, harto más capaz y decente que la mía de Tánger. Díjome el
Marqués que aquel valijón sería mi compañero en la caminata que iba yo a
emprender. Si me agradaba llevar tan buen acomodo para mi ropa, luego,
cuando levantó \emph{Sebo} la tapa de la maleta y vi lo que contenía, el
estupor me hizo prorrumpir en exclamaciones disonantes. Vi ropas de
cura, bonete, breviario, viejos librotes, la \emph{Summa} y los
\emph{Lugares Teológicos}. Riéndose de mi asombro, me rogó el Marqués
que me probase la sotana, para ver si caía bien a mi estatura y talle.
Así lo hice, riéndonos todos, que era lo procedente en la extraña y por
mí no entendida metamorfosis que se me preparaba. A mi casa llevarían la
maleta para meter en ella mi ropa de paisano, en la cual no debía faltar
un traje de color enteramente igual al de los ataúdes.

Pues, señor, ya veríamos en qué paraba aquella farsa, y cuáles eran el
propósito y fines de mi noble protector, en cuyo humorismo claramente se
advertían vislumbres de extravagancia. Marchose el feísimo y ordinario
\emph{Sebo}, y a poco entró un joven muy simpático y bien vestido, a
quien todo Madrid llama familiarmente \emph{Manolo Tarfe}. Yo le había
conocido en aquella misma casa poco antes de mi partida para Cádiz y
Ceuta, y no tuvo necesidad Beramendi de presentarme a él. Comprendí que
entre los dos estaba el juego y se escondía la clave de aquella
conspiración o mundana intriga. Lo primero que me dijo Tarfe fue que me
afeitase toda la cara, limpiándomela del bigote y de las barbillas ralas
con que adornada la tengo en la presente edad histórica\ldots{} Ya no
hay duda de que me disfrazan de clérigo para esa misión que me va
pareciendo una humorada carnavalesca. ¿Qué será? Por Dios que rabio de
curiosidad, y que doy gustoso mis barbas por salir de esta
incertidumbre.

Ante mí hablaron de política Tarfe y Beramendi. Ambos son partidarios
frenéticos de O'Donnell; quieren que éste, al volver de África
victorioso, se revista de la mayor autoridad, y tome aliento para una
dominación estable, implantándonos aquí una imitacioncita del Imperio
francés, segundo de este nombre. No hay ahora en España más fuerza que
la Unión Liberal, \emph{sincretismo}, como algunos dicen, que es la
última palabra de la ciencia política, fuerza que ha de ser liberal para
las ideas y despótica para las acciones, conciliadora del progreso y la
tradición, con proyectismo largo de obras públicas y de fomento
material, enseñando siempre la estaca para que el país obedezca y olvide
las bullangas. La Unión Liberal quiere ilustración y silencio; quiere
mejorar a España de comida y ropa, manteniéndola en el encantamento de
las glorias militares. De lo que dijeron colegí que confían en el
porvenir, y que su ídolo, don Leopoldo, tiene cuerda política para mucho
tiempo; pero algún recelo dejaron entrever, algún misterio se esconde en
las altas esferas, que a mis dos amigos trae inquietos y cavilosos.

No pude enterarme bien de los motivos de esta inquietud, porque Tarfe
ponía frenos a su palabra, como no queriendo expresarse con claridad
delante de mí. No obstante su discreción, bien dejaba comprender que
\emph{estamos sobre un volcán} (así solemos designar el próximo
estallido de una conflagración); que este volcán no es revolucionario al
modo democrático y popular, sino que alimentan su fuego poderes muy
altos\ldots{} ¿Pero a qué devanarme los sesos por descifrar el enigma,
si poco había de tardar la satisfacción de mi curiosidad? Beramendi,
cuando me despedí, me ordenó volver a la noche, para ponerme en autos de
lo que debo hacer, y darme sus instrucciones con la prolijidad que exige
asunto tan delicado.

Acudí puntualmente, y el criado me notificó que el señor Marqués había
salido a un asunto urgente, y me suplicaba que le esperase. Por dicha
mía, fui recibido por la señora Marquesa, que me acortó el plazo de
espera con una graciosa y amena plática. Es mujer tan amable y discreta,
que, oyéndola, no repara uno en la poca gracia de su talle y rostro.
«Pues verá usted, Santiuste---me dijo haciéndome sentar a su lado.---Yo
me alegro de que Pepe haya tenido que salir, porque así puedo darle a
usted mi parte de instrucciones. Yo también conspiro; yo también me
entretengo en mis trabajitos de zapa. ¿A usted no le han dicho aún Pepe
y Manolo que anda por debajo del suelo que pisamos una tremenda
conjuración? Pues yo se lo digo para que tiemble un poquito. Yo, si he
de hablar a usted con franqueza, no he temblado ni pizca cuando lo he
sabido. ¿Quién conspira? Los absolutistas. ¿Quién los mueve? Pepe y
Manolo, que son los descubridores de tal enredo, me aseguran que los
hilos de la conjura los mueven dos grandes familias hermanas, la una
fuera de la Península, la otra en nuestra propia casa, y llamo así a
\emph{Palacio}, porque \emph{Palacio} es la Nación\ldots{} por el lado
solariego y heráldico. ¿No tiembla usted?»

---No, señora: ni el más ligero temblor me sacude los nervios\ldots{} Me
asombro, sí, de que ahora no se azoten las dos ramas, sino que se
injerten y se unan. ¿Contra quién? Contra España y la Libertad, ¿no es
eso?

---No sé qué contestarle, amigo Santiuste; porque como no creo en ese
fragmento de historia inédita que han descubierto Pepe y Manolo, tampoco
sé contra quién vienen las dos ramas unidas\ldots{} Me figuro que es
contra la Unión Liberal, contra el justo medio, \emph{etcétera},
\emph{etcétera}\ldots{} Usted lo entenderá mejor que yo. Lo que veo con
claridad\ldots{} y con mucho disgusto, créame usted, es que Pepe, con
estas cosas, está medio loco. Es hombre que, a poquito que se exalte,
recae en una dolencia que llama \emph{efusión popular, efusión
estética}\ldots{} Nada, tonterías\ldots{} pasión de ánimo, entusiasmo
ardiente por cosas que maldito lo que le interesan\ldots{} Su cerebro es
muy delicado, propenso a la congestión de ideas. Gracias que me tiene a
mí para el alivio de sus manías y aligerarle la carga excesiva de sus
cavilaciones. Soy el sangrador de su pensamiento.

---Sangradora, médica, inteligencia de primer orden. Yo me permito una
pregunta: ¿está usted plenamente convencida de que es absurdo y
fantástico lo que han descubierto el Marqués y Manolo Tarfe?

---Le diré a usted con toda franqueza que me he reído con los cuentos de
la tal conspiración, como con una comedia de esas que son obras maestras
en el arte de los disparates\ldots{} Me he reído, me he reído\ldots{}
pero al fin, tanto me dicen, y tales razones me dan, que he concluido
por ponerme seria. Si no afirmo que las dos ramas estén de acuerdo para
darle un papirotazo a la Constitución, tampoco me atrevo a
negarlo\ldots{} En la duda, espero con un poquito de temor y con otro
poquito de tentación de risa.

---Pues si usted teme, aunque sea riendo, pensemos que es verdad, y
confiemos en el hombre del día, don Leopoldo O'Donnell\ldots{}

---Ayer le ha escrito Pepe contándole estos líos, y dándole prisa para
que arregle pronto los asuntos moros, y acá se venga con su
Ejército\ldots{} Pero me temo que O'Donnell lo tome también a risa, y
que al venir se encuentre en el trono de España a un Rey con quien no
contaba: Su Majestad Carlos VI.

No pude contenerme; solté una risa franca, infantil, y contagiada de mi
buen humor la ilustre señora, los dos concluimos en sonoras carcajadas
sin poder articular palabra alguna. La primera que pudo pronunciar algo
inteligible fue María Ignacia, que dijo: «Temblemos, señor de Santiuste,
que el caso no es para menos, y temblando podremos recobrar la
seriedad.»

---Creo, como usted---dije yo,---que esta comedia es el supremo arte de
los disparates graciosos\ldots{} Y en comedia tan chusca voy yo a
desempeñar un papel de clérigo: ya me han traído la ropa.

---Las cosas que inventa mi buen marido, no se le ocurren a nadie. Menos
mal si con estas tonterías se distrae\ldots{} Y a propósito: oiga usted
mis instrucciones, y sígalas al pie de la letra\ldots{} Pero entienda
que las instrucciones mías son reservadas, y que de esto no debe usted
darse por entendido con Pepe\ldots{} Irá usted, según creo, a un país
que está preparado para levantarse en armas al grito de \emph{Carlos VI
Rey}. No se meta usted donde haya jaleo de tiros y bayonetazos, ni nos
describa batallas sangrientas, sobre todo si en ellas ganan los
facciosos. Mucho cuidado con esto, Santiuste, porque Pepe, cuando le
hablan de triunfos del absolutismo, se me pone tan perdido de la cabeza
y tan arrebatado del temperamento, que me veo y me deseo para traerle a
la tranquilidad. Siempre que haya encuentros y agarradas feroces, con
heridos y muertos, tenga usted cuidado de decirle que ganan los
liberales\ldots{} Fíjese bien, Santiuste: que ganan los
liberales\ldots{} Si a mal no lo toma usted, le recomendaré que hable
poquito de las salvajadas de la guerra civil. Cuéntenos las guerras y
batallas de usted mismo, sus aventuras, cuitas o calamidades;
descríbanos costumbres no conocidas, sucesos que se aparten de lo
vulgar, escenas pintorescas, como lo que le pasó a usted en el
\emph{Fondac}; píntenos personas ridículas o hermosas, la blancura de
\emph{Yohar}, la fealdad negra de \emph{Bab-el-lah}, las hechicerías de
\emph{Mazaltob}\ldots{} Esto le encanta extraordinariamente a mi marido.
Anoche pasamos un rato delicioso leyendo el pasaje de la invisible
odalisca \emph{Erhimo}, y luego, hasta muy tarde, estuvimos discutiendo
si \emph{El Nasiry} le engañó a usted o no con aquella salida de que la
esclava es tuerta y le huele mal la boca\ldots{} Pepe sostiene que hubo
engaño y que \emph{Erhimo} es una preciosidad; yo estoy por la
contraria: creo que no hubo trampa, que \emph{Erhimo} es tuerta y sucia,
y que fue una gran suerte para usted la imposibilidad de libertarla.

\hypertarget{xv}{%
\chapter{XV}\label{xv}}

No seguimos, porque entró Beramendi. Su discreta esposa nos dejó solos,
después de decirle que ya me había informado de la terrible
conspiración, y que habíamos temblado y reído de aquel arcano tremebundo
y jocoso. De mal temple venía el Marqués, sin duda porque acababan de
darle informes nuevos, alarmantes. Ampliando lo que yo por su esposa
sabía, díjome que el actual plan del absolutismo no es un risible
sainete, sino un drama con gran arte compuesto. No se trata de quitarle
la corona a Isabel II, sino de \emph{cuajar el pacto de familia},
aprobado ya, según dicen, por una parte y otra. La rama femenina accede
a bajar del trono, con tal de ver restaurado el poder absoluto, puesta
en la cumbre la fe católica, y la Libertad en la situación que tiene el
diablo a los pies de San Miguel. Desde que la Revolución de Julio del 54
aterrorizó a la familia reinante, andan los de acá y los de allá en
tratos y contubernios. Dicen, y no les falta razón, que conviene
sacrificar algo para no perderlo todo. El Rey Francisco y don Carlos
Luis, heredero de los derechos de Carlos V, han tirado de pluma
grandemente en estos años, y de su continuada correspondencia furtiva ha
salido al fin el amasijo. Don Carlos Luis, Conde de Montemolín, subirá
al trono con la denominación de Carlos VI\ldots{} La actual Reina Doña
Isabel y su esposo se avendrán a una jubilación decorosa, conservando
título y honores de Reyes\ldots{} El hijo de Montemolín se casará con la
Infanta Isabel, y subirá al trono cuando cumpla veinticinco años\ldots{}
Isabel y Carlos reinarán juntos con igual derecho majestático, y se
titularán Segundos Reyes Católicos\ldots{}

«Esto es lo fundamental---añadió Beramendi.---De los principios
políticos que han de ser alma de este cuerpo, no tenemos noticia exacta.
Presumimos que caerá hecha cisco la Constitución, y que se hará un
llamamiento a todos los beatos furibundos para que vayan preparando la
traída de la Inquisición y demás zarandajas\ldots{} ¡Y que no han tenido
poco arte para organizar el movimiento! Existe, aunque esto te parezca
mentira, una \emph{Comisión regia suprema}, organismo hipócrita que se
ajusta dentro de las piezas del organismo visible del Estado. Esta
Comisión, compuesta de personas afectas al \emph{Pacto de familia}, se
ha dado buena maña para meter en todas las Capitanías Generales
individuos que trabajan en la sombra, y que han extendido por España una
red de voluntades absolutistas. Tiene ya la red tal extensión, que no sé
lo que aquí pasará si O'Donnell y su Ejército no vuelven acá de un
brinco. Confían los montemolinistas en que don Leopoldo tiene quehaceres
en África para un rato, y activan su organización\ldots{} Bien se ve que
quieren aprovechar esta soledad de tropa, las Capitanías Generales en
cuadro, las plazas desguarnecidas\ldots{} Lo peor, querido
\emph{Confusio}, es que si no miente el público secreteo, también en el
Ejército de África hay militares de todas graduaciones a quienes ha
comprometido para el alzamiento la maldita \emph{Comisión regia
suprema}. No quiero pronunciar ningún nombre ni dañar a ninguna
reputación, mientras no sepa la verdad. Dudo ya de todo, y no aseguro ni
niego la incorruptibilidad de nadie\ldots{} Vendrán los hechos, y todo
se aclarará\ldots{} La Historia que cuchichea me fatiga, me enloquece.
Venga de una vez la Historia que grita, aunque nos traiga desengaños y
catástrofes.»

---No pongamos tanta atención en la Historia inédita---le dije yo,---en
el caudal corriente de las conversaciones de hombres ociosos, porque
gastando nuestro corazón y nuestra mente en idear y sentir con
intensidad y en falso, derrochamos un tesoro anímico, sin sacar de ello
más que los pies fríos y la cabeza caliente\ldots{} Y pues tengo yo que
ir a donde están encendiendo la hoguera facciosa, dígame ya qué tengo
que hacer. Si efectivamente he de hacerme pasar por clérigo, sepa yo qué
clase de órdenes debo figurar en mí, pues como sean más de las menores,
en gran compromiso he de verme.

---Vas a un país revoltoso, nidal de fanatismo y \emph{partidaje}, donde
encontrarás infinidad de clérigos que habrán limpiado ya las armas para
lanzarse a pelear por Carlos VI. Conviene que con los curas pacíficos,
así como con los valentones, hagas buenas migas. Llevarás cartas de
recomendación muy eficaces. Con esto y con hacerte tú el apocado y el
santito, dando a conocer tus sabidurías de cosas dogmáticas y
litúrgicas, andarás por todo el país sublevado sin que nadie te moleste,
y observarás, y recogerás gran conocimiento, que me irás contando por
escrito, cuándo y dónde puedas. Hablemos ahora del nombre que te he
puesto, y que va ya expresado en las cartas de recomendación. Yo creo
que el \emph{Confusio} te va bien para segundo apellido. Quédate con el
nombre de pila, añadiéndole un patronímico cualquiera, y llámate
\emph{Juan Pérez de Confusio}. ¿Qué te parece?

---Como el \emph{Confusio} no les suene a mentira o artificio, paréceme
que no está mal mi nuevo nombre, y que da cierto eco de personalidad
erudita y casi filosófica.

---Verás cómo no te faltan lances peregrinos, quizás conquistas más
afortunadas que las de Marruecos. Aplica toda tu atención y el
sortilegio de tus gracias a las amas de cura, que por allá entiendo que
las hay muy guapas. Si pescas alguna, puede serte de mucha utilidad para
el estudio esotérico de nuestras guerras civiles\ldots{} Las cartas que
llevas han de abrirte holgados caminos. A más de las que yo te daré,
Manolo Tarfe te está preparando algunas que te causarán asombro cuando
las veas. Hoy está en Aranjuez. ¿Sabes a qué ha ido? A conseguir que te
recomiende una monjita de San Pascual, parienta suya. Manolo es de la
piel del diablo para estas cosas. En ellas está como el pez en el agua,
y cuando le toma el gusto a la intriga, se embriaga con las
dificultades, y acaba por realizar verdaderos prodigios. Con decirte que
pretende sacarle a sor Patrocinio una carta para no sé qué Provincial o
Prepósito de allá, está dicho todo. Nada, hijo, que irás bien favorecido
y hasta popeado de monjitas y con olor de santidad\ldots{} No te quejes.
Quisiera yo ser tú, y andar en esos trotes\ldots{} Mañana, ya dispuesto,
limpio de barbas, te vienes a recibir las cartas y nuestras últimas
advertencias, que por la tarde sin falta has de salir. ¡Dichoso tú mil
veces! Tú vives en España, tú la tratas íntimamente, tú gozas de ella y
en ella engendras los hijos de tu fantasía\ldots{}

Afeitadito, con todo el aire de un motilón ordenado de menores, me
presenté al día siguiente en la casa de mi protector, donde ya me
aguardaba el saladísimo Tarfe con las cartas que había conseguido en San
Pascual, de Aranjuez. Una le fue dada por su prima doña Margarita de
Barcones, monja profesa; otra llevaba la respetable firma de don Mateo
Valera, administrador del Real Sitio, y la tercera ¡ay!, la tercera
traía todo el olorcillo de un sagrado mensaje. Habíala escrito la mano
divina y llagada de la Madre reverenda. Iba dirigida al venerable
Vicario de Ulldecona, varón docto y bien calificado de virtudes,
carlista por los cuatro costados, con brillante hoja de servicios en la
anterior guerra civil, que ilustró con ruidosas hazañas. De mí decía la
carta lindezas que debo agradecer, aun considerándolas dictadas de la
travesura de Tarfe. Yo soy, según la carta, un joven de buena familia,
aplicadito desde mi tierna infancia a la piedad primero, a los estudios
religiosos después. Descuellan en mí las virtudes de humildad y
castidad, las cuales, con el adorno de mi sabiduría, me hacen amable, y
dueño de la simpatía de cuantos me tratan. ¡No me pusieron poco hueco
los elogios que hacía de mí la santa Madre!\ldots{} Mis nobles amigos me
recomiendan con la seriedad más socarrona que procure hacerme digno del
concepto que merezco, y me exhortan a seguir la senda de aplicación y
honestidad por donde llegaré a coger la breva eclesiástica que Dios
reserva a sus elegidos. En la carta de la Madre, así como en las otras
que Tarfe me ha traído, se dice que voy a completar mis estudios en el
Seminario Tarraconense, al paso que tomo posesión de una capellanía
heredada de mis ilustres antecesores\ldots{} Bueno, señor. Adelante con
la farsa, y Dios me saque vivo y sano del laberinto en que quieren
meterme estos exaltados caballeros.

Pasé un rato delicioso oyendo a Tarfe la descripción del interesante
convento de San Pascual, de Aranjuez, cuya importancia histórica quedará
bien patente con decir que a él tienen que acudir Narváez y O'Donnell
cuando desean el Poder o temen perderlo. Las manos guerreras que han
blandido la espada heroica, agarran un cirio y acompañan, con devota
flojera de miembros y ojos caídos, las procesiones que alrededor del
claustro limpio y oloroso se organizan un día sí y otro no para solaz
del Rey don Francisco de Asís. Según Tarfe, la enseñanza de señoritas
tiene en aquella casa una organización perfecta, según el moderno estilo
francés, sin que falte el adorno de piano y bailecito conforme a
etiqueta. La beatísima Patrocinio será lo que se quiera; pero de tonta
no tiene un pelo. La placidez y blancura de su rostro mueven a confianza
y piedad. En un aposento dispuesto con cierto artificio teatral y
amorosas obscuridades que inducen al misterio y la ilusión, tiene la
Madre su divino \emph{Cristo de la Palabra}, el cual, en instantes de
pío recogimiento, dice todo lo que debe oír y entender el candoroso
espíritu de la Reina. Ya está cansado el buen Señor de recomendar a
todos los individuos de las dos ramas borbónicas que hagan las paces y
vivan como hermanos; no se ha mordido la lengua para decir que por
ningún caso sea reconocido el Reino de Italia, y que se pongan todos los
obstáculos a la desamortización y venta de bienes de la Iglesia.
O'Donnell y Narváez, a cuyos oídos llegan más o menos pronto los buenos
consejos del Santísimo Cristo, no saben a qué santo encomendarse para
dejar contentos a todos, Trono y Pueblo, Altar y Tribuna.

Recorrió y examinó Tarfe todo el convento (que allí la clausura no rige
con los poderosos), y lo que más maravillado le dejó, despertando en él
envidia del ameno vivir de aquellas santas señoras, fue la magnífica
pajarera que allí tienen éstas para su recreo. No hay en todas las
Españas colección de pájaros tan variada y nutrida. Su Majestad el Rey
no repara en gastos para reunir allí las avecillas más bonitas, las más
exóticas, las de plumaje vistoso y las de canoro pico. ¡Vaya con el
museíto ornitológico! ¡Y que no se embelesa poco la Madre con los
tiernos hijuelos que a falta de otros le depara su valimiento! Monjas y
educandas se esmeran en instruir a las especies habladoras,
familiarizándolas con las formas corrientes del lenguaje. Cuenta Tarfe,
y porque él nos lo ha dicho lo creemos, que en la sección de loros hay
uno tan bien enseñado, que dice \emph{Jesús} cuando Sor Patrocinio
estornuda.

Escribo en mi casa el final de esta larga epístola, para dejarla con su
debido remate antes de lanzarme por el camino de mis desconocidas
andanzas. Concluyo diciendo que como el tiempo apremia y tengo que
prepararme para la partida, dejé la morada de Beramendi. Éste me dio sus
últimas instrucciones en cuatro plieguecillos de papel bien aprovechados
de letra, y me encargó muy encarecidamente que por el camino me aprenda
de memoria el texto de los pliegos, y luego los rompa. A los libros de
Teología que llevo, agregó un tomo del \emph{Concilio de Trento},
\emph{El Genio del Cristianismo} y la \emph{Vida de Jesús del Padre
Rivadeneyra}. Ha insistido en que no debo escribir con la idea de que
sea él mi único lector: conviene que mis relatos vayan mentalmente
dirigidos a mayor público y a la misma Posteridad, que nunca podría
decir: «de aquella agua no beberé.» Sin pensarlo, vengo yo aderezando
mis cartas como si hubieran de ser gustadas por innumerables lectores.
Ahora lo haré con más determinado propósito, alentado por mi Mecenas, el
cual me recomienda una y otra vez que, por miedo a una publicidad
remota, no recorte ni desfigure la narración de mis sucesos y
trapisondas personales. Está muy bien: como me llamo \emph{Confusio},
que así lo haré.

Me ha marcado el Marqués este itinerario: saldré en la diligencia de
Guadalajara y Zaragoza, siguiendo en ella de un tirón hasta Alcolea del
Pinar. En este pueblo, un amigo y colono de mi protector cuidará de
encaminarme a Molina de Aragón; traspasaré después la Sierra Menera para
entrar en la provincia de Teruel. Las observaciones que haga por el
camino me indicarán si debo dirigirme a la noble Alcañiz o a la vetusta
Morella. En una o en otra comarca ha de estar la mayor rescoldera del
volcán por donde voy a pasearme. Quedo en libertad de escoger la ruta
conveniente, según lo que oiga y vea por esos endiablados pueblos.
Dineros llevo cuantos pueda necesitar, pasaporte en regla, y cartas para
señores sacerdotes o caballeros pudientes, que mirarán por mí si me veo
en algún peligro. Yo nada temo; confío en mi buena estrella, y en salir
con donaire de cualquier mal paso en que mi curiosidad o mi arrebatado
temperamento me metiesen.

Arreglo mis asuntos con la patrona; doy la última mano a la ordenada
estiba de mi ropa y libros en la maleta; me da el corazón una o más
punzaditas al acordarme de Lucila y Vicente, a quienes no veré
más\ldots{} me acuerdo también de \emph{El Nasiry}, y hago voto de
decirle algún día cuatro frescas si descubro que me engañó poniendo
lacras y pestilencia sobre el invisible rostro de la hermosa
\emph{Erhimo}\ldots{} Entra Beramendi en mi modesto cuarto; me da prisa.
Escribo rápidamente el final de ésta, y se la entrego para que la lea y
archive\ldots{} Adiós, Madrid mío. Ahí te queda un suspiro del pobre
\emph{Confusio}.

\hypertarget{xvi}{%
\chapter{XVI}\label{xvi}}

\textbf{Foz Calanda}, \emph{Abril}.---¡Ay qué pueblos, qué posadas, qué
caballerías, qué arrieros de Dios y qué caminos del diablo! He recorrido
con mala sombra una de las comarcas más características de la guerra de
facciones. La humanidad, lo mismo que la geografía, se me han
representado como expresión viva de la bárbara epopeya
cabrerista\ldots{} Dudo si el país por donde voy hizo la campaña, o es
obra y hechura de ella. Ruinas y desolación veo por todas partes,
veredas de guerrilleros, emboscadas de asesinos, burladeros naturales
para la sorpresa y la traición\ldots{} Más acá de un pueblo que llaman
\emph{Cosa}, estuve a punto de perecer ahogado, vadeando un río nombrado
\emph{Pancrudo}; y al venir de Montalbán a Gargallo, faltó poco para que
me despeñara en una sima, por cuyo borde serpentea el camino pedregoso.
Las lomas y cerros, de un conglomerado rojizo, eran como sangrienta
visión que me seguía tomándome las vueltas. Entre Alcorisa y este lugar
donde escribo, se me cambió en próspera la adversa suerte, porque
acompañado vine por un cura viejo y bondadoso que, emparejando su
jamelgo con el mío, me entretuvo por todo el camino con su conversación
amena. Mi buena facha, mi lenguaje modoso debieron de cautivarle, porque
no esperó a que yo le mostrara las cartas que llevo, para ofrecerme,
como párroco de este pueblo, campechana hospitalidad en su casa.

Y aquí me tenéis bien alojado y bien comido en esta vivienda modesta,
mas no desprovista de sabrosas vituallas; vedme tratado hidalgamente por
el cura, que es un bendito, y asistido hasta con mimo por dos amas
viejas, corcovaditas\ldots{} El sitio y las personas me recordaron los
tranquilos días de Samsa, en las inmediaciones de Tetuán\ldots{} Aquí
recibo los primeros rumores del anunciado alzamiento que motiva mi
viaje, noticias que al cura y a mí nos han parecido fantásticas. Mi buen
párroco no es menos pacífico que yo ni menos aborrecedor de la
guerra\ldots{} Como digo, las noticias traían todo el cariz de un
tremendo embuste. Ved la muestra: El Rey Carlos VI había desembarcado en
los Alfaques con un poderoso ejército. ¿De dónde venía? De la isla de
Ibiza o de islas de Italia: a punto fijo no se sabe. Al desembarcar en
tierra española se pronunció Tortosa\ldots{} Ya iba el Rey camino de
Zaragoza, engrosando a cada paso su ejército, pues todas las tropas de
Isabel se agregaban a las de su primo\ldots{}

Con recelo de que tal notición fuera verdad, un ejemplo más de la
verosimilitud de lo absurdo en nuestra patria, me dormí aquella noche,
arrullado de mi cansancio, y a la mañana siguiente, cuando una de las
viejas me trajo el chocolate, entró don Miguel Castralbo, que tal es el
nombre de mi huésped, y me dijo: «Ya van llegando vientos de verdad, que
desvanecen las mentiras que oímos anoche, señor de \emph{Confusio}.
Parece cierto que ha llegado el Montemolín con tropas sublevadas de no
sé qué islas; pero no ha tenido, al parecer, recibimiento feliz, porque
los mozos que de estos pueblos salieron armados para guerrear en la
facción, vuelven a toda prisa. He visto a algunos; les he preguntado, y
no dicen más sino que vuelven y corren para acá, porque han visto que a
la carrera volvían los de Calanda y Alcañiz. Por allá deben de soplar
aires de miedo\ldots{} Mientras fijamente no se sepa lo que ocurre, yo
que usted, señor de \emph{Confusio}, no me movería de ésta su casa,
donde puede estarse todo el tiempo que le pida su cansancio.» Las amas,
que ya empezaban a tomarme ley, apoyaron con chillones encarecimientos
esta exhortación a la holganza; di las gracias, y echándomelas de muy
valiente, les aseguré que, aunque hubiera de pasar por el cráter de un
volcán en erupción, seguiría mi camino sin vacilar\ldots{} Discutimos;
no me convencieron\ldots{} Partí.

~

\textbf{Alcañiz}, \emph{Abril}.---En Calanda y aquí he visto confirmadas
la dispersión y retroceso de los que iban al juego de la guerra civil.
Alojado estoy en un decente parador, y por la ventana de mi cuarto, que
da a la plaza, veo el lindo frontispicio del Ayuntamiento. Me encanta
este rincón monumental casi tanto como las dos mozas que me sirven, la
una tirando a lo gótico, la otra a lo ático\ldots{} Nada, que me gusta
este pueblo, en el cual he admirado bellas iglesias románicas y del
Renacimiento, amén del mujerío, que es de orden compuesto, quiero decir,
de la hermosa mesticidad celtíbera y moruna\ldots{} Los compañeros de
mesa me han informado del levantamiento carlino, calificándolo de
fracaso tan escandaloso y grotesco, como ha sido insensata y absurda la
intentona. Dijo uno que Montemolín había venido de Mallorca con la
guarnición sublevada de aquella isla; otro aseguró que vino de Marsella;
un tercero puso las cosas en su lugar, refiriendo que de Baleares llegó
el general Ortega, cabeza visible del alzamiento, con las tropas de su
mando, las cuales al punto de tocar tierra se llamaron andana y
dejáronle solo\ldots{} Pronunciamiento más desatinado no se había visto,
ni operación militar que más se pareciese a una correría de traviesos
muchachos.

Como liberal habló uno de los huéspedes, desatándose en injurias contra
los montemolinistas y sus auxiliares por haber hecho tal barrabasada
cuando tenemos en África casi todo el Ejército. Alzáronse al oír esto
voces que apoyaban al preopinante, otras que lo contradecían, y del
extremo de la mesa soltó un bárbaro la bomba de que algunos de los
Generales de África estaban comprometidos, entre ellos Prim. ¡Jesús, la
que se armó cuando el nombre del héroe sonó en medio del tumulto! El que
parecía liberal dijo al otro que mentía: mediaron tonantes vocablos de
cólera; levantáronse uno y otro, y venciendo a saltos el espacio que los
separaba, agarráronse de manos y tiráronse de pelos\ldots{} A separarlos
corrimos los demás; yo fui de los más presurosos en poner paz, lo que me
costó un rasguño, varios pisotones, y en el brazo izquierdo un golpe que
me hizo ver las estrellas.

\textbf{Ulldecona}, \emph{Abril}.---El hilo que solté en el comedor de
Alcañiz, lo recojo ahora para proseguir desde aquel punto la relación de
mi viaje y aventuras, que hasta los últimos días, en lo que ahora voy a
contar, no ofrecen sino sucesos comunes indignos de ser escritos. Salí
de Alcañiz con marcada variante de mi rumbo presupuesto, porque las
muchachas bonitas, gótica la una, ática la otra, que servían en la
posada, me aconsejaron que no tomara el camino de Valdetormo y
Calaceite, directo a Gandesa y Tarragona, porque allí corría el riesgo
de que me salieran, si no facciosos, bandidos que en aquellos caminos y
puertos hacen de las suyas. Demostrándome más interés que el que yo
merecía por el simple hecho de alabarles la hermosura, me señalaron como
más práctico y seguro, aunque más largo, el camino que, cortando tierras
del Maestrazgo, va a salir por la Cenia a las tierras bajas del Ebro.
Así lo hice, y llegado sin tropiezo de ladrones a donde ahora me
encuentro, no puedo decir si el consejo de las lindas mozas a mi ventura
o a mi perdición me ha conducido.

Toda la noche anduve en una tartana que iba nada menos que a Vinaroz, y
llevaba, a más de mi persona, dos monjas de una Orden para mí
desconocida, viejas y adustas, y un señor de edad provecta, con trazas y
rudeza de hombre de mar. Ni ellas ni él hablaban más que catalán
cerrado, que yo no entendía, y todos mis esfuerzos para entablar
conversación me resultaron inútiles, viéndome condenado a un hosco
silencio que me hacía más molestos los tumbos y sacudidas espantosas de
aquel vehículo del diablo. Aun entre sí, no eran comunicativos mis
compañeros de suplicio, pues las monjas no hacían más que rezar, y el
marino, si es que lo era, compartía el tiempo entre las modorras con
ásperos ronquidos y las maldiciones seguidas de toses y carraspeos.
Nunca tuve ni padecí travesía tan mala y tediosa.

En vano traté de congraciarme con las monjas, haciéndoles comprender mi
carácter sacerdotal, ya con algún latinajo, seguido de exhortación a la
paciencia, todo sin venir a cuento, ya procurando que el gesto y el
mirar expresaran mi estado y mansedumbre; pero ni por ésas. No he visto
seres más huraños y recelosos. Sin duda son religiosas de clausura que,
al ir de trasiego de un convento a otro, van espantadas por el mundo,
como el ganado lanar cuando lo hacen pasar por las calles de una
población\ldots{} Mi terrible encierro con semejantes fieras tuvo su fin
en un caserío de cuyo nombre me alegro de no acordarme, pues en él mis
desventuras no hicieron más que cambiar de forma. ¡Qué tal sería el
pueblecito, que me vi y me deseé para encontrar algo parecido a un
colchón donde tender mis huesos por unas cuantas horas, y algún alimento
con que engañar el hambre! Habíanme dicho que allí abundaban las
tartanas de alquiler; pero ninguna pude hallar, ni aun ofreciendo pago
doble y triple de lo acostumbrado. ¿Dónde diablos estaban las tartanas?
Una vieja cejijunta, displicente y con ojos de sibila, me dijo que los
coches se habían ido a los juncales del Ebro, y allí se los había
tragado el fango.

Al cabo de mil diligencias y pasos fatigosos, me sacó de mis apuros un
trajinante con quien ajusté dos caballerías, una para mí y otra para él
como escudero y portador de mi maleta. Y heme otra vez en camino, a
media tarde ya, sufriendo la bofetada continua de un viento que de cara
nos azotaba cruelmente. Ambas caballerías venían cansadísimas de
anteriores trabajos, sin pienso, y para curarlas de su pereza no había
otra medicina que los palos. Mi jaco era de tan aviesa condición, que en
algunos repechos del camino no andaba ni adelante ni atrás\ldots{} Fue
mi viajecito más triste y desesperante al entrar la noche; el viento no
amainaba; los caballos vengaban en mí la ruindad de su amo; a éste
hubiera dado yo los palos que las pobres bestias recibían; eché de menos
la tartana de la noche anterior, y acordándome de las monjas, me las
figuré graciosas y amables: tal era mi furor en aquella desgraciada
travesía. Para mayor enojo mío, el maldito jayán escudero se había
vuelto mudo. Hacíale yo preguntas, que bien respondidas habrían dado
algún alivio a mi dolorosa impaciencia. ¿Tardaremos mucho? ¿Cuánto hay
de aquí a la Cenia? ¿Qué caserío es éste?\ldots{} Pues el muy bestia,
resguardándose con la blandura de su manta el pecho, pescuezo y boca, o
no decía nada, o me soltaba un ronco mugido, como un mastín con más
ganas de morder que de ladrar.

Deploraba yo además la soledad, el no encontrar arrieros ni caminantes;
y tanto silencio y monotonía, sin oír otra voz que la del viento ni ver
caras de personas, me desesperaba\ldots{} «¿Pero dónde estamos? ¡Qué
país tan desolado y triste!» A esto, mi escudero no decía más que
\emph{muú}, y en mí se acentuaban las ganas de pegarle un tiro\ldots{}
Grande alegría me causó de improviso ver una luz lejana. ¿Estaría en
aquella luz el paso de la barca? \emph{Muú}\ldots{} ¿Era luz de un
farol, luz de un hacho? \emph{Muú}\ldots{} Los caballos, contagiados de
mi impaciente gozo, avivaron un tanto su perezoso andar\ldots{} Nos
acercábamos a la luz, y la luz hacia nosotros venía presurosa\ldots{}
Por fin, me vi frente a unos cuantos hombres que gritaron ¡alto! La luz
era una antorcha resinosa, los hombres un hato de bárbaros insolentes.
Vestían el traje catalán con faja colorada, y en vez de barretina
llevaban pañuelo liado a la cabeza, a estilo valenciano más que
aragonés. Todos iban armados con escopetas, trabucos o pistolas. Mi
primera impresión fue que había caído en poder de bandidos. Luego,
oyendo sus preguntas atropelladas, me creí frente a una de esas
terribles organizaciones político-militares que llamamos partidas.

Mi escaso conocimiento del catalán me bastó para entender las preguntas
que me hicieron aquellos brutos: «¿De dónde vienen ustedes? Sepamos
quiénes son\ldots{} ¿A dónde van? ¿Han dejado atrás fuerzas del
Ejército? ¿Viene Guardia civil?» Contestaba \emph{muú} mi escudero, y
yo, con mejor tono y cortesía, expresé la verdad. No debí de
convencerles\ldots{} desconfiaban de mí. Con malos modos me mandaron que
me apease. Uno me tocó todo el cuerpo, preguntándome si llevaba
pistolas. Díjeles que, como sacerdote que soy, no llevo armas ni para
nada las necesito. Hablaron de registrar mi maleta, y no me opuse: al
contrario, abriéranla cuando quisieren, y verían en ella tan sólo mi
ropa, mis libros de religión, y las cartas que llevo para diferentes
personas del clero y la nobleza, todas muy calificadas\ldots{} El que
parecía sargento de tan desaliñada tropa me mandó con grosero despotismo
\emph{arrear} a pie, y obedecí silencioso, emprendiendo la marcha
rodeado de aquellos gandules. Delante iba el que alumbraba. La antorcha,
con la furia del viento que desgreñaba la llama y consumía las hebras de
fuego deshaciéndolas en chispas, perdió su fuerza y su luz; el viento
devoró las últimas ráfagas, dejándonos a obscuras. Seguí yo andando a
trompicones, sin saber dónde ponía los pies. A mi lado iba el sargento o
lo que fuese; detrás mi escudero; uno de la partida llevaba de la brida
los dos rocines, que agradecieron mucho que se les aliviara de nuestro
arreara cuando el temor de caerme en un hoyo o de tropezar en una piedra
obligábame a moderar el paso.

Y en aquella procesión lúgubre, me acordé de las instrucciones
consignadas en los pliegos de Beramendi, leídos cien veces por mí entre
Madrid y Guadalajara, y después de bien aprendidos, rotos y dados al
viento. Descollaba en mi memoria un substancioso parrafillo, que así
decía: «Si llevas muchas probabilidades de ser obsequiado de curas,
favorecido por sus amas, y de que todos se rindan a tu talento y
simpatía, también las llevas de caer en manos de guerrilleros feroces,
que te fusilen por primera providencia. En este caso, mi querido
\emph{Confusio}, sabrás morir como cristiano caballero y como sacerdote,
apartando con desprecio tus ojos de las vanidades humanas, y
volviéndolos a la vida perdurable, donde hallarás el premio de tus
virtudes.»

\hypertarget{xvii}{%
\chapter{XVII}\label{xvii}}

«¡Ay de mí! ¡Pues tendría gracia---pensé yo en el obscuro camino---que
estos animales me pegasen cuatro tiros!\ldots» Pensándolo, vi luces
rastreras, como de farolitos llevados a mano\ldots{} Se movían delante
de nosotros, con lenta derivación hacia la izquierda\ldots{} Este mismo
rumbo tomamos siguiendo un recodo del camino\ldots{} Cuando estuvimos
cerca distinguí un grande y negro caserón, y varios hombres que con sus
propias sombras se confundían. Del grupo se destacó un corpacho. Le vi
llegarse a mí. Era un sujeto de muy aventajada estatura, cincuentón, y
vestía con más decencia que los otros. «Este tío---pensé yo---será el
capitán de la partida. Su facha es de persona de calidad, aunque el
gorro de pieles que trae calado hasta las orejas le da cierto aspecto de
ferocidad montuna.» De sus hombros pendía suelto de mangas un capote.
Toda su ropa era negra, y el pantalón gris colán; llevaba botas de alta
caña. Apenas llegó frente a mí, repitió las preguntas de los otros con
voz tan bronca y adusta, que temblé al oírla, y me dije: «Este tío me va
a dar un disgusto.» Reiteré mi respuesta: que yo no sabía si venían o no
detrás de nosotros tropas del Gobierno. «Pues un batallón salió esta
mañana de San Mateo---dijo el talludo y truculento señor.---¿Dónde están
esas tropas? ¿Han ido a Vinaroz?\ldots{} Si saben ustedes el camino que
han tomado y no quieren decirlo, a uno y otro les participo que lo
pasarán mal\ldots» Y otra cosa: «La Guardia civil de los puestos de
Chert y Ballestá, ¿dónde se ha ido? ¿Por ventura supo que estamos aquí y
nos cogió miedo?» Yo declaré no saber nada, y poniendo en mi acento toda
mi sinceridad, esperaba que mi inocencia quedaría bien clara. El que yo
creía sargento habló en voz queda con el cabecilla. Y éste ordenó que se
nos registrase detenidamente. Entramos todos en el caserón, y el
hombracho iba tras de mí rezongando con ira y mofa: «Ha dicho que es
sacerdote\ldots{} Ya lo veremos. Y trae cartitas de
recomendación\ldots{} Las veremos, sí, señor, las veremos, y ojalá sean
para quien yo me figuro.»

Metidos en un cuarto estrecho, donde vi una mesa manchada de vino,
porrones medio vacíos, cortezas de pan, una silla de paja con el asiento
casi deshecho, y un banco desvencijado como los que hay en ínfimas
tabernas de aldea, se procedió al registro de mi maleta, el cual fue por
extremo detenido y escrupuloso. El cabecilla presidía la operación en
pie, junto a mí, y no quitaba ojo de lo que iban sacando los
registradores. Éstos eran dos, y dos brutos más habían entrado para mi
custodia. Desdoblaban la ropa, y en las prendas que tenían bolsillos no
había hueco ni pliegue que no escudriñaran. Los libros eran cogidos por
el jefe, que al leer las portadas con cierto énfasis, revelaba más
sorpresa que pedantería. Cuando salió de entre otros papeles mi
pasaporte, le echó con avidez la garra, y leído por dos veces, dijo
entre burlón y receloso: «¡Qué apellido tan raro éste de
\emph{Confusio}!\ldots{} Es la primera vez que veo un cristiano que así
se llame.» Yo le advertí humildemente que la familia de los Pérez de
\emph{Confusio} es muy conocida en Medinasidonia y otros pueblos de la
provincia de Cádiz. Antes de que pudiera oírme, vio las cartas de
recomendación, y cogido el no pequeño rimero de ellas, las fue
examinando, y a cada nombre que leía, soltaba de su boca una breve
expresión de asombro, acompañada de un mohín de labios o chasquido de
lengua. Las expresiones eran: «¡Anda!\ldots{} ¿Pues y ésta?\ldots{}
¡Vaya, vaya!\ldots{} Bien, bien\ldots» Al llegar a una que despertó su
interés más que las otras, rápidamente la desdobló y con ansiosa lectura
enterose de su contenido, pasándola de la cruz a la fecha. Después, sin
mirarme, volviose a los bárbaros, que, una vez vaciada la maleta,
golpeaban el fondo y costados por si el sonido les denunciaba trampa o
secreto, y con imperiosa voz les dijo en catalán: «Ea, basta ya: ¿no
veis que no hay nada? ¡Pues no sois poco sobones!\ldots{} Digo que
basta\ldots{} Idos afuera.» Salieron los hombres atropellándose, que ya
sabían cómo las gastaba su jefe; cerró éste la puerta, y llegándose a
mí, me indicó con ademán cortés que me sentase\ldots{} Obedecí al
momento. No me dio tiempo a pensar nada de aquel extraño cambio de voz y
maneras, y antes de sentarse frente a mí, me habló en castellano neto de
este modo: «Al ver esa carta para el Vicario de Ulldecona, me picó tanto
la curiosidad, que\ldots»

---Puede usted leerlas todas si gusta---le contesté, correspondiendo a
sus buenos modos con los míos.

---No\ldots{} gracias, señor de \emph{Confusio}\ldots{} Pues ha de saber
usted que el Vicario de Ulldecona soy yo.

Prorrumpí en exclamaciones de sorpresa, y atropelladamente me congratulé
de la felicísima casualidad que me deparaba el Acaso, o por hablar
mejor, la Providencia. ¡Quién había de decirme\ldots! «Vea usted, señor
Vicario, cómo las situaciones más desfavorables, o si se quiere más
obscuras y pavorosas, se iluminan de improviso por el divino rayo de la
verdad.»

---Exacto: usted me temía, y ahora un rayo de verdad nos hace
amigos\ldots{} Pero no me llame usted señor Vicario, que en esta
diócesis no está en uso tal denominación. Soy el Arcipreste de
Ulldecona. Más de una vez he dicho a la \emph{Madre}, cuando he tenido
que escribirle, que no me llame Vicario, sino Arcipreste; pero no se
acuerda, no se acuerda\ldots{} Y ante todo, ¿cómo está la \emph{Madre}?

---Tan buena\ldots{} Fresca como una rosa, y sin perder nada de aquel
despejo, que es, digo yo, uno de los dones más maravillosos que debe al
Señor.

No me pareció muy vivo el interés del Arcipreste por la bendita y
llagada monja. Su pregunta no había sido más que fórmula fácil de
rudimentaria cortesía. Al instante varió de conversación. Refregándose
la frente con una mano, después con otra, como quien quiere aligerar su
pensamiento de preocupaciones y cuidados opresores, me dijo: «Se habrá
usted enterado de lo que aquí pasa\ldots»

---Sí, algo sé. En Alcañiz oí noticias confusas, incompletas\ldots{}
Desembarco de tropas en los Alfaques.

---En San Carlos de la Rápita desembarcó la locura. Venía guiada por la
necedad, y a recibirla salió la ceguera. ¡Ja, ja!\ldots{} ¡Y nos habían
hecho creer que todo lo tenían muy bien dispuesto\ldots{} que Francia
estaba en el ajo\ldots{} que Madrid se pronunciaba, que \emph{Palacio}
se pronunciaba, y que Prim en África se pronunciaba!\ldots{} ¡Majaderos,
canallas, mentecatos!\ldots{} Lo que aquí se pronuncia es el sentido
común, que no quiere ser español, y se va; la vergüenza, que se va; el
arranque y las ternillas de hombre, que tampoco quieren estar en esta
tierra gobernada por mujeres. Bien merecido les está el fracaso, por
fiarse de Ortega, por fiarse de los de Madrid, por fiarse de\ldots{}

Hizo breve pausa, comiéndose el final de la frase\ldots{} Clavó sus ojos
en mis ojos, y posando su mano en la mía, me dijo: «Pues hemos de ser
amigos, contésteme pronto a lo que le pregunto: ¿a más de la carta que
he leído, no tiene para mí un mensaje verbal de la Madre o de otras
personas?»

---No, señor Arcipreste.

---Y para otros señores eclesiásticos o seglares, ¿no trae recadito de
palabra, debajo del disimulo de las cartas de recomendación?

---Aseguro a usted---respondí con desahogada sinceridad---que no traigo
más que lo que ha visto.

---Por las Ánimas del Purgatorio, o hay confianza o no hay
confianza\ldots{} Usted teme\ldots{} Aún no se le ha pasado el susto de
esta sorpresa\ldots{} Serénese y dígame la verdad.

---La verdad he dicho. Soy un seminarista obscuro, alejado de toda
intriga, y aquí vengo no más que al negocio particular de mi capellanía
y a mis estudios.

---Así será\ldots{} Perdóneme. Me pasó por el magín la idea de que nos
traía usted instrucciones\ldots{} que ya no serían instrucciones, sino
cataplasmas tardías de los que en Madrid calentaron este movimiento y
luego se han quedado fríos, zurrándose de miedo\ldots{} Pensé que usted
venía para decirnos: «Perdonen por hoy, que otra vez será.» Veo que se
asombra de oírme\ldots{} Voy creyendo que está completamente en ayunas
de todo lo que pasa aquí y en Madrid, y en Francia y más allá de
Francia. Si es usted un ángel, nada más tengo que decirle sino que le
aproveche su inocencia.

---Un ángel soy, no vacilo en decirlo, en todo eso que a usted tanto le
afana.

---¿Y no sabe que contábamos con el apoyo de ese zascandil, de ese
peine\ldots?

---¿Quién, señor?

---Es usted, en efecto, el más puro de los serafines si no sabe que nos
ofreció protección, y no ha cumplido, ese buscarruidos, ese\ldots{} no
quiero llamarle por su nombre\ldots{} \emph{el marido de la
Eugenia}\ldots{}

---¡Napoleón III!

---Así lo llaman los que creen en el imperio francés\ldots{} ¡Farsa,
mujerío indecente!\ldots{} Pues en Madrid, digamos en \emph{Palacio}, se
habrán echado atrás, por influencia de la Inglaterra. ¿No cree usted lo
mismo?

---Yo, señor Arcipreste, nada entiendo de esas cosas.

---¿Pero no saben que Inglaterra protege al Progreso y a la Masonería,
porque así se lo manda el Protestantismo? Los progresistas cuentan con
el apoyo de Inglaterra, protectora de la Unión Liberal, de O'Donnell, de
Prim, y de este maldito Dulce, que manda en Cataluña\ldots{} La
Inglaterra se ha metido donde no la llamaban, y \emph{Palacio} se ha
zurrado de miedo. La familia reinante usurpadora había entrado ya por el
aro, aviniéndose al arreglo y transacción de los derechos de unos y
otros Borbones; acordada estaba ya la forma y modo de establecer la gran
Monarquía católica, perpetua y definitiva\ldots{} y ved aquí que los
reinantes de Madrid dicen \emph{yo no juego}, y se vuelven atrás,
dejando a los leales en la estacada\ldots{} Ello habrá sido por
metimiento de la Inglaterra\ldots{} Pues espérense un poco, que ya
recibirán su merecido. Con el apoyo y el dinero inglés, los progresistas
y O'Donnell y toda esa taifa darán cuenta del Trono\ldots{} Créalo
usted, señor \emph{Confusio}: hemos de ver a \emph{la Isabel} emigrada y
sin un real, teniendo que lavar la ropa de \emph{la Eugenia} para
ganarse un triste cocido\ldots{} No se ría, ángel, que eso lo verá
usted, que es un joven, y yo también, que ya voy para viejo\ldots{}
porque irá de prisa, muy de prisa, la descomposición y ruina de las
cosas.

Se puso en pie con viveza juvenil, y abrió la puerta para llamar a su
gente. «¡Eh, canalla, venid aquí!» Apenas entró la turba de gaznápiros,
el Arcipreste dijo al que me había registrado la maleta: «Pon todo
conforme estaba. ¡Eh!, colocar cada cosa en su sitio\ldots{} ¡Cuidado,
bruto!\ldots» Y a otros: «Tú, Gasparó, llevarás a casa la maleta. Tú,
Rufulet, coge un farol y alúmbranos.» Y a mí: «Señor \emph{Confusio},
despache a su espolique y véngase conmigo.» Salimos\ldots{} Andando
entre bardales, por un caminejo de cuyos peligrosos altibajos me
defendía la ondulante claridad del farol delantero, dije al que ya
consideraba como amigo: «Señor Arcipreste, ignoro dónde estoy. ¿Es esto
Ulldecona?»

---No, señor: esto es Rosell de la Cenia. Tengo aquí una masada, donde
suelo venir a pasarme algunos días de campo con mi familia o parte de
ella. El lunes me vine acá\ldots{} quería descansar de los berrinches de
estos días, por el desembarco de necios y locos\ldots{} y de paso, dar
gusto a las aficiones, al deber que uno tiene de no perder ripio\ldots{}
¿Usted me entiende? Me traje unos cuantos escopeteros con idea de
acechar el paso de la Guardia civil\ldots{} Parece que olieron mi
presencia, y se fueron por otro lado. Fácil nos hubiera sido merendarnos
a los guardias, y lo mismo digo de la tropa, no siendo mucha.

Yo callé. Volví a sentir miedo del hombre en cuyo poder estaba\ldots{}
Pero me dejé llevar de él confiadamente, pensando que la mejor regla de
conducta en toda vida de aventuras es entregarnos a la desconocida
voluntad del Destino, o de su hermana la Providencia. Sin hablar cosa de
interés, pues no lo tuvieron las breves observaciones acerca de la
molestia del viento y de la obscuridad de la noche, recorrimos en unos
veinte minutos el camino que nos llevó a la masada, y en ésta, saludados
de perros y recibidos por un viejo y dos mujeres, entramos en el caserón
campesino, que al primer vistazo me pareció alegre, holgón, cómodo y
bien abastecido para un vivir regalado. Del portal ancho, lleno de
aperos, pasé a una gran estancia, donde vi una escalera de fábrica, que
a los pisos superiores en dos tramos conducía; al fondo, otra pieza que
era la cocina, con resplandor de fogata y excitantes olores de comida, y
a derecha mano, un aposento blanco y espacioso con mesa ya puesta para
tres personas. Allí nos metimos, y el señor Arcipreste, desembarazado de
la gorra de piel y del capotón, se me presentó en toda su gallardía
simpática. Era un hombre alto, sanguíneo, vigoroso, de perfecta
escultura esquelética y muscular, arrogante de actitud, ardiente la
mirada, garboso el gesto. Iluminado de lleno el rostro por la luz de una
buena lámpara, su edad me pareció de más de cincuenta años, o de sesenta
desmentidos por una salud venturosa. Era su color encendido, su nariz
enérgica, su boca desconfiada, el cabello espeso, cortado al rape, y
blanquecino por las sienes, la dentadura recia y blanca.

A la mujer de mediana edad que recogió el capote y montera, le ordenó
que nos diese pronto de cenar, añadiendo: «Para este caballero y para mí
solos.» Su voz y su acento sonaban a dominante autoridad sin altanería.
Otra mujer, de apacible madurez, puso la mesa, en que advertí blancura
de manteles y fineza de loza que me causaron sumo agrado. ¡Y con el ama
presente, ya eran dos las que yo veía! La tercera apareció después
trayéndonos una sopa calduda, hirviente, con huevos, capaz de matar el
hambre con sólo la rica fragancia que despedía. Mi apetito era
monstruoso, como de náufrago perdido en una isla desierta. Pedí permiso
al Arcipreste para caer sobre la sopa con devorantes ansias, y me lo
concedió risueño, asegurándome que él haría lo mismo\ldots{} Y comiendo,
no perdía yo la cuenta de las amas que veía, ni dejaba de observar el
rostro de la tercera, que era bonita, aunque demasiado pálida, con
cierto aire y mohín lacrimoso de Virgen de los Dolores, de buena talla,
pero ya deslucidita de pintura y barniz.

De mis disimuladas observaciones me distrajo el señor Arcipreste,
dándome noticias de su persona, antecedentes y circunstancias. «Por mi
habla---me dijo---habrá usted conocido que no soy catalán. Hablo
castellano, sí señor; he mamado esta lengua de los mismos pechos que
Cervantes, el portento de la literatura, porque nací como él, en Alcalá
de Henares, y allí me crié y viví hasta que, ya mocetón hecho, me
llevaron mis padres a Híjar, tierra de Teruel. Ésta es mi patria
efectiva, pues en ella fui hombre y recibí las órdenes sagradas,
desempeñando varios curatos buenos, hasta que me trajo a este
Arciprestazgo, diez años ha, mi amigo don Isidro Losa, de quien me viene
mi conocimiento con la madre Patrocinio. Mi nombre es Juan Ruiz; añado a
este primer apellido el de mi madre, que es \emph{Hondón}, por lo cual
unos me dicen mosén Hondón, y aquí, entre mis feligreses, se ha hecho
moda, por aquello de abreviar y dar gusto a la lengua, llamarme
\emph{Don Juanondón.»}

En esto vi que con el ama que empezó a servirnos entraba otra. ¡Ya eran
cuatro, Señor! Y no era lo peor que fuesen cuatro, sino que la última, o
sea la cuarta, era más joven, por lo menos más lozana que la parecida a
la Virgen de los Dolores, y seguramente más bonita: una rubia ideal, de
azules ojos, cara como las rosas, no muy alta de cuerpo, pero éste muy
bien modelado en sus partes todas, y con admirable distribución de
carnes en sus contornos y bultos, resultando de tales armonías una
combinación feliz de la agilidad y el buen desarrollo. Allí se juntaban
las dos bellezas fundamentales: la gracia y la salud.

\hypertarget{xviii}{%
\chapter{XVIII}\label{xviii}}

Habían acudido al comedor las dos amas, sobrinas o lo que fuesen, porque
eran necesarias a nuestro servicio. La joven de dorados cabellos mudaba
los platos; la jamona, que era de buen ver, como un ocaso de dorada
tibieza, descuartizaba unos pollos que pronto habíamos de comer. Los
movimientos de una y otra no se me escapaban, aun poniendo las
apariencias de mi atención en don Juan Ruiz, que así proseguía contando
su novelesca historia: «En mi curato de Híjar, y antes en los de
Albalate y Samper de Calanda, me hice querer de mis feligreses. Siempre
fui bueno para ellos: a los pudientes respeté, y a los pobres favorecí
cuanto pude. Estalla en esto la guerra, y\ldots{} Nada, que mi voluntad,
lo mismo que mi convencimiento, me llevaron a la causa de don
Carlos\ldots{} Fue un arrebato del corazón, ¡rediez! Me tiraba el campo
de batalla. Yo era gran cazador\ldots{} Me sacaba de quicio la guerra,
que es cazar hombres con hombres\ldots{} Combatí en la partida de
Quílez: yo era el ojo y el caletre de la partida, yo su pie derecho, por
mi conocimiento del país y de las vueltas de montes, las distancias,
alturas, atascos y torrenteras\ldots{} Pues hice bravamente toda la
campaña. Pregúntenle a Ramón Cabrera si cumplí o no cumplí\ldots{} Supe
mandar, supe obedecer, supe dar recompensa y castigo\ldots{} Maté
cristinos y urbanos, copé columnas, desbaraté batallones, y aunque usted
se asuste, ángel, fusilé prisioneros, no uno ni dos\ldots{} No hay que
asustarse\ldots{} Fusilé y aterroricé porque así me lo dictaba la ley de
guerra\ldots{} Tiene el soldado su conciencia muy distinta de la
conciencia del cura\ldots{} Nada tiene que ver una conciencia con
otra\ldots{} Las vidas no suponen nada\ldots{} Por delante de las vidas
ha de ir la Causa\ldots{} y Dios, que es la Causa de las Causas, mira
por lo suyo\ldots»

Esto decía acabando de comerse un pollito, pues era hombre de buen
diente y mejor estómago. Yo tampoco lo hacía mal. Pidió el Arcipreste
vino blanco; acudió la rubia con la botella, y cuando lo escanciaba en
los vasos (que allí no vi funcionar el castizo porrón) oí su voz, que me
sonó a gorjeo delicioso. El catalán hablado por mujer es una de las más
bellas músicas de la boca humana. Así me ha parecido siempre, y más aún
en aquella placentera noche\ldots{} La jamona sirvió después un plato de
pescado, y al recomendármelo el Arcipreste como exquisito manjar, me
dijo que dispensara la cortedad de la cena. ¡Cortedad, y tras el pescado
trajo la rubia un plato de carnaza, y después \emph{ali-oli!} ¿Señor,
qué casa era aquélla?\ldots{} Como yo alabase la substanciosa y
abundante mesa, don Juan Ruiz añadió a su relación histórica este dato
interesante:

«¡Bendito sea Dios que me ha concedido un buen vivir! Sabrá el señor
\emph{Confusio}, que allá por el 41, un pariente mío por parte de madre,
solterón y gran propietario en Belchite, murió\ldots{} Natural fue que
cascara el buen señor, pues ya pasaba de los ochenta\ldots{} Me quería
tanto, y era tan ferviente admirador de mis hazañas en la guerra, que me
dejó por heredero de toda su hacienda, que no era grano de anís. Vea por
qué vivo bien y doy buen trato a los amigos\ldots{} También debe saber
que no soy tacaño ni guardador; no me excedo ni tampoco escatimo, y
cerca de mí no hay pobre que no sea remediado\ldots{} Y en mi casa son
tantas bocas a comer, que a menudo me equivoco en la cuenta de ellas.
Las amas y sobrinas que me sirven, aquí se están hasta que quieren, o
hasta que hallan novio con buen fin que pida casamiento. Yo a ninguna
despido, y la misma regla observo con mis mozos de labranza, criados y
medianeros. Verdad que también les exijo lealtad y buena conducta, eso
sí, y el que no cumpla, ¡rediez!, se ha divertido.»

Me encantaba aquel tío rudo y noblote, gran señor a su modo en la paz,
como había sido esforzado paladín en la guerra. Durante su relación, ni
un momento vi en él al sacerdote. En la punta de la lengua tuve este
concepto: «Dígame, señor Arcipreste, ¿cuántas amas y sobrinas tiene?»
Pero antes de pronunciar la primera palabra, vi la indiscreción de tal
pregunta. Acabamos la cena no sin catar a la postre azucarados bollos,
rosquillas de miel, con buen vino dorado, trasañejo. Salimos al central
aposento, donde está la puerta de la cocina, la escalera que a las
alcobas conduce, la comunicación con despensa, cuadras, patios y
corrales, y allí nos repantigamos en un banco de madera, junto a
ventrudas tinajas. De la cocina no podía yo ver más que el resplandor
vivo de la lumbre, ni oír más que el rumor alegre de los que allí
comían. Muchos eran, a juzgar por la variedad de voces. Parecíame que
había más mujeres que hombres, y más juventud que vejez. En el
desconcertado ruido distinguí voces castellanas entre el silabeo blando
del catalán. Reconociendo en tales voces la innumerabilidad de las
sobrinas del Arcipreste, creí que ellas me contestaban la pregunta que
no osó salir de mis labios.

Encendimos buenos puros. Por las órdenes que dio don Juan a sus criados,
entendí que saldríamos de madrugada, para estar en Ulldecona a las
primeras horas del día. De pronto, el Arcipreste, volviéndose a la
cocina, gritó: «¡Donata!» Y apenas sonado este nombre en la cavidad
anchurosa, apareció una mujer en el hueco iluminado por la roja claridad
del fogón. Salía sin presteza de la cocina, mascando el último bocado.
Acudía con diligencia grave al llamamiento de su señor, como servidora
que sabe no ha de ser reñida por tardanza o pereza. Fue para mí una
visión sorprendente y deslumbradora. Creí ver la expresión sintética de
la hermosura de mujer, tal como yo la soñé, sin verla nunca realizada.
«Donata---le dijo don Juan Ruiz,---ya sabes que nos vamos antes de que
amanezca. ¿Has guardado en las maletas todo lo mío que se ha de llevar?
Anda, hija, ve y dispón todo: no olvides mis pistoleras; no olvides
tampoco tu trajecito de payesa, ni mi sable, ni la caja de puros\ldots»

Tragado lo que mascaba, la hermosa Donata (el nombre ya se había grabado
en mi mente) habló en buen castellano endurecido por acento aragonés.
Dijo que nada quedaba por guardar más que las pistolas, espuelas y otras
cosillas; pero que al momento subiría para recogerlo. «Oye---le dijo el
señor, cuando ya iba la beldad hacia la escalera,---se me olvidaba
mandarte que arregles la cama para este señor en el cuarto de la
esquina\ldots{} Podrá dormir cómodamente cuatro o cinco horas\ldots{}
Oye, no corras tanto: ven acá. El cuarto de este señor lo arreglará
Carmeta\ldots{} Vete tú a los demás quehaceres, y no te descuides.»
Subió Donata, y embobado estuve mirándola hasta que desapareció en lo
alto de la escalera. Don Juan llamó entonces a Carmeta, una de las
jamoncitas que nos recibieron al entrar, y repitió la orden de preparar
mi descanso. Era esta ama bien parecida, conservada en una blanda
madurez otoñal; pero después de ver a Donata, no había mujer tierna ni
madura que hiriese mi atención ni cautivara mi espíritu.

Aturdido por la deslumbradora visión, no pude hacerme cargo de las
diversas órdenes que para la partida dio el cura a las muchas personas
que salieron confusamente de la cocina. Sólo entendí bien esta
disposición: «Con vosotras, en la tartana de Quirico, que saldrá
primero, irán Donata y Carmeta\ldots{} Conmigo y el señor
\emph{Confusio}, vendrán Toneta y Olegaria.» Ésta era la rubia, Toneta
la \emph{Dolorosa}\ldots{} Mucho me incomodó la orden de que Donata no
hiciera el viaje en la tartana donde yo iba. Pareciome ofensa,
desconsideración, un desaire manifiesto, como lo fue asimismo el mandar
que Carmeta y no Donata arreglase mi cuarto. ¡Vaya con el tío aquel,
déspota celoso y bárbaro! Al entrar en el aposento que me destinaron, vi
a Donata que de uno próximo salía con brazados de ropa. Se aproximaba
con los ojos bajos; pero al pasar junto a mí los alzó para mirarme.
¿Estaba yo loco, o tenía razón al pensar que algo muy intenso quiso
decirme con su fugaz mirada? Pasó veloz. El ruidillo de sus pisadas algo
también me decía.

Encerrado en mi alcoba, excitadísimo y sin ganas de acostarme, a pesar
de mi cansancio, vi a la guapa moza en mi mente con más lucidez que en
la realidad habíala visto, y mejor podría describirla por el retrato
mental que en mí llevaba, que por su presencia efectiva. Era más delgada
que gruesa y más alta que baja, estatura y talle contenidos dentro del
arquetipo de la humana belleza. Negros ojos, boca ideal, cabello
abundante, recogido con helénica gracia, melancolía, desconsuelo,
añoranzas, ambición de amor\ldots{} todo esto vi en su rostro, y con tan
ricos elementos lo compuse\ldots{} El cuerpo de aquella divina mujer me
revelaba la suma donosura, la soberana previsión de Naturaleza, la
sabiduría del Criador\ldots{} Belleza tan acabada no habían visto nunca
mis ojos.

Con más fatiga corporal que sueño me tendí vestido, y en el estupor
letárgico que embriagó mis sentidos, algo como borrachera o vaporización
de pensamientos, incurrí en el más extraño desbarajuste de las cosas
reales. No diré que soñé, sino que creí sueño todo lo que me había
pasado desde mis travesuras en la casa de \emph{El Nasiry} hasta la hora
presente; sueño, mi conversación con el renegado, mi salida de África,
mi regreso a Madrid, mis careos y tratos con Beramendi; sueño, la
conspiración absolutista y mi viaje para observarla; sueño, que yo
estuviera donde estaba. Lo verdadero y real era que aún permanecía en
Tánger, y que reposaba en el poyo de mi camarín sobre tapices morunos. Y
allí recreaba mi mente con la imagen de Donata, que no era Donata sino
\emph{Erhimo}, la esclava de ideal hermosura, sólo comparable a los
ángeles de los cielos católicos y mahometanos. En esclavitud vivía
Donata, digo, \emph{Erhimo}, y a mí me enviaba Dios para libertarla de
la garra de \emph{El Nasiry}, digo, del fiero sultán \emph{Mosén
Hondón}. Sonábame este nombre como el más bárbaro que pudiera inventar
la rudeza oriental o marroquí. Era el tirano celoso y feroz que guardaba
dentro de cerrados muros a la odalisca, y ésta quería libertad, y por
Dios que yo había de dársela.

Salté del lecho, llamado por suaves golpecitos que dieron en la puerta.
Era hora de partir. Yo no vi la mano cuyos nudillos hicieron la tocata
en la madera. Pero mi adivinación prodigiosa me permitió afirmar que
había sido Donata la que con el lenguaje de los golpecitos me decía:
«Levántate, salvador mío, que ya nos vamos a donde podrás, con tu
agudeza y mis advertimientos, sacarme de este serrallo y hacerme tuya.»
Cuando bajé, ya estaba la Donata ideal agazapadita en la tartana que
había de conducirla con otras mujeres. Entre ellas vi a la que parecía
\emph{Dolorosa}, despintada y amarillenta pidiendo barniz. Fue una
visión fugaz, a la débil luz de faroles, pues aún era noche
obscura\ldots{} Partió la tartana, y en ella no pude ver bien más que
los ojos de Donata, que ya se entendían maravillosamente con los míos.
Don Juan Ruiz me ofreció café: lo tomamos juntos, acompañados de
Olegaria, la rubia. En la mesa vi las tazas con poso de café, donde lo
habían tomado las amas y sobrinas que iban delante. Reconocí, ¡oh
inspiración!, la pieza de loza en que había puesto sus rojos labios mi
odalisca\ldots{} ¡Oh!, la taza y sus sedimentos negros también me decían
algo, que traduje del lenguaje porcelanesco al lenguaje humano. «Yo voy
delante de ti\ldots{} Desde tu tartana mira el polvo que levanta la mía,
y me verás en él\ldots{} Yo miraré el polvo que levanta la tuya, y te
veré\ldots{} Cuando llegue a Ulldecona me ocuparé un rato en las cosas
de la casa; luego iré a la iglesia\ldots{} Oigo misa todos los
días\ldots{} Ve tú también a oírla, y en la iglesia nos veremos\ldots{}
Ningún sitio mejor que la iglesia para que las esclavas y sus
libertadores se pongan de acuerdo.»

Salimos. Yo miraba el camino delantero; pero no veía el polvo de la
primera tartana, sino el de otras que marchaban en contraria dirección.
Las luces del alba me permitieron observar que el país no era nada
bonito\ldots{} Me parece que vadeamos un río; no estoy de ello bien
seguro. Mi espíritu atendía más a sus interiores paisajes y horizontes
que a los de fuera. Don Juan Ruiz me habló de guerra más que de
política. El día anterior se había entretenido con unos cuantos
escopeteros de confianza en dar gusto a su afición favorita, que era la
\emph{caza de hombres con hombres}. No pudiendo hacer nada de
fundamento, porque la Causa en aquella ocasión estaba perdida (tan
disparatado había sido el movimiento), intentaron gastar sus cartuchos
en la Guardia civil y tropas que habían de pasar de San Mateo a
Ulldecona. Pero les salió mal la cuenta: la fuerza del Gobierno se fue
por otro lado, y los cazadores facciosos no cobraron más que \emph{un
ratón}. Yo sólo, el pobre \emph{Confusio}, inofensivo, había caído en la
celada. Añadió don Juan Ruiz que se iba desconsolado: hubiérale sabido a
gloria copar a la Guardia civil en el paso angosto de Rosell de la
Cenia, próximo a su masada. Pero la Providencia dispuso las cosas de
otro modo. A su casa y parroquia se volvía el hombre tan tranquilo: los
escopeteros, cernícalos de vuelo rápido, habían volado ya, cada cual a
su nido en los montes de Godall y Muntciá.

Destartalada y fea me pareció la villa de Ulldecona, donde, según iba
entendiendo, reinaba como sátrapa o cacicón mi amigo el Arcipreste. Ya
era día cuando llegamos a la soberbia vivienda parroquial: junto a la
puerta vi la primera tartana, que había llegado con veinte minutos de
ventaja. Miré sus ruedas y atalajes blanqueados del polvo, y en todo
ello leí el pensamiento de Donata, que me decía: «He llegado
bien\ldots{} Búscame luego en la iglesia.» Antes que mis ojos, que todo
lo miraban, dieran con el templo, don Juan Ruiz me señaló un armatoste
arquitectónico de diferentes estilos y pegotes que alzaba su
insignificancia ostentosa no lejos de la casa.

Entramos: la casa es grandona, laberíntica, resultante de varios
edificios comunicados interiormente, con distintas alturas de techo,
diferencias de nivel en los pisos. No se va de una parte a otra en
aquella jaula de cal y canto sin dar vueltas y quiebros de sala en sala,
y bajar o subir escalones. Plano y brújula necesita el huésped de esta
mansión misteriosa y dramática. Pasada la primera impresión de
aturdimiento al verme llevado por aquel interior tortuoso, la casa fue
muy de mi gusto. En ella vi escenario romántico; supuse escondrijos de
citas amorosas, dorados camarines invisibles, recogimientos de
harén\ldots{} Por aquellos desiguales recintos vi que iban y venían
mujeres muchas, las de la masada y otras. Vi ancianas, niños de ambos
sexos. Era un mundo, un microcosmos la casa de \emph{Don Juanondón},
Arcipreste, Patriarca y Califa.

Invitome mi huésped a tomar chocolate; él no lo tomó, porque tenía que
decir misa. No quise recordarle que había bebido café en la masada; en
lugar de esto, le pregunté con mucho interés que a qué hora diría la
misa, pues yo deseaba oírla. Respondiome que antes de una hora saldría
al altar\ldots{} Nos hallábamos en una pieza como de tránsito, que daba
acceso a diferentes salas y a dos corredores, y desde allí vi a las
chicas que pasaban y repasaban, como solícitas hormigas, ocupadas en el
trajín casero. Vi a la \emph{Dolorosa}, a la rubia, a otras menos
bonitas; pero a Donata no vi. Estaba yo elogiando la diligencia y
laboriosidad de las incontables sobrinas del señor Hondón, cuando pasó
por allí la jamoncita Carmeta con un cubo de agua y estropajos para
lavar el suelo de baldosines rojos. Don Juan Ruiz le dijo con dureza:
«¡Buena tenéis la casa! Hoy\ldots{} bien puedes decirlo a todas\ldots{}
no me ponéis los pies en la calle, haraganas. Y como no es día de
precepto, no tenéis por qué ir a misa. La Toneta y la Donata irán si
quieren; las demás a la obligación, que es primero que nada\ldots» Sin
chistar oyó Carmeta el réspice: se fue a una pieza próxima, donde había
suelos que lavar. Don Juan Ruiz me dijo: «Tengo que estar siempre encima
de estas mozas para combatir la ociosidad\ldots{} Son buenas,
sencillotas; pero no puedo descuidarme. En cuanto se las deja hacer su
gusto, se pasan el día de charloteo\ldots{} Algunas tengo que se
inclinan a la beatería; pero a éstas hay que dejarlas en su gusto de lo
espiritual, y no quitarles de la cabeza las devociones extremadas,
porque con el pío pío del rezar continuo llegan a ser unos pobres
ángeles\ldots{} y de los ángeles hace uno lo que quiere.»

\hypertarget{xix}{%
\chapter{XIX}\label{xix}}

No eché en saco roto la lección del Arcipreste, pensada y dicha en
conformidad con su sistema de vida, y aplicada por mí a ideas y planes
de orden muy distinto. Él quería decir que las chicas embebecidas en
vanas devociones son fáciles al dominio de quien posee la clave de lo
espiritual, y que por tal camino sabía él traerlas al rigor de los
deberes domésticos y a la corrección externa y visible\ldots{} Atento a
mis propósitos, en cuanto mi huésped me dejó solo (por haberse ido con
Olegaria a la inspección y revista de su bien poblado gallinero), me
metí en la iglesia, que era, conforme a los gustos de la moderna piedad,
sombría, casi lóbrega, invitando a somnolencias dulces y a borracheritas
de la mente. Vi trozos del esqueleto de una robusta arquitectura,
mutilada, recompuesta, vestida de mil requilorios ornamentales y de
bárbaros colorines; vi santos en paños menores y profetas barbados, de
cara fosca; vi un altar mayor, cuya sencillez elegante se perdía tras un
matalotaje de cortinas, arañas, candelabros y pabellones; vi en la
cabecera de la nave lateral un altar de la Virgen, que era la más
descabellada y furiosa expresión del churriguerismo, obra, al parecer,
de pastelería, compuesta de delgados y retorcidos bizcochos, de
hojaldres quebradizos, de dorados y relucientes caramelos. La santa
imagen apenas se distinguía entre la chillona profusión de metales,
tisúes y flores de trapo, rodeada de ángeles de pastaflora y ex-votos de
mazapán que la comprimían y ahogaban.

Bajé después hacia el pie de la misma nave, donde vi, en soledad
tétrica, olvidado de la devoción, un Cristo de espantosa anatomía, de
espeluznante horror traumático, piernas y brazos en carne viva, con
cárdenos bultos y cuajarones de sangre, que resultaban de una realidad
viva por la reciente mano de barniz. Su cabellera natural, despeinada y
polvorienta, le caía sobre el pecho. No tenía velas encendidas ni
apagadas en su altar desnudo, baldío\ldots{} Cuando pasé hacia la
capilla bautismal, entró Donata, ¡ay, qué hermosa!, con su velito negro,
en las albas manos el Ordinario de la misa. Acudí a darle agua bendita,
y cuando sus dedos de los míos la recibieron, me miró sin sorpresa. Sin
duda me esperaba. No me equivoqué al pensar que su mirada placentera me
decía esto: «yo rezaré a la Virgen; haz tú lo mismo, y con el rezo mudo
y sin mirarnos, nos entenderemos hasta que llegue el momento en que
podamos hablar.» Avanzó ella hasta la capilla de la Virgen. Yo me quedé
en la nave central, debajo del púlpito, sitio reservadito desde el cual,
protegido de la penumbra, podía ver a Donata y cebarme en la
contemplación de su interesante figura. La vi de rodillas; al levantarse
para tomar asiento en un banco, observé en su movimiento perezoso la
intención de buscar un propicio instante para mirarme. Y una vez
sentada, aprovechaba ella todo ruido de gente que entraba o salía, para
mover su cabeza y producir el divino cruzamiento de su mirar con el mío.
Mientras permaneció sentada, no cesaba el flecheo; jugamos a la pelota
con nuestras almas mandándolas de un lado para otro.

Salió el coadjutor a decir misa. Donata la oyó de rodillas, y en todo el
oficio nuestra comunicación fue puramente espiritual y magnética. Sus
ojos mantuvieron en el carcaj del disimulo todas sus flechas. Pasada la
misa, ya sacamos alguna, y tiramos con gran tensión de arco. Poco duró
este grato ejercicio, porque salió don Juan Ruiz a decir su misa en el
propio altar de la Virgen. Me pareció prudente retirarme de mi gazapera
bajo el púlpito\ldots{} Desde mayor distancia, resguardado por un grupo
de hombres, vi y admiré al Arcipreste revestido con espléndida ropa. Era
rito encarnado, y estaba el hombre guapísimo, interesante, casi
majestuoso. Celebraba de prisa, mas sin quitar al oficio su poesía y
solemnidad. Al volverse al pueblo, su mirada intensa parecía recoger en
conjunto la voluntad de todo el rebaño que delante tenía. Y véase un
caso que no vacilo en llamar aberración de mi pensamiento. Por la
mirada, en el momento de decir \emph{Dominus vobiscum}, por las líneas
de su rostro más caballeresco que místico, don Juan Hondón se me pareció
a \emph{El Nasiry}. Sin fijarme en la diferencia de ropaje, calidad y
estado, ni en que el uno tiene barbas y el otro no, encontraba yo gran
semejanza entre los dos caballeros renegados. ¿Por ventura la semejanza
moral no era aún más efectiva y patente?

Terminada la misa, y cuando salía la gente, vi que Donata se metió en la
sacristía de la capilla. Con ella entró también Toneta, de mustia cara,
parecida a una Dolorosa retirada del culto. Comprendí que las dos eran
camareras de la Virgen, y que la vestían y desnudaban de sus bordadas
ropas, y le adornaban el pastelero altar. Tentaciones tuve de colarme
tras ellas; pero las refrené pensando que de nada me valdría mi
entrometimiento, pues no había de encontrar a Donata sola. Sospechando
que el camarín de Nuestra Señora tendría comunicación con la rectoral
por patios profundos interiores, y que era inútil esperar más, salí
despacio de la iglesia, y me entretuve hablando con unas viejas que en
la puerta pedían limosna. Les di cuartos, y sin entender su lengua más
que a medias, departí con ellas de la capacidad de la parroquia, y de la
virtud y llaneza de las sobrinitas del señor Arcipreste. A este
propósito, dijeron algo que no llegó a mi conocimiento por no poseer
bien la lengua catalana. Yo les hice repetir sus dichos para
traducirlos; ellas los repetían y ampliaban con el feo sonreír de sus
desdentadas bocas, que para expresar la malicia tenían que imitar al
buzón del correo; y estando en esto, oí la voz del Arcipreste y las dos
muchachas, que salían de la iglesia. Corté mi conversación bilingüe con
las viejas, y estreché la poderosa mano de don Juan Ruiz, felicitándole
por el arte exquisito con que en su misa hermanaba la brevedad con la
edificación.

Llamado al pueblo el Cura por negocios graves, no podía entretenerse. En
la misma puerta de la iglesia se despidió de mí, y mientras él se perdía
en una calle estrecha, las muchachas y yo seguimos hacia la casa. La
suerte me favorecía, porque habiendo ya charloteado con la
\emph{Dolorosa} cuando nos sirvió el chocolate, fácil me fue entrar en
conversación, y lo hice con el tópico de rúbrica, que era la hermosura
de la Virgen y el lindísimo adorno de su altar. Toneta me habló con
desahogo; Donata, cohibida y medrosa, no echaba de su linda boca más que
los mugiditos de la timidez: «Sí\ldots{} naturalmente\ldots{} eso
es\ldots{} ¡Oh!, no\ldots{} ¡Oh!, sí\ldots» Entramos. Yo me sentí con
ánimos para obtener de la ocasión las mayores ventajas, siempre que no
sobreviniesen entorpecimientos invencibles\ldots{} Cuando avanzamos por
las primeras salas de la mansión laberíntica sin encontrar a nadie,
Toneta se adelantó rápidamente; escabullose por un pasillo con recodo, y
solos nos quedamos Donata y yo en una pieza, que era el obligado paso
para mi habitación\ldots{} ¿Fue la escapada de la \emph{Dolorosa} un
quiebro convenido entre las dos para dejarme solo con Donata? Si no fue
ardid preparado, lo pareció, y me apresuré a sacar de la instantánea
soledad todo el partido que me ofrecía\ldots{} En mí sentí la
inspiración, la sublime audacia de un caudillo que en la violencia de la
primera embestida ve la más segura probabilidad de victoria.

Creo que no pasaron más de dos segundos entre el verme solo ante Donata
y el arrancarme a los increíbles atrevimientos de palabra que voy a
referir. En un monólogo brevísimo, mental relámpago, me dije: «Ésta es
la mía\ldots{} Inspíreme Dios\ldots{} y deme el logro feliz de esta
grande aventura.» Donata se dirigió con paso lento a una puerta de
cuarterones que no sé a dónde conducía\ldots{} Yo corrí hacia ella
diciéndole: «No tenga prisa, Donata, y espérese un poquito, que tengo
que hablar con usted.» Como estatua quedó ella, la mano en la
puerta\ldots{} y yo seguí: «En la calle dije que es bonita la
Virgen\ldots{} Más bonita es usted, Donata. Ni en la tierra ni en el
cielo hay mujer que se iguale a usted en hermosura\ldots» La exageración
de mi arrebato le facilitó la respuesta, que había de ser de
incredulidad y burla. Su condición de señorita inocente, u obligada a
simular inocencia, no podía inspirarle más que esta salida: «¡Ay qué
pillísimo!\ldots{} ¡Ay qué desvergonzado\ldots{} ¡Y también blasfemo!»

---Perdóneme usted\ldots{} No sé lo que digo\ldots{} El amor que prendió
en mí desde el instante en que mis ojos vieron a Donata es hoguera
inextinguible\ldots{} Mi razón se turba, mi conciencia se
obscurece\ldots{} Ni me acuerdo de la religión, ni respeto las cosas
santas. Todo se borra en mi mente\ldots{} No veo más que a Donata, que
es el cielo, la gloria, la salvación de mi alma.

---¡Por Dios\ldots{} Jesús!\ldots{} ¿Está loco?---dijo ella, sin salir
de las muletillas que el decoro impone a una muchacha honesta.

---La salvación de mi alma he dicho, y no me vuelvo atrás\ldots{} Sin
usted no quiero salvarme, ni vivir siquiera\ldots{} Al infierno entrego
mi corazón, abrasado por los ojos de una mujer. Donata, sea usted
piadosa\ldots{} impida mi condenación eterna\ldots{}

---¡Virgen Santísima! ¡Ay qué locura de hombre!\ldots{} Modérese\ldots{}
¡Cómo había yo de creer\ldots! Entre en razón\ldots{}

---De usted depende que yo vuelva a la razón. Dígame que sí, dígame que
puedo esperar\ldots{} que algún día podrá usted quererme\ldots{} que sí,
Donata, que sí\ldots{} Pronuncie usted el sí, dos letras, que de la boca
se salen solas a poquito que su voluntad las empuje.

---¿Pero cómo he de decirle que \emph{sí}? ¡Oh, eso no puede
ser!\ldots{} ¡Que \emph{sí}!\ldots{} Usted no se hace cargo\ldots{}

Dijo esto poniéndose muy seria. Su palidez y gravedad la embellecían
más. Yo eché el resto con estas ardientes expresiones: «Donata, no me
diga usted que \emph{no}\ldots{} dígame siquiera que lo pensará, que
verá\ldots{} Pero un no redondo no me diga, porque ese \emph{no} sería
mi muerte.»

---Bueno, bueno: no se apure\ldots{} Para que se le vaya quitando la
furia, no diré el \emph{no}\ldots{} Vamos, debo decirlo; pero lo callo
por ahora\ldots{} Pero el \emph{sí} tampoco se lo digo\ldots{} ¡No
faltaría más! Usted mismo, si yo dijera el \emph{sí}, no pensaría de mí
nada bueno\ldots{}

Del corredor tortuoso vino un ruidillo no sé de qué, de toses, de pasos,
quizás rumor de las puertas de casa vieja, que suenan como enigmáticas
palabras de duendes. Donata desapareció como si se filtrara por la
pared, y yo me quedé solo en la destartalada estancia\ldots{} Mis ojos
se fijaron, sin darse cuenta de lo que veían, en un cuadrángano vetusto,
colgado en la pared. Mirando después con gran atención, he visto en él
informes bultos, que lo mismo pueden ser frailes que sacas de carbón.
Todo es allí negro y fúnebre\ldots{} ¡Atrás, expresiones de muerte! Dad
paso a la vida.

A mi cuarto me recogí, y en verdad que no estaba yo descontento del
ímpetu temerario con que inicié mi aventura. Herida vivamente en su
voluntad y en su corazón había quedado la bella Donata, y yo con más
ardor prendado de ella. Ya me parecía que la conquista de tan linda
mujer era cosa segura, y no pensaba más que en las paralelas que había
de empezar a poner aquel mismo día para llegar a la posesión de ella y
hacerla mía y llevármela, que éste había de ser el airoso remate de tal
empresa. Lo que no pude hacer en la casa de \emph{El Nasiry}, quizás por
las marrulleras artes del guasón renegado, lo haría en la de don Juan
Ruiz, cuya semejanza con el español africanizado cada día se
representaba en mi mente con más vigor. Los harenes europeos no están
tan cerrados al soborno y a la captación como los africanos, y sus
odaliscas o barraganas no se hallan tan cohibidas para pedir al mundo
externo su salvación, siempre que haya valientes caballeros que en esta
honrada empresa pongan toda la energía de sus bien templadas almas.

La primera paralela puse aquel mismo día, escribiéndole una carta con
todo el fuego de amor que mi ambicioso anhelo me dictaba. Cada concepto
era una flecha capaz de atravesar corazones de piedra. Y firme en mi
idea de que la presteza y resolución rectilínea me conducirían a un
rápido triunfo, desde aquella primera carta le propuse la evasión, el
rapto, el cambiar su vida prisionera por la libertad y el amor, huir
juntos en busca de la paz y la felicidad a regiones distantes. Bien
sabía yo que a la primera carta contestaría negativamente o con
alambicados melindres; pero a la segunda y tercera seguramente se
desplomaría su voluntad, y allí estaban mis brazos abiertos para
recogerla y escapar con ella. Doblé y cerré la epístola en la forma más
breve, y ya no me faltaba más que una coyuntura propicia para
entregársela, la cual al cuidado de Dios estaba, y no tardó en
presentarse.

Comimos aquel día solos don Juan y yo, servidos por una jamona pasadita,
nombrada Monsa, y por la que yo llamo la \emph{Dolorosa}. La comida fue
opípara. Como yo expresase a mi huésped mi sorpresa de encontrar trato
tan exquisito y mesa tan señoril en un pueblo casi rústico, y en región
como aquélla, donde parece muy lenta y premiosa la evolución de las
costumbres, me dijo que él había recibido la enseñanza del buen vivir, y
de las comodidades y limpieza de casa, mesa y demás, de un prócer que
fue muy su amigo en la guerra pasada, a quien llamaban don Beltrán de
Urdaneta, dechado y tipo de caballeros aragoneses, el cual a mí quizás
no me sería desconocido, porque su nombre y hechos andan en papeles, y
aun en un libro donde se refieren las gestas de Cabrera en el
Maestrazgo. Aquel noble señor, tan entendido en cosas del mundo y de la
civilización extranjera, dio a don Juan lecciones del arte de comer y de
cuanto atañe a tenimiento de casa y al buen porte y modales de persona
fina. No fueron perdidas por \emph{mosén Hondón} las enseñanzas del
caballero, y cuando fue rico puso en ejecución toda la ciencia, que, una
vez probada, le pareció admirable para ir pasando los días en este valle
de lágrimas. «Antes de que me cogiera de su cuenta el gran
maestro---añadió don Juan Ruiz,---yo no sabía salir de la rústica
ignorancia y sencillez grosera de los pueblos en que me crié. Para mí no
había más mundo que la cocina con su enorme campana, el ollón sobre el
fuego, alimentado con \emph{fajuelos}, el candil de aceite, las
\emph{cadieras}, la bazofia que comíamos, y luego el dormir en camas
altísimas con apretados colchones\ldots{} En fin, tras aquello vino
esto, gracias a don Beltrán, a mi herencia y al natural mío, que desde
niño con secretas voces me tiraba a lo rumboso y elegante. No me pesa de
ser como soy, que así puedo obsequiar dignamente a los amigos, y
sorprendo a los forasteros, como usted, dándoles en este villorrio las
comodidades y el trato y trote de las poblaciones ricas.»

Pareciome excelente lo que el cura me decía, y queriendo yo también
darme alguna importancia, ya que alardear no puedo de buen vivir, díjele
que mi lujo era el saber y mi elegancia el estudio. Desde mi tierna
infancia no había para mí mayor goce que el manejo y lectura de libros.
Alabó don Juan Ruiz mis gustos, que nada encaja tan bien en la conducta
señoril como dar aliento y protección a la gente estudiosa. La
benevolencia del clérigo, excitando mi amor propio, fue causa de que se
me desbordara la fácil erudición que poseo. Sin que viniera muy a
cuento, le solté a mi amigo un chaparrón de Teología, de Tomismo, y al
fin todo lo que sé del Concilio de Trento, por haberlo leído en el
camino\ldots{} Pronto eché de ver que el Arcipreste se aburría con mi
ciencia; fui recogiendo mi verbosidad, y acabé rogándole que me
permitiera entretener mis ocios en su biblioteca. Soltó la risa Hondón,
y con graciosa sinceridad me dijo: «Criatura, yo no tengo biblioteca, ni
me hace falta para nada. Jamás abro un libro, porque sé que en él he de
encontrar lo que ya sé, o sabidurías enrevesadas que, por razón de mi
edad, ya no puedo aprender. Mi biblioteca, señor \emph{Confusio}, es la
Humanidad, y mis libros las flaquezas, las pasiones, las envidias, las
luchas humanas por el pan o por el palo\ldots{} ¿Le parece a usted que
esto no es estudiar, y afilar uno las ideas, y quemarse las pestañas?»

\hypertarget{xx}{%
\chapter{XX}\label{xx}}

Mi respuesta, puramente mental, a los métodos científicos del Cura, fue
así: «Conformes, amigo Ruiz. Yo también revuelvo esa biblioteca y
compulso esos libros. Pues ahora vas a ver cómo de tus estantes te quito
el libro más substancioso, más inspirado y profundo, el estampado con
más lindos caracteres, porque ese libro me gusta a mí, y quiero leérmelo
y desentrañar su ciencia honda y su intensísima belleza.» En efecto: don
Juan Ruiz se fue a sus quehaceres en la ciudad, y yo, solo en la casa,
hice de ella un estudio topográfico, bajando luego a las huertas
amenísimas y al gallinero populoso. Hallándome en la admiración de éste,
tuve la dicha de que Donata me diera la contestación a mi primera carta.
Entró ella a recoger huevos, y al salir, de la misma falda en que los
llevaba sacó el papel, y ruborosa me lo dio, suplicándome que no le
escribiera más. Yo le dije que esto no podía ser, y que al día siguiente
se dispusiera a recibir la segunda en la iglesia. En sus ojos y labios
puso los más graciosos remilgos para decirme que no volviese a
escribirle. Pero harto comprendía yo que los remilgos significaban:
«Escríbeme más, y mañana recogeré tu carta en el momento de tomar el
agua bendita.»

Deliciosa era la epístola, que con su sintaxis pueril y su anarquía
ortográfica me representaba la mujer tal como mi amante ambición la
requería. Cierto que no se omitían en ella los inevitables aspavientos
pudorosos, ni la monadita de espantarse de mi atrevimiento; pero luego
venía la confesión de que era muy desgraciada, y el temor de que sus
desdichas no pudieran tener remedio. Entre col y col, decíame que yo no
le era \emph{hindiferente}, y que me agradecía mucho la \emph{idalguía}
de querer libertarla; pero que no podía ser, y vuelta con que no podía
ser\ldots{} En fin, leída la carta en la soledad de mi cuarto, me
apresuré a redactar la segunda, esmerándome en hacerla más incendiaria
que la primera, y más arrebatada en la elocuencia de amor. La semejanza
de Donata con la imagen que me forjé de la bella \emph{Erhimo} era cada
día más patente. Yo vestía mentalmente con el traje oriental a la
sobrina, o lo que fuera, del señor Arcipreste, y veía realizado en su
rostro y talle la suprema hermosura de mujer, sintetizando los
ejemplares más perfectos\ldots{} Sus ojos son todo el cielo, su boca
toda la vida existente entre cielo y tierra, y de su seno para abajo los
profundos abismos de creación, donde nacen los ángeles. Yo estaba loco;
yo amaba tiernamente a Donata, con ilusión de poesía, y con el santo
anhelo de fundir ésta en la prosa de la vida común.

Al siguiente día, realizado el plan presupuesto, entregada la carta en
la obscuridad junto a la pila, oída la misa, salimos todos con don Juan;
pero éste, en vez de dejarme ir a la casa con Donata y la otra, que no
era Toneta, sino Olegaria, me llevó consigo por el pueblo. Entendí que
iba, como el día anterior, a quehaceres importantes, enfadosos\ldots{}
Sorteando baches y montones de basura, recorrimos angostas calles sin
empedrar, que me recordaban las de Tánger y Tetuán. Por donde quiera que
iba don Juan Ruiz, era saludado con respeto: hombres y mujeres le abrían
paso, y le besaban la mano los chiquillos, homenaje de que yo
participaba alguna vez, por mis trazas de curita vestido de seglar. Con
diversas personas que encontramos cambió el Arcipreste animadas
observaciones acerca de la cosa pública. A dos payeses arrogantes y de
buena ropa les dijo: «Parece que a Ortega le condenan a muerte,» y los
otros no mostraron asombro ni lástima. Luego, llegados mi amigo y yo a
una plazoleta solitaria, nos detuvimos un instante, porque así lo quería
el interés que tomó de súbito nuestra conversación.

«Bien merecido le está---declaró mi amigo.---¿Qué menos pueden hacerle a
ese tarambana de Ortega que pegarle cuatros tiros? Figúrese usted que se
plantó aquí con los batallones de la guarnición que tenía en Palma de
Mallorca; los embarcó como quien embarca sacos de almendras, sin
decirles: «vamos a esto, vamos a lo otro.» ¿Qué había de suceder? Llegan
a San Carlos a media noche. ¿Él qué se creía? Que le esperaban aquí
tropas sublevadas; que toda Cataluña estaba en armas, y que Madrid había
dado el grito\ldots{} Ni Madrid dio ningún grito, ni aquí estábamos en
pie de guerra, porque no se preparan esas cosas como preparamos una
merienda, ¡rediez!\ldots{} El que dio el grito fue Ortega al saber que
O'Donnell ha firmado la paz. Gritó \emph{sálvese el que pueda}, mientras
las tropas que trajo gritaban \emph{¡Viva Isabel II!} En fin, ello fue,
señor \emph{Confusio}, el mayor desastre y la chiquillada más necia que
se ha visto desde que hay facciones en el mundo\ldots{} Huyó don Jaime
Ortega\ldots{} ¡qué había de hacer el hombre!\ldots{} Hubiera sido
Cabrera el desembarcante en la Rápita, y yo le juro a usted que, aun
viniendo solo, no habría tenido que escapar como un colegial travieso.
Pero ese botarate, ese Orteguita, que se deja engañar por los de la
Romana, tal vez por algún comisionado de Francia, quién sabe si por
algún catacaldos venido de Madrid, y luego engaña él a su vez tontamente
a Montemolín y lo hace venir de Marsella, ¿cómo pudo creer que los
leales de acá le íbamos a recibir armados y organizados?\ldots{} ¿Para
qué, rediez? ¿Para que nos pudriéramos la sangre en esa Cataluña y en
ese Aragón, y echáramos el bofe sin resultado alguno?\ldots{} No puede
ser\ldots{} con estos locos no puede ser\ldots{} La Causa seguirá
dormida\ldots{} y dormiremos hasta que suene la hora. La trompeta que ha
de tocar la hora está enfundada.»

---Bien---le dije:---muy santo y muy bueno que estén enfundadas la
trompeta y las armas; pero la humanidad, señor Arcipreste, no debe
estarlo. No me negará usted que por la Causa condenan a muerte al
desdichado Ortega. ¿Por qué, cuando el hombre salió azorado y huido, no
le dieron ustedes escondite para que pudiera salvar la pelleja?

Bien porque se cansara de la paradita, bien porque había de pensarlo un
poco antes de darme la respuesta, el Arcipreste me cogió del brazo, y
silencioso me llevó por una calle torcida, de vulgares y pobres casas,
hasta llegar a una de aspecto vetusto, con una puerta que había sido
monumental y conservaba ornamentos heráldicos ya carcomidos del tiempo.
Allí se detuvo, y bajando la voz, aunque nadie había en la calle que
oírnos pudiera, me dijo: «No tienen todos los locos y majaderos derecho
a que se les ampare y se les libre de la muerte. ¿De dónde ha salido ese
Ortega? ¿Dónde está su abolengo carlista? Nosotros no podíamos atender a
su escondite, porque teníamos que mirar por otros majaderos de más
cuenta, el Rey y su hermano, que tan sin tino se metieron en esta
malandanza. Bastante hemos hecho, ¡rediez!, con salvarlos del bochorno
de ser cogidos y avergonzados en público por esta canalla del Gobierno.
Y salvos quedaron gracias a mí y a otras buenas almas que miran por la
Causa. ¿Para qué estábamos en Rosell de la Cenia más que para cortarle
el paso a la Guardia Cívica que venía, según supimos, al olor de las
cabezas reales? Mientras allí estaba yo con mis aguiluchos de confianza,
otros condujeron al Rey y Príncipe a Vinaroz, desde el arrabal de
Ventalles, donde los teníamos escondidos. Y en Vinaroz se había
preparado un falucho; del falucho pasaron a un vapor, y allá se fueron
mares adelante. Ya ve el amigo \emph{Confusio} que hemos apurado nuestra
humanidad para sacar del atascadero al Soberano. A ese Ortega que lo
salve su madre, si la tiene, o Napoleón de Francia, o sálvelo la Isabel,
que es de corazón blando, según dicen\ldots{} Con que, amigo y tocayo,
yo en esta casa me quedo, que tengo que visitar a la vieja más cócora de
esta villa, una \emph{Trotaconventos} y \emph{Tragahostias}, que me
tiene frita la sangre con un pleito\ldots{} un enredo de
intereses\ldots{} Ya le contaré. Es tía de aquella Donata, de aquella
pobre huérfana que tengo en casa\ldots{} Abur. Váyase usted a dar la
vuelta grande del pueblo. ¿Ve usted ese callejón y al fondo unos
árboles? Sale usted por aquí, y se encuentra en el convento de Santo
Domingo\ldots{} Ya no hay frailes, ni falta que nos hacen. Ahí verá
usted una olmeda. Es sitio ameno. Después, tirando a la izquierda, por
una calle con porches, vuelve a entrar en el pueblo, y derecho, derecho,
sale a la parroquia, y a casa\ldots{} Ea\ldots{} no se vaya a perder.»

Metiose por el portal, y yo seguí el camino que me había indicado. Vi el
convento, la olmeda: todo me pareció tristísimo y de vulgaridad
villanesca, bien porque así fuese, bien porque, llena mi alma de la
hermosura de Donata y del ansia de su conquista, no había forma ninguna
de la Naturaleza que pudiera serme grata. No sé por dónde anduve\ldots{}
Mis pies me llevaban a donde querían, y al fin, por ejidos polvorosos,
por calles costaneras, lleváronme a la parroquia sin que mi voluntad les
ordenase aquel camino. A la vuelta de un recodo, vino sobre mi vista la
torre de la iglesia, como si diera algunos pasos a mi encuentro\ldots{}
Vi la casa, cuyo negro frontis pareció sonreírme\ldots{} ¡ay!, y en
efecto, me sonrió, porque vi a Donata en una de las ventanas altas
sacudiendo una colcha\ldots{} Miré a la colcha y a Donata sin decir
nada; después seguí hacia la puerta, afectando la mayor indiferencia,
porque había gente en la plaza: el coadjutor, una mujer y un
burro\ldots{} mejor será decir un aguador que lo llevaba.

En mi cuarto aceché el paso de Donata por las estancias próximas; mas no
la vi. Todas las hembras jóvenes y maduras de la populosa familia del
Arcipreste pasaron, menos la que era luz de mi vida. Sin duda se ocupaba
en contestar a mi carta, faena para ella lenta y difícil por la torpeza
de su escritura. Llegada la hora de comer, salí antes que me llamasen.
El señor Arcipreste no había vuelto aún, desusado y rarísimo caso que
sólo en ocasiones extraordinarias ocurría. Advertí en las amas y
sobrinas un ceño de inquietud; iban de un lado para otro interrogándose
con fugaces monosílabos; enfilaban desde una ventana la calle frontera y
larga por donde el reverendo había de venir. Pasaba tiempo, y cada
minuto aumentaba la incertidumbre y ansiedad del rebaño mujeril\ldots{}
Oí cuchicheos en los corredores, como si celebraran consejo para adoptar
alguna resolución\ldots{} Por fin, Olegaria, que estaba de centinela en
la ventana, volvió gozosa con el feliz anuncio de que ya venía\ldots{}
¡Oh!, ya venía, ya entraba en la casa; ya se sentía el resoplido del
león en el portal, en la escalera.

¡Por las once mil Vírgenes, cómo venía el buen señor! Daba miedo
verle\ldots{} Despavoridas huyeron hacia la cocina las chicas, las
grandes y medianas, y yo temblé viendo la cara que traía mi don Juan, y
observando los gritos y patadas que fueron su entrada y saludo en la
patriarcal vivienda. Algo debió de pasarle aquella mañana, que le
sacudió los nervios, le encendió la sangre, y desató la mal enfrenada
bestia de su genio mandón y arbitrario. Pidió la comida con fuertes
voces, tiró el gorro, se quitó el balandrán como un estorbo para sus
manotazos, y cogiéndome cual si quisiera pegarme, me llevó al comedor y
a la mesa, diciendo: «¿Qué es esto, rediez? ¿No comemos hoy?\ldots» El
hombre se salía, por decirlo así, de su pellejo. Creyérase que en su
alma llevaba una gran tempestad, más terrible por ser de esas
agitaciones del corazón y de la mente que a nadie pueden comunicarse.
Sus ojos despedían lumbre, limpiábase el sudor del cogote, rechinaba los
dientes apretando las mandíbulas, dejaba caer sobre la mesa la palma de
su mano con tanta fuerza y pesadez, que temblaban de susto los pobres
platos, vasos y copas. «Serénese, don Juan---le dije yo, no menos
trémulo que la loza.---Coma tranquilo y no se altere por tan poco. ¿Qué
es ello?\ldots{} El pleito, la vieja cócora\ldots»

Y él, después de quemarse con la primera cucharada de sopa, gritaba:
«¡Por vida de los cojilondrios, esta sopa es puro fuego!\ldots{} ¡Pero,
chicas!\ldots{} ¿qué puñaletes de sopa es ésta?\ldots{} Os voy a matar,
os voy a arrancar el moño, haraganas, hijas putativas del
infierno\ldots» Y volviéndose a mí: «Loco me tienen ya. A todas de buena
gana las fusilaría\ldots{} y a usted también, señor Confusio\ldots{} ¡a
usted, cuatro tiros!\ldots{} Hoy estoy tremendo, estoy como en los días
peores de la guerra; hoy me han sacado de quicio, han desencadenado a la
fiera que Dios me puso dentro.»

Traté de sosegarle, y deseando hurgar su enojo para saber la causa, le
dije: «¡Que una vieja \emph{Trotaconventos} y \emph{Tragahostias} le
sulfure a usted de ese modo\ldots{} por un pleito de reales
mezquinos!\ldots{} Calma, mi amigo; no turbe su digestión por esas
bicocas\ldots»

---Sí, sí\ldots{} Son como viejas\ldots{} dos viejas, que mejor estarían
hilando que saliendo a pescar coronas\ldots{} La culpa tiene quien da su
vida por tales y tales\ldots{} ¡Qué cojilondrios!, ya no más, ya no
más\ldots{} ¡Váyanse a la porra, a la santísima porra\ldots{} con cien
puñales de peines\ldots{} y con la maldita leche que mamaron de su madre
putativa!\ldots{} ¡Quieren que me ponga las botas! Para darles un
puntapié me basta con las zapatillas, o con los zapatrancos que gasto
para andar sobre terrones.

No conseguí aplacar su furia. Para acabar de arreglarlo, las pobres
mujeres, aturdidas quizás por la tardanza del señor, descuidaron la
comida. La \emph{escudella}, que solían servirle al cura dos veces por
semana, estaba sin sal; la \emph{pelota} de carne, parte principal de
aquel popular condimento, había quedado medio cruda; la \emph{saboga},
sabroso pescado ribereño, quedó hecha papilla del exceso de cochura, y,
por fin, el asado del pato de los juncales, \emph{coll-vert}, se había
quemado y amargaba. Resistió el fiero \emph{don Juanondón}, sin protesta
ruidosa, la ruindad de los primeros platos; pero al llegar al
\emph{coll-vert}, que era manjar muy de su gusto, estalló su ira en la
forma más descompuesta. «Esto ya es zurrarse---gritó, poniéndose en pie
con gallarda impavidez de guerrillero frente al peligro.---Canallas,
cuerpo de liberales, ¿qué porquería es ésta que traéis a vuestro amo?
¿Qué cojilondrios hacéis todo el día, bigardonas, zarrapastros?\ldots{}
¿En qué pindonguerías pasáis el tiempo? Así os vea yo comidas de tiña.
¡Fuera de aquí, perras, ladronas, hijas de malas madres!\ldots»
Escupiendo estos despropósitos, cogió platos, vasos y lo que más cerca
de su mano encontraba, y empezó a descargarlos como proyectiles de mano
contra las infelices que le servían. Como en gran número habían acudido
al vocerío y escándalo, todas fueron blanco de la rociada. Las piezas de
loza volaban por el aire y se estrellaban contra la pared, o en el
cuerpo de las consternadas mujeres, que defendían su rostro con las
manos, chillando furiosamente; los cascos de porcelana, los pedazos del
pato, el salero, los tenedores, la ensalada, iban cayendo aquí y allá, y
las amas y sobrinas huyeron despavoridas hacia el interior con lamentos
de resignación más que de ira. Vi a Donata, que fue de las últimas en
huir, y oí bien claramente su voz que gritaba: «¡Santa Virgen!, ¿qué
culpa tenemos nosotras?\ldots»

\hypertarget{xxi}{%
\chapter{XXI}\label{xxi}}

Ciertamente: ¿qué culpa tenían las pobres? Así lo reconoció \emph{don
Juanondón} cuando su furia, una vez traspasado el punto culminante, fue
perdiendo su ardor insostenible, y dando lugar a la serenidad.
Limpiándose el sudor de la frente, con resoplidos más que con voces, me
dijo: «Estas tontas lo pagan\ldots{} ¿Qué culpa tienen ellas de que yo
esté lastimado en mi honor militar? Dispénseme, señor \emph{Confusio}:
hoy no ve usted en mí al Arcipreste, sino al cabecilla\ldots{} No sabe
uno cuándo es cura ni cuándo es soldado\ldots{} El soldado, el hombre
que sacrifica su vida por la Causa, salta cuando menos se piensa\ldots{}
Y yo me digo a veces: «¡Qué cojilondrios!, ¿es cuerdo que uno se haga
matador de hombres por los derechos o los torcidos de Príncipes
ingratos? ¿Valen esas coronas tan disputadas el sacrificio de hombres
dignos y valientes?\ldots{} ¡Con que he de ponerme las botas!\ldots{}
¡Con que soy un cobarde si no me las pongo!\ldots» ¡Que oiga uno estas
cosas!\ldots{} Dígolo por las viejas, que debieran ponerse a hilar antes
que meterse en estos trotes. No vaya usted a creer que es otra
cosa\ldots{} Juan Ruiz se ha sublevado, créalo usted, y se sublevará
cuantas veces sea menester, porque ha visto y ve en los españoles un
pobre pueblo sacrificado a los fanfarriosos de Madrid\ldots{} Yo he
tirado contra el Gobierno que agobia a España con las contribuciones, y
no da ningún bienestar a los pueblos\ldots{} El pueblo no come, y allá
los ricos holgazanes viven de estrujar a la pobreza. Por esto me he
sublevado\ldots{} Y yo le dije a Cabrera cuando escoltábamos a don
Carlos: «Ni tú ni yo combatimos porque sea Rey este alcornoque. Cuando
lo sea, no valdrá más que la Isabel, ni remediará la miseria del
pueblo.» Y Ramón me echó los cinco, y nos apretamos las manos, diciendo:
«Cierto es, y algún día nos pedirá Dios cuenta de la sangre que hemos
derramado por estos acebuches.» Yo debí haber hecho lo que Ramón: irme a
Londres, y hacerme inglés, y no pensar más en este país ingrato. Pero la
tierra nos llama, y el pedazo de pan que uno tiene aquí\ldots»

---Yo que usted, hombre independiente y adinerado---le dije,---no
andaría más en la compostura y lañado de Causas, y me dedicaría en paz y
gracia de Dios a cuidar mis tierras y dejarme cuidar de mis
sobrinas\ldots{}

---No puede uno\ldots{} Se impone lo hecho ya, se impone la gente que a
uno le rodea\ldots{} Cuando uno es fuerza, dominio, autoridad en un
pedacico de tierra, no puede abandonarlo. Los que aquí quedaran serían
devorados por ese Gobierno maldito. Aquí soy fuerza y poder. ¿Por qué,
amigo \emph{Confusio}? Porque protejo a todos, porque reparto entre los
infelices lo que a mí me sobra. La mitad de los vecinos de esta villa
viven de mi amparo. Si no lo cree, salga por ahí, pregunte y entérese,
¡qué cojilondrios! No me gusta alabarme; pero me alabo, ¡rediez!, cuando
llega el caso\ldots{} Y por hacer tanto bien, y amparar a tanta familia,
no hay aquí quien me tosa, y el Gobierno, haga yo lo que hiciere y
conspire todo lo que se me antoje, no se mete conmigo\ldots{} Me tiene
miedo; sabe que está en mi mano la paz o la guerra en todo el territorio
de la Cenia y del alto Maestrazgo\ldots{} Si yo abandono esto, otro lo
cogerá, y por todo paso menos porque me quiten mi mandamiento\ldots{} Ya
me pusieron los puntos para echarme de aquí\ldots{} ¿Quién dirá usted?
Pues los mismos de la Causa, cabecillas de cuartel, como decimos, y
hasta convenidos de Vergara. ¡Y que no trabajaron poco hace tres años
con el Obispo para birlarme el Arciprestazgo!\ldots{} En poco estuvo que
se salieran con la suya. Pero yo me lié el manteo y me planté en Madrid.
Por don Isidro Losa me puse en relación con la \emph{Madre} Patrocinio,
y ésta me lo arregló a mi gusto. Total: que aquí vine triunfante, y me
zurré en mis enemigos, los de Gandesa, y en el Obispo y su pistolera
madre.

---Ya ve cuán buena es sor Patrocinio, y cómo mira por los defensores
del Trono y el Altar---dije yo, sin miedo ya de que mis ironías le
ofendieran.

---¿La \emph{Madre}? Aquí, que nadie nos oye, déjeme decir que no ha
nacido bribona semejante. Si usted cree en sus llagas, con su pan se lo
coma\ldots{}

Dijo esto, y soltando luego toda la voz, gritó: «Chicas, venga café,
vengan copas.» Tomando el café que Olegaria nos trajo, y que por cierto
estaba muy bueno (con la chillería y el disparo de platos, las pobres
sobrinas habían puesto sus cinco sentidos en el servicio), continuamos
nuestra conversación, él más sosegado de su ira, yo pinchándole más para
que me descubriese todo su interior. «¿Quiere usted saber cómo estoy de
ortodoxia? Pues sepa que creo todo lo que me manda creer la Iglesia
Santa, y no pongo el menor pero, ¡qué cojilondrios!, a ningún dogma de
los que me enseñaron y enseño\ldots{} Pero fanatismo no verá en mí por
ninguna cosa de fe, como no sea por la adoración y culto de la Virgen
María. Eso desde chiquito lo llevaba en mi alma, y a Dios gracias no lo
he perdido ni pienso perderlo. A la Virgen acudo yo en mis lances
desgraciados, y la verdad, nunca me faltó, ni tengo queja de mi
abogadica celestial. Ella me sacó en mi niñez de toda enfermedad; ella
me libró de mil peligros de muerte en los combates y aprietos de la
campaña; ella fue mi sanidad en las heridas que recibí, mi escudo contra
el fuego que cien y cien veces a boca de jarro dispararon contra mí;
ella es indulgente con mis pecados, y ella me inspira las buenas
obras\ldots{} Todas cuantas caridades hago, a ella se las aplico, y
firmísimo en este amor de Nuestra Señora, espero que la tendré a mi lado
a la hora de mi muerte\ldots»

Así habló con solemnidad semejante a la que había yo notado en su
varonil rostro cuando decía la misa. Terminada la interesante
declaración de su ortodoxia, en la cual resplandecía la luz de un
apasionado culto mariano, paladeó su café, acompañado de la copita de
aguardiente. Con esto, y mis dulces exhortaciones a la paz del ánimo,
fue recobrando la que había perdido en el ya descrito berrinche, y, por
último, en actitud extática, la cabeza echada atrás contra el respaldo
del sillón, los ojos fijos en el techo, recitó esta oración arcaica:
«Santa Virgen escogida,---de Dios Madre muy amada,---en los cielos
ensalzada,---del mundo salud e vía\ldots» Esta oración---dijo luego
llevándose a los labios la copa---me la enseñó mi madre cuando era niño,
y siempre la digo al acostarme y levantarme. No es ésta la única que mi
madre sabía; otras que recitaba de continuo también me enseñó. Oiga
usted la que digo siempre que me veo en un gran aprieto: «¡Oh Santa
María,---luz del día!---Tú me guía,---dame gracia y bendición---e de
Jesú consolación\ldots» Para los lances apurados de guerra, cuando
atacábamos a la bayoneta, o dábamos carga de caballería, tenía yo otra
plegaria, que por el sonsonete redoblado y vivo me parecía muy propia
para el paso de ataque. Oiga usted: «Tú, Señora,---dame agora---la tu
gracia---toda hora,---que te sirva---toda vía\ldots» Nunca dejó de
ampararme la Madre de Dios. Por eso podrán decirme que si creo tanto más
cuanto, en lo tocante a otros puntos de religión; pero en este punto,
¡rediez!, nadie puede decirme nada.»

---Las oraciones que acaba usted de recitar---le dije,---son del
Arcipreste de Hita, varón docto, muy devoto de Nuestra Señora, poeta y
sabio, aficionadísimo al buen vivir y al trato de mujeres, según él
mismo nos cuenta en su magno \emph{Libro de buen amor}. Menos en lo de
acaudillar tropas y andar en guerra contra cristianos, usted y él en
todo entiendo yo que se parecen; y para completar la semejanza, el de
Hita era, como usted, hijo de Alcalá de Henares; como usted Arcipreste,
y también se llamaba Juan Ruiz\ldots{}

Ya tenía entre los dientes mi amigo algún discreto comentario sobre su
semejanza con el de Hita, glorioso poeta, cura, gastrónomo y mujeriego
del siglo XIII, cuando su atención fue repentinamente sustraída por
Olegaria y Toneta, que de puntillas a la puerta llegaron, queriendo ver
si había pasado la nube. «Entrad, entrad sin miedo---les dijo don
Juan.---Bigardas, mostrencas, ya estáis recogiendo los cascos de la loza
que os tiré a la cabeza. Limpiad suelo y paredes de la grasa y piltrafas
del pato, que no se podía comer. ¿Verdad, \emph{Confusio}, que no se
podía comer?» Animadas por el tono tranquilo del clérigo entraron otras,
entre ellas Donata, y se pusieron a recoger los despojos de la refriega.
Apenas comenzaron, sonó el aldabón de la puerta de la casa.
Estremecimiento general, zozobra y susto repentino del Arcipreste.
Donata, que había corrido a una ventana para ver quién llamaba, volvió
azorada diciendo: «Señor, es mi tía\ldots» Y don Juan Ruiz exclamó con
todo el estruendo de su voz: «¡Cojilondrios, me llaman otra vez!\ldots{}
Tengo que ir allá.» Acudiendo a recoger su gorro y balandrán, recobró el
aspecto terrorífico que había traído de la calle cuando vino a comer.
Sus ojos echaban lumbre, se le encendió el rostro, en su maxilar veíamos
la vibración del músculo\ldots{} Dando un empujón a Donata, le dijo: «A
tu tía, que voy en seguida\ldots{} ¡Por los cojilondrios de San Pedro,
que no me hurguen, que no está este león para tafetanes!\ldots{} «Tú,
Señora,---dame agora---la tu gracia---toda hora\ldots»

Viéndole tan enfurruñado, le pregunté si quería que le acompañase; me
respondió que iría solo. Al bajar la escalera se volvió para decirme:
«Si pasea usted esta tarde, lléguese al bodegón de Llopis\ldots{} ya
sabe\ldots{} al fin de esa calle de enfrente, torciendo a la
derecha\ldots{} Por allí me pasaré cuando de esta pejiguera me
desocupe\ldots»

¡Qué bien me venía quedarme solo en la casa con el rebaño mujeril!
Mientras ayudaba solícito a recoger los pedazos de loza y vidrio, supe
que ya tenía respuesta mi segunda epístola. En un momento en que solas
conmigo quedaron en el comedor la \emph{Dolorosa} y Donata, ésta, con
sólo medias palabras, el mirar revelador y el gesto expresivo, me hizo
saber que me daría su carta en cuanto Toneta saliera. Dicho y hecho:
diez minutos después de esta telegrafía rápida, el papelito estaba en mi
poder. Mientras la familia comía, me bajé a leer a la huerta, como el
día anterior. Entre las hojas del primer tomo del \emph{Concilio de
Trento}, libro que me interesa tanto como la \emph{Vida de Bertoldo},
metí el mensaje de mi odalisca, y bajo los frondosos árboles que rodean
la noria, lo leí muy a mi gusto. De la primera a la segunda carta había
madurado la dulcísima fruta del amor de Donata, hasta el punto de que ya
manifestaba resueltamente, con amoroso abandono, sus deseos de libertad.
No podía ya vivir en tan horrible suplicio\ldots{} Dios le había enviado
consuelos con mi presencia, y la Virgen, hablándole al corazón, le decía
que soy un hombre bueno y honrado, incapaz de engañar a la pobre
prisionera que en mí confía\ldots{} Decía también que ella es religiosa,
y que la entusiasma verme tan aplicadito a la lectura de libros
sagrados\ldots{} que la Virgen la absolverá del pecado de su fuga, si en
efecto puede lograrla, porque su fin no es otro que buscar la paz y la
virtud fuera de aquel triste caserón.

Todo esto decía, y aún más, pues no faltaban expresiones de intenso
cariño. ¡Qué triunfo, Dios mío; qué admirable victoria ganada por mi
audaz estrategia de amor, con las armas de mi mérito personal y de la
fogosa elocuencia que pongo en mis cartas! Sólo faltaba determinar el
plan completo de la fuga, con toda la tramitación prolija de tan
peliagudo negocio\ldots{} No bajó aquella tarde Donata al gallinero,
prudencia y disimulo dignos de alabanza. Pero en otra ocasión y lugar
próximos me mostró la hermosa joven su agudeza y sus instintivas artes
amorosas, porque sabedora de que yo había de salir para juntarme con don
Juan en el figón de Llopis, hizo tan exacta distribución de sus
quehaceres y tan feliz medida del tiempo, que cuando yo salí estaba ella
barriendo el portal.

Bendije la casualidad, que era de las previstas, y me regalé con un
diálogo delicioso en su apurada rapidez. Pocas palabras bastaron para
repetir y afirmar el pacto de amor\ldots{} Otra vez escribiría
yo\ldots{} Ella me señalaría en su respuesta sitio y hora para celebrar
una entrevista en la cual dejaríamos acordada la hora de evasión,
etc\ldots{} Preguntele yo si podíamos contar con su tía\ldots{} Pedile
noticia breve de los negocios, pleitos o diabluras que tenía el
Arcipreste con aquella señora anciana, y quise saber el motivo de la
furia del buen señor\ldots{} A esto no contestó Donata más que con un
vacilante \emph{no sé}, frunciendo el entrecejo y mirándome como en
demanda de perdón por no ser más explícita. Comprendí que no debíamos
hablar de semejante cosa: a su razón y tiempo se hablaría\ldots{} y con
esto terminamos. Donata me indicó que saliese, y la obedecí,
condenándome al suplicio de no mirar atrás cuando atravesaba la
plazuela\ldots{} No puedo expresar el alborozo que llevaba yo en mi
alma: era como un sol vivísimo que me alumbraba el entendimiento, y como
celestial música que me lanzaba el corazón a un danzar frenético. ¡Oh
portento de la hermosura, oh \emph{Erhimo}, ya tu apasionado caballero
abre los brazos para traerte a la libertad, a la paz y al amor! Hierros
del harem, rompeos en mil pedazos. Astucias y malas artes de \emph{El
Nasiry}, ya nada podréis contra las invencibles armas de
\emph{Confusio}.

\hypertarget{xxii}{%
\chapter{XXII}\label{xxii}}

Era el bodegón de Llopis un local telarañoso y mugriento, donde bebían
los que tenían sed y jugaban a los naipes algunos holgazanes viciosos:
en él vi el boceto, el trazo rudimentario del moderno casino, sitio de
reunión, de vago charlar, mentidero y bebedero público, con el aspecto y
colorido que tenían estos lugares en tiempos del Arcipreste de Hita;
pero algo más era, pues allí, no el de Hita, sino el de Ulldecona,
celebraba juntas, recibía embajadas y mensajes, dictaba órdenes,
ejerciendo las funciones de su califato político, social y
militar\ldots{} Entré en el humano pesebre con el propósito de esperar a
mi señor don Juan; mas resultó que ya él a mí me esperaba. Los
parroquianos que en sucias mesas comían o jugaban, me miraron con
curiosidad y respeto, mientras un vejete adiposo, que parecía dueño del
establecimiento, me señalaba una escalera de palo, diciendo: «Arriba
está \emph{Don Juanondón} aguardándole.» La escalera, de añoso castaño
ennegrecido, chillaba con todas las tablas de sus desvencijados
peldaños, cuando uno subía por ella: era un son de coplas con cadencia
de romance gangoso, recitado por bocas sin dientes\ldots{} El ritmo de
la escalonada madera me llevó a un cuartucho ahumado, que recibía la luz
de dos agujeros, más que ventanas, con barrotes en diagonal. Mesa larga
del mismo castaño musicante ocupaba el centro, y junto a ella vi a mi
señor Arcipreste sentado en un banco, hablando con dos tíos de
zaragüelles, grandones, macizos, terribles cuerpos para el trabajo y
para la guerra. En lo que don Juan les decía, creí entender órdenes de
permanecer pacíficos, y advertencias concernientes a la labranza, todo
mezclado; extraño amancebamiento de Marte y Ceres. En el atezado rostro
de aquellos interesantes bárbaros, vi la ingenuidad del hombre
medioeval, laborioso en la paz, matón en la guerra, defensor de su
terruño y de sus rudas creencias con fanático heroísmo\ldots{}
Despidioles don Juan a punto que entraba el hostelero con un jarro de
vino blanco y pastelitos tortosinos, que llaman \emph{panolis}.

---Siéntese, amigo y tocayo---me dijo el clérigo, a quien noté
totalmente aplacado del berrinche,---y charlemos; que una charla sabrosa
es el mejor alivio de los ánimos destemplados\ldots{} He mandado traer
este blanco de Sitges y estos pasteles para reparar nuestros estómagos;
que hoy apenas comimos\ldots{} con el jaleo que armé en casa\ldots{} y
las torpezas de aquellas chicas.

---Me place mucho su compañía, señor Arcipreste---le dije,---y no
rechazo el vino y los pastelitos. A lo que entiendo, este tugurio es
para usted salón de embajadores, cuartel general, sala de
audiencia\ldots{} Aquí dicta la guerra o la paz\ldots{}

---Cierto, cierto, y acabo de dictar paces. No hay quien me saque de mi
ten con ten, ni por ningún interés de fantasmones me meto yo en
aventuras sin elementos para llevarlas a su término debido.

---Aquí ejerce usted su cacicato; aquí convoca el sínodo de los curas
que de usted dependen, y les dicta órdenes guerreras\ldots{}

---Y órdenes espirituales, amigo mío: de todo hay\ldots{} Aquí me las
tengo tiesas con los de Tortosa y con los de Tarragona, cuando hay
alguno que me quiere fastidiar, llámese Obispo, Gobernador militar o
Jefe político\ldots{} Ésta es la oficina de mis auxilios a los
campesinos que andan estrechos; aquí dispongo darles tanto más cuanto de
trigo para simiente, dinericos para la contribución, y aquí me traen
ellos sus bendiciones\ldots{} No digo esto por alabarme, sino para que
usted lo sepa, y salga a mi defensa cuando vaya por ahí, y algún
ignorante o malicioso le hable pestes del \emph{Cabezudo} de Ulldecona,
como suelen llamarme en Tortosa y Gandesa\ldots{}

---Yo diré de usted todo lo bueno que he aprendido en su hospitalidad y
compañía\ldots{} Y espero decirlo pronto, señor Arcipreste, porque ya
está pesando sobre mi conciencia la ociosidad.

---No le diré que se detenga más, porque si mucho me honro con tenerle
en mi casa, también me inquieta el pensar que lleve retraso en sus
diligencias.

No podía yo discernir si esto me lo decía con sinceridad, o si era
delicada fórmula para indicarme que ya estoy de más aquí.

«¿Y ya no me pregunta nada el amigo \emph{Confusio}---me dijo
riendo---de las viejas impertinentes, que me han dado ayer y hoy la más
grande matraca que puede sufrir un cristiano?»

---Nada de eso pregunto---respondí,---porque entiendo, señor Arcipreste,
que usted no me respondería la verdad.

---Así es, amigo y tocayo, pues nadie está obligado a referir todas las
cosas; que algunas hay que por su intríngulis no deben salir nunca del
encierro de la discreción\ldots{} Cuando vuelva usted por acá de paso a
Madrid, si es que va por Valencia\ldots{} y ya sabrá que de Valencia a
Madrid tenemos ferrocarril, y hecho está un buen pedazo del que de
Valencia viene hacia acá\ldots{} cuando vuelva, digo, le contaré estos
lances para que se divierta un poco y tome apunte de ellos, por si le da
la gana de escribir algún día unas miajicas de Historia.

---No le aseguro a usted que no las escriba. El arte de referir los
hechos públicos o que deben serlo, me seduce, y algunos ensayos tengo
escritos de este arte difícil.

---Es usted un sabio, señor \emph{Confusio}, y pocos habrá que en edad
tan corta hayan reunido en su caletre tanta ciencia y tal caterva de
conocimientos, de los que se sacan del alma fría de los libros. Lo que
yo dudo, y con franqueza se lo digo, es que todo ese caldo soso de
bibliotecas le sirva de algo\ldots{} ¿De veras está decidido a cantar
misa? ¿No teme que de aquí al momento de las órdenes mayores puedan
venirle arrepentimientos, o siquiera tibieza de la vocación?

---Me parece que no, señor Arcipreste. Cada día siento mayor seguridad
de que no han de faltarme los alientos y el entusiasmo que me llevan por
ese camino.

---Muy bien: yo le felicito por su constancia. Sin duda tiene usted un
temple tan apagadico, que no temerá las zaragatas entre lo divino y lo
humano, ni se verá en riesgo de pecado, o de faltar gravemente a lo
divino\ldots{} Yo, como hombre tan largo de experiencia que se pierde de
vista, puedo aconsejarle\ldots{} No se asuste porque le diga que si
siente quemazones de lo humano, tan fuertes que amenacen con abrasar lo
divino, no les eche agua fría de penitencias, que esto a la postre es
malo, así para el cuerpo como para el alma\ldots{} No sé si me explico
bien\ldots{} Yo he notado que es usted encogidico; pero como he visto
tantos zorronglones de ojos caídos y semblante mustio, que luego han
salido unos grandes peines, no sé qué opinión formar de usted. Podrá ser
usted lo que parece, y podrá no serlo\ldots{} ésa es mi duda. Quizás la
misma duda tenga usted; que el hombre no se conoce tal como es hasta que
llegan ocasiones singulares de la vida que sacan lo escondido y hacen
ver a cada cual lo que tiene dentro\ldots{} Pero, en fin, por lo que
valga, yo le doy a usted mis consejos, y usted los toma para hoy o los
guarda para mañana, como esas cosas de apariencia inútil que guardamos
creyendo que para nada han de servirnos, y el mejor día, ¡pum!, resulta
que nos hacen mucha falta.

---Dígame, dígame lo que quiera---respondí gozoso y atento,---que sus
opiniones son oro puro para mí. Yo quizás no sea todo lo cuitado que
parezco; quizás me encuentre en el punto ese sutil en que no puedo decir
con certeza si me siento bien seguro en las virtudes de humildad,
castidad y limpieza de pensamientos, o si, por el contrario, me asaltan
temores y barruntos de caer en esos infiernos de lo humano que me
cerrarían la puerta de lo divino\ldots{}

---Poco a poco---dijo el cura, echándose atrás el gorro después de
atizarse una copa del blanco vino.---No estoy porque a lo humano se le
llame infierno\ldots{} ¿Cómo pudo hacer nuestro Criador la Humanidad
para el sufrimiento y la privación de sí misma? No, no: lo humano es
obra de Dios como lo es lo divino\ldots{} En fin, amigo \emph{Confusio},
hablemos claro, y cada cual de lo suyo. No quiero meterme en
filosofainas, sino presentar a usted hechos particulares míos, tan míos
como mi cuerpo y rostro. La verdad y la ciencia están en lo \emph{que a
uno le pasa}, y lo demás es viento de sabidurías vanas\ldots{} Pues a mí
me ha pasado que no he podido echar de mí el amorcico de mujer\ldots{}
Entiendo que sin mujer no vive el hombre; y cuanto me digan en contrario
téngolo por una pesada broma que nos quiso dar el judío Moisés, o errata
de imprenta de los sagrados Cánones. Nunca dijo Nuestro Señor Jesucristo
de que los sacerdotes habíamos de vivir del aire de mujer, y nada más
que del aire\ldots{} ya usted me entiende\ldots{} y en todo caso, paso
por que ello sea mérito, obligación nunca\ldots{} ¿No está usted
conforme conmigo?

Asentí sin quitarme la máscara de mi timidez, pues esto en ningún modo
me convenía, y con hábiles réplicas le incité a clarearse más y
descubrirme todo su interior. Al desbordamiento de su sinceridad
contribuía la frecuencia con que se atizaba vasitos y más vasitos de lo
añejo. «¿No cree usted como yo que la mujer es una de las más apañadas
creaciones de Dios?\ldots{} ¿Me negará usted que ha nacido para recibir
los obsequios del hombre, y que estos obsequios son la sembradura de las
generaciones?\ldots{} Cierto que en la gran caterva de mujeres las hay
impertinentes, desabridas y fastidiosas, y de éstas debe huir el hombre
de gusto; pero las hay también adornadas de mil encantos, ¿no es verdad?
¿Y no observa usted que hay mil y mil pobrecicas que quedan sueltas y
horras, porque no se casan todos los hombres que debieran casarse? ¡Ay!,
el Arca del matrimonio es cada día más estrecha, y en ella no caben
todas las parejas de animales, o sea de hombre y mujer. Debemos mirar
con caridad a las hijas de Dios que no han encontrado colocación en el
Arca\ldots{} Yo he sido bueno para ellas; las he amparado, y a muchas
proporcioné buen casamiento después de tenerlas algún tiempo a mi
servicio\ldots{} A otras, que eran holgazanas, las he arregostado al
trabajo; a las sucias, enseñé limpieza y curiosidad. Di de comer a las
hambrientas, y a las ignorantes, como fieras cogidas con lazo, les di el
pan de la enseñanza: lectura y escritura. He sido, aunque me esté mal el
decirlo, un gran civilizador, y si me apuran, el buen pastor de esa
parte del rebaño femenino condenada por el mundo a la pena capital de
vestir imágenes.»

No pude contener la risa. Con el vino y la natural malicia del asunto
tratado, se iba poniendo el Cura en un punto de alegría y gracejo que
daba mayor encanto a su sinceridad. Digno era de envidia, por haber
arreglado su vida tan a gusto, agenciándose riqueza, autoridad sobre los
hombres, dominio sobre las mujeres\ldots{} «Muchas me han querido cuanto
se puede querer---dijo poniendo un poquito de amargura en sus
remembranzas;---otras han sido ingratas. No me han faltado sofocos y
peloteras. Naturalmente, dejando entrar en el alma las pasiones que
halagan, no podemos librarnos de las que nos atosigan: la cólera, los
celos malditos. No puedo decir que he sido violento y malo más que una
vez. La Virgen me lo perdone, si no me lo ha perdonado ya\ldots{} Verá
usted: fue en lo más duro de la guerra, siendo yo cura de Albalate y
jefe de la Caballería de Quílez. \emph{Hablaba} yo entonces, para
decirlo decorosamente, con una muchacha de Alcaine, que era un sol de
bonita, morena como el trigo, con un sonreír de ángeles y unos ojos de
fuego que disparaban bala rasa\ldots{} Para decirlo de una vez, me
enamoré de ella como un bestia\ldots{} La puse en casa de una tía suya
en Valdeconejos, a donde iba yo a verla siempre que el trajín de la
facción me lo permitía\ldots{} Lleváronme el soplo de que la Fabiana me
estaba faltando\ldots{} No lo creí. Lleváronme otro cuento: que me
faltaba con un teniente de la partida del Royo\ldots{} Ya dudé\ldots{}
Lleváronme el chisme de que Fabiana y el teniente hacían escapadas de
noche por las huertas del pueblo\ldots{} Allá me fui\ldots{} aceché, no
vi nada\ldots{} Aceché más, vi\ldots{} Vamos, que los cogí haciéndose
fiestas. ¡Usted figúrese\ldots{} con mi genio! Salté del zarzal en que
estaba escondido\ldots{} Agarré al teniente por un tupé muy empinadico
que gastaba, y asegurándole de modo que no podía moverse, le disparé mi
pistola en la sien derecha\ldots{} El tiro salió por la sien
izquierda\ldots{} La Fabiana voló chillando, y no he vuelto a
verla\ldots{} ni me ocupé más de esa trotera putativa, que quedó bien
castigada, con cien mil pares de cojilondrios\ldots»

Una ráfaga de frío corrió por todo mi cuerpo al oír el trágico suceso
del Cura, y al figurarme la escena bárbara y breve que con terrible
concisión me contaba. Díjele que difícilmente podía Nuestra Señora
perdonarle tan brutal homicidio; pero él, que de copa en copa iba
cayendo en un estado, no diré de embriaguez, pero sí de alegría voluble,
dispersión juguetona de sus pensamientos, no hizo caso de mis severas
palabras, y me invitó a secundarle en la empinación del codo. Resistime
yo a ello, y él entonces con hipérboles de cariño, entremezclando los
acentos de alegría con acentos llorones, me dijo: «\emph{Confusio} mío,
sigue mi consejo y toma las órdenes, sin cuidarte de lo que ahora o
después te digan en contra del estado religioso tus nervios y tu
sangre\ldots{} No seas cuerpo sin alma\ldots{} También ser alma sin
cuerpo es mala cosa\ldots{} Veo la vida como un jardín. Todo lo bueno
que Dios hizo en este jardinico es para nosotros\ldots{} para el hombre
todo lo bueno, no para los burros\ldots{} El burro es el que se priva de
lo bueno\ldots{} Lo mejor entre lo bueno es el amor\ldots{} y lo más
santo, lo divinamente divino. Ríete de los que dicen no a todo lo bueno
y sabroso\ldots{} Yo digo: la serpiente tenía razón\ldots{} mi señora la
serpiente supo lo que se hacía\ldots{} Adiós, \emph{Confusico}, vete a
Tarragona\ldots{} dale memorias al Deán, al Obispo y al
Archipámpano\ldots{} y que te echen pronto la sagrada crisma\ldots{}
Adiós, hijo mío, que seas bueno, que metas el dedo en la olla de la miel
prohibida\ldots{} Adiós.» Después, su creciente alegría se extremó en un
canticio, golpeando la mesa con el vaso, con ritmo de paso doble: «¡Oh,
María!,---luz del día,---tú me guía---toda vía\ldots»

\hypertarget{xxiii}{%
\chapter{XXIII}\label{xxiii}}

Al día siguiente del suceso, más bien de la sabrosa espontaneidad del
buen Arcipreste en el tabernáculo, se precipitó el curso de mi aventura
con Donata, hasta llegar al punto que ella y yo deseábamos\ldots{} En
nuestras últimas cartas, y en una breve entrevista que tuvimos, ya
después de anochecer, quedó concertado el plan de su evasión y fuga
conmigo. No ocultaré que si la proximidad de mi dicha inundaba mi alma
de gozo, no me veía libre de algún punzante recelo cuando pasaba por mi
mente la imagen de don Juan Ruiz, a quien veía en las formas de su enojo
antes que en las de su bondad. Recordaba el caso de fiereza que me había
contado en el bodegón, y su poder en toda esta tierra, donde la
muchedumbre de sus amigos y adeptos favorecerá sus venganzas. Y
aumentaba mi intranquilidad la confusión en que me tiene la persona
moral del Arcipreste, cuyo carácter verdadero no he podido penetrar en
trato tan corto. En él veo cualidades excelentes, virtudes afeadas por
el vicio, barbarie y talento en increíble mezcolanza, y otro revoltijo
no menos extraño de orgullo feudal y supersticiones, de crueldad
sectaria y democracia piadosa. Sin conocerle a fondo, ¿cómo discernir el
sistema de defensa que debo emplear contra él?\ldots{} Confiaba yo en
que aportase mi amada nuevos datos para el estudio del personaje, que
bien pronto había de ser nuestro mayor enemigo.

Para terminar la parte de mis aventuras fechadas en esta villa de
Ulldecona, consigno aquí las resoluciones que adoptamos Donata y yo para
la evasión y huida. Yo me despediré de don Juan a las diez de la mañana,
saliendo con mi equipaje, en dirección a Tortosa y Tarragona, con toda
la tranquilidad que simular pueda, y a la mitad del caminito, poco más,
en una villa nombrada Santa Bárbara, me detendré, despachando para
Tortosa la tartana con mi maleta. Acto seguido me personaré en la casa
de un alquilador de coches llamado Manalet, y ajustaré otra tartana, en
la cual me volveré a Ulldecona, a punto del anochecer, entreteniendo el
tiempo de modo que no llegue aquí hasta las doce de la noche. Al pueblo
me aproximaré, rodeando, hasta un sitio que llaman \emph{Los Olmos}, por
la parte del camino de la Cenia, a espaldas de la parroquia y casa
rectoral. Allí, junto a unos molinos aceiteros, debo esperar con mi
tartana; allí se juntará conmigo \emph{Erhimo}, digo, Donata.

Tal es la parte mía en el plan; ved ahora la de mi cómplice. Donata,
encargada de cerrar el portalón de la huerta, hará todo lo contrario,
que es fingir que lo cierra y dejarlo abierto. Se acostará como siempre
en el cuartito alto, donde también duerme Toneta. Ya cuenta con que ésta
la favorecerá con su ayuda y su silencio. Recogerá en un lío toda la
ropa que pueda llevar, y a media noche bajará descalza o con alpargatas,
llevando para el perro queso y pan con que acallará los ladridos del
honrado animal. Ya \emph{Sultán} la conoce: es su amigo y no ha de
hacerle ninguna mala partida en el crítico momento\ldots{} Arriesgadillo
es el complot; pero confío en mi buena estrella y aguardo lo que el
destino quiera depararme\ldots{} Adiós, Ulldecona; adiós, orgulloso
Arcipreste, y que en la próxima noche sea pesado tu sueño y ligeras las
sandalias de mi amante odalisca\ldots{} Punto final. A ti me encomiendo,
Beramendi amigo, para quien son estos desaliñados renglones.

\textbf{Tortosa}, \emph{Abril}.---Entiendo que los divinos ángeles y San
Antonio bendito, protector de los enamorados, se pusieron de nuestra
parte en aquella memorable noche, porque todo el plan presupuesto quedó
cumplido sin la más leve contrariedad. ¡Jesús mío, qué suerte! A la
media hora de estar yo en la espera de Los Olmos con la ansiedad que
puede suponerse, vi que de la obscuridad se destacaba un bulto, cargado
con otro bulto menor, o lío de ropa. El corazón, antes que la vista, me
dijo que era Donata. No hallo términos con que pintar mi alegría, y la
priesa con que introduje a mi fugitiva en la tartana, y di al tartanero
las órdenes de salir a escape. Comprenderéis, oh insignes Marqueses,
Mecenas míos, que los primeros instantes de nuestra viajata fueron
consagrados a la celebración del santo suceso, la divina libertad
lograda con manifiesto auxilio del Cielo, y que el himno de júbilo y las
felicitaciones consiguientes se confundían con amorosas ternezas, y con
las caricias que a mi audacia consintieron la timidez y encogimiento de
Donata. Luego se recogió ella en su piedad, rogándome que le permitiese
rezar el rosario, a lo que no pude oponerme, por más que ni el rosario
ni ninguna otra forma de devoción estaban en mi programa. Hícele mil
preguntas, a las que contestó que, no creyéndose segura hasta pasar de
Santa Bárbara, convenía que nos encomendáramos a Dios, dejando para las
horas de tranquilidad las explicaciones y comentarios de lo que atrás
quedaba y de lo que teníamos camino adelante. Hube de acompañarla en la
enfadosa recitación del rosario, y en verdad, poco me importaba esta
corta interrupción de nuestra dicha, teniendo ya en mi poder a la bella
\emph{Erhimo}, sacada por mi astucia del harem de don Juan Ruiz.

Antes de amanecer, pasado ya el lugar de Santa Bárbara, vi a mi
\emph{Erhimo} repuesta de su ansiedad y susto. Quise que satisficiera mi
curiosidad en algunos hechos observados y nunca comprendidos durante mi
residencia en Ulldecona, y empezó por aclararme el enigma de aquel
misterioso casón de puerta heráldica, y del berrinche que allí había
cogido el Arcipreste, el día de la voladura de los platos.

«Te habló de una vieja cócora y pleitista---me dijo Donata,---para
desorientarte. En aquella casa están escondidos el Rey y su hermano.
Nadie lo sabe. Yo y algunas de nosotras lo sabemos. En la casa vive una
señora anciana, rica, noble, y no partidaria de la Causa. Mi tía es
criada de la señora, que se llama doña Tiburcia\ldots{} Él ha querido
que don Juan se lance al campo con su gente; don Juan no estaba por eso.
Insistió el Rey con malos modos, y de ahí vino el sofoco del Cura y la
furia que desahogó en casa con nosotras\ldots{} Una cosa te pido, Juan,
y es que al llegar a Tortosa, a nadie hables del pueblo y casa en que
están escondidos el Rey y Príncipe; que no debemos meternos a
delatores.»

Pareciome muy atinado y prudente este propósito de discreción, y allá se
entendiera el Gobierno con aquel Rey de pega, que no sabía por dónde
salir del pantano. Luego me informó Donata de algo muy interesante, que
hasta entonces era otro enigma para mí. En Tortosa nos aposentaríamos en
la casa de una prima suya, llamada Polonia, con quien sostiene
relaciones de amistad cariñosa. Se criaron juntas, se quieren como
hermanas. Viuda de un zapador, Polonia vive del corto rendimiento de una
modesta casa de pupilos, puesta bajo los auspicios de la guarnición de
la plaza: son sus huéspedes un capitán, el Músico mayor y uno o dos (en
esto no estaba muy segura Donata) capellanes castrenses. Ya había
escrito a Polonia notificándole su resolución de abandonar, por el
procedimiento de la fuga, pues no había otro, la casa y el nada honroso
patronato del Arcipreste. Segura estaba de ser bien acogida, y de que en
casa de su prima podríamos trazar sosegadamente nuestros planes del
porvenir. A mi recelo de que en aquel refugio nos alcanzase la
persecución del celoso don Juan, opuso Donata esta afirmación
tranquilizadora: «No temas, \emph{Confusio}. No va el Arcipreste a
Tortosa ni atado. Allí son pocos sus amigos, muchos sus enemigos, y hay
unos cuantos que se la tienen jurada.» Esto me dio un buen pie para
pedir a Donata su opinión del carácter de don Juan. ¿Qué pensar de tal
hombre? ¿Es bueno, es malo, o un plexo intrincado de cualidades
recomendables y perversas?

«Es bueno---dijo la guapa moza;---todo lo bueno que puede ser el que no
vive como es debido. La mala es Olegaria\ldots{} envidiosa, egoísta, y
además tan torcida y dañada de religión, que si se va a mirar, en nada
cree: si no es atea, le falta poco\ldots{} Piensa y dice cosas que hacen
estremecer al Santísimo en su altar\ldots{} Y no has visto otra más
ambiciosa: todo lo quiere para sí\ldots{} Te roba las estampicas, los
pañuelos, las agujas y dedales, y hasta un bollo que tengas guardado
para tu merienda\ldots{} ¿Y golosa?\ldots{} más que una gata. ¿Y
acusona?\ldots{} un horror. Ella es la que con sus chismes y
cuentilorios trae revuelta a toda la familia.» Bien claro me decía
Donata que sus antipatías se concentraban en la rubia. Los motivos Dios
los sabrá\ldots{} Quedábame yo en el Limbo de mis dudas respecto al tipo
moral del Arcipreste; y por más que reiteré mis preguntas, no pude
obtener de Donata más que confusiones semejantes a las mías. «En
conciencia---me dijo,---no puedo responderte como tú deseas. ¿Es bueno,
es malo? Yo, pobre mujer sin mundo, no puedo darte sentencia fija sobre
un hombre como ése, tan raro en sus sentires, en sus pensares y en sus
entenderes. Como bueno, bueno, no es, digo yo, pues siempre está
faltando, Juanico, faltando a lo que manda Dios, y haciendo faltar a los
demás\ldots{} Como malo, malo, no es tampoco, porque a lo mejor te saca
unos arranques de hombre bueno que te dejan pasmado. Así es que no sé,
no sé\ldots{} Tú, que eres sabio, sabrás esto de que un hombre pueda ser
malo y ser bueno\ldots{} y de que haya bondades malas y maldades
buenas\ldots»

Camino adelante, repetíamos de vez en cuando las tiernísimas expresiones
de nuestro afecto, al rodar trompicoso de la tartana; nuestra
conversación se iniciaba con cualquier asunto, y siempre, sin saber
cómo, derivaba hacia la familia, casa y asuntos del Arcipreste. Por esta
razón me enteré de interesantes particularidades, que quiero consignar
sin demora para satisfacción de mi amigo Beramendi, y de los ociosos que
en edad próxima o lejana leyeren estas deshilvanadas aventuras. Pues,
según las referencias de Donata, son muy variadas la procedencia,
categoría y funciones de cada cabeza de ganado en la femenina grey del
buen Hondón. Mujeres hubo allí que debieron al amor su ingreso en el
hogar; mas esto no era lo común; mujeres hubo que entraron simplemente a
servir; otras que eran hijas de antiguas servidoras; otras que llegaron
inopinadamente sin más razón que la caridad del Arcipreste, gran
amparador de huérfanas, y aliviador de viudas ahogadas, y de familias
venidas a menos. No todas las muchachas que entraron con este carácter,
dando a la casa vislumbres de hospicio, incurrieron en debilidad de amor
con el Cura. Hubo casos rarísimos: feas que pecaron, y hermosas que
salieron tan puras como habían entrado.

Comprendía Donata en la síntesis de \emph{familia} a las que yo
designaba, por su edad, en las dos clases de \emph{amas y sobrinas}. Y
resultaba que eran sobrinas algunas que yo tuve por amas, y al
contrario; y otras, las más, no tenían nada de sobrinas por razón de
parentesco. Por ejemplo, Carmeta, ya madura, era sobrina efectiva, hija
de una hermana de don Juan; Toneta, la \emph{Dolorosa}, era hija de
Monsa, una de las más viejas amas, prima hermana de don Juan.
Clasificadas por el lenguaje, resultaban los dos grandes grupos,
aragonés y catalán, dominando el primero, porque de tierra de Teruel
solían mandarle al poderoso Arcipreste remesas de lucidas zagalonas para
que las amparase y pusiera en la carrera de matrimonio. Olegaria, la
pérfida y venenosa rubia, es catalana, y no tiene vínculo de sangre con
el patrono, ni con ninguna de sus amas o amadas de diferente edad y
abolengo: vino al cotarro como de aluvión. Si la costumbre de no
despedir a nadie acreditaba el buen corazón del Cura, por otra parte era
grave mal, porque la familia crecía desmedidamente, con riesgo de
choques y zaragatas. Por último, de sí misma habló Donata muy poco, y
aún ignoraba yo totalmente su origen, el cómo y cuándo entró en la
familia, y otros mil pormenores y circunstancias que eran sin duda de
grande interés. A mis insinuaciones pidiéndole estas para mí preciosas
noticias, se anticipó así: «No te impacientes, Juanico, que tiempo
tenemos de hablar de todo, y de que yo te cuente lo que es fácil de
decir y lo que no se dice sin trabajo y pena.»

Nuestro viaje se acortaba por momentos, y a las primeras luces del día
vimos un paisaje en que Donata reconoció las inmediaciones de Tortosa.
Ya estaba cerca la caudalosa corriente del Ebro; ya se veían los cerros
que circundan la histórica ciudad; ya llegábamos a nuestro refugio, y
empalmábamos el fin de una vida con los comienzos de otra, que habrá de
ser felicísima\ldots{} Estimulados ambos por la frescura de la mañana y
por el gozo que trae siempre un nuevo día, renovamos nuestro juramento
de amor, y sellamos el pacto con arrebatadas ternezas. Libertad dijimos
al salir de Ulldecona; voluntaria esclavitud proclamamos al enfilar el
puente de barcas para entrar en la venturosa ciudad, que a Donata y a mí
nos pareció la más bella y alegre del mundo\ldots{} como que fue espejo
en que nuestra felicidad se reproducía.

Y a medida que nos internábamos en la población, dejado el suplicio de
la tartana, mayor alegría sentimos. Hízome admirar Donata la diligencia
con que acudían los hombres a sus varias industrias y trabajos, la
belleza y lozanía de las mujeres, la no menos opulenta hermosura de los
frutos del suelo, que en el mercado acreditan la feracidad del vergel
circundante\ldots{} Estas impresiones, y el cielo azul, la luz vivísima
que hacía sonreír a todas las cosas, y el caudal majestuoso del Ebro,
penetraban en mí con las formas de amor, de esperanza.

Pensó Donata que antes de entrar en la que había de ser nuestra casa,
situada en lo que llaman \emph{el Rastro}, debíamos ir a la Catedral a
dar gracias a Dios y a pedir a la Virgen de la Cinta que nos amparase en
la vida nueva que emprendíamos. Me pareció muy bien. A la santa iglesia
nos fuimos, la cual por fuera es de un greco-romano mazacote y
pedantesco, interiormente bella, mística, ornada de primores artísticos
y de ingenuas fruslerías costosas, que mueven a la devoción. La Virgen
de la Cinta, ante cuya majestad estuvimos arrodillados largo rato, es
linda, consoladora, de expresión divinamente afable. Ninguna imagen he
visto que me haya cautivado tanto como ésta, ninguna que tan bien
sintetice en su rostro la dulzura y la gracia\ldots{} Nunca vi manos tan
puras como las que muestran la milagrosa Cinta, ni cabeza en cuyo
contorno brille con tan celestial resplandor la corona de estrellas.

Trabajillo me costó sacar a mi amada de la espléndida capilla. Por su
gusto se hubiera estado allí todo el día reza que reza, sin acordarse de
que hemos de alimentar nuestros cuerpos desmayados del insomnio.
Salimos, y por calles para mí desconocidas, risueñas, animadas del
hormigueo alegre de la vida tortosina, nos fuimos a la casa de Polonia,
quien nos recibió poco menos que con palio; tan satisfecha estaba de
tenernos en su compañía. Mi primera diligencia, después de tomar
chocolate con lucido acompañamiento de tiernos bollos, fue salir a
recoger mi maleta, y a despachar al tartanero de Ulldecona, breve
ocupación en que me guió el asistente de uno de los pupilos de
Polonia\ldots{} Ésta nos instaló en lo más alto de su vivienda, donde
estaríamos, según dijo, algo estrechitos, pero con preciosa
independencia, aislados del bullicio de la casa. A mi odalisca y a mí
nos agradó el aislamiento, y no nos molestó la estrechez, porque así
estábamos más juntos el uno del otro. Mi querencia de las comparaciones
me hizo ver en el palomar alto y recogido una reproducción fiel de aquel
otro en que anidé con la blanca \emph{Yohar}, por arte y gracia de
\emph{Mazaltob} y \emph{Simi}\ldots{}

Permitidme, oh nobles Marqueses, que guarde en mi mente y en mi corazón,
apartadas del descaro de las cosas escritas, la tarde de amor\ldots{} la
noche de intenso descanso, de un dormir hondo y dulce\ldots{}

\hypertarget{xxiv}{%
\chapter{XXIV}\label{xxiv}}

Acordaron Donata y Polonia que comeríamos en la cocina, pues aunque
somos huéspedes, nos consideramos de la familia. Este apartamiento fue
muy de mi gusto, y no porque nos molestaran los pupilos; al contrario,
en ellos encontramos afabilidad y cortesía. El Músico es un ángel; el
Capitán un aragonés de lo más corriente y francote que he visto en mi
vida; el Castrense (no son ya dos, sino uno) un señor picoteado de
viruelas, de mediana edad, un poco duro de semblante, pero sencillo y
cariñoso en su trato, persona excelente, si no me engañaba el primer
vistazo. Observo con gusto que mi Donata se afana desde hoy por ayudar a
su prima en los trajines domésticos. En la cocina están las dos tan
entretenidas, que da gusto verlas. Otra observación fugaz: Polonia es
guapa, frescachona; pero no llega ni con mucho a la clásica belleza
hispano-árabe de Donata-\emph{Erhimo}.

Un día más, y sigo observando. El Capellán consagra diariamente un
mediano rato al arreglo de las cuentas de Polonia. En un librito le va
poniendo el gasto, sin omitir lo más insignificante y menudo, y por otro
lado van los ingresos. Gracias a don Jesús Portela (que así se llama) la
simpática patrona lleva sus negocios con admirable claridad y limpieza.
No podrán decir lo propio las innumerables pupileras esparcidas por el
ancho mundo. Mi dominante espíritu de comparación háceme pensar en
Lucila y en el novio administrativo que le ha salido para enderezar su
existencia hacia las ordenadas esferas de la Economía Política y
Privada\ldots{} Otra cosa: no sé de dónde habrá sacado mi buen Capellán
que yo soy un gran teólogo, y que cuando llegue a Tarragona saldrá el
Arzobispo a recibirme como a un enviado del Papa, o poco menos. Esta
idea del buen Portela, me le pinta como un administrativo forrado de
inocencia paradisíaca.

Sigo observando y enterándome de todo: el Capitán se empeña en llevarme
a ver el Castillo, que desarrolla su imponente grandeza en los altos de
la ciudad. Me dejo llevar y querer, y en los baluartes, oficiales de
distintas armas se nos unen\ldots{} Me cuentan el suceso de la Rápita,
que aún no ha dejado de ser aquí la diaria comidilla de todas las bocas.
¿De qué se ha de hablar más que de la calaverada orteguista, del
estúpido desenlace de aquel drama político, el peor aderezado y
compuesto que nos ofrece nuestra Historia, primer teatro del mundo en
sediciones y pronunciamientos?

Reproduzco una noticia breve, fugaz nota recogida de un testigo
presencial, Teniente del Provincial de Tarragona: «Salimos de San
Carlos. Ignorábamos a dónde se nos llevaba. Esto fue el día 2. Hasta
entonces nada sospechábamos, o por mejor decir, ninguno de nosotros
sacaba del corazón su vaga sospecha\ldots{} Habíamos visto dos tartanas
que iban delante de las tropas a regular distancia. Cuando el General a
ellas se acercaba, se descubría con todo respeto y reverencia\ldots{} Ya
empezaba a correr un cierto run-run de boca en boca. Llegamos a un sitio
llamado Coll de Creu, donde se hizo alto para comer\ldots{} Formamos
pabellones, y los soldados se quitaron las mochilas. En la vanguardia se
sirvió la comida al General y a cinco o seis personas más, debajo de
unos árboles\ldots{} Yo no puedo referir lo que pasó\ldots{} sólo diré
que en nuestro batallón corrió de punta a punta una ráfaga de luz, de
inspiración; nos pusimos todos en pie, abandonando las raciones; sonó
toque de llamada; los soldados echaron mano a las mochilas. Nuestro
Teniente Coronel nos habló a gritos: «¡Hijos, vamos vendidos!\ldots{}
¡Viva Isabel II!» Yo no sé lo que pasó, vuelvo a decir. Sé que algunos
soldados señalaban una nube de polvo en que iba Ortega con cuatro más, a
galope tendido. Desaparecieron\ldots{} Los del Provincial de Lérida nos
contaron luego que a los desconocidos caballeros de la tartana les cogió
el pánico cuando estaban comiéndose un pavo que llevaban entre papeles.
Cada uno de ellos se arregló como pudo con un alón o pata, y comiendo
iban cuando arreó disparada la tartana, y se perdió también en nube de
polvo.»

No se abren aquí las bocas más que para decir algo del desgraciado
Ortega. Los que no hablan de su insensato alzamiento, hablan de su
captura. A Ortega encontramos en la sopa y en la \emph{escudella};
Ortega sale a relucir en toda charla de paseantes; Ortega, en la sala y
en la cocina. En la de Polonia estábamos cuando entró a encender un
cigarrillo en las brasas del fogón el castrense don Jesús Portela, y nos
contó cómo había sido capturado el General en su fuga\ldots{} Tan ciegos
estaban él y sus compañeros de locura, que en vez de correrse a la costa
en busca de un falucho que les llevara mares adentro, se metieron en el
corazón de España. No podían desechar la ilusión de que el país se
sublevaba por la Causa. Soñaban con el levantamiento general, con Madrid
convertido a la fe montemolinista. Siguiendo este fantasma, se
internaban de pueblo en pueblo, camino de su perdición. El hijo del
conde de Sobradiel, ayudante de Ortega, era un valiente soñador que
creía encontrar en cada pueblo lo que no encontraron en Tortosa\ldots{}
Todo su afán era llegar a Alcoriza, donde contaban con fantásticos
auxilios del Barón de la Linde\ldots{} Pero en Calanda se acabaron las
ilusiones: los fugitivos chocaron con un alcalde que los reconoció y los
puso debajo del recaudo de la Guardia civil\ldots{} Todo esto nos
refirió el Capellán, que acabó abominando del carlismo como ciudadano
consecuente que milita en la Unión liberal, y debe su posición a Posada
Herrera.

Pasa otro día, y se ensancha la esfera de mis amistades. Conozco y trato
a sinnúmero de oficiales de la guarnición y de los batallones que en mal
hora trajo de Baleares Ortega. Éste no tiene la cabeza buena, en
concepto de muchos, y sólo así se explican sus inauditas rarezas y actos
extravagantes. En Palma, cuando preparaba la desatinada expedición, iba
de taller en taller, vestido de paisano, con botas, vigilando la
compostura del armamento\ldots{} Pues al traerle prisionero desde
Calanda a Tortosa, los que le custodiaban sufrieron acerbas quejas y
reproches del desgraciado General, irritado de las incomodidades
inherentes a su triste situación. Pedía lo que no podían darle, y
reclamaba lo que en aquellos míseros pueblos no existía. Es hombre de
hábitos elegantes, hecho a los refinamientos del tocador. Le desesperaba
el no poder mudarse de ropa. En Alcañiz pidió un traje negro de pana, y
no hubo más remedio que hacérselo en breve tiempo. Vestir de negro, con
botas altas de charol, guantes color lila, era un atavío muy del gusto
de aquel hombre, a quien la conciencia de su buena figura y porte, y los
éxitos sociales, inclinaban a la presunción.

Temiendo un arrebato de locura o despecho, los guardianes del General no
le permitían afeitarse, con lo que movían mayores arrebatos de la
presunción. La idea de estar feo y poco galán sacaba de quicio al hombre
tanto como le irritaba su fracaso militar y político. Pero aún hubo de
ser más vivo el enojo del pobre Ortega cuando se le sirvió la comida sin
cuchillos ni tenedores, que esto es de rigor tratándose de presos en
quienes se supone con fundamento la demencia suicida. La porquería de
comer con los dedos le sublevaba; ponía el grito en el cielo; clamaba
contra sus verdugos; protestaba de su buena intención patriótica en la
empresa frustrada, y decía: «Yo haré saber a la Europa este bárbaro
tratamiento que se da a un General español, por el hecho de querer traer
a su patria la paz definitiva. Yo no soy carlista, no soy
absolutista\ldots{} quiero la fusión de las dos ramas, deseo ardiente de
todo español honrado\ldots{} Yo defiendo la causa \emph{fusionista}, y
por ella moriré, si así lo quieren mis enemigos.»

¡Infeliz hombre! Mimado de la sociedad y favorecido de las damas, su
buena figura y sus relaciones no habían tenido poca parte en los fáciles
adelantos de su carrera militar. Era un caso del \emph{señoritismo}
endiosado, que desvanecido con los triunfos sociales, acaba por creerse
un derecho y una fuerza. Fuerza ilusoria es, bomba de vidrio, fundida en
salones y tertulias, y que al salir disparada de estas esferas, se
estrella en mil cascos contra el primer muro que encuentra. ¿Verdad,
amigo Beramendi, que Ortega no es más que una víctima del
\emph{señoritismo}, y que éste debe ser atado con cintas de seda para
que nunca intente salir de los dorados espacios de la frivolidad al
campo de la acción?

El risueño vecindario de Tortosa se entristece con la visión del próximo
suplicio de Ortega. Empezó creyéndole criminal, y al fin le tiene por
más merecedor del manicomio que del patíbulo. La execración y burlas
injuriosas de los primeros días derivan rápidamente hacia la compasión.
Dulce, Capitán General de Cataluña, ha llegado a Tortosa reventando de
inflexibilidad. O'Donnell, desde África, ha dicho que no hay perdón, y
en Madrid, el blanco corazón de Isabel se pone frenos para no dar lugar
a la clemencia\ldots{} Cuando nos aseguró el Capitán que el fallo cruel
es inevitable, Donata y yo caímos en gran tristeza. Ortega no nos había
hecho ningún daño. Dicen que el daño grande lo ha hecho al país; pero
este perjuicio, si es cierto, se reparte por igual entre todos los
españoles, y la porción que a nosotros nos toca es inapreciable por su
pequeñez. Donata me dijo: «Vámonos a rezar a Nuestra Señora de la Cinta
para pedirle que haga lo que no quieren hacer Dulce en Tortosa,
O'Donnell en África y la Isabel en Madrid.» Y yo, que cada día me siento
más sumiso a la bella \emph{Erhimo}, digo, Donata, le respondí: «Recemos
a la Virgen para que entre la sentencia y el pecho de Ortega interponga
su Cinta milagrosa.»

En la capilla de la Virgen pasamos la tarde. Luego fuimos de paseo hacia
la puerta del Temple y el Astillero, y en nuestra conversación,
divagando lentamente, sentados al fin en un recuesto donde
contemplábamos la majestuosa corriente del río, surgió un pequeñísimo
punto de discordia que me ha hecho cavilar más de lo que yo quisiera.
Ello fue que, en el ardimiento de mi pasión, me arranqué a declarar que
es broma todo lo que he dicho de cantar misa, y que mi verdadera
vocación es el vivir laico en la turbulenta lucha del mundo. En mi amada
noté algo como desvanecimiento súbito de una ilusión. Largo rato
permaneció callada y seria, mirando las aguas del Ebro. Comprendí que mi
sinceridad no fue de su gusto. Lo que a mí me parecía muy natural,
perturbaba sus ideas. Vi ante mí, o entre mi persona y Donata, un mundo
extraño y anormal, que nunca pensé pudiera existir. La idea laica, con
su natural secuencia de matrimonio y vida regular, no era de su gusto.
¡Monstruoso fenómeno de psicología artificial, obra de las direcciones
equivocadas de la existencia!\ldots{}

Emprendimos el regreso con cierta esquivez el uno del otro, y sólo
hablamos de cosas insignificantes. La tristeza que el incidente descrito
me produjo, se desvaneció por la noche viendo a mi Donata como siempre
amorosa, quizás más que de ordinario, cual si quisiera desagraviarme.
Por último, se franqueó del modo más lisonjero para mí, diciéndome:
«\emph{Confusio} mío, dejo aparte mis gustos en lo tocante a tu carrera.
Seas tú lo que fueres, y cantes misa o dejes de cantarla, yo a ti
pertenezco para toda la vida, porque tú has querido tomarme, y yo darme
a ti con entera voluntad. Más te quiero cada día, y tan enamorada estoy
de ti y tan cautivada de tus prendas, que si me faltara tu cariño, me
faltaría también la vida.» Con ardientes caricias pagué el regocijo
intenso que me dio esta declaración, y ella la corroboró con nuevas
ternezas, terminando nuestro nuevo pacto de amor en el alto aposento
recogidito.

Repitió Donata al siguiente día sus oraciones a la Virgen de la Cinta
para que se apiadase de Ortega, trayéndole el indulto, ya que ablandar
no podía la dureza del Consejo. Éste fue de los que llaman ordinarios, y
de él formaba parte mi compañero de vivienda, el capitán Albuerne, quien
me contó que el pobre reo había protestado airadamente de no ser juzgado
por un Consejo de Generales, como por su calidad le correspondía. Habló
Ortega cuanto quiso, y leyó un escrito largo ante el adusto Tribunal;
mas no pudo obtener clemencia, y fue condenado a morir, tremendo fallo
que espeluzna. Dicen que esto es necesario para que subsista en su
inmaculada doncellez la Disciplina Militar, y en ello convendríamos
todos si no supiéramos que ya está bien violada de innumerables
seductores, aunque se guarda como un dogma el convencionalismo de que
substancialmente convivan la violación y la virginidad.

Despojado de la dignidad militar, Ortega no dejó de ser elegante en el
mayor aprieto de su vida y en los preparativos para su muerte, y encargó
un traje negro de paisano, según su particular gusto, bien ajustado, con
botas altas, y capa corta, que airosamente llevaba. Preso en el
Castillo, era el galán peripuesto, que se desvivía por que su presencia
y figura fueran admiradas de cuantos pudiesen verlas. Ante el Tribunal
como ante el público, su tribulación se aliviaba revistiéndola de cierta
elegancia melancólica. Su romanticismo no le abandonó un instante.
Después de sentenciado, soñaba con la evasión, con el indulto, emanado
del tierno corazón de Isabel; confiaba en las vehementes gestiones de la
Montijo y de su hija la Emperatriz Eugenia. Hacia estas empingorotadas
damas volvía mentalmente sus ojos, paseándose en la prisión con su
capita terciada, en gallardas actitudes.

Una noche más\ldots{} El castrense, vestido de hábitos, lo que fue para
mí una transfiguración de su persona, me propuso llevarme a ver a Ortega
en la capilla. No quise acompañarle. Las barbaries de la ley me llenan
de frío el corazón, incapacitándolo para execrar las de los malhechores.

\hypertarget{xxv}{%
\chapter{XXV}\label{xxv}}

Acompañé a Donata a la Santa Catedral. Quería mi pobre odalisca apurar
su piedad y sus oraciones para mover a la Virgen a un acto generoso por
las vías comunes, o por la vía del milagro si necesario fuese\ldots{}
Disparatada fue, según Donata, la conducta de Ortega; pero ¿cómo dudar
que en sus propósitos estaba el defender la religión? La Causa
últimamente abrazada por el infeliz General, no sólo predica las buenas
leyes, sino el reinado de la fe. Pues bien merecía Ortega que se le
mirara como soldado de Dios, salvándole de una muerte ignominiosa.

En estas reflexiones y en el afán de sus rezos la dejé, porque me
esperaba don Higinio, el Músico Mayor, con quien quedé citado en el café
para irnos a ver la ejecución. Es el prototipo de la franqueza
angelical, un hombre de esos que llamamos \emph{todo corazón}, mejor
será decir \emph{todo nervios}, porque no he visto otro más vivo, más
cambiante y movedizo en sus sensaciones, ni que mejor traduzca su
temperamento en un lenguaje que musicalmente puedo expresar con la
notación de \emph{presto agitato}. Por la mañana me contó que había
visitado a Ortega en la capilla, hablando con él un ratito. ¡Qué mal
rato pasó! Aunque se tiene por hombre de fibra, capaz de resistir las
más patéticas impresiones, no pudo eximirse de la pena del caso, ni
disimularla frente al reo. Éste, presumido y bien compuesto de rostro
hasta en los trances últimos de su vida, se mostraba ante los curiosos
sereno y grave, con una melancolía de buen tono. Había escrito a su
mujer una carta afectuosa; se había despedido de sus amigos y cómplices,
Elío y Cavero, y hablando del suplicio próximo, trazaba un programa de
él, comparándose con el bravo don Diego León, de cuyo heroísmo ante la
muerte quería ser imitador fiel. Como expresara su propósito de mandar
el fuego, el cura que le asistía le arguyó que es más cristiano el valor
callado que el jactancioso. Así pudo quitarle de la cabeza lo de dar las
voces de \emph{¡apunten, fuego!}, que revela el apego a las vanidades
terrestres en el momento de cambiarlas por la eternidad gloriosa.

Todo esto lo contó el Director de la banda a un su amigo que le
acompañaba y a mí. Era el amigo un hombrachón espigado, fuerte, como de
treinta años largos, con trazas de marino, por su traje azul y lo
curtido del rostro. ¿De qué habíamos de hablar sino de Ortega y de su
trágico fin? Como algo dijera yo de la descabellada expedición, el
desconocido me informó de que él había venido de Palma con el General
rebelde. «¿Es usted marino de guerra?» le pregunté. Y él: «No, señor: lo
fui. Cinco años estuve en el servicio. Después me metí en lo mercante;
pero no me amaño al mucho trabajo con poco provecho, y ahora me vuelvo a
los barcos del Rey.» Nada más me dijo, ni yo a él, porque nos apremiaba
lo que era motivo principal de nuestra reunión, y salimos los tres
camino de la explanada de Remolíns, donde nos dijeron que dejaría de
vivir el General Ortega. La verdad, no iba yo muy a gusto: desconfiaba
de mantenerme entero ante tal espectáculo, y la compasión ocupaba en mi
alma más espacio que la curiosidad. Pero don Higinio, en quien la
energía y animoso temple contrastaban con la pequeñez del cuerpo, se
burló de lo que llamaba mi pusilanimidad. Otro tanto hizo el hombre de
mar, declarando que conviene presenciar las desdichas ajenas para que no
nos cojan de nuevo las propias, y que cuando sepamos que arden las
barbas del vecino debemos ir a verlo, para aprender cómo hemos de poner
en remojo las nuestras.

Ya estaba la explanada de Remolíns llena de gente cuando llegamos; pero
gracias a los codazos y empujones con que se abrió camino en la masa
humana el hombre de mar, nos colocamos en sitio donde podíamos ver
cómodamente la función. Hubo un momento en que ésta se presentó en mi
mente como función trágica de teatro, que nos da la emoción patética y
compasiva. Al influjo del arte, llora uno y se aflige; mas todo ello es
como si nos pusiéramos máscara de espanto. Debajo están el rostro sereno
y la conciencia de que es mentira lo que vemos entre telones. Nos
retiramos alabando el arte del dramaturgo y el bello fingir de los
cómicos\ldots{} En esta ilusión de tragedia teatral permanecí mientras
estuvimos en espera del acto, y la causa de mi error no fue otra que el
aspecto del apretado público, y su bullicio de impaciente curiosidad.
Bullía y bufaba como una muchedumbre de parada militar, de teatro, de
toros\ldots{}

A la derecha, y a bastante distancia de lo que puedo llamar nuestro
palco, había una puerta de fortificación. Por allí salieron tropas a
caballo, después tropas a pie: traían al reo, y en el momento de verle,
mi teatral ilusión desapareció, sustituida por un sentimiento congojoso
de la realidad. La figura vestida de negro, con botas; el hombre
elegante y melancólico que yo me representaba en mi mente por lo que de
él me contaran, estaba vivo ante mí; y vivo, conducido entre dos
clérigos, fue llevado a un sitio frontero a mi palco. La distancia que
de él me separaba no era tal que pudiera escapar a mi vista la figura
aristocrática, la cabeza hermosa y descubierta, el rostro pálido, el
bigote rubio, la blanca frente, que al sol relucía como espejo\ldots{}

Sentí aflicción hondísima, terror, vértigo, cual si me viera al borde de
un abismo negro y sin fondo. Quise huir, mas ya no era posible: la
multitud me enclavijaba en su cuerpo macizo. En mi retina se estampó la
imagen del reo, calificado de traidor. Lo sería; mas a mí se apareció
revestido de todo el esplendor de la dignidad\ldots{} Cuando vi que se
apartaban de él los curas; que le dejaban solo, cruzado de brazos, sin
vendar los ojos, y que él miraba impávido los fusiles que pronto
apuntarían a su pecho, cerré los ojos\ldots{} No quería yo ver tal
ultraje a la Naturaleza. Mi temblor y el temblor de todos anunciaban un
cataclismo del mundo moral\ldots{} Repentino acceso de curiosidad me
hizo abrir los ojos. Fue en el mismo instante del tremendo disparar de
los fusiles. El cuerpo de Ortega saltó en rápida voltereta. Vi las
suelas de sus botas, como si patearan el espacio\ldots{}

El murmullo de la multitud acarició el cadáver como una onda con gemidos
de responso. ¡Oh iniquidad, baldón de la Naturaleza, bofetada y palos en
la propia persona de la Divinidad! ¡A las tres de la tarde, en un
espléndido día de Abril, cuando el sol alegra los campos, y la tierra
fecunda echa de sí para regalo del hombre toda la magnificencia de
flores y frutos, la ley nos ofrece su auto siniestro de la Fe jurídica y
militar, remedo de los sacrificios idolátricos! ¡Y se llama ley lo que
es contrario al sentimiento y a la razón; ley, la violación salvaje del
principio cristiano! ¿En qué te diferencias, ley matadora, de los
criminales que matan? En que revistes tu crimen de etiquetas y trámites,
y en que has sabido cohonestarlo con fórmulas hipócritas de moral falsa
y de religión contrahecha. Tan execrable eres tú, perversa ley, como tus
auxiliares, los hombres trajeados de negro, cuya misión en el patíbulo
es comprometer a Dios a que sancione la barbarie llamada pena de
muerte\ldots{} A mi delirio de furiosa protesta puso fin un triste
accidente que a mi lado ocurrió. Fue tan viva la congoja del pobre
músico don Higinio ante el terrible espectáculo, que todo el artificio
de su presumida entereza se vino al suelo, y lanzando un ¡ay! lastimero
cayó al suelo con un síncope. Con no poco trabajo lo sacó de entre los
pies de la multitud nuestro acompañante el gigantesco marino, y viéndolo
sin sentido se lo echó a la pela. Mujeres hubo a quienes pasó lo mismo;
mas no encontraron a un atleta que del oleaje tan gallardamente las
sacara.

Poco pesaba el Músico \emph{mayor}\ldots{} Véase por dónde vinieron a
interrumpir la convulsión trágica risas de sainete. Chiquillos vi, y aun
mujeres animosas, que hicieron gran fiesta de ver al don Higinio llevado
en brazos por el hombracho. Éste reía también, oyéndose llamar San
Cristóbal. Avanzando a paso de procesión, pudimos llegar a donde no nos
abrasaban los rayos del sol y se aclaraba la espesa multitud. Recobró su
sentido el músico, y tan sorprendido como avergonzado, se limpiaba el
sudor de su frente calva. «Es muy raro esto que me ha pasado---nos
decía.---No vayan ustedes a creer que fue susto: soy hombre de terrible
entereza\ldots{} ¡Pero tengo el estómago más canalla y más perro que
ustedes han visto!\ldots{} Esta mañana comí unos \emph{muscles} que me
trajeron de Ampolla, y sobre ellos bebí aguardiente\ldots{} Ya lo ven:
me han hecho daño\ldots{} Lo peor es que se me va la vista, y me
tiemblan las piernas\ldots{} Horrible ha sido el fusilamiento, ¿verdad,
amigos?\ldots{} Entiendo yo que la pena de muerte es una
brutalidad\ldots{} es un asesinato\ldots{} También lo es comer
\emph{muscles} y encima aguardiente\ldots{} verdadero asesinato.»

Con menos trastorno exterior, quizás la impresión mía fue más honda y
lacerante que la del buen músico. Dejando a éste en casa, me fui a la
Catedral en busca de Donata, a quien vi consternada, en un corrillo de
clérigos y devotas, condoliéndose de que no hubiese venido el indulto.
Bien claramente echó de ver mi amada que el trágico suplicio me había
descompuesto. Más que condolido del triste fin de Ortega, me mostré
indignado de la hipocresía de las leyes, que sacrifican a un hombre en
el ara de la Disciplina Militar, mil veces violada y escarnecida. La
traición resultó más ridícula que tremebunda, sin ocasionar muertes. ¿A
qué tanto rigor contra un soldado iluso a quien debíamos acusar
principalmente de necedad inofensiva? «Ya ves, ya has visto---dije a
Donata---de qué te han valido tus rezos, y cuán indiferente es la
Divinidad a nuestras miserias y dolores. El General muerto tenía mujer,
tenía hijos, que habrán rezado tanto como tú, y con más afligido
corazón\ldots{} ¡Valiente caso les han hecho! Y es que la proyección de
la Divinidad sobre nosotros en forma de culto, es tan falsa como la otra
proyección de la Divinidad en forma de justicia. Todo es mentiroso, todo
compuesto para el servicio exclusivo de un grupo de poderosos, que se
han alzado con el mundo moral y con el mundo físico\ldots{} ¡Ay, Donata,
repugnancia y miedo me da esta oligarquía, formada con la triple casta
de soldados, legistas y curas!\ldots{} ¡Y dicen que así ha de ser; que
no existe mejor sistema; que en la majestad de Dios se apoya este
armadijo!\ldots{} ¡Paciencia! Cantemos las glorias de los que nos
esclavizan y atormentan.»

Presumo que Donata no entendió mis ideas, expresadas con vaguedad
febril\ldots{} Agarrándose a lo que afirmé de la ineficacia de sus
rezos, me dijo: «La Virgen no ha querido salvarle\ldots{} bien claro
está\ldots{} no ha querido, porque Dios y la Virgen habrán determinado
que la Causa tenga mártires.»

¡Ay, con qué pena oí este brutal concepto en boca de la mujer tan
tiernamente amada! Quizás debí callarme, respetando un error nacido de
la propia sencillez y rusticidad de la guapa moza; pero no lo hice, y
movido de un ímpetu sectario y de mi locuacidad discursista, solté un
sinfín de acusaciones y diatribas contra la Causa y sus príncipes,
contra sus caudillos y tropas. Donata me oía consternada, poniéndose ya
lívida, ya roja, y haciendo con su linda boca graciosas muequecitas de
ira, de burla, de desdén. Sin duda dije mil simplezas, y arrebatado de
mi propensión a la vana oratoria, endilgué pedanterías hinchadas, de
esas que comúnmente no entran en el cerebro de una mujer. No la
convencí, no: en la rudeza de sus ideas macizas, recibidas de la
tradición, se estrellaban mis razonamientos como la ola en la peña.
Díjele al fin que el vivo ejemplo y símbolo de la Causa lo tenía en el
que fue su amo y sultán, de cuyo brutal poder habíala yo librado con
ayuda de Dios. La monstruosa Causa se personificaba en el monstruo
llamado \emph{don Juanondón}.

Balbuciente salió Donata a la defensa del Arcipreste, del cual dijo que
si estaba cargado de faltas, también poseía virtud y mérito grandes.
«No, no, \emph{Confusio}; no seas injusto. Don Juan es valiente, es
piadoso\ldots{} Piedad y valor tiene, según lo requiere la necesidad. Tú
no le conoces bien, y hablas como un papagayo que no sabe lo que dice.
Que don Juan peque, no significa que deje de servir a Dios cuando es
caso de servirle\ldots{} Y como talento macho, con luces para entender
de cuanto hay, ¿quién se iguala con él? Yo digo que donde está don Juan,
que se quiten todos\ldots{} Hombres así debieran ponerse a gobernar la
Nación\ldots{} ¡Qué derechos andarían los españoles con un tío como el
Arcipreste!\ldots{} ¡Bien les sentaría la mano, bien!\ldots{} y ellos,
los muy cuitadicos, agradeciéndolo, Juan, agradeciéndolo.»

Me acometió un reír convulsivo\ldots{} Hablar quise, y de mi boca no
salía más que la risa desbordada y frenética. Donata se asustó al verme,
y cuanto más carantoñas y mimos me hacía para calmarme, más locamente me
disparaba yo en aquella infernal risa. Acudió primero Polonia; después
el bondadoso Castrense, que además de administrativo tenía sus puntos de
médico: en mí vio un fuerte ataque nervioso, y me ordenó cenar todo lo
fuerte que pudiese para combatir la debilidad. Negueme a tomar alimento;
me hicieron acostar; trajéronme aguas cocidas, infusiones en las cuales
echó don Jesús no sé qué polvillos\ldots{} Lentamente se me sosegó el
mal de risa que me sacudía los hipocondrios y me quebrantaba la cintura.

Solo con mi moza, ésta procuraba con blando arrullo y expresiones suaves
incitarme al sueño. Yo quería dormir; mas algo había en la estancia que
me despabilaba, tentándome al furor y a la risa. Veréis lo que era.
Algunos días sacaba Donata de mi maleta las prendas de clérigo, sotana y
bonete, que en mi equipaje con socarrona intención pusisteis, ¡oh
insignes Beramendi y Tarfe! Estimaba mi odalisca en mucho aquellos
negros atavíos; cuidaba de ventilarlos de tiempo en tiempo para que no
se picase la tela, y después de cepillarlos con esmero y quitarles el
polvo, y arreglar con la aguja algún deterioro que en ellos notase, los
guardaba de nuevo respetuosamente. Pues aquella noche estaban colgadas
frente a mí las feas vestiduras que debían servirme de disfraz en la
farsa de mi viaje. En ellas clavé mis ojos espantados, y cuando Donata
me incitaba a dormir, yo le dije: «No me deja, no me deja dormir\ldots»

---¿Qué tienes, Juan, qué miras?

---Eso, Donata; eso no me deja dormir. Quítalo y dormiré. Si no te
decides a quemar ese horror, esa funda negra, y el nefando gorro de
cuatro picos, guárdalos, vida mía, para que yo pueda coger el sueño.
Sacerdote quiero ser; pero nunca pondré sobre mi cuerpo ese traje de
ajusticiado o de ajusticiante.

Solícita, me libró Donata de la vista de aquellos lúgubres objetos; y
hablando de religión, de la misericordia divina, de la redención de
nuestros pecados por el arrepentimiento, del amor a todas las criaturas
sin distinción de castas, clase ni estado, de lo bueno que es Dios y de
la maternal indulgencia de la Santísima Virgen, me quedé dormido como un
ángel.

\hypertarget{xxvi}{%
\chapter{XXVI}\label{xxvi}}

Desperté sosegado y sin ningunas ganas de reír. Sentada junto al lecho,
había Donata recostado en éste su cabeza y parecía dormir profundamente.
Las ideas que me asaltaron en aquel rato de sedación suave, fueron
desconsoladoras. Pensé que me había dejado llevar de la imaginación al
encarecer desmedidamente la hermosura de Donata. Aunque es muy propio de
poetas sublimar el semblante, el color y las líneas corpóreas de la
mujer amada, entiendo que hice un derroche abusivo de comparaciones
poniendo el cielo en los ojos de la mía, en su boca todas las gracias y
en su cuerpo no sé qué ideales paganos de perfectísima gentileza. La
miré bien, dormida, y si en efecto, no puedo menos de reconocer que es
una linda hembra, también reconozco que hay no poca distancia desde sus
atractivos a la perfección de nuestra madre Eva, o a la de las diosas
gentílicas, con quienes en mis arrebatos de amor propio la he comparado.
Me propuse rectificar en la primera ocasión oportuna aquel juicio mío
inflado de hipérboles optimistas, y así lo hago, manifestando a los
señores de Beramendi que rebajen un poquito mi poética descripción de la
\emph{Erhimo} aragonesa.

Pues de las observaciones que aquella noche hice ante Donata dormida,
pasé a revolver en mi mente recuerdos de lo que ella me había contado
días antes referentes a su niñez y crianza. En Alcoriza, tierra de
Teruel, nació la que por especiales motivos románticos llamo mi
odalisca, y fue su padre el sacristán de la iglesia principal del
pueblo. Con sutil discreción, me dijo que el sujeto que ante el mundo se
llamaba su padre le tenía ella por tal en concepto putativo, y que el
verdadero progenitor era persona de más categoría. A la muerte del
sacristán (acaecida cuando Donata no pasaba de los cinco años), quedó su
mujer de sacristana, porque así lo dispuso el generoso párroco, hombre
de opinión y de buena presencia, y en todo lo que no fuera servicio
litúrgico de altar desempeñaba la viuda las mismas faenas del difunto.
Ved aquí cómo creció la chiquilla en aquella vida sacristanesca. A su
madre ayudaba en el barrido de la iglesia y capillas, en alimentar las
lámparas de aceite, en lavar las imágenes, y desnudarlas o vestirlas
cuando era menester, en disponer los altares para el diario y las
funciones mayores. De aquí que estuviera la odalisca tan versada en las
cosas del culto y fábrica, y en los ritos de cada festividad.

Así llegó a ser mocita. Me contó que miraba mucho por ella el buen cura,
y que la guiaba paternalmente por los caminos de la virtud y de la
honestidad, dándole además la instrucción rudimentaria de leer,
escribir, y contar un poquito. Por desgracia, al cumplir Donata los diez
y ocho abriles, falleció el bendito señor, dejándola sin más amparo que
el de la madre; y menos mal que ambas continuaron en el albergue y
oficio sacristanil, por tolerancia del nuevo cura. Era éste un bravo
mocetón, furibundo carlista, gran cazador, rumboso, jovial. Sin duda no
tuvo a la moza por saco de paja, porque quiso estrecharla más en su
servicio y compañía, llevándola a la propia vivienda. Luchó la madre
contra este propósito del superior jerárquico, y de la lucha vinieron
disgustos, y la intervención de otro curita joven, de un próximo pueblo.
En resolución, la sacristana hubo de poner en salvo de aquellas disputas
a su querida hija, y no halló medio mejor que remesarla al señor
Arcipreste don Juan, varón de grandísimo crédito y autoridad en aquel
territorio. ¿Fue remitida Donata como alumna o pupila de un colegio, o
como criatura torcida que necesita de un maestro y corrector que la
enderece? Esto no supo decírmelo mi amada. Ya me lo explicaría mejor al
proseguir la historia de su vida\ldots{} ¡triste vida desarrollada en un
medio sombrío, sotanesco!

Las reflexiones que me sugirió el \emph{ensayo biográfico} de Donata,
reproducido en mi memoria, las contaré cuando Dios quiera. Hoy tengo que
reanudar el cuento de aquella noche, diciendo que Donata despertó cuando
yo me hallaba en lo más intrincado de mis reflexiones. La pobrecilla
mostró un interés muy vivo por mi descanso. Quería que durmiese más, o
que en su compañía, charlando de íntimas y dulces cosas, repusiese mi
espíritu del susto de la tragedia. Con buen acuerdo, nada me habló de la
monstruosa ficción legal política y religiosa que levantaba en mi alma
oleaje de terror y de ira. Lo que me dijo fue para mí de gran alivio; en
sus palabras vi la expresión fiel del pensamiento. La esclava
\emph{Erhimo}, redimida por mí, puede tener, y tiene sin duda en su
mente, todo el mazorral tenebroso que daba de sí la singular crianza que
me contó ella misma; pero es buena, hay en su alma un fondo de rectitud
y ternura, sobre el cual puede fundarse una regeneración espiritual con
auxilio del tiempo.

Reproduzco sus expresiones, que creo interesantes: «Mira,
\emph{Confusio}, mientras tú dormías, yo he llorado\ldots{} he llorado
como San Pedro cuando, al oír cantar el gallo, cayó en la cuenta de que
había negado a su Maestro. A ti, que eres mi maestro, te he negado yo
diciéndote lo que no debía decirte. ¡Ay!, yo no debí defender al
Arcipreste, ni meterme en músicas de si la Causa es mejor o peor que
otras Causas\ldots{} Verás por qué estuve yo tan impertinente y tan
fuera de lo que soy. En la Catedral me arrimé a un gran corrillo que
formaron en la capilla de San Rufo unos señores sacerdotes y media
docena de mujeres, o señoras, que todo podían ser, de las que están allí
mañana y tarde engolfadas en la santidad. Sea esto santidad verdadera, o
\emph{turris burris}, como dice \emph{don Juanondón}, ello es que en mi
pobre cabeza metieron todo lo que iban diciendo, y cuando me recogiste
venía yo trastornadica\ldots»

---Tu principal defecto---le dije---es que el último que llega te hace
suya, Donata.

---Pues tenme siempre en tu influencia---respondió besándome las
manos,---y si me quieres como yo a ti, \emph{Confusio} mío, no me dejes
ni un momento de tu poder\ldots{} Yo te pido perdón de lo que pensé y
dije, y no quiero ser sino al modo que a ti te plazca\ldots{} Esclava
soy desde que nací, y de unos a otros dueños he pasado; ahora soy
esclava tuya. No me has comprado con dinero, sino con tu amor, y en el
amor tuyo quiero vivir siempre.

Bastaron estas tiernas declaraciones, que del corazón le salían en
hermoso torrente, para que yo me calmase de aquel estado de malquerencia
y enojo de todas las cosas. A tal estado llegué por el terror de la
ejecución de Ortega, que en mi espíritu se desató el fiero pesimismo.
Ver morir a un hombre en aquella forma de glacial justicia sin entrañas,
era bastante motivo para que se ajaran ante mis ojos todas las formas
del mundo que me rodea, entre ellas la misma Donata, cuya belleza rebajé
despiadado con cierto furor iconoclasta. Mas lo que a la madrugada me
dijo, y el hechizo de su ternura y arrepentimiento, la repusieron en mi
adoración, y si no recobró la ideal hermosura de los días románticos,
quedó restaurada lo suficiente para ser una hembra muy distante de la
vulgaridad.

Por la mañana subió a mi camaranchón el castrense don Jesús. Mis
primeras palabras, contestando a su saludo, fueron para suplicarle que
no me hablase de la \emph{intentona}, ni de ningún tema que con la cosa
pública tuviera relación. «¡Pues, hijo, no está usted poco incomunicado
con el mundo!---me dijo risueño.---¿De modo que no quiere saber que se
ha encontrado la pista del falso Monarca y del falso Príncipe?\ldots{}
Sí\ldots{} ya saben hasta los perros que Montemolín y su hermano están
en Ulldecona, muy agazapaditos en un convento de monjas\ldots» No pude
sustraerme al interés de estas noticias. Sintiéndome gradualmente
animado, me vestí y arreglé para sacudir la tristeza y volver a la vida
normal\ldots{} Poco después estaba yo en la cocina, donde supe por
Polonia que don Higinio había convidado a comer a su amigo, el marino
atlético, que en brazos lo sacó de las apreturas del gentío momentos
después de la ejecución. Ambos estaban en el comedor con el Capitán,
éste leyendo periódicos de Madrid, don Higinio haciendo cigarros de
papel en una maquinilla.

Allá me fui tratando de dar a mi espíritu algún esparcimiento, y saludé
con afecto al marino, deseando mostrarle mi simpatía. Al verle en pie,
para corresponder a mi saludo, admiré su arrogante figura y la ruda
belleza del rostro en que habían escrito sus rigores el viento y el sol.
«Paréceme usted un gladiador de mar---le dije,---y tan lucida y airosa
es su facha, señor mío, que le dan a uno ganas de llamarle
\emph{Neptuno.»} A mis galanterías dio esta contestación, que me dejó
atónito: «No me llamo \emph{Neptuno,} sino Diego Ansúrez, para servir a
ustedes.»

Con ardientes expresiones mías estalló la \emph{anagnórisis}, que así
llaman los retóricos al reconocimiento de personajes. Era de los míos.
No pude decirle: «¡oh padre, oh hermano!» como es de cajón en las
\emph{anagnórisis}; pero le dije: «Soy muy amigo de su padre de usted,
Jerónimo Ansúrez\ldots{} de su hermana Lucila, que es, mejorando lo
presente (por allí andaba Polonia trasteando en el aparador), la mujer
más guapa del mundo; de su hermano Leoncio, armero habilísimo; de su
hermano \emph{Ruy}, pensionado en Bélgica por el marqués de Beramendi
para perfeccionarse en la música, y por fin, conozco y estimo
grandemente a su glorioso hermano Gonzalo, que de España se pasó a
Marruecos y de Cristo a Mahoma, y hoy es un caballero poderoso llamado
\emph{El Nasiry.»}

No menos asombrado que yo, el Ansúrez de mar me pidió con interés febril
noticias de todo el familiaje que nombré. De Lucila sabe que ha
enviudado y que posee hacienda pingüe; de su padre recibió carta
hallándose en Vinaroz en el mes de diciembre último; con Gonzalo no se
cartea; pero sabe por Jerónimo que está bueno y vive en grande, con
sinfín de mujeres, y valimiento en la corte del Sultán. Dile yo cuenta
de mi amistad con \emph{El Nasiry}, y de lo que es y supone en aquel
Imperio, quedando él y los que nos escuchaban maravillados de tan
portentosa metamorfosis. Don Higinio, el Capitán y el Castrense mismo,
no ocultaban sus ganas de vestirse a lo moro, de hablar el árabe, de
tener provisión de hermosos caballos y un rebaño de lindas mujeres
sumisas. Buena cosa es la poligamia, matrimonio múltiple sin
suegras\ldots{} Después de responder a las preguntas del celtíbero de
mar, tocome preguntar a mí, y lo hice pidiéndole informes de su hermano
Gil o \emph{Egidius}, que vaga por estas tierras. Contestome que,
gracias a Dios, no anda ya Gil en trotes de bandolero: de otras
granjerías vive, no muy honradas que digamos, pero menos expuestas a dar
contra la justicia y a tropezar con el presidio.

«Por estos pueblos de la costa andaba con el compañero valenciano que le
ha enseñado esa industria---nos dijo.---En Hospitalet nos encontramos un
día, y le eché los tiempos\ldots: «¡Ah, tunante, si no te marchas de
esta tierra donde te conocen y puedes ser descubierto, yo te haré salir
a puñetazos!» Pasaron a Falset; de allí al Priorato, donde ganaron mucho
dinero, según oí, y ahora están hacia Mequinenza sacando todo el jugo a
su negocio\ldots{} Veo que están ustedes llenos de curiosidad por saber
en qué negocio trabajan esos pillos, y van a quedar satisfechos sin
demora. Mi hermano Gil es agudo como el hambre, vivo como la pólvora, de
rostro muy moreno, el labio un poco grueso, los ojos como endrinas. Con
un gorro encarnado, unas bragas azules, chaquetón o balandrán con
botones de moneditas y adorno dorado, se hace un empaque como el de esos
griegos o turcos que vemos en los muelles de Marsella y de Génova. En
los puertos levantinos aprendió el valenciano la industria que luego
enseñó a Gil, enseñándole también a mascullar la lengua turquesca o
tunecina que habla toda la pillería marinera del Mediterráneo. ¡Y qué
industria se traen los caballeros! Ello es vender cositas piadosas de
Tierra Santa, que llevan en un carro grande como una casa, donde viven y
hacen su comida, con lo que, a más de darse mucho tono, ahorran el gasto
de posada\ldots{} En cuanto llegan a un lugar, paran el coche en medio
de la plaza, y con grandes voces, en catalán o castellano chapurrado,
según los pueblos, llaman a la gente, y mi hermano, que tiene gran
despejo para sermonear, larga una plática pregonando la santidad y la
virtud de la mercancía. Acuden las mujeres como moscas, oyen aquellos
disparates, y ya las tienen ustedes trastornadas. La ignorancia, el poco
seso y la beatería caen en el garlito; empieza el compra y vende, y
antes se cansarán ellos de coger dinero que ellas de dárselo por las
baratijas milagrosas\ldots{} ¡Y que no es floja la tarifa de precios!
Las piedrecitas del Monte Sinaí, donde Dios le dio a Moisés las Tablas,
se venden al peso, por adarmes, y valen dos, dos y medio, y tres reales:
en relicario con cristal, valen seis y ocho reales. Las botellas de agua
del Jordán, para lavar los ojos enfermos, u otra parte del cuerpo en
algún caso, varían, según el tamaño, de siete a doce reales, y lo mismo
los rosarios de huesos de aceitunas del Getsemaní. Las hierbas del mismo
huerto son a precios convencionales, y quien las quiera del propio sitio
en que posó sus pies y rodillas Nuestro Señor, ya tendrá que pagar un
pico. Las que están cogidas en el ruedo de aquel sitio sagrado, van
valiendo menos, conforme se alarga la distancia; y las llamadas rosas de
Jericó, que son al modo de unas escobitas para rociar agua bendita, se
pagan caras, pues es cosa que estiman mucho las embarazadas, y mujeres
hay tan ciegas de fanatismo, que no paren a gusto si no les dan la
rosa\ldots{} Para el completo engaño de la gente, llevan esos pillos
testimoniales de cada cosa: son papeles escritos en arábigo, y
traducidos al español por un monje que acredita la procedencia del
género, y luego firman y dan fe priores, abades, y hasta cónsules
mismamente. Todo es falso; pero tan bien apañado, que la filfa parece
verdad: las mujeres enloquecen, los hombres aflojan los cuartos, los
curas bendicen, los alcaldes toleran, y los malditísimos charlatanes se
van a otro pueblo cargados de dinero, sin más trabajo que ir recogiendo
por el camino las piedrecitas del Monte Sinaí.»

\hypertarget{xxvii}{%
\chapter{XXVII}\label{xxvii}}

«¡Si serán listos esos sinvergüenzas, que me han engañado a
mí!---exclamó el Capellán, dando un golpe en la mesa;---a mí mismo,
señores, que siendo, como soy, católico ferviente, no creo en
milagrerías. Ello fue en Gandesa, cuando servía yo en el Provincial de
Teruel. Una patrona que allí tuve, y cuyo nombre no hace al caso, se
emperró en que le llevara la rosa de Jericó: estaba la pobre en meses
mayores. Llegaron por allí esos tunantes. Me acuerdo de verles en el
pórtico de la iglesia, donde el cura les hacía el artículo, y a todos
recomendaba que se proveyesen de aquellas porquerías. Llegué yo a
comprar la rosa, porque la patrona no me dejaba vivir. ¡Maldita
casualidad! Las rosas se habían concluido; pero me ofrecieron las
\emph{hojitas del propio lugar en que estuvo el Señor orando}\ldots{} Yo
no quería\ldots{} O rosas, o nada. Pero los mercachifles y el cura mismo
me querían hacer tragar las \emph{hojitas}, diciéndome que la eficacia
era tal, que no había parto desgraciado con semejante droga. Total, que
caí en la trampa: ¡veinticuatro reales me sacaron por unas hojuelas
arrugadas, con las que no se podría hacer un cigarrillo de
papel!\ldots{} ¡Lástima de dinero! La patrona se murió de sobreparto:
Dios la tenga en gloria\ldots{} Meses después, me encontré al cura que
había tenido su parte en aquel timo, y le dije: «Oiga usted, so
farsante: tiene usted que darme un duro y una peseta que con su garantía
me sacaron los ladrones aquellos de Tierra Santa.» Y él se echó a reír,
y convidándome a refrescar, me dijo que a él le habían sacado mucho más,
pues por unas botellas de agua del Jordán para curarle los ojos a su
sobrina, las cuales valían tres duros, les arreó media onza, y ellos, al
darle la vuelta, le encajaron un doblón de a cuatro\ldots{} más falso
que Judas\ldots{} «Y la sobrina, ¿curó de los ojos?\ldots» «Sí, señor,
curó\ldots{} El agua no era falsa: la tengo por legítima del Jordán. Aún
me queda un poco: se la ofrezco a usted para curarse ese grano que tiene
en la nariz.» Le mandé a la porra, y no le he visto más.»

La graciosa historia de los vendedores de santas bagatelas, y el
incidente que contó el capellán, nos divirtieron hasta la hora de comer.
Tanta simpatía me inspiró el Ansúrez acuático, que por disfrutar de su
grata conversación me fui por la tarde al café con don Higinio y el
Capitán. Reunidos en amena tertulia, nos contó Diego lances peregrinos
de su vida de navegante; luego nos dijo que posee un buen falucho, en el
cual saldrá dentro de unos días con carga para distintos puertos de la
costa, rindiendo viaje en Cartagena, donde dejará el barco a un compadre
suyo, y pedirá reenganche en la Marina de guerra. De lo mucho que habló
el hombre de mar, no he podido colegir que sea casado, aunque sin duda
lleva mujer consigo. Como yo le manifestara deseos de hacer un viajecito
por la costa para ver mundo y esparcir los pensamientos, me invitó a
navegar en su compañía de aquí a Cartagena. Si por el momento decliné
muy agradecido la invitación, en el curso de la tarde, por inesperados
sucesos, me sentí muy inclinado a no rechazar cualquier proporción de
viaje que se nos presentara.

Ved lo que ocurría: llegué a mi casa con objeto de recoger a Donata para
dar un paseo, y a quien primero vi fue a Polonia con una taza en la
mano. «Voy a darle a ésa un poco de tila---me dijo.---Se nos ha puesto
malucha.» Encontré a mi pobre odalisca demudada, llorosa. Con trémula
voz me dijo: «¡Ay, mi \emph{Confusio}, qué ganas tenía de que vinieras!
¿No sabes lo que pasa? Olegaria está en Tortosa: Polonia la ha visto. Ya
sabes lo mala que es\ldots{} Que te cuente Polonia lo que le han
dicho\ldots{} Corre por la plaza el rumor de que el Arcipreste está aquí
también, disfrazado de payés\ldots{} y ha venido\ldots{} ya puedes
suponerlo\ldots{} A Polonia le han dicho\ldots{} que te lo
cuente\ldots{} le han dicho que ni tú ni yo nos reiremos de \emph{don
Juanondón}\ldots{} Yo tengo un miedo horrible\ldots{} Cuando ésta me
dijo que vio a Olegaria en los porches de la plaza, creí morirme\ldots{}
Juanico mío, no me dejes un momento sola\ldots{} A ti y a mí nos
matarán. Lo que te dije: no nos perdonan lo que hemos hecho\ldots{}
¡Fugarme de su casa!\ldots{} ¡sacarme tú de su casa!\ldots{} Polonia,
\emph{Confusio}, escóndanme bien\ldots{} discurran cómo hemos de
escaparnos a lugar más seguro\ldots{} lejos, lejos\ldots»

Dejando para después el discutir si debíamos o no marcharnos a un lugar
más distante de la esfera de acción del Arcipreste, Polonia y yo
procuramos expulsar del cerebro de mi aragonesa los pensamientos
terroríficos que en él se habían metido. Cuando nos quedamos solos,
Donata se estrechó más contra mí, oprimiendo mi cuerpo con un abrazo
forzudo, y me dijo: «Tuya soy, tuya me hiciste por amor, y a ti me pego,
y no habrá quien de ti me separe\ldots{} ¿Te acuerdas de lo que hablamos
en la tartana viniendo de Ulldecona? Tú me preguntabas si el Arcipreste
es bueno o es malo, y yo no sabía qué contestarte\ldots{} Ahora te digo
que es malo, o que está en la vena de volverse demonio. ¿No te contó él
lo que hizo con el teniente que le quitó a Fabiana? Pues lo mismo querrá
hacer contigo\ldots{} ¡Qué horror! Vámonos, vámonos pronto de
aquí\ldots{} ¿Permitirá la Virgen de la Cinta que ese hombre se vengue
de ti por haberme robado y de mí por dejarme robar? No: la Señora no lo
permitirá. Yo le diré a la Señora que don Juan no merecía mi
constancia\ldots{} Yo he pecado\ldots{} él más\ldots{} él es, como quien
dice, monstruo, y su casa\ldots{} como eso que me contaste de los
harenes\ldots{} ¿no se llaman así?\ldots{} Te diré una cosa, y también
he de decírsela a la Virgen de la Cinta. Don Juan me compró a mí por mil
quinientos reales\ldots{} No te asombres. Es como te lo cuento: mil
quinientos reales dio por mí\ldots{} Mi pobre madre necesitaba la
cantidad, porque le habían embargado el huerto de la Diezma, única
hacienda que teníamos\ldots{} Y ello fue porque mi padre dejó una deuda
que al principio era de poca cuenta, pero crecía, crecía, año tras año,
como una mala hierba que se corta, mas no se arranca. Para librarse de
esta trampa empeñó mi madre la Diezma\ldots{} Mi prima Polonia, que
vivió con nosotros muchos años en la sacristía de Alcoriza, podrá
contarte las fatigas que pasó mi madre. Pidió al rico don Juan que le
prestase dinero para el desempeño de la Diezma, y no quiso dárselo. Lo
que él decía: «Estoy harto de hacer beneficios. Saco a estos pobres de
la miseria, y en la miseria se vuelven a meter.» Y yo digo: «No tienen
la culpa los pobres, sino la miseria de los pueblos, que es mayor que
toda caridad\ldots» Pues nada: don Juan iba todos los años a Alcoriza,
donde tiene tierras muchas\ldots{} Mi madre le daba matraca, y él que
no, que no. Me parece que le oigo\ldots{} Al fin\ldots{} el año pasado
no, el otro\ldots{} a poco de salir de allí el Arcipreste para
Ulldecona, mi madre, desesperada, discurrió ofrecerme a mí por el
dinero. Un arriero, apodado \emph{Mañas}, fue quien trajo y llevó los
recaditos para el arreglo del negocio, y ese \emph{Mañas}, en el mes de
Octubre, no en el último Octubre, sino en el de más atrás, me trajo a
mí, y llevó a mi madre los mil quinientos reales\ldots»

La pena y bochorno de estas revelaciones me hicieron enmudecer. Antes
que Donata me refiriese su \emph{caso}, había yo visto que en España
tenemos esclavitud mal encubierta de formas legales o de sociales
artificios. «¡Y este hombre---prosiguió ella---se atreve a disputar su
esclava al que me ha comprado por amor, no por dinero!\ldots{} También
te digo que don Juan no tomaría venganza de ti y de mí si esa perra de
Olegaria no le pusiera en el disparadero con sus arrumacos\ldots{}
porque él\ldots{} vuelvo a decírtelo\ldots{} enteramente malo no
es\ldots{} Tú lo comprendes, Juanico\ldots{} Dentro de él andan a la
greña los ángeles y los demonios.»

De esta conversación surgió la idea de aprovechar la oferta de Ansúrez
para buscar refugio en pueblo más distante. A la siguiente mañana,
anduve en persecución del hombre de mar, sin poder dar con él. Supe al
fin, por un posadero de la calle de Tablas Viejas, que había bajado a la
villa de Amposta, donde tenía su embarcación. Acompañome el Músico Mayor
en las últimas vueltas que di por la ciudad, no cuidando de recatarme,
sino de afrontar la presencia del Arcipreste, si acaso con él me topaba.
Por cierto que don Higinio, una vez pasada la gran congoja del
fusilamiento, se volvió a revestir de falsa entereza, y no hablaba de
otra cosa que del suceso trágico. Todo su empeño era presumir de haberlo
visto mejor que yo, y poner reparos a la descripción que yo hacía. Según
mi amigo, no eran de color lila, sino de color de paja, los guantes que
Ortega llevó al cuadro. Me porfiaba que la levita no era negra, sino
azul, y la capa, un capote de caballería\ldots{} Sobre tales pormenores
disputamos en la mesa del café, y la intervención y juicios de otros
amigos, en vez de aclarar los hechos, más los obscurecían y
embrollaban\ldots{} Y es que estos espectáculos siniestros, iluminados
por el relámpago de nuestra curiosidad terrorífica, no impresionan con
igual forma y colorido la retina de cada espectador. Un soldado del
piquete que hizo fuego sobre el General, nos dijo que éste llevaba una
casaca roja\ldots{} ¿Sabéis lo que motivó este error del soldado,
haciéndole ver tanto rojo? La cruz de Calatrava que Ortega llevaba en su
pecho.

Antes de anochecer, Polonia nos llevó noticias que rectificaban las que
habían consternado a la pobre Donata. Muy revuelta estaba Ulldecona con
las diligencias que hacía la tropa para encontrar a los Príncipes
escondidos. El Brigadier Ballesteros, hombre templado muy atento a su
obligación, destituyó al Ayuntamiento, que ha sido el primer amparador
de carlistas, y metió mano a todos los cabecillas de aquellos contornos.
En el casco de la villa fue registrada la casa del Arcipreste, que
escapó antes que entrara la Guardia civil. De las innumerables amas y
sobrinas, algunas huyeron con su señor; otras volaron por su cuenta, y
alguna se quedó al amparo de los propios guardias, que era lo más
seguro. El Arcipreste había ido a parar a la Cenia, según unos; otros
creían haberle visto camino de los Alfaques\ldots{} Tranquilizaron a
Donata estas nuevas, pues si eran verídicas, ya no debíamos temer la
presencia de don Juan en Tortosa. Mas como ni aun así podíamos estar
libres de inquietud, uno y otro, al cabo de mucho charlar de las
probables contingencias, resolvimos marcharnos\ldots{} ¿A dónde? Nuevas
dudas. ¿Iríamos a Cartagena en el falucho de Ansúrez, o a Tarragona en
tartana?\ldots{} ¿Y por qué no a Madrid? ¿No sería esto lo más seguro?
Tan indecisos estábamos, que entres veras y bromas propuse a Donata que
lo echáramos a la suerte, y el modo y forma de consultar el Destino fue
diferido para la mañana siguiente. Ved ahora, amigos míos y amables
lectores, Marqués y Marquesa de Beramendi, la solución que nos dio el
Oráculo, bajo la sagrada representación de Virgen de la Cinta.

Levantose Donata muy tempranito, casi al amanecer, y con Polonia se fue
a la Catedral. De regreso estaba cuando yo me vestía. Risueña entró en
el palomar, y con tiernas caricias me notificó la divina solución de
nuestras dudas. Bastaron medias palabras para que yo comprendiera que la
Virgen hablado había dentro del corazón de Donata con misterioso
lenguaje sólo entendido de la sincera piedad. En resumen: decía la voz
del cielo que sin miedo ni vacilación alguna nos embarcáramos en la nave
del señor Ansúrez. «Y para que veas, \emph{Confusito} de mi
alma---agregó Donata,---cómo ha correspondido la verdad natural a las
voces que hablaron en mi corazón, sabrás que al bajar las gradas de la
Catedral, vimos pasar a tres hombres, uno muy alto, vestido de azul.
Polonia saltó y dijo: «Mírale\ldots{} ése es el amo del falucho. Parece
que Dios te le envía.» No me atreví a correr tras él. En cuatro brincos
fue Polonia; le paró, hablaron\ldots{} Le encontrarás toda la mañana en
el Astillero\ldots{} Búscale en el tinglado de un calafate nombrado
\emph{Lleó.»}

No puedo ocultar que Donata me comunicó su anhelo de huir por mar.
También sentía yo en mí la corazonada, las tenues voces íntimas que me
aconsejaban lo mismo que la Virgen sugirió a Donata, y esto prueba cuán
extenso y variado es el reino de la superstición. Salí en busca del
marino; pero no quiso Dios que mis pasos fuesen tan derechos como yo
quería, porque al atravesar la Plaza de la Ciudad, sentí tras de mí la
voz del Castrense, y antes que me volviera, su mano me cogió del brazo.
«Vamos, querido \emph{Confusio}---me dijo,---vamos a ver con nuestros
propios ojos a Carlos VI y a su hermano. ¿Pero qué?\ldots{} ¿ignora el
gran acontecimiento? Anoche les han cogido. Se sabe por un correo que
llegó muy temprano. Ya no tardarán en entrar en Tortosa, pues a las dos
de la madrugada salieron de Ulldecona. Vamos, amigo, a prisita\ldots{}
no haga el demonio que entren antes de la hora prevista y perdamos esta
fiesta. ¡Cuándo veremos otro Rey, aun siendo de papelón y sin ningún
derecho a la Corona, digan lo que quieran!\ldots» Me llevó; dejeme
llevar hacia la calle del Arsenal, donde está la Comandancia General. A
mí, como a don Jesús, se me había despertado la curiosidad vivamente.
¡Carlos VI\ldots{} un perfil histórico\ldots{} la encarnación de una
idea, tras de la cual corre el caudaloso torrente de la guerra civil!
Hay que ver, hay que ver esa cara, dibujada por Clío\ldots{} con un
hueso mojado en sangre española.

Ya había gente en la calle del Arsenal, gente en la Barana y en la calle
de Pont\ldots{} Aquí nos encontramos a don Higinio con otros amigos de
la tertulia del café. «Higinio---le dijo el Castrense,---¿cómo no te has
traído la banda para darle a tu monarca un golpe de Marcha Real?» Y el
Músico chiquitín replicó: «Lo que le daría yo es un golpe de \emph{Himno
de Riego}, y mejor del \emph{Trágala}\ldots{} Por de contado, que le
fusilarán. Y a ese fusilamiento sí que no falto, aunque mi estómago se
me ponga de uñas\ldots{} ¿Que no le fusilan? ¿Pues qué justicia es ésta,
ajo? Al otro pobre cuatro tiros, y a éste, chocolate con mojicón.» Por
acuerdo razonable de todos, fuimos a situarnos en la puerta de la
Comandancia, donde forzosamente había de parar lo que don Higinio llamó
el \emph{cortejo}, y en efecto, a los diez minutos de espera vimos que
entraban en la calle cuatro guardias civiles a caballo, detrás una
tartana\ldots{} más guardias civiles, otra tartana\ldots{} y una escolta
pequeña de húsares\ldots{}

¡Ya estaban aquí! ¡Qué interesante es ver a la Historia apearse de un
carricoche, con aire mohíno, y codearse con los simples mortales que no
salen de los espacios grises de la vida privada!\ldots{} De la primera
tartana vi bajar a Montemolín, un joven alto, de buena presencia,
pálido, con una nube en un ojo, barba que renacía tenuemente después de
afeitada, como cerquillo obscuro en los bordes del ovalado rostro; vi
detrás al hermano, más pálido y ojeroso, menos interesante que el
primogénito. Ambos, al entrar en la Comandancia, pasaron rozando
conmigo: observé sus gabanes largos llenos de polvo; las hilachas de sus
pantalones; la chafadura de los hongos negros de seda, blanqueados del
polvo; los cuellos de camisa no mudada en luengos días; el deterioro
general de sus ropas; los guantes por cuyos descosidos asomaban los
dedos; noté las caras soñolientas, mustias, avergonzadas\ldots{} ¡Oh,
qué historia tan triste! Sentí lástima de la Causa y de sus hombres, que
parecían conducidos a un patíbulo sin muerte, o a una muerte histórica
sin dignidad.

\hypertarget{xviii-1}{%
\chapter{XVIII}\label{xviii-1}}

Apeose el Teniente de la Guardia civil, hermano del Castrense don Jesús,
y éste, después del abrazo, le asestó las preguntas que resumían la
curiosidad ardiente de los que en el portal estábamos. «¿Le cogisteis en
la casa que llaman de Gandalla, a la salida del pueblo? ¿Es cierto que
tuvisteis que entrar por el tejado?\ldots» Según nos dijo el Teniente,
no habían subido los guardias más arriba de una ventana o balcón, pues
el dueño de la casa, Cristóbal Raga, que también venía preso, no quiso
abrir la puerta, pretextando que se le había perdido la llave. En un
aposento alto, muy pobre, y con cortinaje de telarañas, encontraron a
Montemolín, a su hermano y a un criado. Se vestían a toda prisa cuando
entraron los guardias. Montemolín dijo gravemente su nombre, y la frase:
«Estamos a disposición de ustedes\ldots» Les llevamos a nuestro cuartel,
donde se les ofreció chocolate: lo tomaron con panecillos, y\ldots{}
¡hala!, en marcha.

El Mayor de plaza, que había venido en la primera tartana, nos contó que
don Carlos Luis es hombre de fino trato. El tonillo de persona Real,
benévola y cortés con los inferiores, no se le cae de los labios. Elogió
mucho a la Guardia civil, calificándola de \emph{admirable cuerpo}. En
el extranjero se le citaba como el mejor de su índole organizado en
Europa\ldots{} y él no cesaba de poner en las nubes su buen porte,
policía, y puntualidad en el cumplimiento del deber. Don Fernando
hablaba poco, y sólo se permitía repetir como un eco las opiniones de su
hermano\ldots{} El Cristóbal Raga, que les había dado asilo, era un
honrado labrador que procedió con noble y franca generosidad, movido de
sentimientos humanitarios. El Arcipreste Ruiz le había dicho: «Guarda a
estos señores, que corren peligro,» y no necesitó más para darles
albergue piadoso. Les guardó cuanto pudo; pero, según cuenta, no cesaba
de decirles: «Caballeros, váyanse, que me están comprometiendo.» Dulce
había ofrecido diez mil duros por los Príncipes. Cristóbal Raga no los
habría entregado ni por un millón.

La página histórica se desvanecía en la insignificancia. Ya no trataban
las autoridades tortosinas más que de proporcionar a los primos de Su
Majestad alojamiento decoroso. A toda prisa se arregló la casa del
Comandante de Ingenieros para que Sus Altezas se aposentaran como
personas de sangre Real. Recobrado el equipaje, que les fue cogido en la
fuga, pudieron vestirse de limpio. Lo primero que pidió Montemolín fue
que se le permitiera poner un telegrama a su esposa, y al punto le fue
concedido. La página histórica terminaba con un recadito a la familia:
«Estamos buenos. Se nos trata con la debida consideración.»

A medida que se enfriaba en mi espíritu el interés de aquel negocio
público, recobraba su calor el asunto propio. Dejé a mis amigos para
seguir el camino que me había propuesto al salir de casa, y al llegar a
la Puerta del Temple tuve la suerte de encontrarme de manos a boca con
Diego Ansúrez, que del Astillero venía con una caterva de
\emph{menadors, filadors y calafats}, la plebe más bulliciosa y maleante
de esta ciudad, carpinteros de ribera los unos, los otros fabricantes de
cuerdas de cáñamo para la marina. ¿Quién no ha visto en los puertos de
mar la interesante obra de torcer el cáñamo, al aire libre, obteniendo
los cabos de diferentes gruesos, desde las guindalezas y calabrotes
hasta las sutiles drizas para izar banderas? \emph{Menadors} son aquí
los que dan vueltas a la rueda, \emph{filadors} los que con el cáñamo
liado a la cintura hilan y tuercen andando hacia atrás. Éstos, y los
calafates y careneros, y los manipuladores de filástica, constituyen un
gremio característico en todo pueblo de costa; gremio que vive en
salvaje independencia, con tanto desahogo de costumbres como de
lenguaje. Antes de que yo pudiera decir a Diego Ansúrez lo que me
proponía, él y los que le acompañaban me preguntaron con viva
impaciencia: «¿Llegaremos a tiempo?»

---¿De qué, señores? ¿A dónde van ustedes?

---A ver el fusilamiento\ldots{} Nos han dicho que han cogido a los
Príncipes.

---Es verdad. Acaban de llegar Sus Altezas.

---¿Pero no fusilan? ¿Qué es esto?---me dijo en catalán, echando fuego
por los ojos, uno de los \emph{menadors} más decididos.

Me reí de la bárbara inocencia de aquellos hombres, tan apartados del
sentir general y del flujo de la opinión. Y uno de ellos, que era sin
duda el más inocente y el más bárbaro, gritó con desaforadas voces:
«¿Pues no son ésos los causantes? ¡Vaya una justicia de porra! ¿Y qué
significa el ofrecer diez mil duros por esas cabezas? ¿Para qué quieren
esas cabezas si no es para pegarles cuatro tiros, o cien tiros, una vez
cogidas?»

---Creímos---gritó otro, ávido de exterminio---que con sólo identificar
las personas\ldots{} cuatro tiros\ldots{} y a paseo.

---¿Pero es verdad que no hay fusilamiento? ¡Nosotros, que veníamos tan
alegres a verlos caer patas arriba!

Traduzco lo que querían decir, no la viveza y gracia de la dicción
catalana expresando tales atrocidades. Que el que esto lea lo
adivine\ldots{} Al fin, desengañados, viendo por tierra sus justicieras
y trágicas ilusiones, siguieron con Ansúrez y conmigo hacia el centro de
la ciudad, por si faltaba algún acontecimiento que diera efusión a sus
almas inquietas, ardorosas. Lo que vimos no fue, en verdad, muy
interesante para ellos; para mí sí, pues me siento encariñado con las
decadencias históricas, considerándolas como el completo derribo de una
época, que nos permite cimentar en el mismo solar otra más fuerte y
vividera. Quizás me equivocaré; quizás la vulgaridad e insignificancia
del fin de la famosa intentona no remata la brutal epopeya carlista,
sino que es un falso desenlace, como los que en las obras de imaginación
sirven para preparar mayores enredos y trapatiestas.

Contaré que después de refrescar con Ansúrez y su gente en un figón
próximo al Arsenal, vimos un espectáculo que al pueblo sirvió de
diversión, y a mí de grave enseñanza por las razones expuestas\ldots{}
De la Comandancia salieron los serenísimos Príncipes, o si se quiere, el
augusto Monarca y su hermano, con los mismos trajes que al entrar
llevaban, revelando ya reciente cepilladura: los pantalones habían sido
remediados de cascarrias, no de los flecos que colgaban por abajo; los
hongos de seda ya no tenían polvo, pero los agujeros de los guantes
seguían ventilando los dedos. Era lástima muy distinta de las otras
lástimas la que inspiraban aquellos señores tan mal trajeados, y que ni
con su humildad y cortesía, ni con la distinción de sus maneras,
lograban inspirar respeto. A su lado iba el Comandante General
hablándoles no sé de qué: debía de ser de algo referente al buen tiempo
que disfrutamos. ¿De qué se habla a los Reyes? Y a Reyes y a Príncipes
como éstos, que sólo parecen tales por el hecho de que hay ilusos que se
dejan matar por ellos, ¿qué se les dice? Comprendí lo comprometido que
debía verse el Comandante General para dar conversación a tales
prisioneros. Al otro lado iba el conde de la Torre del Español, Alcalde
de Tortosa; detrás más militares y dos canónigos de la Catedral\ldots{}
Éstos hablaban entre sí\ldots{} ¿Qué dirían?

Batidores de este cortejo eran los chiquillos que iban delante, haciendo
cabriolas. A un lado y otro, mujeres y hombres del pueblo contemplaban
el paso de los hijos de don Carlos María Isidro. ¿Qué pensarían? Tal vez
en la mente de todos revivía el trágico fin de Ortega, la figura del
caballero que sabe morir por una idea o por un error. ¡Cuánto más
hermoso y más grande el aventurero castigado que el falso Rey sin
majestad y sin corona, pues ni aun la del martirio ha sabido conquistar!
El pueblo no pensaba sino que aquel pobre señor y su hermano estaban mal
de ropa. Peor estaban de entendimiento\ldots{} Al fin, gracias a Dios,
había concluido el oprobio del escondite. ¡Lo que habrían sufrido,
teniendo que dormir en pajares, comiendo porquerías, y sin las
satisfacciones que da la etiqueta a los que de ella disfrutan por el
lado ancho! Pero ya estaban alojados dignamente; ya iban a ocupar la
vivienda que se les había preparado conforme a su rango elevadísimo.
Poco tuvieron que andar por las calles: la Comandancia de Ingenieros,
donde se les instaló, no estaba lejos.

Nos contaron que nada falta allí de lo que puede hacer grata la
existencia de Príncipes trashumantes. Verdad que se tapiaron puertas y
se reforzaron ventanas, y se pusieron centinelas en todos los costados
del edificio, a fin de garantizar la seguridad de los presos. ¡Escaparse
ellos! ¿Para qué? ¿Y a dónde habían de ir que estuvieran mejor? Ya
sabían que no se les haría ningún daño, y que la prisión, los cerrojos y
guardias, no eran más que aparato regio de comedia para sostenerles en
su ilusión de testas coronadas. Cuando vieron la buena casa que tenían,
¡ay!, se llenaron de gozo, y preguntaron si había capilla. ¡Ya lo creo
que había capilla! Y si no, ¡ay!, pronto se la habrían improvisado.
Pidieron los serenísimos caballeros con gran fervor que se les dijese
misa todos los días, pues llevaban mucho tiempo privados del consuelo
religioso\ldots{} ¡Pobrecitos! Y como Dios les quiere tanto, por ser
Dios primer lema de su bandera, ¿qué menos hacer podía que visitarles a
menudo?\ldots{} El que en pensamiento no les visitaba era Ortega. Oí que
ni una sola vez preguntaron por él.

Vista la marcha nada triunfal de los asendereados pretendientes, me
bastaron pocas palabras para entenderme con Diego Ansúrez, el cual fue
tan expresivo en su alegría por llevarnos, como yo en mi gratitud por
favor tan grande. Pero no estaba próximo como yo pensaba el día de la
partida, porque la carena del falucho en un playazo de Los Alfaques no
había terminado: con esta faena y la de la carga había para una semana.
Propúsome luego que nos trasladásemos a Amposta, donde él nos
proporcionaría un holgado y no costoso alojamiento\ldots{} Aún fue más
allá su bondad ofreciéndonos una barca bien acondicionada, en la cual
podríamos bajar al son de la corriente, paseo delicioso en las noches de
luna\ldots{} Cuando fui a mi casa con estas nuevas y el plan de salida,
Donata me conoció en el rostro la alegría que yo llevaba. Poco tardó el
contento en pasar de mi corazón al suyo; y en ella se movió y enardeció
tanto la voluntad, que toda espera le parecía larga, y se puso a recoger
la ropa con idea de partir esta misma noche\ldots{} En clase de varón
prudente, eché frenos a su impaciencia. «¿A qué tanta
precipitación?\ldots{} No vamos a apagar ningún fuego\ldots{} Partiremos
mañana.»

\hypertarget{xxix}{%
\chapter{XXIX}\label{xxix}}

\textbf{Amposta}, \emph{Abril}.---Contentos partimos, no sin dejar
alguna fibra de nuestros corazones prendida en la bella y hospitalaria
Tortosa, en su buena gente, en los leales amigos que allí dejábamos.
¡Qué hermosura de viaje; qué navegación maravillosa por la corriente del
ensueño, por un río de plata, bajo un cielo de intenso azul! La luna
llena, lámpara rostral, iluminaba y miraba el agua por donde íbamos, y
la tierra de una y otra ribera. Su claridad penetraba en nuestras almas,
y nuestros ojos requerían los ojos cóncavos del planeta y su boca sin
sonrisa. La redonda cara corría ladeada por el cielo, y nosotros,
después de mirarla en lo alto, la veíamos abajo, en el cristal sereno y
profundo. ¡Oh Ebro, español río, cuán soberano y bello al dejarte caer
perezoso con toda la hinchada pompa de tus aguas en los brazos de la
mar!

En la primera parte de la expedición íbamos mudos, subyugados de la
hermosura que nos rodeaba; luego cada cual empezó a lanzar del alma sus
observaciones; cerca ya de Amposta, Donata, más comunicativa que yo, me
dijo: «Voy confiada en la protección de la Virgen. Verás cómo no nos
pasa nada, y salimos tranquilamente de este río para el mar grande, que
nos llevará lejos. No me engaño, no, en esta confianza. Aunque mucho he
pecado, y pecando estoy siempre, la Virgen me saca de mis tribulaciones.
La Virgen no castiga, la Virgen a todos ama, intercede por los
pecadores, y perdonándonos nos enseña a ser buenos\ldots{} La Virgen es
la verdadera cristiandad. ¿No lo crees así?»

Respondí afirmativamente, pues no era cosa de ponernos a discutir en
medio de las aguas, ni tampoco estoy seguro de que sea un error el giro
absolutamente feminista que algunos dan a la idea religiosa. Donata, con
más calor de frase, prosiguió así: «He prometido a la Virgen que tú y yo
haremos alguna penitencia para ganar méritos que nos alivien del
pecado\ldots{} Yo digo: el que no haya casamiento, porque no puede
haberlo, ¿quiere decir que nuestro amor no tenga la indulgencia divina?
Éstas son mis dudas.»

Y las mías también. Vi que la testarudez de Donata no abandona la idea
de que yo me vista las negras ropas. ¡Arcano inmenso de un alma
enamorada! Preferí sortear con frases ambiguas este endiablado problema.
Y ella: «La Virgen nos dirá lo que debemos hacer\ldots{} La advocación
de la Cinta será siempre para mí, donde quiera que esté, la más
venerada, la que más adentro se mete en mi corazón\ldots{} También adoro
la de la Providencia, y aquí en mi pecho llevo en un saquito, como
escapulario, las estrellitas milagrosas, que son el juguete de los
angelicos en el Santuario de \emph{Mitan Camí.»}

Ya conocía yo estas estrellitas de cinco picos, que no son más que
fósiles, denominados por la ciencia \emph{encrinites}. Las tortosinas
las veneran como objeto milagroso, y algunas hacen y toman caldo de
ellas creyéndolo el más excelente específico tocológico. Millones de
estos fósiles diminutos se encuentran en un cerro próximo al santuario
de \emph{Mitan Camí}, llamado también de \emph{Cabrera}, porque en él
tuvo su beneficio, cuando estudiaba para cura, el famosísimo
guerrillero. No quise cuestionar con Donata, ni destruir la leyenda del
carácter sagrado y milagroso de los \emph{encrinites}. La dejé seguir en
su enumeración de los piadosos objetos que lleva, como preservativos
contra el mal, en las aventuras que vamos a correr. «Sabrás también,
\emph{Confusio} mío, que traigo conmigo una \emph{rosa de Jericó}. Creí
que no podría conseguirla; pero Polonia se desvivió por darme gusto, y
entre ella y don Jesús convencieron a una señora de las principales de
la ciudad para que me cediera la flor\ldots{} No creas: es legítima, del
propio Jericó, que bien probado con escrituras lo traen los vendedores
de estas cosas\ldots»

---Sí, sí: no hay duda; legítima será---dije yo lleno de indulgencia
ante tales errores, convencido de que es más fácil convencer al Ebro de
que se vuelva atrás y se suba hasta Reinosa, que arrancar del cerebro de
mi odalisca las nefandas supersticiones que en él se han hecho fósiles.
No sé quién dijo que nadie entrega sus ideas para que le pongan otras.
Lo que llamamos conversión no existe en la realidad; es siempre un
engaño del catequizador o del catequizado.

Medianamente instalados en Amposta, aguardábamos tranquilos el día del
embarque. Me encantaban, en aquella antesala del delta del Ebro, la
amplitud de horizontes, el aire salino, la frescura que enviaba el mar
con vigoroso resuello. El terreno bajo, palustre, nos ofrecía por el
lado del Naciente la extensa marisma, hibridación pintoresca de la
tierra y las aguas\ldots{} Al ser de día, el paisaje anfibio que en la
noche de nuestra llegada apreciamos vagamente a la luz ensoñadora de la
luna, se nos reveló en toda su grandeza, no ya iluminado de plata, sino
de oro. Al sol, la marisma era más risueña, más rica de color, más
hirviente de vidas zoológicas, más reveladora de lo infinito.

Desde el primer día, nos hicimos a una vida placentera, descuidada.
Donata encontró amigas sin salir del posadón, y yo, por la amistad de
Ansúrez, trabé conocimiento con innumerables personas que vivían del
esquilmo de tierra y mar: pescadores, cazadores, explotadores del
carrizo y la enea. Se me pasaban los días cazando \emph{collverts} en
los tortuosos canalizos, embarcado en mi chalana con dos o tres amigos,
o bien recorriendo a pie descalzo los barrizales, con descanso y
merienda en esta o en otra barraca. Alguna vez nos acompañó Donata en la
navegación de chalana por los caños salobres, o nos íbamos en lancha por
el canal grande a comer la sabrosa \emph{sopa de raps} con los calafates
que carenaban el falucho en San Carlos de la Rápita\ldots{} ¡Qué
agradables almuerzos y meriendas, sentados en la arena entre gentes
sencillas, oyendo el suave rumor de los besitos que daba el mar a la
playa!\ldots{} Frente por frente veíamos la Punta de la Baña, que
resguarda la bahía de Los Alfaques, y detrás la faja azul del
Mediterráneo, que nos decía: «Venid a mí, y os llevaré a las partes más
bonitas del mundo.»

Sorprendionos una mañana la grata visita de don Jesús, el Castrense, que
ha venido a pasar un par de días con nosotros. Al punto se agregó a mis
expediciones de caza y pesca, pues no hay otro más aficionado a esta
clase de ejercicios\ldots{} Como aquí me siento tan alejado del mundo,
no me afectan los cuentos que don Jesús me trae del fin y desenlace de
la \emph{ortegada}. En su cómoda residencia de la Comandancia de
Ingenieros, el titulado rey Carlos VI hizo formal declaración de
renuncia de sus pretendidos e ilusorios derechos a la Corona. ¿Quién
pudo pensar que a la trágica epopeya del Carlismo se le pusiera una
escena final de comedia pedestre? Al bajar el telón sobre tal escena,
¿no se oirá la silba en el Polo Norte y en el Polo Sur? ¡Y para esto
vinieron al mundo Cabrera y Zumalacárregui, y anduvo en loca
peregrinación don Carlos Isidro, llevando a rastras la
\emph{Generalísima} su Patrona! Dijeron el Rey y su hermano en su
declaración que hacían la renuncia por libre y espontánea voluntad.
¡Pobrecitos, qué buenos son, y cuánto debemos a sus corazones
magnánimos!

Más interés tenía para mí lo que de nuestra patrona Polonia nos contó
don Jesús. Ya la tiene tan adiestrada en las prácticas de la buena
administración, que bien podrá poner una fonda de las grandes y
desenvolver en ella su negocio. Polonia es mujer de mucha disposición
natural, y don Jesús un hombre muy práctico\ldots{} Cuando la conoció,
el gravísimo defecto de ella era su querencia de las supersticiones más
ridículas. Si un huésped era reacio en el pago, encendía velas a San
Antonio. Ponía los chorizos en cruz para que no se los robase la
cocinera, y tenía repuesto de agua bendita para rociar los garbanzos
duros\ldots{} Y entre tanto, un desbarajuste horrible en la
administración, y el más lamentable desarreglo de cuentas. Pues el don
Jesús la curó de estos despropósitos con su cariñosa enseñanza. ¿Cómo?
¿Qué medios empleó? «El palo, querido \emph{Confusio}---me dijo mi
amigo,---el palo, y crea usted que no hay otro medio\ldots{}
Materialmente no empleé bastón ni garrote\ldots{} ha sido con la mano, a
bofetada limpia\ldots{} Convénzase usted de que a estas hembras criadas
a lo moro no hay otra manera de enderezarlas y de enseñarles el
Catecismo de la vida práctica, para que ellas vivan y hagan llevadera la
vida de los demás.»

Estas y otras cosas muy entretenidas me contaba don Jesús, divagando por
los carrizales, juncales y espadañales, donde viven las innumerables
repúblicas de ánsares, cercetas, guardarríos y fúlicas. También se ven
por allá parejas de los flamencos de zancas rojizas. Imaginad las aún
más populosas repúblicas de moluscos, lombrices y gusarapos, que sirven
de alimento a tantísimas aves, así nadadoras como andariegas\ldots{} El
último día que aquí estuvo don Jesús, salimos con varios amigos
\emph{caberos}, que así llaman a los habitantes de aquellos partidos
pantanosos, y nos fuimos al de la Enveja, río abajo, por la derecha
orilla. Toda la tarde estuvimos en la persecución de los pobres patos:
fui yo más afortunado en mis tiros que el Castrense; y éste, picado del
amor propio, se corrió con dos \emph{caberos} muy prácticos hacia la
parte más intrincada de la marisma, donde los carrizos y cañas forman un
espeso matorral, en muchos sitios inaccesible.

Oímos tiros de nuestros compañeros; pero tan de tarde en tarde, que
seguramente no hacían cosa de provecho. De pronto, el lejano tiroteo
arreció, y tan repetidos fueron los disparos que nos alarmamos. Ya la
curiosidad y el temor nos llevaban hacia allá, cuando vimos venir a don
Jesús despavorido, y a los dos \emph{caberos} detrás gritando como
energúmenos\ldots{} ¿Qué pasaba? Pues que por aquella espesura andaba un
grupo de cazadores intrusos que más bien parecían bandidos. Después de
insultar a nuestros amigos, les habían hecho fuego. Gracias que de
milagro no les tocó ninguna bala\ldots{} Fui de parecer que debíamos
escarmentar a los intrusos; mas un \emph{cabero} me atajó el paso,
diciéndome: «No vaya, don Juan, que son gente mala, tiradores de
primera\ldots» Vi que a una distancia como de cien pasos se agitaban las
cañas\ldots{} y entre ellas aparecieron hombres, hollando con pisadas de
paquidermo la lozana vegetación. Uno, más insolente que sus compañeros,
saltó de los cañaverales como furioso jabalí, y en dirección de acá
lanzó amenazas o burlas que no entendimos. Cuando yo le apuntaba, el
\emph{cabero} me gritó: «Quieto, quieto, que es el Arcipreste.»

---Aunque sea el Obispo---repliqué con la obstinación que me daba la
conciencia del peligroso lance. Los \emph{caberos} se abalanzaron a mí,
parándome los movimientos, y don Jesús me dijo: «Si no le hostigamos, no
nos embestirá. Así es el león, así el jabalí: como no le hagan fuego,
pasa tan tranquilo\ldots» Miré al hombre, que a distancia se mantenía en
un claro del ondulante bosque de cañas: sus facciones no pude
distinguir; mas por el aire y la estatura me pareció, en efecto,
\emph{don Juanondón}. Le vi alzar y agitar los brazos, que se me
figuraban aspas de molino, y claramente llegaron a mi oído estas voces:
«¡Eh!\ldots{} \emph{Confusio}\ldots{} aquí estoy. ¿No me
conoces?\ldots{} Yo a ti te conozco\ldots{} Hasta luego, hijo\ldots{} Ya
nos veremos.» Los penachos de las cañas oscilaron de nuevo, y
desapareció la figura\ldots{}

\hypertarget{xxx}{%
\chapter{XXX}\label{xxx}}

Las grandes superficies de agua conducen muy bien el sonido. Prestando
atención e imponiéndonos silencio, oíamos salpicar en el espacio sílabas
de palabra humana, que se confundían con el lenguaje de las aves de la
marisma. Era la conversación del Arcipreste y los suyos alejándose por
los fangales de la Enveja, al través de carrizos, charcos, salinas y
espesuras de eneas y barrillas. Un \emph{cabero}, que era como cabo de
nuestra partida, apodado \emph{El Topo}, me dijo: «Van a los altos de
Muntciá, donde duermen. Todo el día andan por aquí de caza y
pesca\ldots{} No se puede con ellos: donde quiera que van, se hacen los
amos.» Como yo le indicara que la Guardia civil podía bajarles los
humos, \emph{El Topo}, con grave acento semejante al que usan los
políticos, me contestó: «Nosotros los \emph{caberos} no nos pondremos
nunca de parte de los que dañen a don Juan, porque don Juan es bueno,
aunque cabecilla, y socorre a los pobres de la Enveja y de la Caba, de
Camarles y Campredó, sin distinguir \emph{carlinos de isabelos},
\emph{ni asolutos de liberalos.»}

Volví a mi casa, ya cayendo la tarde, sin poder disimular mi inquietud.
El Castrense, bien enterado por Polonia de mi situación ante el
Arcipreste, me aconsejó que, para evitar alguna escena desagradable, nos
fuésemos a San Carlos tempranito, y nos metiésemos a bordo del barco en
que hemos de partir. Pareciome atinado el consejo, y en cuanto
despedimos a nuestro amigo, que tornó a Tortosa en burro, comuniqué a
Donata mis inquietudes y el plan de ponernos en salvo por la vía de agua
más corta. Menos temerosa que yo, Donata confiaba con cerrada convicción
en el amparo de la Virgen\ldots{} Yo pensé que, sin dejar de fiarnos de
la Virgen, debíamos correr todo lo que pudiésemos. Y nuestras prisas
coincidieron con las órdenes de Ansúrez, que al partir nos dejó recado
de que no nos descuidáramos, pues el barco estaba listo para darse a la
vela\ldots{} Si el enemigo desde tierra nos expulsaba, el amigo nos
llamaba con cariñosa voz desde el mar. Dios hablaba por él, y a Dios nos
confiábamos en tan críticas horas.

No me fue muy fácil encontrar embarcación que nos llevara cómodamente:
la lancha de pescadores que a duras penas conseguí no era nueva ni
grande; mas la tuve por suficiente, y en ella nos embarcamos con dos
chicos marineros que manejarían los remos o la pértiga, según lo
indicara el practicaje por aguas de tan variado fondo. Retrasados por la
tardanza en encontrar embarcación, dadas las siete metimos a bordo
nuestros equipajes, mi escopeta y cartuchos, luego nuestras personas, y
en marcha, aguas abajo.

Navegamos sin contratiempos unas dos horas; después se nos varó la
embarcación por impericia de nuestros pilotos. Fue menester cargar a
popa los baúles y el peso de Donata, mientras yo y los marinerillos,
metidos en el agua, empujábamos hacia atrás. Cuando logramos coger de
nuevo la parte del canal donde hay más fondo, seguimos bogando
avante\ldots{} De improviso sonaron voces por babor: venían de los
canalizos que comunican con el estanque de Algady\ldots{} Donata
palideció y me dijo: «Es la voz de Rufulet, es la voz de Gasparó\ldots{}
No podemos escapar. Sería preciso volar, y aun volando nos
cogerían\ldots» Apenas dicho esto, vi que tras de nosotros, por un
canalizo que desembocaba entre juncos, apareció una chalana. Ya no había
duda. En ella venía el Arcipreste, y él movía con vigoroso brazo la
pértiga que impulsaba la frágil embarcación. Cuando apareció la nave
enemiga, estaría como a sesenta brazas de la nuestra; pero la distancia
por momentos se acortaba, y el Arcipreste parecía reducirla más con
estos feroces gritos: «¡Cabra loca, detente!\ldots{} Ya no te me
escapas\ldots{} Y tú, más loco que la cabra, para el remo, si no quieres
que yo te le rompa en la cabeza.»

Furioso cogí mi escopeta. No soy buen tirador; pero en aquel momento de
ceguera y coraje, confiaba en que mi intento pudiera más que mi
puntería. Con más presteza que yo, Donata acudió a impedir mi
movimiento. «No, no, Juanico mío\ldots{} Será peor si le das\ldots{}
Estamos cogidos. Sálvenos la Virgen nuestra Madre.»

No tardé en comprender que toda defensa era inútil. Tras de la primera
chalana aparecieron otras\ldots{} Conté tres, cuatro. El cabezudo Ruiz
tenía también su escuadrilla, y suyos eran en la marisma el fango y el
agua. ¿Contra una potencia terrestre y marítima, qué podíamos
nosotros?\ldots{} Acortamos remo, y él llegó ávido. Soltando la pértiga,
echó la manaza a la borda de nuestra lancha para abarloar las dos
embarcaciones. Hecho esto, saltó a la mía, diciendo con horrible
sarcasmo: «¿Creyeron mis amigos que les dejaría marchar sin darles mi
despedida? ¿Eso creíais, sinvergüenzas, canallas?\ldots{} Por los
cojilondrios de San Rufo, que hubiera sentido no poder echaros mi
bendición antes que salierais a la mar. Ya os tengo cogidos\ldots{}
Reíos ahora de mí, cojilondrios; llamad a la Guardia civil marítima para
que os defienda\ldots{} llamad al general Dulce y a la putativa de su
madre; llamad a la Isabel con toda su corte, o a O'Donnell con su
ejército\ldots{} ¡Ja, ja!»

Ni Donata ni yo dijimos nada. Aterrados, mudos, sin otra idea que la de
nuestra pequeñez ante la grandeza del enemigo que con su poder nos
abrumaba; absolutamente convencidos de que nadie había de venir en
nuestro socorro en aquella soledad, éramos como condenados a muerte que
ya no pueden pensar más que en un morir digno. Conté los tripulantes de
la escuadrilla: eran nueve. Apenas entró don Juan en nuestra lancha, dio
un cosque a cada uno de mis marinerillos, y les mandó que se fueran a la
chalana. De ésta pasó a mi embarcación Rufulet, cuya imponente
corpulencia vale por media docena de hombres. Estábamos, pues, no sólo
vencidos, sino maniatados, y con el filo del cuchillo en la garganta.
Rápidamente pensé yo: «¿Qué hará este bruto? ¿Nos degollará? ¿Nos tirará
al agua? Puede que me mate a mí solo, y se lleve a Donata\ldots» En
momento tan angustioso, miré en derredor y no vi más que algunos patos
que al ruido de las embarcaciones tomaban tierra, y graznando se
alejaban por entre cañas\ldots{} Envidié a los patos; envidié a las
anguilas que bajo las aguas deslizan sus resbaladizos cuerpos entre el
fango; envidié a los pulpos, a las almejas y a los más diminutos
bicharracos de la Creación\ldots{} En este paréntesis de mis envidias
estaba yo cuando don Juan, cogiendo mi escopeta como si quisiera
desembarazarme de un estorbo, la dio a un mocetón de la chalana más
próxima, y a mí me dijo: «¿Para qué quieres tú este chisme,
\emph{Confusio}?\ldots{} ¿Escopeta un teólogo? ¡Ja, ja!\ldots»

Donata permanecía como estatua. En su palidez marmórea, en la tensión de
los músculos de su cara, vi una conformidad de tanta fuerza como el
heroísmo. El Arcipreste hablaba por los tres. «Veo que estáis
resignados---nos dijo sentándose en el borde de la lancha, mientras
Rufulet remaba solo.---Comprendéis que tengo razón, y que el que me la
hace, me la paga.» Miré a Donata. Creí leer en su mirada fija esta
terminante admonición: «Callemos\ldots{} dejémosle que desfogue la
barbarie\ldots» En esto, llegamos a un ensanche del canal, formando como
una bahía casera para naves de juguete. Con fuerte voz, don Juan mandó
echar anclas. La escuadrilla, con admirable maniobra, formó un círculo
en derredor de la que llamo mi lancha, que ya no era mía, sino del
enemigo, y dio fondo, arrojando al mar, no las anclas, que esto allí no
se usa, sino las \emph{potalas}, una piedra suspendida con una cuerda.
Mientras daba fondo la armada vencedora, el Arcipreste mandó que se
sirviese la comida, pues eran las doce, según indicó la altura del sol,
que allí no había cronómetros, sextantes ni astrolabios.

En una de las chalanas vi una parrilla sobre plancha de hierro, donde
ardían palitroques, eneas y cañas secas: era la cocina. El almuerzo
consistía en ruedas de saboga asadas, vino y pan. Hiciéronnos los
honores que se deben a los reos en capilla; Donata y yo fuimos los
primeros a quienes se sirvió el frugal almuerzo, naturalmente sin platos
ni servilletas, ni más cuchillos ni tenedores que los santos
dedos\ldots{} Pero Donata y yo, con el pie en el patíbulo, estábamos
absolutamente desganados. Quedose mi amada con la saboga y el pan en la
mano, sin rechazarlo ni comerlo; yo rechacé mi parte cortésmente\ldots{}
«No tenéis gana---dijo el Cura;---yo sí, que esta vida de mar da mucho
apetito.» No pude contenerme más tiempo dentro del horrible cerco de mi
angustia, y con más dignidad que arrogancia dije a mi enemigo: «Señor
don Juan, sepamos pronto, pronto, en qué ha de parar esto. Nada puedo
contra usted\ldots{} Usted puede matarnos, arrojarnos al agua, sin que
nadie más que Dios le pida cuenta de su crueldad.»

---Puedo mataros, echaros al agua con una piedra al pescuezo; puedo
hacer lo que me dé la real gana---dijo el Cura flemático, comiendo y
saboreando el pan y la saboga.

---Dígalo claramente. Somos cristianos y queremos prepararnos para
morir.

---¡Pues no tienes poca prisa! Calma; dejadme comer. Después
hablaremos\ldots{} ¡Estaría bueno que os matara sin atormentaros antes
un poquito!\ldots{} Ea, chicos, traed ese porrón, que tengo sed.

En esto se levantó Donata de la tabla de popa en que había permanecido
desde el abordaje, y se llegó a la cuaderna mayor de la nave, donde
estábamos el Arcipreste y yo. Noté en ella una lividez extremada, y
vibración rápida de los músculos de su boca. Con actitudes y
contorsiones que me parecieron epilépticas, se inclinó hasta tocar con
sus dedos el agua. Mojados los dedos, se santiguó\ldots{} Después sacó
del pecho un haz de ramas secas, semejante a una escobita, y lo mojó en
el agua, diciendo con tartamudez: «¿Es salada ya\ldots{} ya salada?»

---Salada es---murmuró el Arcipreste, que contemplaba con estupor a mi
odalisca.

---Salada---repitió Donata,---y como salada, bendita. Todo el mar es
agua bendita\ldots{} ¡Salve, Madre de Dios, estrella del Mar!\ldots{}

Con la prodigiosa escobita, que hacía veces de hisopo, roció al Cura
tres veces, diciendo con voz grave, cavernosa, que yo no había oído
nunca en ella: «En nombre de la Reina de los Cielos, de la Tierra y del
Mar, te mando que huyas, enemigo de las almas, y dejes en paz a estas
infelices criaturas pecadoras, que a Dios darán cuenta; a Dios y a la
Virgen, no a ti, que eres malo. Si has tomado forma de diablo para
atormentarnos, suelta esa forma vana y mentirosa, o vete con ella a los
Infiernos\ldots» Así concluyó el exorcismo; y una vez dicha la última
palabra, cayó Donata al fondo de la barca, como si con el esfuerzo de su
voz mística quedase rendida y exhausta. Era una epiléptica, una
iluminada, que en momento crítico recibía fuerza y voz de los espíritus
celestes para combatir a los malignos\ldots{} Contagiado yo de aquel
delirio, también quedé mudo y paralizado de todos mis miembros, y en el
Arcipreste advertí, cuando acudió a levantar a Donata, temblor de manos,
fruncimiento de cejas y alteración total del fiero rostro.

~

Rociamos con agua bendita, esto es, agua salada, el rostro de la
iluminada mujer, y cuando la tuvimos medio repuesta de su arrebato
místico, sentadita en la tabla, con el apoyo y sostén de mis brazos, don
Juan, en tono muy distinto del que había usado hasta entonces, habló
así: «Ni tú, gran mocosa, ni ningún nacido me gana en devoción a Nuestra
Señora\ldots{} Pero esos arrumacos estaban de más. Suprímelos para otra
vez. Yo, sin perder la chaveta con supersticiones y tonterías
milagreras, digo con toda mi alma, cuando el caso llega: \emph{Tú,
Señora,---dame agora---la tu gracia---toda hora---que te sirva---toda
vía}\ldots{} Si me hubieseis dicho esto cuando entré en vuestra barca,
yo os hubiera respondido:---No os haré ningún daño. Vengo no más que a
despediros y a daros consejos.» Dicho esto, dio la orden de levantar
anclas, o sea \emph{potalas}, y navegando la escuadrilla con rumbo hacia
La Rápita, \emph{don Juanondón} escondió las uñas de su fiereza, aunque
no las de su ironía.

«Sois felices, y os queréis mucho, ¿no es verdad? Pues a ti,
\emph{Confusio}, te felicito. No te llevas una mujer, sino una santa.
¿Has visto alguna vez beatería más recargada de supersticiones que la de
tu odalisca? Así la llamas: lo sé todo\ldots{} Pues a ti, Donatilla,
también te felicito. Te llevas, no diré un hombre, sino un profeta, un
sabio, un padre de la Iglesia. Entre los dos vais a reformar el mundo.
¡Ja, ja!»

Luego moduló suavemente su tono hasta llevarlo a esta humana y más
verdadera expresión de lo que sentía: «Eres un gran majadero,
\emph{Confusio}; eres un chiquillo sin conocimiento, esclavo de tu
imaginación y de las mil vaciedades románticas que has sacado de los
malditos libros\ldots{} ¿Te acuerdas de lo que hablamos aquella tarde en
el bodegón de Llopis? ¿Has olvidado lo que te dije? Pues te dije que en
la vida, y no en las bibliotecas, debes atracarte de lectura y estudio.
En fin, ya estás aprendiendo, y mucho más aprenderás en la compañía de
esta visionaria\ldots{} Ya verás, hijo. No te arriendo la
ganancia\ldots{} Recordarás que te encajé mi teoría de que todo cuanto
bueno hay en el mundo es para nuestro goce, y que Dios no hizo a la
mujer para que la despreciemos, sino para todo lo contrario\ldots{} No
la hizo de piedra, sino de carne. ¿Por qué no me dijiste entonces que
querías a Donata?\ldots{} Yo te la hubiera cedido\ldots{} gustoso, sí,
gustosísimo. Ya estaba yo pensando en el cómo y cuándo de
colocarla\ldots»

Esta declaración del maldito Arcipreste me llenó el alma de turbación,
de vergüenza\ldots{} No había yo conquistado una mujer, sino robado una
esclava, como pude haber cogido furtivamente la cabra o el gallo del
vecino. Socialmente considerada mi aventura desde el punto de vista del
Arcipreste, era el más lamentable desengaño. Callé para evitar
discusiones que habrían embrollado las cosas. Se me hacían siglos los
minutos que tardábamos en perder de vista al diablo de Ulldecona. Para
fastidiarme por completo, me dijo: «Pues tus aficiones te llaman a la
Teología y a la vida eclesiástica, persevera en ellas, que por tu
talento has de llegar a donde llegan pocos. Con esto, y la guapa sobrina
que te llevas, serás dichoso\ldots» Nada contesté\ldots{} temía
encenderme en cólera\ldots{} Miré a Donata, y en su rostro sorprendí la
ola de satisfacción que levantaban en su alma las ideas del que fue su
señor. Para ella, el cambio de dueño había sido un triunfo, la
realización del vago adulterio de amor libre y delirio religioso. Para
mí, ¿qué era? No lo sabía entonces, no daría con el \emph{quid} de mi
problema psicológico mientras no pudiese reflexionar y sondearme a gusto
en la soledad del mar.

¡El mar! ¡Oh!, ya estábamos en él\ldots{} La Rápita desplegó ante mis
ojos su espléndido panorama. Remando fuerte, llegamos al falucho, en
franquía ya, dispuesto para salir. Antes de que transbordáramos, don
Juan nos dio los últimos consejos. «Sed buenos y no escandalicéis, o
escandalizad lo menos posible\ldots{} Al acecharos y perseguiros hoy, no
ha sido mi objeto haceros daño, sino daros un gran susto, y luego
despediros con afectos y con mi bendición. Donata, mira lo que haces:
persiste en tu amor a la Virgen, pero sin arrumacos ni requilorios. Tú,
\emph{Confusio}, métete en lo eclesiástico, que ése es tu camino y para
eso has nacido. Yo me quedo aquí amparando a los pobres, y mirando por
la guerra, que la guerra es la sacudida que damos al pueblo español para
que se despabile y aprenda a tomar lo suyo\ldots{} Porque todo es
suyo\ldots{} y nada es del maldito Gobierno\ldots{} Con que adiós, hijos
míos. Se me olvidaba deciros que si para el viaje necesitáis dinero, a
prevención he traído mil reales\ldots»

Le dimos las gracias, sin aceptar su generosa oferta. Subimos al barco,
y el buen Ansúrez mandó levar anclas, pues no esperaba más que por
nosotros. Desde la borda miramos a \emph{don Juanondón}, que con vaga
tristeza nos saludaba moviendo cabeza y manos. No sé qué casta de
diabólica filosofía se aposentaba en el ánima de aquel hombre malo y
bueno, según Donata. ¿Sabréis vosotros, nobles Marqueses de Beramendi,
descifrarme este complicadísimo enigma? ¿Y de mi aventura qué decís?
¿Pensáis que voy contento, que voy triste? ¡Ay!\ldots{} se puede apostar
a que tampoco me descifraréis esto. ¿Hallaré junto a Donata el apacible
y durable encanto de amor, o tendré que salir un día gritando: Quién me
compra una odalisca?

No sé, no sé más sino que ya estoy en la mar, y que la mar me da todos
sus alientos. ¡Oh, qué grandeza de horizontes, qué frescura de aires, y
en las ideas que aquí surgen de mí, qué amplitud, qué extensión de
esperanzas! Algo me ha de traer la vida más allá de estos términos del
agua movible\ldots{} Adiós, hechos pasados que entrego al papel\ldots{}
Hechos futuros, ¿dónde iré a buscaros?\ldots{}

Nota para concluir. Al comienzo de mi relato de la salida de Amposta,
poned fecha de Vinaroz. Aquí lo escribo, y aquí lo firmo con el
clarísimo nombre de \emph{Confusio}.

\flushright{Madrid, Abril-Mayo de 1905.}

~

\bigskip
\bigskip
\begin{center}
\textsc{fin de carlos vi en la rápita}
\end{center}

\end{document}
