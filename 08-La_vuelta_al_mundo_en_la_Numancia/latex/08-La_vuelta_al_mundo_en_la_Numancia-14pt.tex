\PassOptionsToPackage{unicode=true}{hyperref} % options for packages loaded elsewhere
\PassOptionsToPackage{hyphens}{url}
%
\documentclass[oneside,14pt,spanish,]{extbook} % cjns1989 - 27112019 - added the oneside option: so that the text jumps left & right when reading on a tablet/ereader
\usepackage{lmodern}
\usepackage{amssymb,amsmath}
\usepackage{ifxetex,ifluatex}
\usepackage{fixltx2e} % provides \textsubscript
\ifnum 0\ifxetex 1\fi\ifluatex 1\fi=0 % if pdftex
  \usepackage[T1]{fontenc}
  \usepackage[utf8]{inputenc}
  \usepackage{textcomp} % provides euro and other symbols
\else % if luatex or xelatex
  \usepackage{unicode-math}
  \defaultfontfeatures{Ligatures=TeX,Scale=MatchLowercase}
%   \setmainfont[]{EBGaramond-Regular}
    \setmainfont[Numbers={OldStyle,Proportional}]{EBGaramond-Regular}      % cjns1989 - 20191129 - old style numbers 
\fi
% use upquote if available, for straight quotes in verbatim environments
\IfFileExists{upquote.sty}{\usepackage{upquote}}{}
% use microtype if available
\IfFileExists{microtype.sty}{%
\usepackage[]{microtype}
\UseMicrotypeSet[protrusion]{basicmath} % disable protrusion for tt fonts
}{}
\usepackage{hyperref}
\hypersetup{
            pdftitle={LA VUELTA AL MUNDO EN LA NUMANCIA},
            pdfauthor={Benito Pérez Galdós},
            pdfborder={0 0 0},
            breaklinks=true}
\urlstyle{same}  % don't use monospace font for urls
\usepackage[papersize={4.80 in, 6.40  in},left=.5 in,right=.5 in]{geometry}
\setlength{\emergencystretch}{3em}  % prevent overfull lines
\providecommand{\tightlist}{%
  \setlength{\itemsep}{0pt}\setlength{\parskip}{0pt}}
\setcounter{secnumdepth}{0}

% set default figure placement to htbp
\makeatletter
\def\fps@figure{htbp}
\makeatother

\usepackage{ragged2e}
\usepackage{epigraph}
\renewcommand{\textflush}{flushepinormal}

\usepackage{indentfirst}

\usepackage{fancyhdr}
\pagestyle{fancy}
\fancyhf{}
\fancyhead[R]{\thepage}
\renewcommand{\headrulewidth}{0pt}
\usepackage{quoting}
\usepackage{ragged2e}

\newlength\mylen
\settowidth\mylen{...................}

\usepackage{stackengine}
\usepackage{graphicx}
\def\asterism{\par\vspace{1em}{\centering\scalebox{.9}{%
  \stackon[-0.6pt]{\bfseries*~*}{\bfseries*}}\par}\vspace{.8em}\par}

 \usepackage{titlesec}
 \titleformat{\chapter}[display]
  {\normalfont\bfseries\filcenter}{}{0pt}{\Large}
 \titleformat{\section}[display]
  {\normalfont\bfseries\filcenter}{}{0pt}{\Large}
 \titleformat{\subsection}[display]
  {\normalfont\bfseries\filcenter}{}{0pt}{\Large}

\setcounter{secnumdepth}{1}
\ifnum 0\ifxetex 1\fi\ifluatex 1\fi=0 % if pdftex
  \usepackage[shorthands=off,main=spanish]{babel}
\else
  % load polyglossia as late as possible as it *could* call bidi if RTL lang (e.g. Hebrew or Arabic)
%   \usepackage{polyglossia}
%   \setmainlanguage[]{spanish}
%   \usepackage[french]{babel} % cjns1989 - 1.43 version of polyglossia on this system does not allow disabling the autospacing feature
\fi

\title{LA VUELTA AL MUNDO EN LA NUMANCIA}
\author{Benito Pérez Galdós}
\date{}

\begin{document}
\maketitle

\hypertarget{i}{%
\chapter{I}\label{i}}

Divagando por el \emph{Mare Internum} en el falucho de Ansúrez, con
pacotillas comerciales de Vinaroz a Denia, de Torrevieja a Ibiza, o de
Mahón a Cartagena, pasaron Donata y \emph{Confusio} luengos días
apacibles, sin inclemencias azarosas del viento y las aguas. En la dulce
soledad marítima, aprovechando el ocio de las bonanzas, contó Diego
Ansúrez a sus amigos diferentes sucesos festivos y graves de su inquieta
vida, desde que abandonó a la familia y al padre para lanzarse a correr
ásperas aventuras de mar y tierra; y lo que mayormente sorprendió y
cautivó a los amantes fue la forma o modo peregrino con que hubo de
encontrar y conocer a la hembra que tenía por esposa, o cosa tal\ldots{}
El singularísimo hallazgo de mujer fue dispuesto por Dios con un
golpetazo furibundo que a continuación se refiere.

En Febrero del 49 fue a Játiva Diego Ansúrez a negociar cambalache de
aguardiente anisado por pieles y arroz (que así el menudo comercio
cambiaba las especies, empleando el dinero tan sólo para las
diferencias). Dos días no más estuvo allí; y cuando, ultimados los
tratos y arreglos, a su vivienda se retiraba en noche tenebrosa por
calles solitarias y torcidas, sufrió un grave accidente pasando al ras
de los muros de un convento que llaman \emph{Consolación}. Iba el hombre
con el cuidado de la obscuridad echando las manos por delante, los ojos
al suelo fangoso y a los traicioneros dobleces de las tapias, cuando de
improviso le cayó encima un grande y pesado bulto\ldots{} El golpe fue
tremendo, más por la pesadumbre que por la dureza del objeto caído. ¿Qué
era, vive Dios?

Si al recibir el topetazo pensó Ansúrez en el desprendimiento de un
balcón o de un trozo de alero, no tardó en reconocer que el bulto podía
ser un disforme lío de esteras que tuviera por ánima huesos, lingotes de
hierro, quizás un par de macetas con plantas arbóreas. El grito
sacrílego que dio al sentir el trastazo en su cabeza y hombro derecho,
fue contestado por un lamento que del propio bulto salía, el cual no era
rollo de esteras, ni colchón relleno de objetos duros, sino un ser
humano, grande como lo que llamamos persona\ldots{} Al quejido siguieron
voces que indudablemente delataban espanto de mujer\ldots{} Dolorido del
cuello y de los lomos, inclinose Ansúrez vomitando blasfemias, y vio
ropas negras y blancas\ldots{} El bulto calló, como si de la conmoción
de su caída perdiera el conocimiento, y el hombre, para verlo mejor, se
puso de rodillas diciendo: \emph{«¡Ajos, cebollas, berenjenas y
cohombros!}\ldots{} Yo pensé que era un pedazo de torre o un cacho de
cornisa, y ahora veo que es usted una monja\ldots{} Por poco me mata en
su caída\ldots{} diré mejor en su fuga\ldots{} ¿Se descolgaba usted con
esa soga que tiene en las manos?\ldots{} \emph{¡Ajos y cebolletas!} ¿Por
qué no cogió un chicote de más poder?\ldots{} ¿Se le rompió antes de
llegar al suelo?\ldots{} Ya pudo avisar, señora, y yo me habría puesto
en facha para recogerla\ldots{} Por las verijas de San Pedro, que me ha
derrengado un hombro, y me ha roto una oreja\ldots{} y en el quiebro que
hice creyendo que se me venía encima una torre, pienso que me he roto
por la cintura, del dolor que siento, ¡ay!\ldots{} A ver, comadre, si
puede levantarse\ldots{} ¡upa! No puede\ldots{} ¡Upa otra vez,
valiente!\ldots»

La señora monja parecía cuerpo muerto: sus manos ensangrentadas
agarraban la cuerda tosca con presión formidable de los dedos, como si
aún estuviera pendiente de ella; su rostro encendido, su boca
entreabierta y muda, expresaban terror; sus ojos abiertos parecían
privados de la visión\ldots{} No tardó Ansúrez en acometer el más airoso
lance de aquella singular aventura, y movido de su caridad o de su
gallardía caballeresca, probó a levantar en peso a la caída y derrengada
monja. Al primer esfuerzo, su energía titánica flaqueó por efecto del
quebranto que en su propio cuerpo sentía; pero estimulados los músculos
potentes por la más briosa voluntad que puede imaginarse, el atleta tomó
en brazos a la señora y la llevó por el dédalo de calles, diciéndole:
«Comprendo que su reverencia se ha escapado como ha podido\ldots{} ¿Qué
ha sido? ¿Malos tratos?\ldots{} ¿ganitas de volver al siglo?\ldots{}
Serénese, y como no tenga su reverencia hueso roto, haga cuenta de que
el salto ha sido feliz, y que no ha pasado nada.»

No era saco de paja la mujer caída; antes bien, notó Ansúrez la carnosa
opulencia de las partes próximas al apretón de los brazos de él. Por dos
veces tuvo que aliviarse del peso para tomar resuello, y al fin dio con
su preciosa carga en la posada donde tenía su alojamiento. Grande fue el
asombro del huésped y de los dos amigos que esperaban al patrón del
falucho para emprender el viaje a Denia. El primer cuidado de todos fue
tender el desmayado cuerpo en un fementido catre y proceder a su
reconocimiento, por si las partes lastimadas en la caída reclamaban
auxilio del médico. No fue cosa fácil el examen, porque la esposa del
Señor opuso toda la resistencia que su remilgado pudor monjil le
imponía. Declaró que bien podían reconocerle cabeza y brazos; pero que a
la jurisdicción de las piernas no permitiría que llegase mirada de
hombres, aunque en aquella zona tuviese todos los huesos partidos y
deshechos\ldots{} Respetaron los discretos varones estos refinados
escrúpulos, y serenándose más a cada instante la buena mujer, les dijo
que sentía magulladuras dolorosas y quebranto en diferentes partes de su
cuerpo venerable; pero que no creía tener fractura en ninguna pieza de
su esqueleto, agregando que sufriría con paciencia, y hasta con gozo,
todas las averías de la máquina corpórea, con tal de ver para siempre
conquistada su libertad. Mientras así hablaba la monja, pudo hacerse
cargo el buen Ansúrez de que su rostro no carecía de belleza y gracias,
y apreciar la excelente proporción de partes y formas ocultas por el
hábito dominico.

La mujer y criada del posadero encargáronse de curar y bizmar las
erosiones y rozaduras de la religiosa, y de aplicarle compresas de
vinagre allí donde era menester. Luego, por indicación del marino,
quitáronle hábito y toca, vistiéndola con las prendas usuales del traje
popular valenciano. Esta rápida metamorfosis dio mayor tranquilidad a la
fugitiva del claustro. Ansúrez, que gradualmente se hacía dueño de la
situación, recomendó a la familia posaderil que guardara impenetrable
secreto sobre aquel extraño caso, y a la señora propuso que se dejase
llevar fuera de la ciudad, pues no estaría segura mientras no pusiese
entre su persona y el convento grandes espacios de tierra y de mar.
Aceptó la señora sin vacilación, que su espanto le daba prisa, y alas le
ponía su atrevimiento. «Vamos, buen hombre; lléveme a donde
quiera---dijo echándose del lecho y recorriendo la estancia con la
cojera que le imponían sus doloridas coyunturas.---Lléveme lejos, lejos,
a donde no puedan alcanzarme.»

Con el apremio que requerían las circunstancias dispuso Diego la
partida. Pronta estaba la tartana. En ella metieron a la monja,
acomodándola con almohadas y ropa de abrigo, y añadiendo mediano
cargamento de provisiones de boca. Con Ansúrez y su venturoso hallazgo
entraron en el coche dos amigos del primero: un marinero tortosino y un
traficante balear. Partieron a escape\ldots{} A las ocho de la mañana
entraban en Denia, y sin detenerse en las calles corrían hacia el
puerto. Antes de las nueve estaban a bordo del falucho, el cual,
acelerando su despacho y listo de papeles y víveres, dio sus velas al
viento, que era nordeste fresco y traía el lento son de las campanadas
con que el reloj consistorial cantaba las once\ldots{} Recostada en la
borda, la prófuga lloraba de alegría, viendo alejarse el caserío
dianense, las alturas del Mongó\ldots{} después las rocas y el faro del
cabo San Antonio\ldots{} Creía soñar\ldots{}

\hypertarget{ii}{%
\chapter{II}\label{ii}}

La continuación de estas noticias biográficas dejó en la memoria de
\emph{Confusio} y Donata los puntos más salientes, a saber: la edad de
la monja fugada no pasaba, según su cuenta, de los veintiséis años. En
el siglo llamábase Angustias, y había nacido en un pueblo próximo a
Granada, de familia buena y humilde. Mal sonaba en los oídos de Ansúrez
el tristísimo nombre de la que, arrojada de los aires y cayendo sobre él
como un bólido, fue coscorrón y donativo de la Providencia; y así,
cuando llegaron a completa concordia y se avinieron a recorrer juntos la
cuesta de la vida, resolvió él con franca autoridad rebautizarla y
ponerle nombre de \emph{Esperanza}, que al ser pronunciado ensancha el
corazón en vez de oprimirlo\ldots{} Al mes no entero de la evasión
efectuaron sus bodas, sin más trámite que su firme voluntad de correr
igual suerte en lo futuro; y el día de Navidad de aquel mismo año 49 dio
a luz doña Esperanza, en Palma de Mallorca, una niña, que puesta debajo
de la advocación y patrocinio de la Virgen del Mar, se llamó Marina, y
por elipsis del habla familiar quedó para siempre con el breve nombre de
\emph{Mara}. Este hecho del nacimiento de la criatura demuestra que los
desconciertos morales y canónicos podrán traer efectos revolucionarios
en el terreno legal; pero no traen el acabamiento de la especie humana,
la cual, contra viento y marea, continúa cantando bajito el himno de su
fecundidad.

Supieron asimismo Donata y \emph{Confusio} que el buen Ansúrez, hacia el
50, viéndose perdido en sus negocios de cabotaje, entró por segunda vez
en el servicio de la Armada. Tres veces fue a las Antillas, corrió toda
la mar Caribe, y por fin, en la Expedición científica al Pacífico, pasó
de ida y de vuelta el temeroso Estrecho de Magallanes. En estos viajes,
con descansos periódicos en Cartagena, transcurrieron diez años. El 60,
cumplido el plazo de enganche, restituyose Diego a su hogar y familia,
trayendo sus ahorros y algún dinero ganado en América con el toma y daca
de pacotillas. Era su propósito emprender de nuevo el tráfico costero, y
a este fin compró dos naves, abanderadas la una en Cartagena, la otra en
Palamós. En el primer viaje de esta, entró de arribada en Amposta para
el reparo de averías; y mientras permaneció en Tortosa, ocurrieron
sucesos para él memorables: el suplicio de Ortega, la captura de los
Príncipes y el conocimiento con Donata y \emph{Confusio}. Ya se ha dicho
que estos navegaron con su generoso amigo, visitando puertos del
archipiélago balear y de la Península; queda por decir que en un
pueblecito del Mar Menor, cerca de Cabo Palos, conocieron a doña
Esperanza, esposa putativa de Ansúrez, y a su preciosa hija Mara. En la
primera vieron una señora muy reservada y seria, de belleza fría y sin
encanto, la expresión del rostro más de escultura que de persona viva,
la mirada brillante y quieta, como de imagen barnizada. En cambio, la
chiquilla era una morenita salada y picaresca, pimpollo de gracias
infantiles, que anunciaba la mujer pertrechada de seducciones.

En este punto se desvanece la Historia, y los sucesos se diluyen por la
dispersión de los seres que los informan. Donata y su caballero se
establecen en Cartagena, luego en Murcia. Leves divergencias de carácter
y de gustos se manifiestan en ellos; a las discordias menudas suceden
reconciliaciones tibias; la inarmonía crece; menguan los halagos;
rómpese de súbito un vivo fuego de guerrillas; al desamor sucede la
antipatía\ldots{} y por fin, Donata corre a satisfacer sus ambiciones
del alma en la servidumbre y compañía de un opulento canónigo,
aristocrático y elegante. Deslumbraban a la discípula del Arcipreste de
Ulldecona los ricos atavíos eclesiásticos, las áureas dalmáticas y
casullas, las albas vaporosas, las sotanas de sarga, olientes a raíz de
lirio, o a exquisito rapé del de \emph{la Orza del Papa}. Fastuosamente
vivía el capitular en un palaciote viejo, ornado de muebles arcaicos y
de objetos primorosos. Toda la casa hallábase impregnada de una sutil
fragancia de cedro, sándalo y otras maderas exóticas. La profusión de
fino damasco en cortinas, colchas y almohadones, así como la riqueza de
plata labrada, hacían creer a la simplona Donata que tenía por amo a un
cardenal. Dolorido al principio, pronto consolado, contento al fin de su
divorcio, \emph{Confusio} partió a Madrid ansioso de contar sus buenas y
malas andanzas al Marqués de Beramendi y a Manolo Tarfe.

Si el 60 fue en gran parte venturoso para Diego Ansúrez, el 61 empezó
desgraciado: florecieron y fructificaron sus negocios, y doña Esperanza,
descubierta y reconocida por su familia, entró con esta en relaciones
muy cordiales. Se le perdonaba su escapatoria del convento; se admitía
como ley circunstancial la fuerza de los hechos consumados, y se
declaraba triunfante el nuevo estado de derecho, olvidando su origen
revolucionario y sacrílego. Tanto los hermanos de ella, Matías y Segunda
Castril, como los demás Castriles, parientes próximos y lejanos, que
residían en Loja, en Granada y en Iznalloz, proclamaron a una el indulto
de Angustias y al cariño de toda la familia querían traerla, legitimando
la situación creada por el tiempo y las pasiones humanas. Don Prisco
Armijana y Castril, cura del Salar, tomó a su cargo las gestiones para
obtener dispensa, y santificar la diabólica unión de la monja y el
navegante. Pero las alegrías de Ansúrez por estas disposiciones y
propósitos de la familia de su mujer, se nublaron viendo a esta
rápidamente desmejorada en su salud, sin que los médicos supieran atajar
la dolencia traidora.

En la creencia de que los aires del país natal serían eficaces para la
enferma, Diego la llevó a Lanjarón, de allí a Granada, y por fin a Loja,
donde Esperanza se repuso un poco. Vivían con Segunda Castril, esposa de
un don Cristino López, propietario de un buen olivar y tierras de
sembradura en término del Tocón. Contenta estaba doña Esperanza en la
compañía de su hermana, y no cesaba de recordar con ella los tiempos
infantiles, los rigores del padre, que, por la sola razón de tener
abundancia de hijas y escasez de peculio, metió a una en las
Franciscanas y a otra en las Dominicas de Granada. Con artificiosa
vocación entró Angustias en la comunidad; por ser algo díscola y más que
rebelde a la observancia reglar, fue trasladada al convento de Játiva,
donde, como es sabido, meditó y llevó a feliz término su evasión por el
tejado, sin más socorro que el de una soga. Esta hizo la gracia de
rompérsele con una oportunidad que indudablemente fue obra del cielo.
Cortaron los ángeles la cuerda, y a los diez meses nació Mara.

Entretenida fue para doña Esperanza y su hija la existencia en Loja,
pues no faltaban los quehaceres domésticos ni las relaciones fáciles y
amenas, y además gozaban de las delicias del campo en épocas de
recolección, matanza o trasquila. Si distraídas y alegres vivían las
hembras, don Diego (que así llamaban al navegante sus amigos de Loja,
rodeándole de afectuoso respeto) se sentía confuso y atontado, pues
ajeno hasta entonces a las querellas de la política, veíase transportado
a un vertiginoso torbellino de pasiones y antagonismos locales. El
vecindario de Loja habíase dividido en dos bandos, que se aborrecían, se
acosaban y se fusilaban sin piedad: liberal era el uno, \emph{moderado}
llamaban al otro. No salía el buen Ansúrez de la perplejidad en que el
sentido y la aplicación de esta palabra le puso, pues siempre creyó que
la moderación era una virtud, y en Loja resultaba la mayor de las
abominaciones y el mote infamante de la tiranía. Sin darse cuenta de
ello ni poner de su parte ninguna iniciativa, desde los primeros días se
sintió afiliado al bando liberal, por ser de esta cuerda todos los
Castriles y Armijanas, y los amigos de estos.

No causaron al hombre de mar poca maravilla las noticias que le dio su
concuñado don Cristino de la organización y disciplina masónica que se
impusieron los liberales, para formar un haz de combatientes con que
tener a raya el poder ominoso de la \emph{Moderación}. Esta no era más
que un retoño de la insolencia señorial en el suelo y ambiente
contemporáneos; el feudalismo del siglo {\textsc{xiv}}, redivivo con el
afeite de artificios legales, constitucionales y dogmáticos, que muchos
hombres del día emplean para pintarrajear sus viejas caras medioevales,
y ocultar la crueldad y fieros apetitos de sus bárbaros caracteres.
Representaba el feudalismo la Casa y Condado de La Cañada, en quien se
reunían el ilustre abolengo, la riqueza, el poderío militar de Narváez y
su inmensa pujanza política. Hermanos eran el famoso \emph{Espadón} y el
caballero que imperar quería sobre las vidas, haciendas, almas y cuerpos
de los habitantes de Loja. Sin duda, aquel noble señor y su familia
obedecían a un impulso atávico, inconsciente, y creían cumplir una
misión social reduciendo a los inferiores a servil obediencia; procedían
según la conducta y hábitos de sus tatarabuelos, en tiempos en que no
había Constituciones encuadernadas en pasta para decorar las bibliotecas
de los \emph{centros políticos}; no eran peores ni mejores que otros
mandones que con nobleza o sin ella, con buenas o malas formas,
caciqueaban en todas las provincias, partidos y ciudades de este vetusto
reino emperifollado a la moderna. Los perifollos eran códigos, leyes,
reglamentos, programas y discursos que no alteraban la condición
arbitraria, inquisitorial y frailuna del hispano temperamento.

Contra la soberanía bastarda que la nobleza y parte del estado llano
establecieron en Loja, la otra parte del estado llano y la plebe armaron
un tremendo organismo defensivo. Por primera vez en su vida oyó entonces
Ansúrez la palabra \emph{Democracia}, que interpretó en el sentido
estrecho de protesta de los oprimidos contra los poderosos. Democrática
se llamó la Sociedad secreta que instituyeron los liberales para poder
vivir dentro del mecanismo caciquil; y en su fundamento apareció con
fines puramente benéficos, socorro de enfermos, heridos y
valetudinarios. Debajo de la inscripción de los vecinos para remediar
las miserias visibles, se escondía otro aislamiento, cuyo fin era
comprar armas y no precisamente para jugar con ellas. Dividíase la
Sociedad en Secciones de veinticinco hombres que entre sí nombraban su
jefe, secretario y tesorero. Los jefes de Sección recibían las órdenes
del Presidente de la Junta Suprema, compuesta de diez y seis miembros.
Esta Junta era soberana, y sus resoluciones se acataban y obedecían por
toda la comunidad sin discusión ni examen. Engranadas unas con otras las
Secciones, desde la ciudad se extendieron a las aldeas y a los remotos
campos y cortijos, formando espesa red y un rosario secreto de
combatientes engarzados en a autoridad omnímoda de la Junta Suprema.

A todos los afiliados se imponía la obligación de poseer un arma de
fuego. A los menesterosos que no pudiesen adquirir escopeta o trabuco,
se les proporcionaba el arma por donación a escote entre los
veinticinco. Cada Sección estaba, de añadidura, obligada a suscribirse a
un diario democrático, que era regularmente \emph{La Discusión o El
Pueblo}. Cuando alguna Sección trabajaba en faenas campesinas a larga
distancia de la ciudad, enviaban a uno de los de la cuadrilla a recoger
el periódico (o folleto de actualidad, cuando lo había); y en la
ausencia del mensajero, los trabajadores que quedaban en el tajo hacían
la parte de labor de aquel. Un tal Francisco Navero, apodado
\emph{Tintín}, repartía los papeles democráticos a los enviados de cada
Sección. En estas había un individuo encargado de leer diariamente el
periódico a sus compañeros en las horas de descanso.

La Junta Suprema limitaba a los asociados el uso del vino, y prohibía en
absoluto el aguardiente. Gran sorpresa causó a don Diego saber que por
esta \emph{moderación} de los liberales se arruinaron muchos taberneros,
y llegaron a ser escasísimos los puestos de bebidas. El número de
afiliados creció prodigiosamente desde que comenzaron, en la ciudad y
luego en cortijos y villorrios, los solapados trabajos de propaganda. La
iniciación se hacía en lugar secreto que Ansúrez no pudo ver: allí se
les leía la cartilla de sus obligaciones, y se les tomaba juramento
delante de un Cristo que para el caso sacaban de un armario. Afiliados
estaban no pocos servidores del Conde de la Cañada. En el propio caserón
o castillo roquero del cacique feudal se sentía la continua labor de
zapa del monstruoso cien-pies que minaba la tierra.

La Sociedad, en cuanto se creyó fuerte, no quiso limitarse a la defensa
ideológica de los derechos políticos. Los principales fines de la
oligarquía dominante eran ganar las elecciones, repartir a su gusto los
impuestos cargando la mano en los enemigos, aplicar la justicia conforme
al interés de los encumbrados, subastar la Renta (que así llamaban
entonces a los Consumos) en la forma más conveniente a los ricos, y
establecer el reglamento del embudo para que fuese castigado el matute
pobre, y aliviado de toda pena el de los pudientes. Con tales maniobras,
no sólo era reducido el pueblo a la triste condición de monigote
político, sin ninguna influencia en las cosas del procomún, sino que se
le perseguía y atacaba en el terreno de la vida material, en el santo
comer y alimentarse, dicho sea con toda crudeza.

Frente a esto, la poderosa Sociedad buscaba inspiración en la Justicia
ideal y en el sacro derecho al pan, y decretó la norma de jornales del
campo, estableciendo la proporción entre estos y el precio del trigo.
Véase la muestra. ¿Trigo a cuarenta reales la fanega? Jornal: cinco
reales. Al precio de cincuenta correspondía jornal de seis reales, y de
ahí para arriba un real de aumento por cada subida de diez que obtuviera
la cotización del trigo. Accedieron algunos propietarios; otros no. Los
jornaleros segadores se negaron a trabajar fuera de las condiciones
establecidas, y en las esquinas de Loja aparecieron carteles impresos
que decían poco más o menos: \emph{«Todos a una} fijamos el precio del
jornal. Si no están conformes, quien lo sembró que lo siegue.»

Clamaron no pocos propietarios, y al cacicato acudieron pidiendo que
fuese amparado el derecho a la ganancia. La cárcel se llenó de
trabajadores presos, y tal llegó a ser su número, que no cabiendo en las
prisiones, se habilitaron para tales el Pósito y el convento de la
Victoria. Pero no se arredró por esto la Sociedad, que en su tenebrosa
red de voluntades tenía cogidos a todos los gremios. El buen éxito de la
escala de jornales para el trabajo rural movió a la Junta a continuar el
plan defensivo, justiciero a su modo. Peritos agrícolas afiliados a la
Comunidad revisaron los arrendamientos, y en los que aparecieron muy
subidos, se despedía el colono. El propietario quedaba en la más
comprometida situación, pues no encontraba nuevo colono que llevara su
tierra, ni jornaleros que quisieran labrarla. Igual campaña que esta del
campo hicieron los peritos urbanos o maestros de obras en el casco de la
ciudad. Casa que tuviera demasiado alto el alquiler, según el dictamen
pericial, quedaba desalojada, y ya no había inquilinos que quisiesen
habitarla, como no fueran los ratones. Llegó, por último, a tal extremo
la unión, confabulación o tacto de codos, que ningún asociado compraba
cosa alguna en tienda de quien no perteneciese a la secreta Orden de
reivindicación y libertad.

Sorprendido y confuso el buen Ansúrez, oyó hablar de Socialismo y
Comunismo, voces para él de un sentido enigmático que a brujería o arte
diabólica le sonaban. Poseía el vocabulario de mar en toda su variedad y
riqueza; pero su léxico de tierra adentro era muy pobre, y singularmente
en política no encontraba la fácil expresión de sus pensamientos. Sabía
que teníamos Constitución, Reina, Cortes, partidos Progresista y
Moderado; pero ni de aquí pasaba su erudición, ni entendía bien lo que
estas palabras significaban\ldots{} En tanto, ocurrían en Loja y su
término sangrientos choques: una noche apaleaban a un asociado, y a la
noche siguiente aparecía muerto en la calle un testaferro de los Narváez
o un machacante del Corregidor. Las agresiones, las pedreas y navajazos
menudeaban; la Guardia Civil acudía, siempre presurosa, de la ciudad al
campo, o del campo a la ciudad; las voces de ira y venganza sonaban más
a menudo que las expresiones de galantería dulce y quejumbrosa que
caracterizan al pueblo andaluz en aquel risueño y templado territorio.
La Naturaleza callaba cuando los corazones ardían en recelos, y las
bocas agotaban el repertorio de las maldiciones.

Todo esto lo vio Ansúrez en la ciudad y en el cortijo del Tocón, donde
pasó algunas semanas, huésped de su cuñado Matías Castril. Y para que
nada le quedase por ver, llegó tiempo de elecciones, y los dos enconados
bandos, furia narvaísta y furia popular, dieron la trágica función de
disputas, celadas, recíprocos engaños, escandaleras y trapisondas
horribles. Cruelmente y sin piedad se trataban unos a otros. Represalias
morales había no menos duras que las de la guerra. Al grito de ojo por
ojo que estos proferían, contestaban aquellos con el grito feroz de
cabeza por cabeza. El inocente y honrado Ansúrez, testigo por primera
vez de la bárbara porfía, que era por una parte y otra un burlar
continuo de todas las leyes, exceptuando la de la fuerza bruta, no podía
compararla con nada de cuanto él había visto en sus vueltas por el
mundo. Más conocedor de la Naturaleza que de los hombres, veía en
aquellas agitaciones, designadas con mote político, electoral,
socialista o comunista, una vaga semejanza con las turbulencias de mar.
Cerrando los ojos ante la terrible lucha del pueblo con el feudalismo,
su cerebro le reproducía el silbar furioso de los vientos
desencadenados, y la hinchazón de las olas que corren acosándose y
mordiéndose hasta perderse en el horizonte sin fin.

\hypertarget{iii}{%
\chapter{III}\label{iii}}

Hallábase el navegante fuera de su centro, y la nostalgia del mar y del
trajín costero entristecía sus horas. Por su gusto allá se volvería;
pero su mujer le sujetaba con el descanso que la tierra natal y la
familia daban a sus achaques, y su hija Mara con la intensa afición que
iba tomando al suelo y a la gente de Andalucía. De tal modo reinaban en
su corazón los dos seres queridos, hija y esposa, que al gusto de ellas
subordinaba siempre su conveniencia y toda su voluntad. Las labores del
campo, que al principio le interesaban y distraían, ya le causaban
tedio. La mar inquieta era su campo, que él araba con la quilla de sus
naves para extraer el fruto comercial, único verdadero y positivo. Según
él, las bodegas de los barcos son como estómagos que reciben y dan toda
la sustancia de que se nutre el cuerpo de la Humanidad.

A Loja iba algunas tardes con su cuñado Matías y dos compadres de este.
La última vez que estuvo en la ciudad, pasó largo rato en el café,
respirando espesa atmósfera de humo y rencores, y oyendo el mugido de
las disputas, para él más pavoroso que el de las tempestades. Allí
conoció a Rafael Pérez del Álamo, inventor y artífice principal de aquel
tinglado de la organización democrática y socialista. Embobado le oía
referir sus audacias, y tanto admiraba su agudeza como su indomable
tesón. Aunque parezca extraño, Ansúrez sentía en sí mismo cierta
semejanza con Rafael Pérez. Ambos luchaban con poderes superiores: el
uno con los elementos naturales, el otro con los desafueros del orgullo
humano. Y siendo en su interna estructura tan semejantes, diferían
sensiblemente en la proyección de sus voluntades, llegando a ser
ininteligibles el uno para el otro. Si Ansúrez no comprendía el heroico
trajín de las revoluciones políticas, Rafael Pérez desconocía en
absoluto los heroísmos de la mar. Falta decir que el organizador del
pueblo contra las demasías del poder constituido era un pobre albéitar,
que se ganaba la vida herrando caballos y mulas.

En la última visita que hizo al café, conoció también Ansúrez a uno de
los principales mantenedores del feudalismo narvaísta, don Carlos
Marfori, joven vigoroso y resuelto, emparentado con la familia del
General. Distinguíase por la temeraria llaneza con que descendía de su
posición para discutir con los caudillos de la plebe, cara a cara, las
candentes cuestiones que enloquecían a todos. Invitaba Marfori a Rafael
Pérez a tomar café juntos. Alardeaba el albéitar de convidar a don
Carlos y a los caballeros y genízaros que le acompañaban. Bebían
disputando, juraban, y confundían sus voces airadas sin llegar a las
manos. Por la noche era ella. La contenida saña con que debatían el
villano y el noble, estallaba en las obscuras calles. Por un daca esas
pajas se embestían los dos bandos. Palos, cuchilladas y muertes eran la
serenata usual de las noches que, por ley de Naturaleza, debían ser
plácidas en aquel delicioso rincón de Andalucía.

Recluido en el campo, el pobre navegante sobrellevaba sus añoranzas con
la paz y los goces de la familia. Doña Esperanza no empeoraba, y su
mortal inapetencia se iba remediando con los guisos y golosinas de la
tierra. La chiquilla era un portento de agudeza y precocidad, y el mayor
alivio de las penas de su padre, que la amaba con delirio y no ponía
freno a sus antojos. En Mara, el desarrollo espiritual y físico de la
niña traía tempranamente las gracias de mujer hecha y bien plantada. El
suelo y aire andaluz habían extremado la ligereza de sus pies, y la
flexibilidad de su cuerpecillo en el baile, en los andares, hasta en el
saludo. Habíase asimilado el ceceo de la tierra, el donaire anecdótico,
el arte de las réplicas prontas, epigramáticas, chispeantes de sal y
donosura. Mara reinaba en el corazón de todos, y era para sus padres el
sol de la vida.

Pasaron días; avanzaba el verano; la familia de Ansúrez, invitada por el
cura del Salar, fue a pasar un par de semanas en la casa de este, que
era de gran desahogo y abundancia. Mas no quiso Dios que los forasteros
hallaran tranquilidad junto al generoso don Prisco, porque a los seis
días de su llegada al Salar echó al campo la conjura democrática todas
sus legiones, y la tierra de Loja fue como un volcán que por diferentes
cráteres arroja su fuego. Ya sabía don Prisco que Rafael Pérez preparaba
un alzamiento general, mas no pensaba que fuese para tan pronto.
Diferentes rumores contradictorios llegaron al Salar. Según unos, el
albéitar, preso y encarcelado por el Corregidor, se había escapado de la
prisión, corriendo con sus leales amigos camino de Antequera; según
otros, en Antequera prendieron al herrador, metiéndole en un calabozo
subterráneo, y hacia allá iban decididos a salvarle sus más ardientes
partidarios. De la noche a la mañana, no quedaron en el Salar más que
mujeres, chiquillos y algunos viejos. Salió don Prisco en averiguación
de lo que pasaba; aproximose a los arrabales de Loja; volvió a su casa
sobrecogido y algo tembloroso, diciendo a su sobrina y a sus huéspedes
que la insurrección no era cosa de broma, y que no tardarían en
sobrevenir acontecimientos de padre y muy señor mío.

Aunque el reverendo Armijana era de los buenos amigotes de Rafael Pérez
del Álamo, y sentía por la Sociedad toda la simpatía compatible con la
prudencia sacerdotal, viendo las cosas tan lanzadas a mayores y la
revolución sacada de la obscuridad masónica a la luz de la realidad,
echose atrás el hombre, y no cesaba de pedir a Dios que devolviese la
paz a los ciudadanos. \emph{«Camará---}dijo a don Diego, refiriéndole lo
que había visto,---esto no va por el camino natural, y para mí que al
amigo Rafael se le ha metido algún diablo en el cuerpo\ldots{} Arrimado
al ventorro de Lucas vi pasar \emph{un porción} de hombres que gritaban
como locos. Daban vivas calientes a la Libertad y al Democratismo, y
mueras fríos a doña Isabel, a los Narváez y al Corregidor. Cuando me
vieron, soltaron el grito escandaloso de \emph{¡muera el Papa!}\ldots{}
Por la sotana que llevo, que quise protestar\ldots{} pero no me atreví.
Las turbas armadas empezaron a echar por aquellas bocas tacos y
porquerías horripilantes, no sólo contra el Sumo Pontífice, sino contra
la Virgen Nuestra Señora; y Curro \emph{Tintín}, el vendedor de
periódicos, me amenazó con la escopeta y me dijo que se chiflaba en San
Torcuato, el santo de mi mayor devoción, como hijo de Guadix que soy.
Esto, amigo Ansúrez, pasa de la raya, y yo digo que si no nos manda
tropas el Gobierno de O'Donnell es porque el \emph{gachó} quiere
perdernos, envidioso del poder de Narváez\ldots{} Tropas, vengan tropas,
o nos veremos muy mal, pero que muy mal.»

Apenas enterado de lo que ocurría, Ansúrez no pensó más que en
trasladarse a Granada con su familia; pero cuantas diligencias hizo
aquella tarde para encontrar caballerías o un carricoche, resultaron
inútiles. A la mañana siguiente, se supo que toda la caterva de paisanos
armados se encontraba en Iznájar, Aventino andaluz, donde la plebe se
organizaría con marcial unidad y compostura para ir sobre Roma. Roma, o
sea Loja, era desalojada por los narvaístas, que escapaban medrosos,
llevándose cuanto de valor poseían. Con ellos abandonaron la ciudad el
Corregidor y las escasas fuerzas de Guardia Civil y Carabineros que allí
tenía el Gobierno. De este dijeron los \emph{moderados} que estaba en
connivencia con los insurrectos, y que todo era obra del masonismo, del
protestantismo y de la marrullería de O'Donnell y Posada Herrera, en
quienes el orden no era más que una máscara hipócrita para engañar al
Trono y al Altar. ¿Qué hacían que no mandaban tropas? Esto llegó a ser
en don Prisco idea fija. El buen señor terminaba todas sus peroratas,
como todos sus rezos, con la devota exclamación de «¡Soldados,
soldados!»

No cejaba el pobre Ansúrez en su afán de ausentarse con la familia,
apretándole a ello el grave susto de doña Esperanza y su horror ante la
tragedia. Al menor ruido temblaba la infeliz señora, creyendo escuchar
cañonazos próximos; sus males se acerbaban, y el sueño no quería cuentas
con ella. Por el contrario, la inocente Mara gustaba de la trifulca,
ansiaba ver sucesos extraordinarios y encuentros formidables de hombres
con hombres. Su viva imaginación extraía de los hechos más vulgares la
leyenda poemática. A pesar de esto, viendo a su madre tan empeorada de
puro medrosa, no cesaba de decir: «Vámonos, padre, y que nos acompañe
María Santísima.» Y don Prisco, en vez de \emph{ora pro nobis}, repetía:
«¡Soldados, soldados!»

Buscando medios de transporte, se encontró al fin el borrico de un
salinero: esto por el pronto bastaba. Ansúrez y su hija irían a pie
hasta llegar a la Venta de Lachar, donde esperaban encontrar mejor
acomodo de viaje. Fue con ellos el cura don Prisco hasta el camino real,
y allí los despidió con frase zalamera, deseándoles la protección de la
Virgen, y agregando que esta sería más eficaz si el maldito Gobierno
enviara tropas en apoyo de los altos designios. Siguió adelante la
caravana, doña Esperanza en su borrico, mal encaramada en un sillín de
tijera; la hija y el marido a pie, por un lado y otro, sosteniéndola
para que no se cayese, y delante el vejete salinero, que marcaba el paso
con un tristísimo cantorrio entre dientes.

Diego Ansúrez, cuya mollera continuaba cerrada para las cosas de tierra
adentro, no cesaba de meditar en ellas, buscando una clave de las
absurdas contradicciones que veía. ¿Por qué se peleaban los hombres en
aquel delicioso terreno, en aquellos risueños valles fecundísimos que a
todos brindaban sustento y vida, con tanta abundancia que para los
presentes sobraba, y aun se podía prevenir y almacenar riqueza para los
de otras regiones? La sierra fragosa enviaba a las vegas lozanas el
torrente de sus aguas cristalinas. Daba gloria ver la riqueza que
descendía por aquellas encañadas, la cual asimismo prodigaba tesoros de
sal, mármoles y ricos minerales. Las lomas de secano se cubrían de
olivos, almendros y vides lozanas; en las vegas verdeaban los opulentos
plantíos de trigo, cáñamo, y de cuanto Dios ha criado para la industria,
así como para el sustento de hombres y animales\ldots{} Si los que en
aquella tierra nacieron podían decir que habitaban en un nuevo Paraíso
terrenal, ¿para qué se peleaban por el mangoneo de Juan o Pedro, o por
el reparto de los bienes de la Naturaleza, que en tal abundancia
concedían el suelo y el clima? ¿Quién demonios había traído aquel
rifirrafe de la política, de las elecciones, y aquel furor porque
salieran diputados o concejales estos o los otros ciudadanos? Ansúrez no
lo entendía, y razonando en términos más rudos de los que en esta
relación histórica se indican, acababa por declarar que o los españoles
son locos sueltos en el manicomio de su propia casa, o tontos \emph{a
nativitate}.

Rendidísimos llegaron todos a la Casa de Postas de Lachar, ya entrada la
noche. Doña Esperanza no podía tenerse, y fue menester llevarla en
brazos a un camastro que en el único aposento vividero de aquel caserón
se le ofrecía. Lejos de restablecerse de su pánico, la fatiga y
quebranto del viaje la pusieron en mayor desazón, la cual iba labrando
la ruina en su ánimo más que en su cuerpo. El sueño no vino a calmarla,
por más sugestiones que se hicieron para provocarlo; negábase a tomar
alimento, que si los manjares eran malos, el asco invencible de la
enferma los hacía peores. Ansúrez no sabía, en tal situación, a qué
santo encomendarse. Discurrió enviar un propio al Tocón para que la
familia acudiese en su auxilio: no pudo encontrar para tal servicio más
que a una muchachuela jorobadita, y esta fue y tardó diez horas en
volver con la noticia de que don Matías estaba \emph{en la faición}, y
que las señoras no podían moverse de la casa. No había más remedio que
revertirse de paciencia y esperar lo que dispusiese la Divina Voluntad.
El salinero se despidió, ansioso de agregar su burro a la Caballería
ligera de Rafael; y como la Casa de Postas no podía proporcionar medios
de transporte, pues todos los caballos y mulas se los habían llevado los
señores de Loja en su retirada, resolvió don Diego quedarse allí en
espera de cualquier contingencia favorable.

Tan abrumado, tan fuera de su equilibrio natural estaba el navegante
celtíbero, que no se daba cuenta del tiempo que en aquella lúgubre y
calmosa expectación transcurría. Doña Esperanza languidecía por falta de
alimento, sin que a la soledad de aquel mechinal desamparado se le
pudiera llevar el socorro de médico y medicinas. Mara no se apartaba de
ella; Ansúrez hacía sus escapadas al corralón solitario, donde
únicamente hallaba un par de vejestorios que le ponían al tanto de los
acontecimientos. Los insurrectos, reunidos en Iznájar, descendían
orillas abajo del Genil, y en orden y aparato de guerra caminaban hacia
Loja, de cuyo desamparado recinto se apoderaban, poniendo allí su
capital democrática y el asiento de su fuerza civil y militar. Ya eran
dueños de Roma; ya ocupaban y guarnecían el alto castillo, que de los
moros conserva el nombre de Alcazaba; ya fortificaban los robustos
edificios que fueron conventos, y abrían trincheras en todos los puntos
indefensos de la ciudad. Considerable número de combatientes, que en
totalidad no bajaban de cinco mil, se alojaban en la iglesia Mayor, en
San Gabriel, en Jesús Nazareno y en el santuario de la Caridad, donde
residía la patrona del pueblo. Como no quitaba lo democrático a lo
piadoso, casi todos los prosélitos del temerario Rafael Pérez confiaban
en que nuestra Señora de la Caridad les diese la victoria sobre la
insufrible tiranía. Contaron a don Diego aquellos vejetes que al huir de
Loja los \emph{moderados} quisieron llevarse a la santa patrona de la
ciudad; pero que no les fue posible arrancar la imagen de la peana que
desde inmemorial tiempo la sostenía. Ni con palancas ni con ninguna
suerte de artificios lograron despegarla. Peana y Virgen pesaban tanto,
que ni con cien mil pares de bueyes habrían podido apartarla ni el canto
de un duro, señal de que la Señora no quería cuentas con los narvaístas,
y protegía resueltamente al democrático albéitar Rafael Pérez.

Como Ansúrez no diera crédito a esta conseja, la confirmó con juramentos
y arrumacos una gitana vieja que de Loja venía, agregando que Rafael
tenía ya más poder que el santo ángel de su nombre.

\hypertarget{iv}{%
\chapter{IV}\label{iv}}

Las desgracias del valeroso navegante, que tan furioso temporal corría
tierra adentro, no tenían término ni alivio. Confinado con su familia en
una estrecha y miserable celda del piso alto de la Casa de Postas, no
hallaba medio de proseguir avante ni atrás en el viaje emprendido. Daba
el aposento a un corredor que se extendía por dos lados del patio, y en
el término de una de estas alas estaba la escalera. El blanqueo de las
paredes dentro y fuera de la estancia no era reciente: la suciedad
reinaba en todo el edificio, y los olores de cuadra y cubiles discurrían
de un lado a otro como únicos inquilinos que allí sin estorbo moraban.

Lo peor fue que cuando doña Esperanza, en aparente mejoría, se prestaba
a pasar algún alimento, anocheció sosegada y amaneció en completo
desbarajuste de sus facultades mentales, que ya venían de días atrás
algo descaecidas. Debilitado por el no dormir y el no comer el cerebro
de la buena señora, dio esta en el más extraño desvarío que puede
imaginarse. Fue una retroacción de sus pensamientos, un salto atrás, un
desandar de lo andado en las vías del tiempo. A la madrugada, habíase
tendido Ansúrez en el suelo sobre unas enjalmas; despertole Mara ya de
día claro, diciéndole con palabras angustiosas que algo insólito y de
mucha gravedad ocurría. Lo primero que advirtió don Diego al abrir los
ojos fue que su esposa no estaba en el camastro. Como dormían vestidos,
no tardaron en salir del aposento hija y padre, y con espanto vieron a
doña Esperanza que a lo largo del corredor venía parloteando en alta voz
y gesticulando con demasiada viveza, como si disputara con seres
invisibles. Corrieron a ella, y con gran dificultad la llevaron adentro.

Opuso la buena señora resistencia breve, que se revelaba en su voz más
que en sus ademanes, diciendo: «Déjame, Diego, déjame, que esa tarascona
insolente, Sor Emerilda del Descendimiento, quiere meterme en la leñera.
¿No has oído a mis enemigas las valencianas aullar contra mí? La Priora
es de tierra de Jumilla y no me quiere mal; pero está impedida de ambas
piernas y no puede salir a defenderme. ¡Que no entren, por Dios, que no
entren en esta celda!\ldots{} Es lo que llamamos el desván de la fruta,
y aquí me recojo, aquí me refugio entre calabazas\ldots{} Tú eres el
hombre de los aires, que anda de chimenea en chimenea y horada los
techos\ldots{} Vienes manchado de hollín, porque pasas por los caminos
del humo\ldots{} Silencio, que las monjas vamos al coro\ldots{} En el
coro somos las monjas ángeles que rezan dormidos\ldots{} Despertamos, y
nos volvemos demonios\ldots» Estos y otros disparates que dijo la
señora, pusieron a la hija y al esposo en gran consternación. Con
palabras dulces trataron de apartar su mente de aquel furioso desvarío;
pero las ideas de la infeliz mujer se habían dispersado como pájaros,
cuya jaula se abre por las cuatro caras, y no había manera de atraerlas
de nuevo a su prisión.

Lejos de calmarse con halagos ni con esfuerzos de raciocinio la locura
de doña Esperanza, se fue determinando más en el curso del día, hasta el
punto de que Diego y Mara llegaron a creer que también ellos habían
perdido el juicio. Terrible fue la tarde: la pobre señora persistía en
la demencia de creerse monja, y de repetir en memoria y en voluntad los
actos y sucesos que precedieron a su evasión del claustro. Ya no sabían
el esposo y la hija qué pensar, ni qué hacer, ni qué decir. En vano
pedían auxilio a los viejos y mujeres de la casa, que no acertaban de
ningún modo a sacarles de tan doloroso conflicto. Por la noche, el
delirio de la enferma fue más desatinado y violento. Desconociendo a su
hija, la llamaba \emph{negra}, \emph{intrusa}, y mandábala salir de su
presencia. También a su marido le trataba como a persona subida de
color. Creyéndose monja y de inmaculada blancura, decía: «Quiero
escaparme, quiero salir de esta triste cárcel; pero no me salvarán
hombres tiznados\ldots{} no me salvarás tú, que traes el rostro obscuro
de andar con los negros de Indias.»

Espantosa fue la noche, y más aún la madrugada. Muertos de inanición,
Ansúrez y su hija pidieron alimento a sus aposentadores, que les
franquearon cuanto tenían. Una mujerona huesuda y desapacible, no por
esto privada de sentimientos cristianos, se puso a las órdenes de los
huéspedes; les sirvió sopas y una fritanga, y brindose a velar a la
enferma para que el señor y la niña pudieran descansar algunos
ratos\ldots{} ¡Buen descanso nos dé Dios! Cayó doña Esperanza en un
sopor del que no podían sacarla con sacudidas de los brazos, ni con
voces pronunciadas en los propios oídos de ella. Sudor copioso y frío
brotaba de su frente, y de su boca se escapaba un áspero soplido
cadencioso, que no traía ningún acento de locución humana.

Pensó Ansúrez que aquel singular estado podía ser un recalmón intenso de
los alborotados nervios de su esposa; pero la mujerona de la casa, que
era un tanto curandera y había presenciado bastantes casos como el que a
la vista tenía, dio dictamen muy distinto, y sin nombrar la muerte,
expresó el parecer de que no debían buscar remedios corporales, sino
aplicarse todos, deprisa y corriendo, a encomendar el alma de la señora.
Firme en esta intención edificante, bajó y trajo un cazolillo con
aceite, en el cual sobrenadaban encendidas varias mechas de algodón, que
eran como un holocausto a las benditas ánimas del Purgatorio, y el mejor
socorro que se podía dar a una persona moribunda. Nada dijo Ansúrez, y
comprendiendo que acertaba la mujer en su fúnebre pronóstico, echó todo
su dolor del lado de la resignación, encastillándose en esta con todo el
rendimiento de su alma cristiana. Menos fuerte Mara en su espíritu,
rompió en llanto; y entre lágrimas de la niña, oraciones de la huesuda,
silencio torvo de Ansúrez, y un desaforado ladrar de perros que del
campo venía, los alientos broncos que salían del pecho de doña Esperanza
fueron menguando, hasta que con uno más suave y hondo terminó su
existencia mortal.

La claridad del alba entró a deslucir el amarillo resplandor de las
luces mortuorias. Hija y padre se vieron en plena esfera de la realidad,
y de su propio dolor sacaron fuerzas para ocuparse en dar a la querida
muerta la compostura y grave continente que debía llevar al sepulcro.
Arregláronle el pelo, que se le había desordenado con las manotadas de
su locura. Sin quitarle la ropa interior, pusiéronle su mejor basquiña
negra, y un manto, negro también, que con monjil recato le cubría la
cabeza y busto. Formaba como un rostril ovalado, sujeto con alfileres,
que sólo dejaba al descubierto la cara. En las manos le pusieron el
Crucifijo que consigo solía llevar; hecho esto, se sentaron junto a la
cama por uno y otro lado, esperando la ocasión del sepelio. El cansancio
venció la voluntad de Ansúrez. La cabeza le pesaba más que su propósito
de tenerla derecha, y se dejó caer entre los brazos y sobre el lecho.
Quedose el hombre profundamente dormido, y en sueños le turbaba un ruido
intenso y mugiente: creyó que era el oleaje del Mediterráneo rompiendo
en las peñas de Cabo Palos o en los cantiles de Porman. Soñó que estaba
en aquella costa oyendo la voz iracunda del mar\ldots{} Su hija le
despertó sacudiéndole el brazo, y le dijo: «Padre, ¿oyes ese ruido?»

---Sí, oigo---respondió Ansúrez estre dormido y despierto.---Tenemos
levante duro.

---No es eso, padre. Es ruido de soldados. Los soldados están aquí. No
caben en el corral. Del corral han subido al corredor: algunos han
abierto esta puerta, y al vernos han vuelto a cerrar.

Cerciorose Ansúrez por sus propios ojos de lo que Mara le decía; vio la
inquieta turbamulta militar, que sin duda iba de camino hacia la ciudad
insurrecta, y se le daba parada y rancho en la Casa de Postas. Como
acontece en estas invasiones, no faltan muchachos alegres que se lanzan
a una requisa indiscreta, en busca de las vituallas que comúnmente se
guardan en altos desvanes. Perseguían jamones o cecina, y hallaron cosa
muy distinta de lo que anhelaban. Algunos eran tan desahogados, que el
hábito de la galantería se sobrepuso a los respetos debidos a la muerte;
y ante Mara llorosa junto al cuerpo frío de su madre, repararon en la
belleza picante de la \emph{chavala}, y más prontos estuvieron para
requebrarla que para compadecerla. Viendo que unos tras otros
entreabrían la puerta sin más objeto que curiosear, Ansúrez abrió de
golpe y les dijo: «Pasen, si gustan de ver cosas tristes. Esta señora
difunta es mi esposa, y esta muchacha, mi hija. Si buscan comida, sepan
que aquí no la hay, ni creo que puedan encontrarla en parte alguna de
este caseretón desamparado. Aquí no hay más que soledad y lágrimas.
Íbamos hacia Granada\ldots{} Mi esposa enferma no pudo resistir el
quebranto del viaje ni la falta de todo socorro de víveres y medicinas,
y esta madrugada su alma se ha ido a la presencia de Dios. Mi hija y yo
no saldremos de aquí sino para llevar a nuestra querida muerta a donde
podamos darle sepultura cristiana. Si son ustedes piadosos, como parece,
ayúdennos a cumplir esta santa faena, y les quedaremos muy
agradecidos\ldots{} Guardaremos en el corazón el recuerdo de estos
buenos chicos, aunque no volvamos a vernos. Ustedes van a Loja;
nosotros, al puerto más cercano, que entiendo es Motril, pues yo no soy
hombre de guerra, sino de mar.»

Los soldados oyeron respetuosos estas razones tan sinceras como
expresivas, y el más despabilado de ellos, en nombre de todos, dijo que
de buen grado complacerían al señor viudo y a la niña huérfana,
ayudándoles a la conducción y entierro de la señora finada; pero que
habían de partir en cuanto se racionara la tropa, que ello sería obra de
veinte minutos todo lo más. Detrás llegaría un batallón de Cazadores, y
estos no habían de ser menos generosos y cristianos que los presentes.
Con esto, y con dar a los atribulados hija y padre dos panes de munición
de a dos libras, se despidieron.

Al son de tambor y cornetas se alejó la tropa, y Ansúrez, otra vez solo,
trató con la mujerona y los vejetes de dar tierra a la pobre doña
Esperanza. Convinieron todos, mediante \emph{conquibus}, en facilitar la
indispensable función mortuoria. El cementerio más próximo era el de
Cijuela, distante una legua o poco más. No faltarían cuatro hombres que,
turnando, transportasen el cadáver, y delante iría un propio que
previniese al cura para que no faltara un buen responso. Por fin, como
en el curso del día habían de volver de Granada mozos, caballos y algún
carricoche (que ya con la presencia de la tropa se iba restableciendo la
vida normal), después del sepelio podrían tener el viudo y su hija un
galerín en que molerse los huesos por el \emph{camino de arrecife}, que
así llamaban a las carreteras.

Pasaron al mediodía los Cazadores sin detenerse, y a la tarde se puso en
camino con solemne tristeza y soledad la pobre comparsa que acompañaba
los restos de doña Esperanza, encerrados en una caja tosca que a toda
prisa carpintearon los viejos de la Casa de Postas, y que conducían en
parihuela otros viejos y mendigos alquilones. Seguían don Diego y su
hija en el coche llamado de San Francisco, y tras ellos lucido cortejo
de chicos y gitanas que iban al reclamo de una limosna. Con lento andar
llegó la procesión a su término, que era un camposanto humilde, sin
mausoleos pomposos, poblado de cruces, las unas derechas, otras caídas o
inclinadas con dejadez, como si quisieran descender al reposo que
gozaban los muertos. Un cura del mal pelaje, esmirriado y anémico, que
apenas podía con la capa pluvial, y un monaguillo pitañoso y descalzo,
aguardaban con puntualidad mendicante.

Breve y patética fue la ceremonia. Cuando la pobre doña Esperanza bajó a
la tierra, prorrumpieron las gitanas en teatral llanto, que fue como un
fondo coral en que vivamente se destacaba el verídico duelo de la
huérfana y el viudo. Todo terminó al caer de la tarde, cuando sobre el
rústico cementerio revoloteaban las golondrinas, que en próximos techos
tenían sus nidos. Pagó don Diego los servicios funerarios con largueza
de indiano. Moneda de oro puso en la mano negra y flaca del cura, que,
al recibirla y verla tan brillante, apretó el puño cual si temiese que
se la quitaran. Quedó el hombre muy agradecido, y ofreciendo rogar por
muertos y vivos, se fue a toda prisa, que cenar solía tempranito. A los
portadores recompensó Ansúrez con buenas monedas de plata, que por más
señas eran pesetas columnarias, y entre las gitanas y chiquillos
repartió alguna plata y cobre en abundancia, con lo que todos quedaron
muy satisfechos, y al donante como a la niña desearon largos años de
vida y aumento de sus caudales. Al regreso, las gitanas, ya con más
ganas de canto que de llorera, propusieron a Mara decirle la
buenaventura; pero la niña no quiso escucharlas, sintiéndose en tal
ocasión lejos de todo consuelo.

A campo traviesa anduvieron, guiados por los viejos, dos o tres horas,
pasando por tierras del Soto de Roma, propiedad del inglés Duque de
Wellington, y a las diez de la noche fueron a parar a un ventorro, donde
les esperaba el birlocho dispuesto para proseguir su caminata. Todo lo
que tenía de excelente la moneda de Ansúrez, teníalo de perverso y
desvencijado el armatoste que le alquilaron aquellos chalanes. Tiraban
de él dos caballejos cansinos y llenos de mataduras, y lo guiaba un
perillán tuerto y cojo, que, apenas tratado, daba el quién vive con su
aliento de borrachín y sus trapacerías rateriles. Pero no habiendo cosa
mejor, los viajeros pasaron por todo, que para eso traían grande acopio
de resignación. Dando tumbos, oyendo sin cesar las groserías del cochero
y los palos con que a los pobres animales arreaba, llegaron después de
media noche a un parador de la ciudad de Santa Fe, donde hicieron alto
para descansar algunas horas. Pero la fatiga y el sueño atrasado que
ambos traían les retuvieron en los duros colchones hasta más de las
doce; y como el calor era sofocante, se acordó retrasar la salida hasta
el anochecer, lo que agradecieron los caballos tanto como el gandul que
los regía.

Anhelaba Diego recorrer con la mayor presteza posible la distancia que
le separaba de Motril. Forzoso era pasar por Granada, donde despediría
el carricoche de Lachar para tomar mejor vehículo. En Granada se
detendría lo menos posible: le asustaba la idea de encontrar parientes o
amigos, que con halagos y cumplimientos dilatorios le indujeran a mayor
tardanza. Tal como lo pensó, lo hizo: llegaron los viajeros a la ciudad
morisca al filo de media noche, y en una posada del arrabal del Triunfo
se alojaron, y de allí no salieron hasta saldar cuentas con el
ladronzuelo que les trajo, y ajustar un galerín que debía llevarles
hasta donde alcanzaba el camino de arrecife. Desde Béznar seguirían a
caballo hasta el término de su odisea terrestre. En estos tratos
chalanescos se les fue un día entero y parte de otro. A ningún conocido
vieron, ni hablaron más que con arrieros y trajinantes que en el mesón
se alojaban\ldots{} Partieron en alas, no diremos del viento, sino de la
impaciencia y prisa que empujaban el alma de Ansúrez hacia el mar, y en
los últimos ratos del parador, así como en el trayecto hasta Padul,
tuvieron noticia del desastroso acabamiento de la revolución de Loja.

\hypertarget{v}{%
\chapter{V}\label{v}}

Razón tuvo el Cura don Prisco al poner en sus letanías la piadosa
invocación al brazo militar: «¡Soldados, soldados!» Oída fue por Dios y
por el Gobierno esta devotísima plegaria. Soldados acudieron de Granada,
de Málaga y de Jaén, y reunidos frente a Loja, bajo el mando de un
valeroso General, saludaron a los insurrectos con la estimación de
rendirse y poner fin al democrático juego. Pronto comprendieron los
secuaces de Rafael Pérez que habían perdido su causa, metiéndose en una
plaza que más tarde o más temprano había de ser victoriosamente debelada
por la tropa. La hueste revolucionaria no debió abandonar nunca la
táctica de guerrillas: su fuerza estaba en la movilidad, en la rapidez
de las sorpresas y embestidas parciales. Estacionarse en un punto, aun
contando con defensas rocosas o con trincheras abiertas sin conocimiento
del arte de la castrametación, era ir a muerte segura. Un ejército
disciplinado y regularmente dirigido debía dar cuenta, como aquel la
dio, del tan entusiasta como aturdido ejército popular. Apretado el
cerco con la idea de que no escapase ninguno de los cinco mil
republicanos que en la plaza bullían, resultó que después de andar en
tratos y parlamentos, se escabulleron todos por las mallas de la red.

Se dijo que Serrano había llegado a última hora con instrucciones de
lenidad, que practicó a estilo masónico, haciéndose el cieguecito y el
sordo ante los grupos que huían de la plaza. Serrano era liberal, no
debe esto olvidarse, y en Madrid mandaban un astuto y un escéptico que
se llamaban O'Donnell y Posada Herrera. Si hubiera estado el mango de la
sartén en manos de Narváez, de fijo no queda un republicano comunista
para contarlo. Don Prisco Armijana, espíritu que se balanceaba en los
medios pidiendo mucha libertad y mucha religión, diría frente al
Socialismo vencido: «Soldados, no matéis. Dios quiere que todos
vivan\ldots{} y que todos coman. Soldados y paisanos, comed juntos.»

Venturosa fue la evaporación rápida de los insurrectos, tomando por este
o el otro resquicio los caminos del aire, porque así se evitaron las
duras represalias y castigos. Algunos cayeron, no obstante, para que
quedasen en buen lugar los fueros del orden santísimo. La vista gorda
del General no fue tanta que dejase pasar a todos sin coger los racimos
de prisioneros que debían justificar, llenando las cárceles, la
autoridad del Gobierno. No faltaron infelices que con el holocausto de
sus vidas proporcionaron a la misma autoridad el decoro y gravedad de
que en todo caso debe revestirse. De Rafael Pérez, nada se supo. Luego
se dijo que había ido a parar a Portugal. Hombre extraordinario fue
realmente, dotado de facultades preciosas para organizar a la plebe, y
llevarla por derecho a ocupar un puesto en la ciudadanía gobernante.
Tosco y sin lo que llamamos ilustración, demostró natural agudeza y un
sutil conocimiento del arte de las revoluciones; arte negativo si se
quiere, pero que en realidad no va nunca solo, pues tiene por la otra
cara las cualidades del hombre de gobierno. Representó una idea que en
su tiempo se tuvo por delirio. Otros tiempos traerían la razón de
aquella sinrazón.

Más que en estas cosas de la vida general pensaba Diego Ansúrez en las
propias, corriendo en la galera por el camino que faldea las moles de
Sierra Nevada en dirección a la fragosa Alpujarra. Pasó la divisoria que
llaman \emph{Suspiro del Moro}, sin duda porque allí suspiró y lloró el
desconsolado Boabdil, y también el viudo de doña Esperanza lanzó de su
pecho suspiros hondos recordando su amor perdido, y pesando las
desventuras que su viudez le traía. Luego consideraba el
enflaquecimiento de su bolsa, a la que, con las enfermedades de la
mujer, los viajes, los obsequios y otras socaliñas, había tenido que dar
innumerables tientos. En Granada y Loja habíanle tomado por indiano
rico, y no faltaron parientes pobres, Castriles o Armijanas, a quienes
hubo de consolar gallardamente con algún socorro. Ello es que por el
chorreo continuo de gastos en tan largo periodo de inacción, al mar, su
verdadera patria, volvía con sólo el dinero preciso para llegar a
Cartagena.

Pasando por la memoria, como se pasan las cuentas de un rosario, sus
desdichas en tierra granadina, pensaba el buen hombre que la causa de
ellas no podía ser otra que el haber infringido y olvidado las leyes
morales y religiosas. Su casamiento libre y sacrílego con Esperanza, sin
duda tenía muy incomodado al Padre Eterno, de donde resultaba que fueran
siempre desfavorables los que llamamos designios de la Providencia. Pero
luego, razonando con buen sentido, añadía: «Yo no fui a sacar a
Esperanza del convento de Consolación, sino que ella, descolgándose para
coger la calle y la libertad, cayó sobre mí como si cayera del cielo.
¿Qué había yo de hacer con ella? ¿Restituirla al convento, a donde no
quería volver ni a tiros? \emph{¡Ajos y cebolletas}, esto no podía ser!
Después, mares adentro, el amor, fuero imperante sobre toda ley, nos
casó. ¿Cómo lo habíamos de arreglar, si por el aquel de los malditos
cánones no podíamos casarnos por la Iglesia? Yo no diré nunca, líbreme
Dios, como decían los de Loja: \emph{¡muera el Papa!}; pero sí diré a
gritos: `¡mueran los cánones!'. ¿Y qué culpa tengo yo de que don Prisco
no pudiera sacar la dispensa de votos, ni arreglar todas las demás
zarandajas para echarnos las bendiciones?\ldots{} Culpa mía no es esto,
y porque la culpa es del Papa y no mía, siento mi conciencia muy
aliviada, pues hay cosas en que el deseo debe valer tanto como la
ejecución.» A pesar de la relativa serenidad que le daban estos
razonamientos, Ansúrez no se veía libre de inquietud: el temor religioso
iba ganando su alma, y recordando la escena tristísima del cementerio de
Cijuela, se proponía practicar el culto, cuidar de sus relaciones con
Dios hasta desenojarle.

Siguieron su camino hacia la Alpujarra, bordeando abismos y salvando
cuestas. En Padul descansaron, en Dúrcal comieron, y en Béznar1 se les
acabó la carretera, dejándoles a pie si no franqueaban a caballo las
seis leguas que les separaban de Motril. Las maletas quedaron en Béznar
para ser transportadas en mulo durante la noche. Dos borricos llevaron a
los viajeros a Tablate, y uno solo de Tablate a Vélez. No se crea que en
un asno montaban los dos: Mara iba sentadita en el albardón de un alto
pollino, y Ansúrez lo llevaba del diestro: era torpe jinete, y más a
gusto andaba con sus pies que con los de la mejor cabalgadura.

Pasada la divisoria de Lújar, se ofreció a los ojos de ambos el sublime
espectáculo del mar, grande espacio de azul, tan vago y misterioso en su
inmensa lejanía, que no parecía mar, sino una prolongación del Cielo que
se arqueaba hasta besar la costa. Tal fue la emoción de Ansúrez ante el
grandioso elemento en quien veía su patria espiritual, que le faltó poco
para ponerse de hinojos y entonar una devota oración sacada de su cabeza
en aquel sublime momento. Palabras de asombro, cariño y gratitud
pronunció santiguándose, y no tuvo reparo en mostrar una infantil y
ruidosa alegría, primer respiro del alma del marino después de su viudez
reciente.

El camino que faltaba, no muy extenso y todo cuesta abajo, bien podían
recorrerlo a pie. Así lo propuso el padre a la hija, y ambos se lanzaron
intrépidos y gozosos a la pendiente por ásperos caminos bordeados de
piteras, chumbos y otros ejemplares lozanos de la flora meridional. Sin
novedad anduvieron largo trecho; pero el cansancio agotó las fuerzas de
Mara, y cuando aún faltaban como tres cuartos de legua para llegar a
Motril, la pobre niña, dolorida de los pies y cortado el aliento, dijo a
su padre que le concediera un largo reposo, o buscase algún jumento en
las casuchas que a un lado y otro se veían. «Hija del alma---replicó
Ansúrez, a quien se hacían siglos los minutos que tardase en llegar al
puerto,---no perdamos tiempo en buscar caballería, que aquí tienes a tu
padre que te llevará con tanto cuidado y mimo como si te cargaran los
ángeles.» Dicho esto, la cogió en sus brazos y siguió adelante con ella
sin gran trabajo, pues la chica era de poco peso y él un gigante
forzudo.

Iban por un sendero pedregoso, flanqueado de pitas, cuando les alcanzó y
se les puso al habla otro viajero andante que tras ellos venía. Era un
muchachón de buena presencia y estatura, muy desastrado de ropa, como si
llevara largo tiempo de corretear por caminos ásperos y pueblos míseros.
Visto de lejos, parecía negro: tan extremadamente había tostado el sol y
curtido el aire su tez morena. El polvo, además, lo jaspeaba con
feísimos toques; pero ni la suciedad ni la negrura desfiguraban las
varoniles facciones del sujeto. Las primeras palabras que dirigió a los
Ansúrez fueron contestadas con desabrimiento. ¿Era mendigo, ladrón o
vagabundo? Hija y padre se detuvieron en estas dudas antes de
responderle con urbanidad. «Bueno---dijo Ansúrez, vencido al fin de la
cortesía del extraño individuo negruzco más bien que negro:---no nos
enfadamos porque tú nos hables, ni tenemos a desdoro el hablar con un
pobre. Nosotros vamos en demanda de Motril. Tú, a lo que parece, llevas
el mismo camino.»

---A Motril voy---respondió el hombre ennegrecido y empolvado;---y antes
de que el señor me lo pregunte, le diré que me trae a este puerto el
mucho cansancio y ninguna utilidad que he sacado de trabajar tierra
adentro, en el campo, en el monte, en las canteras de mármol; y ahora
buscaré trabajo en la vida de mar, porque el mar es mi elemento, quiero
decir, que me gusta sobre todas las cosas, y que en él está el hombre
mejor que en tierra. Esto digo, esto sostengo, aunque usted lo lleve a
mal.

---¿Qué he de llevarlo a mal, ajo?---exclamó Ansúrez parándose ante el
hombre de color obscuro y mirándole cara a cara.---¡Si yo, aquí donde me
ves, soy del mismo parecer que tú, y después de los peces no hay nadie
en el mundo que sea más hijo del mar que yo! De tierra adentro vengo sin
timón ni compás, no sé si huyendo de mis desdichas o trayéndolas
conmigo. Al interior me fui con mi esposa y mi hija. Sólo con la hija
vuelvo. El corazón se me ha partido, y la mitad he dejado allá en un
cementerio chico\ldots{}

Ya con esta entrada vieron ambos abierto el camino para una conversación
franca. El negro era listo: su lenguaje contrastaba rudamente con su
bárbara facha y su vestir lastimoso. Por el acento reveló a las primeras
frases su abolengo americano, y a la pregunta que sobre el particular le
hizo Diego, contestó así: «Yo soy del Perú; me llamo Belisario, y en
España estoy por locuras y calaveradas mías, que ahora pago con usura,
pues han caído sobre mi cabeza más desdichas de las que merezco\ldots{}
Ya ve por mi facha lo rebajado que estoy de mi nacimiento y
categoría\ldots{} No le pido limosna, aunque bien la necesito, sino
protección para poder embarcarme y salir a buscar el sustento, aunque
sea con fatigas, que las pasadas en el mar han de consolarme de las que
llevo sufridas en tierra.»

Con esta ingenua manifestación, el americano empezó a ganarse la
simpatía de Ansúrez. En lo restante del camino, hija y padre le pidieron
más noticias de su vida, y él no se cortó para darlas. Había nacido al
pie de los Andes; sus primeros pasos los dio sobre pavimento de barras
de plata. Su padre era español, que cruzó los mares y se fue en busca de
\emph{la madre gallega}, que así llaman allí a la fortuna. Casó con una
limeña muy guapa\ldots{} Las limeñas son las mujeres más bonitas del
mundo, y mejorando lo presente, a todas ganan en desenvoltura y malicia
graciosa. La digresión que hizo el narrador hablando de las limeñas, no
se copia en este relato por no agrandarlo más de lo debido. Habló luego
del mal genio de su padre, que era \emph{más adusto que un pleito}, y
conservaba en su carácter el dejo de las fierezas inquisitoriales, que
en toda alma española están adheridas, como se adhieren a la lengua los
sonidos del idioma.

De la dureza del padre y de la propensión del hijo a la independencia,
resultaron castigos, rebeldías y sucesos lamentables. No tenía veinte
años cuando se emancipó de la autoridad paterna, retirándose al Callao,
donde con otros chicos de su edad, como él indisciplinados y ociosos,
cultivó su afición al mar. Todo el día se lo pasaba en botes o chalanas,
jugando a la navegación de vela y remo. El cariño de la madre le atrajo
de nuevo a la casa de Lima. Pero la inflexibilidad del padre no tardó en
reproducir las discordias. Escapó al fin, buscando la deseada libertad,
y se fue a las islas Chinchas, donde halló medio de ser admitido en la
tripulación de una fragata inglesa que le trajo a Europa. Contar todo lo
que en el viaje le pasó, desde su salida de las Chinchas hasta su arribo
a Valencia, sería historia larguísima y fastidiosa para el señor y
señorita que le escuchaban\ldots{} Terminó diciendo que el recuerdo de
su madre y hermanos no se apartaba de él, y que ignoraba en absoluto lo
que había ocurrido en su familia desde que su delirio de aventuras le
separó de ella.

No sabía Diego si creer todo o una parte no más de lo que el americano
refería. Pero a su desconfianza se impuso su buen corazón, y dijo al
vagabundo que él no era más que un pobre naviero de faluchos de costa, y
en tan pobres barcos no podía ofrecerle empleo ventajoso. Pues buscaba
trabajo de mar, le llevaría gustoso a Cartagena, donde hallaría medios
de enrolarse en buenos buques mercantes, o en los de guerra si le
llamaba y era de su gusto la marina militar. A esto dijo Belisario que
el ser llevado a Cartagena lo consideraba como la mayor caridad que
podía recibir, y con grandes aspavientos y cierto lirismo en su dicción
fácil, expresó su gratitud al generoso señor y a su bella hija.

\hypertarget{vi}{%
\chapter{VI}\label{vi}}

Horas no más estuvo Ansúrez en Motril, el tiempo preciso para fletar una
hermosa lancha y disponerla para su viaje. Belisario le trajo las
maletas desde la ciudad al varadero, media legua larga, y luego embarcó
con el padre y la hija, cinco marineros y el dueño de la lancha. Largó
esta la vela, y al amor de un poniente frescachón que felizmente
reinaba, se alejó rascando la costa. La nave era excelente, y a las dos
horas de su salida pasaba frente a la Sierra de Adra. Toda la noche
siguió navegando con gallardo andar; los tripulantes vieron de lejos la
luz de Almería, y al amanecer montaron el Cabo de Gata, siguiendo
después con menos marcha, al socaire de los altos montes y cantil, que
también tienen nombre de Gata. A proa iban Belisario y los marineros, y
Ansúrez a popa con su hija. Sobre las tablas de la sobrequilla habían
arreglado, con petates y mantas, el mejor acomodo posible para que la
señorita descansara, ya que dormir no pudiera.

La caída del viento fue causa de que emplearan casi todo el día en
recorrer la costa hasta Cala Redonda. De aquí, con una fácil guiñada,
demoraron frente al puerto de Águilas, y en él se metieron para pasar la
noche. Al amanecer continuaron: reinaba un lebeche suave que levantaba
marejadilla. Alguna molestia sufrió Mara con las cabezadas de la
embarcación; pero pasado Cabo Tiñoso se les presentó mar bella, y por
fin, bien entrada la noche, gozosos y satisfechos del tiempo y de la
nave, dieron fondo en la bahía de Cartagena. Saltó a tierra Ansúrez con
su hija, y sin tomar respiro subieron a su habitual residencia, que era
una vetusta casa no lejos de la Catedral Antigua, situada en punto
culminante, desde donde se gozaba la vista del puerto y de los dos
gigantes castillos que lo custodian: Galeras y San Julián.

Apenas instalado en su domicilio, se ocupó Diego en reanudar sus
negocios, enterándose de la situación de los faluchos. La ausencia del
amo había embarullado las cuentas, y para ponerlas en claro hacía falta
paciencia y actividad. Dejaremos ahora en estos afanes al pobre naviero,
para decir que la casa donde hija y padre vivían era la de un compadre y
amigo llamado Roque Pinel, socio de Ansúrez en otro tiempo, y a la sazón
ocupado en la compra y embarque de esparto. La cordialidad y buena
armonía entre ambos mareantes no se alteró nunca. Habían sido compañeros
en el servicio del Rey, y juntos corrieron, en la navegación y el
comercio, aventuras borrascosas, con varia fortuna. Cuando Ansúrez vivía
en Cartagena, llevaban a medias los gastos de la casa, y del gobierno de
esta cuidaban la esposa y hermana de Pinel, dos mujeres cincuentonas,
sentadas y de gran disposición para el caso. Bien podía confiarles
Ansúrez la custodia de Mara en sus ausencias. Contaba con la docilidad
de su hija, que aún ceñía falda de adolescente. Pero el padre recelaba
que, en llegando a mujer hecha, no había de ser tan fácil retenerla en
una disciplina rigurosa. Al propio tiempo, no estaba nada satisfecho de
la educación de Mara, limitada, por aquellos días, al leer correcto, a
un mediano escribir y deficientes nociones de Aritmética. Pensaba el
celtíbero en un buen colegio de doncellas, o en escuela regida por
monjas aseñoradas, que la instruyeran y la pulimentaran en todo lo
concerniente a dicción, etiqueta y modales.

Antes que me pregunten por Belisario, diré que Ansúrez le consiguió
trabajo en la descarga de carbón, con lo que se puso el hombre más negro
que lo estaba en el instante de su aparición en el camino. Después fue
recomendado a una empresa de hornos y fundición en las Herrerías, y allí
ganó dinero y se hizo querer de sus patronos. No se asombraron poco
Ansúrez y Mara cuando le vieron entrar en su casa lavado y bien vestido,
en tal guisa, que tardaron en conocerle, según venía de limpio y
elegante. Sus trazas de caballero iban bien con el habla fina que usaba,
y con los dejos líricos que del alma le salían a poco interés y calor
que tomara el diálogo. Lo más substancial que dijo en aquella visita fue
que había empezado estudios de pilotaje en la Escuela de Cartagena, y
que por necesidad continuaba en las Herrerías, sin otro objeto que ganar
algún dinero con que emprender vida más de su gusto; o en otros
términos, para mayor claridad, que él pedía el auxilio de Vulcano para
obtener los favores de Neptuno. Sonriendo miró Mara a su padre, como
interrogándole acerca de aquellos señores Neptuno y Vulcano, que ella
jamás había oído nombrar. Concluyó en aquella ocasión, como en otras, la
visita de Belisario con las donosas burlas que hacía la chica del sutil
lenguaje del americano, sin que por ello lograra enojarle, como sin duda
se proponía; antes bien, llevábale a mayor admiración de ella y a más
desenfrenado lirismo.

Bien entrado ya el 62, se supo que Belisario se había embarcado para
Marsella en un buque francés que dejó en Cartagena cargamento de guano.
Por Navidad del mismo año, le vio Ansúrez en Palma vendiendo azafrán y
comprando almendra. El 63, reapareció en Cartagena, vestido con
singularidad, el rostro demacrado y tristón, como si convaleciera de una
enfermedad penosa. Sus operaciones mercantiles no salían entonces del
terreno espiritual: comerciaba con las Musas, y sus remesas eran
poesías, que más de una vez aparecieron en los periódicos locales. Los
entendidos en estas cosas aseguraban que las odas, silvas, canciones y
elegías del americano no carecían de mérito, y algunos vates
cartageneros las ensalzaban hasta el cuerno de la luna. Sus defectos
eran sus cualidades prodigadas con hinchazón y superabundancia por una
fantasía sin freno. Abusaba indiscretamente de los ángeles, de la
espléndida flora tropical, y de las conversaciones tiradas que sostienen
los astros del Cielo con los átomos de la Tierra. Todo esto pasó
arrastrado por la corriente undosa de la literatura periodística, que
lleva y derrama las ideas en el mar del olvido. Del mismo modo pasó
Belisario, que desapareció de Cartagena sin despedirse de nadie, ni
decir a dónde iba con sus estrofas y su acentuada personalidad.

En los comienzos del 64, volvió el peruano a dar señales de vida, y ello
fue por una carta que de él recibió Ansúrez en Alicante. Decíale que
acababa de salir del Hospital, no bien repuesto aún de una fiebre
maligna. Movido de su buen corazón, hizo Diego por él lo que podía, y
partió a Valencia, donde estaba la gentil Mara perfilando su educación
bajo la férula de las Madres Ursulinas de aquella ciudad. Los quince
años de Mara eran espléndidos: pasaba de la adolescencia a la juventud
con arrogancia de conquistadora. Sus hechizos inspiraban miedo a las
Madres, miedo también al padre, y sin dejarse ver fuera del convento,
eran conocidos y celebrados por obra exclusiva de la fama. Ni el fuego
ni la hermosura pueden estar ocultos.

En Septiembre del mismo año, dio Ansúrez por finiquitado el pulimento de
la señorita, y se la llevó a Cartagena. Creía el buen hombre que las
Ursulinas habían puesto a su hija como nueva, y que esta era un prodigio
de ilustración y un lindo archivo de conocimientos. Grandemente se
equivocaba, porque Mara, descontado el barniz leve de cultura que le
dieran las monjas (nociones farragosas del arte gramatical y de la
ciencia de la cantidad, un poquito de francés mascullado y un
imperfectísimo tecleo de piano), salía del convento tan rasa y monda de
saber como había entrado, con bastantes malicias y astucias de más, y su
cándida ingenuidad de menos. Algo de esta recobró al volver a su casa,
porque no disimulaba el desafecto que en su corazón dejaron las Madres.

Ansúrez no se cansaba de admirar el ligero barniz, que pronto habría de
deslucirse y perderse, y encantado con su hija, no veía en la sociedad
de sus iguales hombre digno de ella. Y está de más decir que Mara tuvo
en Cartagena, al presentarse acicalada y bruñida de lenguaje, un éxito
loco. Muchachos de diferentes vitolas y abolengo la cortejaron, sin que
ella saliera de su mónita constante: enloquecer a todos, y no dar
esperanzas a ninguno. Cobró fama de ambiciosa y de picar demasiado alto.
Con las gracias discretas nuevamente adquiridas se juntaban, en
delicioso revoltijo, los donaires que se le pegaron en la tierra
andaluza\ldots{} No había criatura que exhibir pudiera mayor conjunto de
seducciones mortíferas, ni que impusiese más terror a los que la
sitiaban con solicitudes amorosas. Su talle sutil, su gracioso andar,
sus decires prontos, que tenían por manantial la boca más fresca y
bonita que podría imaginarse, su rostro trigueño a lo Virgen de Murillo,
se grababan en la retina y en el corazón de infinidad de jóvenes que
vivían desconsolados y como almas en pena.

Por aquellos días, que en buena cuenta eran los de Octubre del 64,
resurgió Belisario en Cartagena bien vestido y con cierto mohín
misterioso, dejando entrever que un magno asunto secreto y de universal
importancia movía su voluntad. Algunos le creyeron conspirador, y en
verdad lo parecía por la sutileza con que esquivaba su persona. Pronto
le llevaron a Diego Ansúrez el soplo de que el peruano había venido en
requerimiento de Mara, y que de noche rondaba la casa disfrazado de
marinero. Acechó Ansúrez; tomó lenguas de los vecinos y de las mujeres
de la casa, y si no pudo echarle la vista encima al caballero rondador,
supo de un modo indudable que había cambio de cartitas, y que a las
manos de Mara, por impenetrable conducto, llegaban voluminosos paquetes
de prosa y verso.

Saber esto y volarse el honrado marino, fue todo uno, y en su furor
corrió derecho al descubrimiento de la verdad, encerrándose con su hija,
e interrogándola de forma ruda y pavorosa, que no era para menos la
rabia que el celtíbero sentía. Atemorizada, negó al principio Mara; pero
la verdad que le llenaba el alma pudo en ella más que el disimulo, y al
fin, con la fuerza de dicción que da un sentimiento poderoso, declaró de
lleno que el peruano la quería, y que ella\ldots{} le había hecho dueño
de su corazón, con inquebrantable propósito de ser de él o de nadie.
Larga y penosa fue la escena, y en ella hubo de todo: gritos, amenazas,
lamentos, truenos furibundos en la boca del padre, y un río de lágrimas
en los ojos de la señorita. Repetido por la noche el sofión, presentes
Pinel y las dos señoras, hablaron todos con tal vehemencia, afeando el
amor de Mara, que la pobre muchacha quedó sobrecogida y muda. Creyeron
que la habían convencido; pero no fue así: más fácilmente se apaga un
volcán que el incendio de un corazón enamorado.

Dos días después, hallándose Ansúrez en la correduría que despachaba sus
buques, se le presentó de improviso Belisario, y sin preámbulos ni
retóricas baldías, en prosa categórica y llana, le dijo: «Vengo, amigo
Diego, a pedirle a usted la mano de su hija.» ¡María Santísima, qué cara
puso el celtíbero al oír lo que juzgaba disparate, blasfemia o cosa tal,
qué relámpago de ira echó de sus ojos, qué sarta de vocablos feos y
sacrílegos de su boca! Repitió el peruano fríamente su demanda; mas
antes de que concluyera, corrió hacia él como un león el enconado padre,
y acudieron los allí presentes a sujetar a uno y otro, salvando de un
grave estropicio al poeta mareante. Dueño este de sí mismo, y
conservando la serenidad que había perdido su enemigo, declaró que Mara
sería suya, quisiéralo o no el señor Ansúrez, porque la ley de amor, más
alta y fuerte que todos los respetos humanos, había de cumplirse. Amor
es ley del universo, y la autoridad paterna es ley social. Amor es
fuerza creadora que engendra la vida y perpetúa la Humanidad; las leyes
sociales que contrarían el amor son esencialmente destructoras como
instrumentos de muerte. Estos y otros desatinos y razones enfáticas dijo
en un tono y cadencia que sonaron a verso en los oídos de los hombres de
mar. Terminó la reyerta con groseras burlas de las retahílas del
americano, y a empujones le lanzaron a la calle ignominiosamente. «Soy
solo contra todos---clamaba,---y no es bien que me traten así\ldots»

Ansúrez, sin que sus amigos le soltaran de la mano, quedó en la
correduría braceando como loco furioso, y repitiendo las maldiciones y
amenazas con que desfogaba su ira. «¡Ajo, dar mi hija a un
coplero!\ldots{} ¡Ajo, maldito sea el instante en que los ojos de ese
bigardo miraron a mi niña!\ldots{} ¡Si no me lo quitan, lo
estrangulo!\ldots{} ¡Suéltenme, que quiero tirarlo al agua con una
piedra trincada al pescuezo!\ldots» No se calmó hasta que regresaron los
que se habían llevado a Belisario, y le dijeron: «No te sofoques, Diego,
ni hagas caso de ese silbante. Hémosle metido en el bote del vapor
sardo, donde está de mayordomo. Descuida, que a tierra no ha de volver.
Ya tienes al vapor desatracado y listo para salir a la mar.» A pesar de
esta seguridad, no tuvo sosiego Ansúrez hasta que vio salir el vapor
sardo\ldots{} Aún rondaba su alma un recelo inquietante. Aguardó la
vuelta del práctico que había sacado al vapor, y las referencias de este
diéronle la certidumbre de que el aventurero gandul navegaba con rumbo a
Génova.

En los días siguientes observó Ansúrez en su hija tan serena placidez,
que la irritación y suspicacia motivadas por el suceso de la correduría
se desvanecieron completamente. Después, tuvo que ir a Mazarrón a tratar
de un transporte de plomos, y regresó a los dos días en un vaporcito
costero. Al saltar a tierra, le recibió su amigo Roque Pinel con la cara
larga y afligida que suelen poner los que se ven obligados a dar una
mala noticia\ldots{} No sabía el buen hombre cómo empezar. Sus palabras
balbucientes, el tono lacrimoso y fúnebre con que las pronunciaba,
levantaron en el alma de Ansúrez una onda de terror, que le cortó el
aliento. Desgracia inmensa y repentina había ocurrido en su casa.
¿Estaba Mara enferma?\ldots{} ¿Se había muerto quizás? Echole Pinel el
brazo al cuello, y anduvieron juntos algunos pasos\ldots{} Sacando
fuerzas de flaqueza, pudo decirle, no que Mara se había muerto, ni aun
que estaba enferma, sino que buena y sana se había escapado de la casa.
¡Jesús!\ldots{} fugada, sí, de la casa y de la ciudad\ldots{} ¡Jesús,
Jesús!\ldots{} arrebatada por el gavilán americano.

\hypertarget{vii}{%
\chapter{VII}\label{vii}}

La terrible impresión de esta noticia no hizo estallar al buen Ansúrez
en bravatas y denuestos sacrílegos. La recibió como una maldición de
Dios, y su dolor tomó forma semejante a las sublimes quejas del santo
patriarca Job. Creyó que Dios lanzaba sobre su cabeza rayos de ira, que
debía revolcarse en un muladar, y convertirse en ceniza o polvo
miserable. Rompió a llorar como un niño. Ni Pinel ni otros amigos
pudieron consolarle.

¿Pero cómo\ldots? ¿Cuándo\ldots? A estas interrogaciones ansiosas fueron
contestando los amigos con discreta lentitud. Lleváronle a la
correduría, y con él se encerraron. Así evitaban el tener que contarle
cosas tan delicadas en medio de la calle\ldots{} ¿Pero cómo\ldots?
¿Cuándo\ldots? Pues la escapatoria fue la misma noche de la partida de
Ansúrez a Mazarrón. Ninguno de los amigos podía explicarse que habiendo
embarcado el ladrón en el vapor sardo, volviese a Cartagena tan pronto.
O no eran ciertas las noticias dadas por el práctico, o el americano
tomó tierra en alguna playa o puertecillo de la costa\ldots{} Lo
indudable, y esto se supo por una muchacha que en la casa servía cuando
Mara volvió del convento, era que los amores de Belisario con la
señorita databan de fecha relativamente larga. Cuando Ansúrez le
socorrió en Alicante, ya había logrado el americano que sus amorosas
esquelas llegaran a la colegiala de las Ursulinas\ldots{} Restituida la
niña a su casa, continuó la correspondencia, que era por una y otra
parte de lo más arrebatado y fogoso, a juzgar por una carta que, después
de la evasión, encontraron en el neceser de Mara; papel que esta se
olvidó de quemar, como había hecho con otros\ldots{} También era
indudable que en Octubre, antes de la violenta escena en la correduría,
estuvo el gavilán en Cartagena; los amantes se veían y charloteaban,
asomada ella a una ventana que da al callejón del Cristo, él en la
calle, arrimado a un doblez obscuro de la pared.

Para que nada quedara por decir, uno de los presentes declaró que, por
confidencia que a una de sus amiguitas hizo Mara, se sabía que el amor
de esta era de los de condición irresistible y volcánica. Otro de los
amigos expuso la idea de que el americano sería todo lo perdido y
vagabundo que se quisiera; pero que alguna cualidad eminente había de
tener para trastornar a una señorita que, con la pasada que le dieron en
el convento, era sin duda muy sentada de cascos. No faltó quien dijese
que la culpa de aquel desvarío la tenían los malditos versos, o la
poesía que, hablando en prosa neta, echaba por su boca el maligno
americano. En resolución, este había cautivado a la paloma Ansúrez con
el gancho de su palabrería poética, y el continuo hablar de ángeles,
corolas, crepúsculos, misterios de la tarde y de la noche, astros
rutilantes, desmayos del amor, y otras mil sandeces que debieran ser
prohibidas por la Iglesia, y perseguidas sin compasión por los jefes
políticos, corregidores y alcaldes pedáneos.

Faltaba lo más importante de la información que al afligido Ansúrez
dieron sus amigos. En cuanto se notó la falta de Mara en la casa, salió
Pinel disparado en busca de la fugitiva. Requiriendo el auxilio de las
autoridades, anduvo de mazo en calabazo toda la noche, sin encontrar ni
a las personas buscadas ni rastro de ellas. Creyó que habían huido por
tierra; pero al día siguiente, la vaga delación de un gabarrero le
indujo a creer que Mara y su raptor habían escapado por los anchos
caminos del mar. ¿Cómo y a dónde?\ldots{} Noticias posteriores dieron la
casi certidumbre de que navegaban con rumbo al Estrecho de Gibraltar en
una goleta de tres palos, norte-americana, llamada \emph{Lady Seymour}.
«¿Para dónde, ajo?\ldots» «Para Río Janeiro, Montevideo y el Pacífico.»
La goleta despachada en Barcelona con carga general, había hecho escala
breve en Cartagena para tomar dos docenas de pasajeros, que iban sin
blanca y con lo puesto, en busca de \emph{la madre gallega}.

Por fin, el buen Pinel, no sabiendo cómo consolar a su amigo, díjole que
unos señores, no sabía si peruanos o chilenos, establecidos en Alicante
y que de paso estaban en Cartagena, conocían a Belisario y dieron de su
familia las mejores referencias. El padre había muerto, dejando un
fabuloso caudal, haciendas muchas y plata en barras, que, puestas en
montón, subirían tanto como la torre de la Catedral de Murcia. De todo
eran ya dueños la viuda y los hijos\ldots{} Bien podía suceder que
Belisario, al alzarse con la moza, tuviera la intención de ir por
caminos malos a un fin excelente, que en esto de elegir caminos, el
hombre es siempre un navegante, y no va por donde quiere, sino por donde
le dejan las corrientes y el viento. Dentro de lo posible estaba que la
pareja loca fuese navegando en demanda del Perú y de la herencia; que en
el Perú se unieran Mara y Belisario en santo matrimonio, y que luego
volvieran acá encasquillados en plata, para dar dentera a media
España\ldots{} Ansúrez le mandó callar: se angustiaba más con el
desenlace de cuento infantil que los amigos querían poner a su infamia.

El suceso que referido queda hundió al celtíbero en negra tribulación.
Ya no había para él contento ni paz. En pocos días se avejentaron sus
cuarenta y dos años, tomando aspecto de hombre más que cincuentón.
Llenósele de arrugas el rostro, la cabeza de canas; la sonrisa y todo
concepto jovial huyeron de sus labios. Hablaba tan poco, que sus
palabras se podían contar como los donativos del avaro. Para que su
semejanza con el santo patriarca Job fuera más visible, a los ocho días
de la fuga de Mara trajéronle la nueva de otra gran desdicha. El falucho
\emph{Esperanza}, que había salido de Torrevieja con cargamento de sal
para Villanueva y Geltrú, fue sorprendido de un furioso ramalazo de
Levante, que lo desarboló, y con graves averías en el casco, lo dejó sin
gobierno, a merced del oleaje. De nada valieron los esfuerzos de una
tripulación heroica: el pobre barquito fue a estrellarse en las peñas
del faro de Santa Pola. Perecieron dos hombres, y la embarcación se
deshizo como un bizcocho\ldots{}

La noticia del tremendo desastre fue escuchada por Diego con resignación
tétrica y sombría, como si antes que la temiese la esperase, persuadido
de que las desgracias no vienen nunca solas. Considerando que el otro
falucho que poseía, nombrado \emph{Marina}, se encontraba en tan mal
estado que su reparación había de costar casi tanto como hacerlo de
nuevo, resolvió el humilde armador desprenderse de todas las granjerías
fundadas sobre el inseguro cimiento de las aguas. Aprestose, pues, a
liquidar los restos de su negocio naviero y mercantil, con propósito de
retirarse luego a vida solitaria, quizás eremítica, lejos del mundo y de
sus engañosas vanidades.

Con fría calma y estoicismo dedicose Ansúrez día tras día a soltar sus
amarras con la industria marítima, y el tiempo que le quedaba libre
pasábalo en el Arsenal, al calor de algunas fieles amistades que allí
tenía. Anselmo Pinel, hermano de Roque y maestro ajustador en los
talleres, fue el primero que consiguió distraerle de sus murrias,
interesándole en los trabajos de la ingeniería naval. A la sazón estaba
en grada un fragatón de hélice con blindaje, que llevaba el glorioso
nombre de \emph{Zaragoza}; y terminada ya, esperaba su armamento junto a
la machina otra gallarda nave, la \emph{Gerona}, de cincuenta cañones y
seiscientos caballos.

La inspección de obras, que suele ser el mejor esparcimiento de viejos
aburridos, dio al alma de Ansúrez algún consuelo: al menos, mientras
curioseaba de una parte a otra, descansaba su espíritu de la
contemplación interna de sus desdichas. Viendo iniciada en él la
tendencia reparadora, Anselmo Pinel, sin apartarle de la idea de
retirarse a vida solitaria, le indujo mansamente a volver al servicio de
la Marina de guerra, pues esta, en su sentir, armonizaba muy bien con el
santo propósito de abandonar los intereses mundanos. La vida del marino
real era toda abnegación y sacrificio, con la añadidura de la soledad,
más completa en la extensión del Océano que en los áridos desiertos de
tierra. En este sentido le habló, aunque con términos más llanos,
haciéndole ver que si le llamaban las austeridades del yermo y el gusto
del sacrificio, debía sin vacilación engancharse por tercera vez,
pidiendo plaza de contramaestre u oficial de mar.

Aunque verbalmente rechazaba Diego esta proposición, bien comprendió
Anselmo, por los términos vagos de la negativa, que la idea penetraba en
el ánimo del infeliz hombre, y allí labraba su nido. Insistía y
machacaba Pinel en su exhortación, reforzándola con discretas razones.
«Aquí tienes al Director de Ingenieros, don Hilario Nava, que se
alegrará de que vuelvas al servicio, y pronto ha de venir el General
Rubalcaba, que te estima, y no desea más que protegerte. No vaciles,
Diego, y date a la mar, que será tu consuelo, tu familia, ya que ninguna
tienes, y tu religión, que buena falta te hace.» Ayudaban al buen
consejero en esta obra catequista dos amigos y compañeros de Ansúrez: el
uno, Cabo de mar, llamado José Binondo; el otro Cabo de cañón, por
nombre Desiderio García. Ambos habían navegado con él largo tiempo en la
goleta \emph{Vencedora}.

Por fin, hallándose Diego en gran perplejidad, el ánimo indeciso,
balanceándose entre la pereza, que le pintaba las dulzuras de la
quietud, y el sentimiento religioso, que le pedía trabajos más duros en
provecho de su alma y de la madre patria, alma y dueña de todas las
vidas españolas, salió una mañana al muelle, y vio fondeada en el puerto
la mas gallarda, la más poderosa y bella nave de guerra que a su parecer
existía en el mundo. Metiose en un bote, y se fue a ver de cerca la mole
arrogante; la examinó y admiró por ambos costados y por proa y popa,
embelesado de tanta maravilla. La estructura y proporciones del casco,
que así expresaba la robustez como la ligereza; el extraño y novísimo
corte de la proa, rematada en forma tajante como un terrible ariete para
partir en dos a la nave enemiga; la colocación airosa de los tres palos;
la altísima guinda de estos; el conjunto, en fin, de armonía, fuerza y
hermosura, le dejaron asombrado y suspenso.

Vista por fuera la fragata, subió Diego a bordo, y acompañado de buenos
amigos que allí encontró, hizo detenido examen de todo; vio el reducto
blindado, el puente y alcázar, la extensa cubierta; en el primer
sollado, las potentes baterías con todos los accesorios para su
servicio; en la profunda caja central las máquinas; subió, bajó y
recorrió los departamentos del inmenso recinto, que era barco,
fortaleza, palacio y refugio de las almas valientes, y se sintió llamado
por voz del Cielo a encerrar su vida en aquel que le pareció santuario
de hierro, no menos grandioso que los de piedra. La \emph{Numancia}, que
así se llamaba el barco, venía de los astilleros de Tolón, nueva,
flamante como un juguete construido para los dioses\ldots{} Entusiasmado
ante tanta belleza, pensó por un momento Ansúrez que su patria había
recibido de la Divinidad aquel obsequio, y que este no era obra de los
hombres.

Y cuando la \emph{Numancia} pasó al Arsenal para completar su armamento
y arrancharse y proveerse de todo lo necesario a una larga navegación,
se fue el hombre a bordo con Pinel; bajaron al segundo sollado, a proa,
donde están los dormitorios de los condestables y contramaestres; se
metieron en uno de estos, y Ansúrez dijo a su amigo: «De aquí no salgo
ya. Arréglame todo como puedas. En casa está mi uniforme guardado con
alcanfor para que no se apolille. Tráemelo, y con él mis papeles. Vete a
ver al Mayor General o al oficial de derrota, que es don Celestino
Labera, mi amigo, y dile\ldots{} lo que quieras, Anselmo\ldots{} En fin,
que me voy; y si no puede ser de contramaestre, iré de cabo de mar, de
marinero ordinario, o aunque sea en el oficio más bajo de la
Maestranza.»

Pinel y los demás amigos se ocuparon activamente en este negocio del
honrado navegante, consiguiéndole plaza de Segundo Contramaestre (el
primero era otro excelente amigo y gran marinero, llamado
Sacristá)\ldots{} Y satisfecho de su empleo, el celtíbero no salió más
del barco, y en él se sentía tan consolado de sus tristezas como
peregrino que, tras un largo divagar, encuentra la magna basílica, y en
ella el misterioso encanto que apetece su alma dolorida.

\hypertarget{viii}{%
\chapter{VIII}\label{viii}}

El 8 de Enero del 65 salió la \emph{Numancia} de Cartagena para Cádiz,
llevando a bordo una Comisión de primates de la Marina, que debía
informar de las condiciones de la fragata. Toda la travesía fue una
serie de probaturas. Dócilmente obedecía la nave, haciendo todo lo que
se le mandaba, y vieron y apreciaron los señores su andar a máquina,
variando el número de calderas encendidas y los grados de expansión, y
el tiempo que tardaba en dar una vuelta en redondo. Probose asimismo el
andar a la vela, desplegando en los mástiles la enorme superficie de
lona. Era un encanto ver cómo el coloso, sensible a las caricias del
viento, hacía sus viradas por avante y en redondo con suprema elegancia
y precisión.

Reventaba de gozo Ansúrez viendo estas pruebas, singularmente las de
maniobras de vela, que eran su fuerte y su orgullo. En ellas ponía su
brío y ardimiento, expresados por su potente voz; ponía también su
corazón, pues solo ya en el mundo, privado de todos los amores que
embellecen la vida, había encontrado en la fragata un amor nuevo que le
salvaba de la tristeza y sequedad anímicas. En pocos días se encendió en
él la llama de aquel cariño nuevo: la fragata era su hija, su esposa y
su madre, y en ella veía el lazo espiritual que al mundo le ligaba. La
\emph{Numancia}, personalizada en la mente del Oficial de mar, era el
conjunto de todas las maravillas de la ciencia y del arte; un ser vivo,
poderoso, bisexual, a un tiempo guerrero y coquetón. La bravura y la
gracia componían su naturaleza sintética. No cesaba de alabar sus
múltiples atractivos, y ya decía «¡qué valiente!» ya «¡qué elegante!»

Había recorrido, de sollado en sollado, los innumerables departamentos y
divisiones de la interior arquitectura del barco, los cuales
correspondían a las necesidades de la guerra, de la vida y de la
navegación. Todo lo había visto y examinado con prolijidad, conservando
en su mente los pormenores de tantas y tan diferentes partes, de cuya
proporción y armonía resultaba la hermosura total. Las baterías le
enamoraban, y la máquina y carboneras encendían en él entusiasmo tan
hondo como el velamen gigantesco. Tenía la nave corazón, sangre, alas,
pies, y un rostro bellísimo, que era la peregrina disposición de las
viviendas donde tantos hombres según sus categorías se albergaban, la
opulencia de las cocinas y despensas, y todo lo concerniente al buen
comer, indispensable función de los hombres de guerra.

El 4 de Febrero salió de Cádiz la soberbia fragata, con mar llana y
Noroeste fresquito. En cuanto se zafó del puerto, puso rumbo a Canarias
con cuatro calderas encendidas. Por la tarde se aprovechó la mayor
frescura del viento, largando las gavias y algunas velas de cuchillo,
con lo que se ayudó el andar a hélice. A la cuarta singladura vieron los
navegantes el grandioso Teide, que desde las brumas del horizonte les
daba el quién vive. Hacia él maniobraron, y a media tarde dejáronlo por
estribor, pasando entre las islas de Gran Canaria y Tenerife. No fue tan
bonancible la travesía de Canarias a San Vicente, porque se les presentó
mar tendida y gruesa del Noroeste, que les cogía de costado; y la señora
fragata, que hasta entonces no había sufrido tal prueba, bailó
graciosamente, con diez balances de 25 grados por minuto, demostrando
que si grande era su ligereza, no era menor su estabilidad\ldots{} En
San Vicente se detuvieron el tiempo preciso para reponer el carbón
gastado desde Cádiz. Un calor pegajoso, un barullo de negros y mulatos,
que como solícitas hormigas metían el combustible en las carboneras,
incomodaron a los tripulantes en los tres días que permaneció el barco
frente a la isla inhospitalaria, desnuda de toda vegetación.

En sitio tan desapacible reverdecieron las melancolías de Ansúrez, y se
turbó la serenidad que desde el embarque en Cartagena traía en su alma.
Una tarde, invitado a la mesa de los maquinistas por uno de estos, que
era su amigo, se entabló conversación sobre cosas y personas
cartageneras, y el tercer maquinista, hombre simpático, mestizo de
francés y catalán, hizo alusión muy transparente al rapto de la hermosa
Mara. Saltó Diego con exclamación pronta y viva, como si avispas le
picaran. Mediaron palabras de curiosidad, excusas, interrogaciones
ardientes, y por fin dijo el maquinista que nadie como él hablar podía
de aquel suceso, porque era muy amigo de Belisario Chacón, y se sabía de
memoria su carácter, sus cualidades y defectos. El estupor de Ansúrez
subió de punto. Nunca pensó que en medio de los mares, a tanta distancia
del escenario de su drama de familia, viniese repentina luz a
esclarecerlo. A las manifestaciones que antes hizo, agregó el maquinista
que podía contar muchas cosas que el padre de Mara ignoraba. La
curiosidad ansiosa de este fue muy semejante a los balances que había
dado la fragata en la última travesía\ldots{} Pero como no era discreto
hablar del caso entre tanta gente, en la confianza de la sobremesa,
acordaron reunirse los dos a prima noche, después de picar las ocho.
Bien podían charlar sin reserva cuando uno y otro estuviesen francos de
guardia.

A la hora prescrita, arrimados al castillo de proa, hablaron largamente
Ansúrez y el maquinista Fenelón, sin más testigo que el vientecillo
terral, que una vez entrados los conceptos en el oído de Ansúrez, se los
llevaba mar adentro. Si no fuera discreto el terral, podría repetir
cláusulas de aquel coloquio en que el semi-extranjero refería sucesos
reales y daba sinceras opiniones. Cogidos en la onda del viento se
reproducen algunos trozos que no carecen de interés. Véase la muestra:
«Ha de saber usted, amigo mío, que en aquellos días de Octubre tenía
Belisario mucho dinero. Del bolsillo sacaba puñados de monedas de oro y
fajos de billetes. ¿Piensa usted que este dinero era mal adquirido? Yo
creo que no. Belisario es una cabeza destornillada, como la de todo el
que anda en tratos con la poesía; pero no pone su mano en lo ajeno: esto
me consta; he podido comprobar su honradez en las ocasiones de mayor
pobreza. Dice usted bien que ese dinero no pudo ganarlo en su comercio
de fruslerías\ldots{} pura farsa romántica\ldots{} Se disfrazaba de
vendedor\ldots{} ponía en verso los números\ldots{} Me pregunta usted si
sé la procedencia del dinero, y contesto que Belisario hacía también la
farsa del guardador de secretos\ldots{} Presumo que recibió fondos del
Perú, enviados por su madre para que se restituyese a la patria.»

---¿Y por qué---observó Ansúrez prontamente---no me habló\ldots{}
\emph{en plata}, para pedirme la hija? Aunque ni pobre ni rico me
gustaba el peruano, con ese adorno de la riqueza\ldots{} quiero
decir\ldots{} no viniendo el pretendiente a palo seco, mi contestación
hubiera sido muy otra de lo que fue.

---Pues\ldots{} Belisario no habló a usted de intereses---repuso
Fenelón,---porque es lo que llamamos un romántico\ldots{} ¿se entera
usted, amigo?\ldots{} porque llevando las cosas por derecho y obteniendo
la mano de la niña según el estilo corriente, no resultaba
poesía\ldots{} Lo poético era meterse por el camino más largo y más
difícil, manteniendo la ilusión, que es la salsa de que se alimentan las
almas románticas. Palabra de honor, que es así.

---No lo entiendo, ni creo que tenga sentido común nada de lo que usted
me dice\ldots{}

---Pues añadiré que también su hija de usted es una romántica de marca
mayor ---afirmó Fenelón riendo.---Romántica vino al mundo; el aire
andaluz agravó lo que bien puede llamarse enfermedad, y las lecciones de
las monjitas acabaron de rematarla\ldots{} ¿Tampoco lo entiende?

---¿Conoció usted a mi hija?

---La vi una sola vez. Sus ojos y las pocas palabras que le oí, me
revelaron su romanticismo agudo. Después, la he conocido mejor por el
reflejo de su alma en el alma de Belisario\ldots{} Pues como decía,
siendo los dos románticos furiosos, bien puede asegurarse que desecharon
todo proceder antipoético, para lanzarse a los fines de amor por los
espacios rosados y lindísimos de lo ideal\ldots{} ¿Tampoco lo entiende?

---No, señor, y líbreme Dios de entender esas monsergas\ldots{} Por lo
que usted me dice, voy comprendiendo que también es usted de esa cuerda
o vitola\ldots{} ¿Cómo llaman eso?

---Romanticismo\ldots{} Pero sepa que yo no soy romántico, ni mis
locuras, que también las tengo, son como las de Belisario y su hija de
usted. Yo, así por el lado catalán como por el lado francés, soy
esencialmente práctico y positivista. Si me hubiera encontrado en el
caso de Belisario, habría ido derecho a la confianza de usted alargando
la mano llena de dinero. Yo no desprecio el dinero, no lo llamo
\emph{vil}, no lo tengo por prosa, sino por la más alta poesía\ldots{}

---Hombre, ni tanto ni tan poco---dijo Ansúrez con inflexión
jovial:---quedémonos en un término medio\ldots{} Pues ahora me ha
entrado curiosidad de usted\ldots{} Dígame quién es, cómo ha venido a la
vida de perros de los maquinistas de vapor, y dónde y cuándo aprendió lo
que sabe, y el aquel que tiene para calar a las personas.

---Yo soy hijo de francés y española; me crié en Cataluña, y mi primera
educación fue para mejor oficio que este de maquinista. Mi padre ha sido
Director de \emph{Forges et Chantiers}, y aún desempeñaba el cargo
cuando se puso la quilla de esta magnífica fragata. Hoy está retirado
por su mucha edad, pero conserva en los talleres y en la Dirección tanta
influencia como cuando todo estaba bajo su mano\ldots{} Yo fui muy
aplicado en mis años primeros, como acreditan las certificaciones de mis
estudios prácticos en el \emph{Creuzot}, y los diplomas que gané en Lyón
y en París\ldots{} Ya que nombro a París, diré que en aquella ciudad tan
grande y bella se inició mi perdición, al tiempo que me asimilaba la
cultura y el saber ameno que allí flota en el aire y se le introduce a
uno, como si dijéramos, por los poros. Yo me di grandes chapuzones de
lectura; me puse al corriente de todo lo antiguo y moderno, así en
novela y poesía como en las demás artes, sin olvidar por eso mi
profesión científica. Pero mientras metía en mi entendimiento tanta y
tanta luz, mi voluntad se la llevaban los demonios, y me lancé a una
vida desarreglada y al delirio de los goces\ldots{} Veo que me oye usted
con la boca abierta, como si yo le contara un cuento fantástico. Usted,
hombre sencillo y patriarcal, no comprende nada de esto\ldots{} Abrevio
mi cuento, y vengo a parar en que mis escándalos tuvieron fin por
intervención de mi familia. Mi padre me sentenció a trabajos duros para
corregirme, por imponerme más segura penitencia, me embarcó de tercer
maquinista en la \emph{Numancia}. Ya sabe usted que la Compañía
\emph{Forges et Chantiers} corre con el servicio de máquina hasta que la
fragata vuelva de su expedición.

---Viene usted, pues, como galeote---dijo Ansúrez,---que así llamaban a
los criminales y perdidos que iban a remar en las galeras del Rey. Bien,
señor Fenelón. Ya veo que es usted hombre de historia, muy corrido en
trapisondas de tierra adentro, y sabedor de cosas de novela y
poesía\ldots{} que para mí son letra muerta, pues de ello no entiendo
palotada. Y veo también que no sólo corrió usted las borrascas en
aquella Babilonia de Francia, que llamamos París, sino que también debió
de andar por España como bala perdida, y en España fue amigo del
sinvergüenza de Belisario. ¿Andaba usted por la costa de Levante en
Septiembre y Octubre de año pasado? Sin que me responda, entiendo que
sí. Cuando el maldito peruano me robaba la niña, estaba usted en
Cartagena\ldots{} y cuando el ladrón y la joya robada se embarcaban no
sé para dónde, usted tomaba la vuelta de Tolón, donde su señor padre le
trincó y le impuso el castigo de galeras en nuestra fragata.

Afirmaba el francés, rechazando al propio tiempo toda complicidad en el
robo de Mara.

«¿Y cómo me explica usted---preguntó Ansúrez, que se resistía bravamente
a entrar en el terreno legendario,---cómo me explica que teniendo aquel
pirata sus bolsillos estibados de buena moneda, sirviera de segundo
mayordomo en un vapor de mala muerte?\ldots»

---Romanticismo, pura farsa romántica. El hombre satisfacía un
irresistible anhelo de disfrazarse y hacerse pasar por lo que no era,
siempre a la mira y asechanza de su propósito novelesco, tal como lo que
había visto en dramas y leído en libros de imaginación. Hacía, \emph{por
ejemplo}, el \emph{Montecristo}, y derramaba el oro para escribir en su
vida una pagina sorprendente de interés y emoción.

---No lo entiendo, no lo entiendo---dijo Ansúrez llevándose las manos a
la cabeza;---y como usted es también poeta, por su desgracia, no puede
contarme las cosas como son, sino como las ve en el farol de poesía que
tiene dentro de su cabeza. Y si esto no me entra en el magín, menos
entrará que Belisario pudiera seducir y engañar a mi niña sin emplear
artes de brujería, bebedizos o algún requilorio enseñado por los
demonios. ¿Cómo pudo ser, Señor, que se dejara trastornar mi hija por un
charlatán sin seso; ella, que era buena de su natural, y además traía
fresca la enseñanza de las Madres, que la instruyeron de moral, y me la
pusieron tan modosita y tan recatada que daba gloria verla y oírla?

---Las Ursulinas, amigo Diego---afirmó el francés,---no enseñaron a la
señorita nada, absolutamente nada. Salió del convento tan borriquita
como entró en él. Lo único que aprendió fue el disimulo de su
romanticismo\ldots{} Y también digo a usted que el alma romántica tiene
su mejor cultivo en el misterio y soledad del claustro, mi palabra de
honor\ldots{} El misticismo le pone luego el capuchón para que se
disfrace y pueda engañar más fácilmente al mundo.

Enorme confusión llevó esta idea al pensamiento de Ansúrez. No sabiendo
cómo contradecir al francés, calló\ldots{} y ambos perdieron sus miradas
en el mar sosegado y dormido que delante tenían. Pensó el contramaestre
que su compañero de navegación había cargado la mano en las dosis de
Jerez con que se confortaba después de las comidas, y que por esta
causa, más que por su embriaguez de cultura literaria, estaba el hombre
a medios pelos.

\hypertarget{ix}{%
\chapter{IX}\label{ix}}

La campana picó el \emph{tan-tan} de las nueve, y aún charlaban
maquinista y contramaestre arrimados a la borda, junto a la amura de
estribor. Repitió Ansúrez sus conceptos de incredulidad; insistió en que
nada comprendía de las explicaciones enrevesadas que daba Fenelón al
suceso de autos, y por fin, buscó nueva luz con esta pregunta: «¿Y qué
hacía Belisario con tanto dinero? Me figuro que emplearía buenos patacos
en pagar a los traidores que le ayudaron en su robo.»

---En esto fue tan liberal el hombre, que hay en Cartagena quien se
\emph{ha puesto las botas}, como suele decirse, con la fuga de la niña
de Ansúrez. La criada, \emph{por ejemplo}, que servía en la casa cuando
usted trajo a Mara del convento, y que luego siguió visitando a la
familia con pretexto de vender tortas y polvorones, se casó en Noviembre
y puso una pastelería en la calle de la Caridad.

---¡Ah!\ldots{} Venancia---exclamó Ansúrez apretando los puños;---¡esa
traidora, que a todos nos engañó!\ldots{} Yo le haría pagar sus
tercerías villanas si ahora la cogiera\ldots{} ¡Indecente, hija de
\emph{tal}, y \emph{tal} ella misma, gran perra\ldots!

---Y no es esa la única que se ha redondeado con los dineros del
amigo\ldots{} Muchos estrenaron ropa y pusieron gallina en el puchero
días y días y semanas. Y aquí mismo tiene usted al Cabo de mar, ese José
Binondo, que también se guarneció el bolsillo\ldots{} mi palabra\ldots{}
con la plata del americano. No me ponga esa cara de santo en éxtasis. Es
usted un inocente, un buenazo, que se fía de cualquiera, y va por la
calle diciendo: «¿No hay por ahí alguno que me engañe?»

---Pues mire usted, señor Fenelón---declaró Ansúrez con franqueza
candorosa:---yo sospechaba de Binondo, yo tenía la idea de que este
amigo no era fiel\ldots{} Y no me fundaba en rumores ni hablillas, sino
en algo que notaba yo en él cuando hablábamos\ldots{} una sombra, un
mirar para otro lado, un tonillo dengoso que tiene la voz de los
traidores\ldots{} Ya puede andar con cuidado el hombre, porque esa
cuenta tiene que pagármela\ldots{} ¿Y cómo ganó Binondo los duros del
peruano?

---Al sacar a la niña, la condujeron a una casa de pescadores en Santa
Lucía. Binondo se encargó de llevarla en su lancha a bordo de la goleta;
servicio arriesgado\ldots{} que realizó al amanecer, después de untar de
amarillo las manos de un cabo de la Comandancia. Cuando esta pesquisaba
con Roque Pinel, y revolvía el puerto y la ciudad, la niña y su amante
se mecían tranquilamente en la goleta, contando los minutos que habían
de tardar en salir a la mar\ldots{}

---Salieron, ¡ajo!---clamó Ansúrez entre suspiros hondos,---sin que la
autoridad de mar ni la de tierra supieran cumplir su obligación. El
dolor de un padre no significa nada para los que mandan\ldots{} La
autoridad, como tal autoridad, no tiene hijas\ldots{} Y dígame usted
ahora, ya que todo lo sabe o dice saberlo: ¿es cierto que la goleta
llevaba la vuelta del Pacífico?\ldots{} ¡Ajo!, pongamos que lleva
retraso de tres meses por malos tiempos y averías gordas\ldots{} Tendría
gracia que la encontrásemos, desarbolada y sin gobierno, que nos pidiera
auxilio, que se lo diéramos, y que al traernos a bordo a los náufragos
viéramos entre ellos a mi querida hija y a mi aborrecido yerno. Sería
como si los pescáramos en alta mar.

---No sueñe usted ni se nos vuelva también romántico. La goleta
\emph{Lady Seymour} habrá pasado por estas aguas\ldots{} sabe Dios
cuándo\ldots{} Pero en ella no van Belisario y Mara: su plan era
quedarse en Gibraltar, y tomar el vapor inglés que sale de allí el 15 de
cada mes para Aspinwall, istmo de Panamá\ldots{}

---Entendido\ldots{} A fe que no son tontos. Esto sí lo entiendo; como
que es de mi oficio de mareante, y aquí no hay romanticismo que valga.
Vea por dónde nos fastidia el condenado istmo. Ya conocen esos pícaros
el atajo\ldots{} Vaya, que la juventud afina\ldots{} sabe más que los
viejos\ldots{} Bien recuerdo que el americano de presa tenía grande
afición desde chiquito a las cosas de mar, y debía conocer los caminos
entre su tierra y Europa, que son caminos endemoniados por acá y por
allá\ldots{} Dios permite que la gente joven se nos adelante y nos tome
las vueltas. Si es cierto lo que usted dice, ya estarán esos locos en el
Perú.

---Por mi cuenta, habrán llegado en Diciembre\ldots{} a no ser que se
los haya tragado el mar\ldots{} que todo podría ser\ldots{}

Ansúrez miró al francés como reconviniéndole por su pesimismo. Golpeando
la borda, dijo: «¡Ajo!, no faltaba más sino que mi niña se ahogara con
ese tunante. Santo y bueno que se haya dejado robar; pero irse al fondo
con él\ldots{} eso no puedo consentirlo\ldots{} Dispense usted, señor de
Fenelón: no sé lo que digo\ldots{} Quiero tanto a esa criatura, que todo
se lo paso, todo se lo perdono, con tal que viva. Si en mi mano tuviera
yo el gobierno del mar y de los hombres que andan en él; si tocando mi
pito de contramaestre pudiera echar a pique una embarcación y salvar a
unos tripulantes y a otros no, yo sacaría del agua por los cabellos a mi
querida Mara, y al negro ese lo dejaría para merienda o almuerzo de los
tiburones. Pero estamos soñando\ldots{} que esto es \emph{hablar de la
mar}, o sea hablar dormidos\ldots{} ¡Quién sabe dónde estará mi hija, ni
si vive o muere, ni si volveré yo a verla!\ldots{} Pongamos a Dios donde
debe estar, por encima de todas las cosas, y no nos metamos en
averiguaciones de las cosas distantes ni de las cosas venideras.»

---Respetemos, sí\ldots{} los caprichos del Acaso---dijo Fenelón
entornando sus ojos con vaga soñolencia,---y lo que sea\ldots{} será y
sonará\ldots{} Yo pregunto: ¿vamos, \emph{por ejemplo}, al Callao?
¿Vamos en son de paz, o en son de guerra?

---Dios y nuestro Comandante don Casto dirán a dónde vamos, y lo que
tenemos que hacer por allá.

Esto replicó Ansúrez, añadiendo a sus palabras un ademán o intento de
santiguarse. Pero la intención se quedó a medio camino entre la mano y
la frente. El maquinista, soñoliento y \emph{ajerezado}, manifestó
deseos de embutir su persona en la litera, y en esto sonó la campana.
\emph{Tan-tan}, \emph{tan-tan}: las diez.

«Usted se acuesta, yo no---murmuró Ansúrez despidiéndose con una
cabezada.---Aquí me quedo pensando\ldots»

Pensando estuvo largo tiempo de aquella noche estrellada y apacible. Por
la mañana, entre la algarabía de pitos marineros y de militares
cornetas, salió de San Vicente la fragata, bien arranchada de carbón,
que gastaba con economía, aprovechando la brisa frescachona para navegar
a un largo con todo su aparejo. Días hubo en que se retiraron los fuegos
de las calderas para marchar en brazos del aire vago. Los pies, o sea la
hélice, reposaban, y sueltas al viento las alas daban un andar de cuatro
a cinco millas. Así transcurrieron días, durante los cuales el buen
Ansúrez no cesó de cavilar en su asunto; y revolviéndolo y mirándolo por
todas sus caras, trataba de reconstruir el rapto de su hija para
convertirlo de novela en historia. De la vaguedad iba saliendo el
sentido real del suceso; y si a veces este se anegaba en las tinieblas
de su origen, de improviso resurgía iluminado por la verdad.

Con los preciosos datos aportados por el hispano-francés, llegó Diego a
modificar su apreciación del hecho que había dejado huella tan honda en
su alma. «Será muy raro---pensaba---que ahora salgamos con que no es el
Belisario tan malo como pensé, y que la condenada poesía y los versos no
le estorban para ser hombre honrado, caballero y buen cristiano. ¿Tendré
yo la culpa, por mi brutalidad de aquella tarde en la correduría; tendré
yo la culpa, digo, de que mi niña se me escapara por el aire, viendo que
yo le cortaba los caminos naturales de tierra? Pero él debió decirme:
`Tengo posición; soy nacido de buenos padres, y quiero casarme por la
ley de Dios y con toda la decencia del mundo'. Si esto no dijo, por mor
de la condenada \emph{romantiquería}, no es mía la culpa, sino de
él\ldots{} O será culpa de los dos, y resultará que yo también soy lo
que se dice román\ldots{} ¡Romántico yo!, no puede ser. Un padre no es
eso, diga lo que quiera ese borrachín de Fenelón\ldots{} un padre no es
poeta en lo tocante a nada de su hija\ldots» Cuando estas cosas
discurría, la fragata cortaba la Línea Equinoccial.

El paso de la Línea fue, como es costumbre en la mar, festejado con
alegría carnavalesca. Ansúrez estaba en todo, firme en sus funciones de
contramaestre, sin dejar de hilar en su interior el pensamiento que le
dominaba. Dos seres, uno dentro de otro, existían en él: el padre de
Mara, y el hombre solitario que amansaba su pena con las obligaciones
fielmente cumplidas, y con el cariño al barco, que era su casa y su
templo\ldots{}

Navegaban ya por el hemisferio Sur; ya no veían las amadas estrellas de
la Osa Mayor; en el firmamento austral servíales de guía la espléndida
Cruz. Ante ella, como en otros días ante la Osa, seguía el buen Ansúrez
hilando su pensamiento; del copo salía la hebra, que nuevamente se
deshacía, volviendo a la maraña de donde salió\ldots{} A los 10 grados
de latitud Sur, en el paralelo de Pernambuco, se hallaba Diego
plenamente convencido de que toda la responsabilidad de su desdicha era
de Belisario y de su arrastrada poesía\ldots{} A los 24 grados, paralelo
de Río Janeiro, creía firmemente que la culpa era suya, y que él también
hacía versos sin saberlo. En los 30 grados, remachaba esta idea,
llegando a sostener que cuanto dijo en la correduría contra el americano
era pura poesía rabiosa, pues también la rabia es romántica, como se
podía ver en el teatro, donde todo el interés consiste en que lloren las
mujeres, y los hombres amenacen y griten como locos\ldots{}

En esto llegaron a Montevideo, donde encontraban descanso, la alegría de
víveres frescos, del bajar a tierra y tratar con españoles. Aunque
políticamente no fueran aquellos nuestros hermanos, por el habla y los
sentimientos no podían negar la casta. Prueba plena del parentesco daban
los valientes americanos con su afición al juego de la guerra civil.
Como nosotros, se dividían en furiosos bandos, y se perseguían y se
fusilaban \emph{por dar gusto al dedo}. Cuando fondeó nuestra fragata en
aguas del Uruguay, había terminado una guerra fratricida; pero como el
abolengo hispánico no se avenía con el reposo de las armas, pronto los
orientales declararon la guerra al Paraguay. El Brasil, que había sido
enemigo, trocose en aliado; la Argentina también sintió ganas de
quimera. Aquellos pueblos, establecidos en las regiones más feraces del
mundo, tenían horror, como su madre España, a la ociosidad militar, que
es la paz. Allá, como aquí, la turbaban por un daca esas pajas, o
simplemente por esa ironía del tiempo que llamamos \emph{pasar el rato}.

Por su mucho calado, la \emph{Numancia} echó el ancla a seis millas de
la ciudad. El carboneo se hacía difícilmente; el trabajo era rudo. En
las clases de marinería y tropa, pocos individuos tuvieron permiso para
saltar a tierra. Oficiales y Guardias Marinas gozaron algunos días de
aquel esparcimiento, y más aún el personal de máquinas. Todos volvían
diciendo que la ciudad parecía un campamento, y que en ella no se
hablaba más que de aprestos militares. A pesar de esto, el amigo
Fenelón, que en la mar se sentía por lo común fuera de su elemento,
pasaba en tierra todo el tiempo que se le permitía, empalmando las
tardes con las noches y estas con las mañanas.

«Puede usted creerme, mi querido Ansúrez---decía contándole a este sus
correrías urbanas,---que las mujeres de este país son preciosas,
francas, sensibles, y más instruiditas que las de allá\ldots{} Bajo mi
palabra de honor, afirmo que me han gustado veintitrés, que me he
sentido enamorado bárbaramente de cinco, y locamente de dos. He vuelto a
bordo con el corazón en pedazos y el cerebro como un volcán\ldots{} Yo
soy así\ldots{} Mi naturaleza es la adoración de la mujer, y mi destino
entregarle mi alma para que juegue con ella, aunque con estos juegos me
deje alma y almario hechos trizas\ldots{} No puedo remediarlo. Si en vez
de tocar en esta ciudad hermosa y culta, hubiéramos arribado a un lugar
de tribus salvajes, no habría faltado una negra bozal que me hiciera
tilín, como ustedes dicen, ni yo habría dejado de enloquecer por ella,
trayéndome acá su negra imagen estampada en mi corazón\ldots{} Ya, ya sé
lo que va usted a decirme: que soy romántico. No, amigo mío: soy
clasicote, un poquito pagano y un muchito sensualista y experimental.
Entiendo que este culto mío de la mujer es una pequeña filosofía, mi
palabra de honor\ldots{} Vámonos a mi camarote, y adormeceremos nuestras
penas con unas copas de Jerez\ldots{} Venga usted, acompáñeme\ldots{}
¿Cuándo seguiremos nuestro viaje?\ldots{} Ganas tengo ya de ver otras
tierras. Usted, que ha pasado dos veces ese infernal Estrecho, dígame:
¿cuál es el tipo y cariz de la hembra patagona? ¿Es bravía, procerosa de
talla, alta de pechos, de ojos flamígeros y boca hasta las orejas? ¿Se
pinta, \emph{por ejemplo}, rayas negras en la cara, y se cuelga de la
nariz un arete?\ldots{} Vamos, no sea remolón: nos espera el amigo
Jerez, que es mi alegría y el descanso de mis penas\ldots{} ¿Se ríe
usted, camarada?\ldots{} ¿Esa risita quiere decir que me admira o que me
compadece?\ldots{} Sea lo que quiera, yo no me enfado, mi palabra de
honor\ldots»

Cogidos del brazo descendieron al segundo sollado, y en el camarote de
Fenelón trincaron de lo lindo. Ansúrez era hombre de fabulosa
resistencia contra la embriaguez; el otro, por la reiteración de su
vicio, necesitaba dosis extremadas para perder el dominio de la palabra
y del pensamiento. Ambos permanecieron en el punto fisiológico a que
habitualmente les llevaba una ingestión no excesiva del precioso licor.
El Jerez del mecánico solía ser alegre; el de Ansúrez era siempre triste
y aplanante. «Mi estimado señor Fenelón---dijo a su amigo:---yo, la
verdad, no me alegro mucho de haber conocido a usted\ldots{}
porque\ldots{} también lo aseguro bajo mi palabra de honor\ldots{} más
me gustaba creer que Belisario era un pillo vagabundo, que no creerle
honrado y caballero de posibles\ldots{} Con odiarle me consolaba yo, y
ahora resulta que\ldots{} \emph{por ejemplo}, como usted dice\ldots{}
debo quererle. Esto me pone triste, pero muy triste, señor de
Fenelón\ldots{} ¡Ajo!, yo le juro por mi sangre, que a veces me dan
ganas de arrojarme al agua. Ahogándome, no me atormentará la idea de que
Belisario es un hombre de bien, y de que mi hija le querrá más que me
quiso a mí. Esto me pone loco\ldots{} He pedido a la Virgen del Carmen
el favor de que no me deje morir sin ver a mi hija\ldots{} He llegado a
creer que me lo concederá\ldots{} pero ¡ajo!, me carga una cosa, señor
de Fenelón. En la cara de la señora Virgen del Carmen, cuando le rezo,
he visto un cierto guiñar de ojos y un cierto mover de labios, como si
se burlara de mí. También la Virgen cree que Belisario es bueno, y que
mi Mara hizo bien en irse con él, dejando a su padre en esta
soledad\ldots{} Y cuando ella lo cree, cierto será que mi hija está
contenta, que ha hecho una gran boda, y que yo debo consumirme de rabia,
condenado a tocar un día y otro el pito de contramaestre para que los
marineros entren en faena; y mientras yo doy mis pitidos, allá están mi
morenita y el negro gozando de sus amores, quizás dándome nietos, que yo
no he de ver\ldots{} Dígame usted bajo su palabra de honor, o por encima
de ella, que esto es muy triste, pero muy triste, y que lo mejor que yo
puedo hacer es tirarme al agua\ldots{} Como estoy de buen año, ya usted
lo ve, ¡vaya una meriendita que voy a dar a los tiburones!»

\hypertarget{x}{%
\chapter{X}\label{x}}

---No te tires, Diego, no te tires---le dijo Fenelón, que en sus
alegrías vínicas trataba de tú a todo el mundo.---El mar es muy
frío\ldots{} Comprendo todos los amores, menos los amores de los
peces\ldots{} Yo me agarro a la vida, y no la suelto\ldots{} ¡Se
encuentra uno tan bien en este mundo, aun estando condenado a
galeras!\ldots{} El galeote rema y rema pensando en la mujer que ha
dejado en tierra, o en la que va a encontrar en el primer puerto de
escala. ¿Cómo será esta mujer esperada? ¿Será morena o rubia?\ldots{} El
galeote la ve en su imaginación, y sigue remando\ldots{} Boga, boga,
marinerito, que la bella te aguarda\ldots{} Mi remo es la hélice; la
máquina mi corazón, la hulla mi sangre\ldots{} Yo te empujo, navecita
mía: llévame pronto junto a mi morena, junto a mi rubia\ldots{}

Vencido de un sopor intenso, Ansúrez empezó a dar cabezadas; Fenelón le
agarró del brazo, y con sacudidas quiso despabilarle. Irguiendo la
cabeza, el contramaestre aprovechó aquel despejo para poner a salvo su
dignidad. Dio a su amigo las buenas noches con palabra tartajosa, y
palpando mamparos llegó a su dormitorio, y en el coy se arrojó, que fue
como si se arrojara en el mar del sueño, porque al instante se quedó
dormido\ldots{} Y antes de amanecer le despertó el viento de la Pampa,
que se inició con un silbar prolongado y lúgubre en el aparejo.
Acudieron los de guardia y los de retén a las maniobras precisas para
defender la nave de la cólera rapaz del pampero, que algo quería
llevarse de arboladura o de cubierta. Calaron masteleros, pusieron al
filo las vergas, y largo tiempo emplearon en trincar todo lo que arriba
o abajo podía ser arrebatado por el huracán: botes, toldos, mangueras y
el sin fin de objetos movibles que toda gran embarcación lleva consigo
como y donde puede. El viento la obliga, cuando menos se piensa, a
meterse sus chirimbolos en los bolsillos, o a sujetarlos fuera con esos
apretados nudos que sólo saben hacer los marineros.

Por fin, tras luengos días terminó el carboneo, y la \emph{Numancia}
zarpó acompañada del transporte \emph{Marqués de la Victoria}, que le
llevaba el combustible para la travesía del Estrecho y mares del Sur del
Pacífico. No empezaba con bendición la nueva etapa, porque a las pocas
horas de salida la máquina dijo que no daba una vuelta más, y no hubo
más remedio que arribar a la boca del Plata y fondear en el Banco
Inglés\ldots{} ¿Qué ocurría? La recalentadura de un cojinete había
inutilizado la máquina\ldots{} En aquellos tiempos cualquier accidente
de esta naturaleza llevaba la consternación y la ansiedad a las almas de
los tripulantes.

Los maquinistas, franceses todos, diagnosticaron con pesimismo; por
fortuna el oficial de Ingenieros don Eduardo Iriondo, tan animoso como
entendido, tomó a su cargo la cura del organismo enfermo, y a las
veinticuatro horas, vencida la parálisis y recobrado el movimiento,
salió la \emph{Numancia} mares afuera, cortando las olas con su
arrogante espolón. El transporte no podía seguirla en conserva; hubo de
moderar la fragata su paso ligero, atizando fuego en sólo tres calderas.
A los dos días de navegar en esta forma, repitiéronse los casos de mala
suerte, y el más lastimoso fue que el segundo Comandante, don Juan
Bautista Antequera, resbaló bajando la escala del falso sollado, y en la
violenta caída se rompió una pierna\ldots{} Desgraciada y reincidente
avería, pues la misma pierna por el mismo sitio se había roto meses
antes en Nápoles, cayendo, no de la escala de un buque, sino de la silla
de un caballo\ldots{} Triste fue aquel día: el Segundo Comandante era
muy querido de iguales e inferiores. Mientras en el camarote de popa los
médicos reducían, entablillaban y bizmaban la rotura del hueso, la
fragata, insensible al accidente, se columpiaba sobre las olas con
cabezadas y balances harto expresivos. Quería juego, y hacer alarde de
arrogancia marinera.

La mala sombra seguía. Un pobre marinero llamado José López, que murió
de fiebre de reabsorción, fue arrojado al agua al amanecer de un brumoso
día. Las tristezas no querían abandonar a la \emph{Numancia}, que
bailando seguía, retozona y ligera de cascos, como adolescente que se
estrena en la vida y no conoce los peligros del mundo\ldots{} Luego vino
mar gruesa tendida, con viento racheado y duro: la fragata, poseída de
verdadero frenesí coreográfico, lucía su elegancia y poder, y ya se
inclinaba hasta hundir el espolón en las turbulentas ondas, ya se erguía
majestuosa, sacudiéndose el agua y despidiendo a un lado y otro
chorretazos de espuma. Menos airoso en su lucha con el viento y la mar,
el caballero que a la dama escoltaba y servía, el buen \emph{Marqués de
la Victoria}, se encontró en gran apuro por la obligación de marchar en
conserva. No tuvo más remedio el pobre galán que ponerse a la capa, con
rumbo distinto del que su señora llevaba, y navegando de tal suerte, se
perdió de vista. La \emph{Numancia} siguió su camino, segura de que el
caballero sirviente parecería mares adelante\ldots{}

He dicho que sin interrupción se sucedían las desgracias, y una de ellas
fue que el Cabo de mar José Binondo, que se hallaba en el palo mayor
aferrando la gavia, sufrió un grave accidente. Apoyaba los pies en el
tamborete, las manos en la verga, cuando un fuerte balance de la fragata
le hizo perder el equilibrio, y cayó sobre el aro mismo de la cofa con
fuerte golpe en el pecho. Tuvo bastante destreza en aquel crítico
instante para engancharse de pies y manos en la burda del mastelero, y
pudo deslizarse hasta coger la escala del obenque mayor. Allí no pudo
tenerse, porque el tremendo porrazo en el pecho le privaba de
respiración. Los compañeros subieron a socorrerle, y no sin dificultad
le bajaron a cubierta, donde le recibió Sacristá, el cual, viéndole
demudado y sin habla, le mandó a la enfermería. Allá quedó el infeliz en
manos del médico don Luis Gutiérrez, que diagnosticó rotura de dos
costillas y hundimiento del esternón\ldots{} El pobre Binondo arrojaba
sangre por la boca, y en los intervalos de sus arcadas angustiosas pedía
que le llevasen el Cura y los Sacramentos, pues ya se veía difunto y
amortajado con las parrillas en los pies, para descender rápidamente al
fondo de las aguas.

Seguía la \emph{Numancia} su rumbo hacia la boca del temido Estrecho. En
aquellos días y noches, Sacristá y Ansúrez no se daban punto de reposo,
alternando en el servicio, o haciéndolo mancomunadamente cuando la
complejidad de maniobras en tan difícil navegación lo exigía. El pito
marinero no cesaba de lanzar al aire su estridor agudísimo, rasgando el
claro son de las cornetas, que llamaban a galleta y café, a zafarrancho
de camas, a baldeo, a instrucción, a ejercicio\ldots{} El Oficial de
derrota no bajaba del puente, y don Casto Méndez Núñez, incansable en
las observaciones y estudio del derrotero, no apartaba sus ojos, con
catalejo o sin él, de las brumas que por estribor ofuscaban la costa.

El 11 de Abril amaneció benigno: cayeron la mar y el viento; la fragata
navegaba con cuatro calderas encendidas, ayudándose de las mayores y
foques; era su marcha arrogantísima; la proa potente saludaba con graves
cortesías a las olas que hacia ella corrían de Sur a Norte, lentas, más
ceremoniosas que hinchadas. En la amura de estribor, Sacristá y Ansúrez
lanzaban sus miradas de aves de mar al paredón neblinoso del horizonte.
Poco después de que el vigía cantase \emph{Tierra} desde la cofa,
Ansúrez, conocedor de aquella región, anunció la recalada al Estrecho.

Llamado al puente por Méndez Núñez, el Segundo Contramaestre saludó como
práctico al jefe. «Mi Comandante---le dijo,---la tierra alta que vemos
es \emph{Cabo Vírgenes}; sigue hacia el Sudeste una tierra más baja,
\emph{Punta Miera}, que los ingleses llaman Pungeness\ldots{} Hay un
banco\ldots{} el \emph{Banco del Cabo.»} A una pregunta seca de Méndez
Núñez, tan hombre de mar como el primero, y que buscaba un buen informe
donde quiera que pudiesen dárselo, Ansúrez contestó con la misma
sequedad y modestia que usar solía don Casto: «Mi Comandante, con cuatro
millas de resguardo no puede haber peligro\ldots»

Lahera ordenó la virada en el punto y ocasión convenientes. Al mediodía
la fragata derivaba hacia el Oeste su proa; poco después tenía por
estribor las alturas patagónicas, por babor las soledades de la Tierra
del Fuego. Montada la Punta, se enmendó la marcha, arrimando a la costa
Norte para precaverse de los bajos del Sur. A las cinco de la tarde
fondeó la \emph{Numancia} en la bahía de \emph{Posesión}, para tomar
respiro y aguardar a su extraviado caballero el \emph{Marqués de la
Victoria}, cuyo rumbo y suerte se desconocían. La dama, intranquila, no
cesaba de preguntar a todos sus tripulantes si sabían o sospechaban
dónde había ido a parar el galante satélite.

A menudo se informaba Diego del estado de Binondo, pues aunque le cobró
gran ojeriza por haber auxiliado al seductor de Mara, como buen
cristiano le compadecía. En peligro de muerte estaba el Cabo de mar, y
sus horas en la enfermería de paz eran de infinita tristeza, que si los
dolores de la caja del cuerpo y las angustias de la respiración le
abrumaban, no se sentía menos agobiado y enfermo del espíritu. Habló con
Ansúrez el médico don Luis Gutiérrez, y después de explicarle el por qué
de hallarse Binondo tan abocado a la muerte, le dijo: «Bien puedes bajar
a verle, que está el hombre deseoso de hablar contigo; y si tardas en
darle ese gusto, quizás no le encuentres vivo\ldots{} Según entiendo,
tiene contigo una deudilla de conciencia: no quiere irse al otro mundo
sin quedar en regla con sus acreedores, y me parece que a ti ha de
pagarte a toca-teja. Algo me ha dicho del caso\ldots{} pero como es
cuenta particular, allá los dos.»

Bajó Ansúrez a la enfermería, y a la tristísima claridad de aquel
recinto, que sólo recibía una limosna de luz solar por la escala de
entrada, y el aire por una manguera de lona, vio al que fue su amigo
postrado en la colchoneta colgante, cubierto de un oleaje de mantas, por
entre las cuales sólo asomaba su cabeza, tocada de un pañolón a guisa de
turbante, y el hombro y brazo derechos. El rostro de Binondo modelo de
fealdad malaya, era de los que no se alteran visiblemente, ni con las
alegrías del vivir, ni con las agonías mortales. Ansúrez no halló en él
otra novedad que el cambio de color amarillo cobrizo en un verde sucio
con arrebato febril en los pómulos. La débil claridad hacía más plano el
rostro, como bajo-relieve tallado en una tabla con muy poco saliente de
las anchas narices aplastadas y de la rasgada hendidura bucal\ldots{}
Los ojuelos negros y chicos, de brillantez canina, animaban aquella
careta que sin el mirar no habría parecido cosa humana. Sentose Diego
frente a su amigo, y puso la mano sobre las mantas, en el bulto que
hacían las rodillas; y cuando pensaba las primeras palabras que había de
pronunciar en la visita, habló el enfermo, y dijo: «Ya ves,
Diego\ldots{} qué malo estoy\ldots{} Se me ha roto el casco por la
cuaderna mayor y el bao real\ldots{} Quebrados tengo los palmajares y
los trancaniles\ldots{} En fin, que me voy de este mundo malo a otro
mejor\ldots{} ¡Y tú, Diego, como si no fuéramos amigos de toda la vida!
Si no te mando llamar no vienes a verme, perro, mal hombre, todo porque
el francés maquinista te puso la bocina en la oreja para decirte que si
yo, que si tal, que si tu niña\ldots{} Óyeme a mí, Diego, que verdad
como la que yo te diga no has de oír de nadie\ldots{} Ya mis aljibes
están llenos del agua limpia de la verdad\ldots{} y para esto se
vaciaron del agua corrompida de la mentira.»

Esta figura, empleada ingenuamente por el rudo marinero, impresionó y
enterneció al amigo que le visitaba. «Ya sé, ya sé---le dijo con
emoción,---que no has de ocultarme la verdad\ldots{} Estás en franquía
para vida mejor\ldots{} ya has comulgado, \emph{ya tienes el práctico a
bordo}\ldots{} No has de salirte con embustes, porque si lo hicieras,
llevarías tu alma llena de contrabando\ldots{} y el contrabando ya sabes
que no pasa, no pasa en aquellas aduanas\ldots{} En fin, José Binondo,
si no quieres molestarte, nada me digas, que yo, sabedor de lo que has
de decirme, te perdono de todo corazón, como cristiano que soy\ldots»

---Poco a poco\ldots---dijo el enfermo extendiendo el brazo que tenía
fuera de mantas.---No te des por enterado con las verdades que te soltó
el francés, y escucha las mías, que son más de ley\ldots{} Él te habrá
dicho que favorecí la escapada de tu niña, y que la llevé a la goleta
con tanto cuidado como hubiera embarcado a mi propia hija, si viviera.

---Sí\ldots{} Te portaste mal\ldots{} Fue acción fea la tuya: olvidaste
nuestra buena amistad\ldots{}

---Poco a poco. Diego\ldots{} Déjame que te diga\ldots{} que te diga el
por qué, pues no hay acción que no tenga su por qué.

---El por qué no me importa ya. Yo te perdono, y con perdonarte queda
liquidada nuestra cuenta, Binondo.

---Déjame, déjame que sea yo quien liquide\ldots{} Lo que dije y referí
a don José Moirón para que me absolviera de mis pecados, ¿no has de
saberlo tú? Nuestro capellán me encargó mucho que a ti te diera mis
razones, y te las doy. Con el práctico a bordo, como dices, te llamo, y
al despedirme de ti te dejo mis razones, Diego; óyelas: yo favorecí la
fuga de tu Mara, porque yo también tuve una hija\ldots{} ya sabes cuánto
quería yo a mi Rosa\ldots{} Era un ángel: feíta, eso sí; ¡pero qué mona
de Dios!\ldots{} Las narices tenía chatas, como yo; los ojos chiquitos,
como los míos, pero con mucho aquel; la color quebrada; el cuerpo con
una salazón que ya ya\ldots{} Se parecía más a mí que a su madre, que
era \emph{Pepona la lagarta}, bien lo recuerdas, lavandera de la ropa de
maquinistas en el Arsenal\ldots{} Pues mi niña era una verdadera rosa
sin espinas\ldots{} Aunque por broma la llamaban la \emph{Rosa amarilla}
o \emph{Rosita la fea}, para mí era más guapa que los serafines\ldots{}
Bien sabes, Diego, cuánto la quería yo, y cómo me miraba en ella\ldots{}
Me muero con gusto, porque sé que voy a verla\ldots{} Así me lo ha dicho
nuestro capellán\ldots{} Pues recordarás que mi adorada hija se
enamoriscó de un fogonero italiano. No era mal chico; pero yo me indigné
de que la niña pusiera en persona tan baja su voluntad. Pues la cogí un
día, y con una estaca le di tal paliza, que quedó mi ángel hecho una
lástima. ¡Ay, ay, Diego, cuánto he llorado aquella brutalidad que
hice\ldots! Mi Rosa, mientras yo la pegaba, me decía: «Aunque usted me
mate, padre, querré siempre a mi \emph{Curtis.»} Así llamaban al
italiano\ldots{} Un día la vi que derrengadita y paticoja, salía en
busca de Curtis, y yo, ¿qué hice?\ldots{} la cogí por un brazo y me la
llevé a casa, donde le di bofetadas y me parece que algún
mordisco\ldots{} ¡Oh, qué malvado fui!\ldots{} Pues desde aquel día la
niña empezó a desmejorar\ldots{} a caer y entristecerse\ldots{} ¡Ay, qué
pena tan grande! La llevé al médico, y el médico me dijo que la niña
padecía mal del corazón\ldots{} En fin, que una mañana la oí
quejarse\ldots{} Corrí a ella, y se me quedó muerta entre los
brazos\ldots{} ¡Ay de mí!, yo no tenía consuelo\ldots{} yo quería
matarme para que me enterraran con aquella prenda querida. Los palos y
bofetadas que le di me dolían entonces en el corazón y en toda el alma.
¡Yo verdugo, y ella una mártir inocente! La enterramos al siguiente día
al anochecer\ldots{} Curtis venía detrás cuando la llevábamos\ldots{} Yo
me moría de dolor\ldots{} Curtis y yo la bajamos al hoyo\ldots{} El
italiano era un mar de lágrimas, y yo un mar de amargura\ldots{}

Vio Diego el llanto que corría por las mejillas verdes y por la cara
plana del Cabo de mar. Contagiado por su duelo, pero sin comprender la
relación que pudiera tener el caso de \emph{Rosita la fea} con el de
\emph{Mara la bonita}, Ansúrez, transcurrida una larga pausa, le dijo:
«Bien, José\ldots{} tu hija se murió\ldots{} Ni Mara ni yo teníamos la
culpa de tu desgracia. Si Dios te quitó a tu hija, ¿qué adelantabas con
quitarme la mía?»

---Poco a poco, Diego---replicó Binondo acopiando todo el aliento
posible para expresar lo que faltaba.---No me has entendido\ldots{}
Sabrás que la muerte de mi niña, de aquel cielo mío, fue una lección que
Dios me daba\ldots{} una lección terrible\ldots{} Dios me decía esto,
Ansúrez: «Padres, antes que dejar morir a vuestras hijas, dejad que se
vayan con sus novios.»

\hypertarget{xi}{%
\chapter{XI}\label{xi}}

No entraba fácilmente en el ánimo del celtíbero la explicación
casuística que de su conducta daba el pobre Binondo. No era mala
filosofía la de casar a las hijas a gusto de ellas antes que se murieran
de desconsuelo de matrimonio; pero este humanitario principio debía cada
cual aplicarlo a su familia, no a las ajenas. Estas y otras objeciones a
las ideas de Binondo se le ocurrían; pero viendo mojado de lágrimas el
rostro chato y verde, se encerró en un buen callar: era impertinente
ponerse a discutir con un moribundo, y turbar su conciencia con
acusaciones y distingos. Quedárase cada cual con su tema, y Dios
juzgaría con suprema equidad. Apagando más su voz, Binondo le dijo:
«Vuelve por aquí cuando estés franco, y te lo explicaré mejor\ldots{} Me
darás la razón, Diego, cuando te cuente el paso\ldots{} y sepas estos y
aquellos pormenores.»

Prometiéndole volver, Ansúrez se despidió muy afectuoso. El Cabo de mar
le retuvo, cogiéndole de la mano para preguntarle dónde estaban y a qué
punto de su derrota había llegado la fragata. «Estamos en la bahía de
\emph{Posesión}---contestó Ansúrez,---ya dentro del Estrecho de
Magallanes, a los 52 grados de latitud Sur\ldots{} Como en este maldito
canal tira la marea lo menos, lo menos, tres millas por hora, hemos de
ir mañana en busca de mejor fondeadero\ldots{} Y a todas estas, no
parece el \emph{Marqués}, que nos trae el carbón; y como no venga,
lucidos estamos\ldots{} El Estrecho es todo angosturas, vueltas,
esquinas y canalizos. Métase usted a la vela en este laberinto, y podrá
decir cuándo entra, pero no cuándo sale\ldots{} ¡Y con barcos de este
calado, válgame la Virgen\ldots! Para desembocar sin tropiezo en el
Pacífico, hemos de zafarnos de este callejón con buenas estrepadas de
hélice.»

En esto llegó a la enfermería el castrense don José Moirón, hombre
excelente, modoso y encogidito. Por su mezquina presencia y delgada voz,
más parecía capellán de monjas que de marineros y oficiales de guerra.
El hombre desempeñaba la cura de almas en la sociedad militar con celo y
modestia, hablando poco y no traspasando jamás el límite de sus
funciones espirituales. A los moribundos asistía con amor; a los
enfermos acompañaba, amenizándoles con su conversación dulce las tristes
horas de encierro en la enfermería de paz. «¿Qué tal, Binondo? Parece
que te animas charlando con tu amigo Ansúrez\ldots{} ¿Y tú, Diego, no
encuentras a José más alentado? Los hombres de mar tenéis siete
vidas\ldots{} Todavía, José, has de ver cómo se te remienda el arca del
pecho\ldots{} Volverás a tu oficio de pasear por las vergas como yo me
paseo en el Perejil de Cádiz\ldots{} Ánimo, hijo\ldots{} No llevo a mal
que lloriquees un poco, porque así se te despeja el corazón de malos
quereres.» Binondo contestó con mugidos blandos a estas cariñosas
palabras. De la cuestión de conciencia nada dijo el Capellán delante de
Ansúrez: hablaron de Geografía y de la feísima pinta del paisaje que
tenían por una y otra banda. «Dichoso tú, Binondo, que no ves el horror
de estas tierras endemoniadas. Vegetación, Dios la dé\ldots{} Y de
animales, ¡qué pobreza! No he visto más que unos pájaros, que no sé si
son nadantes o volantes, que están parados y erguidos mirándonos desde
tierra\ldots{} Su forma es la de botijos con plumas.»

---Esos son los \emph{pingüinos}, que también llaman \emph{pájaros
bobos}---dijo Ansúrez.---Se empinan sobre las patas, y miran como si
pidieran un tiro\ldots{} Pero son mala carne\ldots{} no valen el tiro.

\emph{---Pájaros bobos}\ldots---repitió Binondo con ligero extravío en
su cerebro extenuado.---Como nunca ven gente, no huyen del hombre,
creyendo que es, como ellos, un animal bobo\ldots{} Y el hombre lo es,
porque se pasa la vida haciendo tontadas\ldots{} Sólo tiene listeza y
sabiduría a la hora de la muerte, única hora que no es hora boba.

Sentose el Capellán junto a Binondo, y preguntó a Diego qué noticias
había de los fines del viaje, y cómo estaban los asuntos de España en el
Pacífico. «No sé más que lo que me ha dicho Sacristá---replicó
Ansúrez.---En Montevideo recogió don Casto noticias buenas, no de
oficio, sino particulares\ldots{} Parece que está hecha la paz con el
Perú, y allá vamos a proclamarla con salvas y festejos\ldots» A las
demás preguntas de Moirón no supo contestar el Oficial de mar\ldots{} Si
pasaban con felicidad el Estrecho, llegarían en ocho singladuras a
Valparaíso, donde no podía faltar conocimiento cierto de si iban al
Pacífico en son de guerra, o en son de \emph{pingüinos}, por otro nombre
\emph{pájaros bobos}.

No pudo Ansúrez entretenerse más, y dejó a Binondo con el castrense, que
sin duda le habló de lo buena que es la otra vida, y de la felicidad de
los que van a ella limpios de pecados. La fragata partió de
\emph{Posesión} al día siguiente; pasó con felicidad la angostura de la
\emph{Esperanza}; una fuerte corriente contraria la obligó a detenerse y
buscar abrigo en la ensenada de \emph{San Gregorio}; siguió al otro día,
embocando y recorriendo sin tropiezo la angostura de \emph{San Simón};
penetró luego en el canal más ancho del Estrecho; dobló el \emph{Cabo
Negro}, resguardándose de los bajos y escollos que acechan traicioneros
en aquellas aguas, y por fin dio fondo en el \emph{Puerto del Hambre},
que acredita su fatídico nombre por el aspecto de miseria, desamparo y
aridez lastimosa que allí ofrece la tierra en todo lo que alcanza la
vista.

Ávidos de explorar la misteriosa región magallánica, la Oficialidad
obtuvo permiso para saltar a tierra. En la mayor lancha de la fragata
embarcaron oficiales y guardias marinas, el maquinista Fenelón y ocho
remeros. Ansúrez cogió la caña del timón. No olvidaron las carabinas
\emph{Minié} por si ocurría un feliz encuentro de caza mayor, o por si
era menester defenderse de los bárbaros que habitaban en aquellas frías
latitudes. Dirigiose la lancha a \emph{Punta Santa Ana}, en la costa
Norte de la bahía. Pisaron tierra los expedicionarios, y por aquellos
pedregales discurrieron buscando huellas o rastro de humanidad. No
vieron más que unos pozos de agua dulce, con algún indicio, en sus
bordes, de ser utilizados. A lo lejos se distinguían columnas de humo;
mas no era fácil precisar si salían de algún techo, o de hogueras
encendidas en descampado. El humo subía lentamente hacia un cielo pesado
y gris, que acariciaba con sus masas vaporosas las remotas alturas
blanqueadas por la nieve. Todo el afán de los españoles era ver alguna
muestra de la raza patagona, caracterizada, según los geógrafos de más
crédito, por su estatura gigantea y por la mansedumbre y nobleza de su
barbarie. Pero aunque dispararon al aire sus fusiles con la idea de
llamar y atraer a los indígenas, estos no parecían por parte alguna.
Llegaron a creer nuestros compatriotas que los patagones eran seres
fabulosos, engendrados por la imaginación heroica de los primitivos
navegantes.

Del reino animal no se dejó ver tampoco ninguna muestra, y del vegetal
sólo descubrieron unos matojos verdes de plantitas frescas y talludas,
de la familia de las \emph{umbelíferas}. Por su sabor, eran semejantes
al \emph{apio caballar} de nuestros climas. Corriéndose hacia la
extremidad de \emph{Santa Ana}, reconocieron ruinas que a la primera
impresión diputaron por las de la \emph{Colonia de Sarmiento}. Este
Sarmiento fue un héroe loco, un explorador animoso y exaltado hasta el
delirio, que hizo creer a Felipe II en la conveniencia de establecer, en
medio de todas las desolaciones de la Naturaleza, una colonia
fortificada. La expedición, que al mando de otro loco llamado Flórez
envió el Rey con aquel fin aventurero y fantástico, acabó de la manera
más desastrosa. Flórez y Sarmiento riñeron con escándalo y furia en las
aguas y costas de América, disputándose la precedencia. Flórez se volvió
a España. Sarmiento, más terco que la misma terquedad, se dirigió al
Estrecho con las cinco naves que le quedaban, y aplicó toda su insana
testarudez a la fundación de la plaza colonial. Innumerables hombres,
que eran sin duda los más intrépidos orates de la Nación, perecieron
allí. A muchos se los tragó el mar en las angosturas, o en los esteros
fangosos de la costa Sur; otros murieron en enconada lucha fratricida; a
los que se obstinaron en cimentar la absurda colonia, los aniquiló la
desesperación, y, por fin, el hambre dio cuenta de los últimos\ldots{}

Examinadas las ruinas, entendieron los españoles que no pisaban los
restos de la obra insensata de Sarmiento, sino los de la
\emph{Penitenciaría chilena}, fundada en aquel sitio a principios del
siglo {\textsc{xix}}. Tal vez en los informes vestigios, paredones
corroídos, pilares truncados, había trozos de diferente antigüedad. Eran
ruinas yuxtapuestas, despojos sobre despojos, pavorosa osamenta de dos
arquitecturas muertas y consumidas del sol y el viento. Sobre ellas
rodarían indiferentes las edades. Lo que en la historia humana había
sido completamente inútil, en la Naturaleza servía para que anidaran
cómodamente los \emph{pájaros bobos.}

Desconsolados volvieron a bordo los hombres de la \emph{Numancia}. No
habiendo visto los deseados indígenas, la excursión les parecía
enteramente ociosa. La Patagonia sin patagones era una tierra insulsa y
prosaica\ldots{} En la mañana del día siguiente proyectaron nueva
salida, con idea de emprenderla por un río llamado \emph{San Juan}, que
desemboca al Oeste de la bahía del \emph{Hambre}. Sin duda, internándose
aguas arriba, habían de encontrar a los hombres bárbaros y talludos
dueños de aquellas tierras. En los preparativos de la segunda expedición
estaban, cuando vieron venir por la boca del río una piragua tripulada
por figuras al parecer humanas. La exclamación a bordo fue general.
«¡Hurra, ya están ahí los patagones, hurra!»

Hacia la fragata venía bogando la salvaje embarcación resuelta y
presurosa. Al tenerla cerca, vieron con asombro los de a bordo que eran
mujeres las que remaban, y no con remos, sino con \emph{canaletes},
palitroques rematados en una tabla de forma elíptica. Las hembras daban
impulso a la embarcación con aquellas espátulas, sin punto de apoyo en
la borda, pues la piragua no tenía toletes. En pie venían tres bárbaros
de fea catadura y no muy lucida talla, lo que fue gran desengaño de los
españoles, que esperaban ver colosos formidables y coronados de plumas.
Al llegar los salvajes al costado de la fragata, no expresaron
admiración de la grandeza y hermosura de esta. Con gestos y chillidos
gimiosos, manifestaron su deseo de subir y de comer algo que les dieran.
Sin esperar a que les echaran la escala, los tres hombres se encaramaron
por los tojinos con agilidad cuadrumana. Las dos mujeres remadoras se
quedaron en la piragua, desoyendo las incitaciones de los españoles para
que subieran. O ellas no querían seguir a los machos, o estos no se lo
permitían, que tales etiquetas y reparos habrá sin duda en las
costumbres del salvajismo patagón.

Gran rebullicio y algazara se movió en cubierta cuando pusieron su
planta en ella los tres desgraciados seres en quienes se representaba la
primitiva animalidad de nuestro linaje. Bien se podía decir ante ellos:
«así fuimos.» Eran de mediana estatura y color cobrizo, sucios y sin
gallardía estatuaria. Cubrían parte de su cuerpo con pieles viejas y
astrosas de un animal que llaman \emph{guanaco}. Apestaban a grasa de
pescado; sujetaban sus cabelleras ásperas con una correa de cuero, y
acentuaban la fealdad de sus rostros con rayas negras y coloradas. Su
habla era una mezcla de la modulación y el léxico de las cotorras y de
los ásperos aullidos de los monos mayores. Fácilmente repetían las voces
españolas; pero las de ellos no había boca cristiana que las
reprodujera. Invitados a comer, se les ofreció pan, que miraron con
asombro antes de probarlo. Mayor estupefacción les causó el ver
cucharas, y embobados contemplaron a los marineros que con ellas comían.
Quisieron hacer lo mismo; mas no acertaban a meter la comida en la boca
con aquel adminículo tan extraño para ellos. El vino los entusiasmaba, y
el aguardiente los transportó al cielo de las mayores alegrías. Si no
sabían comer con cuchara, bebían cumplidamente en el vaso, empinándolo
hasta que les caía la última gota. Los chupetazos que daban luego y el
relamerse con sus lenguas sedientas, fueron diversión de los españoles,
que nunca habían visto bárbaros de tan extremada inocencia y
grosería\ldots{} Lleváronlos luego a visitar todo el barco: manifestaban
su asombro riendo como idiotas; pero su regocijo llegó al frenesí cuando
se les invitó a ponerse unos pantalones viejos que allí sacaron. A la
primera lección que se les dio, aprendieron a enfundarse las piernas en
los calzones. El que parecía principal de ellos, ostentando como
insignia de su autoridad mayores chorretazos de rojo en sus mejillas,
fue obsequiado ademas con una levita informe y un sombrero alto, chafado
y roto. Luego que se atavió con estas prendas, lleváronle delante de un
espejo, y al ver la reproducción de su elegante figura quedose
fluctuando entre la risa y un asombro respetuoso.

En tanto que a bordo con estas bufonadas se divertía la gente joven y
alegre, otros habían bajado por los tangones al bote de servicio, y en
este se pusieron al habla o a la mira con las señoras salvajes. Fenelón
era el más empeñado en obsequiarlas, y en honor de ellos escanció todo
el Jerez de una botella. Eran las hembras remadoras más desmedradas que
los hombres, feas y hurañas. Ninguna de las gracias del bello sexo se
revelaba en ellas, y sólo Fenelón, como sacerdote de Venus, extremado en
su culto, entrevió algún encanto en los amarillos rostros de las
amazonas, en sus pechos fláccidos y colgantes, en sus cuerpos
desfigurados por haraposas pieles, que dejando al descubierto el ombligo
y otras regiones poco bellas, tapaban las caderas y demás\ldots{} Bajo
los sucios pellejos asomaban las piernas cobrizas\ldots{} con medias, es
decir, con la canilla y pie pintados de color verdinegro, señal de que
las dos señoras habían chapoteado en el fango del río al lanzar la
piragua. Nadie vio en sus descuidadas greñas adorno alguno que indicase
el menor rudimento de coquetería o de arte del tocador\ldots{} Eran
hembras animales más que mujeres. Trabajillo costaba excitar en ellas la
risa, como prueba de ligereza o agilidad de espíritu. La risa de
aquellas fieras causaba más miedo que alegría, porque ostentaban en toda
su extensión la formidable herramienta dental\ldots{} Por fin, partieron
todos en la piragua, borrachos perdidos los hombres. Uno de ellos,
vestido ridículamente con los guiñapos europeos, esgrimía con grotescos
ademanes un sable viejo y tomado de orín que le regalaron los Oficiales.
¡Infeliz tribu patagona, buena te había caído!

\hypertarget{xii}{%
\chapter{XII}\label{xii}}

¡Albricias, llegó el \emph{Marqués de la Victoria}!\ldots{} Saludada con
gran festejo fue su presencia en \emph{Puerto del Hambre}. ¡Volvían los
compañeros perdidos en el Océano! ¡La fragata tenía ya carbón para
proseguir su viaje!\ldots{} Sin tardanza, fondeado el caballero
sirviente a estribor de la dama, se procedió a meter el combustible en
las carboneras de esta. Todo el domingo, que era Pascua de Resurrección,
se empleó en esta faena, sólo interrumpida en la hora de la Misa y
lectura de Ordenanzas después del Oficio. Don José Moirón despachó la
Misa con prontitud, y el sermón militar de las obligaciones del soldado
fue también muy breve. Todo el tiempo era poco para trasbordar el
combustible. La Oficialidad de ambos buques, no teniendo nada que hacer
a bordo, realizó su expedición al río \emph{San Juan}, sin ver nada de
interés, ni hombres ni animales. Los salvajes no parecían. La Naturaleza
misma se recluía tierra adentro, avara de sus tesoros de fauna y flora,
si algunos tenía. Volvieron los españoles a los barcos con el alma a los
pies, desengañados de toda pasión geográfica y exploratriz, y pasaron el
tiempo de estadía en el \emph{Puerto del Hambre}, desmintiendo este
lúgubre nombre con los buenos víveres que una y otra nave traían. Los
días acortaban ya tristemente, como días vecinos al polo en aquella
estación, que era el otoño austral\ldots{} A las cuatro de la tarde se
iniciaba el crepúsculo, anunciando ya las prolongadas noches invernales.
Espesa penumbra caía sobre la tierra; el cielo tomaba un tono plomizo:
cielo y tierra se vestían de un luto angustioso que avivaba en los
corazones el amor y el recuerdo de la patria lejana, radicante en la más
risueña porción del globo.

Partió la fragata el 19 de Abril, despidiéndose con fraternal emoción de
su caballero sirviente, que a Montevideo se volvía. La nave acorazada
emprendió sola su marcha por aquellos canalizos y desfiladeros, lo que
fue temeridad grande; mas para tal empeño bastaba el esforzado corazón
del soldado de mar que la mandaba. Día claro y sereno favoreció el paso
de la \emph{Numancia} frente al morro de \emph{Santa Águeda}, donde el
paisaje tomó las formas más imponentes y majestuosas. En aquel punto
humillan los Andes sus moles ante la mordedura del mar, que las socava y
desmorona. Por estribor veían los españoles, a lo lejos, el grandioso
espectáculo de las cimas nevadas; de cerca, los cantiles abruptos, las
masas rocosas cortadas como a pico, hurañas y resecas, con vagos toques
de vegetación en algunas encañadas; por babor veían la \emph{Tierra del
Fuego}, merecedora de tal nombre si se le añadiera el calificativo de
\emph{apagado}. Era como un volcán, como un avispero de cráteres fríos,
vestigio y estampa de los más terribles cataclismos geológicos. La vista
de aquellas extrañas formaciones causaba espanto, sugiriendo la idea de
un planeta muerto, perdido en los espacios siderales\ldots{} Para que
pudiera participar de la admiración general, sacaron de su camarote al
Segundo, don Juan Bautista Antequera, obligado a quietud por la
soldadura de la pierna, y muy bien acomodado en una colchoneta, le
subieron al Alcázar. Allí estuvo largo rato, y sus ojos, desperezándose
de la obscuridad del encierro, no se hartaban de ver tanta maravilla.

Hizo alto la fragata en el fondeadero de \emph{Fortescue}. Tras ella
venía una corbeta de vapor, que resultó ser peruana, de guerra.
\emph{América} era su nombre, y había sido construida en Francia. Fue
mirada con recelo; se pensó en disparar sobre ella; pero al fin nada
sucedió. La corbeta dejó caer su ancla por estribor de la
\emph{Numancia}. Esta levó muy temprano, al día siguiente, y atravesó la
más estrecha angostura de todo el paso de Magallanes. Viéronse aquel día
más próximos los elevados montes patagónicos, coronados de nieve, y los
hilos de agua que al derretirse la nieve venían saltando por
innumerables cañadas y repliegues, juntándose luego para formar
risueñas, espumosas cascadas. El paso llamado \emph{Crooked-Reach} es
tan angosto, que los navegantes creían tener al alcance de su mano los
dos cantiles de izquierda y derecha. La fragata marchaba con cuatro
calderas, gallarda como nunca, orgullosa de sí misma, mirándose en las
claras aguas, mirando también su sombra en las rocas del Norte. Dijérase
que todos los ánades y pingüinos de la región se habían dado cita en
aquel paso, porque precedían a la nave como señalándole el camino, y
luego levantaban el vuelo al ver de cerca el espolón y oír el golpetazo
de la hélice batiendo el agua. También aparecieron cetáceos monstruosos,
nadando delante y a los costados de la embarcación, y festejando a esta
con el surtidor que sus furiosos resoplidos lanzaban al aire. La fragata
no parecía insensible a estas demostraciones de la fauna marítima, y
surcaba las ondas con mayor prepotencia y majestad. Era la diosa
Anfitrite, esposa de Neptuno, que paseaba por su reino precedida y
escoltada por la corte de sirenas, tritones y bestias marinas.

Al décimo día de entrar en el Estrecho, salió de él la \emph{Numancia}.
A las cinco de la tarde del 21, con mar sosegada y atmósfera densa que
ofuscaba los términos lejanos, la fragata señaló a babor el \emph{Cabo
Pilares}. Era el extremo occidental del paso y la última tierra del Sur
magallánico, la más desolada que podría imaginarse; tierra que parecía
obra de maldiciones y engendro de pesadilla. Las conglomeraciones
basálticas, de soñadas formas nunca vistas, hacían creer que aquel
extremo del mundo era el osario en que los siglos, terminada la monda
total del planeta, habían arrojado todos los esqueletos de animales
paleontológicos.

Franqueado \emph{Pilares}, entraron los españoles en mar libre y ancho.
Fue para todos descanso y orgullo. Por un canal de más de cien leguas,
erizado de peligros, habían conducido la mayor nave que hasta entonces
se aventuró a pasar por allí. Bien podían envanecerse, aunque el caso no
era milagroso, sino una feliz aplicación de la sintética proclama de
Nelson. Todos, desde el Comandante al último marinero, habían cumplido
su deber\ldots{} Y adelante, adelante, en busca de la ocasión de nuevos
deberes que cumplir\ldots{} Sin contratiempo navegó a hélice la fragata,
con rumbo Norte, hasta los 40 grados de latitud, en que hallando mejor
mar y los vientos generales del Sur, apagó calderas y largó todo su
aparejo. Nunca estuvo Anfitrite tan bella como cuando surcaba las aguas
del Pacífico, con todo el flameante adorno de su ropaje aéreo. Sus
airosas cabezadas expresaban el contento suyo y de todos los
tripulantes, que con ella se identificaban y ponían los latidos de su
corazón al compás de los pasos de ella en el ancho mar. La normalidad
placentera de la navegación no se interrumpió en aquella etapa: todos
vivían alegres, contemplando de día, por estribor, el gigantesco
murallón de los Andes, y aun los menos instruidos sabían leer en
aquellas moles alguna estrofa de la leyenda hispánica.

Visibles fueron los efectos del gradual ascenso de la temperatura: los
pocos enfermos que a bordo había se restablecieron, y el mismo Binondo,
que en el Estrecho estuvo a punto de liar el petate, mejoró
notablemente, como si quisiera entrar en la séptima vida que, según el
dicho popular, gozan los marinos. Aún no podía el hombre valerse; pero
respiraba mejor, señal de que se le iban calafateando los deteriorados
bofes, y todos los días, a la hora de más calor, le sacaban a cubierta
en una silleta, y allí le dejaban parloteando con sus compañeros. En
estos solaces de convaleciente habló de asuntos diversos con su amigo
Ansúrez; pero de aquellos coloquios sólo se cuentan aquí los pertinentes
al caso de Mara.

«Poco a poco, Diego---decía Binondo extendiendo el brazo:---no eches
sobre mí más culpa de la que tuve en el latrocinio de tu hija\ldots{}
que bien mirado, no fue tal latrocinio, sino cumplimiento de la ley de
Dios, que dice: `antes que dejar morir a vuestras hijas, dejad que se
vayan con sus novios\ldots{}'. Esto ha dicho Dios, y a mis oídos llegó
la voz divina, por la cual fui movido a dar aquel paso\ldots{} Que venga
don José Moirón, que venga el santísimo Capellán, y él te dirá si este
mandamiento que te digo no es tan de ley como otro cualquiera\ldots»

---Bueno; ley de Dios será\ldots{} Pero no tenemos por qué llamar al
Cura; que esta ropa sucia, José, en casa debemos lavarla\ldots{} ¡Tú a
encandilarme con tus leyes de Dios, ¡ajo!, y yo a no dejarme
encandilar!\ldots{} Mi sentido natural me dice que no es ley de Dios,
sino del diablo, tomar dinerales por favorecer la fuga de la niña con
aquel bandido.

---Poco a poco, Diego, poco a poco. No te niego la verdad. Pero has de
saber que no fueron dinerales lo que tomé, sino una triste onza de
oro\ldots{}

---¿Triste la llamas? ¿Pues no valía diez y seis duros?

---Los valía, sí\ldots{} Llamo triste a la onza, porque fue poco
estipendio para lo que hice. Toda la noche estuve en vela, fingiendo que
pescaba\ldots{} Los carabineros me habían echado el ojo encima; yo no
hacía más que bogar hacia afuera, y volver y escurrirme a la sombra de
la batería de San Leandro. Pues tomé la onza\ldots{} no quiero dejar de
decirte toda la verdad\ldots{} la tomé porque me hacía mucha
falta\ldots{} como que aún estaba debiendo las visitas del médico y el
entierro de mi niña\ldots{} ¡Ay, Diego de mi alma, no puedo nombrar a mi
ángel sin que me salte el corazón y se me corte el resuello! ¿Ves? Ya
estoy llorando\ldots{} No hay consuelo para mí\ldots{} Y ahora, con esto
de que voy escapando de la muerte, mi pena es mayor, porque yo estaba
muy satisfecho de morirme, por el gusto de ver pronto a mi niña, la mona
de Dios\ldots{} y recrearme en aquel rostro de clavellina parda, y en el
habla bonita, y en el cuerpo salado, tan salado y gracioso, que me río
yo de los ángeles\ldots{}

---No llores, José\ldots{} Como algún día has de morirte, y verás a tu
Rosa entre los serafines, resígnate por hoy a seguir viviendo\ldots{}
¡Ajo!, no eches más babas ni mojes el pañuelo. Cuéntame cómo embarcó la
niña en tu lancha; qué dijo\ldots{}

---Antes tengo que repetir que la onza no fue más que una corta ofrenda
para mi alcancía. La tomé por no desairar. Verdad que después, a bordo
de la goleta, me dio don Belisario diez duros más\ldots{} ¿Pero qué son
diez duros para un servicio tan arriesgado\ldots? Y el peligro fue
tremendo\ldots{} Los carabineros no me quitaban el ojo\ldots{} Tu niña
llegó a las piedras de Santa Lucía, único sitio donde podía embarcar de
noche, acompañada de don Belisario y de la Venancia\ldots{} ya sabes, la
Venancia. Esa sí cogió buen dinero\ldots{} De aquí estoy viendo la
pastelería que ha puesto esa ladrona en la calle de la Caridad\ldots{}
la veo, la veo, Diego\ldots{} Y cuidado que hay distancia del Pacífico,
33 grados latitud Sur, a la pastelería de Venancia, 38 grados latitud
Norte. Pues la veo: cree que la veo.

---Avante en popa, y no barloventees más.

---En brazos cogió a la niña el caballero, y de sus brazos pasó a los
míos, que la pusieron en la lancha\ldots{} De un brinco embarcó don
Belisario. Despidiéronse de Venancia. Trinqué yo los remos, y me puse a
bogar en silencio, arrimado a tierra, buscando la sombra del
monte\ldots{} Te diré, buen amigo, para tu satisfacción, que no había
dado yo tres paladas de remo, cuando la niña rompió a llorar con tanto
sentimiento, que me río yo de la Magdalena. El caballero quería
consolarla: ya sacaba estas razones, ya las otras\ldots{} que Dios, que
el amor, que la felicidad\ldots{} y luego unas retóricas ahiladas que no
entendí, pues tales términos, comparanzas y sutilezas no había oído yo
en mi vida. Mara, tan aferrada a su aflicción que no se quitaba de los
ojos el pañuelo, decía: «Mi padre, ¡ay!, mi padre.» Y él le echaba el
brazo por los hombros, y apretándola con agasajo, respondía: «Tu padre
será mi padre; pero él no quiere serlo\ldots{} Pagará su
soberbia\ldots{} Después le perdonaremos.» En fin, Diego, no puedo
repetirte su hablar finísimo, porque usaba expresiones que mi boca no
sabe pronunciar. La sustancia de aquel relato era esta, verbigracia: «El
amor, que viene a ser el rey, emperador o no sé qué de todito el mundo
terrestre y universal, te condenaba por bruto y descastado a\ldots{}
Diantre, a ver si me acuerdo\ldots{} Pues te condenaba a la pena de
perder a tu hija por tal o cual tiempo\ldots» En fin, Diego, que te
daban el trago de amargura para traerte luego los dulzores de volver a
Cartagena casados y con guita\ldots{} ¿Te vas enterando?\ldots{} Yo no
puedo referírtelo palabra por palabra. Sí te digo que a la niña se le
aplacó el duelo con los abrazos que le daba el novio, echándole en la
misma oreja este bálsamo: «Te juro por mi madre que volveremos\ldots{}
volveremos en tal condición, que tu padre se alegre de recibirnos.»
Luego miraba para las estrellas, y moviendo el brazo con ellas
hablaba\ldots{} A la mar le echó también una gran bocanada de términos
que sonaban muy bien, como una musiquilla de cantares\ldots{} En esto,
llegamos a la goleta: subieron ellos, yo detrás.

»Poco tiempo estuvo la niña sobre cubierta, porque don Belisario y el
capitán la llevaron a un camarote, y en él la escondieron, por lo que
pudiera tronar. Yo esperé al caballero, porque así me lo mandó. Al
cuarto de hora le vi aparecer en cubierta; llevome a la borda, desde
donde se veía Cartagena, más que por sus casas, torres y murallas, por
las luces del alumbrado público; y señalando a la ciudad, dijo: «¡Ahí te
quedas, Cartagena! ¡Ahí te quedas, Ansúrez!\ldots{} Entré con paz, y me
mirasteis como enemigo\ldots{} Al padre me llegué con el corazón en la
mano, y el padre me echó la zarpa para ahogarme. No hay paces con los
bárbaros. Mara no es de su padre, sino mía. Él le dio la vida corporal,
yo le doy la vida del espíritu\ldots» No puedo explicártelo, porque las
palabras que dijo en aquellas proclamaciones son de esas que no se
quedan en la memoria. Lo que sí recuerdo bien es esta frase: «Pues
quisiste guerra, guerra te doy, brutal Ansúrez. Ya puedes echarte a
llorar hasta que volvamos\ldots» Luego que dijo lo que te cuento, nos
despedimos. Me pidió juramento de no contarle a nadie lo que había
pasado, y yo se lo di\ldots{} digo que juré, haciendo la cruz, porque
así me lo mandaba mi conciencia. Y él también tuvo conciencia, porque al
despedirme metió mano al bolsillo y diome los diez duros, que\ldots{}
ahora recuerdo\ldots{} no fueron diez, sino veinte, o hablando con toda
verdad, veinticinco, plata y dos monedas de oro, isabelinas\ldots{} Ya
ves que no te oculto nada. Cuando yo a tierra me volvía poco a poco,
pensaba en mi pobre niña difunta, ¡ay!, en aquel ángel. ¡Que no hubiera
yo podido hacer con ella lo que hice con la tuya, Diego!\ldots{} Dársela
al novio, echarla en brazos del novio para que gozaran de su
juventud\ldots{} Para mí no había consuelo\ldots{} yo bogaba con pereza,
y mis pensamientos iban al compás de los remos. En el cielo como en el
agua oía la voz del Divino Jesús, diciéndome: «Que no mueran las hijas;
que se vayan, que se vayan con sus novios.»

\hypertarget{xiii}{%
\chapter{XIII}\label{xiii}}

El interesante episodio referido por Binondo inmergió al Oficial de mar
en mayores cavilaciones y tristezas. Sus sentimientos, agitados por
pavorosa crisis, no sabían si estacionarse en el amor o en el odio. Sólo
sus obligaciones rudas le distraían en estos internos afanes. El 28 por
la mañana recaló la fragata en Valparaíso, y aproximándose al puerto,
paró y se puso al habla con el Comandante de la goleta \emph{Vencedora}.
¡Qué placer y qué descanso recibir noticias frescas, fidedignas! Los de
la \emph{Numancia} oyeron confirmar la buena nueva de que nuestro
Gobierno había concertado un arreglo con el Perú. La escuadra, al mando
del Almirante Pareja, estaba en el Callao. Hacia el Callao hizo rumbo la
\emph{Numancia} sin perder horas, navegando con cuatro calderas
encendidas y ayuda del velamen. Serena mar y viento Norte fresquecito
facilitaron aquella etapa, por todos estilos venturosa. En temperatura
iban ganando de día en día; la salud era excelente a bordo, y todos
vivían en espera de sucesos pacíficos más que guerreros, aunque no
faltaba quien se apenase de que no sobreviniesen hostilidades duras, que
en la profesión militar nada repugna tanto a los corazones enteros como
la ociosidad.

Aunque se reponía bajo la acción de la subida temperatura, Binondo no
recobraba por entero su vigor y aptitud para el trabajo. O era que se
hacía el remolón para que le dieran mimo y le llenaran la pandorga,
dejándole las horas muertas sentadito al sol, o a la sombra cuando el
sol picaba más de la cuenta. En este periodo avanzado de su
convalecencia, se hizo el hombre muy rezador: andaba siempre con el
rosario entre las manos, y en sus pláticas con los compañeros, a estos
recomendaba que tuviesen el alma preparada para un buen morir, pues en
las dudas de paz o guerra, nadie podía decir «a tal hora viviré.»

«Aquí donde me ves, Diego querido, no estoy menos libre de la muerte que
lo estaba en el Estrecho, porque las cuadernas del pecho no acaban de
arreglarse, y el corazón me dice a cada momento que no cuente con él
para una larga travesía. Pero yo no me apuro, Diego, y tan hecho estoy a
la idea de morirme, que me digo: `Cuanto antes mejor, que de este mundo
perverso no saca uno más que sofocos y berrinches que pudren el alma.
Muérame yo pronto, que eso voy ganando, y así veré a mi querida Rosa en
la Eternidad'. Paréceme que ya la estoy viendo\ldots{} Cuando tengo esta
visión, el aliento se me corta, como si la máquina del respirar quisiera
pararse y decir: `hasta aquí llegué'»

---¡Valiente marrullero estás tú!\ldots{} Con tantos rezuqueos y
visiones lo que busca mi amigo es que no le den de alta, para seguir en
esta gandulería y pasarse el tiempo sentadito en cubierta.

---Poco a poco: yo no trabajo porque no puedo. Ya sabes que como bien,
porque así me lo mandó don Luis. Por mi gusto no comería más que lo
preciso para no desfallecer. Duermo toda la noche y parte del día,
porque así me lo recomiendan los doctores, que el sueño es el estero
donde el corazón se va carenando\ldots{} como te lo digo\ldots{} Pero el
dormir mío no es todo lo sosegado que fuera menester, porque el soñar me
quebranta, y despierto tan molido como si me hubieran pasado de verdad
las cosas que sueño\ldots{} Es el corazón enfermo\ldots{} que
adivina\ldots{} Y a cuento de esto, sabrás que anoche he soñado contigo
y con tu hija\ldots{} Y era lo que soñé tan conforme con la razón, que
desperté creyéndolo cierto. Vas a oírlo\ldots{} Pues soñé que entrábamos
en puerto\ldots{} ¿Sabes tú cuándo llegaremos al Callao?

---Mañana. Esta tarde hemos de señalar las islas Chinchas.

---Dime otra cosa: ¿hay mucha distancia del Callao a Lima?

---Media hora o poco más en ferrocarril.

---Pues no te canses en ir a Lima, porque si vas no encontrarás a tu
hija. Yo he soñado que Mara y don Belisario navegan hacia Panamá,
caminito de Europa. Van casados por la Iglesia y cargados de dinero
hasta las escotillas\ldots{} Llevan la idea de que los perdones Diego, y
les eches tu bendición\ldots{} Pero Dios, que ve tus muchos pecados,
dispone que ni ellos ni tú tengáis la satisfacción de veros y
perdonaros. También ellos son pecadores\ldots{} Dios castiga sin palo ni
piedra, y así, mientras tus hijos van, tú vienes\ldots{} Equivocados
navegáis todos\ldots{} Dios, que gobierna con una mano los corazones y
con otra los mares, te trae al Perú cuando tu hija no está aquí, y a
ella la manda para España cuando tú andas por acá\ldots{} ¿No ves bien
claro los designios del patrón de todo el Universo?

Al oír esto, trabajo le costó a Diego reprimirse. Impulsos tuvo de coger
a Binondo por el cogote y darle un fuerte achuchón contra el cabrestante
próximo, chafándole el rostro hasta dejárselo enteramente raso.
«Tunante---le dijo,---guárdate tus sueños malditos, y no atormentes al
hombre honrado y bueno, que no hace mal a nadie.»

---Bueno eres---replicó Binondo con extremada mansedumbre, acariciando
las cuentas de su rosario;---pero ya sabes que el justo peca siete veces
al día. Que Dios quiera probarte, que Dios pruebe a los justos para ver
su temple y fortaleza, es cosa corriente en nuestra religión, y si lo
dudas, llama a nuestro Capellán y pregúntaselo.

Ansúrez le volvió la espalda. En actitud de oración se mantuvo José, la
cara plana y verde caída sobre el pecho con expresión de recogimiento
budista. El otro, echando sus miradas y sus pensamientos sobre el mar,
también quedó en éxtasis de amarguísimas dudas, del cual le sacó el pito
de Sacristá llamando a maniobra. Poco después se marcaron a barlovento
las islas Chinchas. El terral fresquecito trajo al olfato de los marinos
efluvios amoniacales\ldots{} \emph{Tan-tan} cuatro veces. Se cambiaron
las guardias\ldots{} Y al día siguiente, cuando sólo distaban cinco o
seis millas del puerto del Callao, volvió Binondo a dar tormento a su
amigo con el relato de sus estupendas soñaciones.

«Óyeme, Diego, y pásmate de que Dios se digne revelarme lo que ha de
pasarnos a ti y a mí. Tú y yo somos buenos, y para que seamos mejores
nos manda Dios tribulaciones grandes. He soñado, amigo, he soñado lo que
voy a decirte para que te vacíes de orgullo y te llenes de
resignación\ldots{} Pues ello es que\ldots{} No pongas cara fosca ni me
hagas temblar con tus miradas\ldots{} Yo digo lo que soñé, y tú lo crees
o no lo crees\ldots{} Ello es que ya no verás a tu hija en la tierra,
sino en el Cielo\ldots{} Estamos iguales, amigo del alma, y hemos de
morirnos para ver a las prendas de nuestro corazón\ldots{} Para mí es
esto tan cierto y verídico como el mar es mar, el cielo, cielo, y esta
embarcación la \emph{Numancia} bendita\ldots{} que Dios favorezca para
que viva más que nosotros. Dúdalo si quieres; pero la realidad se
encargará de convencerte\ldots{} A tu hija verás en el Cielo; antes de
ir allá, si vas, no podrás verla\ldots{} Créelo, Ansúrez, y disponte
pronto, pronto para un morir cristiano\ldots{} Debemos prepararnos,
porque nunca sabemos si hemos de vivir estos momentos o los otros. Podrá
ser hoy, podrá ser mañana o en mañanas que aún están lejos. Pero que no
nos coja desprevenidos\ldots{} ¡Qué gozo el tuyo y el mío cuando las
veamos en la Gloria!\ldots{} mi Rosa tan linda, con aquella carita de
marfil ahumado y aquellos ojuelos negros, como los de los ángeles que
encienden los relámpagos y disparan los truenos en una noche de
tempestad\ldots{} tu Mara desmejoradilla y muy rebajada de su
belleza\ldots{} porque has de saber que muere o morirá de parto\ldots»

Ya no pudo tener Ansúrez el arrebato de su displicencia, y le dio un
cosque más que regular, que humilló la cabeza budista y puso la cara
plana a dos dedos de la borda, junto a la cual se hallaban. «Poco a
poco---exclamó Binondo.---Esos no son saludos de los que se acostumbran
entre amigos. Bárbaro estás, rebelde contra las verdades que Dios te
anuncia por mi boca. De tus desdichas no tengo yo la culpa\ldots{} ni de
que Dios ame a nuestras dos hijas por igual, y se las lleve de este
mundo nuestro tan malo, al suyo, que es la Gloria\ldots»

No llegaron estas últimas razones al oído del Oficial de mar, que se
alejó rezongando amenazas contra Binondo. La idea de la muerte de Mara,
sugerida por el zorro malayo, le desconcertaba. A creerla se resistía;
pero la idea penetraba en su entendimiento, como la carcoma royendo y
labrándose su casa\ldots{} Aliviábase el buen hombre de esta confusión
con la esperanza de que el sol de Lima despejara pronto sus dudas.

La entrada en el puerto del Callao fue de teatral efecto resonante. Allí
estaba la escuadra española mandada por Pareja: la componían las
fragatas de hélice \emph{Villa de Madrid}, \emph{Blanca},
\emph{Berenguela} y \emph{Resolución} y la goleta \emph{Covadonga}. El
primer saludo fue para la insignia de Pareja; después se saludó a la
plaza, que contestó al instante; y apenas disipado el humo de estas
salvas, se cañoneó en honor de las escuadras extranjeras allí fondeadas,
inglesa, francesa y americana. Devolvían todos la cortesía con igual
número de estampidos, y aquello fue como una batalla naval con pólvora
sola, espectáculo precioso, inmenso vocerío de guerreros en paz.

Presentaba el puerto en aquellos instantes un golpe de vista espléndido.
Deleitaban los ojos la flotante población de barcos de guerra y paz, y
el bosque de sus mástiles, así como los mezclados colorines de tantas
banderas de diferentes Estados. Entre los buques mercantes, había los
más hermosos tipos de vela entonces existentes en el mundo: fragatonas y
corbetas \emph{clipper}, de cascos elegantes y gallardísimas
arboladuras. Todas estas naves esperaban vez para el embarque de guano
en las Chinchas. Si es maravilla de la Naturaleza el almacenaje secular
del excremento de las aves atlánticas en aquellas ínsulas, no lo es
menos el ingenio y artes del hombre para transportarlo por tan largos
caminos de mar de un hemisferio a otro\ldots{} El labrador piamontés o
valenciano no acababa de comprender que abonaran sus tierras las aves
del Pacífico.

Terminados los saludos, empezaron las visitas. No era sólo el jubileo de
amigos y parientes entre unos y otros barcos: era la curiosidad que en
todas las tripulaciones de las fragatas de madera despertaba la
\emph{Numancia}, potente y airosa; era el prodigio de haber esta
navegado sin tropiezo desde Cádiz al Perú, desmintiendo la opinión de
que un guerrero vestido de armadura no podía sin peligro arrostrar
caminata tan penosa y larga. Pero el Comandante, hombre de arrestos
indomables, la Oficialidad y marinería, orgullosos de su feliz empresa,
decían como Segismundo: «¡Vive Dios que pudo ser!»

Tal invasión de visitas hubo en la fragata, que las escalas crujían del
peso de los curiosos entrantes y salientes. Superiores y oficialidad,
guardias marinas, marineros, en fin, y gente de maestranza, acudieron a
saciar sus ojos, a explayar sus corazones en parabienes, que eran la
expresión de la amistad y el orgullo, fundido todo en un tono general de
patriotismo. La \emph{Numancia} vio subir a su cubierta y penetrar en
sus cámaras y sollados al Almirante Pareja, hombre de mediana estatura,
delgado, con patillas blancas, de continente grave y maneras muy
corteses; a don Miguel Lobo, Mayor General, gran náutico y geógrafo,
hombre de ciencia y de voluntad; a don Claudio Alvargonzález, curtido y
fosco, de barba erizada y ojos fulgurantes, el primer lobo de mar de
España; a don Juan Topete, corazón fuerte, ávido de pelea y gloria; a
don Manuel de la Pezuela, ducho en artes políticas y en el trato de
gentes, que aplicar supo al arte de la guerra; a don Carlos Valcárcel,
marino excelente y guerrero de tesón, y a otros muchos que ganaron
después celebridad. La fragata les recibió con alegría, mostrándoles
todas sus bellezas, así las exteriores como las más ocultas. Convites
parciales y refrescos se improvisaron en los camarotes, y en tanto los
grupos de marineros celebraban con modestas libaciones el feliz
encuentro de amigos y hermanos, en latitud tan distantes del solar
paterno.

Fue por la mañana cuando Ansúrez distinguió entre los visitantes una
cara conocida. «¿Será\ldots? Si no fuera tan gordo, diría que es
Mendaro.» A estas dudas fugaces siguió la exclamación de ambos amigos,
que se abrazaron con júbilo después de una ausencia de cinco años. «Por
el ranchero Ibarrola---dijo Mendaro---supe que estabas aquí. He venido a
verte a ti primero, después a esta hermosa fragata que os traéis
acá\ldots{} ¿Sabes que estás viejo?\ldots{} ¿Qué ha sido de tu vida?
Cuéntame.» Con frase concisa notificó Ansúrez a su amigo la muerte de
Esperanza, y de la pérdida de Mara hizo una indicación vacilante, como
los apuntes con que los pintores esbozan el intento de una figura. A
continuación enseñó al forastero el interior del barco; le obsequió con
Jerez y galletas, y despidiéronse con mutuos ofrecimientos y cariños.

Mendaro y Ansúrez, después de navegar juntos, habían vivido en Cartagena
pared por medio. Su amistad era sólida, íntima, como fundada en las
excelentes cualidades de uno y otro. Enviudó Mendaro el 59 y se embarcó
para el Perú, donde contrajo segundas nupcias. El 65 era poseedor de una
de las más frecuentadas pulperías del Callao de Lima, establecimiento
que bien podía llamarse famoso, porque en él encontraban alivio de su
sed y reparo de su hambre los marineros de diferentes banderas,
cargadores y truchimanes, y allí solían congregarse también mujeres que
al socorro de necesidades no espirituales acudían, buena gente toda,
fermento y espuma de la humanidad afanosa que hierve en los puertos de
mar. En la pulpería quedó citado Ansúrez para comer con su amigo, y
charlar de los reinos de España y de las indianas repúblicas.

\hypertarget{xiv}{%
\chapter{XIV}\label{xiv}}

Ansiosos de admirar la ciudad de Lima, que en todas las imaginaciones
españolas se representaba con formas y colores de un seductor
romanticismo, iban a tierra oficiales y guardias marinas en correctísima
y elegante apostura, con pantalón blanco, indumentaria impuesta por los
12 grados de latitud Sur. Del muelle corrían en grupos alegres a la
estación, y media hora después divagaban por las calles y plazas de
Lima. Esparciendo con avidez sus ojos de una parte a otra, aplicaban su
observación a cosas y personas, juzgándolo todo con juvenil calor, así
en el elogio como en la censura. Tras las abstinencias y soledades de la
navegación, anhelaban la vida social, el trato y compañía de señoras
discretas, finas y hermosas, de mujeres, en fin, sin reparar en su clase
y condición. Por desgracia, encontraban retraída la sociedad. Las clases
opulentas, así como las mediocres, se recluían en sus casas por estímulo
de la gazmoñería política, no menos adusta que la religiosa. La
cordialidad y el agasajo entre naturales y forasteros no existían en
aquellos días de incertidumbre y desconfianza; días turbados, además,
por interna enfermedad revolucionaria.

Los Oficiales españoles recorrían con actividad un poco melancólica la
Ciudad de los Reyes. La sombra de Pizarro les acompañaba; las
remembranzas de la patria salían a recibirles en las fachadas de los
edificios de la época vice-real. A cada instante surgía la
\emph{Anagnórisis}, o sea el descubrimiento y declaración de parentesco.
\emph{Anagnórisis} era el gozo con que los españoles contemplaban el
barroquismo amable, risueño, consanguíneo, de la Catedral fundada por el
conquistador. Nuestro, \emph{de casa}, de familia, era el rostro de
aquel monumento; nuestra también el alma, el interior, impregnado de
dulce misterio y de místico encanto. Igual impresión de parentesco les
daba el palacio de los Virreyes, hogaño presidencial.

De calle en calle, se fijaban en los balcones a la turquesca, en las
rejas y celosías, por cuyos huequecitos veían o creían ver los negros
ojos de las limeñas. ¡Qué ilusión! ¿Pero estaban en la América del Sur,
o en Ronda, Tarifa o Algeciras? La mujer limeña, sutilizada por la
imaginación, era el tormento de aquellas pobres almas españolas,
condenadas por un melindre internacional al desconsuelo de Tántalo.
Cerrado el teatro, suspendidas las reuniones y tertulias, no se
mostraban las limeñas más que en la calle, y para mayor desventura no
eran entonces muy callejeras. Por lo poco que vieron los Oficiales al
paso y de refilón, reconocían y declaraban que era la hija de Lima
traslado fiel de la mujer de acá, más bien refinada que desmerecida en
sus cualidades. Por aquellos días no podían extenderse a más detalladas
apreciaciones del tipo físico y moral de tan seductoras hembras. El
famoso manto negro a estilo de Tarifa ya poco se usaba. Sólo por las
mañanas, cuando iban a misa, se las veía entapujadas con exquisita
gracia y travesura, sin dejar ver más que los ojos: el misterio, el
juego de tapa y destapa, los hacía más ardientes y luminosos, más
afilados de malicia o recargados de amoroso fluido. Por junto al suelo
se veían los pies chiquitos, y se apreciaba el andar ligero\ldots{}
andar de gacelas cuando van al paso.

Y vistas estas preciosidades, que parecían huir de las miradas del
hombre antes que solicitarlas, iban los españoles a las partes
excéntricas de la ciudad, donde percibían el rumor popular, nada
benévolo ciertamente. Esquivando el trato con personas, hablaban con los
edificios: vieron y examinaron exteriores ampulosos de parroquias y
conventos, y a cada paso descubrían rastros del pasado, que confirmaban
el parentesco entre los observadores y las cosa observadas. Clarísimo
resultaba el rastro de la superabundancia frailuna, y el paso de la
Inquisición había dejado huellas indelebles. La fiereza española, todo
lo grande de la raza y todo lo violento y vicioso adherido a lo grande,
permanecían escritos allí en cosas y personas, con más vivos caracteres
que los que aún conserva en su propio rostro la madre común.

\emph{Pulpería de Mendaro}.---Este y su amigo Ansúrez, sentados a los
dos lados de una mesa sin manteles, en un patinillo interior de la casa,
platican de los reinos de España y de los achaques de aquellas
repúblicas, sus hijas.

«Todo este torbellino---decía Mendaro---ha venido, ¿sabes de qué? Pues
de añejos piques y desavenencias entre peruanos y españoles; del pleito
viejo por si reconocemos o no reconocemos la independencia del
Perú\ldots{} del mal trato que aquí dieron a unos catalanes y
valencianos\ldots{} de bofetadas, palos y mojicones que han llovido en
la tierra donde no llueve agua\ldots{} de que España se metió en Santo
Domingo y quiso meterse en Méjico\ldots{} de una gravísima trapatiesta
que hubo en Talambo, peruanos ofendidos, españoles muertos\ldots{} de
que en Chile atropellaron a unos vizcaínos\ldots{} de las muchísimas
desvergüenzas que escriben aquí los periódicos, y, en fin, de que los
Gobiernos de una banda y otra están dejados de la mano de Dios\ldots{}
Allá se les subió a la cabeza el humo de la guerra de África, y acá
tienen los humos de su republicanismo y el no ser menos que la vecina de
abajo, Chile, y que las vecinas de arriba, Ecuador y Colombia.»

---Bien se ve que hay humos. En España se dice que este furor de camorra
nos lo ha pegado la Francia, nuestra vecina por el Pirineo, pues el
imperio segundo que hay allí, obra de ese Luis Napoleón, nos da la moda
de encender guerras con tal o cual país. La miaja de gloria que va
sacando el ejército de mar y tierra, es el torniquete, como quien dice,
con que los mandones trincan y aseguran a los que obedecen.

---Moda es que os viene de Francia. Aquí tenemos otra que recibimos de
los Estados Unidos, y es el cansado estribillo de \emph{América para los
americanos}, que quita el seso a toda la gente de acá. Es moda, manía,
aire natural de estos países, que se mete en el corazón y en la cabeza
de cuantos aquí vivimos. Y así verás que los españoles, a los pocos años
de llegar a estos climas, nos volvemos americanos, y tomamos a este
terruño un amor tan grande como si en él hubiéramos nacido. Nada te
quiero decir de los niños que de padre español nacen aquí, pues yo tengo
uno de tres años, que apenas empezó a soltar la lengua, lo primero que
aprendió fue llamarme \emph{gachupín}, \emph{gallego}, \emph{patón},
\emph{godo} y otras perrerías con que los naturales nos motejan\ldots{}
Pues volviendo al por qué de esta campaña, te diré que el Gobierno de la
Isabel no supo lo que hacía cuando nos mandó a ese Almirante Pinzón con
la \emph{Resolución}, la \emph{Triunfo} y la \emph{Covadonga}. No es que
yo le quite su mérito y circunstancias a ese buen General de Marina que
nos mandasteis; pero\ldots{} hablemos claro. ¡Por los pelos del diablo,
que no era Pinzón hombre para estas incumbencias delicadas, porque tenía
demasiado vapor en sus calderas, y no templaba, sino que metía más
coraje en las almas peruanas! A cada brindis que echaba en las
comilonas, ceceando como buen majo andaluz, se armaba una gran
tremolina. Cosas decía con la idea de meter miedo, para que temblaran
todas estas Américas, como si aún se sintieran en el suelo, a la vera de
los Andes, las patadas de aquel bárbaro y grande hombre que llamaron
Francisco Pizarro.

---No toques, amigo---dijo Ansúrez,---no toques a esos caballeros, a
quienes tengo yo por gigantes que no dejaron sucesión, ni con ellos
compares a nuestra familia enana de estos tiempos.

---Dices bien, Diego, que al comparar modernos con antiguos, resulta que
no levantamos más de media cuarta del suelo\ldots{} Sigo mi cuento. Para
echarlo a perder, nos mandaron también al señor Salazar y Mazarredo, que
por las ínfulas y prepotencia que se traía, cayó muy mal aquí. Y lo que
mayor enojo levantaba era el título de \emph{Comisario Regio}, que en
los oídos de esta gente sonó como el nombre de \emph{Virrey} o cosa tal.
En fin, era corriente aquí que entre Pinzones y Salazares nos iban a
quitar la bendita independencia\ldots{} ¿Y qué te diré de la ocupación
de las islas Chinchas, que fue como quitarle al Perú el corazón y el
estómago? Los españoles no querían ser la buena madre, sino la madrastra
de América\ldots{} Todo iba mal, y esta gente, cada vez más encendida.
Llegó un día fatal, mejor diré, la noche en que se quemó la
\emph{Triunfo}. Te aseguro que la fragata era como un volcán\ldots{} Las
llamas pintaban de rojo todo el cielo.

---Aguárdate, Mendaro, y perdona que te interrumpa---dijo Ansúrez
inquieto, poniendo la mano en el hombro de su amigo.---Mucho me interesa
tu cuento; pero deja para otro día lo que falta, y hablemos de lo que a
mí particularmente me coge toda el alma. ¿Podré saber hoy mismo si está
mi Mara en Lima, si me será fácil verla y hablar con ella? Bien enterado
estás ya de lo que me pasó, ¡Jesús me valga!, y yo confío en que me
ayudarás a encontrar a mi querida niña. Ya te dije que no vengo de
malas; traigo el corazón dispuesto para perdonarlos y hacer las paces,
siempre que ellos quieran hacerlas conmigo.

---Voy creyendo que más que distraído estás trastornado---replicó
Mendaro,---pues ya te dije que nada podré saber de esa cuestión tuya,
mientras no vuelva mi compadre Amador con respuesta al encargo que le di
de averiguarme esos puntos. Yo no conozco a los Chacones más que por la
fama de su riqueza: sé que murió el padre, español bragado y de sangre
en el ojo; que el hijo mayor, coplero, avispado, loco por ver tierras,
se fue y volvió\ldots{} y no sé más. Amador, que conoce a esa familia,
no tardará en traernos informes. No te impacientes, ni con el
pensamiento te vayas a Lima volando por los aires, que luego iremos por
el ferrocarril, y algo hemos de saber de tu hija Mara, que, por lo que
recuerdo, es una morenita muy salada.

---La más salada y graciosa que ha echado Dios al mundo---dijo Ansúrez
conteniéndose para no llorar.---Ella fue toda mi alegría, y después mi
tormento y desesperación. No hablemos de esto; no quiero afligirte.
Sigue tu cuento, y yo haré por escucharlo sin perder gota, digo, sílaba.

---Se fue Pinzón enhorabuena, y nos vino Pareja con las fragatas
\emph{Blanca}, \emph{Berenguela} y \emph{Villa de Madrid}. Este señor
Pareja nos pareció más templado que el otro, y de buena mano para los
arreglos de paz. Así fue: tuvimos paces, y en ellas descansaríamos sin
el maldito suceso del Cabo Fradera, en Febrero de este año. ¡Ay, qué
atroz barbarie! Y tengo que reconocer que esta vez la culpa fue del
Perú, por el descuido y pachorra de estas autoridades\ldots{} Aquí se
armó el tumulto; aquí vimos la reunión de gente vaga, y oímos sus gritos
contra los tripulantes de la \emph{Resolución} que bajaron a tierra. Los
españoles, advirtiendo la que se armaba cogieron las lanchas para
volverse a bordo; quedó rezagado el pobre Fradera; trató de ganar a nado
un bote, pero el botero no quiso recogerle; volvió el infeliz a tierra,
y con los pies en el agua, en la mano un cuchillo, se defendía
bravamente de los malos patriotas que le acosaban. En fin, que muerto
cayó entre agua y arena, y estos perdidos y borrachos cantaron su hazaña
con berridos espantosos. La justicia les metió mano; hubo prisiones y
castigos; pero al mal efecto de aquel atropello bárbaro no se pudo echar
tierra, y por él quedaron las relaciones entre españoles y peruanos tan
agrias y picajosas como las encuentra la \emph{Numancia} al arribar al
Callao.

A este punto llegaba Mendaro de su cuento, cuando compareció en el
patinillo una mujer alta, fornida, de solidez estatuaria, ojos negros,
gruesa y bien formada boca, pecho sobresaliente. No era de abolengo
incaico, ni su regia estampa provenía de imperio del Sol; era una
cuarterona de las que llaman \emph{zambas}, ejemplar excelente de la
mezcla de sangres etiópica y ariana, que suele aunar el cuerpo admirable
y las facciones bellas. Traía de la mano un chiquillo gracioso, que en
cuanto vio a Mendaro corrió hacia él y se montó en sus piernas. El niño
era el hijo, y la mujer, la esposa del pulpero, y los tres se llaman lo
mismo: José, Josefa y Pepito. Con un gesto autoritario indicó la mujer a
los dos varones que se apartaran de la mesa para poner los manteles y el
servicio. Obedecieron. Tan pronto gastaba Josefa su saliva en reñir al
chiquillo, que enredaba con los platos y cucharas, como en recomendar a
su marido que vigilase la tienda mientras la familia se disponía para
comer\ldots{} Y entre col y col, ponía la señora vanidosos programas de
la comida, que era extraordinaria en honor del amigo forastero.

Acudió Mendaro a la tienda con una solicitud presurosa, que era como la
medida de los pantalones que en el gobierno doméstico gastaba su mujer;
y esta, entre tanto, hizo cumplido elogio de los platos que serviría y
de su condimento. «Señor Diego, ¿le gusta a usté el arroz con pato? ¿Sí?
Pues como el que yo he guisado para usté no lo habrá comido nunca, ni lo
comerá mejor la Reina de España\ldots{} ¡Ay, qué cosas dicen acá de su
Reina de ustés, la Isabel!\ldots{} Pues también le pondré un
\emph{tamal} que ha de saberle a gloria\ldots{} Los españoles no saben
hacer buena comida\ldots{} ¿Verdá que en España no hay maíz?\ldots{} Por
eso vienen aca ustés tan amarillos\ldots{} por eso andan doblados por la
cintura, como si se les cayeran los calzones\ldots{} ¿Le gusta a usté el
\emph{sancochado}? ¿En España hay \emph{sancochado}? ¿Qué dice? Ya; que
allá tienen el cocido. Pues yo he comido cocido español, y no me
gusta\ldots{} ¿Es verdá que en España no da la tierra más que garbanzos
y aceitunas?\ldots{} Las aceitunas las como yo cuando el médico me manda
\emph{gomitivo}\ldots{} Y esa Reina que allí tienen, ¿cuándo la
\emph{gomitan} ustés?» Con estos y otros dicharachos puso la mesa, y a
punto volvió Mendaro de la tienda con una botella de \emph{pisco} y dos
de vino del país\ldots{} «Este es el Valdepeñas de acá---dijo a su
amigo.---No es malo; se sube hasta el primer piso, y de ahí no pasa. Si
bebes mucho, te pondrás alegre y dirás lo que dice el nombre de
Arequipa: \emph{aquí me quedo}. Este aguardiente blanco que llamamos
\emph{pisco}, es de vino\ldots{} cosa buena: los que empinan mucho, ven
a Dios en su trono.»

Sentáronse a comer, y con alegría y buena conversación despacharon uno
tras otro los platos que Josefa encarecía pomposamente antes y después
de que fueran gustados. A la sopa de rabioso picante siguió el
\emph{sancochado}, que viene a ser como nuestro cocido; desfilaron luego
el \emph{pejerrey} (pescado chico) y la \emph{corbina} en salsa (pescado
grande); y por fin, con honores extraordinarios, el pato en arroz, que
era más bien como una \emph{morisqueta} con pato. Mendaro, en continua
relación con las botellas del tinto de la tierra, se apimpló un poco;
Josefa hablaba no sólo por la boca, sino por los codos, manifestando en
cada cláusula su ojeriza contra la Reina de España; el chiquillo
amenizaba el banquete, ya con llantos y berridos, ya con risas y copiosa
emisión de babas y mocos. Y cuando por postre comían alfajores y
\emph{chancaca}, la cuarterona, limpiándole la jeta a su criollito, dijo
al convidado: «Señor Diego, lo que le digo ahora no quise decírselo
antes, para que comiera tranquilo, que lo primero es comer, y lo
segundo, decir las cosas que han de decirse, aunque sean malas\ldots{} Y
es que no se canse usté en buscar a su hija, porque Amador vino y yo le
pregunté: `Amador, ¿qué hay de eso?' y él me contestó: `Comadre, hay que
los señores de Chacón no están en Lima'. Con que ya lo sabe. Para verlos
y enterarse, tiene usté que ir al Cuzco.»

\hypertarget{xv}{%
\chapter{XV}\label{xv}}

---¿Y el Cuzco está cerca?---preguntó Ansúrez, sintiendo dentro de sí al
patriarca Job con toda su paciencia.---¿Podremos irnos allá y volver en
una tarde?

Rompió Josefa en carcajadas estrepitosas, que empalmaron con estas
expresiones de su marido: «Sí, hombre, sí\ldots{} Está cerquita\ldots{}
cerquita el Cuzco\ldots{} ahí, a la vuelta del primer cerro\ldots{} Poca
distancia\ldots{} Para que te hagas cargo\ldots{} es como tres veces de
Cartagena a Madrid\ldots{} Caminito muy llano, como una sala\ldots{}
Subes los Andes\ldots{} después los bajas\ldots{} para volver a
subirlos\ldots{} Cuestión de diez y ocho días\ldots»

---Para que vean ustés---dijo la hembra talluda sin dejar de reír---que
los caminos de América son caminos grandes, no como los de España,
caminos de juguete. Aquí no gastamos distancias de broma. O vamos lejos,
o no vamos a ninguna parte.

---No te precipites, Diego, a coger la vuelta del Cuzco, que está donde
Nuestro Señor Jesucristo perdió las sandalias\ldots{} Antes de ir tan
lejos, entérate por ti mismo de lo que ocurre. Bien podría suceder que
mi compadre Amador, aficionadillo al pisco, haya empinado hoy más de lo
regular\ldots{} Vámonos, pues, a Lima, y preguntaremos en la propia casa
de los Chacones.

No necesitó Ansúrez que su amigo se lo dijera dos veces. Propuesto el
paseo a Lima, quiso emprenderlo sin perder minutos. Requirió Mendaro la
chaqueta y sombrero, empuñó un bastón nudoso, y pasando por la tienda,
donde imperante quedaba la gallarda Josefa, salió con Ansúrez a la
calle. Momentos después cogían el tren; a la media hora de traqueteo
suave llegaban a la ciudad de los Reyes, y a buen paso tomaron la calle
que conduce a la plaza. Ni en personas ni edificios ponía su atención
Diego, que llevaba dentro de sí los espectáculos de su personal interés.
«Esta es la Catedral---decía Mendaro con inflexión encomiástica;---aquel
el palacio de los Virreyes, hoy de la Presidencia y Gobierno de la
República\ldots» Contestaba el Oficial de mar con un mugido y una mirada
de indiferencia, y seguían adelante. «Por aquí es---dijo Mendaro,
guiando a una calle que de la esquina del palacio arzobispal arrancaba,
extendiéndose recta en toda su longitud.---Al final, en la última
cuadra, viven los tales Chacones. Repara en las buenas casas de gente
noble que hay por aquí. Muchas son del tiempo de los señores Virreyes;
otras, fabricadas después, tienen la misma traza y adorno de puertas y
balcones.»

La única observación que hizo Ansúrez fue para indicar la semejanza del
caserío de Lima con el de algunas ciudades andaluzas, y el tono claro de
las fachadas, blancas las unas, otras de ocre o azul muy bajo. Fijose
también en que no había tejados, sino azoteas, observación que sugirió a
Mendaro esta otra, pertinente a la meteorología: «Te diré que aquí no
sabemos lo que es llover, ni se conocen los paraguas. No tenemos más que
un rocío, que llaman \emph{garúa}, el cual por las noches, así refresca
la tierra como nos moja y cala hasta los huesos. Por este beneficio del
cielo, no echamos de menos la lluvia, y no se gastan aquí canalones ni
aljibes.»

---Dímelo a mí---observó Ansúrez,---que todas las mañanas me encuentro
la cubierta como acabada de baldear, y el velamen y toldos tan mojados,
que se les podría torcer\ldots{} No te diré yo que sea beneficio el caer
el agua del cielo en esa forma de rocío; paréceme más bien maleficio,
porque si lloviera de golpe, quedarían las calles más limpias de lo que
están\ldots{} ¿Tenéis por ventura río caudaloso?

---Río tenemos: se llama el \emph{Rimac}, y es nombrado, más que por el
caudal de sus aguas, por el magnífico puente de piedra, obra de los
españoles, que luego veremos. Por allí se pasea la gente para tomar la
fresca en las tardes de bochorno\ldots{}

Observó también Ansúrez el grandor y pintoresca hechura de los balcones
de las casas principales, al modo de estancias voladas, con adorno
exterior arabesco y celosías verdes. Eran la comunicación romántica de
la casa con la calle y con el mundo; el conducto de las miradas, del
suspirar y del amoroso acecho; eran el rostro enmascarado de la pasión,
y un emblema étnico más español que la propia España. Hallábase el
celtíbero absorto en el examen de uno de aquellos balcones, el más
historiado y holgón de la calle, al extremo de esta, cuando Mendaro le
puso la mano en el hombro y le dijo: «Esta casa que miras es la de los
Chacones. Veo que está cerrada a piedra y barro, por lo que entiendo ser
verdad lo que nos dijo el borrachín de Amador. Si te parece, llamaremos,
que alguien habrá dentro que guarde el edificio.» Y antes que Ansúrez
respondiera, llegose a la puerta, y agarrando el pesado aldabón, dio
golpes y más golpes, sin que de dentro viniera voz de quién vive ni
respuesta alguna.

La emoción de Ansúrez ante la casa en que moraba la familia de Belisario
fue tal, que no pudo tenerse en pie. Arrimose a la pared frontera, y en
el escalón de una puerta, cerrada también como puerta de inquilinos
ausentes, se dejó caer: llanto amarguísimo vino a sus ojos, y para
disimularlo y esconderlo, con ambas manos puso máscara en su rostro.
Mendaro, dejando pasar medio minuto, volvió a empuñar el aldabón y
repitió los furibundos porrazos\ldots{} La casa hacía esquina, de la
cual partía un callejón estrecho, y a lo largo de este, como por el tubo
de una bocina, vino una voz bronca que gritaba: «¡Quién\ldots{} quién!»
Asomose Mendaro al callejón, y a su vez gritó: «Los quiénes somos
nosotros, gandul, que estamos aquí llamando hace dos horas, sin que nos
responda nadie: ven aquí, y ven con respeto, y dinos dónde están tus
amos.»

Apareció doblando la esquina un hombre que por el color del hocicudo
rostro y la largura de sus brazos y la corva inclinación de su cuerpo,
más parecía cuadrumano amaestrado para racional que racional efectivo, y
apenas le vio Mendaro, lo cogió por el cuello, y con voces descompuestas
le dijo: \emph{«Cholo}, sin vergüenza, ¿por qué no has abierto a la
primera llamada? ¿Así cuidas la casa de tus señores? ¿Qué hacías,
borracho? ¿Dormías el \emph{pisco?»}

---Suéltame, \emph{gachupín---}gritó el hombre feísimo, queriendo
desprenderse de la garra de Mendaro. Pero en este había estallado la
fiereza un tanto insolente del español educado con el catecismo de los
tiempos heroicos, y no soltaba su presa, ni suavizaba su duro
acento.---Ven aquí, perro, y contesta sin mentir a lo que te
preguntamos.

---Suélteme, \emph{¡carachitas!.}.. ¡Ay, ay!\ldots{} Le diré la verdad,
patrón; suélteme.

A los chillidos del infeliz \emph{cholo} (así llaman a los últimos
retoños degenerados de la raza india), víctima de la ingénita altanería
de Mendaro, acudió Ansúrez enjugando sus lágrimas y con formas de
lenguaje más benignas: «Déjale; no le trates con dureza\ldots{} Vele ahí
por qué no nos quieren en América\ldots{} Por eso, José, por tus modos
tiránicos\ldots{} Oiga usted, buen hombre: queremos saber\ldots{}
Esperamos que usted nos diga con toda verdad\ldots»

---No esperes de él la verdad si le tratas con esas blanduras,
Diego---dijo Mendaro.---No te fíes de estos ladinos y traidores. Verás
cómo te sale con algún \emph{despapucho}, con alguna sandez o mentira
gorda que te desoriente y te vuelva tarumba.

---No tendrá tan mal corazón---indicó Ansúrez,---que engañe a un pobre
padre\ldots{} de quien no ha de recibir ningún daño, sino todo lo
contrario, quiero decir, una buena recompensa.

---El caso es este---declaró Mendaro algo amansado de su fiereza por el
ejemplo del amigo:---sabemos que tus amos se han ausentado, y deseamos
saber dónde están\ldots{} pero sin engaño.

Fosco y sombrío, el indio no desmentía la condición suspicaz de su raza
humillada y decadente. No miraba a la cara de los españoles, sino al
suelo, como más digno de sus miradas, y al suelo arrojaba también la
respuesta desdeñosa, que rebotó en pregunta: «No hay engaño\ldots{} yo
no tengo por qué engañar\ldots{} ¿Pero a qué cuento quieren saber los
\emph{gachupines} dónde están mis amos?»

---Este caballero---afirmó Mendaro---es el padre de tu señora, quiero
decir, de la \emph{señorita esposa} que el hijo de tu ama, don
Belisario, ha traído de España. ¿Te enteras, animal?\ldots{} Levanta tus
ojos del suelo, zorrocloco, y mírale, mira a este señor, que es el
padre, el padre\ldots{} ¿Sabes lo que es padre, zopenco?

Recogió del suelo sus miradas el \emph{cholo}, y las paseó por el cuerpo
de Ansúrez. Como este vestía de uniforme, cada uno de los botones fue un
punto en que el mirar del indio se detenía con asombro y una risa
estúpida. Sacó Diego una monedita de oro, y se la mostró como una
hostia, diciéndole: «Esto para ti si hablas con verdad.» Pero a Mendaro
le pareció excesiva la oferta, y quiso atajar el movimiento generoso de
su amigo con estas palabras: «No, no, Diego. Con cuatro \emph{soles}
habría para comprar a todos los \emph{cholos} que quedan en esta tierra.
Ofrécele un \emph{sol} (duro), y el hombre tendrá para comprarse unos
calzones, que, ya lo ves, le hacen mucha falta.»

El pobre indio, que en su desmedrada catadura y cobrizo rostro cuarteado
no revelaba claramente su edad, aunque esta debía estar ya muy lejos de
la juventud, quedose como encandilado al ver la moneda, y alargando
hacia ella sus manos, dio una zapateta en el aire, y soltó la respuesta
que Ansúrez esperaba: «Mi patrón, démela y se lo digo. Me llamo Santos,
y por todos los mis patronos de la Corte celestial, le juro que de mi
boca no saldrá mentira: los amos míos, mi ama doña Celia, mi amo don
Belisario y mi ama doña Marina, están en Jauja.»

Oyó Diego el nombre de Jauja como cosa de burleta o de pasar el rato,
pues aunque no ignoraba la existencia de tal pueblo peruano, en aquel
instante, hallándose en la plenitud de sus ideas españolas, Jauja era el
cuento de los perros atados con longaniza y de los árboles que dan
chorizos y jamones; se acordó de la \emph{Pata de Cabra} y de los mil
chistes jaujanos, y puso en cuarentena el dicho del indio. Pero Mendaro
le sacó de este yerro, diciendo: «Puede ser, puede ser verdad, que allí
tienen los Chacones haciendas muchas.»

---Buen amigo---dijo Ansúrez a Santos, sin dejarse arrebatar la moneda
que este quiso coger antes de tiempo,---necesito más referencias\ldots{}
y que me pongas en conocimiento de muchas cosas que ignoro. ¿Te gusta el
\emph{pisco}? Pues vente con nosotros, y en cualquier pulpería te
convidaremos, para que sueltes la sin hueso y me resuelvas todas las
dudas.

Cuando esto decía el Oficial de mar, ya se habían arrimado al grupo
algunos zanganotes, mujeres y chicos. Ni Ansúrez ni su compañero se
habían fijado en esta adherencia de público, que fue creciendo,
creciendo, cuando los dos amigos y el \emph{cholo} iban camino de la
pulpería más cercana, Mendaro fue el primero en revolverse contra la
molesta escolta, que a los pocos pasos se desmandó, haciendo befa del
uniforme de Ansúrez y arrojando sobre los dos \emph{gachupines}
pelotadas de barro y algunas almendras de arroyo. Movido de su impetuoso
genio, que en trances de peligro siempre se mostraba, Mendaro se plantó
en medio de la calle, y mirando a la chusma se dejó decir: «¿A que saco
la navaja? ¿A que alguno de estos sinvergüenzas nos va a enseñar el
mondongo?» El prudente Ansúrez acudió a contenerle. Santos, en la
expectativa de la moneda de oro, dirigió a la muchedumbre palabras
conciliadoras. Con los dimes y diretes de una y otra parte, la cuestión
fue tomando mal cariz, y en esto acertó a presentarse en escena,
saliendo de una calle lateral, el maquinista Fenelón, vestido de
paisano, con dos amigos suyos limeños de la mejor apariencia social.
Aplacaron estos el incipiente tumulto, declarándose defensores de los
dos \emph{gachupines}, y dispersando a los grupos plebeyos.

Mientras esto ocurría, informó Diego a Fenelón del motivo de su
presencia en aquella parte de la ciudad, y de llevar consigo al indio
Santos. El maquinista, con el aplomo y superioridad que en sus palabras
sabía poner, le dijo: «¡Pobre Ansúrez, yo te habría sacado de dudas a
bordo esta noche! Felizmente, he podido enterarme hoy de lo que pasa en
tu familia, y te lo contaré. Nadie podrá informarte con más exactitud,
mi palabra de honor\ldots{} Este \emph{cholo} te ha dicho que tu hija
está en Jauja\ldots{} Ha mentido sin mala intención\ldots{} no le
pegues\ldots{} O no sabe la verdad, o se le ha mandado que diga lo que
has oído\ldots{} Dale los cuatro \emph{soles}, y que se vaya a la porra.
No es ese el guardián de la casa de los Chacones; no es más que un
galopín del verdadero guardián, Arístides Canterac, francés, con quien
he jugado al billar hace dos horas, mi palabra. Por él he sabido que tu
hija no está en Jauja, sino en Arequipa.»

Sosegados todos, incluso Mendaro, que aún daba resoplidos patrióticos;
desaparecido el \emph{cholo}, que partió con la chusma, guardando su
moneda donde no pudiesen quitársela, los dos españoles, el maquinista y
los peruanos se dirigieron a un \emph{restaurant} francés, donde
refrescarían charlando. Ansúrez les siguió, más que por querencia de
charla y frescura, por calmar el ardor de su alma, sedienta de verdad.
¿Por qué no estaba su hija en Lima? ¿Huía de su padre, o de quién huía?
¿Era dichosa\ldots?

\hypertarget{xvi}{%
\chapter{XVI}\label{xvi}}

«No dudes que los Chacones están en Arequipa---dijo Fenelón al
celtíbero, que permanecía como atontado mientras los demás bebían y
charlaban.---Al partir dieron a su servidumbre esta consigna: `Vamos a
Jauja'. Querían despistar al Gobierno y escurrir el bulto\ldots{} ¿No
comprendes esto, pobre Ansúrez? Pues es raro, porque un español, criado
entre el bullicio de los pronunciamientos, entiendo yo que oirá crecer
la hierba. ¿No has conocido que la revolución late en el Perú? Late y
colea; sólo que anda todavía por debajo de las sillas y de las mesas,
por debajo de las camas, por debajo de los altares. Belisario y su mamá
doña Celia son del partido revolucionario, como amigos y no sé si
parientes del Gran Mariscal Castilla, gigantón de esta fiesta. ¿No caes
en la cuenta de que la razón o pretexto de los revolucionarios es el
tratado de paces con España, que firmaron Pareja y el Presidente Pezet,
arreglo que la gente levantisca considera como la mayor ignominia del
Perú? Este patriotismo gordo y populachero es excelente cosa para
ornamentar las banderas revolucionarias en los países de sangre
española\ldots{} Pues oye más, hombre inocente y sin hiel. Tu yerno
Belisario y tu consuegra ilustre son los adeptos más rabiosos del bando
antiespañol del Perú. Mira por dónde tu graciosa Mara, la morenita del
tipo \emph{Virgen de Murillo}, la de las sales granadinas, la discípula
de las monjas, ha venido a ser una antiespañola furibunda.»

---¡Ajo, eso no!---gritó Ansúrez dando una fuerte palmada en la mesa. El
inmenso estupor con que oía los informes del francés, contuvo su
protesta en esta brutal concisión.

---Yo no aseguro su antiespañolismo; pero lo presumo, porque el amor
funde los sentimientos de marido y mujer. Mara siguió a Belisario
deslumbrada por la poesía exuberante de América. América es ya su
patria; España, clásica, rígida y enjuta, ya no lo es. ¿Qué significa
esto, cándido Ansúrez? ¿Te acuerdas de nuestra primera conversación en
la borda de la \emph{Numancia}, cuando tomábamos carbón en San Vicente?
Todo lo que tú no entendías entonces te lo explicaba yo con una sola
palabra: \emph{romanticismo}. Romántico fue el amor de tu hija;
románticamente te la robó Belisario; al Perú vinieron a realizar su
ensueño; se han casado; son riquísimos\ldots{} Todo esto quiere decir,
\emph{por ejemplo}, que cuando España arroja de sí el romanticismo,
América lo recoge. Los ideales que desechan las madres maduras son
recogidos por las hijas tiernas\ldots{} España coge su rueca, y se pone
a hilar el pasado; tu hija hila el porvenir\ldots{} en rueca de oro.

Diciendo esto, Fenelón se atizó de golpe una copa de coñac. Inquieto y
sofocado, Ansúrez no sabía qué pensar, no sabía qué decir. Llevábase las
manos a la cabeza; luego, sobre la mesa las dejaba caer desplomadas; por
fin, arrancose con estos desordenados conceptos: «Me vuelvo loco\ldots{}
¡Mi Mara antiespañola! ¡Ajo, eso no! ¡Vámonos a España con cien mil
pares de ajos! Llévenme a mi casa, llévenme a mi fragata.» Ya levantado
para salir, los amigos trataron de aliviar su pena, y Fenelón terminó
sus informes con estas advertencias adicionales: «Los Chacones, y tu
hija con ellos, se han marchado al Sur por ponerse a salvo de las iras
del Gobierno, y por vivir donde se guisa la revolución, que es el
territorio entre Arequipa y el Cuzco\ldots»

Era ya hora de volver a bordo; acudieron al tren, y en todo el trayecto
hasta el Callao no paró Fenelón en las amenas referencias de sus
audacias amorosas. Lima era la Jauja del amor; él, vestido de paisano y
hablando francés, burlaba la prevención reinante contra la Marina
española. Todos reían de sus fabulosas conquistas, menos Ansúrez, que no
le hacía ningún caso. Despedidos cariñosamente en el muelle, los dos
vecinos de la \emph{Numancia} volvieron a su vivienda, alegre el
hispano-francés, sumido en profunda y negra melancolía el que llamamos
celtíbero. Las emociones de aquella tarde le tenían medio trastornado:
desconoció, por breves segundos, a su compañero Sacristá; desconoció
también el departamento donde moraba, y en la turbación de su mente hubo
de sacudir su dormida memoria, diciéndose: «¿Dónde estoy? ¿Qué casa es
esta?»

En aquellos días, el Oficial de mar \emph{pagó la chapetonada}, que así
llamaban los peruanos, desde tiempos remotos, a la fiebre de
aclimatación, tributo de que pocos europeos se eximían en la costa del
Pacífico. Era una terciana comúnmente benigna; pero en Ansúrez fue por
excepción bastante intensa y dolorosa, quizás a causa de la tristeza y
depresión del ánimo, que le predisponían a toda enfermedad. Atacado ya
de la terciana, escribió a su hija, poniendo en ello la fiebre que ya le
requemaba la sangre. Escribió también a Belisario y a doña Celia; mas no
contento del sentido de las cartas, las rompía, y así consumió gran
copia de cuadernillos de papel. Tal carta en que con extremadas fórmulas
de amor perdonaba y pedía paces definitivas, le pareció humillante. Los
Chacones eran riquísimos, y él un pobre marinero: lo que en dinero no
poseía, debía poseerlo en dignidad. Por fin, todo el fárrago epistolar
quedó reducido a una sola carta, dirigida a la prenda de su corazón,
diciéndole ternezas y pidiéndole vistas. «Estoy en el Callao, soy
contramaestre en la \emph{Numancia}\ldots{} ¿No quieres ver a tu padre?
Véate yo, hija de mi alma, y muérame después de verte. Tus riquezas no
tienen valor para mí. La luz de tus ojos es mi riqueza: dámela, y
guárdate lo demás\ldots» Estos y otros conceptos amorosos y sutiles
enjaretó. Satisfecho de haber expresado sus sentimientos con el mayor
decoro y sin asomo de interés, cerró su carta, y a tierra la llevó para
depositarla por su propia mano en el correo; que de nadie podía fiarse
en cosa que tan vivamente a su corazón interesaba. Al regresar a bordo,
la fiebre ardiente le tumbó en el coy, de donde no pudo levantarse en
muchos días.

Asistíale don Luis Gutiérrez con cuidado y cariño; Sacristá, que como a
hermano le quería, visitábale con frecuencia, informándose por sí mismo
del curso de la traicionera enfermedad. En los días de remisión febril,
la enfermería de paz era muy frecuentada de amigos y compañeros.
Guardias marinas y Oficiales bajaron al sollado, y el mismo don Casto,
que era un ángel, practicó las obras de misericordia, acercándose con
piedad y afecto al lecho de su compañero en las fatigas de la
mar\ldots{} Y cuando la remisión era intensa, permitían a Binondo dar a
su amigo conversación tirada, y aun leerle vidas de santos, que en
aquellos días el \emph{Año Cristiano} era la ocupación predilecta del
cabo de mar. No acababa el malayo de ponerse bueno, y cuantas veces
intentó trabajar, sus esfuerzos le privaban de aliento. Relevado estaba,
pues, de toda faena, y el pobre hombre empleaba su tiempo en exhortar a
sus compañeros a la piedad, y en hacerles descripciones prolijas de la
Bienaventuranza eterna. Unos se reían de esto, y otros no; pero entre
burlas y veras, Binondo hacía el apóstol o el misionero laico, no sin
cierto desdén y escama del venerable capellán don José Moirón.

«Embelesado estoy ahora---dijo Binondo sentándose a la morisca junto al
lecho de Ansúrez---con la vida de Santa Rosa de Lima, la gran santa de
América; y sobre lo que ya tengo leído de ella en mi \emph{Año
Cristiano}, tres veces he pasado un librito que me trajo de tierra
Desiderio García, en el cual librito se trata de mil pormenores de la
virtud angélica de la divina Rosa. Como mi hija lleva ese nombre, llego
a figurarme que es ella, ella misma la santa\ldots{} y aunque no lo sea,
yo las igualo en la hermosura\ldots{} Dice el librito que aquí tengo,
que la santa nació en la casita de un corral, propiedad de su padre,
Gaspar Flores, y en dicho corral, ya niña, plantaba clavellinas y
mosquetas\ldots{} Un día advirtió que brotaba un rosal en su jardinito.
Patente era el milagro, pues los rosales no se conocían en el
Perú\ldots{} Y la planta milagrosa dio tantas, tantas flores, que toda
la ciudad pudo gozar de ellas y de su hermosura y olor deleitoso\ldots{}
deleitoso dice el libro. Y así como el aroma, o dígase fragancia, de las
flores plantadas por Dios se extendió a toda la ciudad, y de la ciudad a
todos los Perules altos y bajos, del mismo modo la fama de la santidad
de aquella criatura voló por todo el orbe cristiano: así lo dice el
libro\ldots{} hasta Roma mismamente\ldots{} Dios me tocó en el corazón
para que a mi hija diera el nombre de Rosa. Mi hija está en el Cielo con
los ángeles y serafines. Cada vez que pronuncio su nombre, me da en la
nariz el olor, o dígase fragancia, de aquella flor celestial\ldots{}
celestial dice el libro.»

---A la hija mía puse yo nombre de Marina por la Santísima Virgen del
Mar, y no hay nombre que mejor le cuadre, porque lleva en sí toda la sal
del Océano; tiene también su oleaje, el vaivén de las aguas; y para que
la semejanza sea completa, la mueven temporales duros.

Con lúgubre y pausado acento dijo esto Ansúrez; y el otro, pegando su
hebra en las últimas palabras del amigo, continuó así: «Tempestades tuve
yo también, Diego; ciclón terrible me llevó a mi hija, dejándome anegado
de pena. Pero mi Rosa está en el Cielo; tu Mara también. Hagamos por
morirnos tú y yo santamente, y las tendremos a nuestro lado por toda la
eternidad.»

---Mi hija no se ha muerto\ldots{} no se ha muerto---replicó Diego
inmóvil, triste, mirando a los baos del techo.---Pero la ausencia y la
distancia son peores que la muerte. Si esta enfermedad acaba conmigo, no
veré a mi hija, y seré mas desgraciado que tú\ldots{} porque tú la verás
pronto\ldots{} puesto que ya la tienes allá, José\ldots{} Tú no tardarás
en morirte, y en cuanto llegues, verás aquellos ojuelos negros y
chiquitos, como los de los ratoncillos; la nariz chatuca y desdoblada;
verás la color de aceituna, la boca reventona, con aquellos dientecillos
que parecen nieve entre tomates.

---Poco a poco---dijo Binondo picado.---No tomes a chanza la cara linda
de mi niña, que si fue preciosidad en la tierra, mayor lo es en el
Cielo; que allá el jaramago se vuelve clavellina\ldots{} clavellina: así
lo dice el libro de Santa Rosa.

---Mi hija es bella, y no necesita que la lleven al Cielo para que se le
aumente la hermosura---murmuró Diego con cierto desvarío, que indicaba
el recargo febril.---En la vida de América se ha puesto más
bonita\ldots{} es más señora y apersonada, más suelta de lenguaje. No
hay preciosidad como ella en todos los Perules del Sur ni del
Norte\ldots{} Mi hija vive en un palacio\ldots{} la sirven quinientos
criados negros, rojos o amarillos\ldots{} come en vajilla de plata y
bebe en copas de oro. Todos los metales preciosos que dan las entrañas
de los Andes, son para ella\ldots{} ¡Y yo no puedo verla muriéndome,
como verás tú a la tuya\ldots! Para verla, tengo que vivir y navegar
mucho tierras adentro. ¿Y cómo navego yo fuera de mi barco, si de aquí
no puedo salir? Estoy en España; mi hija está en América, lejos, lejos,
y ya no quiere ser española\ldots{} ¡Válgame Dios, qué calor siento!
Dame limón, José; me abraso\ldots{}

Así prosiguió divagando hasta que le cogió el sueño. Rosario en mano,
Binondo rezaba entre dientes. La noche fue tranquila. Siguieron días de
quietud vaga y letárgica, en los cuales, desde el amanecer de Dios hasta
la hora de silencio, iba contando Ansúrez todos los toques de corneta,
campana, tambor y pito que marcaban las distintas faenas, maniobras y
ejercicios que sucesivamente se practicaban a bordo.

La terciana fue más larga que intensa, y hasta Junio no pudo Diego
llamarse convaleciente. La reparación orgánica se retrasaba por causa
del hondo abatimiento en que el ánimo del pobre celtíbero se mantenía.
Lo que mayormente le angustiaba era no recibir contestación a la carta
que escribió a su hija, y todo era cavilar y hacer cómputos de distancia
y tiempo para explicarse la tardanza. Por segunda y tercera vez
escribió, y no habría dado paz a la pluma si el amigo Fenelón no calmara
su ansiedad con razones de mucho peso.

«No seas chiquillo, Ansúrez---le dijo una tarde, sentaditos los dos en
el camarote de maquinistas;---no olvides la extensión de los caminos del
Perú, siempre largos, ahora más, por el trastorno de estas revoluciones
malditas. De lo que me ha dicho Canterac estos días, deduzco que la
familia de Mara no está ya en Arequipa, sino en el Cuzco\ldots»

---Y ese Cuzco\ldots{} entiendo que está en el propio riñón de los
cansados Andes\ldots{} La verdad, no sé para qué levantó Dios esa
cordillera tan alta, de Norte a Sur. Es como un grandísimo pisa-papeles
que puso a lo largo de estas tierras para que no se las lleve el viento
ni las arrebate la mar\ldots{} Dime otra cosa: ¿no fue en el Cuzco donde
tenían la cabeza de su imperio aquellos indios que llamaron incas, y que
eran como hijos del Sol?

---Así es. En el Cuzco tuvieron su capital. El imperio era grandísimo, y
lo poblaba una raza industriosa y guerrera. Francisco Pizarro, que no
sabía leer ni escribir, pero tenía, \emph{por ejemplo}, un corazón más
grande que esos montes que vemos, y en su voluntad volcanes de furor, y
en su cabeza, vacía de letras, pensamientos altísimos, se apoderó en
poco tiempo de aquellas salvajes grandezas y cargó con todo; después
vino y fundó esta Lima hermosa, y en ella puso la simiente de las lindas
limeñas\ldots{}

---De seguro, en ese Cuzco tendrá la familia de Belisario algún
palacio\ldots{} Puede que sea el alcázar mismo de aquellos emperadores
incas o incaicos, como aquí dicen, restaurado y puesto a la moderna.
Será todo de piedra mármol jaspeada, con tropezones de metales
preciosos\ldots{} Yo me lo figuro así, y en él veo a mi hija como a una
reina\ldots{} como a una emperadora\ldots{} ¿Es así, Fenelón?

---Así puede ser, porque los Chacones son riquísimos. He podido
informarme de su caudal; me han hecho la cuenta, al dedillo, de las
rentas que disfrutan. Es un escándalo, Diego; es un ultraje a la
humanidad, que unos pocos posean tanto, y los más se pudran en la
miseria, en un trabajo de animales\ldots{}

---¿Y el cuánto, Fenelón? Dime el cuánto de esa riqueza\ldots{} pero con
verdad. Deja en tu cabeza las mentiras, y échame cifras\ldots{} buenos
números claritos.

---Pues entre doña Celia y sus hijos, que son tres, gozan una renta
de\ldots{} ello se aproxima a cuatrocientos mil soles\ldots{}

---¿Al año?

---Naturalmente. Mi palabra de honor, que la cifra no es de fantasía.

---Pues lo parece, y yo me quedo atontado escuchándote\ldots{} Me
acuerdo ahora de lo que pasó en la correduría de Cartagena, cuando quise
coger a Belisario por los cabezones para tirarlo al mar\ldots{} me
acuerdo también de cuando, caminito yo de Motril con mi niña en brazos,
le encontramos vestido pobremente, negro del sol y del aire, con
plastones de polvo encima de lo negro\ldots{} en fin, que daba lástima
verle\ldots{} ¡Y ahora\ldots! Se vuelve uno loco. Estoy en
América\ldots{} ¿He dado la mitad de la vuelta al mundo, o es el mundo
el que ha dado media vuelta en derredor de mí? No sabe uno lo que le
pasa. Esto es vivir dos veces, Fenelón; esto es haberse uno muerto, y
resucitar\ldots{} en otro mundo.

\hypertarget{xvii}{%
\chapter{XVII}\label{xvii}}

Pasados muchos días, sin que el historiador pueda precisar su número,
volvió Fenelón a su amigo con nuevos y más preciosos informes. Al
anochecer, en la batería para resguardarse de la \emph{garúa},
arrimáronse a una porta y charlaron largamente, sentados en el suelo,
sin más testigos que la formidable cureña, y el cañón que al mar
apuntaba con su boca muda. «Hay grandes novedades---dijo el
hispano-francés,---y la primera es que la revolución, que estaba en
manos torpes, ha pasado a las del General Canseco, Vicepresidente de la
República (entre paréntesis, primo hermano de doña Celia). ¿No sabes lo
que ocurre? Ello parece mentira; pero es verdad, mi palabra\ldots{} Pues
se ha sublevado la escuadra peruana\ldots{} La fragata \emph{Amazonas},
mandada por el Almirante Panizo, navegaba días pasados llevando tropas
al Sur\ldots{} ¿Y qué hizo la tropa? Pues dar el grito, y con el grito,
muerte a toda la oficialidad. Quedó dueña del barco, y como soberana
nombró jefe a don Lisardo Montero, capitán de navío\ldots{} ¿Qué dices,
inocente Ansúrez? (El celtíbero no decía nada.) Lo primero que hizo este
señor fue poner rumbo a Pisco, a la vera de las islas del guano, y allí
estaba la fragata \emph{América}\ldots{} ¿No te acuerdas? Es la que
encontramos en Magallanes. ¿Qué tenía que hacer en Pisco esa otra
fragata más que esperar a que la sublevaran? Montero se le atravesó por
la proa, y enseñándole la andanada, la intimó a que se rindiera\ldots{}
lo que efectuó sin resistencia, porque resistir no podía\ldots{} Después
cayó de la misma manera el vapor \emph{Túmbez}\ldots{} Los sublevados
confían que se les agregará la fragata \emph{Unión}, hermana de la
\emph{América}, que ha de llegar muy pronto. ¿Qué te parece, amigo? ¿Qué
opinas tú de esta trapisonda, que hoy es marítima, y mañana será
terrestre?»

---Como no entiendo yo nada de política---dijo Ansúrez rascándose el
cráneo,---de esta revolución no puedo pensar nada bueno ni malo,
mientras no me digas si con ella estoy más cerca o más lejos de ver a mi
hija y gozar de su presencia.

---A eso voy\ldots{} Tengo motivos para creer que tu hija y su marido y
suegra partieron del Cuzco hace bastantes días.

---Yo he soñado, no sé si anoche o anteanoche\ldots{} que mi hija
estaba, con séquito lucido de caballeros y damas, en una cacería\ldots{}
allá\ldots{} qué sé yo\ldots{} Vi un gran lago\ldots{}

---Ya\ldots{} El \emph{Titicaca}. Habría más bien pesca, o cacería de
patos. Puede ser que tu sueño fuera una visión de la realidad distante.

---¿Y ese lago es muy extenso?

---Calculo que es del tamaño de la isla de Puerto Rico. Ya ves qué
charquito. Y no te diré yo que sus márgenes, o gran parte de ellas, no
sean propiedad de tu hija.

---¿Y qué distancia hay del Cuzco a ese pedazo de mar dulce?

---Como treinta leguas, por caminos endemoniados\ldots{} Pero no hay
distancias para los ricos. Las damas y caballeros que en sueños has
visto irían montados en avestruces\ldots{}

---No hay avestruces en este país, creo yo, Fenelón\ldots{} Irían en
llamas, en guanacos\ldots{} o sabe Dios cómo irían.

---En palanquines, tal vez, cargados por indios\ldots{} Me parece, buen
amigo, que no debemos referir tu sueño al lago \emph{Titicaca}, sino a
otro más pequeño que está en territorio muy distante de la zona del
Cuzco. Para mí, tu hija y los Chacones están ahora en el \emph{Cerro del
Pasco}, donde tienen sus minas, y seguramente, a más de las minas,
palacios, grandes cotos y montes para sus diversiones. Puede que hayan
resucitado allí la antigua caza de cetrería: pájaros rapaces hay aquí
muy para el caso. Como Belisario es poeta, habrá querido dar a su
esposa, \emph{por ejemplo}, el espectáculo de aquellas cacerías tan
magníficas, de los tiempos en que no se conocía la pólvora\ldots{} Lo
que te digo: Belisario lo convierte todo en poesía. Después de cazar con
halcones y gerifaltes en la ribera del \emph{Lago de Junín}, que así se
llama, habrá inventado diversiones acuáticas, mandando construir un
magnífico galerón, como el que tenía el Dux de Venecia para salir a
casarse con la mar, y en él paseará Mara por el lago con sus damas,
pajes y acompañamiento rico y aparatoso\ldots{} Y desde la embarcación
dispararán flechas contra los ánades o cisnes, para que todo sea
poético, conforme a los usos de la edad en que la vida era más bella que
ahora.

---Dará gusto ver a mi hija---dijo Ansúrez en éxtasis,---tendiendo el
arco\ldots{} así, como una diosa, y disparando la flecha con tan buena
puntería, que no habrá pato que se le escape\ldots{} Y puede que también
disparen flechazos contra los peces\ldots{} aunque mejor lo harán con
arpones, que para mí habrá en ese lago abundancia de peces de gran
tamaño, así como toninos o golfines.

---Mi palabra de honor, que también tú, querido, te nos vas volviendo
poeta\ldots{} En ti veo la influencia de América, y la inspiración que
te da el amor a tu hija, porque el amor es el manantial de la
poesía\ldots{} Mira por dónde lo que fue tu desesperación ha venido a
ser tu consuelo.

---¡Oh!, no, Fenelón\ldots{} dejemos estas tonterías---replicó Diego
tornando a la realidad, como el aeronauta que da salida al gas para
descender a tierra.---Tú eres quien me ha trastornado con tus
invenciones románticas de la caza de cetrería y del pasear en galerón
por esos lagos de engañifa\ldots{} Dime la verdad, Fenelón amigo: tú has
bebido hoy más de la cuenta.

---Cuatro copas no más he tomado después de comer. Economizo mi Jerez,
que se me concluye, y no sé cómo reponerlo. Tú eres el que ha bebido con
exceso.

---Borracho estoy, sí; pero no me trastornan las copas, sino mis
pensamientos tristes, la ansiedad en que vivo por no tener contestación
a las cartas que escribí a la prenda de mi corazón.

---Sobre eso tengo que decirte que es locura pensar en la puntualidad de
correos, mientras duren las circunstancias de revolución en tierra y
mar, y la tirantez de nuestras relaciones con el Perú. ¿Quién asegura
que tu hija recibió las cartas que le escribiste? Y si las recibió y te
ha contestado, ten por cierto que su carta quedó en el camino. Ya sabes
que nuestro correo nos llega por el Consulado inglés, y que lo recogemos
en la capitana del Comodoro Harvey.

---Por ahí viene el correo de España; pero una carta del interior del
Perú nunca pensé que nos llegara por mano inglesa.

---Pues no la esperes, Diego. Vuelve a escribir a tu hija\ldots{}

---¿A dónde, ajo?

---Al Cerro del Pasco\ldots{} Para mayor seguridad, yo iré mañana al
Chorrillo; veré a Canterac, y le preguntaré a dónde debes
escribir\ldots{} Advierte a Mara que te dirija la carta \emph{al
cuidado} del comodoro Harvey.

---¡Virgen del Carmen---clamó Ansúrez levantándose presuroso y corriendo
al camarote de Sacristá, donde comúnmente tiraba de pluma,---escribiré
al instante!\ldots{} ¡Ajo, tanto tiempo perdido!\ldots{} y ahora\ldots{}
vuelta a empezar\ldots{} Dios no me quiere ya. Tiene razón
Binondo\ldots{} Estoy lleno de pecados.

Ved aquí al pobre hombre nuevamente inmergido en la faena epistolar, que
era gozo y tormento de su alma. Pensamientos nuevos puso en el papel; su
inspiración era inagotable. Con esto se entretenía, descendiendo al
fondo de sus amarguras como un buzo que desea explorar y reconocer las
cavernas recónditas del mar\ldots{} Y en esto desfilaron unos tras otros
los días de ociosidad, y llegó uno memorable por haber aparecido en el
puerto del Callao la flota insurrecta o \emph{Restauradora}, compuesta
de las fragatas \emph{Amazonas}, \emph{América} y \emph{Unión}, al mando
de Montero. Dirigió este a los jefes de las escuadras extranjeras
oficios en que manifestaba su propósito de intimar a la plaza la
rendición; mas no le hicieron caso, que era como negar la beligerancia
que los revolucionarios solicitaban. Fondearon las fragatas junto a la
isla de San Lorenzo, donde mataban el tiempo tirando al blanco; y al
fin, desconsoladas, se fueron a las Chinchas.

Corrieron monótonos los días, y el 17 de Agosto entró en el Callao el
\emph{Marqués de la Victoria}, caballero sirviente que fue de la
\emph{Numancia} en el viaje de Montevideo al Puerto del \emph{Hambre}.
No era joven el \emph{Marqués}, y sus calderas y máquinas se resentían
del largo servicio, sin las reparaciones debidas; así es que cojeaba en
su lento andar de ocho millas. Pero si flaqueaba de los pies, no así del
corazón, y dispuesto se le vio siempre a correr nuevas aventuras, bajo
la rienda de su valeroso comandante don Francisco Castellanos\ldots{}
Salió la escuadra el 31 a efectuar un crucero de instrucción. Convenía
navegar para obtener mediana limpieza de los cascos, que en las
prolongadas estadías en aguas tropicales se llenaban de broza y
escamujo. Trasladó Pareja la \emph{Numancia} accidentalmente su
insignia; la escuadra hizo diferentes evoluciones, probando el andar a
la vela de cada buque, y a los cuatro días regresó al Callao, donde a
todos esperaban interesantes noticias traídas por el correo.
Consecuencia de ellas fue que Pareja, con todas sus naves a excepción de
la \emph{Numancia} y \emph{Marqués de la Victoria}, saliera para
Valparaíso. ¿Qué ocurría, qué determinaciones del Gobierno motivaban la
prisa con que se alistaron las fragatas de hélice para marchar a los
puertos de la República de Chile?

\emph{Camarote de Sacristá}.---Han comido juntos Sacristá, Mendaro y
Ansúrez, y de sobremesa charlan y \emph{trincan}.

{\textsc{Sacristá}}.---Os lo explicaré yo si puedo. Sabéis que en Chile
teníamos un embajador, o legado\ldots{} no sé cómo esto se llama\ldots{}
que llevaba veinte años en aquella República, con vida ociosa y
divertida. Fácilmente se van haciendo al vivir regalado los
diplomáticos, y el nuestro acabó por ser más chileno que español.

{\textsc{Mendaro}}.---He oído que don Salvador Tavira, que así se llama
nuestro Ministro en Santiago, estaba muy agarrado a los cariños
chilenos. Si el Gobierno español lo sabía, ¿por qué no lo retiró del
empleo y puso en su lugar a otro? Veo que aquí se cargan todas las
culpas a la cuenta de los americanos, y esto no es justo. Yo, español,
digo y sostengo que los políticos de allá tienen la mayor culpa de esta
guerra, por haber mandado acá sus primeros mensajeros con tanta
arrogancia, y ahora por el desacierto con que disponen todas las cosas.
¿No están conformes ustedes, españoles a rabiar, con la opinión de este
español tranquilo, que quiere vivir en paz con sus hermanos de América?
Pues lo siento. He dicho.(\emph{Bebe}.)

{\textsc{Sacristá}}, \emph{con solemnidad}.---Dejemos a un lado, amigos
míos, esos pareceres de si ha sido prudente o no el mover guerra con
estos leoncitos de América. Lo hecho, hecho está, y ya no podemos
volvernos atrás. Ese señor Tavira presentó al Gobierno chileno un pliego
de quejas, pidiendo satisfacción de los insultos a nuestro Consulado, a
nuestra bandera y a nuestra querida soberana doña Isabel II, que Dios
guarde. El Gobierno chileno contestó de mala manera, pasándose las
reclamaciones de nuestro Gobierno por semejante parte. Ello era una
guasa\ldots{} Nuestro Ministro, señor Tavira, no admitió las
explicaciones\ldots{} Pasó tiempo, y un día se levanta el hombre de buen
humor, con el mejor humor chileno, ¿y qué hace? Acepta y da por buenas
las explicaciones\ldots{} Van y vienen correos\ldots{} El Gobierno
español se llama a engaño, ¿y qué hace? Desaprobar la conducta del
Tavira y mandarle a su casa; y para llevar las cosas por derecho, nombra
Plenipotenciario al señor Pareja, dándole facultades para reclamar y
exigir las satisfacciones, primero por la buena, y si no entran por la
buena, por la mala, esto es, a cañonazo limpio. España podrá estar loca;
pero de tonta no tiene un pelo. O se le dan satisfacciones de tanto
insulto y vejámenes tantos, o sabrá sacar el pecho como corresponde a su
nombre glorioso\ldots{} He dicho. (\emph{Bebe.})

{\textsc{Mendaro}}, \emph{tamboreando en la mesa con los dedos, después
de beber}.---Tan\ldots{} taran\ldots{} tan. No me meto en si España
desenvaina su espada con razón o sin ella. Español trasplantado en
América, no entiendo bien estas cosas, y lo que quiero y pido es que la
envaine sin deshonor\ldots{} El que viene de aquel hemisferio a este, se
va dejando en las aguas los puntillos de honra. Cuando uno se establece
aquí para ganarse la vida, están muy pasados por agua los orgullos de
allá\ldots{} y esto debe España tenerlo en cuenta antes de sacar de la
vaina el espadón\ldots{} Estos países son hijos del nuestro emancipados,
harto grandullones ya para vivir arrimados a las faldas de la
madre\ldots{} y aunque sean algo calaveras, no debe la madre ponerse con
ellos demasiado fosca. Son republicanos; han roto con la historia vieja,
y se traen ellos su historia. España les dio con su sangre la picazón de
las rebeldías\ldots{} debe tratarlos con indulgencia, y no reparar tanto
en lo que dicen, que de muchachos no debe esperarse mucho comedimiento
en la palabra. En fin, este es mi parecer. Tómenlo como quieran. Soy
español trasplantado: lo que digo es mi pensamiento natural\ldots{} y
algo más que me entra por las raíces. (\emph{Bebe}.)

{\textsc{Sacristá}}.---Pronto hemos de ver grandes acontecimientos. Las
fragatas van a Caldera a tomar carbón, y la \emph{Villa de Madrid} sigue
a marchas forzadas a Valparaíso, donde nuestro General echará su
\emph{ultimatum}, que es dar un plazo para las satisfacciones. Nosotros
quedamos aquí en espera de lo que resulte de esta trifulca peruana; pero
no creo que durmamos mucho en estas aguas. Suceda lo que quiera, yo
digo: «¡Viva Isabel!» (\emph{No beben: pensativos, miran al suelo}.)

{\textsc{Ansúrez}}, \emph{después de larga pausa}.---Yo tengo mi corazón
en América\ldots{} Pero con el corazón en América, también digo: ¡viva
la Reina! Mi bandera es muy grande. Coge medio mundo, desde España al
Pacífico\ldots{} ¿Qué me dice el nombre de este mar? Pues que brinde por
Mara\ldots{} verbigracia, por la paz.

\hypertarget{xviii}{%
\chapter{XVIII}\label{xviii}}

El Chorrillo, la pintoresca playa que al Sur del Callao se extiende, era
lugar de recreo y descanso para la sociedad limeña. Allí concurrían
ricos y semi-ricos, pobres y semi-pobres en busca del trato expansivo y
ameno, de la fresca brisa, de la vida placentera, factor principal de la
vida saludable. En aquel campo de la ociosidad, donde crecían lozanas la
paz, la higiene, la cortesía graciosa y alegre, no podía faltar la
planta viciosa y viciada del juego. Formidables timbas actuaban en
garitos elegantes, donde la juventud florida y la vejez verde exponían
inmensos caudales de oro a la fatalidad del azar. Allí las fortunas
improvisadas con la venta y embarque del guano, pasaban en horas al
bolsón de los banqueros del envite. Como en aquel tiempo la riqueza
principal del Perú procedía de los yacimientos de las Chinchillas, podía
decirse que en las mesas de juego del Chorrillo pasaba de unas manos a
otras lo que las aves oceánicas habían depositado durante siglos y
siglos. Allí dejó cuanto tenía, y hasta las plumas del tricornio, un
altísimo personaje de aquel tiempo, culminante figura militar, política
y revolucionaria, que ni en las postrimerías de su edad achacosa pudo
curarse del funesto vicio. Los años y su jerarquía social dábanle
derecho a una sinceridad chistosa. Cuando le agraciaba la suerte, decía:
«hoy he ganado yo.» Cuando venía la mala: «hoy ha perdido el Perú.»

En ocasiones diferentes obtuvo Fenelón permiso de dos o tres días, que
se pasaba tranquilamente en el Chorrillo gozando de aquella excitante
vida. Vestido con elegancia y hablando francés, mariposeaba en
diferentes casas y familias, sin que nadie sospechara que estaba al
servicio de la Marina española. Por vanidad tanto como por vicio
dejábase caer en la timba, donde era comúnmente desplumado. Un día que
le sonrió la fortuna, se fue a Lima, y en la mejor fotografía de la
ciudad compró una colección de retratos de mujeres, que era el más
variado y sugestivo muestrario de las hermosuras limeñas. Debe
advertirse que en Lima las señoras y señoritas gustaban de ostentar
públicamente su belleza en las vitrinas de los fotógrafos. Esta liberal
costumbre, que debieran imitar las beldades de otros países, no tenía
nada de particular. Lo insólito y raro era que los fotógrafos vendiesen
al público los retratos de todo el mujerío de la ciudad, y que nadie se
ofendiese por esto. Nuestros Oficiales y Guardias marinas, privados del
trato y contemplación viva del bello sexo, se consolaban adquiriendo las
preciosas imágenes. Algunos hacían entre sí cambalaches de ellas, y a
fuerza de contemplarlas y de discutir y comparar los diferentes tipos de
belleza, llegaban a darles personalidad y aun a ponerles nombres: María,
Carmen, Gracia, Lolita, etc\ldots{}

Las cartulinas que llevó Fenelón, como escogidas por su buen gusto, eran
primorosas. En su esfera jerárquica, que era la de oficiales y cabos de
mar, condestables y mayordomos, enseñó la preciosa colección de niñas
bonitas, describiéndolas con acertado criterio estético, y agregando
indicación de las cualidades morales, virtudes o defectillos de cada
una. De este modo, sin declarar que eran sus conquistas, dejábalo
entender; y cuando sobre esto se le interrogaba, se hacía el modesto y
el delicado, y a sus amigos pedía que no pusieran a prueba su extremada
discreción.

De su tercera visita a las timbas del Chorrillo volvió Fenelón con la
bolsa limpia como patena; mas del percance se consolaba con su filosofía
parda y la gramática del mismo color, asegurando que era rico con la
ilusión de un próximo desquite. Días antes de la catástrofe había hecho
corta provisión de vino blanco, parecido a Jerez de poco cuerpo, con lo
que podría remediarse hasta que vinieran tiempos mejores. Convidó a
Sacristá y a Diego a que lo probasen, y estando en ello se dejó caer por
allí Binondo, encorvado y tétrico. Antes de que rompiera en místicas
declamaciones y en el elogio de los santos, le taparon sus amigos la
boca. Invitáronle a probar el vino; defendió con remilgos sus propósitos
de abstinencia; al fin cedió a los ruegos insistentes, y copa tras copa,
llegó a la cuarta, donde hizo punto con extremado escándalo de su
conciencia. Fenelón y Sacristá le tranquilizaron, diciéndole que porque
llegase borracho al Cielo, no habrían de recibirle con menos agasajo del
que merecía.

Ansúrez bebió doble que Binondo, y cuando estaba en la cuarta copa, le
dijo Fenelón poniéndose muy serio y tomando una actitud parlamentaria:
«Tengo que comunicarte un suceso de los que deben ser celebrados entre
amigos con toda solemnidad\ldots{} He querido haceros beber antes de la
noticia, para que con lo que después se beba quede la noticia entre dos
luces espléndidas\ldots{} Veo a todos con la boca abierta, y a Diego con
los ojos saltones y cortada la respiración. Lo diré de una vez\ldots{}
Bebamos a la salud del Oficial de mar y de su ilustre parentela
incaica\ldots{} Ansúrez, abrázame: ya eres abuelo\ldots{} Tu hija\ldots»

---¡Ajo!\ldots{} ¿pero es verdad?

---Mara ha dado sucesión a la regia familia de los Chacones\ldots{} ¿No
te alegras?

---¡Sí me alegro, ajo!---exclamó Ansúrez con llanto y risa que se
peleaban en su rostro.---Es que la sorpresa me ha dejado lelo\ldots{} Me
vuelvo criatura, como si fuera yo nieto de mí mismo. ¿Con que un
hijo\ldots{} y varón? ¡Jesús, qué lindo será\ldots{} y además poeta por
parte de padre!\ldots{} ¿Y mi hija, está bien? En el trance apretado, se
portó como buena española. Me atrevo a sostener que apretó los dientes
para no chillar\ldots{} ¡Valiente como ella sola! ¡Hija del
alma!\ldots{} ¿Qué dices a esto, Binondo?

---Digo que no es verdad---replicó el malayo.---Yo lo he soñado de otro
modo, al modo triste, que siempre es el más verdadero. Verdaderas son
siempre en sueños las visiones del morir; las del nacer no lo son. No
creas, Diego, el cuento de este señor, y ten por seguro que no tienes
hija, ni tampoco nieto, porque antes que ella pudiera dar el ser al ser
del chiquitín, ambos seres dejaron de ser.

Montó en cólera el buen celtíbero al oír esta disparatada sutileza, y
sin poder reprimirse cerró el puño y alzó el brazo con tal violencia y
furia, que si los amigos no atajaran el movimiento, aplastado quedaría
el cráneo de Binondo. «Repórtate---dijo este;---sé buen cristiano,
Diego; aprende la humildad, la resignación, y hazte más amigo de la
tristeza que de la alegría, más del padecer que del gozar.»

---Cállate, fealdad; vete con tus músicas negras a otra parte---gritó
Diego,---y déjanos a los que consolamos nuestras almas con algún rayito
de alegría que Dios manda\ldots{} En fin, no quiero incomodarme\ldots{}
hoy es día de paz, de bailar de gusto y de echar la casa por la ventana.
Venga otra copa. Bebe a mi salud, José, y que Dios te conceda pronto la
muerte que deseas.

Bebió Binondo, limpiándose con la mano la boca en toda su longitud
monstruosa; dijo \emph{amén}, y agarrándose a los mamparos salió con la
lentitud que le imponía su dolencia cardiaca. Apenas desapareció el
malayo, Ansúrez, que no cabía en sí de gozo, pidió a Fenelón pormenores
del fausto suceso. Díjole el francés que la noticia era tan cierta,
\emph{por ejemplo}, como la luz del sol; que el alumbramiento había sido
felicísimo; que el chiquillo era una preciosidad, la madre un portento,
y que doña Celia y don Belisario estaban a punto de enloquecer de
júbilo.

Para que Diego se persuadiera de la verdad del caso, y se disiparan las
últimas sombras de su duda, aseguró Fenelón que le presentaría dentro de
poco una prueba documental irrecusable. ¿Qué prueba, Señor? Pues\ldots{}
Belisario había compuesto una larga y sonora poesía, titulada \emph{Al
nacimiento de mi primer hijo}. Imprimiéndola estaban en Jauja, pues en
el Cerro del Pasco no había buenas imprentas. Con la poesía del feliz
padre recibiría Fenelón otras muchas en variados metros y estrofas,
escritas por los poetas y poetisas de aquella localidad y sus contornos,
y dedicadas al venturoso natalicio del nene de Chacón. ¡Extraño y nunca
visto caso! Los versos, hijos de la fantasía, venían en auxilio de la
razón, y daban testimonio y fianza del hecho real. Los tres amigos
alzaron de nuevo las copas; Sacristá puso su mano cariñosa en el hombro
de Ansúrez, y en su oído estas nobles palabras: «Lo que tú dices:
nuestras bocas gritan \emph{guerra}, y nuestros corazones gritan
\emph{paz.»}

En esto llegó al camarote el Capellán don José Moiron, y antes de tomar
la copa que le ofrecían, desembuchó estas graves noticias: «Ya hemos
declarado a Chile la guerra\ldots{} Ya la revolución del Perú está en
camino del triunfo.» Queriendo poner un comentario a la primera de estas
interesantes nuevas, el buen castrense, modoso y encogidito como un
Capellán de monjas, echó de su boca esta exclamación pagana: «Séanos
propicio el Dios de las batallas.» Y Ansúrez, comentando la segunda
noticia, dijo: «Pues si como hay Dios de las batallas, hay Dios de las
revoluciones, no le arriendo la ganancia al Presidente Pezet.»

El caso era que no habiendo podido obtener del Gobierno chileno las
satisfacciones pedidas en el \emph{ultimatum}, Pareja declaró que las
pediría con el lenguaje de las armas. Metiéronse por medio los
diplomáticos, buscando arreglo; pero la obstinación de los chilenos
cerró el camino a toda solución pacífica. El primer acto militar de
Pareja fue disponer el bloqueo de los puertos de Chile. A los buques de
banderas neutrales se les concedía plazo de diez días para que salieran
cargados o en lastre de los puertos de la República. Las fragatas
\emph{Villa de Madrid}, \emph{Resolución} y la goleta \emph{Vencedora},
sostenían el bloqueo en Valparaíso; la \emph{Berenguela} en Coquimbo, y
la \emph{Blanca} en Caldera. Apresaron cuantos buques chilenos andaban
por aquellas aguas, casi todos de cabotaje, pues el comercio de altura
se hacía principalmente en buques extranjeros.

Llegaron estas noticias por el correo del Sur, y con ellas innumerables
periódicos que ponían a los españoles cual no digan dueñas. Con la prosa
furibunda se mezclaban los versos: las musas que en aquellos países
florecen reventaban de tanto soplar la bélica trompa. Todo esto era muy
natural, y nuestro Almirante y Plenipotenciario no debió incomodarse por
tal efervescencia del patriotismo y de la versificación, cosas ambas que
compiten en lozanía con la flora americana.

«Señores---dijo Ansúrez, en cuyo ser celtíbero resplandecía la
equidad,---yo pienso, con perdón, que el señor Pareja no estuvo discreto
al mandar a los chilenos el memorial de agravios el mismo día en que
celebraban el aniversario de su independencia. Señores, cada país tiene
sus cariños y sus memorias alegres o tristes de sucesos pasados. El Jefe
de Escuadra\ldots{} lo digo con todo respeto, en cuanto oyó ruidillo de
cohetes y escandalera de patriotismo, debió echarse mar afuera con todos
sus barcos, y cruzar un par de días, para volver luego cuando estuvieran
ya roncas y cansadas las voces patrioteras\ldots{} Y entonces era la
ocasión de decirles: `Ea, caballeros, ya ven que les he dejado desahogar
los corazones. Ahora vamos a tratar de nuestro asunto, poniéndolo en los
términos de la razón'. Y esto y lo otro, y vengan explicaciones, y vaya
indulgencia para pedirlas, sin exigir demasiado, con cierto tira y
afloja, como hace la madre cariñosa que reprende al hijo calavera, sin
olvidar nunca que es madre\ldots{} Esto me parece a mí que debió hacer
nuestro General; y si es disparate, no hagan caso\ldots{} que yo no soy
quién para tratar de estas cosas; pero digo todo lo que me sale del
cacumen de mi sentido natural\ldots»

Ni Sacristá ni el Cura apreciaron en lo que valía esta opinión sesuda,
que sólo fue apoyada por el francés maquinista. Ello es que los
españoles necesitaban de una fuerza grande de virtud para no dejarse
inflamar por el rencoroso fuego que contra ellos enviaban los
americanos. El correo del Sur traía, con las noticias de la declaración
de guerra y el fárrago de versos patrióticos, un clamor inmenso y
unánime que pedía la coalición del Perú y Chile contra el maldito
\emph{godo}; clamor que más bien iba buscando el convencimiento fácil
del partido revolucionario que el del Gobierno del Presidente Pezet.
Casi juntamente con las noticias del furor chileno, llegó a bordo de la
\emph{Numancia} la del desembarco de cinco mil insurrectos en Pisco, al
mando del Vicepresidente y General Canseco, y del Coronel Prado. Se
situaron en Paracas, disponiéndose a marchar sobre Lima, distante
cuarenta leguas. Pronto se supo que Pezet reunía un ejército de diez mil
hombres, y salía de la capital y tomaba posiciones en los llanos de
Lurín. Arrojados quedaban ya los dados.

Mala la hubisteis, españoles, con aquellas trifulcas de vuestros
parientes americanos, y malísima la hubo también el bonísimo Ansúrez,
que apenas acarició las dulces esperanzas de comunicarse con su hija,
viose nuevamente defraudado y a punto de volverse loco, porque el
Comandante no permitía bajar a tierra, temeroso de conflictos y choques,
provocados por la turbamulta de Lima y el Callao. Valiéndose de los
rancheros y de su amigo Mendaro, envió Diego a tierra una carta que
debía confiarse a los buenos oficios del señor Canterac, para quien dio
el maquinista una esquela de recomendación. Pero la epístola volvió a
bordo con el recado triste de que el señor Canterac no estaba en Lima:
había ido al bateo del herederito de los Chacones, y se ignoraba cuándo
volvería.

Y ya tenemos otra vez a nuestro buen amigo dedicado a la imitación santa
del Patriarca Job, de quien se creía discípulo en paciencia, aunque casi
casi iba ya para maestro. Sirviole de solaz y consuelo en aquellos
tristes días la mediana carga de versos que le dio Fenelón, y fue
remitida por una amiga de este. Era el Florilegio del Natalicio, y en él
figuraba como pieza mayor la composición de Belisario, en silva; seguían
innumerables octavas, décimas, quintillas, romances, cantatas y otras
formas de poesía, que ensalzaban con entusiasmo ardiente el familiar
suceso, subiéndolo hasta las mismas barbas de la Historia. Aunque
Ansúrez no entendía ni palotada de poesía, ni en su vida las había visto
más gordas, todo lo leyó y releyó sin perder sílaba, gozando en la frase
sutil, en el número y cadencia, en el sonsonete de las rimas. La
exuberancia de los ripios, a gloria le supo. Admiraba los privilegiados
caletres que daban de sí tan bellos pensamientos, y los reducían a un
lenguaje que era sin duda el idioma vulgar de los serafines. Los
renglones largos y cortos de Belisario, en combinación musical, le
sonaban como una orquesta que imitara el rumor de la marejada, los
golpetazos de la hélice y las caricias de un Nordeste frescachón. Los
otros versos también eran bonitos. ¡Qué comparaciones, qué galanas
frases y qué melindres cariñosos!\ldots{} ¡Y qué cosas le decían a la
hermosa Mara! ¡Ajo, vaya una lluvia de flores!\ldots{} \emph{La perla
española}\ldots, \emph{la flor de Castilla}\ldots, \emph{la paloma
emigrante}, \emph{que en alas del amor}\ldots{} En fin, que había hecho
su nido \emph{a la sombra de los Andes}.

\hypertarget{xix}{%
\chapter{XIX}\label{xix}}

Las revoluciones americanas se parecían a las nuestras como una castaña
nueva a una castaña pilonga. Sus incidentes y desarrollo, su desenlace
infeliz o venturoso, eran casi siempre los mismos; sus héroes, ya
coronados del éxito, ya hundidos en la derrota, llevaban en su conducta
y lenguaje los propios caracteres. Resulta, pues, para nosotros el
relato de la revolución peruana en 1865 como un amaneramiento
histórico\ldots{} Clío se ve obligada a contar, con formas gastadísimas,
sucesos ya conocidos por su lamentable repetición. Será preciso referir
con trazo nervioso y rápido los acontecimientos que arrojaron de la
Presidencia al General Pezet, para poner en su lugar al General Canseco.
Fuera de la escaramuza naval en aguas de Pisco, la revolución no
presentó ninguna originalidad, ni dejó de amoldarse a los precedentes
que para uso de los pueblos ibéricos archiva la Historia de esta
Península.

Mientras los dos caudillos se iban acercando con parsimonia, y alzaban
las cortadoras espadas queriendo renovar la pelea entre don Quijote y el
Vizcaíno, los pueblos se amotinaban aprovechando la debilidad de las
guarniciones y el desequilibrio de aquellas autoridades tambaleantes,
que tenían un pie en la legalidad y pie y medio en la rebeldía. La
República chilena, interesada en celebrar con el Perú pacto de odio
contra España, atizaba candela en favor de Canseco, y valiéndose de
hábiles agentes, laboraba en la capital y en su puerto, así como en las
ciudades del Norte. Lima era un campo de continuos desórdenes, y en el
Callao saltó un motín seguido de saqueo, que fue la página más movida de
aquel drama de escaso interés.

En esto, el bueno de Pezet y el arrogante Canseco renunciaban a toda
semejanza con don Quijote y el Vizcaíno; y poniendo hielo en la furia de
sus primeras amenazas, envainaron los aceros. No tiene explicación la
conducta de Pezet, que, dueño de excelentes posiciones, primero en
Lurín, después en Bella Vista, dio media vuelta a la izquierda y acudió
a embarcarse en una corbeta inglesa. En tanto, Canseco daba media vuelta
a la derecha y caía sobre Lima, donde hubo de luchar con dos militares
tercos que sabían su obligación: era uno el Ministro Gómez Sánchez, y
otro el Coronel Sevilla. Pero, al fin, la fuerza y el número imperaron.
Quedó Canseco dueño de Lima, con el nombre de \emph{libertador}, entre
el delirio y espasmos patrióticos de la muchedumbre; y para completar el
amaneramiento del desenlace, siguieron las fiestas, los escándalos, las
libaciones y atropellos, que en esta clase de cambios políticos suelen
ser el fin de las alegrías y el comienzo de las dificultades.

Desde la \emph{Numancia} pudieron los españoles echar un vistazo fugaz a
la revolución, que por sí y por sus hechos interiores sólo debía
moverles a curiosidad. Por sus consecuencias internacionales les movía
quizás a mayores inquietudes. La situación a bordo era de incertidumbre
y zozobra. Gran número de familias se habían refugiado en barcos
mercantes españoles. Con estos se comunicó Méndez Núñez, ofreciendo a
los prófugos amparo más seguro si fuera menester. La hostilidad entre la
plaza y la fragata era cada día y a cada hora más ostensible. De tierra
venía un aire de cólera que daba en el rostro a los tripulantes de la
fragata. Habrían sido rostros de mármol si no respondieran a las
demostraciones airadas con fruncimiento de cejas por lo menos. Cada cual
tiene su alma en su almario.

Una profecía de Fenelón, hecha por aquellos días en círculo de
camaradas, daba la medida de su mundología y agudeza. Dijo el
hispano-francés que una vez exaltado Canseco a la Presidencia, se había
de ver entre la espada y la pared, entre la realidad del gobierno y los
compromisos que había contraído para encender y arrastrar a las
muchedumbres. El revolucionario tenía que darse de cachetes con el
hombre de Estado, porque aquel lanzó a la populachería la idea de anular
el arreglo con España, calificándolo de ignominioso, y este se veía
forzado, por ley de conservación, a librar a su país de los azares y
quebrantos de la guerra. Así sucedió, en efecto: Canseco inauguró su
presidencia con ejercicios de consumado equilibrista en la cuerda floja.
Había predicado la guerra. ¿Cómo predicar ahora la paz? Largos días
emplearon en negociaciones el Ministro de Estado y nuestro
Representante, señor Albistur, repitiendo los equilibrios del
Presidente. Este inventaba fórmulas, obras maestras de
pastelería\ldots{} Pero no hubo manera de oponerse a la efervescencia
popular, atizada por los agentes chilenos, de prodigiosa actividad y
travesura. Tanto empujó la ola del partido belicoso, formado casi
exclusivamente de militares, que al fin Canseco hubo de comprender cuán
expuesta es a quebrantos la pastelería política, y obligado se vio a
resignar el mando y Presidencia. En su lugar, los revolucionarios,
asistidos de los agentes chilenos, elevaron al Poder supremo al Coronel
Prado, con el nombre de \emph{Dictador}. El nombre no más tenía y la
estampa corpórea, que la verdadera cabeza dictatorial era Gálvez, hombre
impetuoso y sugestivo, que con la brillantez de sus ideas y la
exaltación de su antiespañolismo circunstancial, se llevaba consigo a
toda la juventud peruana.

Desvanecidas con la dictadura las esperanzas de concordia, la situación
de la \emph{Numancia} era bastante crítica. En aguas del Callao la
retenía el cuidado de nuestros compatriotas, guarecidos en barcos
mercantes, el acopio de provisiones para sí y para los demás buques, y
la observación de los movimientos y planes del pueblo, que ya se
mostraba como resuelto enemigo. Evidente era ya que el Callao quería
fortificarse. A los oídos españoles llegaban los proyectos de baterías
formidables, de cañones potentes\ldots{} Más que estas amenazas,
ofendían a los españoles las demostraciones de hostilidad negativa. Los
peruanos no querían dar víveres, regateaban el agua\ldots{} La
incertidumbre y el recelo entristecían la vida de todos los tripulantes.
Se doblaron las guardias; se extremó la vigilancia; se temía, no sin
fundamento, el acecho de las naves americanas. Lanzadas las
imaginaciones al campo de las conjeturas, se hablaba de unos artificios
llamados \emph{torpedos}, imitación del pez de este nombre, que,
dirigidos sin ruido a larga distancia, explotaban dentro del agua y
podrían destruir traidoramente el barco más poderoso. Por esto, y por
creer que era conveniente acudir a reforzar el bloqueo de los puertos de
Chile, la \emph{Numancia} levó anclas el 5 de Diciembre y puso proa al
Sur, llevando a remolque a su galán \emph{Marqués de la Victoria}, que
dolorido de los pies y quebrantado de las coyunturas, no podía dar un
paso. Delante salieron, cargados de carbón y provisiones, los dos
transportes \emph{Vasconga} y \emph{Valenzuela}. ¡Adiós, Callao; adiós,
Lima hermosa; adiós, ingratas limeñas! Un hado maligno y burlón nos hizo
enemigos. Maldito sea.

Navegó hacia Chile la fragata con mar bellísima y sosiego delicioso del
viento. El Pacífico parecía inmenso lago, o un estanque sin fin; la
atmósfera, limpia y transparente, permitía contemplar la majestad de los
Andes. Tanta serenidad contrastaba con la expectación de los navegantes,
que por secreteo misterioso del alma presagiaban alguna desdicha
escondida en el fondo de aquella mansedumbre soberana del cielo y la
mar. Seis días duró el navegar calmoso, con placidez acompasada y
rítmica, marcada por las vueltas de la hélice.

Dos hombres no más había en la fragata que, recogidos en su vida
interior, se aislaban de las preocupaciones comunes a toda la
tripulación. Eran Binondo y Ansúrez. El primero, bajo la acción
deprimente de sus achaques, e incapaz de todo trabajo corporal,
zambullía su espíritu en la lectura, y ya llevaba medio devorada, aunque
no digerida, la biblioteca del Capellán, compuesta de dos o tres docenas
de libros. Después de consagrar dos horas al \emph{Año Cristiano},
picaba en el \emph{Sermonario} y en un tratado de Teología; por fin, le
metía el diente al \emph{Genio del Cristianismo}, al \emph{Perfume de
Roma}, a las \emph{Ruinas de mi Convento}, y a otros volúmenes tan
entretenidos como piadosos\ldots{} El continuo leer y el meditar en lo
que leía, le iba poniendo en comunicación familiar con lo infinito, y su
cara plana y cadavérica revelaba un desprendimiento gradual de las cosas
terrenas. La vida interior de Ansúrez era de un orden muy distinto y
puramente imaginativa. Su pasión paternal, llevada al último grado de
exaltación por el nacimiento del nietecillo, de que daban testimonio los
retumbantes versos, tomaba en la soledad formas de delirio, y a sí
propio se engañaba, construyéndose interiormente un simulacro de la
realidad. Era la imitación a veces tan perfecta, que Ansúrez no dudaba
de la autenticidad de lo soñado. Sin desatender a sus obligaciones,
entregábase el hombre a una solitaria labor de vida imaginada, trajín
muy propio de mareantes, apartados del mundo en largas travesías.

Desde que supo la existencia del pequeñuelo, en él puso el celtíbero
todos los ardimientos de su corazón, tan dispuesto al amor de familia.
Su familia era Mara; mas un destino cruel le vedaba su presencia. El
amor conyugal y los afectos de su nueva parentela la retenían como
prisionera en regiones distantes. Del chiquillo, en cambio, pensaba
Ansúrez que le pertenecía más que la madre. Viéndole con el poderoso
cristal de su imaginación, llegó a construir caprichosamente sus lindas
facciones, su angélica sonrisa y sus donosas travesuras. Por misteriosa
ley divina, aquel niño amaba a su abuelo más que a sus padres: con esto
se creía compensado de tantas fatigas y tristezas. Así, cuando se
aproximaba al puerto de Caldera, ya llevaba Diego varias noches con el
niño a su lado, y aun de día imaginaba intensamente la presencia de la
criatura llevándola en brazos de un lado para otro. Si se pudiera dar
forma visible a tan extraordinaria ficción de la realidad, resultaría el
buen Ansúrez la perfecta imagen de San José, suprimida la vara de
azucenas y cambiado el traje bíblico por el uniforme de diario de un
Contramaestre.

Y en este imaginar ardoroso, Ansúrez no hacía caso del tiempo, ni lo
tenía en cuenta para nada. El día anterior había llevado en sus brazos
al nieto, figurándoselo en una edad como de año y medio, ya destetado,
avispadillo y juguetón. Pues bastó un lapso de veinticuatro horas para
que lo tuviera consigo en edad de más de tres años, con gorrita de
marinero, ya muy parlanchín, sin dar paz a su media lengua deliciosa.
¿Dormía el hombre?, ¿soñaba despierto? Esto era lo más aproximado a la
verdad. Ignorante del nombre que pusieran al chiquillo, él se había
permitido dárselo a su gusto. Llamose, pues, \emph{Carmelo}, como traído
al mundo bajo la protección de la Virgen del Carmen. El delirio del
Contramaestre llegó a suponer que su hija le enviaba el chiquillo con
estas cariñosas expresiones trazadas en una carta: «Ahí lo tienes,
padre; llévatelo, para que navegando te entretengas con él.» Nada más
decía; pero era bastante.

En brazos lo cogía, y su primer cuidado era enseñarle la soberbia
embarcación: le mostraba todo, como le mostraría un fabuloso y
complicado juguete que acababa de comprarle. «Vamos, hijo, por aquí, y
verás qué bonito es esto. Te gustará mucho. Pues todo es para ti, para
que juegues, para que juguemos los dos y nos divirtamos mucho\ldots{}
Vamos\ldots{} pasemos bajo el puente\ldots{} Esto es el Alcázar\ldots{}
Entremos por esta puerta. ¿Ves qué bonita cámara?\ldots{} Aquí viven los
principales del barco\ldots{} Entremos más: allí está el camarote del
Comandante, que se llama don Casto\ldots{} No podemos pasar: el
Comandante nos reñiría\ldots{} a ti no, a mí sí\ldots{} porque aunque
nos quiere mucho, por encima de su cariño está la ordenanza. Salgamos
ya\ldots{} Vamos\ldots{} Por esta escala bajaremos a la batería\ldots{}
¿Ves qué preciosa es la batería? Mira cuántos cañones: aquí uno, y
siguen otro y otro, asomados a las portas para ver la mar y los
peces\ldots{} Estos cañoncitos los dispararás tú cuando quieras\ldots{}
Mi niño no se asustará del ruido. Vamos hacia proa\ldots{} ¿Qué te
parecen estas cadenitas? Son las de las anclas\ldots{} Puedes echar y
recoger el ancla cuando quieras\ldots{} Vamos ahora a ver la máquina.
Nos asomaremos por aquel agujero\ldots{} Verás, verás qué cosa tan
bonita. Mira cómo relucen las piezas de acero, y cómo suben y bajan
aquellos vástagos, y qué ruido hace todo, como si estuvieran aquí dando
patadas contra la quilla cuatrocientos mil caballos de tierra o de mar.
Aunque sé que no te dará miedo bajar a la máquina, no bajaremos, porque
nos pondríamos perdidos\ldots{} Sigamos\ldots{} allí tienes, a popa, el
comedor de Oficiales\ldots{} Vámonos ahora al otro sollado\ldots{} Por
esta escalera bajaremos\ldots{} Ya estamos abajo. Allí\ldots{} a proa
tienes nuestro dormitorio; más allá tenemos un pañol, donde guardamos
nuestra comidita. Aquí, a los costados de babor y estribor, duerme la
tropa\ldots{} se arman y se desarman las camas\ldots{} Sigamos: comedor
de maquinistas\ldots{} y a popa dormitorio de oficiales\ldots{} Bajemos
ahora al otro sollado, que tú no tienes miedo\ldots{} Está un poquito
obscuro\ldots{} Detrás de este mamparo ¿qué hay?, las carboneras\ldots{}
Aquí tienes la enfermería de guerra\ldots{} Esto que pisamos es la
cubierta de los aljibes\ldots{} más allá, despensa, pañoles\ldots{}
¿Quieres que bajemos más? Pues vamos, que el nene no se asusta, y quiere
verlo todo\ldots{} Ea, ya estamos en lo más profundo\ldots{} Por aquí,
por aquí\ldots{} Estamos ahora en el pañol de la pólvora, que llamamos
Santa Bárbara\ldots{} Hacia aquel lado, cartuchos, balas\ldots{} Aquí
podrás jugar todo lo que quieras, y pegar fuego a la Santa
Bárbara\ldots{} con lo que brincaremos todos hasta el cielo\ldots{} Ea,
volvamos arriba, que aquí hace calor\ldots{} ¡Arriba, upa!\ldots{} Ya
estamos otra vez sobre cubierta\ldots{} ¡ajajá! ¡Qué hermoso el
cielo\ldots{} qué soberbia la embarcación! Allí tienes a nuestro amigo
Sacristá, que nos mira y se ríe\ldots{} ¡Ah, pillo!, ya iremos a tirarte
de una oreja\ldots{} Vaya, niño mío, ¿quieres que te suba a la cofa de
trinquete? ¿No te asustarás?\ldots{} Pues si te atreves, subamos.
Conmigo vas tan seguro como si el mismo San José te llevara. Arriba por
la escala del obenque\ldots{} Ajajá\ldots{} Ya estamos arriba. De aquí
sí que se ve bien tu juguete y la mar\ldots{} ¿Ves qué grande, qué
grande? ¿Qué te parece este sin fin de cabos y la largura de las vergas?
Puedes desde aquí jugar todo lo que quieras, y largar y aferrar las
gavias y juanetes a tu satisfacción\ldots{} Mira para el otro lado, niño
mío\ldots{} Allí tienes los Andes\ldots{} ¿Verdad que son
altísimos?\ldots{} Algunos montes de esos son volcanes\ldots{} y tienen
dentro mares de fuego\ldots{} Yo te llevaría con gusto hasta el pico más
alto para que vieras toda la América de la otra banda, y los ríos que
llevan sus aguas al Paraná y al Uruguay y al Plata\ldots{} Todo eso es
España, otra España, ¿te vas enterando?\ldots{} Háblale, salúdala con tu
manecita, y con tu media lengua dile que la quieres mucho, que estás
aquí con tu abuelito, y que también tu abuelito la quiere\ldots{} Bueno:
pues ahora mira para el cielo, niño querido. ¿Ves esa nube que tapa el
sol? No es nube: es una inmensa bandada de pájaros. Míralos bien, verás
que son miles de miles de aves. Vienen de alta mar, donde han comido
peces, y ahora se retiran a las peñas de tierra\ldots{} Se llaman
\emph{piqueros}, \emph{sarcillos}, \emph{gaviotas},
\emph{alcatraces}\ldots{} Traen en sus estómagos mucho dinero, pues el
guano lo es\ldots{} es oro y plata\ldots{} Mira, mira cómo la bandada,
al aproximarse a tierra, se divide en escuadrones, en compañías\ldots{}
Cada familia se va a su casa, y cada pareja busca su nido\ldots{} Ea,
bajemos, que hace ya demasiado fresco\ldots» Terminada esta visión,
empezaba otra; y a medida que las iba produciendo, el celtíbero
celebraba con sonrisa del alma sus propios disparates.

\hypertarget{xx}{%
\chapter{XX}\label{xx}}

Al aproximarse a la ensenada de Caldera, Méndez Núñez, en el puente con
el Oficial de derrota, reconoció con su anteojo las fragatas \emph{Villa
de Madrid} y \emph{Berenguela}; luego vio los mástiles de los
mercantones apresados\ldots{} No le sorprendió encontrar la
\emph{Berenguela}, que había relevado a la \emph{Blanca} en el bloqueo
de aquella zona; pero sí ver a la \emph{Villa de Madrid}, y aún fue
mayor su sorpresa cuando advirtió que esta no arbolaba la insignia de
Jefe de Escuadra, y en cambio, en la \emph{Berenguela} flameaba el
gallardetón de Capitán de Navío. ¿Qué había ocurrido? Diferentes
conjeturas pasaron rápidas por la mente del Comandante de la
\emph{Numancia}, y las visiones de desdichas se sucedieron con la
fecundidad pesimista de nuestra imaginación, que a veces las exagera y
abulta con la idea de que resulte menos fuerte la desdicha real, al ser
conocida\ldots{} Pronto saldría de dudas\ldots{} Era don Casto Méndez
Núñez de estatura mediana tirando a corta, recio y bien plantado. Sobre
su rostro moreno vagaba siempre, en ocasiones ordinarias, un mirar dulce
y una vaga sonrisa. Su voluntad de hierro no era de las que tienen por
muestra al exterior un entrecejo duro, ni su voz, robustecida en las
conversaciones con el viento y la mar, llegó a perder las blandas
inflexiones gallegas\ldots{} Quedó, como se ha dicho, con el alma
suspensa de un enigma cuya solución esperaba, y la atención presa en los
topes de las dos fragatas. Los de la una, por arbolar insignia, algo le
decían; los de la otra, por no tenerla, le decían más.

El Segundo, don Juan Bautista Antequera, ocupaba su puesto a proa,
atento a la maniobra de dar fondo. Saludó la fragata con siete cañonazos
la insignia de Capitán de Navío; contestó la \emph{Berenguela}; y apenas
disipado en vagos jirones el humo, se vio desde el puente que del buque
insignia venía un bote hacia la \emph{Numancia}. Echose a la cara Méndez
Núñez los anteojos, y al ver que el bote traía la visita del Capitán de
Navío, don Manuel de la Pezuela, su asombro fue extraordinario. Con toda
su curiosidad y todo su asombro a cuestas, Méndez Núñez bajó al portalón
para recibir al visitante\ldots{} La clave del estupor de don Casto nos
la da un hecho, de estos que sin estar consignados en los libros de
Historia, a ella pertenecen por el tributo que la vida particular paga a
la vida pública cuando menos se piensa. Antes de que la \emph{Numancia}
saliera de Tolón, era su Comandante Pezuela, amigo y protegido del
Ministro de Marina, General Armero. Lista la fragata blindada para
prestar servicio, y destinada a la campaña del Pacífico, elegido fue
inopinadamente don Casto Méndez Núñez para mandarla y conducirla en tan
larga navegación, nunca intentada por naves de tal porte y pesadumbre.
Las razones que tuvo el Ministro para este nombramiento no debían
deprimir a Pezuela, que gozaba de buen crédito como navegante y militar;
pero le amargaron enormemente. Debemos considerar que el enojo de
Pezuela se fundaba en un noble sentimiento, la emulación, alma de los
cuerpos armados de estructura aristocrática.

El caso fue que desde el día en que la \emph{Numancia} cambió, como si
dijéramos, de galán o de novio, Pezuela y Méndez Núñez no volvieron a
dirigirse la palabra. Al primero se le dio el mando de la
\emph{Berenguela}, novia que ni por su edad ni por su belleza podía
competir con la que le quitaron en Tolón, y fue al Pacífico en la
escuadra de Pareja; el segundo emprendió después su viaje de leyenda con
la \emph{niña bonita}. Cuando esta llegó al Callao victoriosa,
desmintiendo los augurios pesimistas de los técnicos, los dos rivales no
cambiaron ninguna demostración de amistad en todo el tiempo que
permanecieron en aguas peruanas. Si Pezuela visitó en la \emph{Numancia}
al segundo de esta, don Juan Antequera, fue en ocasión de estar en
tierra Méndez Núñez pagando la visita oficial\ldots{} Por la feliz
realización del viaje, ascendió Méndez Núñez a Brigadier de la Armada;
Pezuela seguía en su empleo de Capitán de Navío\ldots{} Todo esto que
brevemente aquí se cuenta, pesó en la mente de don Casto cuando hacia el
portalón bajaba. Era hombre tímido, y la situación que se le presentaba
después del largo eclipse de amistad con Pezuela, le ponía nervioso y
cohibido. Viéndole subir por la escala, pensó que su rival despejaría el
nublado con breves palabras. Así fue.

«Mi General---dijo Pezuela con grave cortesía, estrechando la mano de
Méndez Núñez,---vengo a saludarle y a resignar en usted el mando de la
escuadra que accidentalmente he tomado, y que a usted por su graduación
corresponde. Ha muerto Pareja\ldots»

A la interrogación de pena y asombro, expresada por don Casto con la
mirada y el gesto, más que con la palabra, contestó así Pezuela: «Tengo
mucho que contarle, mi General. Por de pronto, acepte usted para esta
empresa, que se nos presenta obscura y difícil, la cooperación de todos
mis compañeros y la mía particularmente. Estamos a tres mil leguas de
España, con su honor y su bandera entre las manos\ldots{} Miremos tan
sólo a sacar avante estos grandes intereses, y olvidemos todo lo
demás\ldots» Con estas caballerescas expresiones, puso Pezuela a los
pies de Méndez Núñez todos sus piques y agravios; lo mismo hizo el otro.
Se abrazaron como buenos compañeros que en aquel instante se veían más
que nunca subyugados por la religión del deber, y dirigiéronse a la
cámara. Antes de llegar a ella, la impaciente curiosidad de Méndez Núñez
iba soltando interrogaciones ansiosas. «Se ha pegado un tiro,» dijo
Pezuela ya dentro de la cámara; y lo decía con cierta sequedad, como si
más que lástima sintiera desdén del pobre suicida, General
Pareja\ldots{} Sin dejar espacio al asombro de don Casto, soltó la
segunda parte de la trágica noticia, que más bien debía ser primera:
«Hemos tenido una desgracia\ldots{} Nos han apresado la
\emph{Covadonga.»}

Solos en la cámara, hablaron de las causas del suicidio del General, que
habían de ser algo más que la pérdida de la goleta. «Yo me lo explico o
quiero explicármelo---dijo Pezuela,---por la depresión de su ánimo ante
el mal cariz de la campaña. El bloqueo nos resulta un fracaso. Los
Comandantes de las escuadras extranjeras no cesan de ponernos mil
obstáculos; nadie nos ayuda; nadie nos da una noticia, como no sea mala.
Vivimos en el mayor aislamiento, rodeados del odio de todo el género
humano. Hasta se ha dado el caso, aquí, en este mismo puerto, de entrar
una fragata inglesa, y pasar junto a la \emph{Blanca} sin hacer saludo.
Luego saltó a tierra su Comandante sin pedir permiso a Topete, y a los
dos días volvió a bordo, trayendo a un personaje chileno: era el
Intendente del departamento. Empavesó la fragata para recibirlo, le
saludaron con \emph{hurras}, y le hicieron extremados honores. Que le
cuente a usted Topete el berrinche que esto le costó y las ganas que le
quedaron de cañonear al inglés\ldots{} No sabía qué hacer. ¿Quién podía
prever un caso tal de descortesía, más bien de burla?\ldots{} Presumo yo
que Pareja se sentía hundido bajo el peso de su responsabilidad por
haber propuesto al Gobierno las actitudes belicosas a todo
trance\ldots{} Exageró quizás la debilidad de Tavira. Hizo creer al
Gobierno en una victoria fácil\ldots{} no sé, no sé.»

---¿Y últimamente, qué instrucciones recibió Pareja de Madrid?

---¿Lo sabemos acaso? Yo presumo que después de recibir órdenes para
llevar la cuestión por la tremenda, han venido órdenes de templanza y
transacción. ¡Vaya usted a saber\ldots! Habíamos acusado a Tavira de
traidor y desleal, y Tavira enseñaba una carta de Narváez, en que este
le decía: «No haga usted caso del Gobierno, y negocie la paz.» Esto es
inicuo\ldots{} Nos mandan al cabo del mundo, como si el venir acá y
emprender una guerra es estas latitudes fuera cosa de juego\ldots{} y
todo ello sin criterio fijo\ldots{} ¿Saben allí dónde estamos, y el modo
de ser de estas repúblicas? Y verá usted cómo nos faltan recursos cuando
sean más necesarios, y cómo nos veremos el mejor día sin una galleta,
sin un quintal de carbón y sin un real.

Luego contó Pezuela el triste caso de la \emph{Covadonga}. Carecía esta
goleta en absoluto de poder militar y de agilidad marinera\ldots{}
Cojeaba de la hélice; asma padecía en sus calderas; manca estaba la
tripulación, y el arma que llevaba (dos cañones en colisa) no servía más
que para matar pájaros\ldots{} Mandar estos inválidos a una guerra
lejana, era un verdadero crimen\ldots{} En Coquimbo estaba la pobre
veterana, con pata de palo y ambos brazos en cabestrillo\ldots{} Servía
para llevar y traer recados\ldots{} La infeliz navegaba por mares
enemigos, y a la vuelta de cada esquina o de cada cabo, acechábanla
embarcaciones de más poder\ldots{} En Coquimbo mismo entró a su bordo la
traición con pretexto de pedir informe referente a una presa norte
americana\ldots{} Los extranjeros, llamándose neutrales, ayudaban con
ardor a los chilenos, haciéndoles el servicio de espías. Los españoles
no tenían espionaje, ni podían tenerlo como no acudieran a las aves o a
los peces\ldots{}

Partió la pobre \emph{Covadonga} de Coquimbo para Valparaíso, cumpliendo
órdenes de Pareja, que ya estaba con el alma en un hilo recelando el mal
fin de la pobre mensajera\ldots{} El domingo 26 de Noviembre pasaba la
goleta frente a un puerto llamado El Papudo: amaneció con neblina; del
seno de esta salió como fantasma una corbeta, que izó bandera
inglesa\ldots{} No se dio por engañada la \emph{Covadonga}, y preparo
sus inútiles armas y avivó su andar premioso, renqueando por aquellos
mares de Dios, más bien del diablo\ldots{} Navegaba la corbeta de vuelta
encontrada por estribor\ldots{} Cuando se halló a popa, orzó rápidamente
y descargó su andanada sobre la goleta\ldots{} En seguida izó el
pabellón chileno. La goleta no tenía defensa\ldots{} El combate no podía
ser brillante por ninguna de las partes; mas por la parte española, que
era la suma debilidad, resultó de un heroísmo obscuro. La impotencia
hizo más de lo que humanamente podía. Los hombres se multiplicaron para
defenderse y para dejarse morir. Los de la \emph{Esmeralda} podían
dividirse, pues su barco valía por diez del nuestro.

Descansado fue para los chilenos el apresamiento de la \emph{Covadonga},
después de matar y herir a muchos de sus tripulantes. Cogida la nave
inválida, a remolque la llevaron al Papudo con algazara triunfal. El
Comandante Fery había sucumbido por falta de medios materiales que
dieran a su entereza la debida eficacia. Con mal sino fue a la guerra:
le tocó la china de tener que combatir con hombres bien armados, y para
esto no llevaba más que una caña y armadura de papel\ldots{} Los
prisioneros fueron llevados a tierra e internados hasta Santiago, donde
se les trató con rigor y crueldades que no merecía su glorioso
vencimiento.

A una interrogación inquieta de Méndez Núñez, contestó Pezuela que el
Jefe de Escuadra no había tenido conocimiento del desastre de la
\emph{Covadonga} hasta que fue a notificárselo el Cónsul americano
Nicholson, que, dándoselas de amigo de España, favorecía con toda clase
de manejos y soplos la causa chilena. Y añadió el Comandante de la
\emph{Berenguela}: «Ya he dicho a usted que estamos aquí en un
aislamiento horrible\ldots{} No tenemos la simpatía de ninguna
nación\ldots{} Nadie nos ayuda, nadie da calor a nuestra causa, como no
sea un grupo de españoles fanáticos, unidos a unos cuantos franceses
mercachifles, que no sabemos qué fines se traen ni a qué móviles
obedecen\ldots»

---Estamos bien---dijo don Casto triste y ceñudo,---y en estas
condiciones bloquee usted con cinco barcos un frente de mil quinientas
millas\ldots{} En Madrid no tienen idea de lo que es esto. Comprendo la
desesperación del pobre Pareja\ldots{} Sin base de operaciones, teniendo
que llevar a cuestas la comida y el carbón, estamos a nueve mil millas
de la patria. ¿Dónde podríamos reparar una avería de importancia?
\emph{En el cementerio}, como dijo el General Álvarez; en el mar\ldots{}
Eso sí: por cementerio no podremos llorar, que el que aquí tenemos es
bastante ancho.

En este punto del coloquio, llegaron don Claudio Alvargonzález y don
Miguel Lobo, Comandante y Mayor General de la \emph{Villa de Madrid}, y
hablando todos de los graves sucesos, no añadieron nueva luz a las
causas del suicidio de Pareja. Resultaba como causa única y bastante
poderosa la convicción del fracaso de su política en el Pacífico. Se
sentía responsable de haber llevado las cosas al camino escabroso por
donde iban a la sazón. Contaron asimismo los jefes de la \emph{Villa de
Madrid} que después de la visita de Nicholson, observaron en el General
Pareja una tranquilidad melancólica, que en otra persona no podía ser
alarmante; en un militar, si lo era. Hablando con Lobo, le preguntó con
flemática frialdad: «¿Cree usted que nos habrán apresado también la
\emph{Vencedora}?» Y Lobo respondió: «Mi General, lo creo posible y
probable; que estos pobres barcos, indefensos y que andan con muletas,
llegan de milagro a donde se les manda.» Por la tarde, el General comió
con mediano apetito; después paseó un rato en la toldilla, fumando un
cigarro. Bajó a su cámara\ldots{} Tenía costumbre de tirar desde el
balcón con revólver a los pájaros marinos. Así lo hizo aquella
tarde\ldots{} Tres veces disparó\ldots{} Pasó tiempo\ldots{} El cuarto
disparo sonó en los oídos del Comandante y del Mayor General con mayor
estruendo que los anteriores. Pero apenas se fijaron en la intensidad
del ruido\ldots{} De pronto salió de la cámara dando gritos el asistente
italiano del General. Acudieron, y hallaron a Pareja tendido en la cama,
sangrando de la cabeza. Aún tenía en su mano derecha el revólver\ldots{}
En la mesa vieron un papel, en que había trazado el suicida con firme
pulso sus últimos pensamientos, dirigidos a Pastor y Landero, su sobrino
y secretario. Tres pensamientos eran: \emph{Te estoy agradecido\ldots{}
Que no me sepulten en aguas de Chile\ldots{} Que todos se conduzcan con
honor.}

Oído todo esto, y algo más que por no incurrir en prolijidad aquí no se
cuenta, Méndez Núñez suspiró fuerte, y dejó ver en sus ojos cierta luz
que anuncio parecía de resolución firme\ldots{} Era Jefe de la Escuadra;
la autoridad, así como la responsabilidad de Pareja, habían pasado a ser
suyas\ldots{} ¿Cómo continuar la empresa trágicamente interrumpida? Al
abandonar el mundo y la vida, arrojó Pareja sobre un papel una idea
sentimental: \emph{que no me sepulten en aguas chilenas}; y tras esto,
una generalidad de las que vulgarmente llamamos de clavo pasado.
¡Conducirse con honor! Esto ya lo sabían todos, y no había la menor duda
de que así se cumpliera\ldots{} Pareja pudo legar a su sucesor una idea
militar, un plan, un criterio\ldots{} Pero nada de esto dejó, sin duda
porque no lo tenía\ldots{} La Historia se continuaba; al caudillo muerto
reemplazaba el caudillo vivo. Quizás lo que no dijo el papel fúnebre de
Pareja, decíanlo los ojos de Méndez Núñez: \emph{Concentración de
fuerzas}\ldots{} \emph{Tomar la ofensiva}.

Aquella misma tarde trasladó Méndez Núñez su persona y su insignia a la
Villa de Madrid, y salió para Valparaíso.

\hypertarget{xxi}{%
\chapter{XXI}\label{xxi}}

La \emph{Numancia} permanecería en Caldera hasta que llegasen los
transportes de vela \emph{Valenzuela Castillo} y \emph{Vascongada}, que
del Callao salieron con víveres y carbón. Aún había para rato, por causa
de las calmas de aquellos días. Aburridos quedaron los tripulantes de la
fragata y como desengañados, pues muchos de ellos creían, al partir del
Callao, que iban a una función militar de importancia. Otros veían en la
ausencia de su General un vacío melancólico, cual si Méndez Núñez se
hubiera llevado consigo toda la grandeza y ardor guerrero del primer
barco de la Nación. Mientras allí estuvieran las fragatas, debían
custodiar el enorme rebaño de buques apresados que con los transportes
formaban una impedimenta fastidiosa y pesadísima. No teniendo España, en
la inmensa extensión de la costa debelada, ningún puerto, ni siquiera un
islote, para refugio y abrigo de sus operaciones, veíase forzada a
conducir consigo la reata de barcos viejos que le servían de carboneras,
de almacenes, de talleres, y de enfermería en algún caso. Se comprenderá
cuán molesta y embarazosa era esta mochila para el guerrero que allí
necesitaba toda su agilidad y desenvoltura.

Las dos fragatas y todas las embarcaciones de vapor tenían siempre
encendida sus calderas; la vigilancia era minuciosa; en la lancha de
hélice, o en botes, los Guardias marinas bordeaban de día y de noche.
Dos tercios de los tripulantes velaban desde la puesta del sol hasta su
salida. En la plenitud del verano austral, eran las noches claras,
estrelladas, de solemne hermosura. Marineros y oficiales de mar,
oficialidad y jefes armaban sus tertulias nocturnas en los sitios
correspondientes a cada jerarquía\ldots{} Los mentideros más animados
eran los populares, a proa. Junto al cabrestante formaban un ruedo
animadísimo Sacristá, Fenelón, Ansúrez y otros amigos de Máquina y
Maestranza. Binondo, que también hocicaba en aquel ruedo, se apartó
bruscamente de él y se fue hacia un grupo de marineros que charlaban
junto a la borda. «Me vengo aquí---dijo,---huyendo de las conversaciones
indecentes de esos perdidos\ldots{} Me escandalizo de oír los cuentos
asquerosos que refiere el francés de las mujeres que ha conocido en
Lima, Callao y el Chorrillo. Ningún hombre de buenos principios puede
oír tales porquerías. De una dice que tiene el cuerpo blanco como la
leche; de otra, que es morenita tostada, y encendida de su fuego
natural\ldots{} Y como el hombre ve que le ríen y alaban estas
suciedades, no se para en barras\ldots{} ni en pechos, y ahora decía que
los tiene muy bonitos una que llaman Susana, sobrina de no sé qué
General, y prima del señor Arzobispo\ldots{} Aquí me vengo, porque ese
condenado le hace pecar a uno de intención, y en estos casos yo corto
por lo sano, quiero decir, corto por las intenciones.» Oído esto por los
muchachos, dejaron solo a Binondo y se fueron al ruedo.

Las aventuras amorosas acometidas con singular audacia por Fenelón, y
consumadas triunfalmente, embelesaban a los pobres mareantes, tan rudos
como crédulos. Los más de ellos se tragaban sin chistar las enormes
bolas que de su boca fecunda iba soltando el maquinista. El cual,
henchido de fatuidad ante el éxito de sus embustes, lanzábase a los
mayores atrevimientos de la inspiración y de la fantasía. Terminó su
mujeril relato con esta síntesis gallarda: «Yo, que he recorrido las
Américas divirtiéndome cuanto he podido, y cursando, \emph{por ejemplo},
toda la carrera del amor hasta el doctorado, aseguro a ustedes que las
mujeres más hermosas de este continente son las costarriqueñas: diosas,
estatuas vivas las llamo yo. Las más graciosas y apasionadas, las más
seductoras y las más tiranas del hombre, son las del Perú; y en
ilustración, a todas ganan las de este país en que ahora estamos, las
chilenas, señores, que no por sabias y discretas dejan de ser
bonitas\ldots{} mi palabra. Ocurre que en Valparaíso o en Santiago está
usted haciendo el amor a una señorita, y a lo mejor la señorita,
contestando con gracia, le habla a usted de Kant o de otro filósofo muy
nombrado\ldots» Los contramaestres y cabos de mar oían estas cosas con
la boca abierta; y aunque no sabían quién fuese aquel Kant, celebraban
la ocurrencia y enaltecían al orador.

Derivó luego la conversación a un asunto distinto. Desiderio García,
cabo de mar andaluz, muy amigo de Ansúrez, excelente hombre, un poco
dado a la taciturnidad, fue instigado por sus compañeros a tratar de un
tema que a él le trastornaba y a muchos divertía. Debe indicarse que
había navegado por el Pacífico en buques mercantes y de guerra, y
conocía no pocos lugares de la costa y algunos del interior. Contaba
(sin que pueda garantirse su veracidad) que había vivido en una tribu de
indios bravos, y recorrido largas extensiones del continente, al otro
lado de los Andes. «Pues queréis que hable, hablaré---dijo.---Óiganme y
aprendan. Yo sé lo que sé, y de mi saber de este negocio no me arranca
nadie. Estamos en Caldera\ldots{} El monte altísimo que allí vemos, por
encima de la ciudad, lejos, lejos, ¿cómo se llama?»

---Es el \emph{Bonete}---dijo Sacristá:---seis mil metros de altura.

---Más al Sur. ¿Pero no lo sabéis? Tendré yo que deciros que esa altura
es \emph{Come caballos}, y que allí hay una garganta o puerto por donde
pasamos a la otra banda y a un río que llaman Bermejo, el cual lleva sus
aguas al Paraná. Todos esos territorios he corrido yo, y sé que entre un
pueblo que se llama \emph{Tinogasta} y otro que nombran
\emph{Copacavana}, hay unas peñas en lugar descampado y yermo\ldots{} y
en esas peñas abertura estrecha por donde se entra a una cueva tan
grande como cuatro veces la catedral de mi pueblo, que es Córdoba. Pues
en esa cueva, guardada en unas al modo de arcas de piedra, hay tal
cantidad de plata en barras, que puede calcularse en seis o siete
millones de quintales de ese metal\ldots{}

Pausa, en la cual se oyó un grave murmullo: de asombro era, o de burla
mal contenida. Acallado el rumor, prosiguió Desiderio, y dijo que él
había visto el tesoro; que conocía su existencia por un indio viejo,
patriarca en la tribu, llamado \emph{Zapirangui}, padre del famoso
\emph{Cuarapelendi}, indio guerrero. El tesoro allí estaba muerto de
risa, como quien dice, y no faltaba más que ir a cogerlo y transportarlo
a un puerto de mar, empresa que requería grande y costoso convoy de
acémilas y un mediano ejército para custodiarlo. Declaraba el Cabo de
mar, con la más pura convicción y seriedad, que ofrecía la mitad del
tesoro a quien concurriese con él a extraerlo del escondido antro en que
yacía desde el tiempo de los señores Incas. No quería comunicar el
secreto al Gobierno de Chile. Como buen español aguardaba las victorias
de España y la ocupación de toda la América del Sur por los españoles,
para tratar con el Jefe de la Escuadra de la forma y modo de traer la
plata a la costa, llevándola después a España en dos mitades: una para
el descubridor, y otra para Isabel II.

Refería estos disparates el Cabo de mar con tanto aplomo, que los
incrédulos y guasones, que eran los menos, no se atrevían a
contradecirle. Temían su furor, pues era hombre que súbitamente se
encendía cuando alguien negaba o tomaba en solfa el depósito de plata.
Como no le tocaran este asunto, no había hombre más pacífico y
razonable. Ansúrez, que al principio había tenido con su compañero
agarradas tremendas por el tesoro de \emph{Copacavana}, ya empezaba a
creer en él, como primer paciente del mal de soñación, que suele atacar
a los navegantes en las travesías dilatadas. «Mayor simpleza que lo del
tesoro---se decía el buen Ansúrez con sinceridad candorosa---es creer
que tengo aquí a mi adorado nietecillo Carmelo, y que le acuesto en mi
coy, le visto y le arreglo, y le saco en brazos a pasearle por la
cubierta. Cierto que esto es una sinrazón, lo reconozco\ldots{} pero
momentos hay en que a ojos cerrados lo creo, por el consuelo que me da
la mentira\ldots{} En esta soledad chicha, sin ningún cariño a nuestro
lado, nos moriríamos de pena si no encendiéramos las calderas del
pensar, y no navegáramos a un largo por el mundo de la ilusión\ldots{}
En fin, me voy abajo, quiero estar solo\ldots{} Solo, piensa uno lo que
quiere, y se divierte con su propio engaño.»

Todos iban cayendo, como he dicho, en la soñación endémica, y el más
atacado era Binondo, que en la ociosidad física cultivaba más que los
otros la vida espiritual. Una noche, viendo a Desiderio García asomado a
la borda, mirando a tierra con atención alelada, llegose a él y le dijo:
«Yo creo en tu tesoro; Dios me da vista bastante larga para ver el lejos
de las cosas, y para conocer que el hombre espiritado, como tú lo estás,
sabe dónde moran los bienes escondidos\ldots{} Fíjate, Desiderio, fíjate
en la estrella que ahora está sobre \emph{Come caballos}. ¿La ves? Pues
esa estrella tan bonita no sigue la marcha que llevan las otras en el
cielo, sino que va dejándose caer, dejándose resbalar por detrás del
horizonte\ldots{} Estas noches me las he pasado observando la rareza de
su movimiento, pues cuando todo el cielo deriva, como sabes, de Oriente
a Occidente, ella va de vuelta encontrada. No podía yo comprender ni
explicarme esta cosa nunca vista\ldots{} pero al oírle decir lo del
tesoro guardado entre peñas montunas a la otra banda de los Andes, he
caído, Desiderio, he caído en la verdad\ldots{} Pienso que será esa
estrella un sino con que el Padre, el Hijo, el Espíritu Santo, o
verbigracia los tres, nos marcan el lugar del tesoro para que vayamos a
cogerlo y regalárselo a nuestra España querida.»

Echó Desiderio al malayo una mirada fulgurante, acompañada de temblor de
mandíbula, que en el Cabo de mar anunciaba siempre un acceso de cólera.
Sobrecogido, Binondo puso en juego toda su astucia y labia persuasiva
para despertar confianza en el espíritu del maniático. Entre otras
extravagancias, le dijo: «Fíjate bien en la estrella, y verás que tiene
rabo, un rabito que apenas ahora se distingue y que va creciendo,
creciendo hasta media noche. La estrella baja y se pone a contra-cielo;
aún se verá la punta del rabo cuando el alba empiece a comerse las
constelaciones. Si no crees en la maravilla, y en que el Eterno, que así
decimos, por medio de luces celestes y angélicas con corona o con rabo,
y de otras señales y avisos, guía los pasos del hombre, no llegarás a
recoger tu tesoro.» Tanto y tanto le dijo y arguyó, y tan sutilmente
supo enlazar las ideas religiosas con la superstición, que a la media
noche Desiderio veía la estrella, su cola y movimiento, tal como el
malayo lo describía. Y ambos, en ardiente coloquio, determinando la
relación entre los tesoros de la tierra y los del cielo, convinieron en
que la fe vivísima es el medio más seguro para llegar a poseer unos y
otros.

Todos soñaban; el delirio descendía del cielo transparente y estrellado,
para introducirse en las cabezas de los pobres mareantes, que ya
llevaban casi un año ausentes de su familia en países enemigos,
empeñados en empresa guerrera que hasta entonces les ofrecía más fatigas
que gloria, privados de todo cariño y del trato de mujeres, sin pisar
tierra o pisándola hostil, resentidos ya de la poca variedad y frescura
de los alimentos, esperando la solución bélica que nunca venía, y
preguntándola, sin obtener respuesta, al Pacífico inmenso y a la muda
esfinge de los Andes.

Todos desvariaban, todos padecían la nostalgia que impele a la
construcción de una vida ilusoria para llenar con ella los vacíos del
alma. Fenelón evocaba la persona de una dama limeña, a quien había visto
en el Chorrillo sin poder cambiar con ella más que cuatro palabras de
saludo ceremonioso; a su lado la traía; paseaba con ella del brazo por
la cubierta, por el alcázar y la batería; llevábala a su camarote;
platicaban de amores, reían, se ponían serios, eran dichosos\ldots{}
Ansúrez se persuadió una noche de que su hija Mara, deslumbrante de
hermosura y elegancia, entraba en la fragata por el portalón: hablaban
hija y padre tranquilamente, como si nada hubiera pasado, como si se
hubieran visto el día anterior; el chiquillo tenía ya seis años;
Belisario regalaba a su suegro una vajilla de plata; doña Celia era una
señora con muchos moños y lacitos en el pelo gris, cargada de esmeraldas
y rubíes, de habla graciosa y dulce, como la de las gaditanas\ldots{}
Sacristá vio a su mujer de cuerpo presente en su casa de Cartagena: las
luces macilentas que alumbraban a los mayordomos en el pañol de proa, le
dieron esta impresión fúnebre que desechar no pudo en tres o cuatro
noches sucesivas\ldots{} Binondo y Desiderio reducían a formas reales
sus teorías de la intervención divina en el descubrimiento de tesoros; y
el Cabo de mar, en un minuto de sinceridad efusiva, vació sus
pensamientos más recónditos en el oído del malayo, diciéndole: «A ti
solo, José, confiaré lo que aún no he querido confiar a nadie, lo más
reservado, lo más secreto, y es\ldots{} escúchame sin miedo: debajo de
la cueva de \emph{Copacavana}, donde están, en arcas de piedra, los
miles de millones de barras de plata, hay otro covachón más hondo, con
bajada secreta, y en ese segundo sollado subterráneo, no
tiembles\ldots{} hay como unos doscientos bocoyes llenos de pepitas de
oro\ldots{} y no te digo más.»

Y por este estilo soñaban todos los demás, en las jerarquías nobles, de
Guardias marinas para arriba; sólo que sus delirios tomaban otras formas
y caracteres. Eran sueños de guerra, de acciones heroicas. Quién soñaba
con el engrandecimiento personal, quién con sacrificios y extremadas
virtudes. Unos veían entre brumas gloriosos triunfos de la patria;
otros, grandes desventuras y catástrofes.

\hypertarget{xxii}{%
\chapter{XXII}\label{xxii}}

Al Sur de Caldera está Calderilla, que también llaman \emph{Puerto
inglés}, y allí cambiaron por primera vez los españoles sus disparos con
disparos de tierra. Se supo que en Calderilla preparaban los chilenos un
torpedo, montándolo en un vaporcito de ruedas. A quitarle al enemigo
ambas cosas, vaporcito y torpedo, fueron dos animosos oficiales: Alonso,
en la lancha de vapor de la \emph{Numancia}, y Garralda, en un bote a
remolque. Arriesgadilla era la empresa, porque la guarnición de Caldera
se corrió a Calderilla y tomaba posiciones en las rocas que protegen el
puerto. Llegaron los oficiales a donde se proponían, y a la vista de los
chilenos se hicieron dueños del vapor. Ya salían con él a remolque,
cuando se vieron obligados a sostener vivo fuego con los enemigos,
apostados en la orilla Norte. Heridos fueron Garralda y un marinero, y
en gran compromiso se vio la pequeña expedición al querer salvar la boca
del puerto, de unos ochocientos metros de anchura. La suerte de los
españoles fue que los chilenos no acertaron a ocupar más que el costado
Norte de la barra, desamparando el lado Sur, llamado la
\emph{Caldereta}. A esta se arrimaron Garralda y Alonso, sosteniendo el
fuego con las tropas de la otra banda. Su arrojo y serenidad, así como
el auxilio que les prestó la \emph{Berenguela}, acercándose a la entrada
del puerto y cañoneando a los de tierra, les salvaron de un copo seguro.
No pudiendo sacar el vapor aguas afuera por lo que tiraba la marea, lo
echaron a pique, y allí se quedó con su torpedo, si es que lo tenía.

Llegaron por fin la \emph{Vascongada} y la \emph{Valenzuela Castillo}. A
esta podía llamársela el buque milagro, pues de milagro se sostenía
sobre las aguas y milagrosamente llegó a Caldera, gobernada por el
Alférez de Navío don Antonio Armero. Su viaje desde el Callao había sido
un naufragio constante. La vieja fragata, de inmemorial edad, se
descosía, se desarmaba, y sus tripulantes no tuvieron en la travesía
momento seguro. Toda la navegación fue un perenne picar de bombas, un
remendar infatigable de averías y una horrible lucha de la vida con la
muerte. De los quebrantados palos se caían los marineros, y al caer se
mataban y herían a sus camaradas. Héroes fueron aquellos infelices, y el
Oficial que los mandaba mereció más premio que si hubiera ganado una
batalla. A toda prisa se procedió a descargar a la veterana
\emph{Valenzuela}, que no deseaba más que quedarse vacía para tumbar sus
pobres huesos en un playazo. Todos los víveres y municiones fueron
trasladados a los pocos barcos útiles, y se acordó pegar fuego a las
presas, que no servían más que de estorbo, sentencia que fue
rigurosamente ejecutada cuando la \emph{Numancia} y \emph{Berenguela},
obedeciendo a órdenes del Superior, zarpaban para Valparaíso. Fue un
espectáculo espléndido, un simulacro de volcanes marítimos. Los viejos
barcarrones tenían una muerte más brillante que la que les habrían dado
las tormentas deshaciéndolos en las soledades oceánicas. Sus exequias
eran fiesta extraordinaria de las aves y los peces.

Concentrada en Valparaíso toda la escuadra, tuvo eficacia el bloqueo,
reducido al puerto principal de la República. Y ahora, hablando
nuevamente de los españoles que soñaban, designamos a Topete y
Alvargonzález, Comandantes de la \emph{Villa de Madrid} y de la
\emph{Blanca}, como los que en mayor grado padecieron hasta entonces el
desvarío heroico, pues afrontaron una de las empresas más temerarias que
cabe imaginar. Deseando Méndez Núñez buscar al enemigo en los lugares
inaccesibles donde tenía su refugio, los esteros y canalizos del
archipiélago de Chiloe, preguntó a los dos marineros Alvargonzález y
Topete si se atreverían a penetrar en aquel dédalo para sorprender en su
escondrijo a las naves aliadas.

Pudieron responder los dos guerreros de mar que tal empresa era
imposible, mortal de necesidad para barcos y hombres; mas no dijeron
esto, sino que, antes que fueran otros, deseaban ir ellos sin pensar en
el peligro, ni medir los inconvenientes náuticos y militares de aventura
tan descomunal. Salieron las dos fragatas. Justo es declarar que al
verlas partir, casi todos los soñadores que en Valparaíso quedaban,
pensaron que no volverían a verlas\ldots{} Pero se engañaban, porque a
las dos semanas o poco más reaparecieron con su casco y aparejo
intactos, o con no visibles averías. Habían consumado proeza semejante a
las de los argonautas, penetrando en laberintos habitados por monstruos
que devoraban al que osaba llegar hasta ellos. El monstruo era una
Naturaleza hostil, armada de toda clase de asechanzas y peligros, que
para el enemigo de los españoles era refugio y defensa. Alvargonzález y
Topete entraron con esforzado corazón en el laberinto por el golfo de
\emph{Guaytecas}, boca Sur del Archipiélago; navegaron por un angosto
mar, parecido a estanque de recortadas orillas, y dieron fondo en
\emph{Puerto Obscuro}. Indígenas de mal pelaje les dieron noticia de la
madriguera en que se agazapaban las naves chilenas y peruanas.

Prodigiosa fue la marcha por angosturas y desfiladeros, sin más auxilio
que imperfectas cartas, obra de navegantes que habían recorrido aquellas
aguas en cachuchos de corto calado. La \emph{Blanca} y \emph{Villa de
Madrid} andaban al paso, sin dejar de la mano la sonda, temiendo a cada
instante dar en un bajo. Hallábanse a los 42 grados de latitud Sur; la
marea entrante y saliente tiraba con fuerza de seis o siete millas. Tal
o cual paso, donde por la mañana había un fondo de quince a veinte pies,
a la tarde estaba seco. Ángulos y dobleces aparecían, que apenas daban
espacio a las viradas\ldots{} Navegaban las fragatas como los ciegos,
tanteando el suelo con su palo y palpando las paredes cercanas\ldots{}
La \emph{Blanca}, de menor calado, iba delante reconociendo el terreno;
seguía la \emph{Villa de Madrid}, obediente a las indicaciones de su
compañera\ldots{} ¡Qué tales serían las calles y callejones de aquella
Venecia desconocida, que los peruanos y chilenos, guiados por gentes del
país, perdieron allí dos fragatas! ¡Cuando los de casa perdían allí las
botas, qué no perderían los forasteros!

Pero una deidad o encantador benigno miraba por aquellos temerarios
hombres, Alvargonzález y Topete, cuando no se dejaron allí las fragatas
y las vidas y hasta el nombre de España. Por noticias más certeras que
las recibidas en \emph{Puerto Obscuro} supieron que los barcos enemigos
estaban en un estero de la isla de Abtao, y allá se fueron. La temeridad
rayaba en locura. Había que encomendarse a Dios o al diablo para
penetrar en el tortuoso callejón que separa del Continente la recortada
isla\ldots{} Entraron, y en un ángulo recto que forma la ratonera vieron
los españoles el cadáver de la fragata \emph{Amazonas}, tumbado en el
arrecife. Debieron la \emph{Blanca} y \emph{Villa de Madrid} mirarse en
aquel espejo y volverse atrás; pero la calentura heroica pudo más que la
razón. ¡Avante, que el enemigo no podía estar lejos! En efecto, a la
salida del callejón, las fragatas vieron los mástiles de los buques
enemigos; aún navegaron largo trecho pare divisar los cascos.

Chilenos y peruanos hallábanse resguardados por arrecifes, que eran como
una valla imposible de salvar desde fuera. Apenas se echaron la vista
encima, empezaron unos y otros a cañonearse. La distancia no podía ser
acortada por las naves españolas. Habían de darse por satisfechas con
causar algunas averías a los barcos enemigos y matarles o herirles
algunos hombres\ldots{} Y allí terminó la hazaña, porque el monstruo de
la Naturaleza, que en aquellos laberintos habita, sacó del légamo la
cabeza y dijo a los atrevidos argonautas: «Retiraos, locos, ilusos, y no
abuséis de mi paciencia y de la benignidad con que os he dejado llegar
aquí. ¿Qué pensáis, qué queréis, hombres o niños grandes, que habéis
entrado en mi reino con sólo vuestros corazones, dejándoos fuera la
razón? Salid pronto, que a poco que os detengáis, retiro las aguas y
quedaréis en seco\ldots{} De vuestros barcos haré leña para mis
hogueras, y de vosotros no quedará uno solo para contar al mundo vuestra
locura.»

¿Qué habían de hacer los infelices más que obedecer a tan imperiosa
conminación? Unas horas más en los canalizos, y seguramente no podrían
contarlo. Se volvieron, en busca de la salida del laberinto, no sin que
Topete, con terquedad maniática, se parara en un sitio más despejado que
los anteriores, y con la voz tonante de sus cañones, llamase a los
contrarios, diciéndoles: «Venid aquí, enemigos y compañeros; dejad el
enrejado de peñas en que os guarecéis\ldots{} Salid a este campo, y nos
veremos las andanadas\ldots» Pero los otros no salían. Estaban muy a
gusto en sus cómodas huroneras. Las fragatas se desenvolvieron de la
madeja intrincada de Chiloe, y tornaron a Valparaíso. Contado lo que
habían hecho, nadie quería creerlos. El Almirante inglés Denman, que
visitó la \emph{Villa de Madrid}, oyó de boca de don Miguel Lobo el
relato de la expedición, y a creerla no se determinaba. «La empresa
marinera que usted cuenta---dijo---cae dentro de la esfera de lo
fabuloso, y no le daré crédito si usted no la garantiza con su palabra
de honor.»

Verdaderamente, la entrada en Chiloe, el cañoneo en Abtao y la salida
del Archipiélago, no menos admirable que la entrada, eran un prodigio de
habilidad y audacia marineras. Bien podían contarse Alvargonzález y
Topete entre los más heroicos argonautas del mundo. De la eficacia
militar de la expedición no podría decirse lo mismo: las naves
americanas no abandonaban su resguardo, ni admitían combate en aguas
abiertas.

El relato que hicieron los expedicionarios avivó más el fuego de las
imaginaciones soñadoras, y el propio Méndez Núñez quiso repetir por sí
mismo la expedición, llevando de guía o práctico a Topete, que ya
conocía el obscuro dédalo de Chiloe. Salieron la \emph{Numancia} y la
\emph{Blanca} con gran entusiasmo y alegría de sus tripulantes, y cuando
al Archipiélago se aproximaban, les salió viento duro del Sudeste y mar
tan gruesa, que la blindada causó alguna inquietud por la violencia y
amplitud de sus balances. La terrible deidad que imperaba en el
laberinto salió al encuentro de don Casto y le dijo: «¿También tú vienes
acá, Capitán de estos locos y el primero en las vanas locuras? Vuélvete,
y no esperes que sea contigo menos riguroso que lo fui con tus atrevidos
compañeros. Más te perjudica que te favorece traer contigo ese armatoste
blindado, que por su peso y corpulencia estará expuesto a quedarse en
mis dominios, y yo te aseguro que si no viras en redondo y te vuelves a
donde estabas, haré por merendarme tu fragata, que es bocado
exquisito\ldots» Esto oyó Méndez Núñez; mas no hizo caso, y se metió en
Chiloe por las \emph{Guaytecas}, que era la puerta más expedita y
franca.

Viendo el fantasma del Archipiélago que los locos persistían en su
desvarío, desplegó contra ellos una niebla que en sus velos densísimos
los envolvió, cegándolos para que no pudieran andar un paso. Las hélices
daban unas cuantas estrepadas lentas, y en seguida tenían que parar. Aun
en estas condiciones, persistieron en su temeridad, y aprovechando las
claras de la niebla llegaron hasta el mismísimo Abtao, que era llegar al
interno cubículo donde el monstruo habitaba. Pero este salió a
manifestarles con más burla que ira la inutilidad de su expedición,
porque el enemigo se había retirado a un recoveco más inabordable y
escondido, al cual no podrían llegar los barcos españoles si no se
trocaban en anguilas.

Nuevamente les conminó el monstruo a que se largaran, y se dispusieron a
obedecerle; repetía las amenazas otra deidad marina, la bajamar,
diciéndoles que se quedarían en seco si no tomaban el portante. Luchando
con las dificultades del poco fondo, de los arrecifes, de la niebla,
salieron al ancho mar, y a Valparaíso volvieron sin otra novedad que
haber hecho en el camino tres presas: un vapor con pasajeros, que
resultaron reclutas del ejército chileno, y dos fragatas con carbón del
país, que era contrabando de guerra. En Valparaíso encontraron la
escuadra norte americana, recién llegada con cuatro magníficos barcos de
hélice y un monitor llamado \emph{Monadnoch}, que al decir de la gente
se comía los niños crudos.

La flota yanqui, así como la inglesa y los barcos italianos y franceses,
venían al apoyo moral de Chile por la simpatía, y a quebrantar a los
españoles por el despego y la callada hostilidad que en toda ocasión les
mostraban. Así, la incauta y soñadora España llegó a encontrarse sola
frente a dos repúblicas que ante ella desplegaban un frente de costa
casi de mil leguas; y contra aquel frente tenía que combatir sin ayuda
de nadie, sin amparo de ningún pedazo de tierra, llevando consigo las
armas, la comida, el carbón y la bandera. Pocas manos eran para tantas
cosas.

\hypertarget{xxiii}{%
\chapter{XXIII}\label{xxiii}}

El 23 de Marzo saludó el fuerte de Valparaíso con vivo cañoneo a las
banderas de las aliadas de Chile, que a más del Perú, eran Bolivia y
Ecuador; sorpresa histórica, pues ningún agravio ni cuestión pendiente
con la madre tenían estas dos repúblicas. En tanto la madre, llevada por
lastimosos errores de toda la familia a los extremos del coraje, no
tenía más remedio que saludar a Chile con algo más que ruido y humo de
pólvora. Los enojos no aplacados y los ultrajes no satisfechos,
forzosamente conducían a la violencia; que las naciones, cuanto más
viejas, más aferradas viven a la rutina caballeresca del honor. El honor
no existe sin valentía. La valentía puede salvar las situaciones de
hostilidad entre dos países, y es a veces más eficaz que el derecho y
que la razón misma. El apocamiento del ánimo no resuelve nada, ni aun
cuando le asiste la razón. Así lo comprendió Méndez Núñez cuando dispuso
el bombardeo de Valparaíso, acto inevitable ya, derivación lógica y
fatal de los hechos pasados.

No lo comprendían así los Jefes de las escuadras inglesa y americana,
que protestaron del bombardeo, y aun se pusieron los moños de que lo
impedirían\ldots{} Para no llegar a la extremidad de tirotearse con los
españoles, el Contralmirante Denman (inglés) y el Comodoro Rodgers
(yanqui) llevaron a tierra sus buenos oficios para conseguir del
Gobierno chileno las tan disputadas satisfacciones que España pedía.
Pero Chile no quiso darlas por no parecer pusilánime. Las cosas habían
llegado al punto delicado en que se pasa por todo antes de dejar salir
al rostro la menor sombra de miedo. Verdaderamente, las hijas no
mostraban ningún respeto a la madre, olvidando que de ella habían
recibido sus virtudes guerreras, así como sus flaquezas políticas.
Debieron ser las primeras en ceder de su rigurosa tirantez, y
seguramente la madre no se habría quedado atrás en las concesiones para
llegar a las paces. Pero, en fin, el acto de fuerza era inexcusable; don
Casto no podía envainar la espada, y cuando los Comandantes de las
flotas extranjeras daban a entender que se interpondrían entre los
españoles y la plaza, les decía con arrogante concisión que no le
importaba perder sus barcos si conservaba su honra.

Dados los correspondientes avisos al Comandante militar de la plaza para
que señalara con bandera blanca los puntos que debían ser invulnerables,
hospitales, casas de asilo, iglesias, etc., y para que se retirasen los
no combatientes, se señaló el bombardeo para el 31, Sábado Santo.
Amaneció este día con inquietud grande de los españoles. ¿Se decidirían
los extranjeros a proteger la plaza, obligando a Méndez Núñez a desistir
de su propósito? Este recelo se disipó bien pronto, porque apenas
iniciado el movimiento de las fragatas para situarse en los puntos de
ataque, ingleses y americanos levaron anclas y se retiraron mar afuera,
dejando libre el campo\ldots{} \emph{Resolución}, \emph{Blanca} y
\emph{Villa de Madrid} fueron las designadas para cañonear la ciudad. La
\emph{Berenguela} se retiró al fondeadero de Viña del Mar, al cuidado
del convoy. La \emph{Numancia}, después de aproximarse a la población
para dar, con dos cañonazos sin bala, la señal de que empezaba la
función, se volvió a retaguardia de las tres naves combatientes.

A las nueve se rompió el fuego, dirigido exclusivamente contra los
edificios del Estado más próximos: Ferrocarril, almacenes de la Aduana,
Intendencia y Bolsa. Al fuerte se lanzaron también gran número de
proyectiles sin obtener respuesta, pues los cañones estaban desmontados,
y los artilleros no tenían allí nada que hacer. Un disparo certero de la
\emph{Villa de Madrid} partió el asta de la bandera chilena, que ondeaba
en el Fuerte. Los edificios condenados a sufrir el bombardeo dieron
pronto señales del estrago que causaban nuestros proyectiles. La Aduana
y almacenes caían a pedazos; columnas de negro humo señalaban el
incendio en diferentes puntos de la ciudad. Era un espectáculo deslucido
y triste. Faltaba la excitación y armonía del combate, la acción
ofensiva de una parte y otra. Los españoles no celebraban ciertamente la
indefensión de la plaza, y habrían visto con gusto que el Fuerte
respondiera al fuego con el fuego. No les satisfacía la forma de
escarmiento que tomaba en aquella ocasión la guerra, ni se sentían
airosos manejando los instrumentos de castigo. Sus arreos eran las
armas, no las disciplinas.

Todo terminó a las doce menos cuarto. El cañoneo no llegó a durar tres
horas: ya era bastante; aun era quizás demasiado para simple castigo o
reprimenda de una madre austera, harto pagada de su carácter venerable y
de sus históricos blasones. La hija, herida y maltrecha de los crueles
disciplinazos de la madre, miraba a esta desde tierra con el más agrio
cariz que puede suponerse. Hasta entonces, sólo íbamos ganando en el
Pacífico la malquerencia de las Repúblicas. España, al fin y al cabo,
pagaba las culpas de sus diplomáticos y de sus gobernantes. Toda guerra
tiene o debe tener una finalidad militar o mercantil: los fines de la
nuestra en el Pacífico no se veían claros, como no fueran el fin sin fin
de abandonar los principios de la historia nueva para reanudar una
historia concluida.

Tres mil hombres mal contados constituían la dotación de las cinco naves
de combate y de las embarcaciones auxiliares y de convoy que
representaban a España en las aguas del Pacífico. Aquellas tres mil
voluntades, de diferentes categorías, eran o creían ser la voluntad
integral de la Nación; las tablas o las planchas de hierro en que los
hombres se sostenían, eran el suelo mismo de la Patria flotando sobre
las olas; la bandera que flameaba en los aires era el nombre, la
historia, el \emph{qué dirán} de los países extranjeros, el
\emph{primero soy yo}, que así gobierna las almas de los individuos como
las de los pueblos\ldots{} Bien merecían alabanzas los tres mil hombres
de mar comprometidos en aquella singular aventura inconsciente, más que
empresa meditada. No habían alcanzado aún, ni probablemente alcanzarían,
esa gloria brillante y ruidosa que traen consigo los hechos eficaces de
finalidad clara y bien comprensible. No se les podía disputar la gloria
obscura y pasiva, alcanzada por el valor silencioso y la paciencia, por
el cumplimiento del deber, sin más recompensa que la conciencia de
haberlo cumplido. Dignos eran de alabanza, y también de lástima, porque
sin ver ni aun de lejos los frutos de la campaña, se sentían agobiados
de privaciones y sufrimientos. Fueron penitentes en el desierto sin fin
de un mar enemigo.

Después de la dura lección a Valparaíso, la penitencia de los españoles
se acentuaba, sin que se agotara ni mucho menos el caudal de abnegación
que las almas llevaban consigo. Incomunicados con tierra, se alimentaban
de substancias secas, de carnes y tocinos en mediana conservación. El
tabaco, que hace llevadera la soledad y el exceso de trabajo, escaseaba
de tal modo, que cualquier porción de hierba fumable adquiría fabulosos
precios. Pero la falta de buena comida y de estimulantes no quebrantaba
la salud de los tres mil hombres tanto como la vida de continua ansiedad
y alarma en que todos vivían, obligados a una vigilancia minuciosa y sin
respiro. Fatigosos eran los días, cruelísimas las noches. Entre los
barcos de combate y los del convoy no se interrumpía el ir y venir de
lanchas, faena de hormigas presurosas, que acarreaban víveres,
utensilios de maquinaria. Era la escuadra como una ciudad que tenía
todos sus arrabales sobre el agua, y no precisamente en aguas
tranquilas, que algunos días la fuerte marejada dispersaba la procesión
hormiguera.

De noche, los hombres se consagraban a la silenciosa operación de
reconocimiento y patrulla, voltijeando en derredor de la ciudad
flotante, bien al remo, bien en la lancha vapora. Felices eran los que
por turno podían descabezar un sueño de media hora, sin manta, bajo la
acción de la humedad y el sereno. Y no había esperanza de descansar a
bordo, porque las primeras luces del alba traían imprevistas
obligaciones, a más de las tareas ordinarias. Ni los cuerpos se rendían,
ni las voluntades desmayaban. La rutina del deber en pie les mantenía,
esperando un reposo que bien podía ser el de la muerte.

Las sombras de tristeza que dejó en todas las almas el vapuleo de una
plaza inerme, cruzada de brazos ante el fiero castigo, no podían
disiparse sino repitiendo el ataque contra un enemigo armado de todas
armas, como era el Callao. ¿Qué hacían, que no iban corriendo allá? El
Perú les provocaba con la jactancia de sus baluartes novísimos y el
montaje de cañones potentes. Para acudir a la cita del furioso enemigo,
se esperaba el refuerzo de la fragata \emph{Almansa}. Felizmente, esta
se incorporó a la Escuadra el 9 de Abril, que fue día de gran regocijo y
algazara, porque todos echaron su cana al aire, recibiendo con
aclamaciones a los que venían de España de refresco, y traían, con las
memorias de la Patria, algo de comer, y de beber y de fumar. Mandaba la
\emph{Almansa} el Capitán de navío Sánchez Barcáiztegui, y venía muy
airosa y envalentonada: había hecho la travesía desde Montevideo a la
vela, por el Cabo de Hornos, con tan buena fortuna, que no se podía
pedir prueba más decisiva de su poder marinero\ldots{} Sin perder
tiempo, se dispuso la salida para el Callao en dos divisiones. ¡Otra vez
hacia el Norte, a lo largo de la costa, dilatada con prolongaciones de
pesadilla! ¡Otra vez la visión ensoñadora de los Andes, que parecían más
altos, más ceñudos, más enemigos de los que venían a turbar la juvenil
alegría de las repúblicas!

Hacia el Perú navegaban los tres mil con la ilusión de un acto decisivo
que pusiera fin a la campaña; ya era tiempo de tomar tierra en alguna
parte, aunque fuera en el más desolado rincón del mundo. Sobre esto
sostenían en la \emph{Numancia} largos coloquios Ansúrez y Fenelón, el
cual aseguró que sin mujeres no nos ofrece la vida ningún bienestar, y
que las guerras y revoluciones no son ni han sido nunca más que
movimientos instintivos de los pueblos para ir en busca de nuevo surtido
de mujeres, o para cambiar las conocidas por otras de ignorados
encantos. Al propio tiempo, a sus amigos repartía tabaco, obsequio
recibido del maquinista del transporte \emph{Uncle Sam}, que antes del
bombardeo de Valparaíso había llegado de San Francisco de California con
víveres. El tabaco era \emph{virginio}, de la clase fuerte, capaz de
tumbar la cabeza más firme y de volcar los estómagos más equilibrados;
pero por sus cualidades mortíferas lo estimaban y preferían los
marineros de blindadas fauces. Aceptaron estos muy agradecidos las
cortas raciones que Fenelón les daba, y hacían de ellas partijas para
obsequiar a otros amigos. Binondo tomó cuanto pudo, ocultando las
porciones recibidas para que le dieran otras, y así juntaba en previsión
de futuras escaseces.

Trabajaba el pobre malayo en ayuda de los mayordomos y rancheros,
llevándoles las cuentas, y en sus ratos de ocio se engolfaba en la
lectura, prefiriendo la del \emph{Sermonario}, a su parecer la más
devota, la más apropiada a la ruindad de los tiempos y a las calamidades
previstas. Muchos trozos de aquel libro, compuesto para socorro y guía
de predicadores, se le quedaron en la memoria, y vinieran o no a cuento,
a los compañeros los endilgaba. «Dame, hijo mío, limosna de tabaco, que
si no acudes a mi pobreza, no acudirá Dios a la tuya, que será el
desamparo en que te veas a la hora de la muerte si antes no te limpias
de tus pecados\ldots{} En verdad os digo que si no miráis por el pobre,
el pobre no mirará por vosotros, y os pondré el caso de un mendigo que
recibía zoquetes de pan, y era tan santo y bueno, que Dios le dio la
facultad milagrosa de multiplicar los mendrugos que recibía. Y sucedió,
pues, que en la ciudad donde aquel pobre moraba, llamada Gangópolis, si
no me falla la memoria, sobrevino una gran hambre desoladora, por el
aquel de un cerco que le pusieron los del reino vecino de Capadocia; y
hallándose todo el pueblo moribundo del no comer, presentose el mendigo
y mostró almacenes de pan, que era la milagrosa multiplicación de los
mendrugos, con otro milagro encima, a saber: que la dura masa se había
enternecido, y parecía recién sacada del horno\ldots{} Pues bien, hijos
míos: lo que hizo con los mendrugos aquel venturado de Dios, puedo
hacerlo yo con las hojitas de tabaco que me dais, y bien podrá suceder
que os las multiplique cuando llegue la gran carencia de todo lo
comible, bebible y fumable\ldots»

En estas y otras accidentales conversaciones y sucesos, indignos de la
historia, transcurrió el viaje. Si el mar y el viento fueron bonancibles
en toda la travesía, la inquietud de las almas crecía conforme se
aproximaban al Callao. En el momento solemnísimo de reconocer el puerto
peruano, Ansúrez no pensó en el duelo empeñado entre España y la plaza,
ni en la artillería y baluartes de esta. Mirando hacia tierra, veía tan
sólo los ardientes ojos de Mara, fulminando ira contra los barcos
españoles. ¡Ingrata, ingrata! ¡Y él, mísero padre, obligado a disparar
contra ella!

\hypertarget{xxiv}{%
\chapter{XXIV}\label{xxiv}}

Apenas llegaron al Callao las asendereadas naves españolas, los tres mil
(o los que fueran) que las montaban, no pensaron más que en acometer,
sin perder días, la militar empresa, apretandose a ello la noticia de la
fortísima resistencia que habían de encontrar y del grave daño que les
harían los cañones de monstruoso calibre traídos del viejo
continente\ldots{} La Escuadra echó sus anclas en el fondeadero de la
isla de San Lorenzo. No se le cocía el pan a Méndez Núñez hasta poder
enterarse por propio conocimiento de la fuerza y defensas de su
contrario; con esta idea montó en la \emph{Vencedora}, que por su poco
puntal podía ceñirse fácilmente a tierra, y recorrió todo el frente
fortificado y artillado, examinando las obras a que innumerables
trabajadores daban la última mano.

Al Norte de la ciudad vio don Casto dos baterías rasantes, con veinte
cañones la una, la otra con doce, y en medio de ellas una torre blindada
con dos piezas \emph{Armstrong}. En los extremos de la batería había
cañones del sistema \emph{Blakely}. Las baterías al Sur de la población
eran tres, y se extendían hacia la punta en cuyo término está el
\emph{Boquerón}, entrada del puerto para embarcaciones menores. En
aquella parte contó el General unas treinta piezas, entre ellas algunas
de los poderosos tipos antes citados, y vio otra torre blindada, como la
del lado Norte. Frente al muelle vio los monitores \emph{Loa} y
\emph{Victoria}, armados de cañones, y un \emph{Blakely} campaba en
mitad del muelle. Las viejas fortificaciones del tiempo del virreinato
estaban desartilladas, como indignas de desempeñar en las epopeyas
modernas otro papel que el de espectadoras. El \emph{Castillo del Sol}
parecía decoración de teatro, arrumbada por inútil. En él no había
piedra que no hablase del último \emph{ayacucho}, el heroico
Rodil\ldots{} Las defensas nuevas revelaban en su disposición y
estructura manos muy expertas y una dirección inteligentísima.

Mientras los peruanos no se daban punto de reposo para rematar sus
imponentes aprestos de guerra, los españoles, en el fondeadero de San
Lorenzo, no se descuidaban. Todos los barcos desmontaron sus vergas y
calaron los masteleros, dejando no más que los palos machos a la
exposición de los tiros enemigos. Algunas de las fragatas de madera
blindaron con cadenas la parte central de sus costados, correspondiente
a la caja de la máquina, y todas pintaron de negro las fajas blancas de
las portas. Interiormente se previno lo necesario y lo accesorio para
acudir a las eventualidades del combate, y las enfermerías de guerra
quedaron listas para recibir a cuantos heridos quisiera enviarles la
suerte adversa. Desde los cañones hasta los botiquines, todo fue puesto
en punto de servicio eficaz. No faltaba más que la acción, el fuego, el
ardor de las almas, y la divina sentencia que había de dar o negar la
victoria.

Falta decir que los diplomáticos extranjeros se presentaron al General,
apenas fondeó la Escuadra, con la súplica de que aplazara el ataque por
unos días para dar tiempo a la salvación de los neutrales. Méndez Núñez
concedió cuatro días, y en esto su generosidad de caballero fue más allá
que su precaución de caudillo, pues en media semana podía el Perú
perfeccionar sus medios ofensivos. La guerra había llegado a concretarse
en el trámite decisivo de un duelo personal entre los dos combatientes.
Incapaz la torpe diplomacia para dirimir las cuestiones pendientes entre
España y las Repúblicas; ciegos los Gobiernos de acá y de allá, y
encastillados en ridículos puntos de amor propio, quedó la Marina sola,
con toda la responsabilidad sobre sí, a tres mil leguas de la Patria, y
obligada a proceder con acción tanto diplomática como militar, hasta dar
por liquidada y conclusa una empresa cuya finalidad era tan obscura en
el terreno comercial como en el político.

Hizo don Casto cuanto pudo por sacar a su país de aquel atolladero
dispendioso. No hallando ocasión de batirse con las escuadras chilena y
peruana, fue a buscarlas a los caños y esteros de Chiloe. A esta
expedición ardua, que era un reto para que los enemigos salieran a mar
abierto, respondieron ellos encerrándose más en sus inabordables
refugios. Obligado se vio entonces al castigo de Valparaíso, acto de
penosa y desigual lucha, que a su corazón de soldado repugnaba; y
sabedor de que el Callao se pertrechaba de armas, allá corrió, anhelando
el duelo final y decisivo entre el viejo y el nuevo hispanismo, entre el
hemisferio Norte y el hemisferio Sur del planeta, que ya desde las
edades heroicas se conocían.

Al duelo final iban los españoles sin reparar en que el contrario se
había provisto de mayor fuerza que la de los barcos, con la ventaja de
combatir en tierra, en la cabecera de una Nación, de la cual obtendría
todo lo que perdiese mientras los españoles no tenían tras sí más que el
Pacífico inmenso, y en él los peces que se los habían de comer en caso
de un desastre\ldots{} En esto pasaron los cuatro días de plazo que
había dado el General para la retirada de los neutrales\ldots{} Gran
número de españoles que se habían refugiado en una fragata francesa
trasbordaron a la Escuadra, entre ellos el simpático Mendaro, que fue a
embarcar en uno de los transportes del convoy\ldots{} Serena y recamada
de estrellas habladoras fue en sus primeras horas la noche última del
plazo fatal; luego se enturbió de celajes, y en cerrada neblina amaneció
el día, más fatal que la noche, 2 de Mayo de 1866.

El mal de soñación se hizo epidémico, con gravísimos caracteres de
fiebre patriótica, al amanecer de aquel día que todos creyeron había de
ser glorioso. La embriaguez de martirio enardece a los cuerpos armados
en vísperas de batalla. Aún no han bebido la primera pólvora, y ya están
borrachos. Acabó de trastornar a marineros y tropa la proclama que a las
nueve de la mañana fue leída en todos los barcos, y era conforme al
patrón consagrado por la costumbre en casos tales. Con más laconismo del
que suelen usar los caudillos españoles, Méndez Núñez fijó los tópicos
imprescindibles, la perfidia del enemigo, la urgencia de castigarlo, la
recomendación de que todos se aplicaran al castigo con decisión y
entusiasmo, y, por fin, la seguridad de añadir una página a las glorias
de la Nación, etc\ldots{}

Terminada la lectura, todos aquellos infelices, quebrantados ya de la
navegación larguísima, mal comidos y sufriendo mil privaciones,
prorrumpieron en exclamaciones delirantes, declarando el gusto que les
causaba morir por una Reina que no habían visto nunca, y por una Patria
que a tres mil leguas de distancia no pedía otra cosa que la terminación
de la guerra insensata. Roncos quedaron del furioso entusiasmo\ldots{}
En el Callao, a la misma hora, pasaría lo propio, y se oirían
exclamaciones semejantes proferidas en la misma lengua. En tierra y en
el mar se invocaba el fantasma de la gloria, y allá como aquí se pediría
el auxilio de Dios y los Santos, que se habían de ver bien perplejos
para contentar a todos. Por de pronto, los peruanos habían puesto su
mejor batería bajo la tutela y patrocinio de Santa Rosa de Lima,
suponiéndola muy enojada con los españoles. Difícil era, no obstante,
que la santa, con ser de ideal hermosura mística, tuviese bastante
valimiento para lograr que quedase desairada la Virgen del Carmen, a
quien casi todos los marinos nuestros, verbal o silenciosamente, se
encomendaban.

Levaron anclas todos los barcos, y acudieron a las posiciones que les
designaba el telégrafo de banderas en el mesana de la \emph{Numancia}.
Esta y la \emph{Blanca} y \emph{Resolución} habían de batir las
fortificaciones del Sur; las del Norte corrían de cuenta de la
\emph{Berenguela} y \emph{Villa de Madrid}; la \emph{Almansa} con la
\emph{Vencedora} se encargaban de los monitores fondeados en el muelle,
así como de causar todo el estrago posible en el interior de la
población. La Capitana, a la cabeza de la división del Sur, llegó la
primera frente a las baterías enemigas. Claramente distinguían los
españoles las piezas peruanas y sus servidores, en pie junto a ellas con
rigidez marcial. Y apenas las vieron, disparó la \emph{Numancia} sus
primeros tiros, colocándolos en la batería que llevaba el nombre de
\emph{Santa Rosa}. Contestó sin tardanza el Perú. Tronaron luego las
demás fragatas, conforme iban llegando frente a las baterías, y bien
pronto el humo denso envolvió la tragedia, y un estruendo pavoroso
arrojó de los aires todo el silencio de la Naturaleza. El tiempo era
absolutamente olvidado. Sólo lo sabían los cronómetros, que al empezar
la función marcaban poco más de las once y media.

Desde la \emph{Numancia} no se podía saber con exactitud lo que pasaba
en el ala del Norte. El humo tapaba las partes lejanas, y no podía la
atención distraerse del cuidado próximo. No obstante, en una clara, se
vio que la \emph{Villa de Madrid} pedía remolque. Había quedado sin
gobierno por avería considerable. Acudió la \emph{Vencedora} con
prontitud a sacarla fuera, y la \emph{Berenguela} quedó sola cañoneando
las baterías y la torre blindada, cuyas piezas de gran calibre inutilizó
al poco tiempo. En el ala Sur, la \emph{Numancia} requería la mayor
eficacia de sus disparos aproximándose a tierra\ldots{} Pasó muy cerca
de los artificios que los peruanos habían dispuesto para inutilizar las
hélices; llegó a tocar en el fondo; tuvo que dar atrás
precipitadamente\ldots{} En aquel instante, la batería de \emph{Santa
Rosa} y la torre multiplicaban sus disparos contra la fragata. Méndez
Núñez, en el puente, acompañado de Antequera y un Oficial, en todo ponía
sus ojos vivos, y con ellos el alma.

Sereno casi siempre, risueño cuando veía el torbellino de humo y de
polvo que levantaban los parapetos de la batería llamada de \emph{Abtao}
al recibir los proyectiles de la \emph{Resolución}, iracundo al sentir
que su barco tocaba en el fondo, don Casto no perdía un instante la
majestad que sus graves funciones le imponían en medio de sus
subordinados y frente al enemigo. Al gritar \emph{¡Cía!}, su voz
dominaba la voz de los cañones\ldots{} La fragata salió al fin del mal
paso, removiendo con su hélice el fango de la bahía, y continuó la
función sin que la maniobra marinera interrumpiese el fuego. Méndez
Núñez hablaba con las dos fragatas de su división, como si ellas
pudieran entenderle. Era un acto instintivo, de que él mismo no se daba
cuenta en momentos tan críticos\ldots{} y no les hablaba por el nombre
de ellas, sino por el de sus Comandantes. «¿Qué haces, Topete? No te
acerques tanto\ldots{} Valcárcel, firme contra esa batería de
\emph{Abtao}, que con \emph{Santa Rosa} me entenderé yo\ldots{} Y los
tres a una tiremos contra la torre blindada\ldots» Cuando esto decía, un
proyectil pasó entre el brazo derecho y el costado del General,
rozándole\ldots{} Los astillazos que el mismo proyectil despidió del
pasamanos del puente y de la bitácora, causaron en las piernas de don
Casto heridas de menos importancia que la recibida en el brazo.

Que no era nada dijo, y lo mismo creyeron los que estaban a su lado. El
fuego arreciaba por una parte y otra; las baterías peruanas redoblaban
su furor. Pasaron minutos. Méndez Núñez, por la pérdida de la sangre que
del interior de la manga descendía enrojeciendo la mano, sufrió un
desvanecimiento; le sostuvieron los más próximos a su persona\ldots{} Se
le bajó al Alcázar\ldots{} Tomó el mando el Mayor General don Miguel
Lobo, sin decir palabra, pues la ocasión no permitía el rigor de los
trámites\ldots{} En el Alcázar acudieron en auxilio del General los
médicos Oliva y Gutiérrez, y cuatro marineros que le bajaron a la
enfermería. Tendiéronle en la cama\ldots{} Viendo que corría la sangre
por distintas partes de su cuerpo, palpaban los médicos aquí y allí para
reconocer los sitios lesionados; y cuando empezaban a desabotonarle
levita y chaleco, un marinero atrevido tiró de navaja, y cortando de
cuatro tajos la ropa, facilitó la operación de apartar las telas y
descubrir el cuerpo herido.

Al punto procedieron los facultativos a contener la hemorragia\ldots{}
En aquel punto llegaron a la enfermería vivas exclamaciones de la gente
de batería y cubierta. Había volado la torre blindada de los peruanos,
con terrible estruendo y espantoso escupitazo de humo, que por largo
rato impidió distinguir los efectos de la explosión. Fue que una granada
española penetró en aquel recinto, incendiando las grandes masas de
pólvora allí depositadas. Al disiparse el humo, se advirtió que la torre
estaba hundida, y en completa inutilidad sus terribles cañones. Luego se
supo que habían perecido los defensores de la torre, y con ellos el
popular Gálvez, Ministro de la Guerra, el Coronel Zabala, hermano de
nuestro General del mismo nombre, y otros militares de graduación. Cada
una de las tres fragatas que contra la torre disparaban se atribuía la
gloria de haber mandado proyectil que tan tremendo daño causó al
enemigo; pero Topete, que era el más próximo a tierra, sostenía su
derecho con razones que difícilmente podían ser debatidas. Cuando voló
la torre blindada, los cronómetros marcaban las doce y diez minutos.

\hypertarget{xxv}{%
\chapter{XXV}\label{xxv}}

Al poco tiempo de estar don Casto vendado y quieto la enfermería,
recobró todo el esplendor de sus facultades. Quieto estaba, pero no
tranquilo. Llamó al Oficial de la tercera división de la batería. «¿Qué
hay, Garralda? ¿Cómo va el fuego?»

---Muy bien, mi General. La torre de \emph{La Merced} ha volado. Ya no
hacen fuego más que cuatro o cinco cañones en \emph{Santa Rosa}.

---Ánimo, hijos míos. No desmayar. Yo estoy bien\ldots{} esto no es
nada. ¡Volada la torre! Es más de lo que podemos desear\ldots{} ¿De cuál
de los tres barcos sería la granada que causó ese desastre al
enemigo?\ldots{} Difícil será saberlo\ldots{} Pero yo juraría que la
mandó ese diablo de Topete\ldots{}

Díjole después Garralda que la \emph{Almansa} había inutilizado el cañón
\emph{Blakely} montado en el muelle. Luego preguntó Méndez Núñez si
había vuelto la lancha de vapor que, al mando de Lazaga, corría las
órdenes de un punto a otro. Poco antes de caer herido, el General había
ordenado que se le llevasen informes seguros de lo ocurrido en la
\emph{Villa de Madrid}. Antes de que se retirase Garralda entró Lazaga,
que así dio cuenta de su comisión: «Pocos disparos había hecho la
fragata contra la batería del Norte, cuando recibió por el costado de
babor una granada \emph{Armstrong}, que al estallar dentro de la batería
mató trece hombres; veintidós quedaron heridos por la metralla y cascos
que despidió el proyectil en su explosión. No paró aquí el desastre,
porque la misma granada, al chocar en el cabrestante, lanzó un molinete,
que fue a parar a la caja de calderas, destrozando el tubo conductor del
vapor. Esta avería no es grave; pero se necesita tiempo para repararla.
En todo el día de hoy la \emph{Villa} estará privada de movimiento. La
he dejado fondeada en la isla. Cuando me retiré, don Claudio, poseído de
furor, no paraba de maldecir su suerte.»

---Ha quedado sola la \emph{Berenguela} frente a las baterías del
Norte---dijo Méndez Núñez desobedeciendo al médico, que le recomendaba
tranquilidad.---Corra usted a la \emph{Almansa}, y dígale a Barcáiztegui
que inmediatamente vaya en apoyo de Pezuela.

Salió Lazaga más pronto que la vista\ldots{} Continuaba el cañoneo, y su
fragor indecible retumbaba de un modo pavoroso en el hospital de sangre.
El techo de este era por la cara superior suelo de la batería. El
estruendo de los disparos, las pisadas de los que servían las piezas,
los gritos de los oficiales que mandaban las cuatro divisiones, los
alaridos y voces de guerra de tantos hombres iracundos, sonaban dentro
de las cabezas de los infelices que allí yacían malparados. La batería
era el Infierno, y la enfermería su catacumba, encierro de los
condenados a la duda de vivir o morir. En el fondo del lúgubre sollado,
a proa, se distinguía, entre faroles, la figura triste del Capellán con
sotana y roquete, dispuesto para dar los Santos Óleos a quien los
hubiese menester. A su lado, como acólito, estaba Binondo de rodillas,
esperando, quizás deseando entrar en funciones.

El amigo Ansúrez tenía su puesto en el más profundo sollado, rigiendo a
los que conducían la pólvora y municiones desde los pañoles a la
batería. Hallábase, pues, debajo del agua, en un punto en que no podía
ver el espectáculo del combate, y sólo lo apreciaba por el ruido. A cada
instante creía que el cielo se desgajaba sobre la tierra y el mar, o que
las profundidades del barco eran el interior de un volcán. A ratos
trepaba por la escala llegando hasta la enfermería, y echaba un vistazo
a los heridos, deteniéndose con singular lástima y atención en el
General, que fue de los primeros en quedar fuera de combate. Y era, sin
duda, el herido de más consideración. Los demás no eran muchos ni
graves. Ningún proyectil había hasta entonces entrado por las portas:
todos habían perdido su fuerza en la coraza.

Pero llegó al fin, cuando Dios quiso, una granada \emph{Armstrong}, que
habría causado inmenso daño, quizás la inmersión violenta de la fragata,
si no la protegiera la robusta armadura que llevaba sobre sus lomos.
Eran las dos y media de la tarde, cuando un topetazo monstruoso hizo
retemblar la embarcación, como si fuera de hojalata. Ansúrez, que en
aquel momento bajaba al tercer sollado, sintió el golpe por estribor, en
un punto a su parecer correspondiente a la línea de flotación, debajo de
la batería, entre la cuarta y quinta porta contando desde popa. Al punto
creyó que su fragata se rompía en mil pedazos, y que todos bajarían sin
pérdida de tiempo a los profundos abismos\ldots{} Sacristá, que se
hallaba en el tercer sollado, fue el primero en determinar el sitio del
tremendo choque, y como los duelistas de esgrima gritó: «¡Tocado!»
Fácilmente se apreciaba por dentro la caricia de proyectil. La cuaderna
presentaba una sensible alteración de su curva; un tornillo de los que
sujetan el blindaje había horadado la plancha, abriendo una vía de agua
de escasa importancia. Acudieron los oficiales de mar a reparar el
desperfecto y restañar el agua, que poquito a poco se colaba dentro.
Para ello emplearon cemento y ladrillos, que son la cura quirúrgica que
en estos casos se emplea, añadiendo limadura de hierro para mayor
eficacia. El emplasto quedó hecho en poco tiempo, y la \emph{Numancia},
que apenas sentía el escozor de la herida, gracias al peto y espaldar de
su armadura, invocó a Nuestra Señora del Carmen y siguió tan fresca
disparando balas, granadas y demonios coronados contra \emph{Santa
Rosa}.

«Gracias a la Virgen de Carmen---dijo Sacristá,---esto no ha sido nada.»

---La Santísima Señora---observó Ansúrez---ha sido la salvación del
barco, poniéndose a nuestro lado en forma y substancia de blindaje.
Bendita sea la Virgen y los que inventaron estas vestiduras de hierro.

Subió Ansúrez, llamado por el General, a informarle de la reparación de
la avería, y antes de que concluyese, llegó por segunda vez Lazaga con
la noticia del casi milagroso caso de la \emph{Berenguela}, que fue de
este modo: «Sola frente a las baterías del Norte, después de la retirada
de la \emph{Villa}, siguió cañoneando la veterana \emph{Berenguela}, y
logró inutilizar los cañones \emph{Armstrong} de la torre blindada. Pero
luego le tocó una china de las gordas, un proyectil \emph{Blakely}, que
entró por la porta como en su casa, destrozó a muchos hombres, y
corriendo en dirección oblicua, fue a salir por el costado opuesto
debajo del agua. Al salir se llevó una tabla, abriendo brecha enorme,
por la cual se precipitó una cascada que en minutos habría inundado el
barco, si la Providencia y la tripulación no acudieran con prontitud al
único remedio posible en tales casos. Antes de que se les diera la
orden, los marineros llevaron los cañones a brazo\ldots{} ¡a brazo,
parece mentira!, de la banda de babor a la de estribor, para escorar la
embarcación, sacando así del agua la brecha\ldots{} Y estando en esta
faena, entró en el sollado otra bomba que al reventar hirió a mucha
gente y pegó fuego a las carboneras\ldots{} La enfermería, llena de
víctimas, se vio asaltada del agua y del fuego\ldots{} los pobres
heridos gritaban con espanto entre los dos horrores: morir ahogados o
morir quemados\ldots{} Por momentos estuvo la fragata a dos dedos de
irse a pique\ldots{} Gracias a la rapidez con que los cañones pasaron de
un costado a otro, se salvaron el barco y sus hombres de una muerte
segura. Escorada se retiró de la acción, y apagó con el trajín de bombas
su propio fuego. Fondeada y segura está ya en la isla, tapándose el
boquete con lonas hasta encontrar maderas para echarse unas buenas tapas
y medias suelas. Las bajas son muchas: no he visto propiamente muertos,
pero sí hombres muriéndose.»

---Esto va bien, hijo mío---dijo don Casto estrechando la mano de su
subalterno.---Yo me encuentro regular. Me pone nervioso el verme preso
en este camastro\ldots{} Pero estoy contento\ldots{} Adiós, hijo; vamos
bien\ldots{}

Las ironías de la guerra revoloteaban como avecillas negras y doradas en
torno al lecho del General. Con su canto seductor infundían alegría en
el relato de los hechos luctuosos, y matizaban de gloria la cruel muerte
y los sufrimientos humanos. Quedó solo el General con Pastor y Landero,
que le dio cuenta de cuanto arriba, en el Estado Mayor, ocurría. Lobo y
Antequera permanecían en el castillo de popa con los Tenientes de Navío
Lahera y Basáñez. Alonso mandaba la batería; Barreda continuaba en
funciones de Segundo; Pardo Figueroa estaba en cubierta. Las cuatro
divisiones de batería seguían a las órdenes de los Alféreces de Navío
Liaño, Garralda, Silva y Armero, con los Guardias marinas. Todo el
personal se encontraba ileso. Íbamos bien, muy bien. Entró después
Lahera, y con él el ingeniero don Eduardo Iriondo; ambos ponderaron las
condiciones inmejorables de la fragata. Era un barco invencible; el
combate, aún no concluido, daba la mejor prueba de la eficacia del
blindaje. Con otras dos \emph{Numancias} sobre la que teníamos, la
destrucción de las defensas de Callao habría sido obra de
minutos\ldots{} Los barcos de madera ya no podían entrar en fuego con
fortificaciones modernas, sin llevar dentro de sus tablas mayor grado de
heroísmo del que debe exigirse a le hombres de guerra: eran héroes de
vocación y mártires a sabiendas. No debemos ir desabrigados contra el
frío, ni desnudos contra el fuego. La realidad nos demostraba que sin
una escuadra compuesta totalmente de \emph{Numancias}, no iríamos a
ninguna parte. Las consideraciones y las ideas técnicas no podían seguir
adelante, que era ocasión de aplicar todo el entendimiento al empirismo
inmediato. Lahera trajo al General la noticia de que la \emph{Blanca} se
retiraba por habérsele acabado las municiones. Topete estaba herido, no
de gravedad\ldots{} De la \emph{Almansa} se tenían noticias ciertas. En
su batería reventó una granada, matando trece hombres. El Guardia marina
Rull quedó hecho pedazos, y al instante le sustituyó otro Guardia
marina, Hediger, que antes sirvió en la \emph{Villa de Madrid} y en la
\emph{Numancia}. Al estrago de la explosión siguió el incendio de la
pólvora de los guarda-cartuchos; los que conducían las cajas quedaron
abrasados; el fuego se extendió rápidamente hasta el antepañol de la
\emph{Santa Bárbara}\ldots{} El fuego no se apaga sino con agua\ldots{}
Urgía inundar el sollado, abriendo los grifos\ldots{} Prodújose entonces
una terrible situación dramática. ¿Qué era preferible? ¿El peligro
evidente de volar, o el desaire de suspender la lucha? Esta duda
fatídica inspiró al animoso Barcáiztegui una frase que había de ser
célebre: \emph{Hoy no mojo la pólvora}\ldots{} Así fue: retirose la
fragata; fue extinguido el incendio sin mojar la pólvora, y antes de
media hora ya estaba otra vez frente a las baterías del Norte vomitando
contra ellas todo su coraje.

Las cuatro y media marcaban los cronómetros, cuando ya sólo tres cañones
peruanos tenían voz y balas. La noche estaba próxima. Enterado de todo,
Méndez Núñez dijo a Lahera y a Pastor: «Mi opinión es que se dé por
concluido el combate.» Poco después, Lobo mandó hacer la señal de que
cesara el fuego. Subió a las jarcias la marinería, y dio tres vivas a la
Reina, que fueron el último aliento del furioso Marte en aquel terrible
día. Los barcos españoles se retiraron tranquilamente al fondeadero de
San Lorenzo. Durante la corta travesía de la \emph{Numancia}, Méndez
Núñez fue llevado de la enfermería a su cámara, donde el Mayor General
le dio cuenta del resultado total de la acción. Ambos lo conceptuaron
lisonjero, pues sólo el hecho de no haber perdido ningún barco
significaba una indudable victoria. Declaró Lobo que los peruanos se
habían conducido con bravura y tesón. Calculaba que sus bajas habían de
ser superiores a las nuestras, y sólo con la torre de la \emph{Merced}
tenían para llorar un rato y para hacer cuenta larga de desdichas. Pero
a pesar de esto, no podían negar que en el duelo de aquel día todas las
ventajas fueron suyas, y nuestras las mayores desventajas. Combatían en
tierra, alentados por la opinión próxima, en un ambiente de entusiasmo,
con todo un pueblo por reserva. Sus artilleros podían hacer buena
puntería. Los combatientes tenían retirada segura hasta los Andes, y aun
más allá. En cambio, los barcos españoles no veían más retirada que la
mar, sin recursos de vida, sin medios de reparación para los hombres
extenuados y los buques maltrechos, faltos de todo.

Mientras navegaban hacia la isla, Ansúrez no apartaba sus ojos de la
plaza y sus baterías, en las cuales era visible el estrago causado por
las balas de los españoles. Con inmensa piedad miró hacia tierra, como
si entre los muros rotos y entre las ruinas humeantes viese despojos de
seres amados, o algún ser vivo ligado a él con vínculos estrechos. Como
estaba el hombre con los codos apoyados en la batayola y el rostro
vuelto hacia la tierra, que a cada instante se alejaba más por la
neblina y la distancia, nadie pudo ver las lágrimas que resbalaban por
sus curtidas mejillas. Lloraba de remordimiento de haber cañoneado a los
suyos, a su hija, a su nieto, a los demás de la familia, que también se
habían hecho suyos. ¿Quién le aseguraba que alguno de ellos, tal vez la
propia Mara, hallándose por casualidad o de intento en el Callao, no
había sido cogido por las balas que mandó con tanto furor la
\emph{Almansa} contra las casas del pueblo?\ldots{} Y sobre todo, Señor,
¿quién había inventado aquella maldita guerra, y quién dispuso las cosas
de modo que él no pudiese odiar al Perú, ni tenerlo por enemigo? ¿A qué
venía tanta furia contra el pobre Perú, delicioso país sin duda, por el
hecho de estar en él la hermosa Mara?\ldots{}

Momentos después de estas tristezas y reflexiones, vio a Fenelón, que de
la máquina salía jadeante, pintado el rostro de grasienta negrura. Había
hecho servicio durante todo el combate\ldots{} Más fatigado de la
suciedad que del trabajo, buscaba un cubo de agua con que baldearse y
recobrar su ser ordinariamente limpio. «¿Qué cuentas, Fenelón?---le dijo
el celtíbero.---¿Qué opinas tú de esto?»

«Que por una parte y otra, todo ha sido una función de\ldots{}
romanticismo\ldots{} ¿Consecuencias, dices? Ninguna, como no sea esta:
que se retrasará un cuarto de siglo, lo menos, la reconciliación de
España con las que fueron sus colonias. El combate de hoy ha sido,
\emph{por ejemplo}, el acto final de una guerra en verso\ldots{} No
pongas esa cara de asombro. Acá nos han mandado para que cantemos una
oda en el Pacífico. Los americanos han respondido con otra
canción\ldots{} \emph{y he aquí todo}\ldots{} Ahora España envaina sus
versos, y se va por esos mares a la casa paterna, donde también habrá,
cuando lleguemos, poesía a todo pasto.» Dicho esto, el francés dio con
un cubo de agua, y requiriendo un pedazo de jabón, empezó a fregotearse
con furor de limpieza.

\hypertarget{xxvi}{%
\chapter{XXVI}\label{xxvi}}

No cesaba el cuitado Ansúrez de voltear en su mente la idea sugerida por
Fenelón de que toda la guerra y el combate final eran cosa romántica,
como la fuga de Mara con Belisario, como el trasplante al Perú de la
prenda de su corazón, y como la fabulosa riqueza y felicidad indudable
de la niña en América. Hay, sin duda, romanticismo público y nacional,
como lo hay privado y doméstico. Las naciones hacen versos lo mismo que
esos vagos que llaman poetas\ldots{} En la siguiente mañana, las
obligaciones de su cargo le llevaron a un acto tristísimo, por su propia
tristeza y desolación empapado en idealidad romántica. Encargado del
transporte de muertos a la isla de San Lorenzo, donde se les daría
cristiana sepultura, salió Diego de la \emph{Numancia} en la lancha
vapora, y fue de barco en barco recogiendo los botes en que ya estaban
depositados los cadáveres, y dándoles remolque hasta el desembarcadero.

La solemnidad de dar tierra a las cuarenta y tres víctimas del combate
del Callao, dejó en el alma del contramaestre una impresión angustiosa.
Desde el amanecer ya estaban en tierra unos veinte hombres cavando las
sepulturas de sus compañeros. A los dos guardias marinas, Godínez,
muerto en la \emph{Villa de Madrid}, y Rull, en la \emph{Almansa}, se
les enterró envueltos en la bandera nacional. Los cabos de cañón,
condestables y marineros, fueron al hoyo con la misma vestidura, pero
ideal, porque para tantos no había banderas. Asistían a la ceremonia un
Oficial y un Guardia marina de cada barco, y presidía el Segundo
accidental de la \emph{Numancia}, Teniente de Navío don Emilio Barreda.
Los capellanes de todas las fragatas, arrimados a las sepulturas, daban
al viento el tristísimo latín de los responsos, más fúnebre cuanto menos
entendido. José Binondo, que fue de los primeros en la cava de los
hoyos, y en el apañar y soterrar a los pobres difuntos, se multiplicaba
como si le nacieran muchos brazos para las operaciones mecánicas y bocas
muchas para los rezos en castellano y latín macarrónico que a cada
muerto dedicaba. Para rematar dignamente el acto religioso, se puso en
mitad del terreno de las sepulturas una cruz de madera pintada de negro,
que a toda prisa carpinteó un calafate de la \emph{Numancia}. Ansúrez
habíala llevado en la vapora. Binondo ayudó a clavarla en tierra,
afirmando su base con pedruscos.

«Yo te aseguro---dijo a su amigo mientras le ayudaba en la colocación de
piedras---que al llorar a nuestros queridos compañeros difuntos, debemos
también envidiarlos, porque ellos están ya gozando de Dios, y nosotros
aquí quedamos como pobres desterrados, navegando y muriendo, sin
morir\ldots{} Porque ya ves; nuestra vida no es vida, sino más bien
muerte, y nuestro comer es ayunar, y nuestras alegrías penas y
quebrantos. ¿No valdría más que nos echaran al agua de una vez para que,
ya que nosotros no comemos, comieran los pobres peces?\ldots{} Dios
cuida, ya lo sabes, de dar su diario sustento al pajarillo y también al
pececillo\ldots{} y quien dice pececillos, dice ballenas, tiburones y
tintoreras\ldots{} En verdad te digo que debemos envidiar a los muertos,
porque, al morir por la bandera, quedaron absueltos de sus culpas, y en
la gloria están todos ya, salvo algún renegado a quien echen cuarentena
en el lazareto del Purgatorio.»

---Si ellos están absueltos y mondos de pecados---dijo
Ansúrez,---también nosotros, que sobre lo ya sufrido tenemos lo que aún
nos espera en estos malditos mares. Tierra firme paréceme a mí que ya no
pisaremos. Y viviendo en el mar, trashijados de hambre, nuestros víveres
son las ilusiones y nuestra bebida la poesía, que más emborracha que
alimenta.

---Verdad. ¿Pero qué te importa si así eres feliz? Has llegado a creerte
que tu hija vive, cuando está más muerta que mi abuela; crees también
que nada en plata y oro, cuando ya no puede nadar en cosa alguna, como
no sea en la divina misericordia\ldots{} En verdad te digo que no te
salvarás si no te haces amigo de la muerte. Aquí me tienes a mí deseando
siempre que me llegue la hora\ldots{} Vivo muriendo\ldots{} o como dijo
la otra, muero porque no muero.

---Déjame en paz, farsante, y guárdate tus sermones---replicó Diego
cogiéndole por el pescuezo,---que entre poesía y poesía, prefiero yo la
que me alegra el alma\ldots{} Y dime ahora: ¿todavía rezarás a Santa
Rosa, que nos estuvo abrasando con los cañones de su batería, hasta que
Topete y la Virgen del Carmen le metieron en la torre una granada?

---Yo le rezo a la Santa, pero con reservas. Rosa se llamó en el mundo
mi querida hija\ldots{} Yo les rezo a las dos Rosas, y hago mi
separación de cañonazos y santidad. A este lado la guerra, al otro las
ganas que tengo de salvarme. Nada tiene que ver el Credo con las
témporas\ldots{} Si la Virgen del Carmen mira por los españoles y Santa
Rosa por los peruanos, allá ellas. Yo, Pepe Binondo, me pongo todo en mi
alma, y al cuerpo mío, que es témpora, le doy un puntapié y le digo:
«Muérete, cuerpo asqueroso. Cómante peces o meriéndente gusanos, lo
mismo me da. ¡Viva mi alma, y \emph{amén!»}

---Buen tuno estás tú\ldots{} Acaba pronto y vámonos a bordo---le dijo
Ansúrez tirando de él. Embarcados en la lancha vapora, siguieron
charlando. Binondo no soltaba el hilo de sus estrafalarias teologías;
pero Ansúrez le llevó a un tema más positivo, anunciándole que si se
concertaba un armisticio con el Perú, podrían los españoles hacer
provisión de comida fresca y abundante; a lo que respondió el malayo,
con verdoso fulgor en su mirada de santo budista: «Buena falta
hace\ldots{} En verdad te digo que el comer es necesario hasta para la
devoción, pues un estómago vacío trastorna el entendimiento, y si la
cabeza no gobierna como es debido, puede uno llegar encandilado a la
muerte, y no ver la puerta de la salvación.»

Para que no tuvieran aquellos infelices ni un momento de descanso, las
reparaciones de los barcos descalabrados en el combate les ocupaba día y
noche, sin desatender el trajín de aprovisionamiento de carbón y
víveres. Por ser la comida escasa y mala, el repartirla daba mucho que
hacer. Lo menos malo era para los heridos, que no bajaban de ochenta,
con añadidura de sesenta y tantos contusos. En uno de los barcos del
convoy, llamado \emph{Mataura}, tuvo Ansúrez el gozo de encontrar a su
amigo Mendaro. Las desdichas por ambos sufridas les desbordaron en una
conversación calurosa, interminable, sobre lo divino y lo humano, sobre
lo privado y lo público. Refirió Mendaro que sus parroquianos habían
dado en llamarle espía, y su misma esposa, Josefa, le quemaba la sangre
a toda hora, hablando pestes de la Reina doña Isabel. Por más que él
guardaba la mayor compostura, y no se permitía públicamente decir
palabra que sonase mal en oídos peruanos, a cada paso le injuriaban,
azuzándole con dicterios soeces. Antes de que le expulsaran se expulsó
él a sí mismo, con propósito de regresar a su casa en cuanto los barcos
españoles volvieran la espalda, dígase las popas. El hervor del
patriotismo peruano pasaría pronto, que en aquella tierra, como en
España, no había constancia en el odio, lo que es signo de buen natural.

De estos y otros temas particulares pasaron Mendaro y Diego a los de
interés colectivo: se habló largamente del combate del día 2, del coraje
y valentía que unos y otros desplegaron, de la catástrofe en la torre de
la Merced, del brío y agilidad de las fragatas, terminando en
consideraciones y barruntos de lo que sobrevendría. ¿Duraría más tiempo
la guerra o se hallaba ya en su conclusión y finiquito? Esto era lo más
probable y la opinión corriente en la Escuadra, donde todos sentían la
imposibilidad de mayor resistencia. La comida escaseaba y era de la peor
calidad. ¿A dónde irían en busca de víveres frescos? Dijo a esto Mendaro
que en el tiempo que llevaba en el convoy su constante pensamiento era
comer algo más nutritivo y grato; dormía mal, con ensueños de oler y
gustar un buen \emph{sancochado} y un platito de \emph{seviche}, que es
pescado crudo con zumo de limón.

«Pues yo---dijo Ansúrez---sueño que estoy en Cartagena, comiendo
pimientos y \emph{aladroque}, y al despertar paréceme que conservo en la
boca el gusto de aquellos comistrajes tan sabrosos\ldots{} Yo creo que
la guerra se ha concluido, y que vendrán pronto las paces.»

Opinó Mendaro que la paz no podían hacerla los españoles allí presentes,
sino otros que mandaría después el Gobierno con más papeles que
cañones\ldots{} A este propósito, repitieron lo que en la Escuadra se
daba como hecho corriente, divulgado de boca en boca. En sociedad tan
estrecha y cordialmente unida como las tripulaciones de los barcos, no
había nada secreto, y las disposiciones del Gobierno de Madrid, apenas
llegaban al Pacífico, eran conocidas y comentadas en la España flotante
y en su vecindario de tres mil almas, algo mermado ya por las bajas de
la guerra. El hecho que debe ser puesto aquí, como guión de los que
marcan el paso de la Historia, fue el siguiente: Nuestro Gobierno de
entonces, ni más cauto ni más animoso que los que le precedieron y
después le heredaron, se sintió de súbito aterrado de la prolongación
dispendiosa de la campaña del Pacífico. Quizás vio, tarde ya, la locura
de haberla emprendido por un impulso de pueril fiereza, cediendo a los
estímulos de la moda imperialista (segundo Imperio francés) que a la
sazón reinaba, moda que imponía con los miriñaques otras cosas vanas,
como la hinchazón de guerras sin sentido común, para deslumbrar y
dominar más fácilmente a los pueblos. Conocidos el error y la tontería,
no vio el Gobierno más camino de arreglarlo que decretar la terminación
de la campaña; y al efecto, mandó al Pacífico al señor Álvarez de
Toledo, Alférez de Navío, con pliegos para Méndez Núñez, ordenándole el
\emph{inmediato regreso} de la Escuadra.

Defectuoso y precipitado era este modo de concluir, como fue impensado y
calaveresco el modo de empezar. El Enviado español tomó el camino más
corto, que era el de Panamá, y en el Callao apareció el 1.º de Mayo,
cuando ya la Escuadra española estaba haciendo puntería, como si
dijéramos, contra las defensas de la plaza. Y véase aquí cómo procede un
caudillo valiente que tiene en su mano la bandera de su país y el honor
de las armas. Méndez Núñez leyó el papel, y devolviéndolo al mensajero
le dijo: «Mañana 2 bombardeo al Callao. Usted no ha llegado todavía;
llegará pasado mañana, y en cuanto me comunique la orden del Gobierno,
me apresuraré a obedecerla.» Así se hizo. ¡Honor a los hombres que, en
circunstancias tan solemnes y críticas, saben desobedecer obedeciendo!

\hypertarget{xxvii}{%
\chapter{XXVII}\label{xxvii}}

De este suceso, del grande ánimo de General y de su heroica marrullería,
hablaron los dos amigos extensamente, tratando luego de los medios de
proporcionarse algún alimento de mediana calidad y frescura. Pero la
requisa escrupulosa que hicieron de despensa en despensa no dio
resultado alguno. Separáronse, y cada cual fue a entretener y amodorrar
su hambre con las obligaciones. Ansúrez se aplicó a la faena de la
reparación de averías en los barcos de madera.

En la agitación de estos trabajos les sorprendió la noche del 5, que fue
de gran alarma y ansiedad, porque vieron confirmado el temor de que les
atacaran con torpedos u otros aparatos infernales y traicioneros.
Gracias a la vigilancia con que a estos riesgos atendían, pues aquella
pobre gente no descansaba en las noches claras ni en las obscuras,
pudieron librarse de una catástrofe. La \emph{Berenguela} fue la primera
en anunciar con cañonazos el peligro. A favor de las tinieblas se
aproximaba un remolcador conduciendo una barcaza en que venía el
torpedo, diabólico artefacto lleno de fulminante, que por medio de un
sutil mecanismo, al chocar con un cuerpo duro se inflamaba y hacía
terrible explosión, pudiendo así destruir la nave más poderosa. La
Providencia, que a los españoles favorecía en aquellos angustiosos días
de trabajar duro y apenas comer, deshizo el plan siniestro de los que
habían armado el bárbaro artificio. Una bala de la \emph{Berenguela}
rompió la palanca que debía transmitir al depósito de explosivos los
efectos del choque, y el torpedo quedó ineficaz. A la mañana siguiente
pudieron desmontarlo con minuciosas precauciones, y salieron al fin
ganando, porque el vaporcito que traía la muerte quedó con vida
incorporado a la Escuadra. ¡Lástima que en vez de enviar vaporcitos
portadores de fulminante, no los mandaran cargados de jamones, pavos,
manteca fresca y demás pólvoras alimenticias!

Deseaban Sacristá y Ansúrez visitar al General para felicitarle por su
mejoría y recibir sus órdenes, y antes de que pusieran en ejecución este
noble pensamiento, Méndez Núñez les mandó llamar. Ello debió de ser el 7
o el 8 de Mayo. Halláronle levantado, el brazo en cabestrillo, pálido y
decaído de fuerzas físicas, ya que no de ánimos. Con su bondad ingénita,
que en el trato de los inferiores generosamente se mostraba, les
recomendó que se previnieran para un viaje larguísimo y tal vez de
contingencias desfavorables. «Al retirarnos de estas aguas---les
dijo,---no podemos seguir juntos\ldots{} Yo me voy en la \emph{Villa de
Madrid}, con la \emph{Blanca}, \emph{Resolución} y \emph{Almansa}, a Río
Janeiro; vosotros, con la \emph{Berenguela}, emprenderéis la derrota de
Filipinas, para seguir luego hasta España por el Cabo de Buena
Esperanza. Ya veis: ocasión se os presenta de mostrar otra vez que sois
excelentes marineros. Lo que hicisteis para ayudarme a traer acá esta
fragata, repetidlo ahora\ldots{} No me arriesgo a llevar la
\emph{Numancia} conmigo, porque ha de ser muy difícil embocar en esta
estación la entrada occidental del Estrecho. Hemos de ir por el Cabo de
Hornos y a la vela. ¿Quién nos dará carbón de aquí a Montevideo?
Vosotros llevaréis mejor camino, y antes de llegar a Filipinas haréis
escala en alguna isla de Archipiélago de la \emph{Sociedad}\ldots{}
Menester será emplear la vela el mayor tiempo posible, porque no
llevaréis carbón más que para algunos días. Viento de popa y corriente
favorable tendréis al salir de aquí; navegaréis con rumbo Sudoeste hasta
los 17 grados; luego, al Oeste: la corriente os ayudará a llegar a las
islas. Ocupaos hoy mismo en guindar todo el aparejo, asegurando los
estáis y poniendo al corriente todo el juego de brazas de los tres
palos, que si os cogen calmas, habréis de largar todo el trapo y las
arrastraderas. Repasad bien el velamen, y si hay que hacer reparación en
las gavias, no os descuidéis: lona tenéis de sobra\ldots{} Me figuro que
habréis de dar algunas puntadas en las mayores y en los foques, que
bastante trabajaron para traernos acá\ldots{} Y nada más os digo, porque
os conozco, y sé que sabéis cumplir con vuestro deber\ldots{} Deseo que
podamos volver a vernos allá. Ello no es fácil, porque como de esta
hecha hemos quedado todos, cuál más cuál menos, bastante estropeaditos,
y heridos del corazón tanto como de los remos, no será extraño que
algunos vayan cayendo al agua por el camino. Sea lo que Dios quiera.
Amigos, hasta Cádiz\ldots{} o hasta el Valle de Josafat.»

Con emoción y gratitud salieron de la cámara del General los dos
contramaestres. La llaneza bondadosa de don Casto les afianzaba en el
cariño que por él sentían, y era el mejor estímulo para el cumplimiento
de cuanto les mandaba. Sin perder tiempo se consagraron a guindar toda
la arboladura, y a disponer el velamen, que pronto había de ser
entregado a las caricias del viento. Después de trabajar como negros en
estas operaciones, cayó el buen Ansúrez en hondas melancolías. La idea
de abandonar las aguas peruanas sin poder saltar a tierra, le abrumaba.
¿Qué razón había para que el General no hiciese paz honrosa con el Perú,
echando pelillos a la mar, sin pensar más que en la reconciliación de
dos pueblos hermanos? ¡Ajo! ¿Para cuándo dejaban el tierno abrazo de
americanos y españoles? Retirarse a España dejando las cosas como
estaban, era una mala partida, un pastel indecente\ldots{} ¡una
traición, con cien mil pares de ajos! No había consuelo para el infeliz
padre cuando pensaba que tenía que volverse a Europa dando al mundo la
vuelta grande sin ver a su hija y abrazarla. ¡Ni siquiera le permitía
Dios el mezquino placer de comunicarse con ella, de recibir cuatro
renglones trazaditos en un papel por su linda mano! ¿Qué crímenes había
él cometido para estar condenado a dar vueltas alrededor del globo sin
ninguna pausa ni alivio de su inmenso pesar? Esto era horrible, Señor;
esto traspasaba los límites del dolor humano. Mejor que esto era el
Infierno; mejor el Limbo, con su privación eterna de bienes y males.

Para mayor tortura del pobre celtíbero, hasta la consoladora visión del
niño Carmelo había desaparecido. Por más que se esforzaba en traer a su
imaginación la angelical persona del nietecillo, no podía disfrutar de
aquel consuelo. La imagen alada y sutil se escapaba, se escabullía,
perdiéndose en los espacios más remotos del ensueño. «¡Señor, Virgen de
Carmen---decía clavándose los dedos en el cráneo,---si será todo
mentira!\ldots. ¡si me habrá engañado el maldito francés y los que
declararon que mi hija estaba en Jauja, en el Cuzco, en Arequipa, o en
las Batuecas de los Andes! ¿Serán también una farsa los versos con que
quisieron darme fe del alumbramiento de la niña? ¡Ajos!, no me falta más
sino que tenga razón ese puerco mojigato de Binondo, que me asegura la
muerte de Mara y su viaje al otro mundo para no volver de él. Sáqueme
Dios de estas dudas, o me entregaré a los demonios para que me cojan, me
zarandeen, y me zambullan en sus calderas de plomo derretido.»

En esta consternación y turbulencia de su espíritu estaba el hombre sin
ventura, cuando llegose a él Mendaro, que a despedirse iba. Llorando a
moco y baba se echó Ansúrez en brazos de su amigo, y le dijo: «Pepe de
mi alma, por lo que más quieras; por tu mujer guapetona, que perece una
reina, por el príncipe tu hijo, ten compasión de este padre desgraciado,
y en cuanto vuelvas a tu casa, busca el medio de ponerte al habla con
Mara o con su familia; revuelve a Lima, a Jauja y al piñatero Cuzco
hasta dar con ella. Si para esto necesitas gastar algún dinero, aquí
tienes todo el que guardo de mis pagas\ldots{} No dudo que me harás este
favor, hijo: yo te lo agradeceré mientras viva\ldots{} Y si logras ver a
esa ingrata, cuéntale mis amarguras, y hazle ver lo que he penado por
ella, y lo que aún me falta, ¡ajo!, que es mucho dolor este de volver a
España por la vuelta de Filipinas y el Cabo de Buena Esperanza sin ver a
mi hija, sabiendo que está en el Perú\ldots{} No sé, no sé cómo
consiente Dios este desavío tan grande\ldots{} ¡Y para esto ha hecho el
hemisferio Sur y el hemisferio Norte, y los caminos de la mar! Navegue
usted nueve mil millas, fondee delante del Perú, y resígnese a navegar
ahora veinte mil millas sin ver logrado un deseo tan natural y tan santo
como es el abrazar un padre a su hija\ldots{} Yo le digo a Binondo que
no hay Dios, y que si lo hay está trastornado de su eterno
caletre\ldots{} Y si no lo estuviera, ¿cómo había de permitir estas
guerras estúpidas, que no son más que bambolla y quijotismo? ¿Qué
ventajas nos da el sin fin de bombas y granadas que hemos tirado contra
esos infelices?\ldots{} Pero, en fin, no nos entretengamos, Pepe, que tú
tienes prisa, y nosotros aguardamos la pitada que nos mande levar
anclas. Toma las diez y siete cartas que en estos días escribí a mi
ingrata: se las das todas para que se entretenga leyéndolas. En la
última le digo que en cuanto lleguemos a Cádiz, me quedaré franco de
servicio, y me vendré al Perú por Panamá, y veré a mi adorada, si es que
vive\ldots{} y a Dios le digo que si no me arregla el venir acá, y el
encontrarla buena y sana, y el hacer mis paces con ella y con su
familia, me volveré ateo\ldots{} Ateo seré, como hay Dios; te lo
juro\ldots{} Con que ya sabes: en ti confío; guarda las cartas\ldots{}
De lo que averigües me escribirás a Filipinas, donde haremos
escala\ldots{} Y si recibiera carta de ella, me volvería loco, y se me
quitaría el ateísmo\ldots{} Adiós, hijo: a ti me encomiendo. Que te vaya
bien. Ya suena el pito de Sacristá\ldots{} A levar se ha dicho\ldots{}
Adiós, adiós.»

Prometió Mendaro cumplir con toda solicitud el encargo de su amigo, y
resistiéndose a tomar el dinero que este le ofrecía, se
abrazaron\ldots{} «¡Adiós, América!» dijo el uno. Y el otro: «¡Adiós,
España!\ldots» Media hora después, la \emph{Numancia}, andando a
máquina, doblaba majestuosa la punta de San Lorenzo, y al entrar en el
ancho mar tendía las alas de su velamen, abandonándose en brazos del
viento suave y amoroso. Toda la Escuadra navegó en conserva el día 10
con rumbo SO., y a la puesta del sol se separaron las dos divisiones. La
despedida, con los silbatos de vapor y el sube y baja de banderas, fue
patética, y dejó tristísima impresión en todas las almas. Pusieron las
proas al Sur los que iban por el Cabo de Hornos, y la \emph{Numancia},
\emph{Berenguela} y \emph{Vencedora}, con el \emph{Marqués de la
Victoria} y los mercantones \emph{Uncle Sam} y la fragata Mataura,
enmendaron su rumbo, poniéndolo al Oeste con cuarto al Sur.

El descanso de los tripulantes en aquella expedición era tedioso y
lúgubre. Enfermos de excitación anímica y de rudos trabajos, ingresaban
en vida de hospital, donde el malestar o las lesiones que cada uno
llevaba salían a la superficie estimuladas por el reposo. Sobre todos
los males imperaba el mal comer, contra el cual no había remedio
mientras no llegasen a tierra de abundancia. Carne salada, tocino en mal
estado y galleta mohosa, eran el alimento corriente para todos, altos y
bajos. El hambre se juntaba con la inapetencia, y la repugnancia cortaba
el paso al apetito. Y para colmo de desventuras, la carencia de tabaco
llegó a ser absoluta. Hombres había que se dolían más del no fumar que
del no comer. Llegó un día en que el mismo Binondo, almacenista en
pequeña escala de hoja \emph{virginia}, no suministraba ni una hebra.
Hombres industriosos hubo, tan ávidos del vicio, que discurrieron fingir
el tabaco con raspaduras de maderas dadas de sebo rancio. Las virutillas
que así sacaban eran liadas en papel, como picadura, y venga chupar y
escupir, engañando el gusto y rodeándose de humareda pestífera.

La tristeza era general: nadie cantaba ni reía. El aplanamiento físico y
moral sobrevino con verdadera difusión epidémica. La pereza embotaba la
voluntad: nadie trabajaba; fatigábanse algunos del menor esfuerzo, y
todos caían en tétricas modorras. Para sacudir los cuerpos enmohecidos,
se discurrió darles gazpacho dos veces al día, pues no faltaba vinagre a
bordo; y para mover las almas, se ordenó que se pusieran en práctica
todos los medios de regocijo. El que supiera cantar, que cantase, y
lucieran sus habilidades los tañedores de guitarra, bandurria, flauta, o
siquiera del güiro. Diose permiso para bailar y recitar romances y
jácaras. Mientras los marineros organizaban un festival de zapateado, o
de las danzas peruanas la \emph{Zamacueca} y la \emph{Zanguaraña}, que
algunos sabían, los Guardias marinas repartían y ensayaban el socorrido
\emph{Puñal del godo}, para dar una representación solemne y pública en
el Alcázar. Hasta se quiso incluir en el programa un número de
prestidigitación y otro de volatines, que había en la Maestranza dos
muchachos muy fuertes en estas divertidas profesiones.

De nada valían tales artificios para atraer la alegría cuando esta no se
dejaba coger. Si por momentos resplandecía sobre algunas extravagancias,
pronto se iba, difundiéndose en el aire calmoso. Lo que al barco llegaba
y en él ponía su alojamiento era el escorbuto, el mal marinero que
destruye las tripulaciones cansadas, mal comidas y agobiadas de tristeza
en las grandes soledades oceánicas. En la \emph{Berenguela} y
\emph{Vencedora} menudeaban los casos; en la \emph{Numancia} empezaron
las manifestaciones de mal a los tres días de salir de Callao. Los
médicos vieron venir la terrible infección, y sin poder aplicar más que
paliativos, suspiraban por llegar a cualquier isla donde hubiera
limones. El primer atacado fue Desiderio García, que además tenía una
herida de casco de metralla en el muslo, aún no cicatrizada; cayeron
después un marinero vizcaíno, llamado Ansótegui, y dos fogoneros
gaditanos. Empezaban con un recrudecimiento de la general tristeza, y
con extremada flojedad, abatimiento y fatiga; seguía la hinchazón de
encías, síntoma determinante del mal; luego la reapertura de las
heridas, el que las tuviera, las manchas equimóticas que degeneran en
úlceras, la emisión de sangre negruzca, la caída de los dientes, y, por
fin, el marasmo, la muerte\ldots{}

En el pobre Desiderio García, no ofrecieron gravedad los primeros
síntomas escorbúticos; pero el recrudecimiento de las heridas trajo
complicaciones alarmantes, y el enfermo se vio acometido por dos males
que encarnizadamente se lo disputaban. Al mismo tiempo que aparecieron
las \emph{petequias}, forma incipiente de la equimosis, y la hinchazón
de encías, se presentó una fiebre intensa, fatiga, dolores que indicaban
graves alteraciones viscerales. En dos días cayo el infeliz en
postración hondísima. Crueles hemorragias anunciaban su acabamiento; las
encías tumefactas no le cabían en la boca; su respiración no era más que
el ansia de respirar. Una tarde, entre dos síncopes, disfrutó de breve
descanso, y pudo emitir sonidos, palabras y aun conceptos. Llamó a sus
amigos, y una vez que los tuvo junto a su lecho, les cogió las manos, y
con pausado acento les dijo: «Ansúrez, Sacristá, Binondo, quiero que
sepáis que aquella sinfinidad y catálogo de millones de plata y oro que
os conté, y el escondimiento del tesoro en una cueva de Copacavana, son
mentiras y embaucaciones que no sé si saqué yo de mi cabeza, o me las
asopló un diablo que quería perderme. Si creísteis aquellas trolas,
descreedlas ahora, y decid que os engañé por estar yo engañado\ldots{}
Ya confesé al Capellán mi falsedad, y a vosotros ahora la
confieso\ldots{} Perdón les pido, y que recen por mi ánima.»

Alentáronle los amigos con frases cariñosas, y Binondo dijo que no
siendo esta vida más que una ensoñación, soñar con tesoros es un
barrunto y vislumbre de la gloria eterna. Media hora después,
reconciliado por el Capellán \emph{y con el práctico a bordo} para
emprender su viaje a la Eternidad, tuvo otro momento lúcido, en el cual
pidió el último favor a su amigo Ansúrez. «Me pondrás en los pies---le
dijo---dos balas del mayor calibre; en la cintura una parrilla, y en el
pescuezo\ldots{} aquí\ldots{} un par de lingotes, para que cuando me
arrojéis, pueda yo irme derechito al fondo. ¿Sabes por qué te digo esto?
Pues anda por aquí una tintorera que viene dando convoy a la fragata
desde que montamos la punta de San Lorenzo. Tú la has visto, la han
visto todos. Te aseguro que cuando yo la miraba desde la borda, la
condenada no me quitaba los ojos\ldots{} Con sus ojos me decía: `Te
como, te como'. Créelo: como hay Dios que nos viene siguiendo, porque
sabe que me arrojaréis\ldots{} Estos animales son muy listos, y todo lo
entienden. Pero si tú haces lo que te pido, ponerme mucho hierro, mucho
peso, yo me reiré de la tintorera, y a escape bajaré a lo profundo,
diciéndole. `Fastídiate, tintorera. No me comes, no me comes'»

Al poco rato expiró, y fue en busca de los tesoros eternos. Era un buen
hombre, de imaginación poemática\ldots{} Sus amigos le lloraron; y para
cumplir su última voluntad, Binondo cuidó de arrojarlo al agua con
oraciones y hierros de extraordinaria pesadumbre.

\hypertarget{xxviii}{%
\chapter{XXVIII}\label{xxviii}}

El cabo de cañón Ansótegui y los dos fogoneros se sostenían en los
medios de sufrimiento, con esperanza de mejorar en cuanto llegaran a un
país bien surtido de limones y naranjas. Era el viaje de una lentitud
desesperante, por lo apacible del viento y el poco tirar de la
corriente. La Numancia con todo su aparejo al aire no daba más de cuatro
o cinco millas por hora. Como arreciara el mal escorbútico en los otros
barcos, se les dio orden de abandonar la navegación en conserva,
adelantándose cada cual todo lo que pudiese. \emph{Berenguela} y
\emph{Vencedora} y los transportes se perdieron de vista; quedó sola la
blindada, arrastrándose como podía por las aguas quietas, con sus
tripulantes medio muertos de inanición y de quietismo tedioso. Lentos,
monorrítmicos, transcurrieron días de Mayo, días de Junio\ldots{} El
tiempo navegaba por las aguas dormidas de la laguna Estigia\ldots{} Y
los hombres, como atontadas moscas, caían del aburrimiento a la
enfermedad, unos con síntomas de escorbuto, otros de fiebre maligna, no
pocos atacados de mal desconocido, cuyo síntoma visible era la mortal
tristeza. En la enfermería no cabían ya tantos hombres. Era un dolor
verlos caer y humillarse a la pereza, y requerir el olvido de lo que
fueron.

El mismo Sacristá, fuerte como un roble, sucumbió a un acerbo quebranto
y dolor de sus cansados huesos; otros estaban como atacados de locura:
padecían el terror del escorbuto, y apretaban los dientes creyendo que
se les caían. Los fumadores sufrían el aplanamiento agudo de la
privación de tabaco\ldots{} Oficiales y Guardias marinas desaparecieron
del servicio y vivían confinados en sus camarotes, pidiendo limonadas
que no se les podían dar. Había pescadores maniáticos que se pasaban el
día y la noche en la borda, echando al mar aparejos que no enganchaban
bicho viviente. Maniáticos había de ver tierra, que en cada nube del
horizonte señalaban montañas, volcanes, a veces casas con blancas torres
y chapiteles que brillaban al sol.

A mitad de Junio no bajaba de ciento el número de hombres atacados de
diferentes dolencias. El único que se conservaba fuerte, activo y
hablador era Binondo: a todos quería consolar con ideas del galardón que
reserva Dios a los justos, y a los \emph{padecientes} y \emph{llorantes}
en esta cárcel de la vida terrenal. Aseguraba el malayo que él no
necesitaba comer para sostenerse, y que su gran piedad y la fortaleza de
su espíritu hacían las veces de alimento, dígase carne, pescado, y las
demás materias nutritivas de que se forma nuestra sangre.

El 16 de Junio, cuando el vigía de cofa señaló el monte de
\emph{Fatu-Hiva}, salieron todos a verlo, y aquel recreo de los ojos
difundió en las almas una ráfaga de alegría\ldots{} Aún distaban cuatro
o cinco días de la isla de \emph{Otaiti}\ldots{} La esperanza levantó
los corazones\ldots{} Por fin, el 22 al anochecer vieron las luces de la
ciudad de \emph{Papeeté}, capital de la ínsula; mas desconociendo el
puerto, siguieron por un ancho canal hasta la bahía de Toanoa, donde
echaron el ancla. Un día más, y se encontraron frente a \emph{Papeeté}
rodeados de una felicidad y abundancia superiores a cuanto habían soñado
los hambrientos, sedientos y maniáticos. ¿Era ilusión lo que veían? ¿Y
aquellos botes y cayucos que rodeaban a la fragata, cargados de pan, de
frutas, de tabaco, eran reales, o fantástica hechura de los cerebros
enfermos? La hermosura del cielo, la tibieza de ambiente, la juvenil
alegría que de todas partes emanaba, las voces de los indígenas
ofreciendo alimentos tan apetitosos, habían trastornado a los sanos, y a
los enfermos devolvían la razón, la confianza, el amor a la vida\ldots{}
Para mayor gozo, vieron fondeados, a pocas brazas de la ciudad, los
demás buques de la segunda división. Participaban todos del delicioso
descanso y festín riquísimo que Dios les enviaba en compensación de sus
horribles trabajos y miserias. «¡Hosanna, loor eterno al Omnipotente!»
clamaba el pío Binondo alzando al cielo las manos, cuando llegaron a
cubierta las primeras cestas de naranjas y limones, subidas por los
indígenas, que eran, dígase con histórica imparcialidad, los seres más
amables de la creación, los más ágiles y risueños\ldots{}

¡Oh incomparable país; oh civilización silvestre, rozagante y desnuda;
oh tierra de bendición y de libertad, coronada de flores y ceñida de
espumas! Tu suelo fecundo y tu temple benigno redimen a los hombres de
la dura ley del trabajo. Aquí la espléndida vegetación, sin las artes de
cultivo, ofrece al hombre cuanto necesita para su sustento; aquí la
dulzura del clima le exime de la complicada cargazón de ropa, no
imponiendo más que el preciso y elemental resguardo del pudor; aquí las
costumbres son proyección fiel de las benignidades de Naturaleza; no
existe ni el rigor de castas, ni el apartamiento receloso entre los
sexos; la ley es suave, el matrimonio facilísimo, la religión alegre, la
virtud generosa, la moral amable, la muerte un dulce tránsito\ldots{}
Tal pensaban y sentían los españoles ante la hermosura de
\emph{Papeeté}, capital de \emph{Otaiti}.

Las primeras cargas de víveres fueron materialmente devoradas por la
tripulación. Arrastrándose subieron algunos enfermos a cubierta;
arrebataban las naranjas y limones, y se los comían con cáscara. A
enfermos y sanos exhortaba Binondo a la moderación, y pegando bocados a
un tierno pan, les decía: «Poco a poco, hermanos y amigos; refrenad el
apetito de golosinas, que si dais demasiado al gusto, os quedará poco
para la salud. Guardad templanza y observad comedimiento, que las
hambres que habéis pasado no os dan licencia para entregaros a la gula,
feísimo pecado.» Estas y otras frases, aprendidas en el libro de
Sermones, iba soltando de grupo en grupo, sin perjuicio de tomar aquí y
allí todo lo que le daban, plátanos, limones, guayabos y otras
peregrinas frutas.

No escatimó el Comandante en aquel día y los siguientes las licencias
para bajar a tierra. Deseaba que su gente se esparciera y refocilara en
aquel edén, buscando su salud en la libertad, el movimiento y la
alegría. Su primer cuidado fue gestionar de las autoridades otaitana y
francesa la cesión de un edificio amplio y ventilado donde colocar a los
enfermos. Concedida para este fin una isla entera, se dispuso trasladar
a tierra a los infelices que penaban en los obscuros sollados. Todo era
bienandanzas en la venturosa isla que, rodeada de arrecifes de coral,
ciñe su contorno de un cinturón de blanca espuma. Por esto fue llamada
\emph{La Cuna de Venus}.

Fondeada la \emph{Numancia} muy cerca de tierra, en aguas quietas y
cristalinas, creíanse los españoles transportados milagrosamente de la
muerte a la vida, y del reino de las amarguras a la morada de todas las
delicias. Iban y venían los botes, surcando aquel mar de juguete suizo,
con agua, casitas, figurillas de movimiento y caja de música, y pisaron
tierra en diferentes grupos oficiales y guardias marinas, cabos de mar,
marineros, condestables, soldados\ldots{} Lanzáronse a recorrer la
ciudad y sus inmediaciones, apreciando cada cual según su criterio y
cultura las maravillas naturales que contemplaban. Tiraron unos desde
luego hacia el campo, atraídos por la opulencia de la vegetación, que a
mayor altura que las chozas y edificios mostraba sus verdes cúpulas y
cimeras ondeantes. Fueron a parar a un espeso bosque de naranjos y
limoneros, silvestre, libre; se admiraron de pisar alfombra de azahares
caídos, y de coger cuanto fruto quisieran con sólo alargar la mano. No
vieron señal ninguna de propiedad personal. Todo era de todos, del
pueblo, que en la enramada frondosa tenía sus bien provistas
despensas\ldots{} El propio comunismo vieron y comprobaron en los
espesos matorrales de guayabas, en las plataneras de luengas
hojas\ldots{} No había cercas, no daban el quién vive guardas adustos ni
perros mordedores. Mujeres y chicos, vestidos de amplias y flotantes
túnicas, andaban por aquellos vergeles cogiendo cuanto anhelaban, y
ofreciéndolo a los extranjeros con risueña cortesía, para que ni la
molestia tuvieran de cosechar lo que les pedía su necesidad y su gusto.

Adelante siguieron por alegres campos: vieron aldeas escondidas entre
palmas de coco y otras especies vegetales rarísimas\ldots{} Las casas de
cañas con singular arte tejidas parecían jaulas o cestas. ¡Qué bien se
viviría en aquellos aposentos cuyos frágiles muros tamizaban el aire, la
luz y las miradas humanas! ¡Feliz \emph{Otaiti}, que no conociendo la
gazmoñería, también desconocía la indiscreción!

Andando incansables entre tantos motivos de regocijo y asombro, dieron
vista a un río que por aquí saltaba gozoso entre peñas con sonoras risas
y espumas, y por allá se remansaba en curvas perezosas hasta llegar a un
punto en que parecía dormirse a la sombra de árboles corpulentos que
sobre él tejían bóveda de ramaje. En aquel remanso vieron los españoles
turba de mujeres que gozosas y picoteras se bañaban. Las que en la
orilla se disponían al baño y natación no se vestían de verde lampazo,
sino que habían soltado la vestidura, quedándose como vinieron al mundo.
Escondidos miraron los curiosos este lindo espectáculo, y oyeron la
algazara que unas con otras hacían. Las que salían de agua empleaban
para secarse el procedimiento más primitivo, que era revolcarse en el
verde césped, y dar al aire sus extremidades con vigorosas zapatetas y
cabriolas. Llegó un momento en que las alegres mozas se percataron de
que eran miradas por los extranjeros, y no hicieron aspavientos de susto
ni chillaron con remilgado pudor. Cambió de tono su griterío y algazara,
y abandonando las aguas transparentes, se vistieron con prisa; operación
fácil y que sólo consistía en encapillarse un ropón largo y holgón,
única vestimenta de su constante uso, prenda única de su elegancia y
adorno mujeril.

Sin secarse ni aliñar las sueltas cabelleras mojadas, corrieron en
alegre bandada las morenitas nereidas, y tras ellas iban, con paso y
ojeo de cazadores, los europeos. Las alcanzaron en un prado verde
rodeado de arbustos, y allí, sin entender ni jota de la lengua que
hablaban las ninfas, se metieron en franca conversación con ellas. Lo
que no expresaban los idiomas desconocidos, decíanlo las risas, los
gestos amables, las miradas alegres, y el tono general harto elocuente,
mas no exento de cortesía. Algunas muchachas corrían con graciosa
ligereza de piernas, y parándose de improviso, disparaban contra los
españoles guayabos y naranjas, o los apedreaban con una frutilla menuda
parecida a nuestras almendras; otras, admitiendo palique a media
comprensión de vocablos, se dejaban abrazar. El idioma primitivo
recobraba sus fueros. Luego que eran abrazadas, se escabullían brincando
como gacelas, y a perderse iban en las enramadas circundantes de las
casas de caña\ldots{} Desde el interior de aquellas jaulas continuaban
disparando contra sus perseguidores risotadas y voces incomprensibles,
que ellos no sabían si eran burlas o amistoso reclamo\ldots{} ¿Estaban
en \emph{Otaiti} o en el Paraíso terrenal?

Los grupos de españoles, que, en vez de tirar hacia el campo y el monte,
tiraron hacia las calles de \emph{Papeeté}, eran la gente ilustrada que
iba en busca de las señales de civilización. No es menester decirlo: se
divirtieron menos que los incultos y casi analfabetos que lanzándose
tras de la Naturaleza y en seguimiento de la raza indígena,
sorprendieron a esta en su prístina sencillez y alegría de costumbres.
Los ilustrados reconocían y admiraban las casas construidas cerca de
muelle por los comerciantes europeos, el palacio de la Reina, y otros
edificios de carácter administrativo y judicial. ¡Qué hermosura! ¡En
\emph{Otaiti} había Administración, había Justicia! Vieron también con
admiración, en las calles, señoras y caballeros indígenas ataviados a la
europea\ldots{} Gracias al protectorado de Francia, que se había metido
en aquel edén para echarlo a perder y privarlo de sus seculares
encantos, en \emph{Papeeté} había zapateros, sastres y hasta
sombrereros, bárbaros correctores de la estirpe humana, que han hecho
una industria de la fealdad, y de la embarazosa sujeción del andar y los
ademanes.

A consecuencia de no sabemos qué rebeldías y trapisondas, cayó la feliz
\emph{Otaiti} en el protectorado francés. Un funcionario del Imperio
ejercía la autoridad con el nombre de \emph{Comisario Gobernador}.
Conservaba la soberanía de figurón una señora Reina, llamada
\emph{Pomaré IV}, morenita y bella, del mejor tipo de la raza. En la
época del arribo de la \emph{Numancia}, ya no era joven Su Majestad
\emph{canaca}; pero conservaba su aire gracioso y cierta distinción
adquirida en el viaje que hizo a París. Fundaba su orgullo en vestir a
la francesa, cuidando de acarrear trajes de última moda, o de imitarlos
con auxilio de figurines. Dígase con todo el respeto que merecía la
bondadosa Pomaré, que enjaezada a la europea estaba para pegarle un
tiro. ¡Cuánto más bonita y seductora sería su facha conservando como
única vestimenta el ropón o camisolín amplio y suelto con que se
ataviaban y cubrían las mujeres del pueblo! El Rey consorte, llamado
\emph{Arii Faité} era un bigardo glotón y borrachín, que no se dejaba
ver más que en comilonas y francachelas. Vestía ridículamente casacón
bordado, y las plumas que debía llevar en su cabeza, según el uso
salvaje, llevábalas en un sombrerote tricornio, como los que usan los
suizos de las iglesias parisienses. Era, sin duda, el hombre más bárbaro
de \emph{Otaiti} y el más feliz de los \emph{canacas}, que este nombre
se daba a los indígenas del Archipiélago de coral.

\hypertarget{xxix}{%
\chapter{XXIX}\label{xxix}}

Los felices españoles de clase humilde que visitaban la isla un día y
otro, contaban a Binondo las maravillas que habían visto, la frondosidad
silvestre de los naranjales y cocoteros, la sencillez y gracia de las
mujeres vestidas de un simple camisón, y tan amablemente abiertas de
voluntad a los obsequios del hombre; y al oír una y otra vez estas
extraordinarias cosas, el malayo se encerraba en grave silencio, que era
sin duda la cavidad mental en que guardaba sus profundísimas
abstracciones. De aquellas honduras no sacaba su pensamiento más que
para mostrarlo al Capellán don José Moirón. Una tarde, cogiéndole solo,
le dijo: «Por lo que cuentan estos perdidos, señor don José, los
habitantes de \emph{Otaiti} no conocen la vergüenza ni ninguna ley
divina ni humana. El nombre de \emph{canacas} me dice que estos
naturales son los \emph{cananeos} de que nos habla Nuestro Señor
Jesucristo en su Biblia, o dígase Moisés, que es lo mismo. Por donde
saco que esta isla es aquella tierra de \emph{Canaam} de que habla no sé
si el Evangelio o la Epístola.»

Contestole el Capellán tapándole la boca, para que no salieran de ella
más desatinos; pero el malayo prosiguió imperturbable: «Desde que
llegamos aquí, me paso las horas pensando qué religión profesarán estos
bárbaros, cómo serán sus templos y qué vitola tendrán sus sacerdotes.
Nada han dicho los muchachos de la religión \emph{canaca} o
\emph{cananea}, por lo que pienso será una indecente idolatría, como el
adorar a la serpiente con pechos de mujer, o a un hombre desnudo con
cabeza de cocodrilo. Por todo lo cual, señor don José, usted y yo no
haríamos nada de más yéndonos a tierra para ver qué casta de religión
profesan estos salvajes\ldots{} y si resulta que es alguna secta
idólatra y gentílica, de esas en que se adora la materia y el vicio,
bien podríamos hacer algo por las almas de estos infelices,
instruyéndolos y catequizándolos para sacarlos de sus errores lascivos y
pestilentes, y traerlos a la verdad de nuestra fe cristiana y
sacratísima. Habrá usted oído que andan las mujeres por esos campos
pisando azahares, sin más vestido que un ropón para cubrir la desnudez
de pechos y caderas. Tales costumbres disolutas y desvergonzadas
significan que aquí no se mira más que al deleite, en el comer, en el
emborracharse y en el danzar deshonesto\ldots{} Bienaventurado sería
usted si consiguiera iluminar con su predicación a esas almas
descarriadas. Yo iría con usted de misionero coadjutor o suplente, y no
haríamos pocos méritos para nuestra salvación particular.»

Tímido y desconcertado, contestó el Capellán que él no tenía otra misión
que la cura de almas de los tripulantes de la fragata, y que no quería
meterse a convertir salvajes más o menos desnudos. Además, la Francia,
protectora de \emph{Otaiti}, cuidaría de cristianizar a los
\emph{canacas}, que para ello tenía personal nutrido de frailes y curas.
Hecha esta declaración aconsejó a Binondo que pues sentía en sí fervor
de catequista, fuese él solo a enseñar el Evangelio a los otaitanos. No
desoyó el malayo este sabio consejo; aquella misma tarde se acicaló y
compuso de rostro y vestido, y agarrando un grueso bastón en figura de
báculo, se fue a tierra y se internó en la campiña de \emph{Papeeté}.
Divagando de un lado para otro, fue a parar al remanso del río en que se
bañaban las \emph{canacas} (de que tenía noticia por relación de sus
amigos) y vio venir a las ninfas con sus holgadas túnicas, sueltas las
cabelleras mojadas. Llegose a ellas risueño y melifluo, echándoles
almibarados requiebros. Debieron las mozas de tomarlo por un mico
vestido de marino español y con risotadas lo cogieron, lo zarandearon y
se lo llevaron a una de las aldeas próximas\ldots{} Se perdió de vista
el pío Binondo\ldots{} desapareció sin duda en el interior de una de
aquellas frágiles casas de caña que parecían cestas.

Al anochecer, volvió el malayo a bordo hecho una lástima; su chaquetón
de cabo de mar había perdido los dorados botones, y mayores averías que
en la ropa tenía en su rostro plano, lleno de horribles arañazos y
chichones\ldots{} Entró en cubierta procurando ocultar con una mano su
desventura; pero no le valió el tapujo. Sus amigos hicieron gran befa y
chacota. La explicación que dio fue que, habiendo entrado en una casa de
infieles \emph{canacas} con idea de predicarles el Evangelio, al
principio fue oído con atención y recogimiento. Mas de pronto
aparecieron unos diablos negros y deformes que le clavaron sus garras en
semejante parte (el rostro), y le estrujaron y le hicieron mil
estropicios hasta dejarle en aquel estado lastimoso\ldots{} Buscó el
santo varón su bálsamo y consuelo en la piadosa lectura, principalmente
en el \emph{Sermonario}, cantera riquísima de donde extraía todas sus
ideas y sus persuasivas formas de lenguaje.

Desde el feliz arribo a \emph{Otaiti} túvose Fenelón por el hombre más
dichoso del mundo. Su nacionalidad francesa le dio vara alta en aquel
país sometido al protectorado imperial. A tierra bajaba diariamente
vestido con rebuscada elegancia, luciendo llamativos chalecos y
corbatas. No tardó en cautivar al Gobernador Comisario, dándose a
conocer con el título y modales de calavera de buena familia, sometido a
expiación por desvaríos amorosos, y a esto debió mayor prestigio y
metimiento en la buena sociedad \emph{papeetana}, compuesta del
Comisario francés Conde de Roncière, del Ordenador de la Marina, del
Cónsul inglés, y de media docena de comerciantes ingleses y americanos.
De esta sociedad le fue muy fácil subir el único escalón que le faltaba
para llegar al Real Palacio. La aspiración del francés se vio pronto
satisfecha, y tuvo el honor de ser recibido y obsequiado por Su Majestad
\emph{canaca}, de quien mereció tan exquisitos agasajos, que sólo podía
referirlos bajo palabra de secreto a los amigos de mayor confianza.

Solía el buen Ansúrez acompañarle a tierra; pero en las primeras calles
de \emph{Papeeté} se separaban, pues era el celtíbero más gustoso del
libre campo que de la ciudad. En los espectáculos de la silvestre
Naturaleza espaciaba sus melancolías, y el trato del pueblo sencillo y
afable le resarcía de la desolación de su árida existencia sin afectos.
Por las noches, de regreso a bordo, contábale Fenelón sus particulares
sucesos del día, y el inocente Ansúrez se lo tragaba todo con crédula
voracidad. «Hoy---decía el francés,---me ha dado Pomaré un rato
malísimo\ldots{} Es en extremo celosa\ldots{} Figúrate que paseando
solos, vimos pasar una \emph{canaca} lindísima: yo la miré\ldots{} no
hice más que mirarla\ldots{} Pomaré furibunda\ldots{} creí que me
arañaba\ldots{} Hermosa y terrible es la mujer apasionada; yo adoro la
pasión; pero la pasión salvaje puede ponerte, \emph{por ejemplo}, entre
las garras de una leona, y esto descompone un poco las más bellas
aventuras.» Otro día contaba incidentes más gratos: «Hoy me ha dicho
Pomaré que no se separará de mí. Pretende que me quede en \emph{Otaiti}
de director de las Reales Máquinas\ldots{} que son una lanchita de
vapor, varios relojes y cajas de música, y un aparato por el estilo de
lo que llamáis \emph{Tío Vivo}, para solazarse en el jardín\ldots» Y
alguna vez no faltaban regias gacetillas: «Hoy se ha puesto tan pesado
ese gandul de \emph{Arii Faité}, que he tenido que darle veinte francos
para que fuese a emborracharse, mi palabra\ldots{} Con unos gritos de la
Reina y un empujón mío le echamos a la calle\ldots{} Yo leo el
pensamiento de Pomaré\ldots{} Si \emph{Arii Faité} reventara de
\emph{delirium tremens}, ya sé yo quién ocuparía su lugar en el trono.»

La oficialidad apenas tenía tiempo para acudir a tantas invitaciones y
festejos. En la casa del Comisario, Conde de Roncière, y en las del
Cónsul inglés y de los opulentos ingleses Brander y Hort, menudeaban los
banquetes, las \emph{soirées}, \emph{asaltos}, meriendas y conciertos.
Para corresponder a tan amables agasajos, determinó el Comandante de la
División dar un baile a bordo de la \emph{Numancia}, y al punto se puso
mano en los preparativos de la fiesta. Destinado el Alcázar a salón de
baile, se le adornó con vaporosas gasas, percalinas vistosas y
terciopelos ricos, añadiendo a los trapos las galas de la Naturaleza que
mayormente habían de contribuir al bello conjunto, el ramaje verde, las
palmas y palmitos, y profusión de flores de tropical fragancia y
hermosura. Completaron el ornamento los pabellones y trofeos de guerra y
mar, las banderas de \emph{Otaiti}, Francia y España en fraternal enlace
y combinación. La batería fue convertida en comedor para la espléndida
cena, la toldilla de popa en salón de juego y descanso, y las cámaras de
los Jefes en tocador para las señoras. La última mano de esta obra
suntuaria fue un soberbio plan de iluminación interna y externa del
barco. ¿Qué faltaba? Orquesta o banda militar. Como nada de esto tenía
la fragata, se acudió al remedio de un piano traído de \emph{Papeeté}.

Con tantas previsiones y el esmero en cuidar del conjunto y perfiles,
resultó el baile tan original como fastuoso. En la fantástica nave,
Marte y Neptuno se dieron cita con Venus, que llevaba de la mano a
Terpsícore, tras de la cual entró también Baco, representado en la crasa
persona augusta del Rey o Príncipe (que de ambos modos se le llamaba)
\emph{Arii Faité}. Concurrió toda la aristocracia europea y
\emph{canaca}, las hermosas señoras y señoritas de las familias
francesas y británicas, las princesas reales \emph{Aimatá} y
\emph{Borabora}, y por último, Su Majestad \emph{Pomaré IV}, para la
cual se arregló una espléndida falúa. Está de más decir que la Reina de
\emph{Otaiti} y sus damas, vestidas a la europea con huecos miriñaques,
ostentando además cuantos faralaes y ringorrangos imponía la moda,
dieron a la fiesta su mayor grandeza y hermosura. Amabilísima estuvo Su
Majestad con todos, mostrando en su exquisito trato la dignidad afable
de los soberanos europeos. Era una excelente Reina, un poco fondona ya,
en el ocaso de su belleza morenita. Hablaba un francés aplatanado y
ceceoso que hacía mucha gracia\ldots{} Honró \emph{Arii Faité} la cena,
repitiendo cuatro veces de todos los manjares suculentos, y tanto él
como el anciano Príncipe \emph{Paraitá}, que había sido Regente en la
menor edad de \emph{Pomaré IV}, no se contuvieron en las libaciones
alegres y copiosas. Al Rey consorte le retiró Fenelón oportunamente,
llevándole a la falúa poco menos que a rastras. No se pudo hacer lo
mismo con el respetable \emph{Paraitá}, que desplegó hasta el amanecer
su elocuencia en diferentes tonos, desde el sentimental al heroico.
Discursos y brindis sin fin pronunció, primero en pie sobre las mesas,
al fin debajo de ellas. El baile terminó con la noche. A la luz del alba
se retiraron los invitados, tras de la Reina vagorosa, indo-europea y
fantástica. Aquella fiesta entre civilizada y salvaje fue el último
ensueño de los españoles en el Paraíso de \emph{Otaiti}.

\hypertarget{xxx}{%
\chapter{XXX}\label{xxx}}

De las delicias de la isla, llamada con razón \emph{Cuna de Venus}, se
ausentaron los españoles con vivo desconsuelo. ¿Cuándo y dónde
encontrarían un oasis, un paraíso semejante? El día de la salida, dijo
Fenelón a su amigo Ansúrez: «No subo a cubierta; no quiero que me vean
los espías de \emph{Pomaré}. Me voy a escondidas\ldots{} Prometí
quedarme de director de las Reales Máquinas\ldots{} Los ruegos, el
llanto de \emph{Pomaré}, me arrancaron una promesa que no puedo cumplir,
mi palabra de honor\ldots» De las inauditas hazañas amorosas que contó a
su amigo, dedujo este que habían sucumbido a los encantos del francés la
Reina y todas sus damas, no pocas señoritas de las colonias inglesa y
francesa, y dos tercios o poco menos del sexo femenino de clase
popular\ldots{} Todo se lo creía el buen Ansúrez, que se hallaba en un
estado psicológico propicio a la ingestión de mentiras. Sus facultades
pendían de la esperanza de encontrar en Filipinas cartas de Mendaro y de
Mara\ldots{} Pero Dios había dejado de su mano al pobre celtíbero,
porque la \emph{Numancia} llegó a Manila después de un viaje de mil
leguas, y en todo el mes que allí permaneció, no parecieron cartas, ni
de ninguna parte llegaron noticias. Grande es el mundo, y en recorrerlo
y darle la vuelta agota el hombre toda su paciencia; mas la de Ansúrez
era un filón sin término, yacente en un profundo pozo. Cuando a sacar
paciencia se ponía, sacaba esperanza. Si en Filipinas no habían parecido
las cartas, en Java parecerían\ldots{}

Pues llegaron a Batavia, capital de la bien regida colonia holandesa, y
nada dijo el correo, por más que Ansúrez con maniática pesadez
diariamente le interrogaba\ldots{} ¡A la mar otra vez! Y la paciencia y
la esperanza unidas se tragaron mil ochocientas leguas mal contadas
entre Java y El Cabo, sin que tampoco en aquella extremidad procelosa
del continente africano se encontrase ningún papel venido del Perú. Lo
extraño era que Ansúrez alimentaba sin ningún fundamento la ilusión
postal, pues no había dicho a Mendaro que escribiese a las más
excéntricas regiones del globo.

¡Ánimo, y venga del fondo del pozo más paciencia, venga más esperanza!
Ya estaban, como si dijéramos, a la puerta de casa, pues ¿qué suponían
diez mil leguas después de lo que habían andado desde que salieron de
Cádiz el 4 de Febrero de 1865? Al mar otra vez, \emph{Numancia}, y no te
arredres. Si cartas no hubo en Manila, ni en Batavia, ni en El Cabo, las
habría en Río Janeiro\ldots{} La distancia no era gran cosa: un
agradable paseo de mil doscientas leguas mal contadas\ldots{} Sucedió
que al término de esta luenga travesía quedaron igualmente fallidas las
esperanzas, aunque no agotada la paciencia que del hondísimo pozo sacaba
el hombre desconsolado. ¿Pero en qué estaba Dios pensando? «Como
lleguemos a Cádiz---se decía Ansúrez,---y no encuentre allí la escritura
de mi hija, juro a Dios que no habrá quien me saque del ateísmo\ldots»
Lo que en Río hallaron fue el Cólera, amén de otras calamidades, entre
ellas el peligro en que estuvo la \emph{Numancia} de volver a
Montevideo. Pero todo se arregló, y al fin la blindada salió para Cádiz
con lento andar y resuello fatigoso, como caballero que a su castillo
vuelve rendido del peso de sus armas. Del mismo modo Ansúrez se
quebrantó de la fortaleza espiritual que le había sostenido en el viaje
de regreso, y si no se le agotó el pozo de la paciencia, ya sacaba de él
tan sólo heces turbias y corrompidas. A ratos no más le asistía la
esperanza, y paralelamente a este descenso moral, se iba marcando en su
constitución hercúlea la dolorosa ruina.

Al pasar la línea ecuatorial, sintió como un terror que a su nostalgia
se unía, haciéndola más negra y pavorosa\ldots{} Navegando hacia San
Vicente, todos los afectos secundarios que endulzaban su existencia se
debilitaban gradualmente, hasta llegar a extinguirse. A unos amigos
apartó de su corazón con indiferencia, a otros con
aborrecimiento\ldots{} Y más allá de Puerto Grande, la ruina física y
moral del buen celtíbero se cristalizó en un estado neurótico agudo, con
depresión considerable de fuerzas que le obligó a encerrarse en la
enfermería. A duras penas podía pasar algún alimento; repugnaba la
compañía de los que fueron sus amigos\ldots{} A la altura de las Islas
Canarias, su pensamiento se descomponía en imágenes y ensueños, que se
manifestaban sobre un fondo de blancura opalina. Soñó que, arrebatado de
este mundo por la muerte, tomaba la vía del Cielo, donde creía se le
deparaba su perdurable residencia. Pero en el Cielo no quisieron
admitirle\ldots{} Íbase luego caminito del Infierno, donde sin ninguna
explicación le dieron con la puerta en los hocicos. «Pues no estoy poco
tonto---decía;---a donde tengo que ir es al Purgatorio.» Hacia allá
tiraba, y le acontecía lo propio que en el Cielo y el Infierno: que ni
por un Dios querían admitirle. Bien claro estaba que en el Limbo le
tenían preparado su descanso. Pues, señor, en aquel lugar bobo
encontraba la misma repulsa. «¡Ajo!---clamaba el hombre con
desesperación en medio del espacio.---¿Dónde meto yo mi pobre alma?»

Soñó esto muchas veces, en igual forma que aquí se cuenta. Añadíase
luego al sueño descrito este otro no menos extravagante: Hallándose el
alma de Ansúrez en medio del espacio sin saber dónde meterse, se le
presentaba un fantasma de rostro macilento y plano, muy parecido al de
Binondo, y le decía: «¿No me conoces? Soy el Ateísmo. Dame la mano; ven
conmigo, y yo te llevaré a mi asilo de eterno descanso.» No se
determinaba Diego a seguir al fantasma. Solo en medio del vago espacio,
sentía inmenso frío\ldots{} creía ver a un ángel que a soplos iba
apagando todas las estrellas.

\hypertarget{xxxi}{%
\chapter{XXXI}\label{xxxi}}

Un día antes de llegar a Cádiz, dio Binondo al Oficial de mar esta
enfadosa tabarra: «Sabrás, Diego querido, que en cuanto yo ponga el pie
en tierra, me voy derecho a la casa de los santísimos Padres
Franciscanos de las Misiones de África. Llegar y pedir al reverendo
Prior que me admita de lego, será todo uno. Recibiré la santa
instrucción frailesca, y acabaré mis días en la paz y santidad de la
Orden seráfica, que me abrirá de par en par las puertas de la
Gloria\ldots{} Imítame, Diego; tómame por modelo, ya que no tienes
familia ni nadie que mire por ti; decídete, y serás conmigo en el
Paraíso.» Nada le contestó Ansúrez: las ideas se le dispersaban, y las
palabras no afluían a su boca.

Un día más. Ya estaban a la vista de Cádiz, cuando Fenelón fue a
buscarle a la enfermería, y casi a viva fuerza le subió a cubierta para
que participara del general regocijo, y viese el espectáculo
sorprendente de la ciudad que sobre las aguas aparecía como ringlera de
diamantes montados en plata. A medida que avanzaba la embarcación, los
diamantes eran casas y torres, aquellas con cristales, estas con cimera
de azulejos, en cuyas superficies jugueteaban los rayos del sol\ldots{}
¡Cádiz! Para gran parte de los tripulantes de la \emph{Numancia} era el
hogar, el nido donde piaban la pájara y los polluelos\ldots{} La emoción
a todos embargaba, demudando el color de sus rostros y cortándoles el
aliento\ldots{} Pasadas las \emph{Puercas}, se mandó empavesar\ldots{}
Los barcos fondeados en la bahía echaron al viento todas sus banderas.
Acudieron multitud de lanchas y botes. La \emph{Numancia} acortó el paso
como el festejado viajero que, recibido por entusiasta gentío, tiene que
apretar infinidad de manos y contestar a innúmeras salutaciones. Del mar
circundante subía un clamor estruendoso de vítores; de la borda del
barco descendía lluvia de voces alegres y de alaridos roncos. Empezó al
instante, en forma de tiroteo nutrido entre la fragata y las
embarcaciones menores, el reconocimiento y saludo de parientes. Sonaban
en el aire como graneado fuego los nombres de padre, hijo,
hermano\ldots{} En medio de esta algazara, subió la Sanidad a bordo. ¡Oh
rigor de una ley inhumana! Como la fragata venía de Río Janeiro, no hubo
más remedio que imponerle cuarentena. La multitud de dentro y fuera del
barco chisporroteó como las ascuas de un brasero cuando se vacía sobre
ellas un jarro de agua.

En esto, Sacristá se acercó al buen Ansúrez que en la borda estaba
mirando a los botes, sin ver nada en ellos, y echándole un brazo por
encima del hombro, vertió en su oído este chorro de fuego: «Diego, ahí
la tienes\ldots{} ¿ves aquel bote que ahora se acerca por la popa de la
falúa de Sanidad?\ldots{} En él viene tu hija Mara: fíjate,
majadero\ldots{} Ahora está el bote abarloado con la lancha de
Pepe\ldots{} ¡Eh, dejad paso a ese bote!\ldots{} Si no lo ves, es que te
has quedado ciego.»

Ciego estaba el hombre; pero no de ceguera propiamente dicha, sino de
emoción, de algo más que emoción, de una turbulentísima sacudida y
revuelo de su alma que quería salírsele por los ojos. El bote avanzó con
dificultad por entre la escuadrilla de embarcaciones. En él venía, en
pie, una mujer arrogantísima que en su mano agitaba un pañuelo\ldots{}
Tan pronto hacía señas con el blanco lienzo, tan pronto se lo llevaba a
los ojos\ldots{} «Es Mara---dijo Ansúrez con una voz tan baja que sólo
pudo escucharla el cuello de su camisa.---Ella es; pero no verdadera,
sino fi\ldots{} sino figurada, como fan\ldots{} como fantasma\ldots»
«Mara---gritó Sacristá,---aquí tienes a tu papaíto asustado de verte.
Está bueno, aunque no lo parezca. Padece mal de tu ausencia\ldots{}
Acércate más; que te vea bien.» Mara tenía un nudo en la garganta, y de
sus labios no quería salir ninguna voz. Por fin, Ansúrez la reconoció
por su hija corpórea y no fantástica. Pasaron segundos, y reconoció
también a Belisario, que se puso en pie para saludarle con esta sencilla
y familiar fórmula: «Diego, ¿qué tal? ¿Buen viaje?» El celtíbero recobró
su aliento, y en el primer suspiro que lanzó se escaparon de su cuerpo
todas las complejas enfermedades que traía. Estalló un vivo y cortado
diálogo.

«Yo bueno\ldots{} cansado no más de viaje tan largo. ¿Habéis venido por
Panamá?»

---Sí, padre\ldots{} Hace tres meses que estamos aquí esperándole a
usted.

---Yo esperaba encontrar cartas, no vuestras personas.

---Escribimos a usted diez cartas---dijo Belisario.

---Y las mandamos a puntos diferentes, padre: una a las islas
\emph{Marquesas}, otra a Manila.

---Otra fue mandada a Zanzíbar, otra a Santa Elena, y qué sé yo\ldots{}
Cartas fueron a medio mundo.

---¿Os ha visto Mendaro?

---Sí: por él supimos que volvía usted a España. Nosotros pensábamos
venir acá. Hemos anticipado el viaje.

---¿Y tu niño, Mara\ldots?

---Está bueno\ldots{} Verá usted qué gracioso\ldots{} Ya le quiere a
usted sin conocerle.

---¡Pues no le quiero yo poco!\ldots{} Mara, ¿vendréis a verme, desde un
bote, mientras dure la cuarentena?

Afirmó Belisario que irían a visitarle diariamente. La cuarentena no
sería larga, pues no tenían a bordo ningún caso de cólera\ldots{} Mara
se sentó. Sosegados los tres, hablaron largo rato de cosas pasadas y
presentes; y en el curso de la entrañable conversación, repitió el
celtíbero más de una vez este sagaz concepto: «Lo que yo he visto y
aprendido es que cuando a uno se le pierde el alma, tiene que dar la
vuelta al mundo para encontrarla.»

\flushright{Madrid, Enero-Febrero-Marzo de 1906.}

~

\bigskip
\bigskip
\begin{center}
\textsc{fin de la vuelta al mundo en la numancia}
\end{center}

\end{document}
