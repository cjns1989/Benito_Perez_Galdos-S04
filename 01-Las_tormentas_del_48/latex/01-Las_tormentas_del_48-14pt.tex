\PassOptionsToPackage{unicode=true}{hyperref} % options for packages loaded elsewhere
\PassOptionsToPackage{hyphens}{url}
%
\documentclass[oneside,14pt,spanish,]{extbook} % cjns1989 - 27112019 - added the oneside option: so that the text jumps left & right when reading on a tablet/ereader
\usepackage{lmodern}
\usepackage{amssymb,amsmath}
\usepackage{ifxetex,ifluatex}
\usepackage{fixltx2e} % provides \textsubscript
\ifnum 0\ifxetex 1\fi\ifluatex 1\fi=0 % if pdftex
  \usepackage[T1]{fontenc}
  \usepackage[utf8]{inputenc}
  \usepackage{textcomp} % provides euro and other symbols
\else % if luatex or xelatex
  \usepackage{unicode-math}
  \defaultfontfeatures{Ligatures=TeX,Scale=MatchLowercase}
%   \setmainfont[]{EBGaramond-Regular}
    \setmainfont[Numbers={OldStyle,Proportional}]{EBGaramond-Regular}      % cjns1989 - 20191129 - old style numbers 
\fi
% use upquote if available, for straight quotes in verbatim environments
\IfFileExists{upquote.sty}{\usepackage{upquote}}{}
% use microtype if available
\IfFileExists{microtype.sty}{%
\usepackage[]{microtype}
\UseMicrotypeSet[protrusion]{basicmath} % disable protrusion for tt fonts
}{}
\usepackage{hyperref}
\hypersetup{
            pdftitle={las tormentas del 48},
            pdfauthor={Benito Pérez Galdós},
            pdfborder={0 0 0},
            breaklinks=true}
\urlstyle{same}  % don't use monospace font for urls
\usepackage[papersize={4.80 in, 6.40  in},left=.5 in,right=.5 in]{geometry}
\setlength{\emergencystretch}{3em}  % prevent overfull lines
\providecommand{\tightlist}{%
  \setlength{\itemsep}{0pt}\setlength{\parskip}{0pt}}
\setcounter{secnumdepth}{0}

% set default figure placement to htbp
\makeatletter
\def\fps@figure{htbp}
\makeatother

\usepackage{ragged2e}
\usepackage{epigraph}
\renewcommand{\textflush}{flushepinormal}

\usepackage{indentfirst}

\usepackage{fancyhdr}
\pagestyle{fancy}
\fancyhf{}
\fancyhead[R]{\thepage}
\renewcommand{\headrulewidth}{0pt}
\usepackage{quoting}
\usepackage{ragged2e}

\newlength\mylen
\settowidth\mylen{...................}

\usepackage{stackengine}
\usepackage{graphicx}
\def\asterism{\par\vspace{1em}{\centering\scalebox{.9}{%
  \stackon[-0.6pt]{\bfseries*~*}{\bfseries*}}\par}\vspace{.8em}\par}

 \usepackage{titlesec}
 \titleformat{\chapter}[display]
  {\normalfont\bfseries\filcenter}{}{0pt}{\Large}
 \titleformat{\section}[display]
  {\normalfont\bfseries\filcenter}{}{0pt}{\Large}
 \titleformat{\subsection}[display]
  {\normalfont\bfseries\filcenter}{}{0pt}{\Large}

\setcounter{secnumdepth}{1}
\ifnum 0\ifxetex 1\fi\ifluatex 1\fi=0 % if pdftex
  \usepackage[shorthands=off,main=spanish]{babel}
\else
  % load polyglossia as late as possible as it *could* call bidi if RTL lang (e.g. Hebrew or Arabic)
%   \usepackage{polyglossia}
%   \setmainlanguage[]{spanish}
%   \usepackage[french]{babel} % cjns1989 - 1.43 version of polyglossia on this system does not allow disabling the autospacing feature
\fi

\title{{\textsc{las tormentas del}} 48}
\author{Benito Pérez Galdós}
\date{}

\begin{document}
\maketitle

\hypertarget{i}{%
\chapter{I}\label{i}}

Vive Dios, que no dejo pasar este día sin poner la primera piedra del
grande edificio de mis Memorias\ldots{} Españoles nacidos y por nacer:
sabed que de algún tiempo acá me acosa la idea de conservar empapelados,
con los fáciles ingredientes de tinta y pluma, los públicos
acaecimientos y los privados casos que me interesen, toda impresión de
lo que veo y oigo, y hasta las propias melancolías o las fugaces
dulzuras que en la soledad balancean mi alma; sabed asimismo que, a la
hora presente, idea tan saludable pasa del pensar al hacer. Antes que mi
voluntad desmaye, que harto sé cuán fácilmente baja de la clara firmeza
a la vaguedad perezosa, agarro el primer pedazo de papel que a mano
encuentro, tiro de pluma y escribo: «Hoy 13 de Octubre de 1847, tomo
tierra en esta playa de Vinaroz, orilla del Mediterráneo, después de una
angustiosa y larga travesía en la urca \emph{Pepeta}, ¡mala peste para
Neptuno y Eolo!, desde el puerto de Ostia, en los Estados del
Papa\ldots»

Y al son burlesco de los gavilanes que rasguean sobre el papel, me río
de mi pueril vanidad. ¿Vivirán estos apuntes más que la mano que los
escribe? Por sí o por no, y contando con que ha de saltar, andando los
tiempos, un erudito rebuscador o prendero de papeles inútiles que coja
estos míos, les sacuda el polvo, los lea y los aderece para servirlos en
el festín de la general lectura, he de poner cuidado en que no se me
escape cosa de interés, en alumbrarme y guiarme con la luz de la verdad,
y en dar amenidad gustosa y picante a lo que refiera; que sin un buen
condimento son estos manjares tan indigestos como desabridos.

¿Posteridad dijiste? No me vuelvo atrás; y para que la tal señora no se
consuma la figura investigando mi nombre, calidad, estado y demás
circunstancias, me apresuro a decirle que soy José García Fajardo, que
vengo de Italia, que ya iré contando cómo y por qué fui y a qué motivos
obedeció mi vuelta, muy desgraciada y lastimosa por cierto, pues llego
exánime, calado hasta los huesos, con menos ropa de la que embarqué
conmigo, y más desazones, calambres y mataduras. Peor suerte tuvo la
caja de libros que me acompañaba, pues por venir sobre cubierta se
divirtieron con ella las inquietas aguas, metiéndose a revolver y
esponjar lo que las mal unidas tablas contenían, y el estropicio fue tan
grande, que los filósofos, historiadores y poetas llegaron como si
hubieran venido a nado\ldots{} Pero, en fin, con vida estoy en este
posadón, que no es de los peores, y lo primero que hemos hecho mis
libros y yo es ponernos a secar\ldots{} ¡Oh rigor de los hados! Los
tomos de la \emph{Storia d'ogni Letteratura}, del abate Andrés, y el
\emph{Primato degli italiani}, de Gioberti, están caladitos hasta las
costuras del lomo: mejor han librado Gibbon, Ugo Fóscolo, Pellico,
Cesare Balbo y Cesare Cantú, con gran parte de sus hojas en remojo.
Helvecio se puede torcer, y Condillac se ha reblandecido\ldots{} De mí
puedo decir que me voy confortando con caldos sustanciosos y con unos
guisotes de pescado muy parecidos a la \emph{Zuppa alla marinara} que
sirven en los bodegones de la costa romana.

\emph{15 de Octubre}.---Advierto que la fisgona Posteridad, volviendo
hacia atrás la cabeza, me interroga con sus ojos penetrantes, y yo le
contesto: «Se me olvidó deciros, gran señora, que tres días antes de
abandonar el italiano suelo cumplí años veintidós; que mi rostro y
talle, según dicen, antes me restan que me suman edad, y que mis padres
me criaron con la risueña ilusión de ver en mí una gloria de la
Iglesia.» Cómo disloque por natural torcedura de mi espíritu la vocación
irreflexiva de mis primeros años, y cómo desengañé cruelmente a mis
buenos padres, no puedo referirlo mientras no me oree, me desentumezca y
me despabile.

\textbf{San Mateo}, \emph{19 de Octubre}. Ayer, no repuesto aún del
quebranto de huesos ni del romadizo que me dejó la mojadura, aproveché
la salida de un tartanero y acá me vine en busca de mejor vehículo que
me lleve a Teruel, desde donde fácilmente podré trasladarme a la
ilustrísima ciudad de Sigüenza. Allí \emph{rodó mi cuna}, si no de
marfil y oro, de honrados mimbres con mecedoras de castaño, y allí
reside desde los comienzos del siglo mi familia, cuyo fundamento y solar
figuran en los anales de la histórica villa de Atienza\ldots{} Adivino
la curiosidad de \emph{i posteri} por conocer los móviles que me sacaron
de mi casa dos años ha, llevándome casi niño a tierras distantes, y allá
van mis noticias. Sepan que, apenas entrado en la edad de los primeros
estudios, diome el Cielo luces tan tempranas, que mi precocidad fue
confusión de los maestros antes que orgullo y esperanza de mi familia,
pues declarándome \emph{fenómeno}, creyeron mis padres que yo viviría
poco, y maldecían mi ciencia como sugestión de espíritus maléficos. Pero
al fin profesores y familia convinieron en que yo era un prodigio, con
más intervención de las potencias celestes que de las demoníacas, y sólo
se pensó en equilibrarme con buenas magras y un cuidado exquisito de mi
nutrición. Ello es que a los catorce y a los dieciséis años ostentaba yo
variados conocimientos en Humanidades y en Historia, y a los diecinueve
era más filósofo que los primeros que en el Seminario de San Bartolomé
gozaban de esta denominación. Devoré cuantos libros atesoraban aquellas
henchidas bibliotecas y otros muchos que por conductos diferentes a mí
llegaron; poseía el don de una memoria tan holgada, que en ella, como en
inmenso archivo, cabía cuanto yo quisiera meter; poseía también la
facultad de vaciarla, sacando de mis depósitos con fácil y seductora
elocuencia todo lo que entraba por las lecturas, y lo mucho que daba de
sí mi propio caletre. Antes de cumplir los cuatro lustros, mis adelantos
eran tales, que los maestros y yo reconocimos haber llegado al
\emph{summum} del conocimiento posible en cátedras de Sigüenza, y que ni
yo ni ellos podíamos saber más.

En esto, un eclesiástico de espléndida fama como teólogo y canonista, D.
Matías de Rebollo, primo de mi madre, protegido de Don José del Castillo
y Ayenza (que como asesor de la Embajada le llevó a Roma, dejándole
después en la Rota), recaló un verano por Sigüenza, y no bien hizo mi
descubrimiento, propuso a mis padres llevarme consigo a la llamada
Ciudad Eterna, para que en ella diese la última mano a mis estudios y
recibiera las órdenes sagradas. Por su posición y valimiento en la Corte
Pontificia podía el buen señor dirigirme en la carrera sacerdotal y
empujarme hacia gloriosos destinos\ldots{} Mi juvenil ciencia, que a
todos deslumbraba, y la dulzura de mi trato inspiraron a D. Matías un
ansia muy viva de cuidarme y protegerme; y a las dudas de mis padres,
que no querían separarse de mí, contestaba con la brutal afirmación de
llevarme aunque fuera entre alguaciles. Por fin, mi madre, que era quien
más extremaba la fuerza centrípeta por ser yo el Benjamín de la familia,
cedió tras largas disputas que de lo familiar subían a lo teológico, y
sublimado su amor hasta el sacrificio, entregome al reverendo canonista,
pidiendo a Dios los necesarios años de vida (que no habían de ser
muchos) para verme volver con mitra y capelo.

Ved aquí el porqué de mi partida para Italia. Sabed también que me
instalé en Roma en Septiembre del 45, bajo el pontificado de Gregorio
XVI, el cual al año siguiente pasó a mejor vida, y que aposentado en la
propia casa de mi protector, fui atacado de malaria y estuve a dos dedos
de la muerte; que restablecido concurrí a las cátedras de la
\emph{Sapienza} y a otros centros de enseñanza, disponiéndome para la
tonsura. De lo que en el transcurso del 46 hice, y de lo que no hice; de
lo que me ocurrió por sentencia de los hados, y de lo que mi voluntad o
irresistibles instintos determinaron, hablaré otro día, pues para ello
necesito prepararme de sinceridad y aun de valor\ldots{} ¿Debo decirlo,
debo callarlo? ¿Qué cualidad preferís en el historiador de sí mismo: la
melindrosa reserva o la honrada indiscreción?

\emph{23 de Octubre}.---Molido y hambriento llego a Teruel. Uno de mis
compañeros de suplicio, que con sus donosas ocurrencias amenizó el
molesto viaje en la galera, me decía, cuando avistamos la ciudad, que se
comería las momias de los amantes si se las sirvieran puestas en adobo
con un buen moje picante y alioli\ldots{} En la posada, un arrumbado
catre es para mis pobres huesos mejor que la cama de un rey, y la olla
con más oveja que vaca, manjar digno de los dioses. Mientras como y
descanso, no se aparta de mi mente el compromiso en que estoy de referir
los graves motivos de mi regreso a la patria. Ello es un tanto delicado;
pero resuelto a perpetuar la verdad de mi vida para enseñanza y
escarmiento de los venideros, lo diré todo, encerrando la vergüenza con
la izquierda mano, mientras la derecha escribe; y por fin, las
precauciones que tomo para que nadie me lea hasta después de mi muerte
(que Dios dilate luengos años), quitan terreno a la vergüenza y se lo
dan a la sinceridad, la cual debe producirse tan desahogadamente, que,
más que Memorias, sean estas páginas Confesiones.

Al relato de mi salida de Roma precederán noticias del tiempo que allí
estuve. Algo y aun algos hay en esta parte de mi existencia que merece
ser conocido. Mi protector era demostración viva de la flexibilidad de
los castellanos en tierras extranjeras; adaptábase maravillosamente a
los usos romanos, reblandeciendo la tosquedad austera del carácter
español para que como cera tomase las formas de una nación y raza tan
distintas de la nuestra. Desde que le vi en Roma, D. Matías me parecía
otro, y su habla y sus dichos, sus maneras y hasta sus andares, no eran
los del clérigo seguntino austero y grave, con menos gracia que
marrullería, siempre dentro del correcto formulario de nuestra encogida
sociedad eclesiástica. Desde que desembarcamos en Civitavecchia, tomó
los aires del \emph{prete} romano y la desenvoltura graciosa de un
palaciego vaticanista. La severidad de que blasonaba en España, cayó de
su rostro como una careta sofocante, y le vi respirando bondad,
indulgencia, y preconizando en la práctica toda la libertad y toda la
alegría compatibles con la virtud. Espléndida era su mesa, y extensísimo
el espacio de sus amistades y relaciones, comprendidas algunas damas
elegantes que frecuentaban su trato sin el menor detrimento de la
honestidad. Digo esto para explicar que no aprisionara mi juventud en la
estrechez de las obligaciones escolares, ni me encerrara en conventos o
seminarios de rigurosa clausura. Confiado en la sensatez que mi
apocamiento le revelaba, y creyéndome exento de pasiones incompatibles
con mi vocación, me instaló en su propio domicilio, fijándome horas para
concurrir a las cátedras de la \emph{Sapienza}, horas para leer y
estudiar en casa, y dejándome lo restante del día en el franco uso de mi
libertad. Debo indicar que ésta consistía en andar y rodear por Roma con
dos muchachos de mi edad, de familia ilustre, que tenían por ayo a un
modenés llamado Cicerovacchio, personaje mestizo de laico y clérigo,
árcade, mediano poeta, buen arqueólogo, reminiscencia interesante de los
abates del siglo anterior.

Que fue para mí gratísima tal compañía, y muy provechosas aquellas
deambulaciones por la grande y poética Roma, no hay para qué decirlo. A
los tres meses de fatigar mis piernas corriendo de uno en otro monumento
y de ruina en ruina, y al través de tantas maravillas enteras o
despedazadas, ya conocía la ciudad de las siete colinas como mi propia
casa, y fui brillante discípulo del buen Cicerovacchio en antigüedades
paganas y papales, y casi su maestro en el conocimiento topográfico de
la \emph{magna urbs}, desde la plaza del Popolo a la vía Apia, y desde
San Pedro a San Juan de Letrán. El \emph{Campo Vaccino} fue para mí
libro sabido de memoria, y los museos del Vaticano y Capitolio
estamparon en mi mente la infinita variedad de sus bellezas. A los seis
meses hablaba yo italiano lo mismo que mi lengua natal; los pensamientos
se me salían del caletre vestidos ya de las galas del \emph{bel
parlare}, y metidos Maquiavelo y Dante, Leopardi y Manzoni dentro de mi
cerebro, me enseñaban a componer verso y prosa, figurándome yo que no
era más que una trompa o caramillo por donde aquellas sublimes voces
hablaban.

No quiso Dios que me durase mucho esta dulce vida, y sentenciándome tal
vez a ser contrastado por pruebas dolorosas, convirtió la tolerancia de
mi protector en severidades y desconfianzas, que poniendo brusco término
a mi libertad iniciaron el incierto, novísimo rumbo de mi existencia,
como diré cuando tenga ocasión y espacio en las pausas de este camino. Y
por esta noche, ¡oh Posteridad que atenta me escuchas!, no tendrás una
palabra más, que me caigo de sueño, y con tu licencia me voy al
camastro.

\hypertarget{ii}{%
\chapter{II}\label{ii}}

\textbf{Molina de Aragón}, \emph{27 de Octubre}.---Vedme aquí alojado y
asistido a cuerpo de rey, en casa de unos primos de mi padre, los
Ximénez de Corduente, labradores ricos, hechos a la vida oscura y fácil
de estos tristes pueblos, con las orejas enteramente insensibles a todo
mundanal ruido. Para obsequiarme a sus anchas, hácenme comer cinco veces
más de lo que soporta mi estómago, y como no valen protestas ni excusas
contra tan desmedido agasajo, me resigno a reventar una de estas noches.
Adiós Memorias, adiós Confesiones mías: ya no podré continuaros: mi fin
se acerca. Muero de la enfermedad contraria al hambre\ldots{} Luego,
estos azarantes primos de mis pecados, curioseando de continuo en
derredor de mí, me privan del sosiego necesario para escribir. Pongo
punto\ldots{} Quédese para mejor ocasión, si escapo con vida de estos
atracones.

\textbf{Anguita}, \emph{29}.---Aquí paso la noche, y en la soledad de mi
alojamiento angosto y frío, me dedico a escribir lo que me dejé en los
tinteros de Molina. Y ahora que estoy, por la gracia de Dios, a nueve
leguas largas de los Ximénez de Corduente, y no pueden refitolear lo que
escribo, voy a vengarme de los hartazgos con que me pusieron al borde de
la apoplejía, y en la libertad de mis Confidencias declaro y afirmo que
no hay mayores brutos en toda la redondez de la Alcarria, si alcarreña
es la tierra de Molina. Respecto a los padres atenuaré la calificación,
consignando que por sus prendas morales se les puede perdonar su
estolidez; pero en cuanto a los hijos, no retiro nada de lo dicho: nunca
he visto señoritos de pueblo más arrimados a la cola de la barbarie, ni
gaznápiros más enfadosos con sus alardes de fuerza fruta y su desprecio
de toda ilustración. Y no tomen esto a mala parte los demás chicos de
Molina, que allí los hay tan listos y cortesanos como los mejores de
cualquiera otra ciudad. Sólo contra mis primos va esta flagelación,
porque son ellos raro ejemplo de incultura en su patria. Ni una chispa
de conocimientos ha penetrado en tan duras molleras, y alardean de
ignorantes, orgullosos de poder tirar del arado en competencia con las
pujantes mulas. Mirábanme como a un bicho raro, y viendo la mezquindad
de mi equipaje al volver de Italia, zaherían mi saber de latín y griego.
Ellos son ricos, yo pobre. No les envidio; deme Dios todas las desdichas
antes que convertirme en mojón con figura humana, y príveme de todos los
bienes materiales conservándome el pensamiento y la palabra que me
distinguen de las bestias\ldots{}

Y sigo con mi historia. ¿Queréis saber por qué me retiró su confianza D.
Matías? Ved aquí las causas diferentes de mi desgracia: la inclinación
vivísima que a las cosas paganas sentía yo sin cuidarme de disimularla;
mis preferencias de poesía y arte, manifestadas con un calor y
desparpajo enteramente nuevos en mí; la soltura de modales y
flexibilidad de ideas que repentinamente adquirí, como se coge una
enfermedad epidémica o se inicia un cambio fisiológico en las
evoluciones de la edad; mi despego de los estudios teológicos,
exegéticos y patrológicos, en los cuales mi entendimiento desmentía ya
su anterior capacidad; la insistencia con que volvía los cien ojos de mi
atención a historiadores y filósofos vitandos, y aun a poetas que mi
protector creía sensuales, frívolos y de poco fuste, pues él, por una
aberración muy propia de la monomanía humanista, no quería más que
clásicos latinos, sin poner pero a los que más cultivaron la
sensualidad. Presumo yo que en esta displicencia del bondadoso D. Matías
no tenía poca parte su grande amigo y mecenas el embajador de España, D.
José del Castillo, el cual nunca se mostró benévolo conmigo, y opinaba
por que se me sometiera a un régimen más riguroso, resueltamente
eclesiástico.

Si no me quería bien D. José del Castillo y Ayenza, yo le pagaba en la
moneda de mi antipatía. Aquel señor chiquitín y enteco, desapacible y
regañón, consumado helenista, mas tan celoso guardador de su
conocimiento que a nadie quería transmitirlo, no fue entonces ni después
santo de mi devoción. Cuando llegué a Roma, examinome de poetas griegos,
y hallándome no mal instruido, pero poco fuerte en la lengua, me indicó
los ejercicios que debía practicar, se jactó de la constancia de sus
estudios y me cantó el versate mane; mas no añadió aquel día ni después
ninguna advertencia o nuevo examen por donde yo le debiera gratitud de
discípulo o maestro. Tengo por seguro que él fue quien sugirió a D.
Matías la idea de encerrarme, porque mi buen paisano no veía más que por
los ojos del traductor de Anacreonte, ni apartarse sabía de la órbita de
pensamientos que su amigo le trazaba. Ningún día dejaba Rebollo de meter
sus narices en el \emph{Palazzo di Spagna}, y ambos se entretenían en
dirigir con el cocinero guisos españoles, o en chismorrear de cuanto en
el Vaticano y Quirinal ocurría. En aquellas merendonas y comistrajes de
arroz con mariscos, nació sin duda la resolución de mi encierro, para lo
cual se escogió el colegio de San Apolinar, regido por los frailes del
inmediato convento de San Agustín. Entre uno y otro instituto, próximos
a la plaza Navona, corre la torcida \emph{via Pinellari}, de interesante
memoria para el que esto escribe.

Duro fue el paso de la relativa libertad a la prisión, y mis ojos,
habituados a la plena luz, penosamente se acomodaban a la oscuridad de
tan estrecha vida, con disciplina entre militar y frailesca. Debo
declarar que los agustinos no eran tiranos en el régimen escolar ni en
el trato de los alumnos, y entre ellos los había tan ilustrados como
bondadosos. Gracias a esto, mi pobre alma pudo entrar por los caminos de
la resignación. Pero mi mayor consuelo fue la amistad que desde los
primeros días contraje y estreché con dos mozuelos de mi edad, reducidos
a la sujeción del colegio con un fin penitenciario. Llamábase el uno
Della Genga, perteneciente a la ilustre familia de León XII, antecesor
del que entonces regía la Iglesia; el otro, Fornasari, milanés, de una
familia de ricos mercaderes. Ambos eran muy despiertos y de gentil
presencia. Della Genga sentía inclinación ardiente a la política y a la
poesía, dos artes que allí no rabiaban de verse juntas, y con sutil
ingenio daba romántico esplendor a las ideas subversivas; Fornasari,
revolucionario en música, nos repetía los alientos vigorosos de Verdi y
sus guerreras estrofas, que hacían estremecer los muros viejos, como las
trompetas de Jericó. Su aspiración era dedicarse a cantante de ópera, y
creía poseer una voz de bajo de las más cavernosas. Pero su familia le
queda clérigo, y le sentenció al internado como expiación de travesuras
graves. Fogoso y sanguíneo, el milanés contrastaba con nuestro compañero
y conmigo, pues ambos éramos de complexión delicada, nerviosa y fina.
Della Genga tenía semejanza con Bellini y con Silvio Pellico.

Si yo había entrado en San Apolinar con fama de inteligente y aplicado,
no tardé en adquirirla de negligente y díscolo, mereciendo no pocas
admoniciones de los maestros y del Rector. No había fuerza humana que me
hiciera mirar con interés el estudio de la Escolástica y de la Teología,
y aunque a veces, cediendo a la obligación, intentaba encasillar estos
conocimientos en mi magín, salían ellos bufando, aterrados de lo que
encontraban allí. Fue que, impensadamente, había yo hecho en mi cerebro
una limpia o despejo total, repoblándolo con las ideas que Roma y mis
nuevas lecturas me sugirieron. Ya no tomaba tanto gusto de las
Humanidades puras, ni encerraba la belleza poética dentro de los áureos
linderos del griego y del latín; ya la filosofía que aprendí en Sigüenza
se me salía del entendimiento en jirones deshilachados, y no sabía yo
cómo podría recogerla y apelmazarla en las cavidades donde estuvo; ya
las nociones primarias de la sociedad y de la política, de la vida y de
los afectos, ante mí yacían rotas y olvidadas, como los juguetes que nos
divierten cuando niños, y de hombres nos enfadan por la ridiculez de sus
formas groseras.

Los tres que nos habíamos unido en estrecho pandillaje ofensivo y
defensivo leíamos a escondidas libros vitandos, y los comentábamos en
nuestras horas de recreo. Della Genga introdujo de contrabando las
\emph{Ideas sobre la Historia de la humanidad}, de Herder, y Fornasari
guardaba bajo llave, entre su ropa, el libro de Pierre Leroux \emph{De
l'humanité, de son principe et de son avenir}. Con grandes embarazos
leíamos trozos de ambas obras, que cada cual explicaba luego a los dos
compañeros. El hábito de la ocultación, del misterio, nos llevó a
sigilosas prácticas inspiradas en el masonismo, y no tardamos en
inventar signos y fórmulas con las cuales nos entendíamos, burlando la
curiosidad de nuestros compañeros. Estaban de moda entonces la masonería
y el carbonarismo, y Fornasari, que era el mismo demonio y se había
instruido no sé cómo en los ritos y garatusas de aquellas sectas,
estableció entre nosotros un remedo de ellas, poniéndonos al tanto de
los sistemas y artes de la conspiración. Nos teníamos por representantes
de la \emph{Joven Italia} dentro de aquellos muros, y con infantil
inocencia creíamos que nuestra misión no había de ser enteramente
ilusoria.

D. Matías, que en los comienzos de mi encierro me visitaba con
frecuencia, reprendiéndome por mi desaplicación, iba después muy de
tarde en tarde, y la última vez que le vi me sorprendió por la
demacración de su rostro y por el ningún caso que hacía de mis estudios.
Otra particularidad muy extraña en él me causó pena y asombro: habíame
hablado siempre mi buen protector en castellano neto, sin que empañara
la majestad del idioma con extranjero vocablo. Pues aquel día mascullaba
un italiano callejero que era verdadera irrisión en su limpia boca
española, y cortando a menudo el rápido discurso cual si su
entendimiento trepidara con interrupciones rítmicas y la memoria se le
escapara, decía: \emph{«Ho perso il boccino,»} y esto lo repetía sin
cesar, dando vueltas por la sala-locutorio con una inquietud impropia de
su grave carácter. Despidiose bruscamente sonriendo, y en la puerta me
saludó con la mano como a los niños, y se fue agitando las dos junto a
su cráneo, sin dejar el estribillo \emph{ho perso il boccino}\ldots{}
(se me va la cabeza).

Grandemente me alarmó la extraordinaria novedad en las maneras y
lenguaje de mi protector, y en ello pensé algunos días, hasta que
absorbieron mi atención sucesos que a mí y a mis caros compañeros nos
afectaban profundamente. La imposición de un fuerte castigo al bravo
Fornasari fue parte a que nos declarásemos en rebeldía franca. Mientras
nuestro amigo gemía en estrecho calabozo, discurríamos Della Genga y yo
las fechorías más audaces, sin otros móviles que el escándalo y la
venganza; y por fin, adoptando y desechando diferentes planes
sediciosos, concluimos por escoger el más humano y atrevido; sacar de su
prisión a Fornasari y escaparnos los tres, aventura novelesca cuyos
peligros nos ocultaba el entusiasmo que nos poseía y la jactanciosa
confianza en nosotros mismos. Lo que de fuerza física nos faltaba lo
suplía la astucia, y en aquel trance me revelé yo de revolucionario y
violador de cárceles, porque todo lo urdí con admirable precisión y
picardía, ayudado del claro juicio de mi compañero. La suerte nos
favoreció, y la Naturaleza coadyuvó al éxito de la empresa, desatando
aquella noche sobre Roma una tempestad que nos hizo dueños de los
tejados, pues ni aun los gatos se atrevían a andar por ellos. Amparados
de la oscuridad y del ruido con que los furiosos elementos asustaban a
todos los moradores de San Apolinar, violentamos la prisión de
Fornasari; provistos de sogas escalamos las techumbres, y envalentonados
por la libertad que de fuera nos llamaba, así como por el miedo que de
dentro nos expelía, saltamos al techo de las capillas bajas, de allí a
la sacristía y baptisterio anexo, y por fin a la \emph{via Pinellari},
donde ni alma viviente podía vernos, pues hasta los búhos se guarecían
en sus covachas, y el viento y la lluvia eran encubridores de nuestra
juvenil empresa.

Ya teníamos concertado refugiarnos en el Trastévere y plantar allí
nuestros reales, por ser aquel arrabal propicio al escondite, y además
muy del caso para el vivir económico a que nos obligaba la flaqueza de
nuestro peculio.

Della Genga tenía algún oro, yo un poco de plata, y Fornasari piezas de
cobre. Reunidos en común acervo los tres metales y nombrado yo tesorero,
nos aposentamos cerca de la Puerta de San Pancracio en una casa
modestísima, donde fuimos recibidos con desconfianza por no llevar más
ropa que la puesta. En el aprieto de nuestra fuga, que no nos permitía
ninguna clase de impedimenta, harto hicimos con procuramos el vestido
seglar que había de cubrir nuestras carnes al despojarnos de la sotana.
Fue primera y necesaria diligencia, apenas instalados, comprar algunas
camisas, para que viesen nuestras \emph{locandieras} que no éramos
descamisados; pero no nos valió este alarde de dignidad, porque la
desconfianza patronil no disminuyó, y en cambio creció nuestro miedo al
reparar que nos habíamos metido en una cueva de ladrones y desalmada
gentuza de ambos sexos. Salimos de allí con nuestras ansias, y rodando
por la gran ciudad dimos con nuestros cuerpos en un casucho situado en
la \emph{Bocca della Verità}, donde hallamos acomodo entre gente
pobrísima.

Indudablemente, nuestro destino nos llevaba a situaciones arriesgadas,
pues sin pensarlo nos habíamos ido a vivir en el cráter de un volcán:
debajo de nuestro aposento, en lugar oscuro y soterrado, había una
logia. Lejos de contrariarnos esta peligrosa vecindad, fue para los tres
motivo de contento, y Della Genga, que era tan antojadizo como tenaz, no
paró hasta procurarnos entrada en aquel antro, donde podíamos satisfacer
nuestro candoroso anhelo de masonismo. Lo que allí vi y escuché no
correspondió al concepto que de los sectarios habíamos formado los tres
en nuestras íntimas conversaciones. Mi desilusión fue, sin duda, mayor
que la de mis amigos. Fornasari largó una noche un discurso lleno de
hinchados disparates; pero su espléndida voz triunfó de los desvaríos de
su lógica, y le aplaudieron a rabiar.

Hubiera yo querido que durante el día nos ocupáramos en algo que nos
trajese medios de sustento, y que destináramos las noches a cosas
distintas del vagar por calles y plazuelas, o del servir de coro trágico
en la logia; pero la desmayada voluntad de Della Genga no me ayudaba en
mis iniciativas, y el otro parecía encontrar en la profesión masónica el
ideal de sus ambiciones. En esto sobrevino la muerte del papa Gregorio
XVI, motivo de grande emoción en Roma, y en nuestra pequeñez no pudimos
sustraernos al torbellino de opiniones y conjeturas referentes a la
incógnita del sucesor. Durante muchos días no hablábamos de otra cosa, y
cada cual tomaba partido por este o el otro candidato: ¿Sería elegido
Lambruschini? ¿Seríalo Gizzi? A tontas y a locas, y sin ningún
conocimiento en que fundar mi presunción, yo \emph{patrocinaba} a Mastai
Ferretti: era mi candidato, y lo defendía contra toda otra probabilidad,
cual si hubiera recibido secretas confidencias del Espíritu Santo. Della
Genga apostaba por Lambruschini, amigo de la familia y hechura de León
XII; Fornasari, oficiando de cónclave unipersonal, votaba por Gizzi, que
gozaba opinión de liberal con ribetes de masónico, como había demostrado
en su gobierno de la Legación de Forli. Iba más lejos Fornasari,
asegurando que Gizzi tomaría el nombre de Gregorio XVII. De mi candidato
Mastai se burlaban mis compañeros, declarando el uno que Austria no le
quería, y que Francia y Bélgica apoyaban resueltamente a Gizzi. En estas
disputas llegaron los perros\ldots{} quiero decir los criados de Della
Genga, a punto que entrábamos en la \emph{trattoria} de la plaza Cenci,
a dos pasos del \emph{Ghetto}, y ayudados de polizontes cogieron al
prófugo caballerito, y poco menos que a viva fuerza se le llevaron.
Escapamos Fornasari y yo corriendo como exhalaciones.

¡Cuán triste fue la pérdida, o digamos salvación, de nuestro amigo!
Aquella noche, viéndonos sin su compañía en el sucio camaranchón,
lloramos como si se nos hubiera muerto un hermano. Y a la noche
siguiente, hallándome yo dolorido de todo el cuerpo, salió Fornasari a
comprar en la tienda cercana algunas fruslerías para nuestra nutrición,
que de manjares, ¡ay!, muy pobres nos sustentábamos. Le esperé toda la
noche, y no pareció\ldots{} Para no cansar: ésta es la hora en que no he
vuelto a verle; ni volvió, ni he sabido más de mi desgraciado amigo.
Digo desgraciado, por no saber qué decir. Pasados tres días de ansiedad
e inanición, salí de mi tugurio, no con intento de buscar al perdido,
sino de alejarme de aquellos lugares, en que de continuo turbaba mis
oídos runrún de polizontes.

Amparado de la callada noche, me fui hacia Monte Testaccio, donde tuve
la suerte de encontrar un alfarero que quiso admitirme, sin más
estipendio que la comida, a las faenas de su industria, aplicándome a
dar vueltas a la rueda del artefacto con que amasaba la arcilla. El
primer día, ¡cosa más rara!, me agradó el continuo revolver de noria,
que a pensar me estimulaba. Pero pronto hube de cansarme de aquel método
de raciocinio, y como el pienso no era bueno ni me daba el necesario
vigor para sostener mis funciones de caballería pensante, me despedí. La
vagancia, la mendicidad, el dormir en bancos al raso o bajo pórticos del
Campo Vaccino, el comer lo que me daban en porterías de hospicios o
conventos, fueron mis modos de existencia en aquellos tristes días.
Harto ya de sufrir ayuno de buenos alimentos, y cubierto de andrajos,
llegué al límite en que mi dignidad se reconciliaba con mis angustiosas
necesidades físicas. Viendo en mí la dramática situación del \emph{Hijo
Pródigo}, me decidí a volver a la casa de mi buen D. Matías. Costome no
pocas ansiedades el resolverlo, y tan pronto caminaba hacia allá, como
retrocedía, con terror de merecidas reprimendas\ldots{} Por fin cerré
los ojos, y llena el alma de contrición y humildad, llamé a la puerta de
mi salvación, en la plaza de \emph{San Lorenzo in Lucina}. Abrió un
criado vestido de luto, que no me conoció: tan lastimosa era mi facha.
Insistí en que no era yo un pobre desconocido que imploraba limosna: mi
voz reveló lo que ocultaban mis harapos. Al fámulo se unió la cocinera,
y con fúnebre dúo de \emph{requiem} me dijeron que mi protector había
muerto. ¡Oh súbita pena, oh inanición cruel!\ldots{} Mi turbada
naturaleza no supo separar el noble sentimiento del brutal instinto, y
llorando me abalancé a la comida que me ofrecieron.

\hypertarget{iii}{%
\chapter{III}\label{iii}}

\textbf{Sigüenza}, \emph{Noviembre}. Al amanecer de hoy, bajando de
Barbatona, vi a la gran Sigüenza que me abría sus brazos para recibirme.
¡Oh alegría del ambiente patrio, oh encanto de las cosas inherentes a
nuestra cuna! Vi la catedral de almenadas torres; vi San Bartolomé, y el
apiñado caserío formando un rimero chato de tejas, en cuya cima se alza
el alcázar; vi los negrillos que empezaban a desnudarse, y los chopos
escuetos con todo el follaje amarillo; vi en torno el paño pardo de las
tierras onduladas, como capas puestas al sol; vi, por fin, a mi padre
que a recibirme salía con cara doble, mejor dicho, partida en dos, media
cara severa, la otra media cariñosa. Salté del coche para abrazarle, y
una vez en tierra, hice mi entrada a pie, llegando a la calle de
Travesaña, donde está mi casa, con mediano séquito de amigos, y de
pobres de ambos sexos, ciegos, mancos y cojos, que sabedores de mi
llegada querían darme la bienvenida\ldots{} La severidad de más cuidado
para mí, que era la de mi padre, se disolvió en tiernas palabras. Verdad
que de mis horrendas travesuras en Roma no le habían contado sino parte
mínima. Seguía, pues, creyendo con fe ciega en mi glorioso destino
eclesiástico, y suponía que, al regresar a la patria, almacenadas traía
en mi cerebro todas las bibliotecas de Italia. Mi hermano Ramón fue
quien más displicente y jaquecoso estuvo conmigo, anunciándome que si no
me determinaba a recibir las órdenes en España, aspirando a un curato de
aldea, o cuando más a una media ración en aquella Santa Catedral, la
familia tendría que abandonarme, dejándome correr por los caminos más de
mi gusto, ora fuesen derechos, ora torcidos\ldots{} De todo esto hablaré
más oportunamente, pues anhelo proseguir lo que dejé pendiente de mi
romana historia.

Pego la rota hebra diciendo que el mayordomo de mi tío, Cristóbal Ruiz,
español italianizado que había sido fámulo en Monserrat, me informó de
la dolencia y muerte del bendito Rebollo. Había sido un lamentable
desarreglo de la mente, motivado, según colegí de las medias palabras de
Ruiz al tratar este punto, por agrias discordias con otros clérigos de
la Rota. De mis desvaríos en San Apolinar y de mi escandalosa fuga y
vagancia no dieron al buen señor conocimiento, pues ya había perdido el
suyo, y desprovisto de memoria y de juicio, su vocabulario quedó
reducido al \emph{ho perso il boccino}, que estuvo repitiendo hasta el
instante de su muerte. Quién se cuidó de participar a mi familia, con el
fallecimiento de Rebollo, mis atroces barrabasadas, es cosa que no he
sabido con certeza; pero, si no me engaña el corazón, el encargado de
esta diligencia fue un secretario del embajador Don José del Castillo.
Díjome también Cristóbal Ruiz que una radical divergencia en la manera
de apreciar no sé qué asunto de derecho canónico había turbado
profundamente la cordial amistad entre el representante de España y su
protegido, llevando a éste al remate de su delirio. Cuando apenas se
había iniciado la dolencia, hizo D. Matías testamento, nombrando
ejecutor de sus disposiciones a otro de sus mejores amigos, monseñor
Jacobo Antonelli, segundo tesorero, o como si dijéramos, secretario de
Hacienda, persona muy bien mirada en la Corte Pontificia por su talento
político y su mundana ciencia. Al tal sujeto habría yo de presentarme;
pues, según Ruiz, debía tener instrucciones de Rebollo referentes al
cuidado de mis estudios y a la paternal tutela que conmigo ejercía.

Vacilando entre la vergüenza de presentarme a Monseñor y el estímulo de
poner fin a mi desamparo, pasaron algunos días que no fueron malos para
mí, pues me hallaba asistido de ropa, casa y alimento, y además libre,
con toda Roma por mía, para pasar el tiempo en amena vagancia,
reanudando mis amistades de artista y de arqueólogo con tantas grandezas
muertas y vivas. Los ruidosos acontecimientos de aquellos días de junio
me arrastraban a vivir en la calle, siempre con la esperanza de tropezar
con mis perdidos camaradas Fornasari y Della Genga. Mientras duró el
Cónclave que debía darnos nuevo Papa, me confundí con las multitudes que
aguardaban ansiosas en Monte Cavallo. En la noche del 16 al 17, corrió
la voz de que había sido elegido Mastai, lo que fue para mí motivo de
grandísimo contento, porque el Espíritu Santo me daba la razón contra
mis amigos. Al día siguiente, vi al cardenal camarlengo monseñor Riario
Sforza salir al balcón del Quirinal, pronunciando con viva emoción el
\emph{Papam habemus}. ¡Y era Mastai Ferretti, mi candidato, el mío,
\emph{qui sibi imposuit nomen Pium IX}! A las aclamaciones de la
multitud uní todo el griterío de que eran capaces mis pulmones, y cuando
el nuevo Pontífice salió a dar al pueblo romano su primera bendición,
creí volverme loco de entusiasmo y alegría. Si mil años viviera, no se
borraría de mi alma la impresión de aquellos solemnes instantes, ni
tampoco la del 21 en San Pedro, inolvidable día de la coronación.
Imposible que dé yo idea del cariño que despertó el nuevo Papa. Toda
Roma le amaba, y yo, con íntima efusión que no sabía explicarme, le
amaba también y le tenía por mío, sin dejar de ver en él el amor de
todos, creyendo cifradas en su persona la felicidad de Roma y de Italia.

Decidido a presentarme al famoso Antonelli, pues algún término había de
tener mi vagabunda interinidad, vi aplazada de un día para otro la
audiencia que solicité. Monseñor fue nombrado Ministro de Hacienda,
después Cardenal. Los negocios de Estado y las atenciones sociales
alejaban de su grandeza mi pequeñez. Por fin, una tarde de julio me
llamó a su casa, y fui temblando de esperanza y emoción. Recibiome en su
biblioteca, y se mostró desde el primer momento tan afectuoso que ganó
mi confianza, haciéndome desear que llegase una feliz ocasión de
confiarle todos mis secretos. Era un hombre alto y moreno, de mirada
fulminante, de rasgada y fiera boca con carrera de dientes
correctísimos, que ostentaban su blancura dando gracia singular a la
palabra. El rayo de sus ojos de tal modo me confundía, que no acertaba
yo a mirarle cuando me miraba. Sujetome a un interrogatorio prolijo, y
con tal arte y gancho tan sutil hacía sus preguntas, que le referí todas
mis maldades, sintiéndome muy aliviado cuando no quedó en mi conciencia
ninguna fealdad oculta. A mi sinceridad correspondió Su Eminencia
poniendo en su admonición un cierto aroma de tolerancia, que del fondo
de su pensamiento a la superficie de sus palabras severas trascendía.

Díjome, entre otras cosas que procurase fortalecer mi quebrantada
vocación religiosa, redoblando mis estudios, aislándome del mundo y
reedificando mi ser moral con meditaciones. Insistí yo en manifestarle
que me sería muy difícil sostener mi vocación; pero que aplicaría a tan
grande intento toda mi voluntad, sometiéndome a cuantos planes de
conducta me señalara y sistemas educativos se sirviera proponerme. No me
acobardaban los estudios penosos; pero el internado y la disciplina
cuartelesca de los principales centros de enseñanza no se avenían con mi
natural inquieto, ni con las osadas independencias que me habían nacido
en Roma, como si al pisar aquella tierra me salieran alas. Sin duda le
convencí, ¡no era flojo triunfo!, porque me propuso hacer conmigo esta
prueba: durante un año emprendería yo formidables estudios, conforme a
un plan superior acomodado a mi primitiva vocación, y sin someterme a la
esclavitud del internado. Enumerando el programa de mis tareas, señalome
el \emph{Colegio Romano} para las ciencias eclesiásticas, la
\emph{Sapienza} para la Jurisprudencia y Filosofía, y para las lenguas
sabias el colegio de la \emph{Propaganda}, regido a la sazón por el
portentoso políglota Mezzofanti. En todo convine yo, con expresiones de
reconocimiento, y éste subió de punto cuando el Cardenal me manifestó
que cuidaría de alojarme, si no en su propia casa, junto a personas de
su familiaridad o servidumbre, en lo cual no hacía nada extraordinario,
pues D. Matías había dejado caudal suficiente para ésta como para otras
sagradas atenciones. Encantado le oí, y mayor fue mi entusiasmo cuando
al despedirme me ordenó volver tres días después.

En la segunda entrevista, disponiéndose Su Eminencia a partir para
Castel Gandolfo, recreo estival del Papa, me indicó que fuese a pasar
las vacaciones a su quinta de Albano, donde hallaría dispuesta una
estancia. Me encargaba del arreglo de su biblioteca, que tenía en gran
desorden: innumerables libros sin catalogar, y todos los que fueron de
D. Matías metidos en cajas, esperando ser clasificados por materias y
puestos en los estantes. No me dio tiempo ni a expresarle mi gratitud,
porque el coche le aguardaba a la puerta. Salió para Castel Gandolfo, y
yo al siguiente día para Albano, gozoso, con ilusiones frescas y ganas
de vivir, creyendo que la vida es buena y que en ella hay siempre algo
nuevo que ver y descubrir.

La residencia del Cardenal en Albano es arreglo de una incendiada
\emph{villa} de los Colonnas, recompuesta modestamente. Elegantísima
puerta del Renacimiento se da de bofetadas con ventanas vulgares. Restos
de soberbia escalinata son el ingreso de la biblioteca, y en las cocinas
hay un friso con bajorrelieves. La misma confusión o engarce de riquezas
muertas con vivas pobrezas se advierte en el jardín, donde permanece un
trozo en setos vivos de ciprés lindando con plantíos nuevos y cuadros de
hortaliza. Hermosa es por todo extremo la situación del edificio, al sur
de la ciudad, no lejos de la nueva vía Apia. Desde la ventana de mi
aposento veía yo el sepulcro de los Horacios y Curiacios, y los montes
Albanos y los pueblecitos de Ariccia y Genzano\ldots{} Tal era el
desorden de la biblioteca, que empleé todo el verano en remediarlo; y
absorto en faena tan grata para mí, se me iba el tiempo sin sentirlo, en
dulce concordia con los habitantes de la casa, que me asistían
cariñosamente y me tenían por suyo. Siete mujeres había en la
\emph{villa}, y aunque viejas en su mayor parte (dos eran niñas de
catorce a quince años), gustábame su cordial trato. Entendí que eran
familias de la servidumbre jubilada del Cardenal, que conservaba los
criados aun en el período de su decadencia inútil. Todo aquel mujerío y
dos hombres, el uno jardinero, cochero el otro, ambos con traza de
bandidos, procedían de Terracina, el país de Antonelli. Las dos
\emph{ragazze}, una de las cuales era bonitilla, la otra jorobada, me
ayudaban juguetonas y alegres en mis tareas de bibliófilo, y al caer de
la tarde nos íbamos a dar una vuelta por las orillas del lago Albano, o
emprendíamos despacito y charlando la ascensión al monte Cavo para gozar
la vista de todo el territorio albano y del mar, incomparable belleza de
suelo y cielo, ante la cual acompañado me sentía de los antiguos dioses.

Terminadas las vacaciones, volví a Roma con cuatro de aquellas mujeronas
y la corcovadita, y empecé mis estudios, instalado en el piso alto del
palacio de Su Eminencia, en el Borgo-Vecchio. Comenzó para mí una vida
monótona y de adelantos eficaces en mis conocimientos. Los estudios de
lenguas orientales en la \emph{Propaganda} me cautivaban; tanto allí
como en la\emph{Sapienza} hice amistades excelentes, y un día de
diciembre tuve la inefable sorpresa de encontrarme a Della Genga, que me
abrazó casi llorando. Sus padres, convencidos al fin de que a la
naturaleza varonil del chico se ajustaba mal la sotana, dedicáronle a la
jurisprudencia y al foro. Estaba mi hombre contento y orgulloso de su
moderada libertad. Restablecida nuestra fraternal concordia, juntos
estudiábamos y juntos nos permitíamos algún esparcimiento propio de la
juventud. Debo declarar con toda franqueza que Della Genga me corrompió
un tantico, y empañó la pureza de mi moral en aquellos días,
comunicándome eficazmente, hasta cierto punto, su innata afición a la
mitad más amable del género humano. Acúsome de esto, afirmando en
descargo mío que mis debilidades no pasaron de la medida discreta. Y
para que todo sea sinceridad, añadiré que no tuvo poca parte en mi
comedimiento mi escasez de dineros, la cual vino a ser un feliz arbitrio
de la Providencia para preservarme de chocar contra escollos, o de ser
arrastrado en vertiginosos remolinos.

\hypertarget{iv}{%
\chapter{IV}\label{iv}}

\emph{Majora canamus}.---Igualábame Della Genga en la admiración al
nuevo Pontífice y en creerle como enviado del Cielo para devolver a
Italia su grandeza, y dar a los pueblos fecundas y libres instituciones.
Toda Roma creía lo mismo. Mastai Ferretti sería como un pastor de todas
las naciones, que sabría conducirlas por el camino del bien eterno y de
la terrestre felicidad. Cuantas disposiciones tomaba el Santo Padre eran
motivo de festejos, y las iluminaciones con que fue celebrada la
amnistía repetíanse luego por motivos de menos trascendencia. Siempre
que a la calle salía Pío IX, se arremolinaba la multitud junto a su
carruaje, y los vivas y aclamaciones, repitiéndose en ondas, conmovían a
toda la ciudad. Por cualquier suceso dichoso, y a veces sin venir a
cuento, se improvisaban procesiones y cabalgatas, y las sociedades que
habían sido secretas y ya se habían hecho públicas, salían con sus
abigarrados pendones entonando himnos. Pasado algún tiempo de esta
patriótica efervescencia, el entusiasmo empezó a degenerar en delirio, y
las demostraciones en vocerío y alborotos.

Era Della Genga devotísimo de las ideas de Gioberti, y yo no le iba en
zaga. Habíamos leído y releído el \emph{Primato degli italiani}, y
soñábamos con la redención de Italia y su gloriosa unidad bajo la sacra
bandera del Vicario de Cristo. Esto pensaba yo, y con inquebrantable fe
pensándolo sigo y me creo portador de tan saludables ideas a mi querida
patria. Pío IX, que en sus virtudes preclaras, en su poderoso
entendimiento y hasta en su rostro plácido y expresivo, conquistador de
voluntades, trae el sello de una misión divina, efectuará la
restauración civil de la península itálica, inmensa obra que no ha
podido ser realidad por no haberse empleado en ella el ligamento de las
creencias comunes, de la enseñanza católica. Roma será, pues, la
metrópoli de la Italia moral, y cabeza de la política, y creará un
pueblo robusto, tan grande por la fuerza como por la fe. El báculo de
San Pedro guiará en esta conquista a los italianos, enseñando a la
Europa entera el camino de la fecunda libertad. De esta idea y de sus
infinitas derivaciones hablábamos mi amigo y yo a todas horas, siempre
que nuestra malicia o la frivolidad propia de muchachos no nos llevaban
a conversaciones menos elevadas.

Y escribíamos sobre el mismo tema político sendas parrafadas ampulosas,
que nos leíamos \emph{ore alterno} buscando el aplauso, y éste
fácilmente coronaba nuestras lucubraciones. Por cierto que un día
(pienso que por febrero de este año) mi orgullo me sugirió la idea de
mostrar al Cardenal una enfática disertación que escribí sobre el magno
asunto de la época, con el título de \emph{Risorgimento dell'Italia una
e libera}, y quedándose con mi mamotreto para leerlo en el primer rato
que tuviera libre, a los ocho días me llamó para decirme que no estaba
mal pensado ni escrito; pero que no robase tiempo a mis estudios para
meterme a divagar sobre lo que ya habían tratado las mejores plumas
italianas. Comprendiendo que ni mi discurso ni la materia de él eran de
su agrado, salí de la presencia del grande hombre un tanto corrido.

Bien entrada ya la primavera, un ataquillo de malaria, que me cogió
debilitado, interrumpió en mal hora mis estudios y hube de guardar cama,
presentándose la calentura tan insidiosa que ni alivio ni recargo sentí
en todo un mes. Por fin, el Cardenal me mandó a Subiacco, acompañado de
la jorobadita y de una de las vejanconas. El puro aire de los montes
Albanos me restableció en otro mes de régimen severo y de mental
descanso; pero no pude asistir a exámenes ni pensar en nueva campaña
escolar hasta el otoño próximo, lo que sentí de veras, porque en la
\emph{Propaganda} me iba encariñando con el hebreo y sánscrito, y en la
Sapienza figuraba entre los más lúcidos estudiantes de Patrología y de
Lugares teológicos, sin olvidar la Jurisprudencia, Concilios, etc.

Y heme de nuevo, apenas apuntaron los calores de julio, en la placentera
residencia de Albano, libre y bien atendido, compartiendo mis horas
entre los paseos por las alamedas que conducen a Castel Gandolfo, o por
la nueva vía Apia, y el trajín de la biblioteca, que me recibió como un
viejo amigo brindándome con todo el embeleso de sus mil libros
interesantes, apetitosos, llenos de erudición los unos, de amenidad los
otros. ¡Oh soledad dichosa, oh dulce presidio!

De un verano a otro, había cambiado el personal de la villa, pues dos
ancianos murieron, otros dos se habían ido a Terracina, y en su lugar
hallé un matrimonio de edad avanzada y dos mozas muy guapas: una de
ellas, a poco de estar yo allí, fue conducida a Frascati, donde
veraneaba el Cardenal con una noble familia polaca. La que en casa quedó
no era jovenzuela, sino propiamente mujer y aun mujerona, de más que
mediana talla, esbelta, gran figura, tipo romano de lo más selecto,
cabello y ojos negros, la tez caldeada, con tono de barro cocido. Su
trato pareciome un poco salvaje, como recién cogida con lazo en los
campos de Terracina; vestía poco, despreciando las modas y prefiriendo
los trajes de su pueblo. ¿Era casada o viuda? Nunca lo supe, pues de sus
palabras a veces se colegía que el esposo había fenecido en la plenitud
de sus hazañas bandoleras, a veces que se había marchado a Buenos Aires.
Esta doble versión podía explicarse por el hecho de que no fuese un
marido, sino dos los que ya contaba en su martirologio. No insistí yo
mucho en inquirirlo, pues noté en la buena moza marcada repugnancia de
los estudios biográficos. Llamábanla Bárbara o Barberina, nombre que le
cuadraba maravillosamente, porque leía muy mal y apenas sabía escribir;
mas con su natural despejo disimulaba tan graciosamente la ignorancia,
que valía más su conversación que la de veinte sabios. Gustaba yo de
charlar con ella, más que por la rudeza de sus dichos, por verle los
blanquísimos dientes que al sonreír mostraba, y admirar el encendido
color de su rostro iluminado por la elocuencia de mujer burlona.

Pero no se crea que las burlas, a que tan aficionada era, escondían un
carácter avieso y malicioso, no. Era muy buena la salvaje Barberina, y a
mí me tomó decididamente bajo su amparo y protección, y me cuidaba como
a hermano. Viéndome tan endeblucho, se desvivía por reparar mi
quebrantado organismo, dándome calditos o infusiones entre horas, y
haciéndome el plato en las comidas con propósito de llenarme el buche de
cosas sustanciosas y bien digeribles. Guardaba en sus bolsillos
golosinas para obsequiarme, de sorpresa, cuando paseábamos junto al lago
con la jorobadita y otras muchachas, y atendía también singularmente a
mi descanso nocturno, evitando todo ruido en la \emph{villa}, y alejando
de mi aposento la caterva de gatos y perros que en la casa tenían su
albergue.

Agradecido a tantas bondades, se me ocurrió la felicísima idea de
pagarle sus beneficios con otros no menos valiosos. Cualquiera, por
egoísta que fuese, habría pensado lo mismo, ¿verdad? Ella cuidaba de mi
corporal existencia, dándome salud y robustez; pues yo cuidaría de
embellecer su espíritu, dándole el jugo de la ilustración, de que se
alimentan los seres escogidos, \emph{etcétera}\ldots{} En fin, que si
ella me nutría, yo la educaba, le devolvía sus obsequios
perfeccionándola en la lectura y enseñándola a escribir correctamente.
Cuánto se holgó Barberina de mi plan de recíproca beneficencia, no hay
por qué decirlo. Al punto empezamos la campaña, brindándonos a ello el
tiempo que en aquel apacible retiro nos sobraba, y el sosiego de la
retirada y fresca biblioteca. La hice leer \emph{I Promessi Sposi}, y
advirtiendo su predilección por lo que más hería su sensibilidad, nos
metimos con los poetas, prefiriendo los modernos, para huir del estorbo
de los arcaísmos. Con tal cariño tomó estas lecturas, que al fin se me
hizo largo el espacio de sus lecciones. Y yo no volvía de mi sorpresa
viendo que todo lo comprendía, que ninguna delicadeza de sentimiento, ni
alegórica ficción, ni gallardía de estilo se le escapaba. Y cuando nos
poníamos a comentar, ¡qué claro juicio en aquella salvaje! Lloraba con
las ternezas religiosas de Manzoni, se entusiasmaba con el fiero
nacionalismo de Monti y de Alfieri, y Leopardi la dejaba no pocas veces
silenciosa y cejijunta.

Menos afortunado era el maestro en la escritura, porque los dedos de la
cerril discípula no conservaban la flexibilidad y sutileza de su virgen
entendimiento. Gustábame guiar aquella dura y fuerte mano, tan bien
modelada que parecía la mano de Minerva o de Ceres. Pero los adelantos
no correspondían a los esfuerzos de ella, acompañados de hociquitos y
muecas con sus carnosos labios, ni a la paciencia y esmero que yo ponía
en mis lecciones. Acababan éstas con los dedos de ambos manchados de
tinta, y con la exclamación de ella lamentando su torpeza. Hecha su mano
al rastrillo, al bielgo, a la pala y a otros rústicos instrumentos, se
avenía mal con la pluma. Por consolar a mi educanda, decíale yo que
trocaría mi buen manejo de escritura por la fuerza y la paz que da la
vida del campo, y que un labrador inteligente es el primero de los
sabios, que con el arado escribe en la tierra el gran libro de la
felicidad humana. Pero estas pedanterías no la curaban de su
desconsuelo, y a la siguiente lección volvía con más empeño a la faena.

Corriendo con lenta placidez los días, Barberina progresaba en la
instrucción, y ambos en la confianza mutua, sin el menor detrimento de
la honestidad. Pedíame ella que le hablase de mi familia y de mi pueblo,
y que le contara cuanto de mi infancia recordaba. De la suya y de su
parentela, así como de su matrimonio, nada me contaba ella, creyendo,
sin duda, que su historia no podía interesarme. Cada día se inquietaba
más por mi salud, y a sus cuidados del orden doméstico añadía discretas
exhortaciones referentes a la vida moral. En sus sermones me incitaba a
la pureza de costumbres, y afeaba mi ardorosa afición a las cosas
paganas. De tiempo en tiempo hacía yo veloces escapadas a Roma,
volviendo con algunos libros o cualquier objeto, cuya compra, según yo
decía, me precisaba. Recibíame Barberina, al regreso, con dolorida
severidad, afirmando que mi salud y aun mi decoro estaban en peligro, si
no me penetraba del respeto que debemos a nosotros mismos y a la
sociedad. Más sutil moralista no he visto nunca. No pude menos de
rendirme a tan sabios consejos, bendiciendo la boca que me amonestaba y
declarando que a cuanto me ordenase había de someterme. Todo el afán de
mi amiga era preservarme de los peligros que en el mundo cercan a una
juventud delicada, y yo, considerando la inmensa valía de esta tutela,
me abrasaba en admiración y reconocimiento.

No disminuía con esto nuestra afición a las lecturas, y si ella leía por
ejercitarse, hacíalo yo por darle el modelo de la entonación y por
entretenerla y deleitarla con útiles pasatiempos. Observé que las cosas
serias la interesaban más que las jocosas, y las humanas, construidas
con elementos de verdad, más que las imaginativas. Después del
\emph{Jacopo Ortis} y de las \emph{Prisiones}, leí parte de la
\emph{Eloísa} de Rousseau, y de aquí saltamos a las \emph{Confesiones},
cuyos primeros capítulos fueron el encanto de Barberina. Burla burlando
llegamos a la presentación de Juan Jacobo en la casa de Madame Warens,
al carácter y figura de ésta, a la maternal protección que dispensó al
joven ginebrino, y por fin, al ingenioso arbitrio de la dama para
preservar a su amiguito de los riesgos que corre un jovenzuelo
impresionable si se le deja solo ante el torbellino del mundo y las
asechanzas del vicio. Admirable nos pareció a entrambos a quel pasaje,
que Barberina alabó con vivos encarecimientos\ldots{} Mi amor a la
verdad me obliga a terminar este relato repitiendo el famosísimo
\emph{quel giorno più non vi leggemmo avanti}.

\hypertarget{v}{%
\chapter{V}\label{v}}

Alegría insensata y sombríos temores alternaban en mi alma desde aquel
día. ¡Amor, conciencia, cuán desacordes vais comúnmente en la vida
humana! Amargaban la dulzura de mi juvenil triunfo sobresaltos y
presentimientos tristísimos, y mi felicidad en ellos se disolvía como la
sal en el agua. Perseguíame el espectro del Cardenal pronunciando la
acusación y cruel sentencia que yo merecía, y en mis sueños me visitaba,
y despierto le sentía próximo a mí. Seguramente no tendría yo valor para
poner mi rostro pecador ante el de Su Eminencia. El temido rayo de sus
ojos me haría caer exánime; me faltaría valor aun para pedirle perdón de
mi vergonzoso ultraje a la ley de hospitalidad.

Algún alivio me dio la noticia, por la propia Barberina comunicada, de
que el Cardenal no parecería en mucho tiempo por Albano, ni aun de paso
para Castel Gandolfo. Desde Frascati, deteniéndose en Roma sólo una
noche, había pasado a Rímini, sin duda con una misión secreta de Su
Santidad para el Embajador de Austria que allí veraneaba. Calculando mis
huéspedes la duración de la ausencia por el equipo y servidumbre que
Antonelli llevaba, presumían que iría también a Viena. No obstante estas
seguridades de respiro, yo no tenía sosiego, y pedía fervorosamente a
Dios que complicase los asuntos diplomáticos de la Santa Sede en
términos tales, que mi protector tuviese que ir también a San
Petersburgo, y de allí a Pekín, atravesando toda el Asia en camello, en
elefante, o en otro vehículo animal de los más lentos.

Por aquellos días empezaron a tomar mal cariz las cosas políticas. La
popularidad del Papa era ya molesta, tirando a la confianza
irrespetuosa: los entusiasmos de la plebe, dirigida por las Sociedades o
Círculos, no eran ya simples alborotos, sino motines en toda regla. Las
concesiones de Su Santidad al espíritu moderno les parecían poco, y ya
pedían la Luna, la Osa Mayor y el Zodíaco entero. El clamor de reformas
era tan intenso, que el adorado Mastai Ferretti se veía compelido a dar
gusto al pueblo nombrando un Ministerio laico. Gustaba yo de la
inquietud, porque no sólo veía en ella la palpitación generatriz del
ideal de Gioberti, tomando carne y forma de cosa real, sino porque el
tumulto y todo aquel revolver de las ondas sociales me parecían a mí muy
propios para que en ellos se escondiera mi delito y quedase ignorado,
impune.\emph{¡Ahi, come mal mi governasti, amore!}

Mas un día, \emph{¡corpo di Baco!}, anunciaron que el Cardenal estaba de
vuelta en Roma, y ya no hubo para mí tranquilidad. Pasó por mi mente la
idea de fugarme: comuniqué este pensamiento a Barberina, la cual me dijo
que había pensado lo mismo. Propúsome que nos fuéramos a España\ldots{}
¡A buena parte!, dije yo. De escapar, a Nápoles para plantarnos en
Egipto, o a Génova para emigrar calladitos a Buenos Aires, donde
pondríamos café, una tienda de bebidas\ldots{} no, mejor un colegio, en
el cual yo abriría cátedra de \emph{omni re scibile}. Felizmente,
ninguno de estos disparates prendió en mi mente, y la irresolución, que
en normales casos suele perdernos, en aquél fue mi salvación\ldots{}
Mientras discutíamos mi amada y yo si nos estableceríamos en Corfú o en
Alejandría, vino un recado de Antonelli, llamándome con urgencia.
¡Ay!\ldots{} ¡ay!

Se me olvidó apuntar que el matrimonio anciano que regía la casa
mirábame ya como cosa perdida. Días antes, notaba yo en sus rostros
cólera, menosprecio, amenaza: cuando me vieron llamado a la presencia
del amo, su actitud era compasiva, como la de los curiosos que asisten
al paso del condenado a muerte, camino de la horca o de la guillotina. Y
en efecto, en mí se determinaba la insensibilidad del reo en la capilla
momentos antes del suplicio. Salí de la casa sin poder ver a Bárbara;
creí que se había encerrado en su habitación. Quise subir, y no me
dejaron. «¡Barberina!» grité desde , y nadie me respondió\ldots{} Partí
con el corazón despedazado, mordiendo mi pañuelo. Luego me dijo el
cochero que aquella madrugada, la buena moza, obedeciendo órdenes
terminantes del Cardenal y guardando el mayor secreto, había partido
para Terracina\ldots{} a pie, sola\ldots{} Y no había miedo de que se
desviara de su ruta, ni que desobedeciera la terrible y concisa orden.
Protesté, lloré, rugí, y el cochero, con filosófico humor y flemático
desdén, me dijo: \emph{«¡Ah, signore!, questo e peggio che
l'Inquisizione. Ma, non dubiti, la sconteranno sti pretacci, figli di
cani.»} Hablamos de política. Pronto comprendí que estaba el hombre
cogido por las sociedades secretas.

«Un hombre, sólo hay un hombre que pueda traernos la revolución.

---¿Y quién es ese hombre?

---Mazzini\ldots»

Mi pena no me dejó espacio para sostener la conversación. ¿Qué me
importaban a mí Mazzini y toda la turbamulta de las logias?

Llegué al palacio del Cardenal con la esperanza de que sus ocupaciones
no le permitirían acordarse de mí, de que no podría recibirme, de que
tendría yo que aguardar horas, días quizás\ldots{} Quedeme aterrado al
ver que el portero, como si me esperase, me mandó pasar en cuanto bajé
del coche, y luego un ujier, sin darme descanso ni respiro, me introdujo
en la biblioteca, donde vi a Su Eminencia despachando con un secretario.
Yo apenas respiraba: yo pensaba en Dios, como el espía, víctima de la
ley de guerra, que es conducido ante el pelotón que ha de fusilarle. Más
atento al despacho que a mí, el grande hombre no se dignó mirarme. Un
cuarto de hora, que hubo de parecerme un cuarto de siglo, duró mi
ansiedad; y cuando el secretario, recogiendo papeles, a marchar se
disponía, yo, paralizado y mudo en el centro de la pieza, extrañaba que
no me vendasen los ojos para el trance fatal.

No vi la mirada de Antonelli cuando me mandó acercarme, porque yo no
podía levantar del suelo mi vista. El tono de su voz no me pareció
demasiado duro. Me atreví a mirarle, y hallé en su rostro un desdén
compasivo, no la cólera de Júpiter que yo esperaba. La angustia que me
oprimía tuvo el primer alivio cuando Su Eminencia me preguntó por mi
salud, aunque debía yo creer que era pura fórmula. Como le contestase,
por decir algo, que no me encontraba bien, díjome que me propondría un
remedio eficaz para la completa reparación de mi organismo. Nueva
sorpresa mía con su poquito de pavor. ¿Cuál era este remedio? No tardó
en decírmelo: el regreso a España. Los aires natales me serían muy
provechosos. Con más miedo que finura contesté que me parecía muy bien.
\emph{Ed egli à me}: «Hijo mío, bien a la vista está que tus esfuerzos
para conservar la vocación religiosa son inútiles. La Naturaleza manda
en ti como señora absoluta, y no sabes cultivar el espíritu robusto que
debe sojuzgarla\ldots» Admirado de tanta sabiduría, nada supe contestar.
Pareciome que aquello de sojuzgar la Naturaleza era también fórmula, y
que Su Eminencia echaba mano de los tópicos que sólo sirven para
aleccionar a la infancia, sin tener más que un valor pedagógico
semejante al de las palmetas. \emph{Poi ricommincio}: «Tus facultades
prodigiosas se pierden en la distracción. Tal vez has errado la vía, y
debes buscar otra en que la distracción misma no sea un impedimento,
sino un estímulo. Para brillar en artes o ciencias no es necesario ser
benedictino. La tutela que me delegó el buen D. Matías, yo la devuelvo a
tus padres, que la ejercerán con más fruto que yo. En Italia te pierdes:
gánate en España, donde empezarás por hacer efectiva tu vocación de
marido\ldots{} Tu familia te procurará un buen matrimonio.»

Pausa. Conmovido pronuncié al fin vagas expresiones de aquiescencia. Y
como indicase que me prepararía para el regreso a mi tierra, dijo el
Cardenal: «De aquí a la noche, recogerás cuanto necesites llevar
contigo, libros y ropa; al amanecer saldrás de Ostia en un barco que se
da a la vela para la costa valenciana.» Dejome atónito esta conminación
que no admitía réplica, y con un gesto manifesté mi conformidad. Ya
sabía yo con quién me las había y cómo las gastaba el caballero. Al
despedirme, sólo me dijo: «En la política de tu país puedes abrirte
camino ancho, que allá tienes dos especies de hombres afortunados: los
tontos y los que se pasan de listos. Procura tú ser de los últimos.» La
sustanciosa frase me hizo sonreír, y besándole la mano, salí para
disponerme a cumplir mi sentencia. Ya no le vi más. Comí, llené de
libros una caja y un cofrecillo, de ropa un baúl, y me entregué al
mayordomo, encargado por Su Eminencia de ponerme en camino. La sentencia
se cumplió \emph{manu militari}, porque un agente de policía fue quien
me condujo a Ostia, a poco de anochecido, no soltándome de su férrea
mano hasta dejarme a bordo de la urca, libre y quito de todo gasto, bien
amonestado el patrón para que pusiese cien ojos en mí mientras el barco
no se diese a la vela.

¡Adiós, Italia; adiós, Roma, corazón del Paganismo, cabeza de la
Iglesia; adiós, Barberina, ara de mi primera ofrenda al tirano Dios! Así
como los antiguos ponían sus muertos en las constelaciones, yo quiero
darte luminosa eternidad en el firmamento\ldots{} Durante las noches de
mi largo viaje, he clavado de continuo mis ojos \emph{nelle vaghe stelle
dell'Orsa}.

\hypertarget{vi}{%
\chapter{VI}\label{vi}}

\textbf{Sigüenza}, \emph{Noviembre}.---Quedamos en que bauticé con el
nombre de \emph{Barberina} la estrella más brillante de la Osa Mayor, la
que los astrónomos, según creo, llaman \emph{Mizar}, y con esto puse
final punto a mi historia de Albano\ldots{}

Cosas y personas mueren, y la Historia es encadenamiento de vidas y
sucesos, imagen de la Naturaleza, que de los despojos de una existencia
hace otras, y se alimenta de la propia muerte. El continuo engendrar de
unos hechos en el vientre de otros es la Historia, hija del Ayer,
hermana del Hoy y madre del Mañana. Todos los hombres hacen historia
inédita; todo el que vive va creando ideales volúmenes que ni se
estampan ni aun se escriben. Digno será del lauro de Clío quien deje
marcado de alguna manera el rastro de su existencia al pasar por el
mundo, como los caracoles que van soltando sobre las piedras un hilo de
baba, con que imprimen su lento andar. Eso haré yo, caracol que aún
tengo largo camino por delante; y no me digan que la huella babosa que
dejo no merece ser mirada por los venideros. Respondo que todo ejemplo
de vida contiene enseñanza para los que vienen detrás, ya sea por fas,
ya por nefas, y útil es toda noticia del vivir de un hombre, ya ofrezca
en sus relatos la diafanidad de los hechos virtuosos, ya la negrura de
los feos y abominables, porque los primeros son imagen consoladora que
enseñe a los malos el rostro de la perfección para imitarlo; los otros,
imagen terrorífica que señale a los buenos las muecas y visajes del
pecado para que huyan de parecérsele. Habiendo aquí, como habrá
seguramente, enseñanza para diferentes gustos, no me arrepiento del
propósito de mis Memorias o Confesiones, y allá voy ahora con mi cuerpo
y mi juventud y mi buen ingenio por el anchuroso campo de la vida
española.

Ya es ocasión de que os hable de mi familia. Propietario de flacas
tierras en este término es, mi padre: poséelas mi madre de más valor en
Atienza; pero reunidos ambos patrimonios no bastaron para el sostén de
familia tan numerosa, por lo cual mi señor padre ha tenido que arrimarse
a la política y a la Iglesia, y tiempo ha que desempeña la Contaduría de
esta Subalterna, y es además habilitado del Clero. Gran administrador de
lo suyo y de lo ajeno ha sido siempre Don José García, y en su honradez,
que la opinión ha consagrado como artículo de fe, nunca puso el menor
celaje la malicia. La vida metódica y sin afanes, la paz de la
conciencia, el ejercicio saludable, le conservan entero y enjuto, sin
achaques de los que a su edad pocos se libran, aunque es algo aprensivo,
y tan friolero que anda de capa todo el año, de Agosto a Julio.

Mi madre es una santa, que hoy vive petrificada en los sentimientos
elementales y en las ideas de su juventud, creyendo a pie juntillas que
la inmovilidad es la forma visible de la razón. La palabra progreso
carece para ella de sentido, y si en modas no ha querido pasar del año
23, cuando vinieron con Angulema los chales de crespón, rayados, en lo
demás que atañe a la vida general no quiere entender de nada: ni discute
novedades, ni comprende constituciones, ni se cura de opinar conforme a
estas o las otras ideas, firme en su inquebrantable dogmatismo religioso
que a lo social y político extiende\ldots{} «Así lo encontramos y así lo
hemos de dejar,» es su fórmula, que a todo aplica, creyendo firmemente
que el mundo, por muchos tumbos que dé, vuelve siempre a lo que ella
vio, conoció y sintió en su florida mocedad. Completan el retrato la
dulzura y placidez de un rostro angelical, que aún parece más divino con
su copete de cabellos blancos, y el mirar confiado y sereno, reflejo de
un alma en que moran todas las virtudes cristianas y domésticas sin
sombra de maldad. Nueve hijos nacimos de esta ejemplar señora: vivimos
siete, con quienes harán conocimiento mis lectores, que algo hay en
ellos digno de la posteridad. A mí me tuvo mi madre en edad
extemporánea, cuando ya nadie esperaba fruto de ella, y por esto el más
joven de mis hermanos me lleva ocho años. Y como coincidieran con mi
tardío nacimiento una aurora boreal, un cometa, con más otros terrestres
acontecimientos, formidable crecida del Henares, y la aparición de una
espléndida luz que en las noches oscuras se paseaba por el tejado y
torres de la catedral, dio en creer la gente que aquellos inauditos
fenómenos anunciaban mi venida al mundo como prodigioso niño, llamado a
revolver toda la tierra. Mi madre se reía de estos disparates; pero
confiaba siempre en que su Benjamín no habría de ser un hombre vulgar.

Mi hermano Agustín, el primogénito, que ya cumplió los cuarenta, casó en
Madrid, y allá disfruta de un buen empleo arrimado a los hombres de la
\emph{moderación}. Mi hermano Vicente casó con una rica labradora de
Brihuega, viuda, y está hecho un bienaventurado patán, con cinco hijos y
labranza de doce pares de mulas; Gregorio, que estudió en Madrid la
carrera de abogado, también anda por allá, buscándose un acomodo en las
Sociedades mineras o de seguros; y Ramón, que es el más joven, no se ha
separado de mis padres, y disfruta un sueldecito en la Subalterna. De
mis hermanas, la mayor, Librada, que ahora tiene treinta y ocho años,
casó en Atienza con un primo mío, ganadero de buen acomodo y propietario
de dos molinos harineros y de una fábrica de curtidos; la segunda,
Catalina, que ya rebasa de los treinta, profesó en el convento de la
Concepción Francisca de Guadalajara, no recuerdo en qué fecha (sólo sé
que a mí me tenían aún vestidito de corto), y luego pasó a La Latina de
Madrid, donde ahora se encuentra. He aquí mi familia, mis sagrados
vínculos con la Humanidad.

Vivimos en la calle de Travesaña, angosta y feísima, pero muy
importante, porque en ella, según dicen aquí ampulosamente, \emph{está
todo el comercio}. La casa es de mi padre, tan antigua, que la tengo por
del tiempo de la guerra de los Turdetanos con Roma, cuando Catón el
Censor puso sitio a esta noble ciudad. A pesar de las restauraciones
hechas en ella, mi vivienda natal, en la cual no hay techo que no se
alcance con la mano, se pierde en la noche de los tiempos; y a pesar de
todo, como en ella vi la primera luz, paréceme la más cómoda y bonita
del mundo. En los bajos hay un alquilado para botica, la cual creo yo
que radica en aquel sitio desde que vino a España el primer boticario,
traído quizás por Protógenes, obispo fundador de nuestra diócesis. Ahora
la regenta un tal Cuevas, hombre muy entendido en su oficio, y es centro
de reunión o mentidero de cuantos en el pueblo discurren con más o menos
tino de la cosa pública.

Seis o siete sujetos calificados clavan allí sus posaderas en sendas
sillas toda la tarde y a prima noche, entre ellos mi padre; D. José
Verdún, coronel retirado; el juez Sr.~Zamorano, el canónigo de esta
Catedral D. Jacinto de Albentós, que entró aquí con Cabrera el año 36,
mandando una partida de escopeteros, bien ajeno entonces de que se le
recompensaría su hazaña con esta prebenda, y otros que no cito por no
transmitir vanos nombres a la posteridad. Cada cual lleva su periódico,
que lee o comenta: mi padre saca \emph{El Faro}, que goza opinión de
sensato; el canónigo desenvaina \emph{La Iglesia} y \emph{El Lábaro},
ambos de su cuerda; el coronel esgrime el \emph{Clamor}, órgano del
Progreso; otro tremola \emph{El Heraldo}, y Cuevas, en fin, enarbola
\emph{El Tío Carcoma}, satírico y desvergonzado, pues algo hay que dar
también a la risa y al honrado esparcimiento. Predomina en la botica el
tinte moderado, y contra una mayoría formidable luchan gallardamente los
dos únicos progresistas, el coronel y el boticario. De entre las
ruidosas peloteras que allí se arman salen airadas voces aclamando el
nombre sonoro del primate a quien cada cual debe su destino, y si el uno
pone sobre su cabeza a Bravo Murillo, el otro no deja que toquen ni al
pelo de la ropa de Seijas Lozano, de Pidal o de Bahamonde.

Allí me enteré de sucesos que ignoraba, y que, siendo ínfimos en la
esfera total del humano vivir, parecían grandes a los pobres enanos que
de ellos se ocupaban. Supe que habían caído los \emph{Puritanos}, y pues
yo no conocía másPuritanos que los de Bellini, pedí informes de tales
sujetos, sabiendo al fin que eran como una cofradía que dentro de la
\emph{moderada} comunidad alardeaba de pureza. Supe asimismo que el Rey
y la Reina andaban desavenidos, él haciendo solitaria vida en El Pardo,
ella en Madrid gozando de la cariñosa popularidad que había sabido
ganarse con su gracia y desenfado; y supe que los narvaístas andaban
locos por volver al Gobierno, y que los progresistas, alentados por
Bullwer, embajador inglés, hacían sus pinitos por colarse en Palacio.
Todo ello me importaba un bledo, como la caída del Ministerio Salamanca,
sucesor de los Puritanos, para dar entrada al temido y ensalzado D.
Ramón, que, según mi padre, es el único que entiende este complejo
tinglado del gobierno de España.

\textbf{Sigüenza}, \emph{25 de Noviembre}.---La comidilla de esta tarde
en la botica ha sido la reconciliación del Rey y la Reina. Vaya,
picaruelos, se os perdona, pero no volváis a poneros moños, que
perturban la tranquilidad de estos reinos. ¡Ay qué cosas han dicho los
tertulios, Santa Librada bendita! Que si costó más tr reconciliar a los
Reyes que casarlos\ldots{} que Serrano y Narváez se entendieron,
retirándose el primero a la Capitanía General de Granada, y cogiendo el
otro las riendas del poder\ldots{} que ello es juego de rabadanes, y
cambalache gitanesco\ldots{} ¡Dios mío, cómo ponen a Serrano mi
boticario y mi coronel por haber abdicado sin dejar el mango de la
sartén en manos progresistas! Los motes menos injuriosos que le cuelgan
son los de Judas y Don Opas. En cambio los otros échanle en cara el
abuso de su poder y su falta de discreción, tacto y delicadeza. Y yo le
digo al tal: «Si me viera en tu caso, haría las cosas mejor, y si no
pudiera escribir la Historia de España con la mano derecha, sabría
educar y adestrar mi mano zurda.»

\emph{27 de Noviembre}.---Esta tarde fui yo quien hizo el gasto
contándoles las magnificencias del rito en la Corte Papal,
describiéndoles con la facundia pintoresca que me permitían mis
conocimientos de las cosas romanas, los restos maravillosos del
Paganismo, el esplendor de San Pedro, de Santa María Mayor y de San Juan
de Letrán, el lujo y señorío de los cardenales, la opulencia artística
de los Museos, las mil estatuas, fuentes y obeliscos, y no necesito
decir que me oían con la boca abierta, suspensos de mi voz, y que
alabaron en coro mi feliz retentiva. Mayor éxito, si cabe, tuve cuando
de las cosas me llevó a las ideas el curso de mi fácil palabra, y les
expliqué la misión que Dios confiere al sucesor de San Pedro en la
segunda mitad del siglo que corre. \emph{Sursum corda}, y álcense unidos
el dogma cristiano y la libertad de los pueblos. Para redimir a Italia y
hacerla una y fuerte, se constituirá una federación bajo el patrocinio
del Soberano Pontificio, y un sabio Estatuto, en que se amalgamen y
compenetren los católicos principios con las reformas liberales, dará la
felicidad a los italianos, ofreciendo a las demás naciones europeas una
norma política, invariable y sagrada por traer la sanción de la Iglesia.

La polvareda que levantó en el farmacéutico senado de este novísimo
\emph{punto de vista}, como decía el juez, fue tremenda. Ya el señor
Zamorano tenía de ello noticia por haber leído párrafos de un artículo
de Balmes en la revista \emph{El Pensamiento de la Nación}. Para los
demás, el asunto era enteramente virgen. Cuevas y el coronel acogieron
la misión papal con benevolencia, afirmando que, pues las ideas de
Cristo eran francamente liberales, su Vicario en la tierra debía
pastorear a las naciones enarbolando en su báculo la bandera del
Progreso. Oír esto el canónigo y soltar la risa estúpida, grosera y
provocativa, fue todo uno. «¡Vaya, que será linda cosa un Papa
progresista!\ldots{} ¡La Iglesia dando el brazo a los hijos de la
Viuda!\ldots{} ¡Cristo entre masones\ldots{} ja, ja, ja\ldots{} y la
Santísima Virgen bordando banderas liberales como \emph{la} Mariana
Pineda!\ldots» Así desembuchaba sus salvajes burlas el sacerdote bizarro
que había entrado en Sigüenza once años antes, \emph{viribus et armis},
asolando el país y llevándose cincuenta mil reales como botín de guerra.
Y luego siguió: «¡Pero este Pepito, qué ruedas de molino se trae de Roma
para comulgarnos! Listo eres, hijo; pero no afiles tanto, que te vemos
la intención chancera. A Roma fuiste con ínfulas de sabio, que debía
tragarse el mundo, y nos vuelves acá con juegos de cubilete para
embaucar a estos pobres patanes. No nos creas más tontos de lo que
somos, y si vas a Madrid llévate allá los chismes de titiritero, y ponte
en las plazas a predicar toda esa monserga del Papa liberal y de la
Iglesia metida con los ateos. Aquí somos brutos, y no entendemos de
fililíes romanos ni de obeliscos, ni de cardenales que visten capita
corta y calzón a la rodilla; pero tenemos los sesos en su sitio, y
debajo del paño pardo guardamos el discernimiento español, que da quince
y raya a todo lo de extranjis.»

Respondí que no intentaba yo convencerle, porque él era como Dios le
había hecho, un clérigo de caballería, de los que defienden el dogma a
sablazo limpio. Contradiciéndole le puse tan desaforado y nervioso, que
no hacía más que morder el cigarro, echar salivazos en el corro, y dar
resoplidos como un flatulento a quien se le atraviesan en el buche los
gases. Intervino Cuevas en la contienda con sus opiniones emolientes, y
mi padre sacó todo el espíritu de conciliación que comúnmente usa,
asegurando que no hay que tomar a chacota mis ideas, pues vengo yo de
donde las guisan; que él no da ni quita liberalismo al Papa, pero que si
éste se liberaliza, habrá de ser siempre \emph{moderado}. Con esto y con
llegar la hora en que a cada cual le llamaban las sopas de ajo de la
cena, terminó la gran disputa. Era el desvaído rumor con que llegaba a
mi rústico pueblo la grave cuestión que entonces inquietaba a todos los
pensadores de Italia.

\emph{30 de Noviembre}.---He aquí que mi hermano Agustín, el gallito de
la familia, que desde Madrid dirige nuestros asuntos encaramado en su
posición política, comunicó por carta felices nuevas de su valimiento en
el Ministerio de la Gobernación, gracias al amparo que le dispensa el
nuevo Ministro D. Luis Sartorius. Extranjero en mi patria, era la
primera vez que oía yo tal nombre. Púsome en autos mi padre refiriéndome
que este Sartorius es un mozo andaluz tan agudo y con tal don de
simpatía que se lleva de calle a la gente joven. Ha brillado en el
periodismo; plumeando en las columnas de \emph{El Heraldo} se hizo
fácilmente un nombre, y\ldots{} periodista te vean mis ojos, que
ministro como tenerlo en la mano. Con sólo este breve informe me fue muy
simpático el tal Sartorius, y me entraron ganas de conocerle. Añadía mi
hermano en la carta que era llegada la ocasión de colocarme, toda vez
que no había para mí, después del desengaño de mi viaje a Italia, mejor
arrimo que el de la Administración Pública, sin perjuicio de aplicarme a
cualquier carrerita de las que en Madrid están abiertas para todo
muchacho que tenga alguna sal en el caletre. Quedó, pues, determinado
que para no perder tan dichosa coyuntura partiese yo a la Corte sin
dilación, llevándome toda la balumba de mis libros, los cuales habían de
ser mi mejor ornamento, y mi garantía más segura de que no se me
volvieran humo las esperanzas cortesanas.

\emph{1.º de Diciembre}.---Mi buena y santa madre, mientras estibaba con
delicado esmero en el baúl mi provisión de ropa, añadiendo no pocas
prendas, obra reciente de sus hábiles manos, me dio estos consejos que
así demostraban su cariño como su bendita inocencia: «Hijo mío, vas a un
pueblo muy grande, donde todo cuidado será poco para precaverte de los
peligros que te cercaran. Mas tú eres bueno, y tu alma paréceme que está
cerrada a piedra y barro para las malas tentaciones. Pero Madrid no es
Roma; en la ciudad que llaman Eterna, creo yo que no habrás visto más
que ejemplos de virtud y buenas costumbres, pues otra cosa no puede ser
viviendo entre tantísimo sacerdote y personas consagradas al servicio de
Dios. Madrid no es lo mismo, y los ejemplos que allí encuentres serán de
corrupción y escándalo, así en mujeres como en hombres. Te recomiendo y
encargo, hijo mío, que contra las innumerables incitaciones al pecado
que has de sentir, ver y escuchar, te fortalezcas con el temor de Dios y
con el recuerdo de las virtudes que habrás observado siempre en tu
familia. Y no insisto sobre punto tan delicado, pues, como dijo el otro,
«peor es meneallo\ldots» Yo confío en tu buen juicio y en la limpieza de
tus pensamientos.» Respondile muy conmovido que ya cavilaba yo en la
manera de sortear esos peligros, pues conocía bastante la sociedad para
distinguir el bien del mal; y que el refrán \emph{a Roma por todo}
quiere decir que allá van los hombres a enterarse de cuanto en lo humano
existe, y a doctorarse en la ciencia del mundo como en todas las
ciencias.

«Bien, hijo mío---dijo entonces mi madre con dulce conformidad.---Pero
hay otro peligro en el cual quiero que fijes tu atención, y es que en
Madrid abundan los envidiosos; y como tú despuntas por una capacidad y
sabidurías tan extraordinarias, no dejarán de caer sobre ti las malas
voluntades y peores lenguas para cerrarte los caminos de la gloria.
Mucho cuidado con esto, Pepe mío. No hagas alardes de ciencia, y tus
razones te acrediten más de modesto que de jactancioso, para que la
envidia tenga menos abrazaderas por donde cogerte\ldots{} Verdad que
casi está de más este consejo, pues de Roma has vuelto ocultando tu
ciencia más que ostentándola sin ton ni son, como hacías cuando fuiste.
Ya no te pones a recitar la retahíla de cánones y decretales; ya no
hablas de la \emph{Summa} de Santo Tomás ni de lo que escribieron
Aristóteles y Belarmino; ya no nos hablas en griego para mayor claridad;
y como no puedo pensar que sabes ahora menos, pienso que eres más
precavido y mejor guardador de tu ciencia, a fin de no dar resquemores a
la envidia y vivir en paz con tanto majadero.

---Algo hay de eso, señora madre---repliqué yo;---pero el principal
motivo de mi reserva del saber es que ahora sé mucho más que antes, y
cuanto más se sabe más se ignora, y más miedo tenemos de incurrir en el
error que de continuo nos acecha. Estudiando y aprendiendo he llegado a
medir la extensión de lo que aún no ha entrado en mi entendimiento, y
sabiendo cada día más voy hacia el término a que llegó el gran filósofo
que dijo: «Sólo sé que no sé nada.» Vea usted por qué parece que sé
menos sabiendo más. No compare usted, señora madre, la ciencia de un
niño con la de un hombre.»

Muy complacida de mi explicación, añadió este último consejo, dándome a
entender con su sonrisa que lo estimaba por muy práctico: «No te cuides,
hijo de mi alma, de lucirte entre los necios, cuyo aplauso para nada ha
de servirte, ni de enseñar a los ignorantes, ni de desasnar a los
torpes. Para divertir y admirar a cuatro gansos no has estado tú
quemándote las cejas desde que eras tamaño así. Toda Sigüenza sabe que
prontitud como la tuya para el conocimiento no se ha visto jamás, pues
aún estabas mamando y las primeras voces que dabas rompiendo a hablar
parecía que eran en latín\ldots{} Digo que te contengas, y que guardes
toda tu ciencia para las buenas ocasiones, desembuchándola como un
torrente cuando te halles en presencia de personas que sepan apreciarla,
pongo por caso, el señor De Sartorius, que dicen es tan sagaz y tan buen
catador de los talentos. Tengo por indudable que le deslumbrarás, y el
hombre no sabrá qué hacer contigo\ldots{} Para mí, y como si lo
estuviera viendo, es seguro que te pondrá en alguna de las grandes
bibliotecas que hay allá, o en la mismísima \emph{Gaceta}, para que
escribas todo lo que se ordena, manda y dispone, y hasta lo que la Reina
le dice a las Cortes, o a otros Reyes, o al mismo Papa.»

Encantado de su \emph{sancta simplicitas} y estimando ésta como un bien
muy grande, corona de las virtudes de mi madre en su patriarcal vejez,
corroboré aquellas ideas, y para fortalecer su inocencia hermosa me
fingí convencido de que Madrid y Sartorius me subirían a los cuernos de
la luna. Lloraba la pobrecita oyéndome, y yo, traspasado de pena, hice
mental juramento de conservar siempre a mi madre en aquel ideal ensueño
que aseguraba la felicidad de sus últimos días.

Partí aquella noche en el coche correo.

\hypertarget{vii}{%
\chapter{VII}\label{vii}}

\emph{14 de enero del 48}.---Carguen con Madrid y su vecindario todos
los demonios, y permita Dios que sobre esta villa, emporio de la
confusión y maestra de los enredos, caigan todas las plagas faraónicas y
algunas más. Rayos arroje el Cielo contra Madrid, pestes la tierra, y
queden pronto hechas polvo casas y personas. Hágase luego gigante el
enano Manzanares, para que con revueltas aguas borre hasta el último
vestigio de la capital, y quede el suelo de ésta convertido en inmenso
charco donde se establezca un pueblo de ranas que cante noche y día el
himno de la garrulería\ldots{}

No tuvo la Villa y Corte mis simpatías cuando en ella entré: pareciome
un hormiguero, sus calles, estrechas y sucias; su gente, bulliciosa,
entrometida y charlatana; los señores, ignorantes; el pueblo,
desmandado; las casas, feísimas y con olor de pobreza. Pero no proviene
de esto el odio que hoy siento, sino de positivas desdichas que en esta
Babilonia de cuarta clase me ocurrieron a poco de mi llegada. Dos
familias, la de mi hermano Agustín y la de mi hermano Gregorio, se
disputaron desde el primer momento la honra de albergarme, y ésta tiraba
de mí por un brazo, aquélla por otro, y en poco estuvo que me
descuartizaran. De una parte a otra iban mis baúles y maletas. Por la
mañana se decidía que mi casa fuera la de Gregorio; por la tarde venía
la mujer de Agustín, cargaba con mi ropa, y era forzoso meterlo todo a
puñados en los baúles. Tres días estuve de mazo en calabazo, comiendo en
una casa, cenando en otra, y a lo mejor me hallaba sin corbata, que se
había quedado allá, o me faltaba la levita, el sombrero, los
guantes\ldots{} Y cuando tras tantas fatigas, triunfante Gregorio, me vi
definitivamente instalado en casa de éste, ¡oh inmensa desventura!, eché
de ver que en los trasiegos de mi persona y de mis cosas entre una y
otra vivienda, se había perdido el manuscrito de mis \emph{Memorias},
todo lo que escribí desde Vinaroz a Sigüenza, mi vida en Italia\ldots{}
¿Hay mayor desdicha, ni más estúpido contratiempo? En vano lo he buscado
en las dos casas, preguntando a los aturdidos amos y a las cerriles
criadas. Nadie lo ha visto, nadie da razón de aquellas hojas en que
vertí la verdad de mis sentimientos y los secretos más graves\ldots{} Y
la idea de que mis apuntes hayan ido a parar a indiscretas manos me
vuelve loco. ¡Escriba usted confesiones con el fin de deleitar e
instruir a la juventud, ponga usted en ellas toda su alma, para que
caigan en manos de un zafio que haga de ellas chacota, o de una
maritornes que las emplee para encender la lumbre!

Aunque las diversas personas a quienes pregunté por mis papeles me
negaban con notoria ingenuidad haberlos visto, yo sospechaba de mi
cuñada, la mujer de Agustín, sin que pudiera decir en qué fundaba mi
sospecha, pues con la mayor serenidad me ayudaba a buscar el tesoro
perdido y lamentábase con desconsuelo verdadero o falso de la inutilidad
de mis investigaciones. Y hoy, cuando ya he perdido la esperanza de
recobrar mi tesoro, persisto en creer que ella lo guarda como un feliz
hallazgo, sin duda con la idea de variar los nombres de personas,
alterar algún incidente y publicarlo como novela de su invención. Porque
ha de saberse que mi cuñada Sofía es lo que llamamos politicómona, con
sus perfiles de literata, pues aunque alardea modestamente de no
escribir, presume de buen gusto y promulga juicios sentenciosos sobre
toda obra poética o narrativa que cae en sus manos. Comúnmente le sorbe
los sesos la batalladora política más que las pacíficas letras, y toda
la mañana la veis en su cuarto, con bata encarnada y una cofia en la
cabeza, devorando periódicos. ¡Y la casa sin barrer! ¡Y la señora no se
peina hasta media tarde!

Permitid que me ensañe en ella, pues le tengo odio y mala voluntad desde
que se me metió en la cabeza que es ladrona de mi manuscrito. Si mi
hermano la supera en discreción, ella le gana en edad; no tiene hijos,
pero sí un bigotillo con más lozano vello que el que a su sexo
corresponde. Por las mañanas, a la hora en que se halla en todo el furor
de su loco entretenimiento, las greñas se le salen por debajo de la
cofia, las uñas guardan todavía luto y las manos le huelen a tinta de
periódico; su gordura fofa se escapa por uno y otro lado, evadiéndose
del presidio de un destartalado corsé, cuyas ballenas no son más que un
andamiaje en ruinas.

Y también digo que a zalamera y engañadora no le gana nadie. Se precia
de quererme mucho y de tratarme como a un hijo. Me riñe con suavidad
cariñosa, si es menester, y me colma de elogios cuando a su parecer lo
merezco. Ella fue quien me notificó, a los ocho días de mi llegada, mi
nombramiento para una plaza en la \emph{Gaceta}. Éste era el \emph{veni
vidi vici}, y pocos podrían alabarse de tanta prontitud en el logro de
sus esperanzas. «Como ahora no se nos niega nada---me dijo azotándome la
cara con el número de \emph{El Clamor},---te hemos sacado ese destinito
con ocho mil reales, que no es mal principio de carrera. Luego se verá.
Me ha dicho Agustín que no tendrás nada que hacer en la \emph{Gaceta}, y
que te recomendará al director para que te perdone la asistencia a la
oficina los más de los días.» A ella y a mi hermano di las gracias,
añadiendo que no me conformo con tan denigrante ociosidad; que pediría
tr, si no me lo diesen, para devolver a la Nación en honrado servicio la
pitanza modesta que pone en mi boca. Y éste no fue ciertamente un vano
propósito, pues al tomar posesión de mi destino hube de protestar contra
la holganza, a lo que me contestó el director, hombre amabilísimo, y el
más zalamero, creo yo, que existe en el mundo: «Ya sé por su hermano que
es usted un prodigio de talento y erudición. Sería imperdonable que por
exigirle a usted la debida puntualidad en esta oficina, le apartara yo
de sus profundos estudios, privándole de consagrar las más de sus horas
a revolver libros y compulsar códices en las bibliotecas públicas.»
Creía, en conciencia, servir al Estado y al país declarándome vagabundo
erudito. Afortunadamente, la \emph{Gaceta} tenía personal de sobra, y
muchos iban allí a escribir comedias o a componer sonetos de pie
forzado. No insistí. ¡Delicioso jefe, fantástica oficina, sabrosa y
dulce nómina!

\emph{12 de Enero}.---En cuanto llegó a Sigüenza la noticia de mi
nombramiento, me escribió mi buena madre vertiendo en las cláusulas de
su epístola todo el cariño y la inocencia de su alma seráfica. Conocía
yo la magnitud de su alborozo por el temblor de su nada correcta
escritura. Todo había resultado tal como ella lo pensara: llegar yo un
viernes a Madrid, y al siguiente viernes, ¡pum!, el destino. Estas
brevas no caen más que para los hombres escogidos, en cuyas molleras ha
puesto el divino Criador toda su sal y pimienta\ldots{} Ya le había
contado a ella un pajarito que el Sr.~Sartorius me recibió poco menos
que con palio, y que yo me puse muy colorado con las alabanzas que tanto
el señor Ministro como los otros señores presentes habían echado por
aquellas bocas\ldots{} «Nadie me ha dicho esto---añadía con candorosa
persuasión,---pero lo sé. No puede haber sucedido de otro modo\ldots{}
Al mandarte a la \emph{Gaceta}, claro es que se ha fijado Su Excelencia
en que el desempeño de aquellas plazas exige las cabezas mejores, y allá
vas tú para poner en buena consonancia de frase todo lo del Procomún y
demás cosas que en tales hojas se estampan. Ya, ya saben esos señores a
qué árbol se arriman\ldots{} Te recomiendo, hijo mío, que no trabajes
demasiado. Ya estoy viendo que muchos de tus compañeros se aliviarán de
su faena recargando la tuya, fiados en que para tu entendimiento
grandísimo son juguete de chico las dificultades que a ellos les
agobian. No seas tan bonachón como sueles, ni tengas lástima de
holgazanes y torpes, que de esos se compone, según me dicen, la
turbamulta de las oficinas\ldots{} Por aquí se corre que has empezado a
escribir una magnífica obra sobre el \emph{Papado y}\ldots{} no sé qué
otras cosas, la cual no tendrá menos de quince tomos. Date prisa, no
vaya yo a morirme sin poder leer aunque sea sólo el título. Dime si es
verdad esto, y cuántos pliegos llevas escritos ya\ldots{} Adiós, Pepe
mío: cuídate mucho, abrígate, y que en esos trajines no se te olvide la
obligación de tus oraciones de mañana y noche. Siempre que puedas, oye
misa toditos los días. Yo no ceso de pedir al Señor que te ilumine y no
te deje de su mano. Recibe todos los pensamientos, el alma toda, y la
bendición de tu madre.---\emph{Librada.»}

En mi contestación, todas las ternezas me parecieron pocas, y poniendo
especial cuidado en no ajar sus ilusiones, le dije cuanto pudiera
conservarla en aquel sonrosado cielo donde su espíritu encontraba la
felicidad. Su vida era un dulce sueño. Antes muriera yo que despertarla.

\emph{28 de Enero}.---Dejo pasar muchas noches sin añadir una línea a la
Segunda Parte de mis Memorias, porque el desconsuelo de haber perdido la
Primera enfría mis entusiasmos de cronista y biógrafo, llenándome de
crueles dudas respecto al futuro destino de lo que escribo. ¿Quién me
asegura que mis confidencias salvarán el largo espacio que desde la hora
presente de mi vida se extiende hasta el reino oscuro de lo que llamamos
Posteridad, la vida y sucesos de los que aún no han nacido o están
todavía mamando? Para que estos renglones lleguen a su destino, hago
firme propósito de resguardarlos de curiosas miradas, y de trazarles un
caminito subterráneo por donde lleguen salvos a manos de un discreto
historiador del próximo siglo, que los acoja, los ordene y utilice de
ellos lo que bien le parezca.

Voy a contarte ahora, oh tú, mi futuro compilador, la vida y milagros de
mi hermano Gregorio, con quien vivo, y verás que, si por el talle y
rostro se distingue de mi hermano Agustín, mayor diferencia has de
encontrar entre uno y otro por los hábitos, gustos y ambiciones. El
primogénito es alto, airoso, elegante, de seductor trato, y cifra toda
su existencia presente y futura en la política; Gregorio es de mediana
estatura, achaparrado, de mal color, aunque de complexión recia; y
desengañado de la poca sustancia que se saca del trajín de la cosa
pública, adulando a poderosos sin ningún valor, o sentando plaza en el
bullicioso escuadrón de majaderos o malvados, ha querido llevar su
existencia por mejores rumbos.

Si diferentes son mis buenos hermanos, mayor desemejanza hay entre sus
respectivas mujeres, pues la de Gregorio no es politicastra, ni
bigotuda, ni gordinflona, sino muy bella y elegante, aunque, dicho sea
en secreto, un poquito retocada con sutiles afeites; sabe cumplir con su
casa y con la sociedad, gobernando muy bien la primera, y atendiendo a
las buenas relaciones, tan necesarias al género de vida que hoy lleva su
activo esposo. Si Sofía estanca a su marido en la charca pantanosa del
politiqueo, Segismunda dirige los pasos del suyo por caminos penosos y
difíciles, pero de sólido piso, y que pueden conducir a las zonas más
fructíferas de la existencia. A poco de tratar a esta segunda cuñada
mía, la tuve por mujer de entendimiento y de voluntad firme. En vez de
afligirse ante las necesidades, busca medios seguros de atender a ellas,
y mirando al porvenir tanto como al presente, fijo el pensamiento en sus
dos hijos y en los que aún pudiera tener, lanza valerosa y cruelmente a
su marido a un tr rudo, no de gabinete, sino de actividad febril,
mañana, tarde y noche, por las anchuras y estrecheces de Madrid.

Y ella por su lado y en su femenil esfera, trabaja también ayudando al
hombre, suavizándole asperezas o allanándole obstáculos. Viste bien,
recibe y paga visitas, aparenta holgada posición, no deja traslucir al
exterior las ascéticas economías que practica en su vivienda. Sonríe
cuando por dentro le andan terribles procesiones; en su pintado rostro
bonito se revela la mujer audaz y codiciosa que desea la buena vida para
sí y para los suyos, y sabiendo dónde lo hay, pone en juego todos los
recursos para traerlo a casa. Anda el pobre Gregorio todo el día como un
azacán, y a marcadas horas recibe mucha gente en su despacho: señores y
aun damas entran y salen sin cesar. Algunos días veo traslucir el
contento tras de la fatiga: los negocios van bien, y el hombre saca de
su cansancio nuevas fuerzas para seguir en tan terrible zarandeo. Ama
tiernamente a su mujer, que ha sido, según puedo colegir, su musa, su
Minerva, y ella también le ama, viéndole realizar con gallardo tesón
cuantos pensamientos ha sabido sugerirle.

Una tarde que estaba yo en el comedor jugando con los chiquillos,
Segismunda se lamentaba de que Gregorio no hubiera tenido aquel día un
rato libre para comer con sosiego. «Pero no hay más remedio---me
dijo,---y en este vértigo hemos de vivir hasta que llegue el descanso.
Seremos ricos, Pepe, tú lo has de ver, y nuestra posición desahogada la
debemos a nosotros mismos, es decir, Gregorio me la debe a mí\ldots{} Te
contaré: al año de casarme vi yo bien clarito que lo de la política es
una guasa indecente. Tres meses o seis con un mezquino sueldo, y luego
cesantías largas, angustiosas. «Esto no puede ser, me dije yo, y buen
tonto será el que lo sufra.» Gregorio no tenía las necesarias agallas
para lanzarse a los negocios; yo discurría por él; concluimos por
discurrir los dos, y al fin, el hombre se penetró bien de mis ideas,
y\ldots{} ¡a trabajar!\ldots{} ¡Qué comienzos tan penosos, hijo! Yo me
consumía, y Gregorio se despernaba. Pero al fin empezó la suerte a
ponerse a nuestro lado. Cuando él quería achicarse, yo me engrandecía,
haciendo papeles superiores a nuestros medios. Esto precisamente, la
figuración bien sostenida, nos acrecentaba la buena suerte, y al fin, ya
ves\ldots{} vamos prosperando, y ya no hay desaliento, sino esperanza:
los asuntos marchan a pedir de boca\ldots»

Aquí cerró el pico. Más poderosa mi discreción que mi curiosidad, no me
atreví a pedirle explicación clara de tan estupenda granjería.

\hypertarget{viii}{%
\chapter{VIII}\label{viii}}

\emph{6 de Febrero}.---Debo consagrar una de estas hojas, o un par de
ellas, a las reuniones que da cada martes y cada viernes mi cuñada
Sofía, bajo un régimen de confianza que excluye toda etiqueta enfadosa,
y que tiene por norma: amenidad, buen gusto y versificación. Suelen
concurrir los compañeros de oficina de mi hermano, con señora y niñas el
que las tiene. De hombres importantes no he visto a ninguno de los que
hoy dan que hacer a la fama. Ni Pastor Díaz, ni Donoso, ni González
Brabo, han pisado hasta hoy aquellos salones. De literatos he visto a
Rubí, sólo una noche, y varias a Navarrete, a Larrañaga, Antonio Flores,
Ariza y el gracioso Villergas. Con arte y rigores de corsé consigue
Sofía meter en cintura su deslavazado cuerpo y tener a raya las
exuberancias que por las mañanas hemos visto salidas de madre. Esto, y
el esmerado lavatorio de sus manos y pescuezo, y la compostura de la
carátula, con algún retoque de colorete y abundantes polvos, le dan
cierta dignidad majestuosa, que ella sabe realzar con su trato fino y
amable. Es justo decir que en sociedad tiene Sofía el tacto de olvidar
sus mañas de marisabidilla, evitando así la ridiculez que caería
seguramente sobre ella. Limítase a exigir de los jóvenes concurrentes
que lean versos tristes o declamen alguna llorona leyenda en prosa sobre
asunto caballeresco. Alaba desmesuradamente toda poesía de moco y baba,
o narración \emph{fatídica}, vaticinando a sus autores que eclipsarán
las glorias de Zorrilla o de Tula (con este familiar laconismo suele
designar a la señora Avellaneda), y luego toca la vez a las señoritas de
piano y solfa: rara es la noche que no tenemos \emph{Fantasía sobre
motivos}\ldots{} y Cavatina de \emph{Beatrice di Tenda} o de \emph{María
di Rudenz}.

Pero nada me divierte tanto a mí como el \emph{rincón de personas
serias} que dignifica la tertulia de mi hermano, cotarro que tiene su
asiento en un gabinetillo próximo a la sala, y del cual son figuras
principales D. José del Milagro, Ferrer del Río, D. Gabino Tejado, un
muchacho muy listo llamado Santa Ana, un viejo de la tanda del año 23,
llamado Muñoz; un D. Basilio Andrés de la Caña, a quien solemos llamar
el sesudo por la gravedad de sus juicios, y otros cuyos nombres no
recuerdo ahora. Ante aquel discreto senado quiere Agustín hacer gala de
suficiencia, y de hallarse muy al tanto de las ideas que en la
actualidad \emph{agitan a los pensadores europeos}, y como la \emph{idea
del día} es el liberalismo papal y la filosofía histórica de Gioberti y
de Balbo, viene a mí por las tardes, un poquito antes de comer, a
pedirme que en cuatro palotadas le dé una \emph{tintura} de estas sabias
doctrinas. No me cuesta tr complacerle. Llega la hora de la tertulia y
cae mi hermano en el corro de las personas serias como un pedrisco de
erudición. La lección que le di, y que lleva pegada con saliva, se
produce en deshilvanados conceptos que van saliendo en tropel de la
memoria, como avecillas prisioneras a las que se abre la jaula.

Sin dejar meter baza a nadie, Agustín desembucha: «Según expone Gioberti
en su \emph{Primato}, el redentor, el jefe, el príncipe de la nación
italiana, en la esfera del pensamiento, debe ser el Papa, cabeza visible
de la Iglesia católica\ldots» «Tengan ustedes por cierto que se formará
una confederación o liga de todos los pueblos y soberanos de Italia bajo
la presidencia del gran Pío IX.» Y recordando luego, no sin fatigas, lo
más intrincado y sutil de la lección, dice: «Contra dos \emph{elementos}
tiene que luchar Gioberti para implantar su \emph{tesis}. El primero es
el filosofismo que niega la revelación cristiana, y por eso veis que
truena contra Descartes y toda la tropa de filósofos alemanes. El
segundo \emph{elemento} enemigo es la intransigencia de los que niegan
la libertad, la ciencia y el progreso humano, y por eso le veis
revolverse contra los jesuitas. Entre la filosofía racionalista y la
intolerancia inquisitorial está el término prudente y conciliador en que
ha de fundarse la sana doctrina de la Libertad por el Pontificado,
término que \emph{nuestro autor} explana admirablemente en su
\emph{Introducción al estudio de la filosofía}\ldots» Y cuando a mi
hermano se le acaba la cuerda, van entrando en juego los demás, cada
cual con su tesis, y oímos opiniones muy originales. Ninguno me hace
tanta gracia como el \emph{sesudo}, que luce su marrullero escepticismo
cerrando las discusiones, al fin de la tertulia, con esta frase: «Y por
último, señores, que lo veamos, que lo veamos\ldots{} Yo voy más allá
que Santo Tomás, y digo: ¿Papa liberal? Cuando lo vea\ldots{} no lo
creeré.»

\emph{8 de Febrero}.---Palabras sueltas que al vuelo cogí de un
reservado coloquio entre Agustín y el \emph{sesudo}, algo más que oí en
el café de los \emph{Dos Amigos}, arrojaron súbita y esplendente luz
sobre la misteriosa granjería del hermano con quien vivo. Yo no sabía
nada, y todo de improviso lo supe, penetrando con mirada sintética en la
negra y pavorosa mina que explota Gregorio. Allí le ve mi pensamiento
arrancando en mal alumbradas cavernas el rico filón, bajo el látigo de
su esposa inflexible, y me tiemblan las carnes sintiéndome tan cerca de
la región de dolor y tinieblas. Consisten estos negocios en agenciar
préstamos con usura, sencillísimo y elemental arbitrio en todo país
pobre, donde se disputan la vida dos fuerzas negativas: la holganza y la
vanidad.

Al desilusionarse de la política, ocupose mi bendito hermano en la
colocación de pequeñas cantidades a rédito subidísimo; no tardó en tomar
el gusto a la carne, y poniéndose en relación con personas adineradas,
trabajó en los préstamos con tal celo, finura de trato, y con tan
escrupulosa puntualidad y honradez, relativa si se quiere, que en corto
tiempo tuvo una clientela formidable de necesitados, y otra no menos
fuerte de poderosos que sin quemarse las pestañas querían aumentar su
peculio\ldots{} En la red que Gregorio tiende han venido a caer
propietarios y labradores de poco seso, señoritos de familia ilustre,
que liquidan el pasado histórico entregando sus vestigios a una
mesocracia insaciable; industriales y mercaderes demasiado atrevidos
viudas y huérfanos predestinados a la mendicidad, y otros infelices a
quienes habría que calificar entre la necedad y la locura, o en ambas a
la vez.

A la fecha en que esto escribo, y trayendo a la memoria dichos y hechos
del que antes no comprendía y ahora sí, tengo por cierto que mi hermano,
sin dejar el manejo de capitales de incógnitos vampiros, opera también
con dinero propio, ganado en tres años de jugadas pingües. Ahora me
explico el sentido de un diálogo breve, a medias palabras, que oí a
Segismunda y Gregorio a los pocos días de mi llegada. Mi hermano, cuyo
corazón y buenos sentimientos no han acabado de atrofiarse, suele tener
reblandecimientos de la voluntad, remusguillos de compasión. Si le
dejara su mujer, alguna de sus víctimas le vería desmayar en el rigor
usurario. Pero así como el intrépido caudillo, al ver los primeros
síntomas de cobardía o desmoralización en el soldado, cae sobre él y a
empujones o sablazos le endereza, le vigoriza y le restituye a la
disciplina y al honor, del mismo modo la fiera Segismunda, de acerado
temple, cae sobre el tímido logrero, y con iracundas voces le pone ante
la vista el porvenir de sus niños nacidos y por nacer, engendrados y por
engendrar; le pinta con brillantes colores el desquiciamiento que puede
sobrevenir en la familia con tales flaquezas, y asienta dos grandes
principios: que la suprema caridad es la que sobre nosotros mismos
ejercemos, y que el verdadero prójimo es la familia; todos los demás
prójimos son fraudulentos, apócrifos y mixtificados.

\emph{Mutatis mutandis}, acabó diciéndole: «¿A qué vienen esas blanduras
sabiendo que nadie las tendría contigo si en igual caso te vieras? Bueno
que se dé una limosna o se haga un favor; pero siempre que no nos
perjudiquemos, porque si ahora te enterneces, todos querrán lo mismo, y
adiós tu negocio y nuestro porvenir. Ya te he dicho que el mundo que
habitamos es como un gran campo de batalla, en que todos luchan por el
pan, por la vida. Entre tantos que aquí combatimos, hay cobardes y
menguados de una parte, valientes de otra. Aquéllos se contentan con un
pedazo de pan: dignos de la victoria son los que van tras el pan de hoy
y el de mañana, tras el bienestar, las comodidades y todo lo que
constituye el decoro de nuestra existencia. El tesón ennoblece; la
sensiblería degrada. ¿Qué vale más, comer o ser comido? Hay que optar
entre estos dos papeles: el del cocinero, o el del pobre animal que cae
en la cazuela.» Esto dijo, y yo, sin variar a sus ideas ni un ápice,
condimento la frase para quitarle su bárbara crudeza.

Esta tarde se reprodujo la cuestión que acabo de referir, y los términos
del diálogo fueron aún más vivos. Oí desde mi cuarto el rumor de la
disputa, y pasado un gran rato, cuando me llamaron a la mesa, vi a
Segismunda que acababa de engalanarse para ir al teatro después de la
comida; contemplé su belleza y la expresión dura de su rostro, que
parecía verdaderamente trágico cuando mostraba de perfil sus líneas
helénicas; me pareció una euménide, o la propia cabeza de Medusa con
serpientes por cabellos.

\emph{16 de Febrero}.---Ved aquí la lista de mis amigos: Bruno Carrasco,
más joven que yo, aficionadísimo a Historia y Literatura, que encuentra
en mí una viviente enciclopedia, y no me deja a sol ni a sombra; Donato
Sarmiento, sobrino de mi cuñada Sofía, buen chico, ávido siempre de
pasatiempos y muy descuidado en el estudio; Pascual Uhagón, bilbaíno,
que estudia para ingeniero; un hermano de Segismunda, que se llama
Leovigildo (en esa familia todos llevan nombres germánicos); un Bringas,
un Pez, un Caballero, un Trujillo, un Arnáiz, un Moreno Isla, un
Trastamara, un Aransis, y otros que irán saliendo en el curso de estas
Memorias. Inmensamente vario es el jardín de mis amistades, y yo me
trato con muchachos de todas las jerarquías. La confusión de clases,
característica de España, tiene su principal fundamento en la
fraternidad de las generaciones tiernas. Amigos tengo de familias del
comercio, de familias vinculadas en la Administración Pública, de
familias aristocráticas. Ricos y pobres alternan conmigo, y tontos y
discretos; jóvenes estudiosos, de gran porvenir, y zotes que no sirven
para nada. En mis preferencias no brilla una lógica sana: es común verme
a distancia de los chicos aplicados, que como yo devoraron muchos
libros; algunos que presumen de sabios porque ganaron laureles en las
aulas, me son antipáticos; otros que hacen vida irregular y nocturna,
con gracioso desenfado de galanes de comedia, me atraen y seducen; los
hay reservaditos y juiciosos, aspirantes a empleados o catedráticos, que
a mí se me sientan en la boca del estómago. Me agradan más los que
brillan con luces naturales que los que las han adquirido en forzados
estudios, y ejercen mayor influencia sobre mí los aristócratas, a
quienes me gusta imitar, seducido por el no sé qué de sus modales y de
su conducta.

Gracias a Segismunda, que con toda su dureza de euménide es una gran
administradora y cuida de vestirme y engalanarme dignamente, poseo un
fraquecito azul con botón dorado, obra de un buen sastre, y todas las
demás prendas accesorias. Hablando con la Posteridad, que está tan lejos
y no puede ni contradecirme ni burlarse de mí, me atrevo a consignar que
mi figura es buena, que no desagrado al bello sexo\ldots{} que algunos
me toman por diplomático, y otros me lo llaman en broma sin saber que la
cuchufleta encierra un elogio. Mis amigos me cuentan maravillas de los
bailes de máscaras en Villahermosa, y yo, que no he podido asistir a
ninguno por carecer de ropa elegante, ahora que la tengo no veo las
santas horas de meter mis narices en aquella diversión, pues entiendo
que el juego de máscaras es cifra de la poesía social.

\hypertarget{ix}{%
\chapter{IX}\label{ix}}

\emph{18 de Febrero}.---¡Ay, ay, ay!\ldots{} Esto no es quejido
lastimero, sino el lenguaje del asombro y confusión que desde anoche
llevo en mi alma, sin que haya podido atenuarlos con el sueño matutino
ni con el paseo de la tarde. ¿Estoy demente, o qué me pasa? De veras
digo que si llevaran rótulo los capítulos o tratados de estas
Confesiones, el presente debía ser encabezado así: \emph{De la singular
y nunca imaginada aventura que le salió al caballero Fajardo en el baile
de Villahermosa con el inaudito encuentro de una misteriosa máscara.}

Las diez serían cuando Aransis, Donato, Bringas y yo subíamos por la
escalera de Villahermosa, que, con tener espacio y anchuras grandes, le
venía muy corta al tropel de personas, con careta o sin ella, que
intentábamos franquearla. En la puerta que abría paso a la antesala y
guardarropa, las apreturas de la multitud impaciente producían gemidos
de asfixia, alguna imprecación seca, y desperfectos de ropa,
principalmente en las delgadas telas de algunos disfraces. Entramos al
fin: nos despedimos de nuestros abrigos con cierta desconfianza de
volver a ponérnoslos, y nos lanzamos en el barullo ardiente, revoltijo
de mil colores, ondulaciones de cuerpos que parecen nadar en el líquido
tibio y perfumado de una redoma\ldots{} Tal fue mi aturdimiento en los
primeros instantes, que tardé en sentirme gozoso. Se me iba la cabeza,
no sabía para dónde volverme: mis amigos se reían de verme tan
provinciano, y me llevaban de un lado para otro, señalándome las
máscaras bonitas, las extravagantes, las que tenían cariz y sello
aristocráticos. A la media hora de navegar en aquel océano, ya recobré
la serenidad; había vencido el mareo: era un mediano navegante y me
permitía dirigir la palabra a las mascaritas que junto a mí pasaban, o
respondía sin cortedad a cuantas bromas venían dirigidas al grupo de mis
amigos, reforzado con otros que allí se nos unieron.

Fuera de Aransis y Trujillo, que iban a tiro hecho, en amorosa
connivencia con determinada mascarita, novia, \emph{compromiso}, o sabe
Dios qué, todos los de la partida íbamos a lo que saltara; algunos, con
esperanzas de fáciles conquistas, cegados por la vanidad; los más, sin
otro móvil que pasar agradablemente el tiempo, recogiendo una dulce
impresión, alguna hoja desprendida de la flor del misterio. Y era yo
ciertamente de los que menos podían esperar, porque escasos eran mis
conocimientos en la Corte, y además carecía del arranque necesario para
lanzarme en busca de la aventura si ésta no quería venir a mí. A medida
que pasaba el tiempo sin la emergencia de \emph{un encuentro fatal},
principio del enredo de amores (ilusión corriente en todo mozalbete),
iba creciendo mi timidez hasta llegar a una sosería que a mí mismo me
daba de cara. A las doce empecé a creer que me aburría; a las doce y
media confesé y reconocí mi soberano aburrimiento; y cerca de la una
declaré que aquel inmenso hastío era incompatible con mi dignidad de
caballero. Mi persona y mi facha, tan semejantes a las de un
diplomático, naufragaban en un mar de ridiculez. Esto pensaba al filo de
la una, y ya encariñándome iba con la resolución de marcharme, cuando el
Cielo, que hasta en los bailes de máscaras cuida de organizar las
tangencias de cosas y personas para que la armónica ley se cumpla, me
puso ante dos máscaras\ldots{} Mejor será decir que el Cielo las trajo
hacia mí, pues yo estaba quieto y como alelado, y ellas avanzaban con
paso vivo, cual si me hubieran buscado y en aquel punto me encontraran.

Vestían traje popular italiano las dos mujeres, desiguales en estatura y
empaque, y la más alta de ellas clavó en mí sus ojos\ldots{} Al través
de los agujeros de la careta los vi, negros, fulgurantes, y
temblé\ldots{} No me quedó gota de sangre en las venas cuando la
máscara, tocándome en el hombro, no por cierto con suavidad, me dijo:
«\emph{Sono Barberina}\ldots» Y sin darme tiempo a expresar mi
admiración, me soltó una retahíla de apóstrofes italianos, de los que
suelen usar las mujeres del pueblo en casos de pasión o de ira,
dejándome absolutamente confuso, lelo y turulato.

¡Barberina! En el barullo mental a que me llevó tan gran sorpresa, vi en
aquella mujer a la propia Barberina de Albano\ldots{} La seguí como un
loco. Su estatura y talle, el aire, el andamento eran los mismos\ldots{}
¡Pues digo, los ojos\ldots! La voz, aun con el disimulo de timbre que se
imponen las máscaras, también me parecía la suya\ldots{} En italiano le
hablé sin poder obtener más que la repetición de los dicterios, y
cruelísimas apreciaciones de mi conducta. \emph{Siete un povero
pazzo}\ldots{} \emph{Vi sprezzo}\ldots{} \emph{bruto villaco}\ldots{}
\emph{Avete obeditto al geloso pretaccio come un eunuco, come un
cane\ldots{} Non sapete aprezzare l'amore d'una donna
passionata}\ldots{}

Debió de durar poco en mí la persuasión de que me hablaba la Barberina
de mi albanesa historia; pero duró, sí, un espacio de tiempo que ahora
no puedo precisar, y mientras subsistió aquel engaño, díjele cuantas
necedades se me ocurrieron en son de disculpa, y mezclando las
explicaciones con el galanteo. Observé la autenticidad del traje de
\emph{ciociara}: podía jurar que era el mismo que Barberina guardaba en
su arca y que se puso un día para que yo lo viese. En cambio, el vestido
de la acompañante a la legua revelaba la confección casera y
carnavalesca, hecho con retazos mal cortados y peor zurcidos para una
noche. También advertí que la segunda máscara, con todas las trazas de
criada o confidente, no pronunciaba una sílaba en lengua italiana.
Barberina, que así tengo que llamarla, me permitió que la acompañase a
dar una vuelta por los salones; pero se negó resueltamente a bailar. Yo
no merecía, según ella, más que odio y desprecio. No me perdonaba mi
abandono, y había venido a España con el solo propósito de vengarse.
Fuérame, pues, preparando yo a recibir el golpe súbito de la más
terrible \emph{vendetta} que en dramas y novelas se ha visto.

Cuando a este punto de nuestro coloquio llegaba mi mascarita, ya se
había disipado en mi mente el primer engaño, y la claridad envolvía mi
aventura. Tan Barberina era ella como yo el Papa; era, sí, una dama o
mujer\ldots{} no, no, dama sin duda, a cuyas manos por ignorados
senderos había llegado el manuscrito de mis Confesiones de Italia. Lo
había leído y quería embromarme con gracia. Díjele así: «Máscara de mis
pecados, si no quieres que yo me vuelva loco, abandona la farsa
ingeniosa de hacerte pasar por Barberina, y dime cómo y cuándo llegó a
tu poder un manuscrito mío en que digo y cuento\ldots{} lo que sabes.
Dos fines aparecen en mi existencia desde esta noche feliz amarte con
pasión, con locura, con frenesí, y recobrar mis papeles. Te diré todo lo
que ordenan los poetas: eres ya el ángel de mis sueños; muéstrame tu faz
para que pueda adorar tu belleza.» Rompió en sonoras risas, diciéndome
en italiano inseguro que yo era tonto, y que así como soñaba con una
belleza que no existía, soñaba también con un libro que no había sabido
escribir.

«Ya es inútil que sostengas la farsa---le dije.---Ni tú eres romana, ni
sabes de aquella lengua más que algunos dicharachos comunes. Tu linda
boca te ha vendido dejando escapar frases en el castellano más correcto.
Seamos amigos. ¿No quieres mi amor? Pues recíbelo como amistad, y
descúbrete, o, sin descubrirte, dime dónde y en qué lugar debo recoger
mi manuscrito.»

Riendo con más gana, repitió dos o tres veces la frase morbosa del buen
D. Matías, que me hizo un efecto terrible pronunciada en medio de la
febril alegría del baile: \emph{Ho perso il boccino}. Por fin, reducirla
pude a que me hablara en castellano. Y oí de sus labios estas palabras
dulces, afectuosas, como reprimenda de hermana mayor: «Eres un chiquillo
inocente, y corres en el mundo inmenso peligro si no caes en manos
piadosas que te guíen.»

---«Pues sean esas manos las tuyas, máscara\ldots{} ¿Quieres que te
llame \emph{hurí}? Te llamaré mi ideal, mi sueño o el oriente de mi
dicha.

---No empalagues con merengues poéticos.

---¿Te gusta la prosa?

---Sí, la prosa correcta y clara.

---Pues te amo, ¿es esto claro? Quítate la careta, y a renglón
seguido\ldots{} te propondré casarme contigo.

---¡Ay, qué prisita! ¿Y si yo no aceptara?

---Al romper el alba me pegaría un tiro.

---Eso no.

---¿Para qué quiero vivir?

---Pues para seguir escribiendo las \emph{Confesiones}.

---Dame la \emph{Primera Parte}.

---No la tengo.

---Eso no es verdad.

---Cortada en pedacitos, fue convertida en papel para tirabuzones.

---Pues dame los papeles con pelo y todo, que si es tuyo me parecerá
cabello de ángel.

---No, que empalaga\ldots{}

---Tú tienes las \emph{Confesiones}: devuélvemelas.

---No me da la gana.

---Te recompensaré poniéndote a ti en la \emph{Segunda Parte}.

---Si tú me conocieras, yo te tendría miedo; pero soy un arcano para ti.
Escribe todo lo que quieras de una máscara vestida de \emph{ciociara}.

---Tú no eres italiana, pero has estado en Roma. Tú eres amiga de mi
cuñada Sofía, de mi cuñada Segismunda.

---Sonsaca, sonsaca, pobre tonto.

---Tú eres persona principal\ldots{}

---Principal con entresuelo: de modo que soy más alta de lo que creías.

---Yo he de conocerte. Revolveré la tierra por descubrirte, porque, ya
lo habrás conocido, ardo\ldots{} ardo en amoroso incendio.

---No veo más que el humo.

---Yo me muero si ese maldito antifaz continúa ocultándome el sol.

---Más vale así: podría deslumbrarte.

---No veo más que tus ojos\ldots{} Déjame que los mire: en el fondo de
esas pupilas negras como la noche, veo mi escritura, veo mis
\emph{Confesiones}. Tú me has leído. Divinos ojos, a vosotros pertenecí
por algunos instantes, y mientras me leías, yo me paseaba por el alma
que está tras de vosotros.

---Entraste en el alma como el burro que se mete en un jardín\ldots{}

---Me comí una flor\ldots{} ¿No lo habías notado? ¿No echaste de menos
alguna?

---Donde hay tantas, ¿qué significa una de menos?\ldots{} Dime: ¿qué
estimas por lo mejor de tus Memorias?

---Lo de\ldots{} \emph{Juan Jacobo fu il libro e chi lo scrisse}.

---No: lo más bonito es aquel pasaje tierno\ldots{} cuando el Cardenal
te manda embarcar, escoltadito por la policía.

---La policía me empujó hacia España, y una mujer enmascarada me atraía,
como el imán al acero.

---El imán era yo. Benditos seamos los imanes.

---Ya que no enseñas tu rostro ni me das el manuscrito, ¿querrás decirme
la primera letra de tu nombre?

---Es la I\ldots{} Imán.

---Puede que en broma me hayas dicho la verdad. ¿De veras empieza con
\emph{i?}

---Pero ahora me acuerdo: es con \emph{h\ldots{}} Hi\ldots{}

---No será Higinia.

---Hombre, ¿y porque fuera Higinia habías de perder la ilusión?

---Ya que no quieres enseñarme toda la cara, descúbreme siquiera un poco
de la barbilla\ldots{} el piquito de la boca\ldots{} Me está diciendo el
corazón que debajito de él tienes un lunar.

---El lunar no está sino encimita. Pero no lo verás, a fe de Higinia.

---Pero ¿de veras es tu nombre?

---Sí, hombre; y para más señas te diré que soy de Puentedeume.

---Esa no cuela: tu acento es de purísima tierra castellana.

---Porque me he criado en Tordehúmos.

---¡Ay qué mentira más gorda!\ldots{} En fin, he llegado al último
paroxismo de la desesperación. Sultana, yo te amo.

---Abencerraje, tu frenesí no llega a embriagarme. No toques más la
guzla, y lárgate de mi lado.

---Serás responsable de mi fin tétrico\ldots{} Dame siquiera una
esperanza. ¿Vendrás al baile del Domingo?

---Vendré con otro disfraz para que no me conozcas.

---¿Te veré en sociedad; sabré de ti? ¿No quedará pendiente esta noche
un hilo, por donde yo pueda\ldots?»

Ya iba a contestarme cuando avanzó hacia nuestro grupo una máscara
procerosa, cubierta más que vestida con dominó negro guarnecido de picos
verdes, horrorosa estantigua que hubo de parecerme funcionario de la
Inquisición o del mismo Infierno cuando la vi gesticular ante mi
desconocida y hablarle en tono displicente como de superior a inferior.
«Sí, sí---dijo la que llamaré Barberina mientras no pueda darle otro
nombre:---son las dos. ¡Qué tarde, Dios mío! Vámonos.» Y el inexorable
tagarote, que con descompuestos modos cortaba rudamente la interesante
ansiedad de mi aventura, se permitió apartarme con un gesto poco urbano.
Por los ademanes le entendía yo más que por las voces, pues hablaba una
endemoniada lengua de mí jamás comprendida. ¡Vascuence, Señor! La
confusión de idiomas dominante en mi aventura, bien pudo hacerme creer
que estaba en la torre de Babel. Y otra cosa me confundía más. Aquel
desaforado vestiglo que me arrebataba mi ilusión, ¿era criado,
mayordomo, amigo o qué demonios era? Obedeciéronle las mascaritas, y sin
volver la cabeza para mirarme, rompieron por entre la muchedumbre.

«¿Qué haces que no la sigues, tonto?»---me dijo Arnáiz, que en la última
parte de mi aventura había cortejado a la máscara chica. Y viéndome como
lelo, me sacudió con fuerte brazo. Estalló mi voluntad, lanzándome por
el camino que ellas seguían, y me abrí paso a codazo limpio, guiado por
la cabezota del vestiglo, que entre mil cabezas fluctuaba de salón en
salón. «Se nos escaparán ---dijo Arnáiz,---porque ellas no se detienen
en el guardarropa y nosotros sí. Tendrán criados en la escalera que les
darán los abrigos.»

---Salgamos sin abrigos---dije sin apartar mi vista de la cabezota, que
más parecía boya arrastrada por la resaca. Así lo hicimos, y al
precipitarnos por la escalera, observamos que otro mascarón ponía sendos
chales de cachemira sobre los hombros de nuestras damas, pues por tales
sin ninguna duda las teníamos ya. En la calle nos escurrimos en su
seguimiento, mientras iban en buca del coche, situado muy lejos, más
allá del portal de Medinaceli. Las vimos subir a un carruaje anticuado,
alquilón, de los más feos que nos han transmitido las generaciones
pasadas, del cual tiraban dos caballotes angulosos, pero de bastante
poder, que arrancaron veloces desempedrando el suelo por la calle del
Prado . Buscó Arnáiz un simón con idea de salir dando caza al armatoste;
mas no lo halló tan pronto como fuera preciso. Emprender a la carrera la
cacería habría sido inútil locura\ldots{} Y en esto, un polizonte se
cuadró delante de nosotros y en tono socarrón nos dijo: «Caballeritos,
vuélvanse al baile, y busquen allí otro enredo, que lo que es éste se
les ha destripado.» Pareciole a Arnáiz juicioso el consejo; a mí no, y
en poco estuvo que lo contestara con un par de mojicones.

Volvimos a Villahermosa, donde vi que la diversión llegaba al período
vertiginoso y candente: sentime agobiado por infinita tristeza, sin
voluntad, sin resolución, y me entregué a un loco devaneo, arrastrado
por mis alegres amigotes. Bailé, di vueltas como una peonza, perdí toda
formalidad y discreción, salieron de mi boca cuantas garrulerías vanas
pueden imaginarse. Para remate de fiesta, caímos a la hora última en el
\emph{ambigú}, y allí, prestándome a la imitación de lo que veía, metí
en mi cuerpo todo el \emph{champagne} que me ofrecieron, y me puse tan
perdido, que renació en mí la erudición que con el tráfago vital se
había ido desvaneciendo. Improvisé versos sáficos, imitando los de
Anacreonte; canté el Amor en prosa poética, y el vino y los placeres;
hablé en latín y en griego, y recité casi todo el \emph{Ultimo canto di
Saffo}, de Leopardi:\emph{Placida notte, e verecondo raggio---della
cadente luna}\ldots{} añadiendo en diversidad de lenguas extravagantes
desatinos, que mis amigos aplaudían a rabiar. De día entré en mi casa,
más triste que loco, y más enfermo que borracho.

\hypertarget{x}{%
\chapter{X}\label{x}}

\emph{26 de Febrero}.---Señora Posteridad, mi amiga y dueño: La
turbación de mi ánimo en estos días me ha privado del gusto de escribir
a usted. Ya comprenderá que no estoy para bromas después de la que me
dio la máscara de negros ojos, y que bastante ocupación han tenido mis
sesos devanándose a todas las horas para desentrañar aquel arcano, sin
haber logrado hasta la presente la claridad que ansío\ldots{} Más de una
vez he preguntado a mi cuñada Sofía si conoce a una dama llamada
Higinia, y a todo el ardid capcioso con que trato de descubrir su
pensamiento contesta con risotadas. La única adquisición que he podido
hacer en esta mi contienda con lo desconocido es la certidumbre de que
fue Sofía quien me robó mis \emph{Confesiones}. No me lo ha dicho
claramente; pero su familiar risa picaresca me declara el delito, al
cual parece dar el carácter de travesura inocente.

Hoy está fuera de sí con las noticias que corren de una revolución en
Francia. Cree Sofía que si las terribles nuevas se confirman, tendremos
aquí grave trapatiesta, y cuando le digo yo que de ello me holgara
mucho, se pone hecha un basilisco. «¿Te parece bien que ahora, por
seguir aquí el ejemplo de Francia, se nos cuelen en el poder los
progresistas, que después de tantos años de oposición deben de traer
hambre atrasada? Pues como levanten la cabeza Olózaga y Don \emph{Juan y
Medio}, Sancho y Madoz, con toda la taifa nueva de los
\emph{democratistas}, ya podemos recoger los bártulos\ldots{} Bien dije
yo que con este idilio del \emph{Papa liberal} se habían trastornado los
caletres de los políticos españoles. Vino Espartero de Inglaterra, y no
supo D. Ramón qué hacer para festejarle. A Olózaga le levantan el
destierro, y hasta le dan indulto al pícaro Godoy. ¿Qué resulta de estas
blanduras? Que los progresistas no agradecen el favor, y que al
calorcillo de tanta liberalidad la gusanera carlista o montemolinista
revive, y ya tenemos a nuestras tropas dando caza a los Tristanys, a
Tintoret de Igualada y al Tuerto de la Ratera\ldots{} Todo ello es por
haber tomado en serio ese poema católico y político del Papado al frente
del liberalismo, y de la unidad de Italia, que en rigor nos importa un
comino\ldots{} Pues ahora, si se confirma el topetazo que anuncian de
\emph{allende el Pirineo}, no sé por dónde van a salir nuestros
\emph{hombres públicos}\ldots{} Las últimas noticias comunicadas por las
torres telegráficas son que en París está el trono patas , y que Luis
Felipe salió con las manos en la cabeza\ldots»

Respondí yo, para hacerla rabiar, que de todo lo que en Francia sucedía
me alegraba, y que vería con gusto que, no ya los progresistas, sino los
\emph{demócratas} (que así se dice y no \emph{democratistas}), cogieran
la sartén por el mango; que me quitaran mi destino, y que a los vagos
como Agustín y otros les dejaran cesantes; que se decretara el
socialismo y el comunismo y los falansterios, con lo cual quedábamos
todos de un color, en el seno de la más perfecta igualdad. Quimerista y
disputadora por naturaleza, tomaba muy a pechos mis desahogos, y
queriendo defender la razón, el justo medio y el buen sentido,
despotricaba más disparatadamente que yo. Sorprendidos por mi hemano en
agria querella, suspendimos las hostilidades para oír las nuevas que
traía. Éstas no podían ser peores. En Francia se había proclamado la
República o andaban en ello. «Pero ¿qué hace Odilon Barrot?---decía mi
cuñada, roja como un tomate.---Si nos saldrá también \emph{grilla}, como
Guizot y \emph{ese} Thiers\ldots» La cara de Agustín revelaba una gran
consternación. ¿Qué iba a pasar aquí? Ya estaba viendo el tricornio del
Duque entrando por la Puerta de Alcalá. ¡Y que vendría el hombre con
pocas ganas de gresca!\ldots{} Sería forzoso apechugar de nuevo con la
Milicia Nacional y soportar los desmanes de \emph{las turbas}. «Ya en
Francia no se dice las \emph{turbas---}indicó Sofía,---sino las
\emph{masas}, nombre nuevo del populacho, y me parece que también por
acá vamos a tener \emph{masas}, que es lo único que nos faltaba.»
Dejéles comentando a su antojo los sucesos de París, y a mi casa me
vine, donde encontré una carta de mi madre, que abrí con presteza para
saborear el consuelo que siempre me trae el vivir ilusorio de la santa
señora. Ved aquí el sabroso mentir de las estrellas:

«Pero ¿tan engolfado estás, hijo mío, en tu ciencia y en la lectura de
impresos y manuscritos, y tan metido en el trajín de archivos y
bibliotecas, que no te queda un rato para llegarte al convento de la
Concepción Francisca, por otro nombre \emph{La Latina}, y visitar a tu
querida hermana, a quien no has visto desde que estás en la Corte? Aquí
supiste que mi reverenda hija fue a las \emph{Concepcionistas Calzadas}
de Talavera acompañando a una señora monja enferma, de notable virtud y
santidad, a quien recetaron los doctores aquellos aires. De Talavera
pasó Catalina a Torrelaguna, siempre en compañía de la venerada
religiosa, y ya la fienes en Madrid. No te disculpes con que no lo
sabías, pues tu hermana me escribe que con mucho interés preguntó a
Sofía por ti cuando ésta la visitó en el convento. Discúlpate con tus
atracones de lectura, y te perdonaré, sí, te perdono, con tal que al
recibo de ésta des de mano a los cánones y a las historias de romanos y
griegos, y te vayas corriendo a ver a Catalina.

»El pajarito que me cuenta tus pasos me dice que renegaste de toda
diversión en los malditos carnavales, huyendo del barullo de las
llamadas máscaras, y prefiriendo el goce de tus libros a ese torbellino
indecente de bailes y comilonas. Pequen otros todo lo que quieran
emborrachándose y dando bromas, y consérvate tú en tu celestial pureza,
dedicado a las ciencias de Dios. Sé también que si algún tiempo has
robado a los estudios, ha sido para consagrarlo a devotos ejercicios en
la iglesia\ldots{} Lo sé, y no venga tu modestia diciéndome que
no\ldots{} No se me cocerá el pan hasta que me digas que has visto a tu
hermana y me cuentes lo que habéis hablado, que ello ha de ser muy
sustancioso y tocante a las cosas de tejas . Porque por estas bajezas,
hijo mío, todo es vanidad, mentira, y afanes inútiles que no conducen
más que a la perdición. Me imagino que tratará de encaminarte por los
senderos que pisabas cuando eras niño. Vuelve, vuelve, Pepe querido, a
esos divinos campos. Haz caso de tu hermana que ya está en salvo, y
quiere verte salvado con ella\ldots{} Me figuro también que por Catalina
trabarás conocimiento con esa bendita monja, su compañera, de quien la
fama refiere tales maravillas que hasta se susurra ya que hace alguno
que otro milagro. Los hará muy sonados cuando menos se piense. Dará
gusto oíros a los dos platicando de cosas divinas, pues la santidad y la
ciencia frente a frente ya tendrán qué decirse. ¡Lástima que no pudieran
escribirse por máquina o cosa tal vuestros coloquios, que ello habría de
ser de gran enseñanza y edificación!

»Quedamos en que cuando recibas ésta, cogerás al instante tu sombrero y
te irás al convento de La Latina, que entiendo está en la calle de
Toledo, bajando a mano derecha, y en la portería llamarás, preguntando
por Sor Catalina de los Desposorios, la cual debe de estar consumida por
verte, y pedirá todos los días al Señor que encamine tus pasos hacia la
Concepción Francisca. Espero tu carta dándome cuenta de la visita, y
contenta de tu virtud, gozosa de tu piedad y aplicación constante, te
manda su bendición tu amante madre.---\emph{Librada.»}

Mi primer impulso, bebiéndome las lágrimas que la carta me hizo
derramar, fue coger la pluma, y responder a su soñador optimismo con el
desengaño de la verdad\ldots{} Hubiera yo dicho, vaciando de golpe mi
oprimida conciencia: «Señora madre, para mí no hay ya más cánones que
los ojos negros de la misteriosa hembra que en el baile se me apareció
dándome el nombre de Barberina; para mí no hay más estudio que el
intrincado enigma de esa mujer, su calidad, su nombre; saber si es tan
hermosa como al través del antifaz la imagina mi amor, y si la lectura
de mis \emph{Confesiones} de Italia despertó en ella un sentimiento que
me haría más dichoso que poseer los tesoros de ciencia encerrados en
todas las Enciclopedias y Antologías del mundo\ldots{} No piense mi
madre que me seducen las muertas bellezas de los libros, goces
ilusorios, que si fueron quizás verdaderos para quien los escribió, no
lo son para quien los lee; busco mi ciencia en las páginas vivas, y en
los textos que respiran y ríen y lloran, compilados en la gran
biblioteca humana. Soy joven: no me pide el cuerpo una decrepitud
prematura averiguando cosas que ya están averiguadas, o consumiéndome en
medir y pesar la vida que otros tuvieron. Anhelo vivir\ldots» Pero si yo
le dijera esto\ldots{} ¡pobre madre! No: malo es engañarla; peor sería
darle muerte.

\emph{27 de febrero}.---Recogidos y alineados mis recuerdos, puestas en
orden todas las piezas de la máquina cerebral para que no resulte
desconcertado el grave relato de hoy, empiezo\ldots{} Pero ¿por dónde
empiezo? Naturalmente, por mi entrada en La Latina, por las palabras que
dije a la tornera, la cual me mandó esperar un ratito\ldots{} Yo no
quitaba mis ojos de la reja, esperando a cada instante la conmovedora
aparición del rostro de mi querida hermana. ¿La reconocería yo después
de tanto tiempo? Habíanme dicho que de su singular belleza apenas
quedaban reflejos pálidos. ¡Pobre Catalina! Yo niño y ella mujercita,
había sido para mí la hermana predilecta, algo como una madre chica: yo
la adoraba, y ella cifraba en mí todos sus amores. Tenía nueve años más
que yo; me llevó mucho tiempo en brazos, y le serví de muñeca para sus
juegos. Nunca he sabido por qué abrazó la vida religiosa. Ello fue
determinación repentina, en pocos días tramitada. Más de una vez
pregunté a mi madre por qué era monja Catalina, y me respondía lacónica
y evasivamente que porque Dios así lo dispuso\ldots{} En aquel plantón
que precedió a la visita, mi memoria refrescaba los días pasados en que
mi hermana vivía con nosotros en Sigüenza y en Atienza; después hice
mental cálculo de su edad: debía de estar ya en los treinta y dos
cumplidos.

El súbito descorrer de la cortina me sacó de mis remembranzas; temblé,
vi el rostro de mi hermana desvaído en las tinieblas como la imagen de
un ensueño. «Gracias a Dios, hombre---fue lo primero que dijo;---gracias
a Dios que te dejas ver.» Se sentó junto a la reja, y llevándose a los
ojos sus blancos dedos lloró un ratito. Dile las necesarias disculpas de
mi tardanza, con no poca turbación, porque también a mí se me saltaron
las lágrimas y no sabía qué decir. Serenados ambos, y hechos mis ojos a
la oscuridad, observé a Catalina y no me parecía tan decaída su belleza
como me habían dicho. Fuera del descuido de la dentadura, que afeaba un
tanto la boca, no hallé su rostro descompuesto: su blancura era como el
mármol, y sus negros ojos conservaban el encanto de otros tiempos. La
voz se había hecho un poco gangosa y desapacible, por el hábito de
hablar compungidamente. «Ya sé---dijo contestando a mis disculpas,---que
te has lanzado al vivir como las mariposas a la luz; pero esto no hay
que decírselo a madre, porque se moriría de pena. Como hermana mayor y
como religiosa, yo tengo que advertirte los peligros que corres, Pepe.
No trataré de renovar en ti una vocación que ya me parece ha volado para
siempre; pero he de procurar que en ese remolino del mundo te trastornes
lo menos posible, y que no te apartes demasiado de la ley de Dios\ldots»

Le di gracias por su benevolencia, y luego prosiguió así: «Pero, hijo,
has dado un cambiazo tan grande en tu carácter, que no conozco en ti al
muchacho formalito, apocado y estudioso que dejé en casa cuando Dios me
llamó a esta vida. Roma, que para otros es medicina y confortamiento del
espíritu, para el tuyo ha debido de ser veneno, pues allí, como las
serpientes mudan la piel, soltaste todas las virtudes y te vestiste de
todos los vicios\ldots{} Y sabe Dios hasta dónde llegaste, hermano, que
el pajarito que a mí me cuenta todo, no me habrá dicho sino una parte de
la verdad.

---¿Qué te ha contado ese pícaro?---pregunté viéndola venir;---porque ya
no dudo de que andan por ahí gorriones que van de oreja en oreja
desacreditándome\ldots{}

---No, lo que es el mío no me engaña. Pienso que se habrá quedado corto
en contarme tu libertinaje de Roma. No quiero decirte los azotes que yo
te hubiera dado si te cojo en el momento de descolgarte, con aquel par
de mequetrefes, de los techos de San Apolinar\ldots{} Pues ¿qué te
habría hecho si te veo entrar en la infernal caverna masónica?

---Querida hermana, tú has leído mis \emph{Confesiones}\ldots{}

---Yo no he leído nada. ¿Necesito yo leer para enterarme? Aquí sabemos
todos los pasos buenos y malos de las personas que nos interesan.

---Entonces\ldots{} ¿Sofía te ha contado\ldots?

---Yo estoy aquí para interrogar, no para que me interrogues tú, mocoso,
a quien he saltado en mis brazos, a quien he dado la papilla, y luego
las sopitas\ldots»

Y pegando su rostro a la reja interior, y ordenándome que a la de fuera
me aproximase, me miró bien, y orgullosa y risueña me dijo: «Pues estás
guapo de veras. En figura el cambiazo no es menos notorio que en lo
demás\ldots{} Bueno: siéntate y escúchame con atención. No quiero hablar
del grandísimo pecado\ldots{} ¡Jesús, Jesús! Fue tan horrible que mi
boca no puede mentarlo. Pero ya tu conciencia sabe a qué pecado me
refiero, al horrendo delito que no deberías recordar sin que se te
cayera la cara de vergüenza.

---Se me cae la cara, sí\ldots{} pero ¿cuándo y cómo has leído\ldots?

---Cállate la boca, y déjame seguir. Digo que no quiero hablar de ese
pecado, porque repugna a mi conciencia, porque mancha mi boca\ldots{}
Pero de su conocimiento y del horror que me causa partiré para la grande
obra de tu redención; porque yo quiero redimirte, hermano querido,
apartándote de los peligros que corre una naturaleza ya dañada y que se
dañará más cada día; quiero formarte una vida nueva, como jaula segura
de la que no puedas escaparte\ldots{} ¿no me entiendes?

---Querida hermana, si pretendes llevarme a una vida para la que no
siento inclinación, desde hoy te digo que pierdes el tiempo.

---No es vida eclesiástica la que te propongo, pues ni tú la mereces ya,
ni la divinidad de esa vida corresponde a tu naturaleza impura. Quiero
echar cadenas a tu libertad para que no acabes de perderte; pero la
esclavitud que te preparo no es la esclavitud de perfección, aunque
también has de ver en ella carácter sagrado.

---Por Dios, que ya te voy entendiendo, hermana. Has de decirme qué
pajarito te ha traído esa idea.

---¡Ah!, un pajarito precioso\ldots{}

---Ruiseñor tal vez.

---No; su belleza no consiste en el canto, sino en el color de sus
plumas: es todo encarnado.

---Será entonces Cardenal.

---Justo\ldots{} y tú le conoces. A mí ha venido y habló en mi oreja,
diciéndome lo que ya te había dicho a ti.

---A mí no me hablan nunca los pájaros.

---Sí, Pepe, sí\ldots{} Hay en Roma un alto personaje, el hombre de
confianza del Sumo Pontífice, un sabio y prudente ministro que, al verte
huérfano de Don Matías, te amparó en su propia casa, y extendió sobre ti
el manto de su noble protección. Cómo correspondiste a su hospitalidad y
agasajos, mejor lo dirá tu conciencia que mi boca: no hablemos de
eso\ldots{} Pero recordarás que al despedirte para España con severidad
dulce de gran señor, levísima pena de tu delito, te dijo estas o
parecidas palabras: `Tienes vocación de marido\ldots{} Que tu familia te
procure un buen matrimonio'. Consejo más sabio no ha salido de humana
boca. Ese remedio, esa medicina recetada por el hombre más sabio de la
Europa, yo te la proporcionaré. Déjame ser tu boticaria\ldots»

No puedo seguir\ldots{} Al reproducir en mi mente aquel coloquio
interesante, mis nervios se disparan, y ved aquí los temblorosos
garabatos que traza mi pluma\ldots{} Intenso dolor de cabeza detiene el
curso de la función mental, literaria\ldots{} No puedo, no. Hasta
mañana.

\hypertarget{xi}{%
\chapter{XI}\label{xi}}

¡Casarme! ¡Dios! Inaudita sorpresa\ldots{} De cuanto en el mundo existe
pensé que me hablaría mi hermana menos de matrimonio. ¡Casarme! ¿Y con
quién? ¿Será con la incógnita dama del baile? Esta sospecha elevó al
máximo grado el inmenso desvarío que la extraña declaración de Catalina
produjo en mi mente. Bueno, Señor; que me la traigan, en su verídica
forma y rostro, pues yo no puedo comprometerme a ser esposo de una
máscara.

---Está bien---dije a mi hermana\ldots.---¿Y puedo saber con quién me
caso?

---¿Quieres callarte, chiquillo?---replicó ella con infantil
enojo.---Apenas se te habla de boda, ya estás pensando en melindres.
Conténtate por ahora con saber que me ocupo en curarte, conforme a la
receta del prudentísimo Cardenal, y espera mis acuerdos con todo el
recogimiento y la honestidad que el caso pide.

---Pero, hermana querida, ¿por qué has de ver malos pensamientos en este
deseo mío, tan natural, tan humano, de saber qué persona\ldots? ¿Acaso
no lo sabes tú todavía?

---Lo sé; pero no quiero decírtelo\ldots{} Empezarías a calentarte la
cabeza, a mirar por el lado de la liviandad cosa tan grande y santa como
el matrimonio. No me repliques: con lo que hoy te digo debe bastarte. Y
ponte muy contento, Pepe\ldots{} da gracias a Dios por haberme inspirado
esta idea de tu regeneración por la esclavitud.

Diciendo esto se levantó. Al verla yo en pie, lanzando sobre mí por los
huecos de la doble reja su mirada fulgurante, fui asaltado de un
pensamiento dulcísimo. Quise rechazarlo, y como un rayo atravesó de
nuevo mi mente. Dios me lo perdone. Vi tal semejanza entre la mirada de
Catalina al través de los hierros y la de la mascarita por los agujeros
de su careta, que creí que monja y máscara eran una misma persona.
Vuelvo a decir que me lo perdone Dios, porque sin duda tal pensamiento
fue de los más ruines, y un agravio soez al decoro monástico de mi
hermana, y a la Orden, y al mismo Jesucristo\ldots{} No podía ser, no, y
sólo en la corrupción de mi entendimiento podía encontrar el germen de
tan desatinada sospecha. No tardé en reflexionar, en comparar\ldots{}
Indudablemente, entre el lenguaje mundano y un si es no es desenvuelto
de la máscara, y el acento quejumbroso, salmista y nasal de Catalina, no
había la menor concomitancia\ldots{} La única relación estaba en los
ojos\ldots{} ¿Pero eso qué?\ldots{} ¡Locura, perversión de jovenzuelo
que en nuestra sociedad se llena de malicias antes de ser hombre! Fuera,
fuera, pensamiento vil.

Sin duda influyen en mí los desvaríos de la literatura corriente: en
Italia como en España se ha puesto de moda introducir en dramas y
novelas personajes monjiles, con desprecio de la dignidad religiosa, y
ya vemos profesas y novicias que se dejan robar, o que se descuelgan de
las rejas a la calle, ya otras no menos desatinadas que burlan la
clausura para salir encubiertas a ver mundo, o a husmear, amparadas de
la noche y de un buen tapujo, en las fiestas de Carnaval. Las aventuras
de monjas, hoy tan del gusto de los poetas, pasan de la creación
literaria a nuestro pensar y sentir en los casos de la vida real.
Perdóneme Sor Catalina de los Desposorios que manchara su pureza
arrojando sobre ella jirones de una literatura insana.

«¿En qué piensas?---me dijo la monja como riñéndome.---En vanidades del
mundo, en corruptelas y vicios\ldots{}

---No, hermana querida: pienso que antes de dar el \emph{sí} a tu
proyecto, necesito saber\ldots{}

---¡Dale!\ldots{} Por hoy, no se hable más del asunto. Déjalo que
madure; espera y calla.

---Sí; pero\ldots{}

---Que calles te digo y te mando. Volverás cuando yo te avise y
hablaremos otro poquito\ldots{} Ya no puedo entretenerme más; dentro de
un instante llamarán a coro.

---En ese caso, debo retirarme.

---Aguarda un momento, que quiero hablarte de otras cosillas. Según
parece, en París \emph{han puesto} la República. Los demonios andan
sueltos otra vez por allá: pronto veremos cómo asoman la oreja o el
cuerno los diablejos de aquí. Cuidadito, Pepe, con meterte entre
revolucionarios. Mira bien con quién andas\ldots{} Y no creas que con
callarte y disimular tus locuras, no las voy a saber. Aquí lo sabemos
todo. No te trates con progresistas, que de ésos sacarás lo que el negro
del sermón. Mantente a distancia de los que alborotan, y no te faltarán
adelantos en tu carrera\ldots{} Bien mirado, no porque haya República en
Francia, hemos de tener aquí Progresismo, que en nuestra tierra sobran
medios para poner un dique a la maldad. En Francia no hay religión, aquí
sí; en Francia no hay hombres que expongan su vida por los Reyes, aquí
los hay. Luego\ldots{} En fin, que me llaman a coro. Otro día te lo
explicaré mejor\ldots{} Adiós, hermanito. Que seas sumiso y bueno.
Escribiré a madre que has venido a verme, y se pondrá muy contenta la
pobre\ldots{} Retírate ya\ldots{} El Señor te acompañe\ldots»

Salí de La Latina con tanta confusión y alboroto en mi cabeza, que en
todo el resto del día no fui dueño de mis pensamientos. Las alusiones al
manuscrito, la propuesta de casorio, la sospecha de que mi hermana y la
máscara no eran personas distintas, y, por fin, las vagas apreciaciones
políticas que oí de sus labios al despedirme, tantas emociones y
sorpresas en el breve espacio de una visita que apenas duró media hora,
eran para volverme tarumba, si no tuviese yo un cerebro muy bien
organizado, gracias a Dios. Por fin, al anochecer empecé a ver claro, y
entendí que la protección de Sor Catalina de los Desposorios (¡vaya que
el nombre tiene miga!) era de un carácter positivo, como fundada en el
cariño fraternal. Debía yo, pues, esperar a que se fueran aclarando las
nieblas que envolvían el pensamiento de mi bendita y muy amada hermana.

\emph{3 de Marzo}.---Las noticias de Francia son cada día más
interesantes, y en ellas palpita el drama político, tan del gusto de
estos pueblos imaginativos y apasionados. La fuga del Rey, las escenas
teatrales de la duquesa de Orleáns en las Cámaras, con sus niñitos de la
mano; las barricadas, la proclamación de la República, llegan aquí como
páginas epilogales del sangriento poema del 93. Es muy comentada, con
evidente exaltación de la susceptibilidad española, la noticia de que la
infanta Luisa Fernanda, duquesa de Montpensier, quedó abandonada en las
Tullerías al huir toda la familia real: en aflictiva soledad estuvo la
pobre niña un mediano rato, oyendo el rugido de las turbas, hasta que se
salvó, nadie sabe cómo, pero ello fue por arte milagroso.

Con estas cosas, y lo que aquí se presume y teme, tenemos el cerebro de
Sofía en espantosa ebullición: su voz no cesa de explanar las causas de
la catástrofe, y la precisión en que estamos de poner una aduana de
ideas en la frontera para que no pase acá la dolencia revolucionaria, ni
se nos cuelen en España esas malditas \emph{utopías}. «Aquí no queremos
\emph{utopías}---repite con un flujo de amplificación que acaba por ser
insoportable,---pues bastante guerra nos han dado las que introdujeron
los caballeros de la emigración.»

Lo único que la consuela del detestable cariz que toman los asuntos
europeos es que al frente de la República francesa aparezca la
interesante figura de un poeta, el dulce y tiernísimo Lamartine, que
ahora debe aplicar al arte político las sonrosadas imágenes, las
opalinas nieblas y los reflejos lacustres de sus admirables versos.
Habla Sofía del poeta que hoy preside los destinos de Francia como si
fuera uno de los más puntuales asistentes a su tertulia. Le alaba y
glorifica, recita o manda recitar fragmentos traducidos de las
\emph{Meditaciones,} y pone los ojos en blanco cuando llega un pasaje de
azucarada ternura o rosadas ensoñaciones. «Hay que reconocer---nos dijo
anoche,---que Francia nos lleva ventaja en lo de enaltecer a los hombres
eminentes de la literatura. Miren qué pronto han puesto en la cumbre
política a uno de sus primeros poetas. Aquí, por mucho que adelantemos,
no se hará jamás otro tanto. Ni nos cabe en la cabeza que un día, al
tener que cubrir la vacante de Jefe del Estado, cojamos a Pepe Zorrilla
y de golpe y porrazo lo nombremos Presidente o como quiera llamársele.
Lamartine al frente de la República francesa es como si aquí,
hallándonos sin Reina constitucional, nombrásemos a \emph{Tula} para
este cargo\ldots{} Si cada cual estuviese en su sitio, ¿quién duda que
Don Juan Nicasio Gallego sería Arzobispo Primado, y que otros ocuparían
puestos altísimos correspondientes a su categoría?» Todos convinimos en
que cuanto decía la ilustrada señora estaba muy puesto en razón.

\emph{6 de marzo}.---Escribo esta noche sin otro objeto que consignar la
trastada que me ha hecho mi jefe, el nuevo director de \emph{La Gaceta},
a quien aquí saco a la vergüenza pública para que la Posteridad le
vitupere y maldiga. Apenas tomó posesión el tal de su altísimo cargo, le
enteró la envidia de que su antecesor me había dispensado de ir a la
oficina, con excepción de los días de la sacra nómina, y al punto mandó
un recadito a mi hermano ordenando que me presentase en mi puesto, pues
había pendiente gran balumba de tr que exigía las inmediatas funciones
de todo el personal de la dependencia. Acudí al cumplimiento de mi
deber, con la idea de que me encargarían alguna faena delicada, propia
de mi grande erudición, como traducir discursos o memorias del italiano
y del francés. Pero no fueron estas ramas del saber las que encomendó el
jefe a mi cuidado, sino otras que no sé si clasificar en el orden de la
Partida Doble o de la Estadística, ciencias que requieren entendimientos
privilegiados para su cultivo. Pues, señor: todo el santo día me han
tenido sacando el duplicado y triplicado de la nota de líneas compuestas
por cada cajista, operación no exenta de aparato, porque las tales
listas van en pliegos de marquilla de lo más fino, y se me exige un
esmero y limpieza de trazos que me ponen en grande apuro. Mi inmediato
jefe, que es uno de los mayores gaznápiros que comen el pienso de la
Administración, no aprueba mis prolijos estados sin fruncimiento de
cejas, prolongaciones de hocico y reparos necios por si eché un rasgo
para en vez de echarlo para , o por si mis cincos parecen ochos,
deformación que, de no sufrir ejemplar correctivo, traería la catástrofe
de todo el mundo aritmético. Esta tarde apuró tanto mi paciencia aquel
prototipo de la imbecilidad, que mi mano estuvo a muy poca distancia de
su calva asquerosa, y poco faltó para que su nariz y toda su jeta se
aplastaran contra el pupitre y los papeles que examinaba. Me contuve;
pero salí de la oficina con la certidumbre de que si mañana se repite
tan estúpido vilipendio, no sabré reprimirme. Dígolo porque de algún
tiempo acá siento en mí estímulos de orgullo y extremado concepto de mi
personalidad. No me rebajo fácilmente a nadie, y menos a un ínfimo, que
sólo es mi superior en el brutal escalafón administrativo\ldots{} Las
once dan, yo me duermo\ldots{}

\emph{7 de Marzo}.---¿Sabes, oh Posteridad, que resultó lo que yo me
temía? Pudo más la rabia de verme humillado que la paciencia y
abnegación propias de un funcionario de corto sueldo, y viendo
gesticular ante mí las patas delanteras de mi jefe, protesté en la más
desabrida forma. Irguiose él sobre los cuartos traseros, y me dijo que
inmediatamente daría parte al director de mi falta de respeto, y yo le
contesté que lo mismo a él que a nuestro director me los pasaba por las
narices; que yo no había nacido para hacer listines de imprenta, y que
antes que a esto a barrer la casa me prestaría. Replicó entonces con
grosería chabacana: «¡pues no tiene el hombre pocos humos!» y yo fui tan
dueño de mí en aquel supremo instante que no le vacié el tintero en la
calva, conforme a mi primera intención, y me contenté con decirle: «me
voy, por no romperle a usted el alma, so mamarracho.» Cogiendo mi
sombrero, salí por entre los compañeros, mudos de asombro.

Vedme aquí, pues, cesante, pues no tengo duda de que mi arrebato es
motivo suficiente para que la señora Administración me ponga de patitas
en la calle. Tendría que oír mi hermano Agustín y mi cuñada Sofía cuando
se enteren del suceso. Pero no me importa. He dado gusto a mi dignidad
ofendida, y no me pesa, no, esta arrogancia que el trato social de
Madrid va despertando en mí. Sabed, \emph{¡o posteri!}, que practico el
\emph{nosce te ipsum}; que por las noches, una vez cumplida la
obligación de emborronar papel, examino mi interior, y hago cómputo y
análisis de mis pensamientos y mis acciones. Pues bien: declaro que me
siento altanero; atribuyo ese fenómeno al efecto del ambiente en que
vivo, y a mi fácil asimilación de caracteres y costumbres. Cuando los
años me den mayor experiencia haré la crítica de esta nueva evolución
mía, ahondando bien en sus causas; hoy por hoy me limito a consignar el
caso, y echo la culpa al tiempo, a la atmósfera, como hacemos comúnmente
en el primer diagnóstico de nuestras dolencias. Añado a lo dicho que
entre mis numerosos amigos, de varia educación, origen y clase, doy la
preferencia a los aristócratas; siento que mi naturaleza se asemeja y
adapta cada día más a la de los que nacieron en elevada cuna y enaltecen
su voluntad sobre las voluntades ajenas. Nacido yo en esfera humilde,
aunque no de las más bajas, ¿por qué me siento noble? Privado de bienes
de fortuna, viviendo al amparo de mis hermanos con sólo un triste sueldo
para ropa y gastos menudos, ¿por qué me atrae y seduce la compañía de
los ricos? No lo sé; pero como es así, así lo digo, sin comprender bien
la razón de esta sinrazón.

Entre mis amigos, como dije en otra confesión, los hay de todas las
categorías y para todos los gustos. Bringas y Arnáiz, ambos hijos de
comerciantes, no me inspiran el mismo afecto; Caballero, hijo de un
pastor, me da lecciones de cultura social; Donato es un tarambana muy
divertido, pero que no ahondará en mi corazón; a Leovigildo, la peor
cabeza de Madrid, desordenado y voluntarioso hasta lo increíble, le
tengo yo mucho cariño. De los dos aristócratas que figuran en mi trinca,
Trastamara no es santo de mi devoción; en cambio, Guillermo Aransis
forma conmigo una pareja indisoluble. ¿Qué parentesco moral, étnico,
fisiológico iguala nuestros gustos y unifica nuestros pensamientos? No
entiendo este \emph{gemelismo} (excusad la palabra), siendo él rico, yo
pobre; él de raza histórica, yo de cepa plebeya. Verdad que físicamente
tenemos gran semejanza, y mayor aún en el temperamento. Nos asimilamos
el uno al otro con pasmosa rapidez. Absorbe él mis ideas apenas yo las
expreso; me apropio yo sus modos elegantes apenas los indica.
Naturalmente, dada la situación social de cada uno, no le arrastro yo a
él, sino él a mí; Aransis me lleva a su esfera, sin que yo me dé cuenta
de ello, por graduales movimientos, tirando de mí; me introduce en el
campo de las aficiones, de los hábitos y, ¿por qué no decirlo? de los
vicios aristocráticos. A mí nada me asusta en el medio de vida a que mi
amigo me conduce: no me asusta la disipación, ni el convencionalismo, ni
el vértigo de las alturas.

\hypertarget{xii}{%
\chapter{XII}\label{xii}}

\emph{12 de Marzo}.---Llevado al mundo por Aransis, gracioso diablillo
que no me deja de su mano, heme metido en casas de las clases alta y
media, y en ellas me han salido conocimientos y relaciones que en mucho
estimo y han de serme de no poca utilidad. Algunos días he pasado en
grande aturdimiento, sin fijarme en nada, más deslumbrado que
sorprendido, confundiendo cosas y personas\ldots{} Pero el mundo nunca
es un páramo, y si lo fuera, la juventud que va por él haría salir
flores del suelo con sólo pisarlo. Eso me ha pasado a mí. Sentíame yo un
tantico aburrido andando sobre tan diferentes alfombras, cuando una
noche, inopinadamente, en una casa de medio tono, modestita y al propio
tiempo distinguidita, vi surgir ante mí flores risueñas y
fragantes\ldots{} Verde y con asa, dirán los que esto lean: ya tenemos
enamorado al confesor de sí mismo. Poco a poco: necesito
explicar\ldots{}

¡Ay, Dios mío!\ldots{} se me olvidó un caso interesantísimo, cuya
preterición podría traer grave oscuridad a este relato. No tengo más
remedio que volver un poquito atrás con permiso de los que dentro del
siglo me lean, y si por acaso no les pareciere bien retroceder conmigo,
espérenme aquí, que pronto vuelvo.

¿No dije, al referir mi querella con el jefe de la oficina, que el
cataclismo era inevitable, y que se decretarla una fuerte pena, quizás
la cesantía? Pues así sucedió a los pocos días del dramático lance; pero
ello fue muy distinto de como yo lo esperaba y temía. Excuso decir que
no he vuelto a parecer por la \emph{Gaceta}, y que me doy por expulsado
ignominiosamente. Pues ved lo que pasó, y asombraos conmigo. Acababa yo
de almorzar, cuando me anunciaron que un señor viejo deseaba verme.
Aunque se me dijo que era de traza humilde y que sin duda venía con
propósito mendicante, mandé que le pasaran a la sala. Imaginad mi
sorpresa cuando me vi ante D. Faustino Cuadrado, mi superior inmediato
en la oficina, al cual ultrajé de palabra más que de obra. Mi
estupefacción llegó a lo terrible cuando el desdichado sujeto, elevando
hacia el techo sus trémulas palmas, exclamó con luctuoso
acento:---¡Cesante!

Yo\ldots---dije extrañando mucho que llorara para darme la noticia. Y él
replicó:

---No: usted no\ldots{} ¡Yo\ldots{} yo\ldots{} cesante yo\ldots{}

---Pues no lo entiendo, señor mío. Usted cumplió con su deber. Yo no
creía compatible mi dignidad con el deber de usted\ldots{} y\ldots{}

---En buena lógica, a usted le correspondía el castigo. ¿A mí, por
qué?\ldots{} ¿Qué hice yo, desdichado de mí, que llevo veinte años con
diez mil cochinos reales; yo, que fui de los que en las Cabezas de San
Juan se unieron a Riego; yo que serví lealmente con seis mil al Gobierno
del Sr.~Zea Bermúdez; yo que en tiempo de la Gobernadora retrocedí a
cinco mil, y luego fue menester que por mí sacara el Cristo el Sr.~de
Istúriz para recobrar los seis?\ldots{} yo que serví con Mendizábal, y
juntos trabajamos en el decretito aquel de las campanas; yo, casado y
con seis de familia, que por llevar a casa unos tristes garbanzos he
apechugado con lo más contrario a mis convicciones, sirviendo con el
mismo celo a Espartero y a Narváez, a González Brabo y a Olózaga, a los
Puritanos y a los Ayacuchos y al demonio coronado; yo que en tantísimos
años no he faltado un solo día a mi obligación, ni tengo la más
insignificante nota desfavorable; yo que con nadie me meto; yo, Faustino
Cuadrado, cesante\ldots{} cesante! ¿Y por qué, Señor, por qué? Sea usted
imparcial, caballero, y diga, ante Dios y los hombres, si yo le he
faltado\ldots{}

---Yo falté a usted, lo reconozco---dije noblemente, sintiéndome
confuso, lastimado por tanta injusticia,---y de todo corazón tengo que
inclinarme ante su desgracia, y pedirle que me perdone aquel arrebato.

---¡Cesante\ldots{} mis hijos sin pan, yo trastornado, pues no sé a qué
santo encomendarme, ni a quién volverme, ni en qué árbol ahorcarme!

---¿Está usted bien seguro de que la causa de su cesantía fue la
cuestión aquella?

---¡Cristo me valga! Pues si el director, cuando me leyó la sentencia me
lo dijo bien clarito: «Por haber faltado al respeto al señor de
Fajardo\ldots» Y luego me salió con que es usted un sabio\ldots{} un
sabio de reputación europea\ldots{} que nos está escribiendo la Historia
del Papado\ldots{} ¡Pues por qué no me lo advirtió, rabo y uñas de
Satanás! ¿Por qué al darme prisa para los listines, y encargarme que no
le tuviera a usted ocioso, no me dijo: «Guarda que es podenco, guarda
que es sabio, guarda que ha escrito la vida del Santo Padre, que para mí
ha sido la vida de Judas Iscariote\ldots?» La culpa la tiene el señor
director, que no me puso en autos\ldots{} Sin duda estaba tan enterado
como yo de la dichosa sabiduría\ldots Y se me figura que también a él le
han acusado las cuarenta, porque cuando me dio el escopetazo, se rascaba
la barba y decía: «Debieran los sabios llevar chapa en el sombrero, para
que los conociese todo el mundo.»

Como yo afirmase con toda sinceridad que no se me alcanzaba de dónde
podía venir el tremendo golpe, puso cara fatídica, y alzando el dedo
índice cual si quisiera horadar el techo, repitió: «De arriba, Sr.~de
Fajardo, de arriba.

---Creo que padece usted una alucinación. Yo puedo asegurarle que a
nadie he dicho nada, ni aun a mi hermano\ldots{}

---¡De arriba, de arriba!\ldots{} Imposible, señor de Fajardo, que usted
no lo haya dicho. Por las once mil Vírgenes, haga memoria.

---De veras: nadie sabe que nos peleamos, que abandoné la
oficina\ldots{}

---Haga memoria, por los clavos de Cristo.

---Recordando estoy\ldots{} Tan sólo a una persona\ldots{}

---¿Lo ve? ¡Cuando digo\ldots!

---Tan sólo lo he contado a mi hermana, a una hermana mía, monja.

---¿Monja? ¡Dios uno y trino, como si lo viera! ¿Conque monjita? ¿Y en
qué convento?»

Cuando le dije que en La Latina, cayó el hombre desplomado en un sofá, y
llevándose ambas manos a la cabeza, apoyados los codos en las rodillas,
quedó un rato como estatua de la consternación, sin otra señal de vida
que un mugido cadencioso. Confuso yo de verle en tan extraña actitud, no
hacía más que contemplar su espaciosa calva granulosa, aquella calva
sobre la cual, días antes, había pensado vaciar el tintero.

«Como si lo viera, como si lo viera\ldots---murmuró
incorporándose.---¿No dije que de , de muy ?\ldots{} ¡Ay, que mundo, qué
país!\ldots{} ¿Verdad que es divertido nacer español?

---No es muy divertido que digamos, principalmente para los que no nacen
ricos.

---O hijos de frailes\ldots{} o hermanos de monjas.

---Pero ¿usted cree\ldots?

---Sr.~de Fajardo---dijo entre suspiros,---viniendo de donde viene el
rayo que me ha partido, ya no tengo compostura como no salga usted mismo
en mi defensa. Pida a su señora hermana mi reposición.

---Sí que lo haré. Mi hermana es buena.

---Será una santa. Diga: ¿y tiene llagas?

---Hombre, no sé\ldots{}

---¿Siquiera postemas?\ldots{} En fin, bendita sea si me socorre. Para
usted propio no necesita pedirle nada, pues a estas horas ya le habrán
ascendido. Bueno es nacer de pie, caballerito; pero aún es mejor nacer a
caballo. Y ya que va usted tan a gusto en el machito, lléveme a la
grupa. Pido bien poco: la reposición, a no ser que usted y la reverenda
monja, considerando que fui yo el ofendido, me consigan el ascenso a
diez mil. No habría nada más justo.»

Dicho esto, se despidió el infeliz hombre, no sin arrancarme formal
promesa de interceder en su favor. Le consolé y alenté con toda mi alma,
y desde aquel punto y hora, la compasión me hizo su amigo y mi
conciencia su protector, comprendiendo que no es el buen Cuadrado tan
tonto como yo creía. Dejome aquella visita una impresión extraña, no sé
si de asombro, no sé si de miedo\ldots{} ¡Mi hermana\ldots{} La Latina!
Por hoy no digo más.

\emph{13 de Marzo}.---Ya estoy aquí otra vez. Perdónenme el plantón los
que no quisieron volver atrás conmigo. Quedamos, si no recuerdo mal, en
que mis futuros leyentes podrían decir: «Ya tenemos enamorado al
confesor de sí mismo.» Pues no hay aún motivo para suposición tan grave
como la de que ardo en amores. Es tan sólo una dulce ilusión, un
regocijo estético. Y al emplear este calificativo, no vacilo en asegurar
que las dos señoritas de Socobio, Virginia y Valeriana (a la que llaman
Valeria), conocidas por mí en los salones, más bien sala y gabinetes de
D. Serafín de Socobio, no son prodigios de belleza. Nadie que las vea
con ojos de crítica, encontrará en las diferentes partes de rostro y
cuerpo la necesaria armonía y proporciones de que resulta la hermosura;
pero también digo que todo el que las mire, las oiga y trate, sentirá un
agrado que bien puede subir a los espacios del amor. Son delgaditas, muy
derechas, torneaditas en donde es debido, esbeltas y flexibles. De cara
se parecen y no se parecen. No sé qué las iguala, qué las distingue.

Por el sentimiento se meten Virginia y Valeria en el corazón de sus
amigos; por su picardía decente y bien sazonada de ingenio los
esclavizan y confunden. Yo paso junto a ellas mis ratos más divertidos,
y las vuelvo locas con las mil niñerías chispeantes que les digo y
cuento. Ambas son muy inteligentes; tienen alguna cultura y anhelan más.
En justicia declaro que no las divierto yo a ellas menos que ellas a mí.
Formamos un trío delicioso, en el cual no falta godeo de amores, sin
formalidad por ahora. Si se me permite mostrarme en toda la fatuidad que
voy adquiriendo, diré que las dos me quieren: a solas conmigo me
pregunto: «¿Es verdadero amor lo que sienten por mí?» Y no pudiendo ser
igual, con exacta medida, el efecto de una y otra, pregunto también:
«¿Cuál de las dos me quiere más?»

No debiendo por hoy consagrar a la interesante pareja de señoritas
desmedido lugar en mis Confesiones, paso a mejor asunto, que aún no he
hablado sino de una parte mínima de las flores que van brotando en mi
camino. Doy la preferencia a la que ahora os presento para que la
admiréis como yo la admiro. Hará cinco noches que vi en casa de Socobio
a una gallarda mujer de tez morena, pelo y ojos muy negros, el talle
reducido al mínimo volumen, el seno al máximo, todo ello sin menoscabo
de la buena armonía. La señora de Socobio me presentó a ella
designándola como de la familia: era también esposa de un Socobio, y su
nombre, Eufrasia, quedó grabado en mi memoria. Pero tan ceremoniosa
estuvo conmigo, y encontré en ella tal desvío y reserva, siempre que
intentaba yo pegar la hebra de una galante conversación, que me retiré a
mis tiendas, reduciéndome a mirarla todo lo posible con un interés que
no dependía exclusivamente de su belleza un tanto moruna. A la noche
siguiente mis queridas niñas hablaron de la dama con más respeto que
cariño. Supe que Eufrasia se había casado en Roma con un tío de ellas,
D. Saturnino del Socobio; mas no supieron o no quisieron decirme por qué
casó en Italia y no en España. ¿Es por ventura italiana? A esta duda
respondió Valeria diciéndome: «No, Pepito: es manchega.» Y agregó
Virginia que el padre de Eufrasia es un progresistón de los que figuran
en el grupo sensato de Mendizábal, Cortina, Infante y Madoz. Según esto,
la mujer morena es hermana de mi íntimo amigo Bruno Carrasco.

Con estas y otras noticias que iban llegando a mi conocimiento,
aumentaba el interés que por la manchega dama sentía yo, y éste subió de
pronto anteanoche, viéndola menos esquiva y casi casi gustosa de mi
conversación. Aprovechando la feliz coyuntura de encontrarnos lejos de
la masa de tertuliantes, díjele que habiendo yo pasado en Roma días
críticos de mi vida, gozaba mucho hablando de aquella gloriosa ciudad
con cuantas personas la hubieran visitado.

Agregué a este exordio calurosa declaración de la amistad que tengo con
su hermano, y protestas de lo mucho que le admiro por su bondad y
talento, y no fue preciso más: entré, entramos en un diálogo vivo. «Ya
me han dicho las niñas que estaba usted en Roma cuando la elección de
Pío IX.» Y ella: «Sí, y aquéllos fueron para mí días muy felices.» Y yo:
«Para mí no tanto.» Y ella: «Lo supongo: perdió usted a su protector, el
Sr.~D. Matías de Rebollo.» Y yo, sin manifestar sorpresa de oírle
nombrar a mi amigo: «Perdí mi sostén, mi guía, mi amparo.» Y ella: «Pero
luego no le faltaron a usted amigos\ldots{} y amigas\ldots» Diciendo
esto, se echó a reír de un modo tan franco, que me sentí como invitado a
mayores franquezas. «Yo creí---le dije,---que se llamaba usted Higinia,
y que era natural de Puentedeume.» «Cállese la boca---replicó,---y no me
haga reír más, que ya estamos llamando la atención.»

Aproximáronse dos damas y hube de suspender mi indagatoria; pero media
hora después, cuando volvíamos del comedor dándole yo el brazo, abordé
la cuestión y me fui derecho al bulto, conforme a los sabios consejos y
reglas de vida que me había dado Aransis. «Ya es inútil---le dije,---que
usted finja más tiempo conmigo.

---Si yo no finjo, ni hay para qué. Trátase de una broma inocente, de la
que no tengo por qué avergonzarme.

---Así, así me gusta\ldots{}

---Pues sí, señor mío, yo soy la máscara. ¿Qué tal?

---Me volvió usted loco.»

Y como siguiera yo expresando con cierta exaltación mi deseo de mayores
explicaciones, dejó de reír y gravemente me dijo: «No hablemos una
palabra más de aquella tontería sin importancia. Aquí, hábleme usted de
la función de anoche, de la nueva moda que ha venido para el peinado en
\emph{bandós}, o de política si le gusta; a mí no. Y de aquella broma,
punto en boca. Si quiere usted saber más, lo sabrá en mi casa. Desde la
semana próxima recibiré a los amigos los miércoles. Mi marido le
invitará a usted. Debo advertirle que mis explicaciones serán breves, y
que no ha de encontrar en ellas ni sombra de malicia, ni el menor asomo
de aventura.» No tuve tiempo más que para decirle con cierta ansiedad:
«Por Dios, no se olvide usted de advertir a su esposo\ldots»

---Sí, sí\ldots{} vendrá usted a casa, o, como ahora se dice, \emph{será
usted de los nuestros}.

\hypertarget{xiii}{%
\chapter{XIII}\label{xiii}}

\emph{14 de Marzo}.---Sin aguardar a que me llamase mi hermana, he ido a
verla; tanto me aprieta el afán de reparar la injusticia cometida con el
pobre Cuadrado. Aunque la espera no fue larga, aburríame el plantón en
la penumbra fría del locutorio, aspirando el singular tufo de convento,
mezcla de olorcillos de humedad, de incienso, de ropas de lana en
continuo uso. Para colmo de hastío, no había en la estancia ninguna obra
de arte con que entretenerme, pues un San Francisco recibiendo la
impresión de las llagas, pintura nefanda, con el lienzo podrido a trozos
y el marco apolillado, más causaba miedo que admiración. Llegó Sor
Catalina presurosa quejándose de que mi visita no anunciada la distraía
de ocupaciones apremiantes; pedile perdón por la inoportunidad, y al
punto explane el caso de Cuadrado y mi disgusto por la absurda situación
en que nos veíamos: él, inocente, castigado; yo, culpable, impune.

Sin mostrarse sorprendida de que yo acudiese a ella para tal negocio,
negó su influencia y puso muy en duda la posibilidad de servirme; pero
bien se le conocía el discreto fingimiento, porque ni aguzaba las
razones ni extremaba el sonsonete gangoso y aflautado. El argumento de
más eficacia que esgrimí fue éste: Querida hermana, si tú no hallas la
manera de reponer a Cuadrado en su destino, me presento yo al Ministro,
y le suplico que dé al otro mi plaza y a mí la cesantía. La abnegación
gallarda de este propósito hizo efecto en Catalina, que muy satisfecha
me dijo: «¡Cuánto me place ver tan al descubierto tu buen corazón! Así,
así quiero yo a mi hermano. Si pudiera yo influir en que se quiten y den
destinos, muy pronto quedaríais complacidos los dos. Pero\ldots{} en
fin, yo veré si puedo\ldots{} No sé a quién podría recomendar\ldots»
Aplicando a estas formulillas hipócritas la clave monjil, las interpreté
como un lenguaje parabólico para decirme que todo se arreglaría, y que
la reparación del grave yerro corría de su cuenta.

Repetía yo con cierta pesadez mi petición para que quedara fija en su
ánimo, cuando entró una señora en el locutorio. Catalina se alegró de
verla. Era la tal pequeñita, ya entrada en años, vivaracha, de semblante
risueño y simpático, y no se contentó con mirarme una vez, sino que en
mí ponía sus ojos con fijeza, como si quisiera tomarme la filiación. «Es
mi hermano,» le dijo Catalina; y oyéndolo la viejecita me saludó muy
afectuosa, obsequiándome con estas finuras: «Ya decía yo\ldots{} la cara
no miente. ¡Y qué guapo es! Sor Catalina, bien puede usted estar
orgullosa\ldots{} Ya, ya le conocía yo a usted, caballerito, por lo que
cuenta la fama\ldots» Dábale yo las gracias por su amabilidad, y ella,
ocupándose más de mi hermana que de mí, introdujo por la reja estas
palabritas: «Eufrasia no puede venir: tiene hoy la casa llena de
mueblistas, tapiceros y doradores\ldots{} Es tan grande el barullo
que\ldots» No acabó el concepto, porque aparecieron tras de los hierros
otras monjas: vi que eran dos, y oí una gangosa y compungida voz que
claramente dijo: «¡Oh, Cristeta\ldots{} qué cara te vendes!» Mi hermana
me indicó por señas que debía retirarme, y así lo hice: salí a la calle
atando cabos, encasillando rostros y casos en mi memoria con el debido
método, en previsión de acontecimientos futuros.

\emph{20 de Marzo}.---Conforme al gracioso anuncio que oí de labios de
su esposa, el Sr D. Saturnino del Socobio me invitó a sus reuniones, y
con esto queda expresada la diligencia con que yo acudí a la casa de
aquel buen señor, en la cual pude advertir que todo era nuevo,
allegadizo, dispuesto por la mano inteligente de la dama moruna. Allí
encontré mucha y buena gente, aunque no la mejor de Madrid, pues había
un poquito de entredicho social contra el tal matrimonio, por lo que yo
supe aquella misma noche y contaré después para la más ordenada
composición de mi relato. Amable con todos la dueña de la casa, lo
estuvo conmigo singularmente, más que por lo que me dijo, por lo que con
cautelosas y bien medidas razones me dio a entender. He aquí la muestra:
«Tengo que advertirle, señor mío, que procure no desentonar en sus
opiniones políticas cuando tenga ocasión de manifestarlas. Hace poco le
hablaban a usted mi marido y sus amigos del liberalismo de Pío
IX\ldots{} y, como es natural, lo condenaban\ldots{} porque ésas son sus
ideas. Cuando el Sr.~de Conard dijo que el Papa actual es un
\emph{Robespierre con tiara}, y que preside las logias masónicas, usted
se indignó, puso el grito en el Cielo y\ldots{} ya recuerda lo demás.
Pues es preciso que varíe de táctica, y que acomode sus opiniones a las
de mi gente, si no quiere que con suavidad y finura le cierre yo las
puertas de mi casa.»

Segunda muestra: «Óigame, Fajardo: no se le ocurra a usted elogiar otra
vez al Paganismo. Siempre que se trate de griegos o romanos, llámelos
\emph{gentiles} o \emph{idólatras}, como a usted le parezca, y póngalos
que no haya por donde cogerlos. Volviendo a lo de la máscara, no
pretenda saber más de lo que ya sabe. Yo fui al baile con el
consentimiento de mi marido, sin más objeto que el inocentísimo de pasar
un rato y ver la gente. No iba con propósito de ver a usted ni mucho
menos. Que se le quite eso de la cabeza. Por mi hermano conocía yo
personalmente a usted: una noche, en el Príncipe, hallándonos en un
palco, me enseñó un grupo en que estaban varios de sus amigos,
designándolos por sus nombres\ldots{} Al encontrármele a usted en
Villahermosa, perdido en el salón grande como un palomino atontado, me
dije: `Ya tengo a quién dar una broma que ha de ser muy divertida'. Y
como el día antes había leído las Confesiones, ya ve\ldots{} todavía me
estoy riendo\ldots{} Y no me pregunte más\ldots{} Cierre el pico y tenga
paciencia.»

Tercera muestra, la segunda noche, invitado a comer: «Otra vez tengo que
reñirle. Por las llagas de Cristo, no hable usted mal de los que antes
abominaron de la desamortización y ahora compran los bienes raíces que
fueron de frailes y monjas. Mire usted que los amigos de casa adquieren
todo lo que sale, y mi marido anda ahora en tratos con la Hacienda para
quedarse con una gran finca que fue de los Jerónimos en la provincia de
Cáceres. ¿Qué le importa a usted que compren o que no compren? Sea usted
cauto y hágase al ambiente. Respecto a sus \emph{Confesiones}, diré que
Sofía las llevó a una monja de La Latina, que no debo nombrar. No se
incomode usted con su cuñada, que el abuso de confianza no significa en
ella más que una grande admiración hacia usted, y el deseo de que todos
participen de esa admiración. La monjita que disfrutó esa historia por
primerva vez después de Sofía, y que es algo literata y no muy
intransigente con lo mundano, me la dio a leer a mí: somos grandes
amigas, paisanas, y a sus buenos consejos debo yo el haber salido bien
de ciertas borrascas que en su día sabrá. Pues de mis manos pasó el
cartapacio a otras: no se asuste. A estas horas lo ha leído medio
Madrid, y tiene usted una celebridad reservada, que no sale en papeles
públicos, mas no por eso menos extendida. Direle que después de dar la
vuelta, tornó el manuscrito al convento, y luego ha vuelto a salir.
Estuvo en poder de Sartorius, que leyó un poquito, y por cierto lo alabó
grandemente; de las manos de Sartorius pasó a perfumadas manos, y ahora
está\ldots{} esto sí que no puedo decírselo.

---Me sumergirá usted en un mar de confusiones si no me lo dice.

---Pues está en una casa muy grande.

---En casa de Montijo.

---No: allí ya estuvo. Eugenia lo ha ponderado muchísimo. La casa donde
ahora está es más grande.

---¿La de Altamira, la de Osuna?

---No: es mayor, mucho mayor.

---Ya.

---No me pregunte usted más.

---Dígame usted sólo una cosa\ldots{} el sexo de la persona que me ha
leído en esa casa grande.

---¡Ah!, le habrán leído personas de ambos sexos.

---Quiero decir, la persona que pidió mi manuscrito.

---Mucho quiere usted saber. Cierre el pico y agradézcame las franquezas
que tengo con usted. Si no corresponde a mi confianza con su discreción,
no cuente ya conmigo para nada.»

¿Qué tal, señores de la Posteridad? ¿Tengo o no motivos para estar estos
días nervioso, distraído, inquieto, como si en torno mío zumbarán
avispas?

\emph{26 de Marzo}.---Mi amigo Aransis, para quien no tengo secretos, me
aconseja que no retrase el declararme a Eufrasia con las demostraciones
más apasionadas, cuidando, eso sí, de hacerle comprender que sabré
emplear la delicadeza más exquisita para no comprometerla. No necesitaba
yo de estos estímulos para lanzarme, y en la primera ocasión propicia,
el miércoles último, le mostré mi corazón lacerado y el trastorno
inmenso que han traído a mi alma las gracias de su persona. Estimando
más interesante que mi declaración la respuesta de la dama, doy aquí
preferente lugar a los retazos más bonitos de la admirable tela que
tejió con sus palabras:

«¿Querrá usted callar? Por Dios, Pepe, ¿se ha vuelto usted loco? Pues a
mí no me enloquecerá usted, yo se lo aseguro, que por naturaleza tengo
la cabeza bien firme, y además las desgracias me la han claveteado y
endurecido. Calma, amigo mío; tenga calma y juicio. Aun cuando yo
creyera que es verdad todo lo que usted acaba de decirme, tendría que
darle un no como esta casa, o como otra casa más grande. Es usted un
chiquillo, y yo, si en años le aventajo más de lo que parece, en
experiencia, ¡ay!, lo que es en experiencia, Pepe, le doblo la edad,
créame\ldots{} No quisiera yo hablar de esto: usted me obliga a recordar
mis amarguras\ldots{} he vivido, he padecido lo que usted no puede
imaginar\ldots{} sé lo que son los diferentes suplicios a que nos
condena nuestra condición; conozco la esperanza hoy viva, mañana
moribunda; conozco la ansiedad, la desesperación, la dignidad herida;
conozco los ultrajes, la cólera propia y ajena; conozco todo\ldots{}
hasta la vergüenza\ldots»

Llevose la mano al rostro. La pausa que entonces se produjo llenela yo
con frases vacías, porque no se me ocurrieron otras. Luego siguió: «Yo
he sido muy desgraciada. Me sería muy fácil demostrárselo contándole
algunos pasos de mi vida; pero no hay para qué\ldots{} Algo habrá quizás
que usted sepa; algo que no ha de saber si yo no se lo cuento. Pero ni
lo uno ni lo otro le contaré: no quiero entristecerme. He sido muy, muy,
pero muy desgraciada. Ahora, válgame la verdad, ahora no tengo la
felicidad, esa felicidad con que se sueña a los veinte años\ldots{} ya
ve usted qué cosas le digo\ldots{} No tengo la felicidad; pero tengo el
sosiego, la paz; y esta paz y este sosiego no los tiraré por la
ventana\ldots{} Sé lo que son pasiones de hombres, y como lo sé, no
cambio por ninguna de ellas mi paz\ldots»

Tomando en seguida un tonillo jovial, y antes de que yo desembuchara los
conceptos que se me habían ocurrido, prosiguió: «Engolosinado usted,
amigo mío, con su aventurilla de Italia y con alguna otra que habrá
tenido por acá, de esas fáciles y para un rato, ha llegado a creer que
todo el monte es orégano. Me coge usted vieja, si no de años, de
picardía y conocimiento del mundo; me coge usted, se lo diré claro, muy
escarmentada\ldots{} Déjese usted de locuras, y seamos buenos
amigos\ldots{} y nadita más, Pepe\ldots{} Una cosa en que yo le aventajo
a usted, ¿a que no sabe lo que es? Pues es el don de conocer y apreciar
lo muchísimo que vale la amistad. Y ésta tiene sus goces, sus
incertidumbres; también sus penitas, dulzuras no digamos, que se
avaloran más con la pureza\ldots{} En fin, mi amigo, haga caso de mí, y
no se le ocurra volver a decirme lo que me ha dicho. ¿Estamos en ello?

---Estaremos en ello y en todo lo que pueda
sobrevenir---respondí.---Claro es que mi primera obligación con usted es
la obediencia. Y yo le aseguro que no tendrá queja de mí\ldots{} Pero
advierta, mi dulce amiga y dueño, advierta que manda usted en mis actos,
no en mi corazón.

---También en su corazón\ldots{} ¡Pues no faltaría más sino que a ese
loquillo le dejáramos hacer de las suyas! Es un niño, créame usted, y a
los niños se les educa, se les guía, y también se les da una buena solfa
cuando es menester.

---Niño será, como usted supone. El niño es comúnmente revoltoso, y
aunque se le castigue, con sus gracias y zalamerías acaba por ser el amo
de la casa. Todos le riñen si es travieso; todos tiemblan cuando le ven
malito. Y la idea de que pueda morirse conturba más que el cataclismo
universal. Este chiquillo que yo tengo en mi pecho pertenece a
usted\ldots{} No me le castigue, por Dios; déjele vivir a su
gusto\ldots{} Yo le respondo de que será obediente, juicioso,
calladito\ldots{} Vivirá en la adoración de usted\ldots{}

---Déjese usted de adoraciones, por Dios.

---En la idolatría, en un culto mudo, escondido a todas las
miradas\ldots{}

---¿Catacumbas tenemos?

---Catacumbas.

---¡Ay, no!, que son muy tristes. Crea usted que he tomado
aborrecimiento a todo lo que sea oscuridad, ocultación, misterio, vivir
con el temor de que me descubran\ldots{} Prefiero la vida en plena luz,
con sólo un bienestar tranquilo\ldots{}

---Yo no le pido a usted que se meta bajo tierra, ni que viva en el
misterio. El que andará escondido seré yo, porque así me lo impone la
que ha venido a ser mi dueño absoluto. No le ocasionaré la menor
inquietud. Amor y abnegación son hermanos gemelos\ldots{} Tan difícil
será que yo altere la paz de usted como dejar de amarla, porque mi amor
es toda mi alma, y nada puedo contra él, como no se puede nada contra
Dios. Es este amor mi suplicio y mi encanto, Eufrasia. Déjeme usted que
en silencio me arregle yo en mi cenáculo escondido. Aquí tengo mi altar,
y en el altar mi divinidad.

---¡Divinidad yo!\ldots{} ¿Ahora salimos con eso?

---Divinidad, a quien adoro más porque ha sido mártir\ldots{} porque ha
padecido\ldots{} Ahora me toca a mí el padecimiento.

---No le compadezco si se empeña en ser tonto.

---Así somos llamados los que adoramos un ideal, los que por ese ideal
vivimos, los que por él estamos dispuestos a morir\ldots{}

---¿Con que ideal?\ldots{} ¿yo ideal?\ldots{} \emph{No me jaga uté reír,
Joselito.»}

\hypertarget{xiv}{%
\chapter{XIV}\label{xiv}}

\emph{28 de Marzo}.---Leído lo último que escribí, me han dado
intenciones de borrarlo, pues si los conceptos de Eufrasia me resultan
hermosos y sinceros, como producto inmediato de la realidad, los míos se
me antojan artificiosos y de poco fuste, pues todo aquello de la
divinidad, del ideal y del altarito pertenece al manoseado repertorio de
los amantes que por primera vez en su vida abordan tan grave cuestión.
Muy santo y muy bueno que con una inocente o novata de amor emplease yo
tales pamplinas; pero con mujer que ha corrido ya temporales duros en el
océano de la pasión, estimo que debí emplear otro lenguaje y método. Sea
como quiera, no borraré nada del texto escrito, porque ante todo ha de
prevalecer la verdad en estas \emph{Confesiones}; y si estuve tonto, que
tonto me vean los que han de leerme, y yo de ello me consuelo con la
esperanza de ser en otra ocasión más agudo.

No creo frustrada mi conquista, por más que la moruna Eufrasia se
mantiene en el firme terreno de la amistad, donde yo le propongo
levantar una tienda para platicar juntos y solos sobre las inmensas
dulzuras de ese sentimiento, que tanto ennoblece a los humanos. Ella no
quiere nada de tienda, temerosa del recogimiento y soledad que este
mueble trae consigo, y prefiere que no tengamos más abrigo que la
anchura de la casa y del mundo, sin escondrijo, ni misterio, ni
arrumacos de ninguna clase. A pesar de esto, voy creyendo que mi
aventura no lleva mal giro. Por cierto que a la consolidación de mi
creencia no contribuye poco la misma Eufrasia \emph{sentando las bases},
como ahora se dice, de nuestro pacto de amistad, y va teniendo ésta tal
extensión que se nos impone el secreto en diversidad de momentos y
casos; amistad muy bonita y amena, con frecuentes consultas de una parte
y otra, consejitos, protección moral y otras cosas dulces. Mejor que por
mis referencias, lo comprenderán mis lectores por la fiel copia de algún
fragmento de los sabios discursos que la dama me endilga:

«Ha seguido usted mis consejos, menos uno, y en él tengo que insistir.
Es forzoso que en el teatro suprima usted el mirar constante con gemelos
o sin ellos. Pero ¿no se hace cargo todavía de que no sólo es
inconveniente, sino de mal gusto? Tome ejemplo de mí, criatura, que todo
lo veo sin parecer que miro nada. Sin clavarle los ojos, le he visto tan
acaramelado que me daba risa\ldots{} Ya notaría usted que la noche de
\emph{Borrascas del corazón} me puse en la cintura el ramito de
verbenas, que son las flores más de su gusto, y lo hice para obtener de
usted ahora la reducción de sus visitas a casa, que no deben pasar de
tres por semana\ldots{} Y a propósito de \emph{Borrascas del corazón}:
¿le gusta a usted esa obra? A mí no: tanta melosidad me fastidia, como
el arrope de mi tierra, que me empalaga, y además me sabe a
botica\ldots{} Pues siguiendo con mis advertencias, diré a usted que sí,
sí, está muy bien que sea expresivo con mis sobrinas Virginia y Valeria;
pero no tanto, caballerito, no tanto, porque son muy tiernas, demasiado
sensibles, y podrían las chiquillas alborotarse más de la cuenta. Su
madre es tonta y nada de esto ve: yo lo veo todo. No me cansaré de
recomendarle que, al ser amable con ellas, no haga diferencia entre las
dos y las iguale siempre en sus demostraciones, para que ninguna se crea
con derecho a tenerle por novio. Mírelas como gemelas en su amistad, o
como aquellas hermanas que estaban unidas por el estómago, por el
costado, no sé por dónde. Así no habrá peligro.»

Para muestra basta lo copiado. Debo decir que el entredicho en que tiene
la buena sociedad a Eufrasia no lleva trazas de concluir. A su casa no
acuden señoras de alto copete, ni otras que, nacidas y criadas en las
zonas medias, son extremadamente melindrosas en la moral casera y
pública. Verdad que mi amiga se defiende valerosa, y con su talento,
amabilidad y exquisito tacto va ganando cada día más voluntades y
atrayendo gente; pero aún le falta mucho para llegar a la rehabilitación
que anhela. El motivo de su aislamiento me lo explicó Ramón Navarrete,
hombre de grande erudición social, y a la sazón mi segundo jefe en la
\emph{Gaceta}. Después del ruidoso tropiezo de la señorita de Carrasco,
bajo el poder de Terry, aventura de que se enteró todo Madrid, anduvo la
infeliz por senderos torcidos, amparándose contra la opinión en las
tinieblas del incógnito. De su existencia en aquellos terribles días
poco se sabe, algo se sospecha, y mucho quizás se miente. Y así como el
río de su patria manchega se mete bajo tierra cuando le parece bien, y
luego vuelve a salir a flor del suelo, del mismo modo, pasado algún
tiempo en subterráneo curso, volvió afuera la dama y el mundo la vio
llevada de la mano por un hombre benéfico, D. Saturnino del Socobio.

Recatábase Eufrasia en aquel tiempo de toda relación social, y hasta de
su propio padre y familia, y como su protector tuviese que emprender un
largo viaje a Roma (que en negocio de capellanías y colaciones tenía no
pocos entuertos que enderezar allá), pidiole ella el extremo favor de
acompañarle, movida no tan sólo del cariño, sino también del deseo de
cuidarle y asistirle (que no carecía de achaques el buen señor);
resistiose D. Saturno temiendo el qué dirán de su familia, así en Madrid
como en Italia; pero con su labia y embelecos de lo más fino salió
adelante la hembra con su gusto, que algunos creyeron capricho y ganitas
de ver mundo.

Roma fue para los dos dichosa tierra, porque D. Saturno mejoró
notablemente de sus alifafes, y ella se reconstituyó físicamente, y se
puso tan lozana que daba gozo. Vieron y admiraron cuanto encierra la
metrópoli del Paganismo y de la Cristiandad; él se esponjó y se hizo más
sociable; ella aprendió un poco de italiano y de literatura dantesca y
petrarquina. Por dicha de Eufrasia les precedió en el viaje a Roma Don
Vicente de Socobio y Suazo, canónigo patrimonial de Vitoria, nombrado
para ocupar la plaza vacante por defunción de mi protector D. Matías de
Rebollo, y una de las cosas en que puso el venerable varón más empeño
fue reducir a buen orden cristiano las relaciones de D. Saturno con la
manchega. Ésta, que por casarse bebía los vientos, desplegó todo su
talento y trastienda para cautivar el ánimo del clérigo, hombre sencillo
y bondadoso que fácilmente vio en la buena moza una Hija Pródiga que en
gran desolación tornaba al hogar paterno, y debía ser recibida y
perdonada.

Conociendo a Eufrasia como la conozco, no necesito que nadie me cuente
las sutiles artes que desplegaría su ingenio en aquella crítica ocasión
de su vida. Sin duda, viendo que su señor y el D. Vicente intimaban
mucho con los Padres del \emph{Colegio Romano}, con los
\emph{Observantes de Santa María de Araceli}, con las monjas de
\emph{Santa Clara en Quirinal}, elevó al grado máximo de intensidad sus
devociones, aficionándose al besuqueo de imágenes, aprendiéndose de
memoria trozos de literatura mística, con todo lo demás que creía
pertinente a la grande empresa de su redención. Resistíase Don Saturno a
dar su consentimiento, atento siempre al qué dirán probable, y temiendo
los escrúpulos de la familia más que los suyos propios. Pero D. Vicente
y otros clérigos que a la santa obra arrimaban el hombro, decíanle que
por encima de la familia estaba el deber, y por encima de la Sociedad,
Dios; que en Eufrasia eran infalibles las señales de arrepentimiento, y
que por fin, su protector o cortejo que con llama inextinguible la
amaba, debía santificar aquellos criminales lazos, y limpiar su
conciencia y la de ella en las aguas purísimas del Matrimonio.

Libre ya de pasiones y de juveniles devaneos, Eufrasia quería sobre
todas las cosas humanas una posición, y en ello puso las dotes
singulares de su espíritu. Como Dios, al fin y al cabo, protege a los
tenaces y agudos contra los romos y debilitados de voluntad, la manchega
vio colmadas sus ansias, y recibió franco pasaporte para el mundo moral.
En la española iglesia de Santiago (plaza Navona), no lejos del
esquinazo en que está la famosa efigie de \emph{Pasquino}, se casaron
Don Saturno y Eufrasia, precisamente en los días de mi segunda
\emph{villeggiattura} en Albano.\emph{¡O tempora, o mores!}
Naturalmente, la primera noticia del casorio levantó en la familia de
Madrid gran polvareda, y cuando el matrimonio llegó acá, manteniendo en
los primeros días una reserva parecida al incógnito, para sofocar hasta
los más leves rumores de escándalo, no faltaron disgustos, rozamientos,
y aun dicharachos ruines. Mas de todo ello fue triunfando poquito a poco
la diplomacia de la manchega, que con sus astutas carantoñas pudo atraer
uno tras otro a los enojados parientes, y hacerse querer de los que
antes la aborrecían. Doña Cristeta, que había sido la más intransigente,
olvidando su amistad con Doña Leandra, se rindió más pronto que ninguna
a la sutil táctica de la dama moruna, recibiendo de ésta cantidad de
preciosas reliquias, huesecillos de santos, acompañados del diploma que
acreditaba su autenticidad, y sinfín de rosarios, medallas, indulgencias
y demás cositas interesantes a los buenos corazones cristianos.

He referido sin ningún recelo lo que sé de la señora de Socobio,
juntando las noticias que me dio Navarrete con las que yo por directo
modo he sacado de la fuente histórica, y puedo escribirlo sin temor de
que mis indiscreciones lastimen a nadie, pues estas páginas quedarán
ocultas, y nadie ha de leerlas hasta que la señora y yo, y los demás que
me veo precisado a citar, hayamos entregado nuestros huesos a la madre
tierra.

\emph{30 de Marzo}.---¡Cómo está mi cabeza, señores! ¿Creerán que con la
golosina de estas vanas crónicas mujeriles se me ha olvidado escribir
que hace días tuvimos aquí una revolución? Ello fue de harta resonancia,
pero de resultado nulo, como obra de unos locos, cuyos nombres oí y ya
se me fueron de la memoria. Corren voces de que se repetirá: los
progresistas exaltados y los demócratas no descansan, ávidos de ocupar
las poltronas, y más que en los elementos revolucionarios de aquí,
confían en el apoyo que les darán los de Francia. La novísima República
establecida en aquel país tiene a nuestros moderados con el alma en un
hilo. Por mi parte, declaro que no me quitan el sueño las políticas
inquietudes, ni los problemas que, según dicen, señalarán el presente
año como uno de los más agitados del siglo, porque he decretado mi
absoluta independencia del organismo general, creando un sistemita
planetario para mi exclusivo uso, y de él no me sacan atracciones
públicas de ningún género. Y creed que no me interesa nada ya la
cuestión del Papado liberal, en la que puse tanta vehemencia y gasté
tanta saliva. Gioberti y Balbo en Italia, y aquí Balmes y Donoso Cortés,
valen para mí, con todas sus retóricas elocuentes, tanto como un comino,
y el buen Pío IX, a quien de veras quise y admiré, ya no me embarga el
ánimo con el supuesto carácter de pastor de los pueblos y patrono de la
regeneración itálica. Vivo ahora de mi propio jugo, y todas mis empresas
son absolutamente mías, principio y fin de mis ideas y sentimientos.
También digo que la Democracia que en forma de virgen en paños menores
se nos aparece salvando el Pirineo, me encuentra insensible a sus
encantos. Ya no me embelesan lecturas de Lamennais o Ledru Rollin, y me
resigno a que la humanidad se regenere sin mi auxilio: ya iré a verla
cuando esté regenerada, y a festejarla y aplaudirla. En tanto consagro
mis horas a proporcionarme todos los gustos posibles, eliminando
sinsabores y rehuyendo penas.

¿Queréis que os hable de los que para mí son capitales acontecimientos?
Pues sabed que de la noche a la mañana me vi trasladado a la Secretaría
de Gobernación con doce mil reales, sin que yo a ciencia cierta
entendiese de dónde me había caído breva tan sustanciosa, pues mi
hermano Agustín me declaró que no era cosa suya. En cambio, al pobre
Cuadrado se le contentó con la promesa de reponerle, y volvió el hombre
a mí afligidísimo, diciendo que ya se había proporcionado una pistola
para poner fin a sus días si no se le daba pronto la debida reparación.
Yo le consolé, y avivé sus esperanzas, socorriéndole de mi bolsillo para
que mantuviera con sopas o potajes a la extenuada familia, mientras el
remedio de su triste situación llegaba. Hablé nuevamente del caso a mi
hermana, y la oí condolerse del pobre cesante con el registro más
gangoso de su voz, para venir a parar en la negación de su influencia.
«¿Qué más quisiera yo que enjugar todas las lágrimas que veo derramar?
Pero, ¡ay!, no puedo hacer más que pedir a Dios que ilumine a los que
dispensan esta clase de favores, y Dios me oirá, Pepe, Dios me oirá: con
tanto fervor se lo pido.»

\hypertarget{xv}{%
\chapter{XV}\label{xv}}

\emph{1.º de Abril}.---Las confesiones de hoy son un poco amargas; pero
allá van para que todo, conducta y conciencia, quede guardado en el
archivo de estas hojas.

Cierto que mi ascenso a doce mil es un felicísimo suceso que cualquiera,
en caso normal, estimaría como don extraordinario de la Providencia, o
premio gordo de Lotería. Pero en mi caso, por distintos conceptos
irregular, ni los doce mil, ni el doble, si doble fuera mi estipendio,
me bastan para la vida que me doy y el pie de disipación en que me he
puesto. Ya se habrán maravillado los que leyeron las anteriores páginas
de cómo logro sostenerme en una sociedad tan superior a mis escasos
medios. Pero hasta hoy, lo digo sinceramente, no he caído en la cuenta
de que voy andando a ciegas por los caminos más arduos de la vida; y lo
peor es que no puedo retroceder, ni me siento con el suficiente brío de
voluntad para detenerme, porque me atraen metas muy seductoras, y corro
tras ideales muy lindos, que embriagan mi mente y adormecen mi razón.
Hablo con desnuda verdad de este desequilibrio en que se desliza mi
existencia, y afirmo que, aun hospedado y mantenido por mi hermano
Gregorio, con el sueldo no tiene mi agitada vida para empezar. Sin
contar más capítulo que el de ropa (y no sé dónde pararía si en otros
capítulos o renglones me metiera), digo que necesitaré dos años de
sueldo para pagar los trapos que en un solo mes he encargado a mi
sastre, cuyo elogio se hace con decir que es el más caro de esta Corte.
Incapaz de contener los estímulos de mi presunción, quiero surtirme de
toda la rica variedad de levitas y fracs impuesta por la moda. En
chalecos poseo maravillas, y París tiene poca inventiva para colmar mi
gusto. De corbatas no hablemos. En perfumería y accesorios de tocador no
me pongo tasa. Ahora, supla la fantasía del pío lector los innumerables
motivos de dispendio inherentes a este lujo de vestir.

Añado que mis hermanos me riñen; que se asusta Sofía, vaticinándome que
acabaré en un hospicio; que Gregorio pone el grito en el Cielo.
Únicamente mi cuñada Segismunda, la Medusa que tiene culebras por
cabellos, no extrema sus reparos, y aun se permite opinar con cierto
dejo sibilino que yo, lanzándome locamente por las trochas y
desfiladeros sociales, llegaré a los más envidiados puestos. El mundo,
según ella, es de los atrevidos, no de los cuitados; es de los que
corren, no de los que miden encogidamente sus pasos. Esta opinión me
consuela de los achuchones que me da mi familia cuando entra en casa
sombrero nuevo, antes de que su antecesor envejezca, o cuando a la
puerta llama el oficial del sastre con rimeros de ropa elegantísima.

Añado también, aunque me cueste alguna vergüenza el declararlo, que hace
dos meses me hizo probar mi amigo Aransis las emociones del juego, y que
desde el punto y hora en que de aquel fuerte licor gusté, ya no he
sabido vencer el anhelo de catarlo cada día, ya por la espera de una
ganancia que engorde mi flaco bolsillo, ya por la simple maña de hacerle
cosquillas a la fortuna, y ver si me sonríe placentera. No debo quejarme
del azar, que me ha sido propicio más de una vez permitiéndome dar
algunos toques a las apariencias de mi vida fastuosa. Sólo en los
últimos días me ha torcido el gesto la deidad voluble, y heme visto
obligado a contraer deudas, algunas muy enfadosas. Pero espero y busco
un glorioso desquite.

\emph{2 de Abril}.---Sepa la Posteridad, y sépalo con satisfacción, que
el desquite es un hecho. Mas no he podido sofocar el tumulto de mis
deudas, porque si algunas reduje o rematé, me han nacido otras por
inevitable exigencia de los compromisos sociales y de nuevas aventuras
que sin saber cómo me salen de debajo de las piedras. Me consuela el ver
que Aransis se halla en mayores apreturas, y en él son más aterradores
los efectos por ser de superior gravedad las causas, pues mantiene
caballos, disfruta coches, gasta bailarinas y figurantas del Circo, y se
mermite otras formas de opulencia propias de un aristócrata. Días
pasados, cuando después de hacerme horripilante descripción de sus
ahogos, me anunció mi amigo su propósito de levantar un empréstito,
echeme a temblar, y al temblor siguió sudor frío cuando me dijo que
nadie como mi hermano podría encargarse de ello. Fue para mí como un
tiro su indicación de que yo hablase a Gregorio\ldots{} ¡Ay!, mucho
quería yo a Guillermo, y por servirle y ayudarle aceptaría cualquier
sacrificio; pero que no pusiese en mis labios semejante cáliz. Atento a
mis razonadas excusas, y sin ofenderse por mi negativa, buscó entre sus
conocidos otros amañadores de tales negocios, y al fin (el lunes lo supe
casualmente) el empréstito de Guillermo ha venido a parar a mi casa. Hoy
me dijo Gregorio con punzante burla: «¡Vaya con tu amiguito! ¡Dios tenga
piedad de la casa de Aransis! Al paso que lleva ese mequetrefe, pronto
empeñará los pararrayos\ldots{} Y como los que le den dinero no cobrarán
hasta que muera su abuelita la señora Marquesa, que aún está de buen
año, entienda Don Guillermo que le harán pagar caro el plantón.»

Informado por Aransis, día por día, de la marcha de su asunto, supe que
tardaba en efectuarse más de lo que él quisiera\ldots{} Surgían temores,
dificultades; la cuantía del préstamo era objeto de meditaciones
aritméticas por parte de los que habían de aflojar la mosca\ldots{} En
mi casa, sin hacer la menor pregunta a mi hermano ni a los dependientes,
yo inquiría y olfateaba, con la mira de comunicar a Guillermo cuanto
pudiese averiguar. Pero nada en limpio saqué de la contemplación de
aquellas esfinges. Yo veía entrar y salir gente; pero iban a otros
degüellos: unos salían conformes, cadavéricos otros y con el mal de San
Vito. Oía yo el rechinar de dientes, y el estertor de las víctimas en el
momento agónico; pero nada pude pescar que a los intereses de mi amigo
se refiriese. Sin duda no se había encontrado el vampiro, y mi hermano y
otros andaban en su busca y descubrimiento. Por fin, en estas
oscuridades, vi aparecer súbitamente una luz, primero lívida, después
resplandeciente, y ello fue en el salón de la dama moruna.

Aprovechó Eufrasia un oportuno ratito para decirme: «¿No sabe usted nada
del empréstito de su amigo Aransis? Trabajillo ha costado a Gregorio
encontrar quien cargue con ese mochuelo; pero al fin veo que\ldots{}
vamos, que parecieron los cuartos\ldots{} No me pregunté quién los dará.
Ni lo sé ni se lo diría aunque lo supiera, que esas cosas son muy
reservadas. Lo que sí le digo y le ruego es que use usted de toda la
influencia que tiene con su amigo para irle quitando de la cabeza esa
vanidad estúpida, pues si no se enmienda, pronto dará en tierra con esa
casa, un día tan poderosa, hoy resquebrajada y tambaleándose como los
borrachos. Y todo lo que he dicho de Aransis, aplíqueselo usted, que
también va por malos caminos, y no tiene casa grande que devorar. Modere
usted a su amigo, y modérese a sí propio, si no quiere que yo le retire
mi amistad, y le deje solo y desamparado en el mundo.

Contestele agradecido, agregando la promesa de sermonear a Guillermo y
de sermonearme también a mí propio, aunque no era menester, porque ella
lo hacía ya con notoria eficacia. Y la dama siguió así: Hágase cargo de
lo que pasa en esta sociedad. La aristocracia, que no sabe administrar
su riqueza, ni cuidar sus fincas, se va quedando en los huesos. Toda la
carne viene a poder de los del estado llano, que cada día afilan más las
uñas, y acabarán por ser poderosos\ldots{} ¡Como que también están
afanando lo que fue de frailes y monjas!\ldots{} Claro que luego
volverán las aguas a su nivel; los que vivan mucho verán cómo se forma
una nueva aristocracia de la cepa de esos ricachos, y cómo recobrará el
clero lo suyo, no sé por qué medios, pero ello ha de ser. El mundo da
vueltas, y al cabo de cada una de ellas se encuentra donde antes estuvo.
Por esto digo yo que andando hacia adelante, andamos hacia atrás.»

Oíala yo encantado de su donaire. A más de los saludables consejos,
saqué en limpio de aquel coloquio dos cosas: la noticia de que es un
hecho la estrangulación de Aransis, y la casi certidumbre de que el
ejecutor es mi amigo D. Saturnino del Socobio, el cual no pierde ripio
cuando le cae un pájaro de esta calidad.

\emph{8 de Abril}.---Consagro la confesión de esta noche, oh amigos
venideros, al que se precia de serlo mío en la hora presente, el esposo
de Eufrasia, por ésta comúnmente llamado Saturno. Comprenderéis esta
preferencia cuando sepáis que anoche fue grandísimo estorbo para mi
palique con la señora, llevándome a un apartado sitio de la sala para
charlar conmigo\ldots{} Vean primero el hombre. Aunque no ha llegado a
los cincuenta, parece haber traspuesto esa línea, porque su naturaleza
viene arruinada, de años atrás, por achaques de que se defiende hoy con
un método riguroso impuesto por su mujer. De cuerpo menos que mediano,
escaso de carnes y de pelo, fatigoso en el habla, todo su ser se
condensa en la viveza de los ojos y en la movilidad de los brazos cuando
pone el paño al púlpito. Gasta bigote recortado y triangular, lo que más
le asemeja a Espartero que a Zumalacárregui, y unas cortas patillas cuyo
trazado le he visto variar en pocos meses. Viste bien; come con grandes
remilgos higiénicos, desechando hoy lo que ayer le gustaba; habla con
elocuencia reposada y construcción castiza; discurre con tino, en su
cuerda, esquivando la paradoja y la hipérbole; es en su trato cortés, en
todo lo social correctísimo. Gusta de la política, y creería faltar a un
deber profesional si no hiciera cada noche un resumen claro y juicioso,
a su modo, de los sucesos del día. Habla despacio, y es de los que se
escuchan. Conocedor yo de su debilidad por el éxito oratorio, pongo
exquisito cuidado en escucharle también con todo mi oído, ya que no con
toda mi alma. Oídle conmigo:

«¿Me preguntan si acepto el sistema parlamentario con todas sus
consecuencias? Lo acepto como ensayo, sin asegurar que pueda caber
dentro de ese molde la vida de la Nación. Es régimen de garantía siempre
que en él se diga: `fiscalicemos'. Pero es régimen de barullo cuando sea
preciso decir: `gobernemos'. Yo, ya lo saben todos mis amigos, no hago
un misterio de mi procedencia, ni reniego de mis antecedentes. Tengo a
gala el haber influido con Maroto para llevarle al convenio de Vergara.
Serví honradamente a D. Carlos\ldots{} fui bastante leal para decirle:
`Señor, esto es imposible\ldots{}'. El 38, cuando la Corte y el ejército
llegaron a las puertas de Madrid, tuve una fuerte agarrada con González
Moreno, en Arganda, y me separé del partido para siempre. Mis hermanos
luchaban en uno de los campos, yo en otro: vimos clara la inútil
inhumanidad de semejante lucha, nos abrazamos, y aquí estoy\ldots{} ¿No
convienen ustedes conmigo en que los tiempos cambian, y en que su variar
continuado trae la evolución?\ldots{} Pues la evolución es como la
conciencia de la sociedad. Yo evoluciono, luego existo.»

Mis noticias son que D. Saturno fue el representante de la familia en el
campo carlista, mientras otros acá la representaban, atentos al
recíproco auxilio, y a mirar por todos cuando Marte decidiera entre
Isabel y Carlos. Sé también que arrimado a los Socobios, que venían
mangoneando en Gracia y Justicia desde el tiempo de Calomarde, D.
Saturno aumentó considerablemente su peculio, gestionando asuntos
eclesiásticos. Heredó luego de su primera esposa un buen caudal.
Arregladísimo en todo, menos en un aspecto muy importante de la vida
humana, el hombre cuerdo y sesudo para los negocios y la política, para
las relaciones varias del organismo social, no era un modelo en la vida
doméstica, ni practicaba con rigor los buenos principios que rigen y
gobiernan las costumbres. Mutilaba y subvertía la ley moral, dejando a
salvo, con no poca sofistería, sus religiosas creencias. De él se ha
dicho que es un valiente campeón católico que ha reformado el Catecismo,
reduciendo a seis los pecados capitales.

Siguió diciendo: «¿Convienen ustedes conmigo en que es preciso
transigir, amoldarse a las circunstancias, a los hechos? Lo digo sin
rebozo. Yo acepto el parlamentarismo y el liberalismo siempre que se
encierren dentro de los límites de la mayor moderación, poniendo por
encima de todo el principio de autoridad y la fe religiosa. Sin estas
dos grandes columnas no hay edificio social que se mantenga en
pie\ldots{} Alguien me ha dicho que debiera yo predicar con el ejemplo
más que con la palabra: a eso respondo yo que no me tengo por hombre
impecable. Al contrario: pecador he sido, y pecador reincidente. Lo
reconozco, lo confieso. ¿Qué más quieren? Mi temperamento ha podido en
otros días más que mi razón\ldots{} Ésta y la edad me han traído la
enmienda. A muchos conozco y conocemos todos que no podrán decir lo
mismo. ¿Es verdad o no es verdad?»

Yo supe que a su definitiva enmienda le habían traído los alifafes más
que la razón. Padecía D. Saturno de sorderas periódicas, de inflamación
de los oídos, de irritaciones gástricas, de dolores en la osamenta,
gajes, ¡ay!, de sus formidables campañas\ldots{} Su última pasión fue la
hija de Don Bruno Carrasco, y si en ella gastó al principio lo que le
restaba de salud y padeció ansiedades y disgustos, luego Dios le deparó
en aquel mismo pecado su salvación, trayéndole por los trámites de ley a
la honrada paz que ahora disfruta. Adelante.

Tres amigos fumadores escuchábamos con benevolencia de estómagos
agradecidos las campanudas estolideces que Socobio nos endilgaba. Uno de
aquéllos, de traza muy respetable, aparecía por vez primera en la
tertulia, y desde que fui presentado a él por D. Saturno puso en mí toda
su atención. En los respiros que nos daba el orador (a quien afligían
ciertas intermitencias del resuello), el Sr.~de Emparán (que así se
llamaba aquel sujeto) mirábame con fijeza inquisitiva y me hacía
preguntas algo extrañas acerca de mis ocupaciones, de mis placeres, de
mis estudios\ldots{} ¡Estudios a mí! Aquel buen señor soñaba despierto:
era quizás de los que me tenían por sabio, y quería obtener informes
directos y personales de mi prodigiosa ciencia. Tentado estuve de
devolverle curiosidad por curiosidad, preguntándole a mi vez: «¿Y usted
quién es, en qué se ocupa? ¿A qué debo el honor de ese prolijo interés
por mi humilde persona? ¿Qué idea le mueve a querer penetrar en el
segundo fondo de mi existencia?» Pero mi buena crianza me libró de
cometer tal grosería con un señor que me triplicaba la edad, y que al
interrogarme disimulaba su impertinencia con la urbanidad más exquisita.
De pronto, una frase del investigador arrojó alguna claridad sobre la
confusión de mi mente. «Sr.~de Fajardo---dijo,---con su señora hermana,
Sor Catalina de los Desposorios, sostenemos mi familia y yo amistad
cariñosa, y aunque de tanto oír hablar de usted casi casi llegábamos a
conocerle como si le hubiéramos tratado, he querido yo tener este careo,
y no me pesa, no me pesa\ldots»

Acercose más a nosotros el dueño de la casa, y dándome palmaditas dijo a
su amigo: «Aquí le tiene usted, Sr.~D. Feliciano. Es buen chico, aunque
un poquillo desordenado y calavera. Pero, si bien se le mira, en su
fondo no hay malicia, y será lo que se quiera hacer de él.» ¡Y yo sin
comprender lo que oía, ni atreverme a pedir categóricas explicaciones!
Levantose el Sr.~de Emparán para despedirse, y después de ofrecerme su
casa y de rogarme que la honrara, me apretó la mano con fuerte sacudida,
diciéndome: «Su señora hermana me ha indicado esta tarde que desea verle
a usted pronto por allá\ldots{} No tarde, D. José, que, según yo pienso,
tiene que decirle alguna cosa\ldots{} que no es baladí; ciertamente no
es baladí.»

Al verle salir acompañado de Socobio, empecé a descifrar el enigma y
poco después lo vi completamente claro en los ojos negros de Eufrasia.

\hypertarget{xvi}{%
\chapter{XVI}\label{xvi}}

\emph{12 de Abril}.---Hace días deserté de la casa y reunión de D.
Saturno, prefiriendo las de su hermano D. Serafín. He querido probar el
juego desdeñoso, y no sé por qué pienso que ha de marrarme. Allá lo
veremos\ldots{} Continúan las dos chiquillas Virginia y Valeria
embelesándome con sus donaires, que ahora van trocando en agudísimas y a
veces mortificantes burlas. Con tal confianza me tratan ya que hasta me
tutean, sin que yo me atreva a rebajarles el tratamiento. Óigalas el que
esto lea: «¡Ay, Pepito, qué lástima te tenemos!\ldots{} Aunque ahora nos
veas reír, puedes creer que por ti hemos llorado\ldots» «Vaya, que no
tienen mal fin tus picardías\ldots{} Ya no más revoloteos, gavilancito.
Ahora te ponen una calza como a los pollos, y te meten en un corral con
las bardas muy altas, para que no puedas escabullirte\ldots» «Esas
bardas son la casa de los Emparanes. No te pongas afligido, Pepe, que la
novia que te han buscado es tan buena que no te la mereces. A talento
podrán ganarle otras, pero a hermosura no\ldots» «Tiene una nariz muy
mona, encorvadita sobre el labio como si quisiera averiguar lo que hay
dentro de la boca\ldots» «Y antes que ver los dientes, ve las encías. El
talle, eso sí, es tan bien torneadito como el globo terráqueo, y lo
mismo se redondea para los lados que de \ldots» «Su habla es graciosa,
sobre todo cuando tartamudea; pero esto no es todos los días, sino
cuando hay viento de Toledo\ldots» «Los ataques le suelen dar los días
en que se saca ánima.» «¡Ay, Pepito, qué feliz vas a ser, con una esposa
lánguida aunque no sin carnes, con una esposa que tendrás que mecer en
tus brazos cuando se te desmaye! Pero tú te harás la cuenta de que no la
cargas a ella, sino a sus talegas\ldots» «Anda, pícaro, y qué bien
rebozada en millones te la dan\ldots{} Tajadas como ésa no pasan de otro
modo.»

Yo me reía, queriendo seguir la broma. Oigan lo que les contesté: «¿Pero
qué desatinos están ustedes diciendo ahí? ¿Y qué novia es esa que no
conozco ni quiero conocer?\ldots{} Yo no me caso más que con ustedes,
con mis amiguitas Virginia y Valeria, con las dos, porque a las dos las
quiero por igual, y ellas a mí me quieren lo mismo la una que la
otra\ldots{} Con las dos, con las dos, que ahora se reformará la ley de
matrimonio, para que un hombre tenga dos mujeres.

---¡Ay, qué pillo, y qué poca vergüenza! ¡Vaya con las indecencias que
dice! ¡Casarse con dos!

---Con una ya es mucho apechugar, cuanto más con dos.

---¡Si es un pilluelo de la calle! Si pudiéramos, le clavaríamos cada
una un alfiler de los gordos para oírle chillar.

---Si le cogiéramos a solas, le daríamos una broma pesada: ofrecerle una
yema llena de polvos de escribir, o echarle tinta del tintero en la taza
de café.

---Nos vengaremos hablando pestes de él y sacándole los colores a la
cara, ya que no podamos sacarle los ojos.»

Río yo y me distraigo con estas burlas donosas; pero la procesión me
anda por dentro. Lo diré sin ninguna reserva: Eufrasia me trae loco,
respondiendo a mi juego de desdenes con una frialdad y displicencia que
revelan la perfección de su histrionismo. Anoche no pude cambiar con
ella dos palabras: diome con la puerta de su mal humor en los hocicos.
Ya ni amigo siquiera. Y esa dulce amistad me hace ahora más falta que
nunca, pues necesito consultar con la morisca dama puntos delicadísimos
de conducta y aun de conciencia. Su marido, en cambio, me asedia con
oficiosas amabilidades y una protección que me enfada sobremanera. Hoy
me le encontré en casa del general Fulgosio, y se permitió reñirme con
tonillo paternal. «Es muy extraño, Pepe---me dijo,---que no haya usted
visitado a los Emparanes\ldots{} ¿Apostamos a que tampoco ha ido usted a
ver a su señora hermana? Vaya pronto, que algo tendrá que decirle Sor
Catalina de los Desposorios; y luego prepárese a ir a vistas\ldots{}
Cada día que pasa está usted más en falta. Hoy me ha dicho mi esposa que
usted no sabe apreciar el bien que se le hace, y con ello viene a
demostrar que no lo merece.»

Contesté con lugares comunes, sin pedir mayor claridad, porque la
claridad en aquel asunto me causaba miedo, y llevé la conversación a la
política, buscando los efectos emolientes y narcóticos. D. Saturno me
dijo que si Narváez no mostraba más coraje, se le vendría encima todo el
Progreso avanzado, con los demócratas, que conspiran descaradamente,
protegidos por Bullwer, embajador de Inglaterra. «Yo no sé en qué está
pensando lord Palmerston, no lo sé, no lo sé\ldots» Yo tampoco sabía en
qué estaba pensando lord Palmerston, ni me importaba.

\emph{14 de Abril}.---Continúo indiferente a lo que piense o deje de
pensar lord Palmerston. Y eso que esta noche, en casa de Alvear, he sido
presentado a Bullwer, Ministro inglés, el cual no se ha cuidado de saber
lo que opino de su cacareado metimiento en los asuntos de España. Me ha
tomado por aristócrata, engañado de las apariencias, y de ello me huelgo
muy mucho. Mañana iré por primera vez a casa de Montijo con Aransis.
Anteanoche estuve en el Príncipe y vi dos actos de \emph{La Rueda de la
Fortuna}\ldots Yo esperaba verla allí; pero no fue: \emph{brillaba por
su ausencia}, como dice Ramón Navarrete. A medida que avanza la
estación, resplandece en los teatros de un modo fatídico el vacío de las
señoras ausentes\ldots{} He querido hacer una figura, y no me sale. Es
que estoy tonto; así quiero hacerlo constar aquí, dejando correr desde
la mente al papel el inagotable chorro de mis necedades; la tristeza que
me consume agrava mi tontería y la hace insufrible. Soy un necio
afligido y un fúnebre mentecato. Mas ahora caigo en que contra estado
tan lastimoso hay un remedio, que es la divina sinceridad, medicina
segura de las turbaciones del historiador. Salga, pues, de mi corazón
ese bálsamo, y váyanse al demonio todos los reparos y las sofisterías
del amor propio.

Sí, señores del venidero siglo: vedme restablecido en mi ser por la
eficacia de las verdades que a revelaros voy. Mis murrias provienen del
diferente y contrapuesto enfado que me causan dos hembras: la una,
después de negarme su amor, resignada o convencida, no lo sé, me retira
también la opaca dicha de su amistad; la otra se enciende en tan loca
pasión por mí, y de tal modo me asedia y mortifica, que llega, ¡vive
Dios!, a serme intolerable. Dos grandes anhelos llenan hoy mi alma: atar
lazos de amor con la una, desatarlos con la otra. ¿Y esta otra quién es?
Porque de ésta, si mal no recuerdo, no he dicho aún palabra, y ello ha
sido por haber clasificado el presente enredillo entre los de puro
pasatiempo, llamados a un facilísimo desenlace sin dejar rastro en la
vida. Pero en su breve curso tomó inopinadamente tal vuelo, y dio margen
a tales enojos míos, que es forzoso historiarlo\ldots{} Pero ¿quién es?,
dirán los señores y amigos cuando esto lean (ya habrá llovido para
entonces). Pues una mujer del pueblo, una \emph{demócrata}, que así debo
llamarla por ser de lo más selecto y fino dentro del tipo plebeyo.
Llámase Antoñita, y pertenece a una familia de cordoneros subdividida
hoy en diferentes tiendas y portales de calles próximas a la plaza
Mayor. Añado que es muy guapa y graciosa, el más delicado ejemplar de
manola que puede imaginarse, y que tiene por esposo a un tal Trujillo,
abominable truhán de Madrid, hijo de una honrada familia de comerciantes
en peletería, hoy apartado de sus padres y de su mujer, viviendo en
oscuros garitos y revolcándose en el más bajo cieno social.

Vino a mí la preciosa Antoñita por conquista de unas cuantas horas,
realizada con jactancia y perfidia. Bringas la cortejaba y la tenía por
suya; yo se la quité en los rápidos galanteos de una tarde. Cambió la
esclava de dueño como si con unas cuantas monedas la comprara yo en un
mercado de Oriente, y desde el primer instante se arrebató en tan loca
pasión, que el cansancio mío hubo de venir más pronto de lo que
pareciera justo, dadas la belleza y donaire de tal mujer. Era su amor
tan absorbente que no dejaba respiro, y de un egoísmo tan bárbaro que en
constante suplicio vivía por ella el objeto amado. Y no me han valido
las ganas y la determinación de ruptura, pues apenas me separaba, venía
la desolada mujer con tales asedios, persecuciones, súplicas y
lloriqueos, que de nuevo me dejaba encadenar, compadecido de aquella
violentísima fiebre, y de aquel amor inextinguible que para su defensa
se amparaba del cielo y la tierra.

Y en otro orden muy distinto (todo se ha de decir), llévame Antoñita con
el vértigo de su pasión a un cruel sacrificio, porque si ella no es en
verdad un juguete caro, y sabe practicar el \emph{contigo pan y
cebolla,} en torno de ella viene contra mi peculio su insaciable
familia, asediándome con brutalidad famélica. Un día es la pobre
abuelita; otro la hermana perlática; sigue el primo que tiene taller de
cordonería, y como padre de diez hijos se ve en fuertes apuros; arremete
también contra mí la tía, que está medio ciega, y anda tras de que la
operen; y por fin se presenta con infalible periodicidad el degradado
esposo, que al despertar de sus borracheras viene a cobrar el alquiler,
canon, peaje\ldots{} o no sé cómo llamarlo. Estos repetidos golpes y
socaliñas me traen a una situación pecuniaria de grande ahogo, porque no
sé negarme al gemido del pobre, y aun cedo a las cobranzas de Sotero
(que así se llama el vil marido), por evitar algún grave estropicio en
la persona de Antoñita.

Quiero zafarme y no puedo, porque para ello tendría que obsequiar
caballerosamente a toda la taifa postulante con una gorda suma de que no
dispongo. Entre tanto, mis recursos bajan, mis deudas crecen como la
espuma, y yo voy cayendo en sorda desesperación. Huyo de Antoñita, y
ella va tras de mí; me la encuentro en la puerta de la Embajada inglesa
cuando salgo, y su tétrica faz y ademanes de loca me infunden lástima; o
bien me escribe lacrimosas cartas despidiéndose hasta el valle de
\emph{Josafaz}. Vienen a contarme que la han sorprendido encerrada en
compañía de un braserillo de carbón, o tratando una pistola en casa del
armero\ldots{} En fin, no sigo, porque escribiendo esta desastrada
historia me pongo malo, y huye de mí la alegría de vivir, que ha sido en
días más venturosos mi sostén y mi encanto.

\emph{17 de abril}.---Esta noche os doy cuenta de un caso
extraordinario. ¿Cómo y por qué conductos ha llegado este romántico
sainete a conocimiento de la sin par Eufrasia? No lo sé, ni ella me lo
ha dicho al arrojarme a la cara el caso de \emph{Antoñita la cordonera}
con todos sus incidentes y perfiles. Pues sí: ayer, después de largo
paréntesis en nuestra amistad, hablamos largamente. Me la encontré en la
calle, saliendo ella de San Ginés en compañía de su amiga Rafaela
Milagro, ambas en pergenio de devociones, vestidas modestísimamente.
Ignoro si venían de confesar, o de alguna Novena o Manifiesto. Las
detuve un instante, y obtenido permiso para acompañarlas, fui con ellas
a la casa de Rafaela, esposa de mi amigo D. Federico Nieto, \emph{alias
Don Frenético}. La sequedad de la manchega, efectivo trasunto del hielo
de su alma para conmigo, o un acabado modelo de simulación, me llevó a
mayor abatimiento.

Hablamos extensamente delante de Rafaela, mejor dicho, habló la morisca
dama todo lo que quiso, y yo la oí, defendiéndome en breves conceptos de
las acusaciones que me dirigió, más en tono de maestro inflexible que de
amiga. Díjome que, sabedora de mis desvaríos, había decidido privarme
del apoyo de sus consejos; pero que a tal punto llegaban ya mis locuras,
que a salvarme acudía, no por mí, sino por mi madre y hermanos, pues ya
miraba próxima la catástrofe. Contome ce por be todo lo de Antonia y los
ataques de su hambrienta familia, y me preguntó si había yo perdido en
absoluto la vergüenza y el instinto de conservación. Como un pobre
estudiante, cogido en graves deslices, le contesté que no son las
rupturas de amores tan fáciles como ella supone, pues lo que en
conversación de personas indiferentes se tiene por muy hacedero, en la
realidad y en la situación particularísima de los interesados presenta
dificultades y peligros enormes. A esto dijo que ella me propondría un
plan de conducta enérgica, para conseguir en breve tiempo la liberación
que me devolvería el honor y la paz. A no estar presente Rafaela,
hubiérale espetado allí mismo nueva y más ardiente declaración de amor,
echándole la culpa de mis desastres por causa del abandono en que me
tiene.

Continuó la dama, en el resto del coloquio, tan frigorífica como en la
primera parte: ni una sola vez vi la sonrisa en sus labios, ni en su faz
morena el encendido tono que al acalorarse le daba singular encanto; sus
negros ojos parecían haberse impuesto la obligación de atenuar la mirada
con el amparo de sus admirables pestañas, y aquel rayo que herirme
solía, a mí llegaba sin la acerada punta, tibio y ceniciento. Desdeñosa,
siguió fustigándome: «Está usted de algún tiempo acá tan desatinado, que
sin darse cuenta de ello, comete las mayores inconveniencias. Al demonio
se le ocurre dejar en casa, para que yo las lea, esas novelas de los
Balzaques, Suéses y Souliéses. Pero ¿está usted soñando? Ya creo haberle
dicho que aunque traje de Roma licencia para leer obras prohibidas, no
quiero hacer uso de ella, por conformarme con los gustos de mi esposo y
no chocar con su familia y amigos. Yo no leo nada de eso, Pepe, bien lo
sabe usted, pues en una casa como la mía no pueden entrar libros
estimados como peligrosos por la moral depravada que encierran.»

Tomando en este punto la palabra Rafaelita \emph{la Frenética}, que
hasta entonces poco menos que muda había permanecido, me dijo: «Yo no
tengo licencia; pero si la tuviese, tampoco la usaría, porque de esos
libros no se saca más que barullo en la cabeza y cosquillas en el
corazón\ldots{} Cuando una llega a cierta edad y ha encontrado su oasis,
buena tonta sería si no se sentara a la sombra de las palmeras de Dios,
esperando allí a ver pasar las caravanas\ldots{} A mí me gusta ver pasar
las caravanas, y me alegro de no ir en ellas.

---¡Dichosa usted que tiene oasis!---le respondí.---Dígame dónde está el
mío, si lo sabe, para irme corriendo a él.

---El oasis de usted yo sé dónde está---me dijo Eufrasia,---y usted
también lo sabe; sólo que, como es un pillo, se hace el distraído y el
desmemoriado, porque le gusta más andar por el desierto, de Zeca en
Meca\ldots{} comiendo dátiles podridos\ldots{}

\hypertarget{xvii}{%
\chapter{XVII}\label{xvii}}

\emph{18 de Abril}.---Muerto de sueño, no pude terminar anoche la
sustanciosa conversación que tuvimos Eufrasia y yo en casa de Rafaela
Milagro. Sigo en el punto en que la dejé, o sea, en lo de que yo me
alimentaba con dátiles podridos a mi paso por el desierto. Nada quise
responder sobre aquella supuesta putrefacción del fruto de las palmeras,
y abordé valeroso el tema que mi amiga me proponía para seguir
peleándonos. Precisamente, allí quería yo verla y escuchar lo que
pensaba de un problema de mi vida sobre el cual no había querido darme
opinión. No he necesitado decir que en la famosa noche de mi
conocimiento con Emparán, relacioné las enigmáticas preguntas que éste
me hizo con el plan casamentero de mi hermana sin consultar para nada mi
voluntad, como si yo fuera un objeto insensible y de poco precio, que se
regala, o de mucho precio y que se vende. Más sorprendido que indignado,
y mirando por el lado de las burlas aquel mercantilismo matrimonial,
corrí a llevarle el cuento a Eufrasia. Al punto advertí en sus ojos una
gran estupefacción, después un rayo de cólera que me colmó de alegría.
Sus palabras, pasada la impresión primera, y echados rápidamente los
frenos del disimulo, no correspondieron al lenguaje anímico de los ojos.

---Está usted divertido---me dijo echándose a reír.---Quieren llevarle
al matrimonio como se lleva al colegio un chiquillo mal criado para
domarle. Es usted un ángel de Dios, Pepe. Deben de conocerle bien los
que así disponen de su corazón y de su mano. Veo que usted lo toma a
broma, y ello prueba una pachorra\ldots{} tan fenomenal\ldots{} vamos,
que si la pachorra fuera motivo de canonización, ya estaría usted
subidito en los altares con una vela a cada lado\ldots» Por más que en
mi respuesta me mostré irritadísimo ante aquel menosprecio que se hacía
de mi voluntad, no logré que cambiara el tonillo sarcástico por otro más
conforme con mis sentimientos. Repitió las burlas, llevándolas hasta un
extremo que jamás vi en ella, y desde aquella noche levantó delante de
mí el muro de hielo que mis atenciones y mi cariño no han podido
escalar, ni menos romper, como he consignado en las confesiones de los
últimos días.

Y ahora, planteada de nuevo la cuestión, le digo: «Estoy esperando,
amiga mía, su pensamiento acerca de eso que llama mi oasis.» Y ella, más
glacial que nunca: En otra ocasión pude reírme de que le quisieran a
usted\ldots{} para mejorar la yeguada de los Emparanes. Ahora,
conociéndole mejor, veo que los que así disponen de usted, saben lo que
se hacen. Y estará loco si no cierra los ojos y se presta\ldots{} al
cruzamiento\ldots{} antes hoy que mañana. Si así no lo hace usted, está
perdido. Nada, Pepe: ahora mismo escriba usted a Catalina dándole prisa
para que lo arregle todo prontito. Le ha venido Dios a ver con esa boda.
Ni usted merece más, ni podría esperar solución más acertada de los
conflictos de su existencia\ldots{} Más le digo: ¿quiere usted que
volvamos a ser amigos?

---Es mi mayor anhelo.

---Pues vaya, Pepe, vaya pronto a esas vistas que le proponía mi marido;
vea y examine el bien que Dios le ha deparado, y cuando usted salga de
aquella casa, comuníqueme sus impresiones\ldots{} Entonces, cuando yo me
entere del estado de su ánimo, le indicaré la forma y manera más dignas
de dar ese sí que tanto se desea.

---Déjese usted de síes y noes, que no tienen sentido común. ¿Será usted
mi amiga, me aconsejará?

---Aconsejándole estoy ahora.

---¿De modo que usted cree\ldots?

---Ya lo he dicho: cierre usted los ojos\ldots{} y ¡adentro!

---Como quien toma una medicina muy amarga.

---Exactamente---dijo tapándose la boca con el libro de rezos para
ocultarme su risa.

Creí observar que el muro de hielo, con aquel reír gracioso, se
agrietaba; pero ella prontamente acudió a repararlo, revistiéndose de
gravedad severa\ldots{} Entraron otras dos señoras que también de la
iglesia venían; tras ellas un sacerdote\ldots{} Eufrasia me indicó que
debía retirarme, y así lo hice, desdoblando lentamente, en el descenso
por los escalones, mis inquietudes y tristezas.

\emph{29 de Abril}.---Tanto tengo que referir esta noche que no sé por
dónde empezar. Con las fatigas de estos días y la tardanza en recogerme
(que casi con las primeras luces de la mañana entro en mi casa), me han
faltado tiempo y gusto para escribir. Procuraré ganar lo perdido, y
presentaros con el posible método y precisión los acontecimientos de
este capítulo de mi historia. Lo primero que debo decir es que Sotero
Trujillo, marido de Antonia, se personó un día en mi casa, proponiéndome
un negocio, en el cual me daría participación si yo le anticipaba la
cantidad necesaria para plantearlo. ¡Vaya una minita que era el tal
negocio! Con él se ganaría el dinero a espuertas\ldots{} \emph{Tocante
al secreto}, a nadie lo revelaría. Fue mi única contestación agarrarle
por un brazo y llevármele como a rastras hacia la puerta. Ya fuera de
ella, quiso el hombre revolverse contra mí; pero mi fuerza nerviosa, que
a falta de la muscular me asiste en casos tales, pudo más que su
impetuosa rabia\ldots{} De un empujón bajó Sotero rodando el primer
tramo de la escalera. Sangraba por frente y narices, escupía
maldiciones, y si no intervienen los porteros que al escándalo
acudieron, sigo tras él y le lanzo a nueva carrera por el segundo tramo.
Hacia la calle le precipitaron los porteros y un polizonte, y no volví a
saber de él en todo el día. Mi hermano y Segismunda me riñeron por el
escándalo, echándome en cara que a casa llegasen tan ignominiosas
visitas, por la desigual vida que yo llevo entre las personas más altas
y las más bajas.

Siguió a este ruidoso acontecimiento, en la serie de aquel día, otro no
inferior en importancia, pero sumamente grato para mí; y fue que aquella
tarde, hallándome, diré que por casualidad, en mi oficina (a la cual yo
no voy más que a fumar cigarrillos y a escribir mi correspondencia),
tuve el honor de ser llamado por Sartorius a su despacho, y recibido por
él con delicada llaneza. Su Excelencia había oído hablar de mí y deseaba
conocerme. La rápida lectura de las primeras hojas de un manuscrito mío
le había revelado disposiciones literarias no comunes, y como protector
de las letras y de sus cultivadores, me incitaba bondadosamente a poner
más atención en los trs de pluma que en el tumulto de la vida social
elegante. Debía yo, pues, probar fortuna en el Teatro o en la Novela,
género muy desmedrado entonces en España, y mejor aún en la historia
nutrida y amena. Nos hacen mucha falta, según Sartorius, buenos
escritores que aprendan y cultiven el arte de la amenidad, y nos libren
de esas investigaciones pesadas y macizas sobre cosas de la Edad Media,
que no hay cristiano que las trague; y convendría también que los de
literatura entretenida abandonasen la cuerda sentimental, que ya
empalaga, reanudando la tradición de la prosa humorística, española,
expresión de la vida real\ldots{}

Representa Sartorius cuarenta años; es de buena presencia, el rostro
expresivo, el bigote corto y rubio, la mirada sagaz, modales y
conversación de exquisita urbanidad. En él veo un raro ejemplo de
aristócratas espontáneos, como yo, es decir, hombres que, sin haber
nacido en dorada cuna, parecen destinados por Dios a ser fundamento de
la nueva nobleza que ha de levantarse sobre las ruinas de la
antigua\ldots{} Terminó Su Excelencia con una indicación que fue signo
de interés y simpatía por mi humilde persona. «A usted---me dijo,---le
convendría entrar en la carrera diplomática, para la cual parece
cortado, no sólo por su ilustración y su conocimiento de lenguas
extranjeras, sino por su buena figura. Podría ir de agregado a París o a
Roma, y en ello habría para usted dos ventajas: la de abrirse una
brillante carrera, y la de ausentarse de Madrid\ldots{} que no le vendrá
a usted mal, según entiendo.» Comprendí que mi hermano Agustín había
sugerido al señor Ministro la idea de echarme de aquí, como el único
medio práctico de cortar de un tajo los innumerables enredos en que
aprisionada está mi pobre existencia. Agradeciendo la noble intención,
me despedí, no sin protestar en mi interior del destierro que me
preparaban, pues la vida esta en que sufrimientos y goces se confunden
con dramático enlace, me cautiva, me embriaga, y como los borrachos, amo
el licor que endulza y alegra mis horas.

Observando un puntual orden cronológico, refiero que aquella noche fui a
la tertulia de Montijo. Nada de extraordinario me ocurrió en el palacio
de la plaza del Ángel, pues no lo fue que la menor de las hijas de la
Condesa, Eugenia, lindísima criatura, de una belleza espiritual cuando
está seria, picaresca cuando ríe (y no escasea la risa), me dijese que,
pues yo poseía el italiano, habláramos un ratito en esta lengua, que
ella con mucho gusto estudia\ldots{} Gramaticalmente la domina ya, y
desea soltarse\ldots{} Hablamos, no tanto como yo quisiera, y pude
recrearme en la gracia, en el ingenio y donosura de esta sin par mujer;
pero de mis ejercicios italianos hubieron de arrebatármela, con los
españoles, Bermúdez de Castro y Roca de Togores, que andaban locos tras
ella, pretendientes más tenaces cuanto menos favorecidos. Eugenia se
divierte con ellos, como con otros, como conmigo, y a todos da cuerda,
mas no esperanzas\ldots{} En el rincón de los políticos presencié una
viva disputa entre Borrego y Salamanca, del cual se dice que ha vuelto
la espalda a Narváez y a la misma Reina, lastimado del alevoso puntapié
que aplicaron a su Ministerio, no más sólido que una estatua de
escayola, como todo figurón que no tiene por ánima, dentro del yeso, una
espada formidable. Aburriome la disputa, en la cual no se oían más que
los comunes tópicos, y me fui al olor de las damas, que no pocas allí
había de mi conocimiento, y algunas a quienes yo solía cortejar con la
audacia propia de galanes españoles, maestros de dar formas finísimas a
la grosería. Detúveme un mediano rato con la de Torrefirme, casi
cuarentona, que me mostraba singular deferencia ya tocante en la
inclinación, y como advirtiese yo aquella noche que la caída hacia mí se
acentuaba locamente, excediendo en desnivel a la torre de Pisa, miné y
destruí su cimiento todo lo que pude para que se derrumbase pronto, como
en efecto\ldots{} Pero de esto no debo decir más ahora.

Esclavo de la escrupulosa cronología, digo que a la mañana siguiente me
despabilaron dos visitas harto funestas: el pobre Cuadrado, que iba al
olor de socorros y esperanzas, y la prima de Antonia, prendera, que me
dio la noticia de hallarse ésta enferma y poco menos que moribunda. Para
recibir y contentar a los dos visitantes derroché tesoros de filosofía.
Ni sorpresa ni alarma me causaron los suspiros y lamentos con que la
prendera me llevó su embajada, porque ya estaba yo hecho a noticiones de
aquel calibre y a las actitudes sentimentales; no obstante, sentí
lástima de la cordonera, a quien no había visto en luengos días, y
sospeché que padeciese hambre o que le dieran tormento los infinitos
diablos que componen su familia. Con promesa de pasar por allá despedí a
la llorona mensajera; a Cuadrado le di todo el contenido de mi bolsa,
que no era mucho, y por consuelo le dije que ya había hablado de su
asunto con el propio Ministro. Esto no era verdad, porque en mi
entrevista con Sartorius, de todo me acordé menos del infeliz cesante;
pero al soltarle la fábula, hice mental propósito de enmendar pronto mi
negligencia: «Váyase tranquilo, amigo Cuadrado, que no pasará esta
semana sin que usted reciba la reposición: corre de mi cuenta.» Y él:
«Como no se den prisa, puede que antes de reponerme esté todo el
Gobierno en medio del arroyo. Oiga usted a los progresistas y
entérese\ldots{} Cuentan con la Inglaterra, que ha mandado ya para acá
sin fin de cajas llenas de libras esterlinas\ldots{}

---¿Usted las ha visto?

---¿Cómo he de verlas, si todavía no han llegado? Ahora vienen por la
travesía de mar. Pero vendrán, y las veremos todos, que buena falta nos
hace. Esto está perdido; en Castilla y Extremadura habrá mala cosecha, y
como siga el \emph{espadón}, tendremos hambre pública\ldots{} Pues digo:
cuentan con la Inglaterra; cuentan con diez o doce batallones\ldots{} ya
comprometidos, y cuentan con gente de mucho dinero, que no tengo por qué
nombrar.»

Preguntele si conspiraba, y con viva efusión, iluminado el rostro por
llamaradas de alegría, me contestó que sí. Conspiraba porque se lo pedía
el cuerpo, porque el conspirar era olvido de las penas, venganza de la
injusticia y fuente de risueñas esperanzas; conspiraba también por
patriotismo, para que la Nación saliera pronto de tantas desventuras.
Como no tenía ocupaciones de oficina ni de nada, se pasaba el día
charlando de la conspiración con sus amigos viejos, o con los nuevos que
en el campo democrático le habían salido. El rincón de un café, el
cuchitril de una portería o las negras estancias de una mala imprenta
eran sus logias, y cuando no se terciaba el arrimo a cualquier tertulia
revolucionaria, satisfacía su anhelo en los corrillos de la Puerta del
Sol, conventículo habitual de cesantes. Díjome que si sus hijos fueran
mayores, a todos pondría un trabuquito en la mano para defender la
primera barricada que se levante. Él y otro amigo no menos enamorado de
las trifulcas, y que con ellas soñaba dormido y despierto, habían
recorrido todo Madrid, barrio por barrio, estudiando sobre el terreno
los puntos más estratégicos para emplazar barricadas, con el menor
riesgo de sus defensores y mayor desamparo de la tropa que tomarlas
quisiese. Las enfilaciones de las calles, la orientación de los
edificios, todito lo tenían bien observado, medido y presupuesto para el
caso muy próximo de \emph{dar el grito} contra Narváez. No era puro
platonismo y \emph{ojalatería}, que también, según dio a entender,
andaba en pasos de \emph{pronunciar} a cabos y sargentos, sirviendo de
auxiliar a otros dos, ya muy duchos en este arte. Despedile al fin,
incitándole a perseverar en su tr, pues aunque yo creía firmemente que
se le repondría, debíamos prepararnos para toda contingencia
desfavorable, y si la gran injusticia no se remediaba, echar a rodar
todo lo rodable, Gobierno, Constitución y el Trono mismo.

Comí con presteza y me eché a la calle, movido de la absoluta precisión
de buscar dinero, pues el cesante había limpiado mis bolsillos: visité a
tres usureros, arreglándome al fin con el más cruel y de más arrebatada
fantasía para elevar hasta el cielo los intereses y remontar mis deudas.
Me reparé de mi necesidad, y aunque me acosaban tristes presentimientos
del abismo a que yo corría, bien pronto el torbellino vital, el
encadenamiento rápido de las obligaciones con los goces, y de los
apetitos con los nuevos deseos, las ambiciones soñadas sucediendo a las
satisfechas, me volvían al normal abandono y a no pensar más que en el
momento presente\ldots{} Sigo contando.

Con dinero fresco, corrí a casa de Antonia, un piso tercero en la Plaza
Mayor, y mi sorpresa fue terrible ante el desastre que mis atónitos ojos
contemplaron al entrar en la estancia. Trujillo, según me explicó la
prendera, allí presente, había cargado con todos los muebles para
empeñarlos o venderlos, no perdonando más que la cama en que Antonia
yacía con altísima fiebre y angustias del ánimo, que se disiparon al
verme. El miserable se había llevado hasta los clavos, haciendo tabla
rasa de cortinas y alfombras, con lo que la casa se había convertido en
nevera. Nada de esto me había dicho en mi casa Margarita, que así se
llama la prima de Antonia, porque lo ignoraba: el villano despojo fue
perpetrado de once a una por el Sotero y dos compinches. Mientras acudía
yo a reanimar con palabras afectuosas a la pobre Antoñita, hizo la otra
una visita de inspección a los aposentos interiores, y volvió con las
manos en la cabeza, diciendo: «Han afanado también toda tu ropa, hija,
no dejándote mas que cuatro pingos. ¡Habrá infames, habrá
trastos!\ldots{} En la salita queda un espejo chico, el lavabo viejo y
unas mantas y almohadas\ldots{} ¡Jesús, Jesús!\ldots»

Sonriendo, y sin quitar de mí sus ojos, nos contó la enferma que al
partir Sotero con el ajuar de la casa le dijo: «Ahí quedan unas cosas.
No vayas a creer que te las dejo. Volveré por ellas esta tarde.»

Yo no tenía más arma que un bastón de estoque. Ya estaba yo viendo el
hierro traspasando de parte a parte al ladrón si volvía mientras yo
estuviese allí\ldots{} Pero no había que perder el tiempo en quejas y
apóstrofes vanos, pues Antoñita necesitaba con premura cuidados,
alimento, medicinas. Llamada por la prendera una chiquilla de la
vecindad que todos conocíamos, muy amable y vivaracha, de nombre
Encarna, empezamos a reparar el gran desavío causado por aquellos
bergantes, y acudiendo algunas vecinas, entró en la casa lo más
necesario en aquel conflicto: caldo, pan, agua caliente, carbón, leche,
velas\ldots{} Bajando y subiendo Encarna con ratonil prontitud las
escaleras, trajo de las tiendas próximas todo lo que el dinero podía
facilitar de momento; y al ver el trajín que unos y otros traían,
Antoñita reía y daba palmadas, no sé si delirando, o por efecto del
extremado gozo que mi presencia le causaba, el cual parecía tener virtud
bastante para sofocar las mayores tribulaciones. Procuré hacer a su lado
la calma: dispuse que todo el mujerío se retirase al comedor y cocina,
di a Margarita cuanto necesitaba para completar la provisión de lo más
indispensable, ordené que fuese llamado un médico, y quedeme solo con la
infeliz mujer, arropándola cuidadosamente para que no se enfriara, y
sosegando su ánimo con dulces acentos de amistad y compasión. Pero si
logré que guardara bien los brazos bajo el rebozo, no pude poner freno a
su desmandada locuacidad.

\hypertarget{xviii}{%
\chapter{XVIII}\label{xviii}}

«¿Qué me importa que ese gandumbas indecente me haya llevado todo lo que
había en casa, trebejos, trapos y chirimbolos, si te tengo a ti? Por la
puerta por donde salieron los trastos entraste tú. Bendita sea la
puerta. Estoy contenta; no me cambio por nadie\ldots{} Bueno. Que Sotero
se ha llevado lo que no era suyo, vaya bendito de Dios\ldots{} pero si
da en robarme también a mí, que soy lo menos suyo de todo lo que había
en la casa\ldots{} ¡ay!, si da en robarme, entonces sí que la hacemos
buena\ldots{} ja, ja\ldots{} No, \emph{Chinito}, no me digas que me
calle después de haber estado quince días o quince siglos sin
verte\ldots{} No, no: todo el palabreo que se me ha quedado dentro de la
boca en tantos días, ahora tiene que salir\ldots{} ¡Si estoy hablando
como una fuente! ¿Callarme yo? Aunque quisiera, \emph{Chinito}, no
podría. Déjame que despotrique, y si me sube la calentura, que suba
hasta el Cielo, y si por hablar he de morirme, muérame con la última
palabra cogida en la boca como un cigarro puro\ldots{} Ayer dije entre
mí: `Voy a figurar que estoy mala, para forzarle a venir. Me meteré en
la cama, y estaré un día sin comer para ponerme languiducha, y tomaré la
yerba que dicen enciende calentura, para que el timo sea completo.' Esto
pensé, y de tanto pensarlo, \emph{Chinito}, caí mala de verdad\ldots{}
Ayer vino Sotero y me contó que le habías echado por las
escaleras\ldots{} Hiciste mal en incomodarte, pues todo el negocio, es
un suponer, no llevaba otra malicia que sacarte doce o catorce reales; y
se habría ido tan contento\ldots{} En venganza de ti y de mí, porque
ayer le dije: `apártate, asqueroso, que tu olor a vinazo me tumba', ha
venido hoy con dos granujas para desvalijarme\ldots{} Es un pillo, un
borracho, un gandul y un sinvergüenza; y si yo no le aborrezco todo lo
que debiera, ¿sabes por qué es? Porque sé que te quiere\ldots{} No, no
te rías. Sotero te quiere. Me ha dicho que eres bueno, y que si alguien
te tocara al pelo de la ropa, ya se vería con él\ldots{} No es
vengativo, ni picajoso; casi, casi es un poquito noble\ldots{} no te
rías. Como te le encuentres por ahí, no le temas, que nunca fue
traicionero. Le sueltas un duro, y verás qué contento se pone\ldots{}

»No callo, no me da la gana de callarme\ldots{} Yo creo que estoy mala
por el aquel de tantos días sin hablar, y es que se me han podrido
dentro las palabras, y de la pudrición del vocablo ha venido esta
calentura\ldots{} Pues no me callo, que parloteando se me despeja la
cabeza\ldots{} Vuelvo a decirte, \emph{Chinito}, que no me importa que
Sotero se haya llevado los ajuares. Déjale que se remedie el pobre, y
que mire por su vicio. ¿Y para qué quiero yo muebles, para qué nos hacen
falta sillas, cómodas ni espejos, si ahora nos vamos tú y yo a vivir a
un monte? Tú me lo has dicho cuando entraste a verme, y ya no puedes
volverte atrás\ldots{} ¿Que no me lo has dicho? ¡Ay, qué
mentiroso!\ldots{} No te hago caso. Quieres divertirte conmigo. Sí que
me lo dijiste\ldots{} Nos vamos a un monte\ldots{} y pronto, mañana por
la mañana, y viviremos en una choza, solitos\ldots{} Ni tú verás más
mujer que yo, ni yo veré más hombre que tú\ldots{} Y que nos entren
moscas. Nos vestiremos a lo salvaje con unos pedazos de pellejos en
donde sea más preciso, y no tendremos que averiguar lo que es moda y lo
que no es moda para vestirnos\ldots{} Mira \emph{Chinito}: no te vuelvas
atrás, que me lo has dicho\ldots{} tú me lo has dicho, y yo te pregunté
que cuándo nos íbamos, y me respondiste que mañana\ldots{} Bueno: pues
ya que estamos conformes, sigo\ldots{} Para nada necesitamos allí mesas
de noche ni mesas de día, ni más batería de cocina que unos
pucheritos\ldots{} Sobre tres piedras pondré yo la olla con que guisaré
nuestra comida. Iremos juntos a recoger leña, y cuando nos paseemos por
el monte, no veremos más que algún conejo que pasa, y las maricas que
volarán delante de nosotros, las abubillas, y algún lagarto que nos mire
y se escurra\ldots{} Pero no veremos lo que acá llaman personas, ni
señores con frac, ni mujeres con zapatitos\ldots{} Yo iré descalza y tú
también, luciendo la bella patita, y por sombrero nuestras greñas, que
nos peinaremos, yo a ti, tú a mí\ldots{} ¡Cuánto me alegro de que Sotero
se haya llevado los trastajos!\ldots{} Así no veré más la cómoda, ni el
lavabo, ni las rinconeras\ldots{} Allá, nuestra choza, con paredes de
piedra y techo de paja, es más bonita que todos estos cuartos segundos y
terceros con entresuelo, y tantísima escalera que bajar y subir\ldots{}
Y nuestra choza no tendrá campanilla para llamar cuando entremos, porque
visitas no habrá más que la de alguna comadreja, o quizás de algún
galápago que entre despacito sin dar los buenos días. ¡Ay qué felicidad!
Yo contigo, sin ver gente, sin tener celos de marquesas y
condesas\ldots{} Porque allá, ¿qué marquesas ha de haber? Ninguna:
¿verdad que no habrá ninguna?\ldots{} Tampoco tendremos allá papeles
públicos, ni libros, ni nada de eso, y así no se quemará mi
\emph{gitano} las cejas averiguando lo que piensan en Francia, o lo que
guisan en Constantinopla\ldots{} Y con esta vida, ¿cuánto viviremos? Yo
creo que doscientos años, es un suponer, y nos moriremos el mismo día y
a la misma hora: ¿verdad, \emph{Chinito mío}? Y en esos doscientos o más
años no nos aburriremos ni un solo ratito, porque tú mirarás mis ojos
verdes, yo los tuyos garzos\ldots{} ¡ay, qué bonitos!\ldots{} y cuando
se nos abran goteras en el tejado, tú subirás a componerlo mientras yo
lavo nuestros camisolines y nuestras pieles en el arroyo que va
corriendo al pie de la cabaña\ldots{} Y otra cosa: allí, lejos de este
mundo maldito, \emph{desaprenderemos} todo lo que hemos aprendido, para
que se nos olvide hasta el nombre de tanta miseria y tanta
porquería\ldots{} Y hasta el alma hemos de cambiar, sacando de nuestras
cabezas un habla nueva, de poquitas palabras, lo preciso para decir
cuánto nos queremos, y nombrar las tres o cuatro cosas que usamos; y esa
habla pienso yo que ha de ser a modo de poesía, o al modo de
música\ldots{} ¿verdad, \emph{gitano}, que tendrá cancamurria de canción
o de verso?\ldots»

Esta charla delirante, a la que ningún freno podían poner mis cariñosas
incitaciones a la quietud y al mutismo, fue interrumpida por el médico,
desconocido para mí, hombre tan pequeño que mis ojos turbados le vieron
liliputiense, que no levantaba una cuarta del suelo. Era un viejecillo
de acicalado rostro, el bigote a lo Espartero, pintado; su sonrisa
mostraba una mala dentadura postiza; su cabeza forrada en un peluquín
negro tirando a rucio; capita corta; las manos con guantes, de cuyos
dedos sobraba la mitad. Suelo yo incurrir en la alucinación de que la
realidad no engendra el arte, sino el arte la realidad. Vi en aquel
mediquillo un ser creado por el prodigioso dibujante Alenza.

Con amable ademán, que inspiraba confianza, examinó a la enferma,
interrogándonos sobre la iniciación de su malestar. Dio mejores
explicaciones que yo la prima de Antonia, parroquiana antigua del
doctorcillo, el cual era especialista en partos, y muy acreditado como
tal entre las vecinas de aquel barrio. Ya llevaba Antonia cuatro días de
indisposición, cayendo y levantándose. No se recataba del frío, y sin
comer, ardiente su cabeza del cavilar continuo, lanzábase a la calle,
ansiosa de buscarme las vueltas y de salirme al encuentro. Comía tarde
alimentos fríos, indigestos; dormía de día, velaba de noche\ldots{} Con
el pecho al aire poníase a lavar la ropa en la cocina, frente a una
ventana por donde entraba todo el frío que arroja sobre Madrid el
Guadarrama\ldots{} Total, que había cogido un dolor de costado, o un
pasmo de todo el órgano de la respiración\ldots{} Hecho el examen de
pulso y lengua, nos dijo el doctor que era pronto para precisar el mal;
mas por el momento había que poner a la enferma un vejigatorio en el
vacío izquierdo, y arroparla y cuidar de que conservase el calor. Recetó
una pócima que se le daría en determinados espacios de tiempo, y se
despidió hasta la siguiente mañana. Acariciando las mejillas de Antonia,
le dijo que por picaruela se veía en aquel mal paso; que a los hombres
hay que dejarlos, y no correr tras ellos, pues mejor sistema que
perseguirlos es hacerles rabiar huyendo de ellos; que él tenía de estas
cosas no poca experiencia por haber sido muy galanteador y pizpireto en
sus mocedades, y que también le habían perseguido casadas y aun
doncellas; añadió luego que él tenía muy buena mano para las enfermas
bonitas, y no se le moría ninguna, ninguna, siempre que hicieran con
gracia y paciencia lo que él mandaba, y durante la enfermedad pensaran
en el médico antes que en los novios o querindangos que las traían a mal
traer\ldots{} Al despedirse de mí en la puerta díjome que el mal parecía
de cuidado, y que se presentaba con cariz de pulmonía del
izquierdo\ldots{} Al siguiente día nos lo diría claramente. Salió, y al
verle yo coronar su cabeza con el desmedido sombrero que usaba, adquirió
proporciones humanas su menguada estatura.

Después de la visita del médico, advertimos en Antonia sedación y
tranquilidad. Hablaba menos y se conformaba con la prisión entre las
sábanas, con tal que la dejara yo tener una de mis manos entre las
suyas. A media noche, viéndola dormida, resolví marcharme, pues aquella
mi larga ausencia de los amigos y de mis entretenimientos nocturnos ya
pesaba en mi ánimo. Prometí a Margarita que antes de retirarme a mi casa
volvería, y allí se quedó ella de guardiana y enfermera al cuidado de
todo. No salí a la calle sin alguna inquietud, pensando en la
posibilidad de tropezar con el bestia de Sotero a la vuelta de la
primera esquina, y anduve cuidadoso, requiriendo mi bastón y la fácil
salida del estoque, con el propósito de acometerle antes de ser
acometido; pero por mi ventura y la suya, llegué a donde iba sin que
fuese menester sacar el hierro de la caña, donde dormía su inutilidad
como el otro duerme sus monas.

\emph{30 de Abril}.---Por indiscutible derecho de lógica primacía,
corresponde este lugar a la carta de mi madre, recibida hoy, y cuyos
párrafos culminantes copiaré para mi vergüenza, y edificación de los que
me leyeren: «Hijo mío, no sabré expresarte mi gozo al tener noticia de
tu ascenso, que sin duda ha sido motivado por tus méritos hoy
reconocidos y aclamados por grandes y chicos; y esta mi creencia quedó
confirmada con lo que me escribió Agustinito de las ganas que el señor
de Sartorius tenía de conocerte, y tanta era su curiosidad que no se le
coció el pan hasta que te llamó a su bufete y estuvo platicando contigo
larguísimo rato. ¡Vamos, que no se quedaría el buen señor poco asombrado
de tu saber!\ldots{} ¡Y cómo se le caería la baba!\ldots{} ¡Ay!, a mí sí
que se me cae, considerando que es hijo mío el que tanto da que hablar
por su sabiduría y aplicación\ldots{} De veras te digo que si no supiera
yo cuán gran pecado es el orgullo, me llenaría de soberbia y vanagloria
pensando en ti noche y día, y no hablando de otra cosa más que de tu
superior inteligencia. Pero yo me contengo en mi entusiasmo, y doy
gracias a Dios por el beneficio que me concede.

»Hijo de mi alma, por el pajarito que me cuenta todo, sé que vives muy
retirado, y que eres Alejandro en puño por la moderación y el tino de
tus gastos. Sírvate ahora el aumento de sueldo para que ahorres y vayas
juntando con qué hacerte dos trenes de ropita decente, negra por
supuesto, que tú llamado estás a ser siempre persona grave, aun siendo
joven, por la seriedad de tus estudios y tus modos reservaditos. Y como,
según me asegura el pajarillo una y otra vez, huyes del trato de mujeres
y mujeronas, y te pones colorado en cuanto te ves en presencia de alguna
hembra, no hagas por quebrantar ese tu honesto y recomendable
encogimiento, aunque algunos bergantes te ridiculicen. No te metas,
pues, en gastos de chalecos vistosos, ni de corbatas de colorines, ni
para nada tienes que usar pantalones claros ajustadicos, que eso, digan
lo que quieran, es cosa fea, impropia de un varón digno\ldots{} Presumo
que de tu sueldo no ha de sobrarte gran cosa si, como te encargué, haces
en las fiestas y días de santo regalillos delicados a Segismunda, con
quien vives, y a Sofía, que tanto mira por ti. Con cajitas de dulces,
algún juguete para los chiquillos de Gregorio, y para tus cuñadas
cualquier alhajita de poco precio, jabón fino, paquetes de polvos o cosa
tal, cumples, hijo. Pensando siempre en esto, y con la mira de que
quedes bien, deseo ayudarte, y allá te mando con el ordinario de Molina
ochenta reales, ahorrados por mí cuarto a cuarto, para que los emplees
en algún esparcimiento decoroso, como ir a la función de teatro, un
domingo, como el \emph{Munuza}, que yo vi el año 23, o una comedia mora,
como \emph{El Delincuente honrado}, que tu padre y yo vimos en
Guadalajara; por cierto que toda la función estuve llorando, creyendo
que cuanto allí pasaba era verdad.

»Además de los ochenta reales en un doblón de a cuatro, dentro de un
paquetito donde he metido la oración de Santa Librada y unos papeles de
perfumería, mando un mediano lío con chorizos, de los que hicimos este
año. Van envueltos en una lona cosida por mí con mucho esmero, y bien
rotulado por tu hermano Ramón, como conocerás por la letra. Los chorizos
son de calidad tan superior que no se hallará en Madrid género igual.
Los mando por el ordinario de Molina, porque éste va más pronto que el
de acá, que se duerme en las largas estancias de Alcalá y Meco. Vete al
parador denominado del Peine, en la calle de las Postas, y pregunta por
Quiterio\ldots{} Me parece que tú le conoces. Te encargo que hagas este
recado tú mismo, y que no te fíes de criados, no vayan a cambiarnos los
chorizos por otros de los que se compran en las tiendas. Tú mismo
recoges el dinero y el paquete grande, y ten mucho cuidado en repartir
los chorizos por partes iguales entre las dos casas. No vayan a ponerte
hocicos por si la una o la otra llevó menos parte.

»No me cuentas nada, picarillo, de la obra que sobre el Papado estás
escribiendo. Si no me hubiera dicho el pajarito que llevas ya lo menos
cuarenta capítulos, nada sabría de tu tr. Imagino que estarán los
libreros y todo el personal de sabios esperando que sueltes el primer
tomo para caer sobre él como lobos hambrientos. Tratarás del Papado
completo, de la cruz a la fecha, empezando por San Pedro y no parando
hasta el Santísimo Pío IX. Materia más interesante no puede haberla. No
sabiendo yo qué leer en estas largas horas de la tarde y la noche, pedí
a D. Julián, chantre de la catedral y profesor del Seminario, que me
trajera algún libro que yo comprendiese, y que conteniendo buena
doctrina, tuviera también recreo para personas legas, y me trajo la
\emph{Historia de los Concilios}, que estoy leyendo con muchísimo gusto.
Ya llevo lo menos treinta hojas, y todavía no he sentido cansancio, sino
más bien un gran interés, admirando las virtudes de tantos Santos Padres
y esperando a saber en qué para tan larga historia. Tú, que todo esto te
lo sabes como el Padrenuestro, te reirás de mí. Me ha dicho D. Julián
que esa obra que estás plumeando será muy larga, y que tú lo has tomado
tan a pechos que no se te queda por registrar ninguna biblioteca profana
de las que hay en Madrid, y que en todas te metes, así en las públicas
como en las privadas, pasando en ellas largas horas de la noche. Hijo
querido, trabaja con calma y prudencia: no consumas tus facultades
abusando de ellas; no te calientes el entendimiento; modera, hijo,
modera, y pon en todo pulso y medida. No desoigas este consejo dictado
por mi cariño; recíbelo, con la bendición de tu amante
madre.---\emph{Librada.»}

\hypertarget{xix}{%
\chapter{XIX}\label{xix}}

\emph{31 de Abril}.---La carta que anoche agregué a mis Confesiones
removió en mi conciencia la turbación que en ella mora, unas veces
adormilada, otras en profundo sueño. Pero los afanes de cada día, que en
la mundana corriente van creciendo y encrespándose como un oleaje
furioso, han ahogado aquel sentimiento trayéndome a inquietudes
inmediatas y más positivas. Parte del día he pasado en la casa de
Antonia disponiendo sustituir lo más indispensable del ajuar robado por
Sotero, y en ello se me fue todo lo que no hace mucho me entregó con
enorme usura el prestamista. ¡Aciaga tarde la de hoy, en la cual he
llegado a creer razonables los delirios de la cordonera, pues no habría
para mí mejor solución que abrazar la vida de ermitaño, con ermitaña o
sin ella, en un solitario y agreste yermo, comiendo raíces y vistiéndome
de lampazos! Cuando vio la enferma que la casa se iba reparando de su
desnudez, empezó a curarse de la manía del salvajismo, y aunque siempre
tiraba al monte, no lo hacía con tanta vehemencia. A sus parientes
míseros, que acudieron maldiciendo su suerte y bendiciendo mi caridad,
tuve que socorrer hasta quedarme sin un maravedí. Por la calle Mayor
adelante, pensaba yo que no poseía en aquel momento más peculio que el
dobloncito de mi madre, aún no recogido del ordinario, y antes que se me
olvidara fui al parador, donde puntualmente me entregaron moneda y
chorizos, todo lo cual llevé a mi casa con gran respeto, como si llevara
el Viático, y después de partir con religiosa equidad entre las dos
familias los embutidos, miré y acaricié y escondí mi doblón bajo llave,
precaviendo de este modo la probable ignominia de ponerlo a una carta.

\emph{1.º de Mayo}.---Con endiablado afán de probar suerte, por
irresistible instinto de mejora, me pasé la noche dando tremendos
estirones a las orejas de Jorge, mas con tan loco desacierto en cuanto
apuntaba, que ni un instante me sonrió la fortuna. La terrible deidad me
asestaba golpe tras golpe, como si fuese yo un excomulgado de la
diabólica secta que tiene por biblia los naipes malditos. Concluí en el
mayor desastre, debiendo a mis amigos sumas que mi abrasada mente
imaginaba fabulosas. Para pagarlas érame forzoso pedir a la usura nuevos
auxilios, que más bien serían dogales con que pronto habría yo de llegar
a mi definitiva estrangulación. Abrasado mi cerebro, dormí con
pesadillas parte de la mañana, y al despertar entráronme una carta de
Sor Catalina en que me afea destempladamente, no sin razón, mi grosero
descuido en la prometida visita a los Emparanes. ¡Dichosos Emparanes! No
vacilo más, y vencida mi repugnancia, me dispongo a cumplir. Almuerzo
tarde, me visto, espero la hora oficial de visitas de etiqueta, y tomo
pausadamente el caminito de la plazuela de Navalón, leyendo en las rayas
del embaldosado de las calles cifras misteriosas de mi destino.

La casa es antigua reformada, grandona, irregular, revocada de amarillo
con rayas que figuran el ajuste de ilusorias piedras, la puerta de
berroqueña con un escudo pintado de blanco, los balcones con palomillas
de hierro, y en ellos las descoloridas palmas de Domingo de Ramos, con
los trenzados en hilachas y los lazos ya desteñidos por la lluvia. En
todo esto reparé antes de entrar, así como en el aspecto del portal, de
una limpieza rara en Madrid. El portero, viejo y medio cegato, limpio
también como la casa, ostentando chaleco rojo y gorra galonada, me
acogió con marcado respeto, y oído mi nombre, díjome con el acento más
satisfactorio que \emph{los señores estaban}, y me franqueó la entrada
de la escalera, lóbrega, sin más adorno que unos faroles de navío y
cuadros viejos, cuyo asunto se pierde en la oscuridad de la ennegrecida
tela.

Un portero de estrado, viejo también y con chaleco rojo, me introdujo en
el salón, que examiné con rápido golpe de vista a la escasa luz que por
los entornados huecos de los balcones entraba. Vi retratos de personajes
del pasado siglo, consejeros de Castilla y de Indias, almirantes,
generales, todos con el peluquín de ala de pichón, los rostros amarillos
y sin relieve, detestables pinturas en su mayor parte; vi santos y
frailes de diferentes Órdenes, de mano de Orbaneja; vi, por fin,
retratos de Papas, en los cuales me fijé singularmente. Aquí, Mauro
Capellari (Gregorio XVI), de aspecto achaparrado; allí, Della Genga
(León XII), de noble rostro; a la otra parte, las finas facciones de
Chiaramonti (Pío VII). Como pintura, estos retratos merecen el fuego,
salvando sus espléndidos marcos. Mil otras obras de inferior y menguado
arte vi en el salón: pinturas milagreras, relicarios con más riqueza que
gusto, autógrafos de monjas en cuadros de plata, dos o tres arquetas de
indudable mérito, y una disforme y amazacotada araña de cristal.
Contrastaban con estas antiguallas los muebles construidos en estilo
modernizante, los sillones y canapés de raso anaranjado, los chinescos
jarrones, las consolas de caoba con adorno de bronce dorado, algún
espejo de marco a la griega, y los candelabros encerrados en fanales.
Movido de no sé qué fanatismo suspicaz, creí ver dentro de aquellos
vidrios las velas verdes de la Inquisición. En todo reparé fugazmente,
maravillándome así de la muchedumbre de objetos que respiraban devoción,
como de la perfectísima limpieza que en lo antiguo y lo nuevo
resplandecía, cual si muchas manos escrupulosas diariamente persiguieran
el polvo, la mugre y toda suciedad por menuda que fuese.

No acabé mi examen, porque un criado me rogó que pasase al próximo
gabinete, donde salió a mi encuentro el Sr.~D. Feliciano de Emparán con
luengo levitón que rápidamente se abrochaba, como si acabara de
ponérselo para recibirme; y estrechándome las manos muy afectuoso, me
hizo sentar en un blando sofá, sobre el cual ostentaba su dulce rostro,
en marco flamante de cornucopia, la imagen de Mastai Ferretti, a quien
yo amaba desde que fue mi preferido y victorioso candidato a la sucesión
de San Pedro. Daba yo a D. Feliciano noticias de mi salud, que con muy
vivo interés me pedía, cuando entró la señora de Emparán, doña
Visitación de Baraona, en bata morada con encajes, y sus primeras
palabras, después de oír mis cumplidos, fueron para redoblar las
interrogaciones acerca de mi salud: «Ayer nos dijo Sor Catalina que ya
estaba usted en plena convalecencia y podía salir a la calle.» Yo
asentí, comprendiendo que mi hermana había disculpado la tardanza de mi
visita con un inocente embuste. «En seguida vendrá María
Ignacia---añadió doña Visita,---que ya está concluyendo la lección de
piano. La pobre no oculta su alegría, porque, a pesar del mal tiempo que
tenemos, se va recobrando de sus alifafes nerviosos.»

Sobre estos alifafes hablábamos, declarando yo su escasa importancia en
el organismo, cuando llegó otra señora mayor, Doña Rita, hermana de D.
Feliciano, en traje de merino negro, con escofieta blanca; y no había yo
concluido de saludarla, cuando vi aparecer la tercera señora mayor,
valdría más decir máxima, Doña Josefa Baraona, tía de Doña Visita,
también uniformada de negro, viejísima, desdentada, pero no falta de
viveza y agilidad. «Ya tenía yo el gusto de conocerle---me dijo cuando
le ofrecí mis respetos.---Le vi una mañana en el locutorio de La
Latina.» Y yo miraba a la puerta esperando que acabara de salir el coro
de Emparanas y Baraonas mayores, pues me habían dicho mis amiguitas
Valeria y Virginia que no bajaba de seis la cifra de venerables matronas
que habitaban allí. Oyendo el remoto cascabeleo de un piano, esperé
ansioso la presencia de María Ignacia, la señorita con quien querían
casarme, tierna paloma que todas aquellas cornejas agasajaban entre sus
plumas.

Debo declarar, poniendo la verdad por encima de mis antipatías, que las
cuatro personas mayores eran de trato muy fino y de exquisita educación,
a la antigua española. Sosteniendo con ellas un coloquio de pura
fórmula, pensaba yo para mis adentros en los artificios de que ha debido
valerse mi hermana Catalina para conquistar el ánimo de aquella familia,
y qué grande ascendiente ha podido adquirir sobre todos para meter en
sus duras cabezas, y darle allí fuerza dogmática, la peregrina idea de
que yo soy el hombre designado por Dios para realizar los grandes fines
de la sucesión \emph{Emparánica}. Sin género de duda, es mi hermana
mujer de extraordinario entendimiento, y de una travesura que bien puedo
llamar política, pues en esa cualidad estriba el dominio de las gentes y
la generación de los grandes sucesos públicos y privados. Ello es que
Catalina, sorbiéndoles el seso, trata de realizar con firme voluntad la
filosofía del gran Antonelli, condensada en esta fórmula: «Tu familia te
procurará un buen casamiento.»

Impaciente Doña Visita por lo mucho que su niña se entretenía en los
musicales ejercicios, fue en su busca, y a poco la trajo de la mano,
diciéndome al presentarla: «Dispénsela usted. Quería mudarse de vestido;
pero como usted es de confianza, puede verla en el trajecito de casa.»
Hago acopio de toda mi sinceridad y rectitud para declarar que la
primera impresión que en mí produjo la niña de Emparán fue atrozmente
desagradable. ¡Válgame Dios qué niña! Y aunque en el breve espacio de
una visita sólo podía yo juzgar el ser físico, éste y el espiritual,
representados en un solo ser, pareciéronme de lo más desgraciado que
Dios ha puesto en el mundo. Es Mariquita Ignacia lo más contrario al
tipo de muchachas que comúnmente vemos en todas las clases sociales,
pues no hay ninguna que en la florida sazón de los dieciocho no tenga en
su persona, siquier sea la misma fealdad, algún rasgo de gracia y
donaire, algún tono de frescura y de seductora juventud. El cuerpo es un
mentís de su edad, que en ella parece un fraude. Rara vez se revisten
los verdes años de aquella gordura desatentada, contraria a todo
sentimiento de proporción, pelmazos de carne distribuidos sin ninguna
lógica en las partes de un defectuoso esqueleto. Abulta el seno
enormemente, saliéndose del círculo natural de la doncellez, y para
acabar de arreglarlo, la cintura y vientre con aquella otra zona quieren
confundirse, rompiendo la esclavitud del corsé y arrollando las filas de
ballenas que martirizan el pobre cuerpo. Son los brazos chicos, el
cuello corto, gordezuelas y bonitas las manos, única nota bella en que
puede recrearse la vista. Ella lo sabe y habla más con las manos que con
la boca.

Hice un mental esfuerzo por descubrir en el rostro de María Ignacia algo
que despertar pudiese admiración o agrado, y no lo encontré, bien sabe
Dios que no lo encontré. En la estricta verdad me inspiro al firmar que
la señorita de Emparán nació desfavorecida de todas las hadas. Deseando
conceder algo, sostengo que es aceptable su rostro cuando la niña
permanece con la boca cerrada; pero en cuanto descorre la cortinilla de
sus labios, aparece el rojo escándalo de sus encías que todo lo afean;
los dientes son desiguales, colocados anárquicamente, sin más atractivo
que una limpieza tan esmerada como la de toda la casa de Emparán. Bien
sabe la niña que su boca es la negación de la juventud, de la alegría y
del amor, y no cesa de hacer hociquitos y muecas para tenerla siempre
tapada. Hay que ver sus apuros cuando, en los incidentes de la
conversación, forzada se ve a la risa franca: de aquí proviene la
seriedad que la hace más desapacible. Rubios tirando a bermejos son sus
cabellos, peinados con arte, y sus ojos claros, sin viveza, miran
medrosos reclamando la compasión más que la simpatía. ¡Pobre María
Ignacia! Yo sentía lástima de ella y de sus padre y familia, que en tan
infeliz persona concentran todos sus afectos y aspiraciones.

Del trato, revelador seguro de las dotes de ingenio, poco puedo decir
todavía, porque María Ignacia no pronunció en la visita más que cortadas
y tímidas expresiones: su condición huraña, nacida de la conciencia de
su fealdad, y el mimo que le daba toda la familia, reducían su
vocabulario a la mínima expresión: las ideas no se manifestaban en ella
más que en forma rudimentaria, y su palabra torpe y balbuciente no hacía
nada por sacarlas a luz. Llevaron los padres y las señoras mayores la
conversación al terreno más propio para que la niña pudiera lucirse un
poco, el terreno de la vida mundana, paseos, teatros, modas, la
esclavitud que traen tantas vanidades; pero ni por ésas\ldots{} Tuve yo
que hacer el gasto, y con facilidad suma traté la cuestión. Las personas
mayores oíanme admiradas; y la pobre niña, que desde que entró hasta que
me fui no quitó de mí sus claros ojos, escuchaba mi acento con una
fijeza que al éxtasis se me parecía. Apunto esto sin vanidad, mirando a
la exactitud de los hechos, y sin que mi relato signifique alabanzas de
mí mismo, pues nada dije que no fuese de lo más común. Ante cualquier
otro joven de mi edad habría pasado lo mismo\ldots{} Por fin llegó el
momento en que yo no podía prolongar la visita sin incurrir en falta de
urbanidad, y me despedí. Invitome a comer la señora de Emparán para día
fijo, a lo que accedí porque no podía eximirme de ello; señalome además
ciertas noches de la semana en que los amigos van a jugar al tresillo y
a pasar un rato en amenas charlas, y prometiendo acudir alguna vez, les
expresé mis gratitudes, y él y ellas me dieron las suyas en la forma más
expansiva. María Ignacia, al decirme adiós, bajó los ojos como
avergonzada.

Salí de la casa de Emparán con simpatía hacia la familia, mas también
con el firme propósito de oponer un inquebrantable \emph{non possumus} a
los planes de mi hermana. Sin duda, el dominio moral de Catalina sobre
aquella gente se fundaba en algo de autoridad religiosa: los Emparanes
debían de mirarla como a ser superior, que llevaba dentro el Espíritu
Santo. Pero si la bendita monja se había hecho absoluta señora del
corazón de la ilustre familia, no podría por ningún medio hacerme esposo
de la desgraciada, de la imposible María Ignacia.

\hypertarget{xx}{%
\chapter{XX}\label{xx}}

\emph{2 de Mayo}.---No he querido que pase el día de hoy sin comunicar a
mi hermana mi decidida protesta contra sus planes de matrimonio. Pero
como, si le manifiesto de palabra mi negativa, es fácil que su carácter
despótico caiga con abrumadora grandeza sobre mi pobre voluntad y acabe
por aplastarla, he preferido escribirle. Al convento mandé esta tarde mi
carta, en la cual vengo a decir con corteses y limpias expresiones, que
no aceptaré la mano de la niña de Emparán aunque me den con ella todas
las riquezas que el mundo atesora. Se casa uno con una mujer, a la cual
no estorban sus talegas si está de buenas y bellas cualidades adornada;
pero no se casa nadie con un capital personificado en una criatura que
carece hasta de los atractivos más elementales. Esto sería venderse, no
casarse\ldots{} En fin, bien hilada va la epístola, y no sé por qué
lógicos vericuetos echará para contestarla Sor Catalina de los benditos
Desposorios.

Hablando de otro asunto, dos cosas me afligen esta noche: Antoñita en
mayor gravedad, y mis bolsillos en absoluta limpieza. He tenido que
apurar a los usureros y porfiar con ellos en la forma más humillante,
para reblandecer estas rocas de la desconfianza y el egoísmo. Por fin
logro extraer de sus arcas alguna cantidad en condiciones horrendas, y
con ello puedo atender a una de las deudas contraídas la noche de mi
catástrofe de juego. Pero aún me falta el compromiso más apremiante, por
tratarse de compinches de timba, que me han fijado improrrogable plazo
para cumplir. Acudo a Guillermo Aransis, que se encuentra en situación
no menos ahogada que la mía, y acudo a todas las potencias infernales
para que me saquen del pantano. Ni del cielo ni de la tierra viene
auxilio para este infeliz. He pasado una tarde horrible y una noche
peor, apretándome los sesos para que de ellos salga la chispa de una
resolución salvadora. Si no estoy loco ya, poco me falta.

Mis pobres sesos dan por fin una luz, resplandor muy lejano, que indica
inciertas probabilidades de éxito: ello consiste en recurrir a mi
hermano Gregorio. Me armo de valor, hago acopio de argumentos aderezados
con sensiblería, y por fin, esta noche abordo la cuestión ante
Segismunda, pues el directo trato con mi hermano en este asunto es
empresa superior a mi audacia\ldots{} ¡Santo fuerte, Santo inmortal,
cómo se puso mi cuñada apenas formulé mi petición! Ni me dejó concluir
la frase angustiosa, trémula y antigramatical\ldots{} Creí ver
enroscarse las serpientes que tiene por cabellos, y su boca griega,
volviéndose cuadrada como las de una máscara de tragedia, vomitó sobre
mi pena injurias que sonaban a sordidez furiosa y a egoísmo de parientes
desnaturalizados. No podré reproducir aquí sus brutales anatemas. Que
cómo y con qué respondo del cumplimiento de mis obligaciones\ldots{} que
si creo posible hacer vida de señorito de la Grandeza sin más patrimonio
que el día y la noche\ldots{} que estoy deshonrando a la familia, y que
he perdido la vergüenza, y acabaré en el Hospicio, si no voy a parar a
la cárcel\ldots{} que si no hago enmienda total trayendo a casa el
dinero de los Emparanes, no espere socorro de la familia, sino
desprecios y maldiciones.

Al discorde ruido que la condenada mujer hacía, no tardó en acudir
Gregorio, el cual, adivinando la cuestión por la lividez de mi rostro y
los apóstrofes crudos de Segismunda, prosiguió la filípica con no menos
ira en los denuestos. Atrevime yo a replicarle, y trabados los tres en
furibunda querella, llegamos al desconcierto más escandaloso. Como
dijese mi hermano que era grande enojo para él tenerme en su casa, por
el continuo jubileo de acreedores que a la puerta venían con atrasadas
cuentas o recibos, sin que hubiera ya palabras con que aplacarles o
persuadirles a la paciencia, estalló Segismunda en nuevas iras,
abominando de los ilícitos enredos que estorbaban el casamiento
patrocinado por la monja; amosqueme yo más de lo que estaba, y subido el
tono y coraje de todos hasta el punto de la ronquera, corté la disputa
con la resolución de renunciar a su hospitalidad y dejarles tranquilos.
Nada dijo Gregorio para contenerme, ni mi designio sirvió de agua mansa
para templar sus arrebatos; antes bien, parecían contentos de que yo
tomara el portante. Recogí lo que podía llevar conmigo, guardé mi ropa
en maletas para que fácilmente pudieran llevármela a mi nuevo domicilio,
y me vine a la casa de Antoñita, donde doy testimonio de mi existencia
escribiendo junto al lecho de esta pobre mujer páginas amarguísimas de
mis Confesiones\ldots{} ¡Dos de Mayo! La fecha no puede ser más lúgubre.
¡Quiera Dios que no sea trágica!

\emph{3 de Mayo}.---Llena está mi alma de presagios siniestros, pues me
siento rodeado de sombras por todas partes, y cerca y lejos de mí veo
los espectáculos más tristes que ofrece la humana vida: a mi lado, la
muerte; a distancia, la deshonra posible, la probable miseria. Escribo
por la mañana, tras largo insomnio, y noto que el acto de trasladar al
papel mis dolorosas impresiones amansa mis penas y las hace tolerables.
Parece que hay alguien que a soportarlas me ayuda, o que mis propios
escritos, transmitidos a una Posteridad lejana, me dicen que la vida es
larga y que en ella no pueden ser duraderos los infortunios como no lo
son las dichas. Tras unos días vienen otros, y la naturaleza rehúye la
uniformidad de las cosas\ldots{} Vienen a mi pensamiento estas
candorosas filosofías velando el sueño inquieto de Antonia, que ha
entrado en un período de suma gravedad, según me ha dicho el médico
enano\ldots{} Si bien lo miro, no sé si estoy aquí porque debo estar, o
porque no puedo estar en otra parte. Sobre esto me interrogo y, la
verdad, no sé responderme categóricamente.

Avanzado el día, entran en esta casa algunas niñas de la vecindad que
andan en el divertido juego de pedir para la Cruz de Mayo. Vestiditas de
limpio, con su pañuelo de talle cruzado a la cintura, y flores en la
cabeza, se disponen a la persecución y despojo de los transeúntes. Entre
ellas, una muy linda, que no tendrá más de cinco años, me hace mil
carantoñas, se sube a mis rodillas, no se contenta con un cuarto ni con
dos, y metiendo su manecita en mi bolsillo, me saca la única peseta que
hay en él. Me resigno a tan dolorosa expoliación, y la despido con
besos. Ella me dice: «caballero, usted me estrena,» y se va ondulando el
cuerpo con meneos graciosos. Salen tras ella las demás, después de
aligerarme del cobre que poseo, y sus risotadas se pierden en la
escalera. ¡Dichosa edad!

Vuelvo a coger la pluma después de un largo rato de tedioso paseo en la
estancia. La pobre Antonia está muy caída de espíritu y en gran
debilidad de cuerpo; pero en sus ratos de lucidez, que son pocos, no
ceja en su manía proyectista: muchas ideas la atormentan, menos la de la
muerte. El hecho de haber yo trasladado a su casa mi vivienda, por el
deseo y el deber de cuidarla, según cree y dice, enciende en su pobre
alma vivísima gratitud, y el ardor de este sentimiento, brotando del
corazón a la piel, entiendo yo que es ayuda y estímulo de la naturaleza
en su lucha contra la fiebre.

No pudiendo apartar mi pensamiento de otros conflictos, de intensa
gravedad para mí, escribo a Guillermo llamándole a mi lado para que vea
mi anómala situación, y me ayude por cualquier arbitrio extraordinario a
salir del compromiso en que estoy, por la deuda de juego no saldada.
Contéstame Aransis a las dos horas que se ocupa de mi asunto, y que
espera resolverlo a prima noche; que de diez a once me espera en el
Casino, y me encarece con vivas instancias que no falte a la cita, pase
lo que pase, pues tenemos que hablar. La carta de mi amigo me hace
recobrar la esperanza, y para mayor consuelo, el médico liliputiense no
hace malos augurios en su visita de la noche. Puedo sin cuidado
alejarme, y en posesión de mi ropa, que al mediodía me trajeron sin otra
merma que un par de corbatas, un chaleco de alepín, y alguna prenda
interior, me visto y salgo, dejando a Margarita bien aleccionada, y con
la advertencia de que volveré pronto, infalible ardid para que esté muy
alerta en su obligación.

\emph{4 de Mayo}.---Déjame, déjame, oh ignoto público de la Posteridad,
si en efecto existes y me lees; déjame que tome respiro y ataje los
vuelos de mi pluma en esta parte de mis Memorias, pues tantas desdichas
en ella se reúnen, que me será difícil transcribirlas con orden para que
aparezcan en la serie aterradora con que me las ha deparado el Destino.
Me río, pueden creérmelo, con risa que es una mixtura increíble de rabia
y gozo, al sentir sobre mi cabeza esta ingente acumulación de males.
¿Son obra lógica de mi propia conducta, o fatal embestida de un espíritu
diabólico que se entretiene castigando a los inocentes? ¿Quién dispone
esta convergencia de todos los dolores en un solo punto?\ldots{} No lo
sé; pero doy en pensar que lo que llamamos Casualidad es un desconocido
método de las cosas invisibles y el superior ordenamiento de las causas.

Aunque gusto más de filosofar sobre mis penas que referirlas, dejo a un
lado las metafísicas y me voy a la relación de los hechos, empezando por
decir que me personé en el Casino a la hora marcada por Aransis y que
éste no tardó en llegar. Con lenguaje precipitado y ansioso me participó
el arreglo de mi asunto, aprovechando una extrañísima coyuntura
favorable que la casualidad le había deparado. Diole su abuela el
encargo de llevar una cantidad de consideración a su primo el conde de
Tarfe, y él ¿qué hizo? Diferir para mañana la entrega, destinando la
mayor parte del dinero a sacarme del compromiso y guardando lo demás.
Era, pues, indispensable que los dos revolviéramos el mundo para reponer
la suma en el fatal plazo. Mucho agradecí a Guillermo el apurado socorro
que me traía; pero con el reconocimiento se confundió el terror del
nuevo y mayor aprieto que para el día siguiente se nos preparaba. A lo
hecho, pecho: tomé el dinero; pagué incontinenti, excusándome de la
tardanza con el aquel de tener en casa un enfermo grave, y mi amigo
Caballero, que era mi acreedor, dejó de serlo y volvimos a encontrarnos
en afectuosas relaciones ante la sociedad y ante el vicio.

Cuestionando con Aransis acerca de la responsabilidad del día próximo,
propúsome mi amigo que con el dinero restante probásemos a obtener del
azar lo que nos hacía falta. Fuera miedo; buscáramos nuestra solución en
el desquite, pues bien podía la suerte mostrarse benigna después de
tantos desdenes. Yo creí lo mismo, que si no hay bien eterno, no hay mal
que cien años dure. Jugamos, y el demonio de amarillos ojos cuando uno
pierde, de pupilas rojas cuando uno gana, se divirtió en balancearnos de
las ansiedades pavorosas a las hondas alegrías. A la una estaba yo
boyante; pero quise más, y a las dos lo había perdido todo. Busqué a
Guillermo con angustiados ojos para que me favoreciera, y advertí que
había desaparecido de la criminal sala. Agencié un empréstito, hice
nuevas cucamonas a la fortuna, y ésta siguió tratándome como a un perro.
A las tres de la mañana, apartándome de la mesa de juego, halleme sin
saber cómo en un grupo compuesto de caras amigas y otras simplemente
conocidas, no todas simpáticas. Entre estas caras destacose la de un
hombre de mediana edad, señalado en la trinca nuestra por su índole
maleante, sus dichos a veces graciosos, groseros a veces, el cual,
riendo con desenfado, me dijo estas palabras: «Si quiere dinero, yo
tengo para usted cuanto necesite.» El valor gramatical de las palabras
era tan distinto del tono con que fueron dichas, que me sentí ofendido,
y respondí en el mismo tono: «Gracias: no juego más. Celebro verle a
usted tan generoso.» Y él con disparada lengua: «Lo soy con los que como
usted ofrecen garantía segura, con los que cultivan mujeres ricas que
les pagan las deudas.»

Tenía yo, al oír esto, apoyada mi mano derecha en el respaldo de una
silla. Ciego enarbolé la silla, apuntando a la cabeza del insolente; mas
interpuestos los amigos, ni la silla fue a estrellarse donde yo quería,
ni pude saciar mi furor con las manos. El tumulto fue ruidoso; se
arremolinaron los amigos y conocidos, unos allá, otros acá, para
separarnos y agrandar la distancia, y entre tantas voces oí la de aquel
bruto que, alejándose a la fuerza, chillaba: «¡Dejarme a ese \emph{Don
Líquido, Catacaldos}\ldots!»

Llámase el tal Jiménez de Andrade, y goza fama de temerón y
perdonavidas. Es de Écija o de Marchena, no recuerdo bien; ha derrochado
dos fortunas; entiende de caballos más que de política, y en ésta quiere
señalarse ahora, ahuecando la voz entre los progresistas exaltados y los
demócratas. Frecuenta el trato de militares, jactándose de seducirles
para la revolución; es, en suma, un bárbaro, que no busca más que el
ruido y el escándalo para sacar su persona de la oscura vulgaridad a que
pertenece\ldots{} No necesito indicar que al instante determiné lavar
con sangre el oprobio que aquel bestia arrojó sobre mí; yo quería
matarle o que él me matara. Mis amigos hicieron suya mi causa, y como
alguno expresara su inquietud por la desigualdad de la lucha entre un
hombre diestro en las diferentes armas y otro que apenas manejarlas
sabe, afirmé yo que tal desigualdad tendrá para mí la ventaja de
proporcionarme una muerte muy expeditiva. «Estoy cansado de vivir---les
dije.---Acabemos de una vez.»

Yo deliraba. Mis amigos procuraron sosegarme, y a ellos me confié para
que cuidasen conmigo de poner en salvo mi honor. Quise nombrar padrino
al Marqués de Bedmar, amigo mío que me distingue y considera; pero no
habiendo podido encontrarle a tan avanzada hora, elegí a Bermúdez de
Castro y a Guillermo Aransis. Dos horas estuvieron mis amigos buscando a
éste, y en casa de unas famosas cucas le encontraron a las tres y media
de la madrugada. En el propio Casino intentaron mis apoderados un
arreglo amistoso, fundados en que Andrade estaba ebrio en el momento del
insulto, y creyendo que gallardamente daría explicaciones al
despejársele la cabeza. Pero ya porque ésta no se despejara, ya porque
su razón nada pudiera contra su brutalidad, no hubo arreglo, y Andrade
insistió en que tendría el gusto de mandarme al otro mundo\ldots{}

Asomaba la aurora por los balcones y ventanas del madrileño horizonte,
cuando mis amigos me trajeron a esta casa, dejándome en el recogimiento
que necesito para la meditación y el descanso. Las vivas emociones, el
insomnio de las noches pasadas habíanme traído a tan gran quebranto de
la naturaleza, que caí en el camastro como en un pozo, y dormí con sueño
parecido a la embriaguez. Mediodía era por filo cuando me despertó
Aransis para decirme que los padrinos del contrario son dos andaluces,
Sánchez Silva y Nicolás María Rivero. A éste le conozco: es muchacho de
mérito, áspero, cetrino, ceceoso en el hablar. Añade que no se ha podido
conseguir de Andrade un honroso acomodamiento, el cual habría de
fundarse en una satisfacción hidalga por parte de él. Digo yo que me
alegro de que no haya componendas artificiosas y cobardes. Me informa
Guillermo de que a pistola será el lance, y no le dejo seguir cuando
quiere puntualizar las condiciones, tantos pasos, avance gradual\ldots{}
Las condiciones, que poco me importan, las conoceré mañana. Me basta con
saber la hora, ocho en punto, y el lugar, la huerta de Moreno-Isla,
cerca de la Fuente del Berro. Insiste con grande interés mi amigo en que
dedique la tarde y parte de la noche a ejercitarme en el tiro de
pistola, a lo cual me niego resueltamente, pues con lo que sé me bastará
para matarle si los hados me favorecen, y lo que aprender pueda en tan
poco tiempo no impedirá mi muerte si está ya escrita y decretada en el
\emph{fatídico} libro de los Sucesos\ldots{} Vase Aransis, y al quedarme
solo, siento lo fatídico en torno mío\ldots{} y se me enfría todo el
cuerpo. Me dejo abrigar por Margarita en un pesado mantón suyo.

\hypertarget{xxi}{%
\chapter{XXI}\label{xxi}}

No tardé en advertir que mi estoicismo era un tanto figurado,
histriónico, y con esfuerzos de la razón me puse en el verdadero punto
psicológico que los hechos imponían, ni medroso ni arrogante, fiado en
que me ampare Dios, y desechando la insana idea de que deseo morir,
fraudulento recurso teatral, cuya procedencia descubro en los afectados
versos de la época. Que yo no estaba en mis cabales cuando Guillermo me
habló del lugar y hora del duelo, lo demuestra que olvidé preguntarle si
había resuelto el conflicto pecuniario que para hoy nos reserva el cruel
Destino. Mañana me lo dirá, si estoy en disposición de oírlo\ldots{}
También podría suceder que me fuese a la Eternidad sin saberlo, ni
importárseme un ardite de las menudencias que aquí se nos hacen
montañas. Yo pregunto cuáles serán las estrellas que se vendrán porque
traigamos a nuestros bolsillos el dinero que a Guillermo entregó su
abuelita\ldots{} ya no me acuerdo para qué.

Vuelvo a tomar la pluma, ya anochecido, y como mis cavilaciones no me
hacen perder la noción del método, escribo que la pobre Antoñita va de
mal en peor, y que ella será motivo de que el Destino se ensañe más en
mí, prolongando indefinidamente la serie angustiosa de sus furibundas
estocadas. Esta tarde nos vimos y nos deseamos Margarita y yo para
sujetarla cuando se arrojó del lecho, pidiendo que la vistiéramos.
Quería irse conmigo a la verbena de San Antonio. «¡Si no es hoy la
verbena, tonta!---le dijo su amiga.---Es mañana, que ahora andan
trabucados los meses, y el 12 de Junio por la tarde viene a caer mañana,
que así lo dispuso el Padre Santo, por ser el año cuatro veces
bisiesto\ldots» Tan ardiente era la calentura que su rostro quemaba, y
brillaban sus ojos como luceros. Logré calmarla, prometiéndole que
iríamos juntos a la verbena, y recostado en su propio lecho sobre las
mantas, para con mis brazos aprisionar los suyos, oí sus expresiones
amorosas, más que nunca impregnadas de ternura. Díjome que yo le
pertenecía, que juntos estaríamos hasta que nos muriésemos, y que
viviríamos un sin fin de años, pues así lo había ordenado el Papa. Desde
la tarde anterior intervenía en su atroz delirio la figurada persona del
Sumo Pontífice, eclipsando con su grandeza las demás figuras que
poblaban la mente trastornada de la pobre mujer. ¿Quién puede fijar de
dónde le vienen las ideas al que enloquece? Vienen quizás de
pensamientos sedimentados antes de incurrir en la demencia.

---«El santo Papa---dijo Antonia dejándose arrullar,---me aseguró ayer
tarde, cuando vino vestidito de paisano y con ramo de azucenas, que me
descasaría de Sotero para casarme contigo, y yo me alegré tanto
que\ldots{} se me saltaron las lágrimas. Bien puedes estar con cuidado
para abrir la puerta en cuanto llame, que esta tarde ha de volver,
revestido de todos los pontificales, con capa colorada. Vendrán con él
siete cardenales; no te descuides, que como la visita es motivada de las
ganas que tiene de conocerte y alternar contigo, es justo que tú seas
fino con él, verbigracia, y correspondas, \emph{gitano} mío\ldots» Cortó
su locuacidad la tremenda sacudida de aquel toser que parecía partir el
tórax en mil pedazos. El silbido del aire en las cavernas de su seno
causaba espanto. ¡Pobre Antoñilla! ¿Por qué Dios no había de salvarla?
Esto me preguntaba yo, entendiendo cada vez menos el misterioso
ordenamiento de muertes y vidas.

Las primeras horas de la noche transcurren amargas disponiendo nuevas
tomas de drogas prescritas por el médico chico, y más vejigatorios, que
acabarán de desollar aquel pobre cuerpo martirizado\ldots{} La enferma
cae al fin en dulce desvanecimiento. Atacado de un furioso pesimismo,
pienso en su muerte y en la mía, por bien diferentes modos de
morir\ldots{} Me paseo por la estancia, de un ángulo a otro, rodeando la
mesilla donde están la luz y los potingues, y en este cadencioso
movimiento de fiera enjaulada transcurre no sé cuánto tiempo. Por fin,
me siento a escribir, apartando las medicinas que me estorban; y apenas
cojo la pluma, oigo que da la hora el reloj de la Casa Panadería. Cuento
las doce campanadas para cerciorarme de que paso del hoy al mañana, o de
que el mañana se pone las insignias del hoy, y empiezo por consignar la
fecha:

\emph{Cinque maggio}.---Lo escribo en italiano porque la fecha trae a mi
memoria la muerte de Napoleón y la célebre oda de Manzoni. ¡Vaya, que no
es floja honrilla morir el mismo día que el primer Capitán del siglo!
Con cierto humorismo me aplico los viriles acentos del poeta:

\small
\newlength\mlena
\settowidth\mlena{\quad Ei fu. Siccome immóbile}
\begin{center}
\parbox{\mlena}{\textit{\quad  Ei fu. Siccome immóbile                   \\
                        dato il postrer sospiro...}}                     \\
\end{center}
\normalsize

Trato de penetrar el arcano de los acontecimientos que mi \emph{Cinco de
Mayo} me guarda en el preñado vientre de la entidad diurna. ¿Qué
sucederá?\ldots{} Pienso después en que habito un mundo apartado de la
ordinaria esfera de mi vida. Ninguna persona de mi familia ha parecido
por aquí. O ignoran dónde estoy, o soy para ellos como un ausente, como
un difunto. Hasta la presente hora no había sentido desconsuelo por este
alejamiento de los míos. Mi hermano Agustín, ¿por qué no viene a verme?
Y mi cuñada Sofía, ¿cómo no deja asomar por aquí sus voluminosas ubres,
ya que no por afecto hacia mí, siquiera por curiosear en estos
desórdenes de mi existencia? No es mala la politicómana, y alguna pena
tendrá de mis infortunios. Aun Segismunda y Gregorio vienen a mi memoria
despojados ya de la siniestra antipatía que nos puso frente a frente en
aquella memorable tarde. Me figuro que uno y otra deploran ya los
arrebatos que me obligaron a salir de su casa. De mis caros sobrinitos,
que sin duda confusos y tristes preguntarán por mí, también me acuerdo,
y a todos desde esta mansión de dolor envío mis ternuras\ldots{}

Divagando por los espacios del mundo que dejé, me propongo estos temas
de adivinación: ¿Sabrá Eufrasia lo que ocurre y dónde estoy?\ldots{} Y
mis amiguitas Virginia y Valeria ¿tendrán noticia de que vivo en el seno
de las tempestades?\ldots{} Sin duda la dama moruna lo ignora todo,
porque de lo contrario no me habría faltado un recadito, carta o mensaje
discreto, que bien podría ser gozándose irónicamente en mis desdichas y
cantándome el \emph{trágala}\ldots{} Corre después mi pensamiento a La
Latina, y veo a mi hermana inquietísima por lo que me sucede. A estas
horas la bendita monja o reniega de mí para siempre, o pone velas a los
santos de su predilección para que me saquen de estos malos pasos. Estoy
viendo las velas, las imágenes, y a Sor Catalina de los Desposorios de
rodillas en devota oración. Por estos espirituales caminos voy hacia mi
buena madre, y al llegar a ella, la exaltación de mis sentimientos no me
deja escribir.

A la madrugada, después de dar las medicinas a la enferma, cuidando de
no despertar a Margarita, que rendida de cansancio duerme en un sillón,
vuelvo a coger la pluma. Paseando se me ha ocurrido escribir una carta a
mi madre, para que Guillermo se la envíe, en caso de que mi contrario se
salga con la suya\ldots{} No sé qué me pasa. Hace un rato veía la carta
bien clara y completa, cual si escrita la tuviese delante de mis ojos, y
ahora nada veo. Todas las ideas se me han ido con vuelo, rápido, como
aves, como sombras, como humo, y ya no sé con qué palabras empezar ni
con cuáles concluir\ldots{} Mejor será que no escriba nada. ¿Para qué,
si Andrade no ha de poder más que yo? Me herirá tal vez\ldots{} pero
matarme, nunca. Rarísimas son hoy las muertes en desafío\ldots{}
Protesto contra la idea de mi muerte, y el duelo sería la más estúpida
de las instituciones si no se concretara a un simple alarde de valor
convencional entre caballeros\ldots{} Y pensando siempre en mi madre, lo
que me importa, si salgo en bien de estas trapisondas, es impedir por
todos los medios que a conocimiento suyo lleguen referencias de mi
conducta y desarreglada vida; que ningún nacido le lleve al desengaño
que habría de matarla; y el villano que lo llevare, sea mil veces
maldito entre los hombres, y condenado en el Infierno por veraz a mayor
suplicio que el que sufren los mentirosos.

Ya amanece. Dormiré un poquito, pues hasta las siete no vendrá Guillermo
a buscarme. ¿Qué quieres que te diga, Posteridad, al despedirme de
ti?\ldots{} ¿Me atreveré a decir: «hasta mañana\ldots?» Sí que me
atrevo, y sí en ello miento, mándame tus quejas a la Eternidad.

\hypertarget{xxii}{%
\chapter{XXII}\label{xxii}}

\emph{6 de Mayo}.---Amigos míos del tiempo futuro, sabed que no me mató
Andrade. Imagino vuestra inquietud y la impaciencia con que aguardáis el
resultado del \emph{temido choque}, y me apresuro a tranquilizaros,
declarando que vuelvo incólume a mi guarida, sin un rasguño, sin el
menor desperfecto en ninguna de las partes de mi interesante
persona\ldots{} En cambio, mi enemigo\ldots{}

Pero no quiero precipitar los sucesos, y proponiéndome que estas
relaciones remeden en lo posible los procederes de la grave Historia,
dejadme que refiera con pausa y método mi lance de honor, con todos sus
preámbulos y secuencias.

Pues cuando llegó Aransis, serían las siete, me dispuse a salir con él,
tratando de escabullirme sin que Antonia se enterase. Ni ésta debía
verme, ni Margarita conocer los motivos de mi salida en hora tan
temprana. Mas no me valieron mis precauciones, porque la enferma, que
con sagaz atención de oreja me había sentido vestirme en la estancia
próxima, me llamó con las voces más fuertes que pudo articular, y a su
lecho corrí, prodigándole caricias e inventando excusas.
\emph{«Gitano}---me dijo,---¿para qué andas en tapujos con tu
\emph{gitana?} Ya sé a dónde vas. Hoy llega de Sigüenza tu madre, y vas
a recibirla\ldots{} La galera de Padriz, que trae los viajeros de
Sigüenza, para en la calle de San Miguel\ldots{} No te descuides,
\emph{Chinito}\ldots{} Has hecho bien en ponerte levita y sombrero
\emph{góndola}, porque con tu madre viene el Obispo\ldots{} Mira, yo que
tú, a esta casa les traería, pues si tu madre viene por las ganas que
tiene de conocerme, es un suponer, véame pronto, ¡caramba! Yo estaré
vestida y peinada cuando vengáis\ldots{} Y que no sobra tiempo\ldots{}
¡Margara\ldots!» Cuantos disparates dijo la pobre mujer, fueron por mí
confirmados, para que su delirio no me estorbara la salida
indispensable, y prometiéndole volver muy pronto, nos fuimos Guillermo y
yo a nuestra fatal obligación.

El coche que en la puerta nos esperaba llevonos a recoger a Bermúdez de
Castro en su casa; de allí nos fuimos a la huerta que había de ser
teatro del lance, y por el camino me explicaron mis amigos los
concertados trámites y condiciones. El aire fresco de la mañana diome
serenidad y una confianza saludable, que me permitió afrontar la
situación con grande entereza, ni encogido ni arrogante, en el exacto
punto de la dignidad conforme a la ley de caballería. Casi al mismo
tiempo que nosotros llegaron los dos médicos, y minutos después Andrade
con sus padrinos. Conferenciaron aparte los amigos de uno y otro
campeón, nos preparamos, se marcó el terreno de la lucha, fuimos
colocados en la fatal línea, se nos dio a cada uno nuestra arma; se nos
advirtió el orden de los disparos, los pasos que debíamos dar, y\ldots{}
¡a matarse, caballeros! Esto no lo dijo nadie; lo dije yo en mi
interior, pensando que si deplorable sería que yo matase al hombre que
me había ofendido, más triste y lastimoso sería que él me matase a mí, o
me hiriese, añadiendo a la injuria el daño material. Sentíame yo muy
sereno y despejado, sin rencor hacia mi contrario, y sobre todas mis
ideas, dominaba la de conservar mi dignidad en el curso del lance
cualesquiera que fuesen sus accidentes\ldots{} Dieron la señal, disparé
yo apuntando muy alto, disparó él\ldots{} sentí pasar la bala silbando
junto a mi oído\ldots{} Avanzamos los pasos designados\ldots{} vi en el
rostro adusto de Andrade no sé qué hostil designio\ldots{} apunté menos
alto\ldots{} disparé, pensando que me sería más sensible morir que dar
muerte, y a mi disparo hizo Andrade un rápido movimiento llevándose la
izquierda mano al otro brazo sin soltar la pistola. Estaba herido: diose
la voz de alto; acudieron sus amigos\ldots{}

Había terminado el juicio de Dios, declarándolo así los jueces del
campo. Andrade y yo resultábamos igualmente caballeros, igualmente
coronados de honor y dignidad, con la diferencia de que yo estaba ileso
y él tenía una bala dentro de los tejidos del antebrazo\ldots{} Llegó el
momento de las paces por tan guerreros caminos traídas, y fui a saludar
al que ya debía ser mi amigo. Antes de que sus padrinos y su médico le
desnudasen el brazo derecho, Andrade me estrechó con efusión la mano
diciéndome: «Ya puedo asegurarle que pronuncié aquellas palabras
teniéndole a usted por otro\ldots{} por otro, no sé por quién. Yo me
había bebido media botella de champagne, y confundía nombres y caras de
personas\ldots{} Pronto conocí mi error; pero en estos casos, si uno se
desdice le toman por cobarde; no tenía yo más remedio que sostenerme en
lo dicho y aceptar el reto\ldots» Con emoción sincera le contesté que
sentía en el alma grandísima pena de haberle herido, y que debíamos
atribuirlo a la fatalidad, no a mi intención\ldots{}

Ya no había que pensar más que en retirarnos todos, rodeando al herido
de los cuidados más exquisitos hasta dejarle en su casa. Dijeron los dos
médicos que el hueso no estaba interesado, y que la bala podía ser
extraída fácilmente. Hablé con el médico de Andrade, un joven muy
simpático llamado Corral; y como yo expresara mi anhelo de tener prontas
noticias del herido, brindose Nicolás Rivero, que médico es también, a
llevármelas en el curso del día, pues a Corral no le era esto fácil,
imposibilitado del tráfago incesante de sus visitas. Emparejados
vinieron nuestros coches hasta más acá de la Cibeles, esquina a la calle
de Barquillo, donde nos separamos por diferentes rumbos, y no eran las
diez cuando volví a esta casa. Al quedarme solo con Aransis, despedidos
de Salvador Bermúdez, le pregunté por el temido asunto que tras la
solución del duelo recobraba el primer lugar en nuestros afanes, y no me
dio respuesta categórica, pues aún estaba en tramitación, con esperanzas
de un dichoso resultado. Prometió volver, y en la puerta nos separamos.
Yo subí a esta jaula donde tengo mi encierro, y no pude saborear el
término feliz del desafío, porque encontré a mi pobre Antoñita en
tristísimo estado, sin conocimiento; a Margarita llorosa, al mediquín
aturdido y rebuscando las expresiones menos aflictivas para pronosticar
la catástrofe.

Con revulsivos enérgicos devolvimos a la pobrecita cordonera una
premiosa vida, y en aquel regateo doloroso ayudaba yo la resurrección
con las palabras más tiernas que se me ocurrían, administrándoselas en
el oído para que con la virtud de ellas reviviese más pronto. Volviendo
por un instante a ser sombra o remedo de lo que fue, Antonia me dijo:
«El Obispo es el causante de que yo no haya podido ver a mi Doña
Librada.» Con disparates parecidos a los suyos teníamos que procurar su
sosiego, pues las expresiones lógicas la excitaban más. Díjome el médico
al salir que pues era tan apretada la situación, y la ciencia se
declararía pronto impotente, dejando su puesto a la fe, debíamos
preparar a la enferma para que como buena cristiana se entendiese con
Dios.

Esta inhibición de la ciencia pronunciándose en retirada, me colmó de
amargura; yo no sabía qué hacer, ni con qué fórmulas piadosas abordar a
los que deben disponerse para el trance último. Consultada Margarita
sobre el particular, puso fin a mis dudas diciéndome que en la vecindad
hay un clérigo que suele asistir a los moribundos pobres. Llámase el tal
D. Martín, y vive en el Callejón del Infierno. Margarita le conoce y
Antonia también. Propúsome la prendera preparar el ánimo de su infeliz
amiga con un caritativo embuste, para que conceptuase natural la visita
del clérigo, y así lo ha hecho esta tarde; véase cómo: Querida, ¿no
sabes a quién me encontré en la plaza hace un ratito, cuando bajé? Pues
a D. Martín, que me preguntó por ti con muchísimo interés. Díjele yo que
subiera a verte, y él dijo, dice: `Ahora no; cuando esté mejor. No
quiero molestarla'. Y yo dije, digo: `Pues mejor está, gracias a Dios y
a San José bendito. Bien puede subir cuando quiera'. Calló Margarita
esperando el efecto de su ficción en el turbado cerebro de Antonia, y
ésta, tras larga pausa, respondió: `Me alegraré que suba pronto D.
Martín, para que me descase de Sotero, pues ya me pesa este vejigatorio
de hombre pegado a mí\ldots{} ¡Y cómo apesta a vinazo!' Determinamos
llamar al cura, y discutiendo estábamos Margarita y yo la ocasión de
esta visita, cuando llamaron a la puerta, y entró Leovigildo, sobrino de
Segismunda. Al fin, mi cara familia se acordaba de mí, y me enviaba por
embajador aquel chico simpático, mala cabeza con excelente corazón y
salidas de lenguaje muy oportunas. Por él supe que allá tenían noticia
del duelo, ¿cómo no, si todo Madrid lo sabía?, y se alegraban de que yo
no hubiese tenido ni un rasguño. Se hablaba mucho de mi valor en el
lance, de mi arrogancia serena, y era motivo de general alegría que lo
hubiese roto un hueso al Sr.~Andrade, que presumía de comerse los niños
crudos.

Díjome también que en el café de los \emph{Dos Amigos} y en el de
\emph{Amato} ha corrido esta tarde la voz de que Andrade está dando las
boqueadas, y que yo soy el héroe del día en Madrid. Contome además las
historias que acerca de los orígenes del lance corrían, y en ellas he
visto cuán locamente levanta el vuelo la fantasía del público. La
versión más corriente era que Andrade había insultado a unas damas, y
que yo, sin conocer a éstas, salí a su defensa, con exaltación de
andante caballero, y de paladín del sexo débil. Eterna loa merezco yo
por tal conducta y también por mi generosidad, pues habría podido matar
a mi contrario con sólo quererlo, como que es mi puntería tan certera
que donde pongo el ojo pongo la bala, ¡anda morena!\ldots{} pero me
contenté con romperle el brazo derecho. Por fin entregome Leovigildo una
carta que habían llevado a casa. Era de la benditísima Sor Catalina de
los Desposorios, contestación a la que le escribí negándome por
conocimiento propio, \emph{ex visu et auditu}, a tragar la píldora
matrimonial que propinarme quería. No se mostraba iracunda mi hermana en
su respuesta, sino burlona y algo maleante, tratándome como a un
chiquillo, y asegurando que no tendría yo más remedio que someterme a
cuanto ella y otras personas dispusieran acerca de mí. Guapezas de monja
no me afectaban mayormente: no hice caso, y con mi amigo hablé de toros,
a que él era muy aficionado, y de teatros, mi predilecta afición.

En ello estábamos cuando entró Nicolás Rivero, que, si bien no disipó la
inquietud que yo sentía por Andrade, deshizo en un instante el embuste
contado por Leovigildo: el herido no estaba peor, y el pronóstico no era
malo. La bala, adherida al húmero, sería pronto y fácilmente extraída.
En esto pasó Leovigildo a ver a Antonia, a quien conocía, por ser hombre
muy bien relacionado en la sociedad de manolas, y Rivero me habló un
poco de política, que a la verdad no despertaba en mí gran interés. A la
curiosidad que en otro orden de ideas me manifestó, hube de responder
explicándole por qué concatenación de circunstancias anómalas me
encuentro aposentado en esta casa; y al saber que hay en ella un caso
grave de pulmonía, invocó mi amistad y su título de médico para que le
permitiese verlo y darme una opinión. Accedí gustoso, y cuando volvimos
a la sala, después de pulsada la enferma, y prolijamente examinada de
rostro y pecho, díjome que la encontraba mal, y que hiciésemos la última
prueba dándole a beber jerez superior, a ver si pega un bote la
naturaleza, ya tan caída, y se levanta. Como buen vitalista, cree inútil
combatir los síntomas y aun el trastorno general que los produce. La
medicina no es más que el arte de ayudar a la vida, y lo que no haga
ésta defendiéndose como una leona, no lo harán la Terapéutica y la
Farmacia. Si esta teoría es la única eficaz en el cuerpo humano, no lo
es menos en el cuerpo social\ldots{} ¿Qué son las revoluciones más que
pura teoría vitalista? Estas generalidades le llevaron a un nuevo
despotrique político, asegurando que España está cataléptica y necesita
de grandes sacudimientos que la despabilen\ldots{} ¡Reformas, reformas!
Es Rivero un talento viril, algo difuso, que fácilmente salta de cima en
cima, con más brillantez que método\ldots{} Oí con gusto su lengua
ceceosa, que al despedirse me dijo: «Ya ze verá si dezpertamo al dormido
y rezuzitamo al muerto\ldots{} Quédeze con Dios, y hazta que noz veamo
por el mundo\ldots{} o en el valle de Jozafá.»

En la puerta se cruzó Rivero con un sacerdote que entraba. Saludó el
andaluz, el clérigo no, y entró en mi casa como en la suya, diciéndome
con fría confianza y sin ningún preámbulo de urbanidad: «¿Se muere esa
niña o no se muere?\ldots» Metiose adentro, y yo tras él, asombrado de
sus extraños modos. En la desmantelada salita donde escribo nos hallamos
frente a frente, y él, sin quitarse la teja, cogió un botón de mi levita
y me dijo: «Aquí me tiene a la disposición de esa enferma y de usted. Yo
me llamo Martín Merino, soy riojano, y no gasto cumplidos. Como tengo
pocos quehaceres, volveré si ahora no es oportuno\ldots{} Ya sabe
Margarita dónde estoy: que me llamen a cualquier hora de la noche. Yo no
duermo\ldots{} quiero decir, duermo muy poco\ldots{} ¿Y usted está
bueno? Lo celebro\ldots{} Con este tiempo variable andan los cuerpos
trastornados, y las cabezas más, más las cabezas.»

\hypertarget{xxiii}{%
\chapter{XXIII}\label{xxiii}}

\emph{8 de Mayo}.---La precipitada serie de acontecimientos que cayeron
sobre mí, con ruido y azote de pedrisco pavoroso, me han impedido tomar
la pluma. Hoy tengo que recoger y archivar todo lo que vino con
abundancia no proporcionada a la brevedad del tiempo, y he de andar
despacio y atento para que me asista mi buena memoria en la reproducción
exacta de tanto dolor y sorpresas tantas, así como en el orden que
traían.

Enlazo este relato con el último hilo del antecedente, diciendo que
aquel clérigo buscado por Margarita para la espiritual asistencia de
Antonia, me pareció muy extravagante. Pasó a ver a la enferma, y
hallándola dormida tornó a la sala, y como yo le invitase a tomar alguna
cosa (de lo que mandé traer para reparo de mi cuerpo desfallecido),
contestome: «Gracias, señor: yo no como\ldots{} quiero decir, como muy
poco. Hablele yo de las dificultades y sinsabores de su ministerio, y me
dijo que él es pobre y que vive con gran estrechez. Como yo le indicase
que debía proporcionarse una prebenda, respondió que, aunque le sobraban
amigos poderosos, ni pretendía nada, ni eran de su gusto las altas
posiciones eclesiásticas. \emph{Odivi ecclesiam malignantium}---me dijo
con fácil expresión latina,---\emph{et cum impiis non sedebo}; o más
claro: aborrezco la congregación de los malignos, y entre impíos no he
de sentarme.» Otros muchos latines hubo de soltar en el transcurso del
diálogo, y explicó su erudición con estas palabras: «Perdone usted que
le hable así: me sé de memoria los Salmos del ritual, y sin quererlo,
todo lo digo por boca del rey David.»

En esto entró Aransis, cuya visita deseaba yo como agua de mayo, y D.
Martín se fue a la alcoba llamado por Margarita. Antes que yo le
preguntara, me dio mi amigo el notición de que había resuelto el
conflicto pecuniario del modo más ingenioso. ¿Cómo? Le dejo hablar, y
así será más fácil la explanación del caso. «Pues me sacó del compromiso
nuestra amiga \emph{Doña Manolita la Cuca}. Cuando estalló en el Casino
tu cuestión con Andrade, yo no estaba allí: ya lo recordarás. Me había
ido a probar fortuna en casa de las \emph{Cucas}; allí encontré a las
dos \emph{pájaras} de Mora, a doña Berenguela, a las \emph{piculinas};
estaban también Pepe Cruz y otros amigos: hallé todo lo de costumbre;
pero no a la Fortuna, que aquella noche no quería cuentas
conmigo\ldots{} En mi rabia, tuve una inspiración, y cogiendo a Doña
Manolita, me la llevé al gabinete amarillo, ya sabes, donde está el
retrato del que dice fue su padre, el caballerizo de Carlos IV, y las
vistas de los Reales Sitios; y tales discursos le eché y tan elocuente
estuve, jurando que me pegaría un tiro si no me amparaba, que la
conmoví, chico, figúrate, y empezó a echar suspiros y a despintarse las
ojeras con el pañuelo. Díjome que no podía darme un maravedí, que lo
siente en el alma, \emph{etcétera}, \emph{etcétera}. En fin, ayer al
mediodía, después del duelo, volví allá con mi cantinela, y tales
extremos hice, que la señora \emph{Cuca} se arrancó con un rasgo de
bondad heroica y me dijo: No tengo el dinero; pero ahora mismo voy a
pedirlo. Si me lo dan, en tus manos estará esta tarde, Guillermito de
mis entrañas\ldots{} Pues ¿sabes a quién pidió el dinero y quién se lo
dio?\ldots{} No adivinas: tu hermano Gregorio\ldots{} mejor dicho, no
fue él, sino Segismunda, quien nos ha favorecido. Por supuesto, no sabe
que es para nosotros. Segismunda suele dar sus ahorros a \emph{la Cuca},
que se los devuelve muy aumentados casi siempre\ldots{} Bueno: para no
cansarte, Doña Manolita, guardándonos el secreto, nos exige que le
firmemos un documento, obligándonos a devolverle los cuartos el día 20,
y aquí traigo el papel para que pongas tu firma junto a la mía.
Conque\ldots{} De aquí al 20 ya tenemos un buen respiro, y si no
pudiéramos cumplir con la \emph{Cuca}, ya nos esperará, y si no, que se
la lleven los demonios.»

Pareciome la solución muy feliz, porque en catorce días bien pueden
venir infinidad de contingencias favorables: que nos caiga la lotería,
que encontremos un tesoro, o que lluevan doblones. Luego me contó
Aransis que en la tertulia de las \emph{Cucas} había oído rumores de
tormenta, es decir, de revolución próxima. Suelen ir al cenáculo de la
cuquería progresistas de los más inquietos; aquella noche estaban
presentes dos tan sólo, y la gravedad de los futuros acontecimientos se
colige de lo que aquéllos dijeron, y más aún de la ausencia
significativa de los que faltaban. «Doña Manolita---dijo por fin
Aransis,---me aseguró, al soltarme la mosca, que están en puerta los
progresistas, porque las tropas que ahora se subleven no se pararán en
pelillos, y obligarán a Su Majestad a poner en la calle a Narváez. No lo
siento más que por mi abuela, que cuando Narváez no está en el poder,
cree en el fin del mundo, y se pone de un humor tan endiablado, que
sacarle dinero es más difícil que extraer aceite de un ladrillo.»

Reapareció el clérigo, que había echado un parrafito con Antonia, y me
dijo: «No está la pobre en disposición de confesarse; pero , se
confesará. \emph{Revela Domino viam tuam, et spera in eo.»}

Mirábale atentamente Guillermo, examinando su cara lívida, pomulosa, sus
ojos ratoniles; midiole de pies a cabeza con sagaz mirada, y al fin,
evocando recuerdos, llegó a la filiación incompleta del estrafalario
sacerdote. «Perdone, señor cura. ¿No he tenido yo el gusto de verle en
casa de Don José de Olózaga? ¿No es su nombre\ldots?

---Martín Merino---respondió el clérigo inclinándose,---y en casa de
Pepe Olózaga me habrá usted visto\ldots{} el gusto es mío: allá suelo ir
algunos ratos\ldots{} También conozco a Salustiano, aunque no le trato
como a Pepe. Riojanos somos: ellos de Ocón, y se criaron en Arnedo, que
es mi pueblo para serviles.

---Pues dígale usted a su amigo y paisano que ahora se armará de
veras\ldots{} Aunque él puede que lo sepa mejor que nosotros, porque
estará en el ajo\ldots{}

---En el ajo están todos los que miran a las cosas pequeñas y no a las
grandes.

---¿Cree usted que triunfará el Progreso?

---Yo no creo nada\ldots{} Y el Progreso ¿qué es? Lo que yo creo es que
el mundo será de los pacíficos\ldots{}\emph{Mansueti autem haereditabunt
terram, et delectabuntur in multitudine pacis}.

---Pues usted es de los mansos que triunfarán y gozarán la paz, como uno
de los pocos progresistas que visten sotana. No será mala canonjía la
que le darán a usted los Olózagas cuando venga la revolución.

---¿A mí?\ldots{} No me hará daño. \emph{Verba oris ejus iniquitas et
dolus}\ldots{}

---Pero ¿de veras no es progresista?

---Yo nada soy.

---¿Ni siquiera masón?

\emph{---Nihil.}

---¿No cree usted que la Reina dará pronto el poder a los progresistas?

---¿Yo qué sé de eso? Y pregunto\ldots{} ¿quién es la Reina? En los
Estados no me pongan monarcas con faldas, sino Rey macho. Yo hablo
siempre del Rey.

---Entonces es usted carlista.

---Yo no\ldots{} Creo en un soberano.

---Y de ese soberano ¿qué opina?

---Poca cosa. \emph{Iniquitatem meditatus est in cubili suo: astitit
omni viae non bonae}\ldots{} «En su cama medita iniquidades\ldots{} anda
en malos pasos.»

---¿Es eso salmo? ¿Y qué tiene que ver con lo que hablamos?

---Nada. Por eso lo he dicho. Sabrán ustedes que yo no hablo, quiero
decir, que hablo poco.

---Y usted mismo no se entiende. ¿Está seguro, Sr.~Merino, de tener la
cabeza buena?»

Esto le preguntó Aransis, y él vacilaba en la contestación, rezongando
al fin: «Buena o mala, no tengo otra.»

Callamos. Acudí a mi pobre Antonia, que me llamaba. Prometile que de
ella no me separaría, y me repitió sus protestas de eterno amor en tono
y estilo de niño quejumbroso. Aseguraba que ya no le dolía el pecho, y
que durmiendo acabaría de curarse; tomaba aliento a cada dos palabras,
en las cuales el acento infantil, de truncados términos y sílabas
primarias, se iba marcando como si los minutos que transcurrían le
quitasen años y días, tornándola a la edad más tierna. Cuando calló,
cerrando los ojos, volví a la sala, y encontré solo a Guillermo. El cura
se había ido, prometiendo volver a la tarde.

Solo y en tenebrosa tristeza estuve en la tarde del 6, pues la compañía
del presbítero D. Martín no era la más propia para mitigar con dulces
coloquios mi pena. Hablábale yo de su ministerio, procuraba sondearle y
descubrir qué clase de espíritu bajo tan extravagantes formas y estilo
se escondía, y a todo me contestaba con versículos de Salmos, no siempre
aplicados con oportunidad a lo que decíamos\ldots{} Tan marcados vi en
la pobre Antonia los signos de su próximo acabamiento, que deseché hasta
las últimas esperanzas que en mi alma querían entrar. ¿A qué esperanzas,
si no había remedio, como no fuera la cristiana resignación? Largo
tiempo estuve a su lado, recogiendo con avaro afán cuanto me decía en
fugaces, desconcertadas, infantiles expresiones: \emph{«Tero
agua\ldots{} tero mimir\ldots{} daca mano tuya\ldots»} Con modulaciones
sólo por mí entendidas decíame que le limpiara la boca del agua que
bebía, la frente del sudor, y que no quitase de su cuello el brazo mío
que le servía de almohada. Serían las cuatro cuando me dijo: \emph{«No
veo a ti, gitano\ldots{} tae luz\ldots{} ¿Por qué tanto oscuro?\ldots»}
La besé una y otra vez, y ella intentaba contestarme del mismo
modo\ldots{} Sus labios no podían ya besarme. Cayó en profundo sopor de
agonía. No había nada que hacer, más que contemplar con dolor callado su
muerte. Traspasado de aflicción, apoyé mi rostro en el lecho; mas D.
Martín me sacudió la cabeza diciéndome: «Atienda, señor: ya concluye.»
Atención puse, y en unos segundos de suprema ansiedad recogí el último
aliento de la pobre Antonia. El cura, de rodillas, encomendaba en alta
voz el alma, y Margarita lloraba sin consuelo. El tiempo flotaba
silencioso entre las cuatro y las cinco de la tarde.

Mi tribulación y desconsuelo eran grandes, pues ya no podía ver las
desazones y enojos que por aquella mujer sufrí, y tan sólo veía el
generoso ardor de su corazón amante, su ingenua, inquebrantable devoción
de mi persona, que más bien era un culto idolátrico. La lloré con el
alma por el amor que me tuvo, y del cual seguramente era yo indigno. Las
incongruencias sociales, contra las cuales nada podemos, fueron las
causas de que aquel amor no tuviese en mí la debida correspondencia, y
de que su ser y el mío no llegaran a la soberana fusión para la que sin
duda habíamos nacido. ¡Pobre Antonia! Error suyo fue amarme; mayor
dislate mío dar alientos a su afición. Yo no merezco piedad del Cielo
por esta falta, y si aquí tienen proporcionado castigo nuestros errores,
no me faltará en la vida que me resta mi parte de Infierno.

Partió el clérigo; acomodamos Margarita y yo en su lecho a la pobre
muerta, la cabeza sobre mullidas almohadas, el martirizado y ya
insensible cuerpo extendido y envuelto en sábanas limpias, y aun no
sabíamos cómo amortajarla, porque el vil marido, entre los efectos que
sustrajo se había llevado el traje negro, medias y zapatos, y las
mejores prendas de ropa que la infortunada mujer poseía. Acordamos al
fin que para vestirla traería la prendera ropa blanca de la suya, y lo
necesario para calzarla decorosamente, y que luego le pondríamos el
hábito del Carmen, por ser esta advocación de la Virgen la más firme
devoción de Antonia. Las seis serían cuando salió Margarita en busca de
la fúnebre vestimenta y de las velas que habíamos de encender junto al
cadáver; yo, solo en la casa, quedeme sentado junto al lecho mortuorio,
contemplando la marchita belleza, que aún conservaba sus lindas
facciones sin la menor descomposición de líneas, como vaciadas en
transparente cera. Tardó mucho Margarita en su diligencia; llamaron al
fin a la puerta, y seguro de quién era, salí y abrí\ldots{} ¡Dios mío,
qué estupor!

La sorpresa dejome paralizado, mudo. Era Eufrasia la que ante mí
apareció en traje muy sencillo, como de ir a la iglesia, con el libro de
rezos en la mano. «Supe que no puede usted salir de aquí---me dijo
trémula,---por\ldots{} vamos\ldots{} esa mujer enferma\ldots{} he
querido saber de usted\ldots{} informarme\ldots{} Alguien ha dicho que
estaba usted herido\ldots» Le señalé el paso, la conduje a la salita, y
ella entró con recelo, temerosa de miradas impertinentes. En mi rostro
debió de leer mi consternación. «¿De veras no resulta cierto lo de la
herida?---me preguntó ya en la sala, negándose a aceptar el sillón que
le ofrecí.---Gracias: no me siento. ¡Si me voy ahora mismo! He salido al
rosario. Acabo de rezarlo en Santa Cruz, y\ldots{} Por Rafaela, que todo
lo sabe, supe anoche el número de esta casa, el piso, y he
subido\ldots{} Subo un momento con el único fin de\ldots{} Me dijeron
que esa señora está muy malita, en peligro de muerte, y, naturalmente,
la situación de usted en esta leonera es poco agradable. Los buenos
amigos deben prestarle su apoyo, ver si en algo pueden servirle\ldots{}
No se asombre usted tanto de verme aquí: sé que es una imprudencia, un
desatino\ldots{} pero antes que mandarle un recado, he querido venir en
persona\ldots{} ¿Y es de veras que está usted solo, enteramente solo con
la enferma\ldots?»

Díjele que estaba solo con la muerta, y por la puerta de cristales que
con la alcoba comunicaba le mostré el lecho, del cual se veía la parte
de los pies, y el bulto de los de Antonia cubiertos por la sábana.
Grande impresión hizo en Eufrasia el ver en la penumbra los pies de la
yacente estatua, como incipiente escultura en el bloque de mármol, y sin
expresar su consternación más que con un ¡ay!, dejose caer en el sillón
próximo, cerró los ojos, y se llevó a la frente el libro de rezos, como
si con él quisiera persignarse. «Mi dolor no lo comprenderán muchos---le
dije;---usted sí lo comprenderá. Antonia me amaba\ldots{} No era su amor
de los que se amoldan a los respetos y se someten al artificio social;
era un amor que llamaríamos loco, revolucionario, que no reconoce más
ley que la de sí mismo. Fue mi suplicio cuando ella vivía, y ahora que
la he visto morir, es mi remordimiento. Yo no era digno de un cariño tan
hondo, tan puro, tan superior a todo interés y a las conveniencias
humanas. ¿Verdad que no lo merecía yo? ¿No piensa usted lo mismo?

---Ciertamente, no era usted digno\ldots---respondió la dama morisca,
echando atrás la cabeza y dejando caer sus dos brazos sobre los del
sillón.---Nadie que viva en sociedad es digno\ldots{} de eso\ldots{} Ni
esas pasiones tan a lo primitivo caben en los moldes de nuestra
vida\ldots»

En esto llegó Margarita con velas y ropas. Eufrasia turbose un poco al
verla; yo la tranquilicé, asegurándole la discreción y delicadeza de la
que había sido mi auxiliar en aquellas tribulaciones. Mostró la prendera
el hermoso hábito del Carmen que había comprado, y Eufrasia, con un
arranque de valor y piedad, que fue mayor brillo de su belleza, se
levantó y me dijo: «Viva no la vi nunca\ldots{} quiero verla
ahora\ldots» Antes que yo me decidiese a ser acompañante de su
curiosidad, Margarita le franqueó el camino, andando delante de ella.
Entraron en la alcoba. Yo vi a Eufrasia desde la sala, fijando sus
miradas en el rostro marchito cuando la otra con pausa y respeto
cariñoso levantó el blanco lienzo que lo cubría\ldots{} Durante un corto
rato, las dos mujeres no estuvieron mudas. Sus cuchicheos lo mismo
podían ser comentario de admiración que afligidos rezos\ldots{} Volvió a
mí la manchega con el rostro mojado por las lágrimas que de sus ojos
corrían; dejó el devocionario en la mesa donde yo escribo, se quitó los
guantes y la mantilla, y me dijo: «Hermosa fue sin duda, y aun muerta
está guapísima\ldots{} ¡Pobre corazón amante! Por amar con tanta
independencia y con tanta fe, despreciando el mundo y toda vanidad,
merece mi simpatía\ldots{} Usted y esta buena señora me
permitirán\ldots{} No se asombre, Pepe. Quiero amortajarla.»

\hypertarget{xxiv}{%
\chapter{XXIV}\label{xxiv}}

Mientras Eufrasia y la prendera se consagraban sin descanso a su piadosa
obra, entró Aransis que venía a traerme dinero, tan necesario para mí en
los días fúnebres como en los alegres días. «Márchate ahora mismo---le
dije,---que hay aquí una señora, mi amiga, a quien no gustará que la
veas.» Invocó él nuestra amistad, que no admitía secretos entre los dos,
para que yo abriese un poco la mano en la confianza; mas no accedí a
ello, y que quieras que no, le expulsé con recomendación enérgica de no
atisbar en la calle la salida de la dama\ldots{} Terminado el acto de
vestir a la pobre muerta, Eufrasia volvió a ponerse mantilla y guantes.
Su palidez intensa declaraba su grande emoción. «Está guapísima---me
dijo,---y la toca blanca da a su rostro una expresión enteramente
mística. Nunca, por mucho que viva, olvidaré esa cara, que tan muerta y
callada me ha dicho cosas muy bellas\ldots{} Yo también le he dicho a
ella\ldots{} algo que sólo se dice a los que no pueden oír\ldots{} con
los oídos naturales.» Encargome luego, camino de la puerta, que en
cuanto volviese yo a la vida regular fuese a verla, pues tenía que
hablarme de cosa urgente: hablaríamos en mejor ocasión y lugar.
Prometile ponerme a sus órdenes muy pronto, y con ella bajé, pues no
quería que en escalera tan bulliciosa tuviese encuentros de gente
grosera y de chiquillos importunos. En el portal nos despedimos,
reiterando yo mi gratitud por su visita, y ella los honores de su
amistad, que en aquel día por especial gracia éranme de nuevo
concedidos.

Al subir, sentí pasos detrás de mí. Volvime y encaré con Sotero, que
llevándose un pañuelo a los ojos, me dijo:

---Don José, lo supe hace un rato\ldots{} por el bruto del
cerero\ldots{} ¡ay Jesús!, que vendió a Margarita las velas.

---Sí, hombre: la pobre Antonia, cansada de sufrir, se nos ha ido a otro
mundo mejor\ldots{}

---¿Y a usted le \emph{costa} que es mejor? Yo no lo sé, ni lo sabe
nadie, como se dice. En fin, yo he sido malo, y la última que hice no me
la perdonó Toña.

---Sí, hombre: te perdonó. No llores por eso\ldots{} ¿Necesitas algo?

---No quiero cansarle ahora. Subiré, si me necesita. De usted para mí,
le digo que es mi deber velarla.

---Hombre, no; te fatigarás. Más que para velar estás tú para que te
velen. Tienes cara de no haber comido desde anteayer.

---Así es, D. José; pero yo nada le pido en esta circunstancia, Dios me
libre\ldots{} Si me apetece subir es por velarla: que yo seré todo lo
perro que quieran, pero tocante a buen cristiano, lo soy como el
primero.

---Eso te honra, Sotero---le dije dándole para comer.---Pero atiende
antes a la necesidad de vivir\ldots{} No tendría gracia que también tú
te murieras ahora\ldots{} Come y bebe esta noche; duermes los garbanzos,
y de madrugada vienes a velar a la pobrecita Antonia\ldots{} Así
alternamos: yo descansaré cuando amanezca, y a fe que estoy rendido.

Tomó lo que le di, y prometiendo volver a las altas horas de la noche,
se despidió con una caballerosa manifestación, la mano en el pecho, los
ojos húmedos, la palabra balbuciente: «Sé cumplir el cometido de mi
deber. Velaré esta noche, y mañana la llevaré hasta el propio
cementerio, como se dice, camposanto, que ésta es mi obligación, D. José
de mi alma, como marido que soy del cadáver.»

A poco de esto, cuando ya teníamos a la pobre Antoñita enteramente
ataviada de muerte, en un lujoso ataúd, llegaron otros parientes, y
lloráronla todos y compadecieron su temprano fin. Era un dolor verla
partir en la edad florida y dichosa. Trajeron algunas flores naturales
muy lindas, con que la adornamos, poniendo en ello un cuidado y esmero
tan grandes como si adornáramos a un vivo. «Aquí hacen mejor las
violetas\ldots{} Las rosas coloradas entre las manos\ldots{} Con las
rosas blancas formemos un cerquillo en derredor de la cara.»

Pasada medianoche, volvió Aransis. Él y yo, en el desmantelado comedor,
cenamos algo, buenos fiambres que trajo un criado suyo, y bebimos de un
rico burdeos. El reparo de mi desfallecimiento me produjo un sopor
intensísimo: no vi salir a Guillermo, no vi nada, porque me quedé
dormido en la silla, recostando mi cabeza en el ruedo que hacían mis
brazos sobre la mesa\ldots{} Fue mi sueño como indigestión cerebral de
las imágenes que en aquel día y los precedentes habían pasado ante mis
ojos. Y como entre estas imágenes descollaba la yacente figura lastimosa
de mi pobre Antoñita, vestida del hábito del Carmen y de cirios
humeantes rodeada, esta visión no me abandonó en todo el espacio de mi
sueño, harto parecido a la embriaguez cebándose en el cansancio. Vi el
cuerpo de mi amada en un alto y aparatoso túmulo a la romana; las velas
se trocaron en antorchas, y el religioso traje en túnica de vestal. Vi
que todo ello se alzaba sobre un monumento de formas ondulantes y
cartilaginosas, en nada parecidas a las clásicas formas de arquitectura;
vi un conjunto armónico de tallos y miembros vegetales, con flores muy
abiertas de monstruosa sencillez. «¿Será esto---me dije yo soñando,---el
tipo de un arte que, andando los siglos, vendrá potente a derrocar los
tipos y módulos que hoy componen nuestra arquitectura y nuestras artes
decorativas?\ldots»

Seguramente, los funerales que en torno de este gran túmulo se hacían a
la pobre cordonera eran espléndidos, con asistencia de innumerables
sacerdotes de no sé qué religión, y de un gentío inmenso, cuyas voces
turbaban mis oídos. Era un estruendo parecido al del mar bravo, que va y
viene, azotando las rocas y plegándose con espumante ira sobre sí mismo.
No podía yo entender lo que decían aquellas voces, ni supe si eran
himno, plegaria, o quejumbrosa oración fúnebre\ldots{} Y luego sonaron
salvas, que a mí me parecieron el más natural ornamento de aquel acto.
Oí un disparo, luego dos, en seguida muchos, sucediéndose y acelerándose
como las notas de una tocata que empieza en \emph{adagio} y acaba en
\emph{presto, prestísimo}\ldots{} ¡Vaya un traqueteo y estallido de
ingenios de guerra! En las tinieblas de mi sueño empezaba yo a
sorprenderme de que las lucidas exequias se celebraran con función
pirotécnica o juego de pólvora. ¿Sería esto también un arte funerario
del porvenir, llamado a reformar los actuales modos de honrar a los
muertos?\ldots{} No sé cuánto tiempo duraron estas impresiones y ruidos
disparatados\ldots{} Ello es que yo iba despertando, y mis sentidos se
mecían entre el sueño y la realidad sin que cesaran los disparos, o al
menos sin que dejase yo de oírlos. Una mano vigorosa sacudía mi hombro,
y lo primero que oí claramente fue la voz de Margarita, que decía: «No
le despiertes, ganso.»

El que me despertaba era un ser de pesadilla, odioso y repugnante. Tardé
un rato en reconocer al maldito Sotero, esposo de la difunta. Era él, él
mismo, desfigurado por una corbata de luto mal liada a su pescuezo, las
greñas en desorden, la cara sin lavar en tres días, el cuerpo en mangas
de camisa, con un chaleco negro que por la holgura parecía de otra
persona. Con tabernaria voz graznaba: «Despierte, despierte, D. José,
que hay revolución.»

«¡Revolución!» Yo me erguí en un desperezo total, queriendo sobreponerme
a mi cansancio. Vi la claridad del día. No creía nada de lo que en torno
de mí se hablaba. Mujeres medrosas decían: «¡Ay, señor, qué Infierno en
la plaza!\ldots» «Venga al balcón y verá\ldots» «La tropa sublevada por
aquí, y enfrente, asomando por el arco de la calle de Toledo, la tropa
del Gobierno\ldots» «Que no salga al balcón; no le suelten un tiro.»
Como el tumulto que de la plaza venía no cesaba, tuve que rendirme a la
evidencia. Soñoliento me asomé al balcón, y en la plaza vi un hormiguero
de soldados y paisanos que parapetados tras montones de piedras hacían
fuego contra otros que en el arco les atacaban. El tiroteo era tan vivo,
que hube de cerrar a escape\ldots{} «Por Dios, señor---dijo una
mujer,---no se asome: cierre vidrios y maderas, que a un vecino del 7,
que se asomó a guluzmear, le han dejado seco.»

Mandé cerrar a piedra y barro, y esperamos\ldots{} ¿En qué pararía toda
aquella gresca? Los parientes de Antonia y otras vecinas aquí
congregadas, se complacían en ilustrarme acerca de aquel hecho político,
que pronto había de ser histórico. Lo que quiere ahora el Progreso es
poner la República y quitar a la Reina, pues la República no es otra
cosa que un Gobierno \emph{todo de hombres}, sin Rey ni Reina, ni cosa
ninguna de Majestad\ldots{} Según afirmó un vejete que entre las mujeres
rebullía, el propio Narváez mandaba la fuerza que abrasaba a los
patriotas\ldots{} Éstos se defendían sin coraje, por no contar con toda
la tropa comprometida, y ello acabaría mal, fusilando a medio Madrid y
cargando de cadenas al otro medio\ldots{} También se dijo que estas
marimorenas no son de nuestra invención, y que todo viene armado de
fuera, de la Europa y de las naciones extranjeras, que están toditas
revolucionadas y dadas a los demonios. El Reino de Nápoles arde; el
mismo Papa no ha tenido más remedio que largar una constitucioncita para
sosegar a los masones; otro Rey italiano, D. Carlos Alberto, va contra
el Austria, para quitarle unas provincias que ya son italianas, ya
tudescas; y un país que se llama la \emph{Hunguería}, porque de él
vienen los húngaros, anda también muy revuelto con un demonio de hombre,
de apellido \emph{Cosuto} (Kossuth), el cual predica la libertad, la
religión libre y otras monsergas libres. La \emph{Hunguería y la tierra
de los austriacos} no son lo mismo, pues la una linda con las Américas,
y la otra es propiamente como una familia real, por lo cual, nombrando a
nuestros Reyes de antaño, se dice la \emph{casa de Austria}\ldots{}

A todos los presentes prohibí que abriesen las tres ventanas de la casa,
y en la sala nos quedamos en fúnebre penumbra, mortecina claridad de los
cirios, que ya gastados se derretían en gruesos cuajarones. La faz
hermosa de Antoñita se descomponía de hora en hora, tiñéndose de una
lividez tristísima. La contemplé largo rato, recogí sobre el rostro la
plegada toca, añadí flores en derredor, y al volverme di de manos a boca
contra Sotero, que mostrándome su atavío me dijo: «Vea, D. José, que así
no estoy decente, y que me van a criticar por no presentarme como
requiere la defunción. Pedí a Dimas, el tabernero, que me prestase ropa
negra, y no ha podido encontrar más que este pingo de chaleco y la
corbata\ldots{} Yo se lo digo al Sr.~D. José para que vea el
ridículo\ldots{} mi ridículo ante la vecindad y ante la comitiva del
féretro. A usted no le ha de gustar que me vean así\ldots{} Soy, como se
dice, el esposo de la finada, y si no estoy todo puesto de luto
riguroso, pero muy riguroso, ¿qué dirán, D. José, qué dirán?\ldots{}

---Bueno, hombre, ya lo arreglaremos. Déjame ahora.

---Mi parecer es que debemos apañarnos como Dios manda, para que no
tengan que criticar\ldots{} Yo sé de un sastre que alquila ropa de
entierros, y allí se puede vestir uno para toda la pompa enlutada que se
ofrezca. Con que, si quiere, allá me voy\ldots{} y pido precios\ldots{}

---Está bien. Irás cuando se concluya la gresca en la plaza. Ahora no se
puede salir\ldots{} ¿Y cómo va eso?

---Parece que van ganando los de Narváez. Ya no atacan tan sólo por la
Sal y por Atocha, sino también por los Portales de Bringas. En una casa
de la calle Mayor con balcones que caen a la Plaza, junto a la
Panadería, metieron tropa, y ya están largando tiros desde el piso
segundo\ldots{} Oiga, Don José: yo he traído mi pistola y pólvora. Si
quiere que dispare desde el balcón contra los republicanistas, verá qué
pronto pongo a dos o tres patas al aire.

---No, no: aquí somos neutrales. Vencerá el Gobierno. No tomemos partido
ni por la revolución ni por el orden.

---Yo estoy siempre con el Orden, y por esto hay en la vecindad más de
cuatro que no me tragan. En la taberna de la calle Imperial oí ayer
tarde runrunes, y así como latines masónicos\ldots{} Me dio en la nariz
olor de chamusquina, y me traje la pistola por lo que pudiera tronar.»

Entreabierto el balcón, noté gran desorden en la plaza y que el tiroteo
era menos vivo. Vi grupos que huían por la calle de Botoneras, próxima a
esta casa\ldots{} Pasado un rato, hallábame en expectativa de nuevos
incidentes y sorpresas, cuando Margarita, muy asustada, vino a decirme
que un señor había preguntado por mí en la puerta, y que sin esperar a
que se le mandara pasar, habíase colado muy resuelto en el pasillo. Salí
al instante, y me encontré con Nicolás Rivero, bastante desordenado de
ropa, que sin ningún preámbulo, ni la menor alteración en su rostro
cetrino y ceñudo, me dijo: «¿Puedo ezconderme aquí, Pepito? Me ha dicho
eza zeñora que a la otra zeñora la tenemoz de cuerpo prezente. Lo
ziento\ldots{} Pero no podía yo zoñar mejor ezcondite.»

Ofrecile todo mi amparo con la mayor cordialidad, y me le llevé al
comedor, donde podíamos hablar sin testigos: «Ezto ze acabó\ldots{}
Adioz mundo amargo.

---Es la primera revolución que veo en Madrid, y la verdad, me ha
parecido una fiesta de pólvora. ¿Es siempre así?

---Ziempre azí\ldots{} tropa contra tropa\ldots{} el pobrecito pueblo en
medio\ldots{} ¡Pueblo crucificado!\ldots{} Dígame: ¿el entierro zerá
ezta tarde? Bonita ocazión para zalir y ezcabullirme\ldots{} por donde
ze pueda\ldots{} Dizpénzeme que me alegre del entierro\ldots{} La
humanidá ez azí\ldots{} Del llanto zale la alegría.»

Dicho esto, renegó de los que no acudieron al puesto de peligro, y tronó
contra Narváez, contra Figueras, Fulgosio, Lersundi y demás instrumentos
del Orden\ldots{} El Orden por sí no es nada, y cuando se ejerce contra
la voluntad del Pueblo, es el Desorden con insignias usurpadas\ldots{}
El Pueblo ama la Libertad\ldots{} sólo que no le dejan
manifestarlo\ldots{} ¿Pues la tropa? ¿Qué es la tropa más que Pueblo con
uniforme?\ldots{}

Entró Sotero a decirme que los soldados de Lersundi ocupaban la plaza, y
que los insurrectos huían por la calle de Toledo. Metíase aquel bestia
con grosero desenfado en nuestra conversación, por lo cual hube de
tenerle a raya; llevémele a un cuarto próximo, y después de prohibirle
salir a la calle, ni aun con el razonable motivo de procurarse ropa de
luto, le pedí su pistola; diómela con la polvorera, rezongando, y en mis
manos el arma, le dije: Como suban polizontes o militares en pesquisa de
algún paisano refugiado aquí, y tú pronuncies una sílaba sola delatando
a este joven, te levanto la tapa de los sesos. De aquí no me sales hasta
la hora del entierro, si nos permiten que sea esta tarde. Margarita
alquilará la ropa de luto, la cual se pondrá Rivero, después de bien
afeitado, figurando como esposo de la finada. Tú quedas relegado al
puesto de \emph{primo del cadáver}, y te vestirás con las ropas que te
traerá Julián, el cordonero de la calle de Bordadores. Cuidado, Sotero,
con lo que haces y dices mientras estés aquí. Ha de venir el celador del
barrio para tomar nota del nombre de la difunta, \emph{etcétera}, de la
hora y lugar del enterramiento: no salgas tú a recibirle; saldré yo, y
diré lo que se ha de decir, lo que me dé la gana, y tú te callas, que
aquí no eres nadie, ¿lo entiendes bien?\ldots{} Si no nos permiten que
la llevemos esta tarde, ya veré lo que se ha de hacer mañana. Y no se
hable más, Sotero: silencio y obediencia, o ten por seguro que te mato.

Con gruñidos iba marcando a cada frase su bárbara sumisión a mis
órdenes.

\hypertarget{xxv}{%
\chapter{XXV}\label{xxv}}

\emph{14 de Mayo}.---Pasó la tormenta, dejando en mi alma gran destrozo,
árboles caídos, caminos deshechos, ruinas y cambios lamentables. Termino
las referencias del día 8, manifestando que todo lo presupuesto se hizo
con arreglo al programa: en un nicho de la Sacramental de San Andrés
guardamos los restos de la enamorada Antoñita, a quien debo en estas
Memorias enaltecer singularmente por su devoción de amor y sus arrebatos
afectivos, sin mentar sus pecados y errores, que de ellos no pudo verse
libre quien tenía la pasión y la fragilidad por componentes del alma. Y
el acto de conducirla a su última morada me sirvió para proporcionar
fáciles medios de ocultarse al amigo Nicolás Rivero, que temía los
rigores de la policía por haber metido sus narices en aquel fregado de
la plaza Mayor. Liquidé cuentas con Margarita, cuentas con Sotero, a
quien di cuanto me pidió a condición de que no volviera jamás a
ponérseme delante, y abandoné la triste casa en que apurado había tantas
amarguras.

Volví fatigado al mundo y a la vida corriente, instalándome en casa de
Agustín, y mi primera visita fue para Andrade, a quien encontré muy
mejorado de su herida, de lo que recibí gran satisfacción. Dos amigos
míos, Uhagón y Pepe Arana, en su compañía estaban, y poco después que yo
entró el que con Rivero había sido su padrino, Sánchez Silva. Del
ruidoso escándalo militar del día 7 hablamos los cinco, y allí me dieron
exacto informe de su móvil inicial y de los pormenores que yo no había
visto. Como apenas pongo atención en las cosas políticas, ignoraba el
argumento del confuso drama cuya principal escena, si no la más trágica,
fue representada tan cerca de mí. Había sido Ruiz de Arana testigo y
actor muy principal en la marimorena, por parte del Gobierno. Él vio a
los soldados de \emph{España} bajar en desordenado tropel por la calle
de la Montera; él corrió de una parte a otra con una sección de
coraceros, llevando órdenes del capitán general Fulgosio; él le vio caer
miserablemente en la Puerta del Sol, a los tiros del paisanaje; él con
tesón juvenil se halló en todos los sitios donde casi era milagroso no
perder la vida. No reproduzco su prolija referencia, que ha venido a ser
histórica, porque, la verdad, ni a mí me interesa grandemente la
detallada relación de los movimientos de la tropa leal y de la tropa
rebelde, con tanto general que va y viene de calle en plaza, o de uno a
otro cuartel, ni creo que la remota posteridad que esto lea con ello se
divierta ni se instruya. Porque, si bien se mira, por lo muy repetidos,
son estos movimientos sediciosos como los amanerados poemas de corta
inspiración y de frase pedestre, y sólo en el caso de que el triunfo los
haga eficaces merecen la atención de las gentes. En los pronunciamientos
fallidos veo yo la más tediosa sarta de aleluyas que nos ofrece nuestra
historia. Mirémoslas de prisa, y pasemos a otro asunto.

Lo más triste de aquella jornada fue la muerte de Fulgosio, necio y
bestial asesinato, sin gloria de él ni de sus inicuos matadores. Fue
mártir antes que héroe. Y por mártires hemos de tener también a los
infelices que en la misma tarde del 7 fueron fusilados a la salida de la
Puerta de Alcalá\ldots{} Eran de tropa, pueblo uniformado, según Rivero,
y se habían batido contra el Orden con locura patriótica y militar
ceguera. ¿Qué se dirían Fulgosio y estos desventurados si en el primer
paso dentro de la Eternidad se encontraron y se vieron?\ldots{} No se
dirían nada tal vez, porque del lado allá no habrá palabra con que
expresar la inmensa estolidez de lo que acá llamamos política, orden y
revolución\ldots{}

Hablamos los cinco del suceso y sus consecuencias, y por mi gusto no me
habría entretenido en puntualizar la psicología de aquel movimiento:
todo era vanidad, interés de personas, Salamanca, Buceta, lord Bullwer,
Gándara, y luego una cáfila de nombres de progresistas, llenaban la
histórica aleluya. Los cinco estábamos conformes en que una férrea
dictadura de Narváez se nos venía encima. Pronto seríamos sometidos
todos los españoles a un duro régimen penitenciario. La tormenta que
habíamos visto estallar aquí era no más que un leve desorden
atmosférico, anuncio de mayores desastres; y en aquel motín o
pronunciamiento tan pronto sofocado, no debíamos ver más que una
centella perdida de la furibunda tempestad que corría por toda Europa.
En Francia, gran diluvio que anegaba el trono; en Nápoles, truenos y
rayos; en Roma, centellas y exhalaciones que aterraban al Papa,
moviéndole a cambiar su política de liberal en despótica; en Hungría,
viento huracanado; en Austria, formidable pedrisco que derribaba el
árbol corpulento de Metternich, y en las demás naciones, azoramiento y
terror por el hondo ruido subterráneo que se sentía, como anunciando
terremotos. Es la voz pavorosa del Socialismo, la nueva idea que viene
pujante contra la propiedad, contra el monopolio, contra los privilegios
de la riqueza, más irritantes que los de los blasones. Tiembla la
presente Oligarquía ante estos anuncios, y no sabiendo cómo defenderse,
sólo pide que esta gran vindicación la coja confesada.

Fue mi segunda visita para Eufrasia, a quien encontré celebrando sesión
de la \emph{Sociedad de Socorros de Religiosas}, de que es Presidenta
interina. Actuaba como secretaria Rafaela Milagro, y como
\emph{informantas} o \emph{procuradoras} otras dos damas a quienes no
conozco, y asistía como asesor un capellán de monjas, antiguo jesuita,
que yo había visto antes en la casa de Socobio. Ya estaban terminando
cuando yo llegué, por lo cual pude acceder a no retirarme discretamente.
Contáronme las damas el gran beneficio que hacían a la religión,
socorriendo a las pobres monjitas expoliadas por Mendizábal, y
abandonadas de estos infames gobiernos sin creencias. Rafaela, por lo
que allí oí, es el alma de la Sociedad, a la que se consagra con tanta
actividad como pasión. En el arte de allegar fondos, excitando la
caridad vanidosa, es maestra consumada; al verla, sus amigas tiemblan.
Madrid entero conoce su labor ratonil, las monjas comen y viven\ldots{}
Los elogios que de la Secretaria hizo el clérigo allí presente sonábanme
a panegírico de santa. Y ella, serena y modestísima, insensible a los
encomios, continuaba extendiendo recibos en el pupitre cercano al sillón
presidencial que ocupaba Eufrasia. Por fin, con el desfile oportunísimo
de las \emph{procuradoras} y del cura, que no abandonó el campo sin
hablar pesadamente de una rifa que se proyectaba, quedeme solo con mi
amiga y Rafaela.

«Siéntese usted a mi lado---me dijo la moruna, que por lo visto, o nada
reservado quería decirme, o no le estorbaba la presencia de la
Secretaria.---Esta tarde recibirá usted una invitación de los Emparanes
para comer mañana en su casa. Ya sabe usted que allí no han entrado por
el uso nuevo de comidas a la francesa, y sirven los garbanzos a la una y
media\ldots{} No vuelva usted a dirigirme la palabra si no acude como un
doctrino al llamamiento de esa familia, Pepe. Se le disculpó a usted la
otra vez por las razones que callo; pero si mañana se excusa o hace
rabona, ya sabe que no habrá perdón, sino azotes, y buena mano tiene
Catalina para dárselos. No le digo más sino que ayer tarde di yo a su
señora hermana mi palabra de empujarle a usted hacia la plazuela de
Navalón, y la seguridad de que el simpático \emph{dandy} no se quedará a
mitad del camino. Con que ya lo sabe. Me parece que ya van resultando
ridículos los papeles de galán melindroso y de caballero que adora los
ideales. Déjese de andar por las nubes, y bájese a la realidad. ¿Quiere
más sermón? Pues se continuará esta noche en casa de mi cuñado Serafín.
No falte.» Quise yo responderle; pero la Secretaria reclamó toda la
atención de la Presidenta para el colosal proyecto de rifa, y me retiré
teniendo buen cuidado de no preguntar por D. Saturno. Temía yo que mi
fórmula de urbanidad fuese como evocación que le hiciese surgir por
alguna de aquellas doradas puertas.

\emph{16 de Mayo}.---¡Con qué ganas de solaz honesto, de desconocidas
emociones, entré esta noche en la sala de mi señor Don Serafín de
Socobio! A mí acudieron gozosas Virginia y Valeria, con gorjeo de
pajarillos, y no me abrazaron por respeto a sus papás. Yo sentí en mi
alma una onda de frescura cuando las vi, y deploré que el respeto social
no me permitiera cogerlas y sentarlas en mis rodillas, una a cada lado,
y darles besos inocentes. Empezaron por acribillarme con dicterios
graciosos y con bromas que no carecían de malicia y picor. Dijéronme
luego que cuando se corrió la voz de que en el desafío había yo perdido
una pata, ambas habían llorado por el hombre y por la pata perdida,
sintiendo que no pudieran ellas pegármela con cola, como la pata de una
mesa. Se acordaban de mí, y sabían las cosas terribles que me pasaron
por mi mala cabeza, sin que el castigo me enmendase; enteradas estaban
también de que ya no tardaré en caer en la ratonera que me han
armado\ldots{} Contra esto hube de protestar, asegurándoles que yo no me
caso con ningún bicho viviente más que con ellas, con ellas dos,
Virginia y Valeria, mis dos novias hoy, mis dos mujeres mañana. Vi sus
rostros pasando de la risa a la seriedad, y por igual impregnándose de
no sé qué melancolía cavilosa. Callaban, y aun querían huir de mi
presencia por no saber qué decirme, pues aquella broma del casorio con
las dos, a entrambas lastimaba, como si fuera la única idea que cortase
de raíz la membrana moral y física que las unía. Sentían quizás el
desconsuelo de ser dos y no una sola\ldots{} También yo me llenaba de
gran confusión, no pudiendo destruir la dualidad sin matar a uno de
aquellos ángeles. ¡Imposible el dualismo, imposible la unidad!

Ya muy tarde pude quedarme solo con Eufrasia en un rincón del gabinete
donde Rafaela Milagro explicaba su magno plan de benéficas rifas a dos
señoras ancianas y al vetusto coronel Sureda, convenido de Vergara,
hombre muy dado a la protección de monjas. ¿De modo que usted---dije a
mi amiga en cuanto entramos en materia,---persiste en que yo no tenga
dignidad y me venda a los Emparanes?

---Esto no es venderse, Pepe---respondió mirándome cariñosa.---No tome
usted actitudes de teatro ni se nos ponga \emph{fatídico}\ldots{}

---Es una venta, señora mía. Yo doy una figura regular, un carácter
ameno, instrucción, hábito social, buenas relaciones, y encima de todo
ello mi libertad y mi felicidad. Ellos lo toman, quiero decir, lo
compran, dándome dos clases de valores: su riqueza, que es efectiva, y
su hija, que es una falsificación de mujer, un valor de engañifa, un
papel mojado, como si dijéramos. ¿Para qué quiero yo a María Ignacia? De
todas las personas que conozco podría yo esperar que me aconsejaran esa
boda, menos de usted\ldots{} y ésta es mi mayor pena, Eufrasia, porque
ya no tengo duda: usted me detesta. Si en algo me estimara, no sería
corredora de esa venta infame.

---Yo creí que era lo contrario---me dijo bajando los ojos.---Por su
mejor amiga, por su amiga franca y leal me tenía y me tengo yo al
agenciarle esa colocación\ldots{} No se ofenda usted de la palabra,
Pepe\ldots{} \emph{Colocación}: no hay otra manera de decirlo; y yo, que
no reparo en soltarle a usted las verdades más amargas, le digo que está
perdido si no se coloca, y que no encontrará, créame a mí, mejor plaza
que ésa, porque no la hay, ni lugar más ancho y cómodo para el descanso
de toda su vida\ldots{} Dé gracias a Dios y a su hermana, que es para
usted como un ángel bajado del Cielo.

---Mi hermana es, sí, el ángel del comercio matrimonial, y usted otro
ángel que ha venido a volverme loco\ldots{} porque si en efecto me
estima, no puede usted aconsejarme la entrega vil de mi persona\ldots{}
porque, si yo sigo su consejo, usted debe despreciarme\ldots{} ¿Y cómo
compagino un sentimiento con otro, el desprecio con la estimación?

---No hay tal desprecio.

---Digo y repito que usted me ha hecho perder la cabeza. Diré con D.
Matías: \emph{Ho perso il boccino}\ldots{} Contésteme: si yo rechazo lo
que me propone, ¿qué seré para usted?

---Será usted un ingrato---replicó fijando en mí sus ojos con dulce
tristeza,---porque no sabrá corresponder al grandísimo interés que por
usted me tomo. Yo le aconsejo la boda porque sé que le conviene, que no
hay otra salvación para usted, que no hay mejor remedio para salir del
laberinto de sus deudas y reconstruir su vida sobre una base
firme\ldots{}

---¿Y llama base firme a un matrimonio en el cual no puede haber amor,
por mi parte?

---No sigamos, Pepe---dijo la dama, viendo que en nuestra discusión,
algo semejante al revolver de una madeja, se había formado un nudo
difícil de deshacer.---Si nos ponemos en lo \emph{fatídico}, no hemos
hecho nada\ldots{} Me da usted, créalo, una pena muy grande rechazando
mi consejo\ldots{} consejo de amiga\ldots{}

---Pero ¿qué amiga es usted, Eufrasia?

---La mejor---afirmó sin disimular su emoción,---la mejor, la única que
ha tenido usted en su vida. Si así no lo aprecia, déjeme, no vuelva a
verme más, y siga, siga en esa vida absurda, que le llevará al
precipicio\ldots{} Yo quiero salvarle, y usted no se deja. Bueno: ya me
dará la razón algún día\ldots{} Ya me dirá: «¡Qué razón tuviste,
mujer\ldots{} a quien no comprendí\ldots!»

Y recelando ser oída, varió de tono, puso freno a su emoción. La vi
pestañear, fruncir la boca; más pronto compuso admirablemente sus
facciones, y sonriendo me dijo: «No hablemos más esta noche, Pepe.
Dejémoslo para otro día\ldots{}

---¿Para cuándo?

---Vuelvo a repetirlo: ¡ingrato, ingrato!\ldots{} No digo más por
hoy\ldots{} Mañana\ldots»

Hizo una larga pausa meditando. El \emph{mañana} y la pausa fueron como
un balancín en que se meció mi espíritu dulcemente.

«Pues mañana\ldots{}

---Acabe usted, por la Virgen Santísima---dije, mareándome un poco en el
balancín.

---Déjeme usted: estoy haciendo cálculos de tiempo\ldots{} Pues sí, a
última hora de la tarde podremos vernos. ¿Dónde? Sorpresita
tenemos\ldots{} Pues al marido de la \emph{Teresona}, criada antigua de
esta casa, le hemos dado la plaza de conserje del Casino. ¿Sabe lo que
es el Casino? No vaya a confundirlo con esa maldita sociedad donde se
pasa usted las noches jugando, y hablando mal de todo el mundo. Hablo
del \emph{Casino de la Reina}, un Sitio Real chiquito, al fin de la
calle de Embajadores, con jardín muy hermoso y un poco de templete y un
poco de palacio; recreo que fue de la Reina Gobernadora\ldots{} Pues el
otro día estuve a ver a \emph{la Teresona}, y pasé un rato muy
agradable. Adoro los jardines, y las flores me enloquecen\ldots{}

---¿Y mañana\ldots?

---Mañana volveré allá, sí, señor\ldots{}

---¿Irá usted sola?

---No puedo asegurar que vaya sola\ldots{} Quizás tenga que llevar a
Rafaela Milagro.

---Bueno: ¿y yo\ldots? Descuide usted, que antes faltará el sol en el
cielo que yo en ese Casino, venturoso rincón del paraíso terrenal.

---No vaya usted a creer que es un Versalles, ni un Pincio, ni un
Aranjuez.

---Será más bello que todo eso; sólo con servir de fondo a la
\emph{belle jardinière}\ldots{}

---¡Ay, ay, ay!\ldots{} ¡qué florido!\ldots»

\hypertarget{xxvi}{%
\chapter{XXVI}\label{xxvi}}

\emph{17 de Mayo}.---No falté, no, a la comida en casa de los Emparanes,
y debo decir que fue muy de mi gusto, y en todo, cosas y personas, hallé
gratísimas impresiones, menos en la señorita de la casa, quien, por
refinada crueldad de mi destino, hubo de acrecentar en mí la antipatía
que me inspiraba. Sentáronse a la mesa conmigo, como invitados, el
coronel Sureda y el Sr.~de Roa, secretario que había sido del Infante D.
Sebastián en la Corte de Oñate, y la siempre vistosa y guapísima Doña
Genara Baraona, viuda de Navarro, de cabello blanco como la nieve,
rostro fresco y sonriente boca. Los años no pasan por ella, o le
tributan los más ricos honores viéndose obligados a envejecerla. Es un
monumento esta dama, cuya belleza va unida a medio siglo de nuestra
historia, con adherencia y comunidad de sucesos interesantes, así
públicos como privados. Desde la batalla de Vitoria, el año 13, hasta la
Regencia de Espartero, el 40, la católica Genara y la profana Clío han
corrido juntas algunas parrandas, y ello se les conoce en la amistad que
las une. Así, no hay historia más instructiva y amena que la que cuenta
esta ilustre viuda cuando alguien incita su natural vanagloria de
crónica viviente\ldots{}

De las señoras mayores que dan lustre y dignidad a la casa, sólo dos
estaban en la mesa, además de Doña Visitación, en todo el esplendor de
su atavío morado, de amplitudes y magnificencia episcopales. Las otras
vestían de negro, con cofias elegantes del año 30, y de pies a cabeza
eran la corrección y la pulcritud más exquisitas. Gravemente amable,
como perfecto caballero de antigua cepa, estuvo conmigo el señor Don
Feliciano, y su esposa le imitaba en cuanto podía, sin llegar al punto y
filo de la perfecta urbanidad castellana. Las señoras mayúsculas
cotorreaban, Genara quería distinguir su elegancia flexible y
modernizada, y los dos personajes carlistas, muy finos, aunque algo seco
el uno, demasiado charlatán el otro, completaron el lucido y decoroso
cuadro. Todo, como dije, contribuyó a mi solaz y contento, menos la
desgraciada niña, que a mi lado tuve, y que en el largo curso de la
comida no supo responder con el menor chispazo de gracia o de ingenio a
las excitaciones que por vía de tienta le hacía yo. Huraña y
melancólica, ni una vez la vi reír, ni salieron de su siempre repulgada
boca más que frases vulgarísimas, o desabridas observaciones. Nunca vi
cortedad semejante, ni mayor indigencia de ideas, ni criatura menos
mujer. Por momentos parecíame un chico gordinflón y mal educado a quien
no habían podido enseñar más arte que el del silencio.

La conversación, que al principio fue bastante amena, porque Genara y
los carlistas se enzarzaron en una controversia recreativa sobre el
casamiento de D. Carlos con la de Beira, recayó luego en temas
fastidiosos. Como estamos en plena romería de San Isidro, las señoras
maduras sacaron a relucir la historia del santo, y después hubo grande
palique sobre el hecho de que se conservase incorrupto el cuerpo del
patrón de Madrid. Aseguró D. Feliciano que lo había visto, y podía dar
fe de su perfecta conservación en estado de mojama, sin que ninguna
parte le faltara, y nos ponderó su gigantesca estatura, como nueva
demostración de la divinidad del bienaventurado labriego. Los carlistas,
que me parecían algo escépticos en materias de milagrería y momias de
santos, contaron anécdotas vascas muy graciosas, que no hay para qué
reproducir aquí, pues de asunto más pertinente a mi persona debo
ocuparme.

Ello fue que en el salón, después de la comida, cuyo suculento aliño a
la española tengo que elogiar aunque sea de pasada, probé a sacar del
pedernal duro de María Ignacia algunas chispas, hiriéndola por uno y
otro lado de su entendimiento con el eslabón de estudiadas preguntas y
proposiciones. Mas no me dio resultado la prueba, y fuera de alguno que
otro rasgo de ingenuidad casi infantil, no daba lumbres la infeliz
criatura con quien querían emparejarme para toda la vida. A posta me
dejaron los padres con ella en un extremo de la estancia, para que la
señorita, sin tener sobre sí la vista y atención de las personas
mayores, pudiera despabilarse; doña Genara me miraba compasiva; los
carlistas, hablando pestes del Gobierno, no nos hacían caso; y María
Ignacia continuaba en el bloque ingente de su estolidez, como un grosero
pedrusco diamantino en el cual no entraba la lima, ni aun el filo de
otro ya bien tallado diamante. No hacía más que clavar en mi rostro, o
en las guirindolas de mi pechera, sus ojos fríos, vidriosos, con una
expresión de arrobamiento que me confundía, y estar pendiente de mis
palabras, como si yo fuese oráculo que debía ser oído religiosamente,
mas no contestado.

Incitarla quise a la risa, y sus esfuerzos por no descubrir el feo
panorama de las encías daban a su boca cierta semejanza con el hociquito
de no sé qué animal. Díjele, por no dejar de ser galante, que estaba muy
bien vestida (y era la verdad, aunque con la perfección del traje no
lograba hermosear su cuerpo), y me respondió que soy un
embustero\ldots{} Vamos, esto me hizo alguna gracia. Luego tuvo más de
un rasgo de suprema modestia, expresada con primitiva sencillez; pero al
instante destruyó el buen efecto con unos solecismos imposibles, y me
preguntó con mimo quejumbroso si iba yo a misa todos los días. «Ya lo
creo---le respondí.---Mi misa de ocho todas las mañanas no hay quien me
la quite.» Decidiose a reír, y volvió a llamarme embustero, y después
malo\ldots{} De diferentes modos me dijo que yo soy muy malo, añadiendo
que si encuentro quien interceda por mí, Dios me perdonará\ldots{} No
hubo manera de sacarla de esto\ldots{} Yo me aburría, lo
confieso\ldots{} Vi con júbilo llegar el momento del desfile, y salí
renegando de mi hermana Catalina, sobre cuya cabeza vería con gusto caer
un rayo del Cielo.

\emph{30 de Mayo}.---El largo paréntesis entre la última y la presente
confesión no sea mirado como efecto de la holganza, sino de las
inquietudes, amarguras y sobresaltos que en el intermedio de las dos
fechas han agitado mi alma y absorbido mi tiempo, no dejándome espacio
para el recreo de estas Memorias. Con la atención prisionera y esclava
de los acontecimientos, ni aun el descanso del cuerpo me ha sido
posible, y no pocas noches pasé de claro en claro, abrasado el cerebro
por las cavilaciones\ldots{} Desembarazada ya mi atención de aquellas
cadenas, quiero ganar el tiempo perdido, y llenar toda esa laguna con
una confesión extensa y sustanciosa.

Pues, señor, el 17 de Mayo (no olvidaré nunca la fecha) se me hacían
siglos las horas, esperando la de la cita que me había dado Eufrasia en
el apartado Casino de la Reina, y en mi loca impaciencia, incapaz de
adelantar el tiempo, me adelanté yo, llamando a la puerta de aquella
posesión a las cinco y media de la tarde. Entré: vi con sorpresa que la
dama me había cogido la delantera, pues allí estaba ya. La vi entre la
arboleda corriendo gozosa, y fui en su seguimiento: se me perdía en el
ameno laberinto, pasando de la verde claridad a la verde sombra, y no
encontraba yo la callejuela que me había de llevar a su lado. Llamé, y
sus risas me respondieron detrás de los altos grupos de lilas. Se
escondía, queda marearme. Corrí por el curvo caminillo que tenía
delante, y luego sonaron las risas detrás de mí. Una voz que no era la
de Eufrasia dijo: «Por aquí, D. José.» Creí escuchar a Rafaela Milagro,
y ello me dio mala espina, porque era un testigo sumamente importuno.
Después reconocí el acento de la doncella de mi amiga. Ésta fue, por
fin, la ingeniosa Ariadna, que con el hilo de sus voces me fue guiando
hasta que pude verme en su presencia y rendirle mis cariñosos homenajes.
¡Qué hermosa estaba, encendido el rostro por la agitación de sus
carreritas y el contento de la libertad! En su peinado advertí alguna
incorrección, sin duda producida por las mismas causas. Vestía con
sencillez deliciosa. Nunca la vi más interesante.

Del ramo de flores recién cogidas entresacó la morisca el más bonito
capullo de rosa para ponérmelo en el ojal, y luego me dijo: «¿Verdad que
es bonito este vergel? Aquí me pasaría yo todo el día si pudiera.»
Satisfecha de mi admiración, que por igual a ella y a la Naturaleza
tributaba yo, quiso enseñarme toda la finca, el \emph{Sitio Real de
juguete}. A cada instante se detenía para señalarme los grupos de rosas
que con insolente fragancia y risotadas de colores nos daban el quién
vive. Por otro lado, me mostraba los cuajarones de lilas inclinando con
su peso las ramas de que pendían, como millares de hijos colgados de los
pechos de sus madres; luego vi el árbol del amor, con su infinita carga
de flores entre las hojuelas incipientes, símbolo de la precocidad
juvenil y de la desnuda belleza pagana; vi el árbol del Paraíso, de
lánguidas ramas que huelen a incienso hebraico, y la acacia de mil
flores olorosas\ldots{} En los cuadros rastreros, los lirios de morada
túnica eran los heraldos de las no lejanas fiestas del Señor, Ascensión,
Corpus, y las blancas azucenas anunciaban la proximidad del simpático
San Antonio.

Mil tonterías dijimos en alabanza de tan bello espectáculo. No sé si el
encanto de éste era cualidad intrínseca del risueño jardín, o estado mío
de alborozo. Ambas cosas serían. Después de divagar solos por aquella
ondulada amenidad, llevome la dama a un templete, erigido entre verdosos
estanquillos. Era de piedra y mármoles, semejante a los que hay en
Aranjuez, pero de juguete, abierto por tres costados de su cuadrangular
arquitectura, y decorado con bichas y quimeras al fresco, un poco
deslucidas por la humedad, todo en el estilo neoimperial de Fernando
VII. Allí nos sentamos. Eufrasia dejó la carga de flores que traía,
señalando un grupo muy grande para sí, un ramo para mí, y apartando
después otro montón de lilas y rosas, acerca del cual me dijo: «Ya sabrá
usted luego para quién es esto.» Entablé sin esfuerzo ni premeditación
un coloquio dulce y cariñoso, que fácilmente afluía de mí sin más
estímulo que la fragancia del ambiente y el aspecto de tanta flor sobre
la verde arboleda. Hablé a la moruna del religioso fervor con que yo
practico el culto de su amistad, haciendo de ésta la clave de mi vida;
entoné otras estrofas, y en variados metros de amor canté mis quejas por
el desdén que me mostraba, y le rendí toda mi voluntad. Cuando
callábamos, oíamos el zumbar de insectos y el vuelo de moscas o moscones
que en el templete requerían la sombra. Por fin, en premio de mis
líricos arrebatos, permitiome Eufrasia besar su mano; y ya tenía yo en
la boca y en el pensamiento intención y palabras para empezar a
desmandarme, cuando sentimos pasos que por lo fuertes parecían de
hombre. Levantose mi amiga, dejándome todo lo suspenso que puede estar
un enamorado, y saliendo a uno de los huecos del templete, dijo:
«Teresona, aquí estamos.»

Salí yo también, a punto que una voz hombruna decía: «Yo pensé que
estaba la señora cogiendo flores.» En la gigantesca mujer que se
acercaba, reconocí, más por los andares y por la facha de osa polar que
por la voz, a la estantigua que en el baile de Villahermosa se apareció
tocando a retirada. Ya me había dicho Eufrasia que la mascarita
compañera era su doncella Rufina. El acento vizcaíno de Teresona la hizo
revivir en mi mente con el dominó negro guarnecido de picos verdes. Por
segunda vez venía el odioso espantajo a cortar bruscamente mi recreo,
mejor será decir mi felicidad. Y lo peor fue que no pareció Eufrasia
disgustada de verla, y que, antes bien, acogía su presencia como se
acoge a quien nos preserva de un peligro. En calidad de cancerbero
teníala allí la dama, y sin duda le había encargado que ladrase con sus
tres bocas en cuanto notara el menor riesgo de fragilidad. «Venga, venga
la señora---dijo Teresona,---y verá la pollada que me sacó ayer la
moñuda.» Y Eufrasia (confúndanla Venus y Cupido), para contrariarme más
y darme el quiebro, alegrose o fingió alegrarse del recreo que la criada
le propuso, porque al punto echó tras ella, llevándome a un corral
próximo a la casa del guarda o conserje. Malditas ganas sentía yo de ver
pollitos; pero no tuve más remedio que acompañar a la dama y hacerle el
dúo en la admiración de la gallina conduciendo y educando a sus
graciosos hijuelos.

Volvimos luego a pasear, mas por sitios elegidos sin duda con astuta
precaución, para que encontrásemos a cada paso, bien a la vizcainota,
bien a su marido o a la doncella, que charloteaba con un jardinero
jovencito.---Bien, señor\ldots{} Adelante\ldots{} «Sé apreciar, amigo
mío, la lealtad de su afecto---me dijo Eufrasia respondiendo a las
protestas apasionadas que de nuevo le hice,---y no le faltarán a usted
ocasiones de conocer lo que vale su amiga.

---Esas ocasiones vengan pronto; pero no se me ordene lo que no puedo
cumplir.

---¿Cómo que no? Hará usted todo lo que yo le mande, todo absolutamente,
sin vacilar.

---Y por esa obediencia mía tan penosa, ¿tendré la recompensa que más
anhelo?

---Déjese de recompensas y de bobadas. Está usted loco con la idea de
que le quieren vender o comprar, y ahora quiere comprarme a mí. Yo no me
vendo ni por su obediencia, que es valor muy grande, ni por nada\ldots{}
Al aconsejarle yo que tome a Ignacia, lo hago porque sé cuánto le
conviene ese cáliz, Pepito. Es un elixir bien probado el matrimonio: con
él tendrá usted la posición que merece, y la libertad que no puede
esperar de esa vida falsa entre tantas esclavitudes, deudas,
compromisos, el quiero y no puedo, que es el más grande suplicio de los
tiempos que corren. Dese usted por convencido, y no hablemos más del
asunto.

---Ni amo ni puedo amar a María Ignacia.»

Eufrasia no me contestó, y mascando un palito de rosa, miraba al suelo.
«Vamos, no sea usted tonto, ni haga uso de un argumento en que no
cree\ldots{} No, no cree usted que eso del amor sea una razón\ldots{}
Fíjese usted en su situación social, y haga caso de lo que le
aconsejamos las que conocemos el mundo, la vida: su hermana Catalina,
que tiene la inspiración del Cielo; yo, que tengo la inspiración de mi
experiencia\ldots{} quiero decir, que mis desdichas me han enseñado la
inmensa mentira de amor.

---Cierto---dije yo,---que debo tener muy en cuenta su opinión\ldots{}

---Y la de otras. Consulte usted el caso con otras amigas\ldots{} ¿Por
ventura la de Torrefirme no le aconseja lo mismo?

---No, señora: me aconsejó lo contrario\ldots{} Hoy no puede aconsejarme
nada, porque hemos roto\ldots{}

---Ya lo supe\ldots{} Esa mujer no le amaba a usted, Pepe. Por no amarle
ni pizca, le aconsejó tan disparatadamente. Quería su perdición, su
ruina, su muerte en la sociedad y en la familia, que es lo que yo no
quiero; no, Pepe, no lo quiero\ldots{} Y como no deseo nada malo para
usted, le aconsejo y le mando que se case\ldots{} Su obediencia es una
virtud que será pagada con mi amistad.

---¿Y cómo será esa amistad?

---Muy cariñosa: una amistad\ldots{} tutelar---declaró después de
pensarlo un ratito.

---¿Y qué más?

---Una amistad entrañable\ldots{}

---¿Y qué más?

---Eterna---dijo volviéndome la espalda, para que no la viese llevarse
la mano a los ojos.

---¿Eterna dice\ldots?

---Sí, sí\ldots{} Ponga usted todos los adjetivos que quiera, Pepe;
siempre serán pocos\ldots{} Y no hablemos más de eso, Pepe, por Dios, no
hablemos más.»

\hypertarget{xxvii}{%
\chapter{XXVII}\label{xxvii}}

En efecto, no hablamos más del asunto; pero con sus ojos más negros que
el alma de los condenados, con la lividez que los circundaba, y con el
timbre opaco de su voz, picando en cosas comunes, me cantaba el poema
más halagüeño para mi vanidad. Bien segura en su conciencia exterior por
el amparo que le daba la guardia de sus cancerberos, y cuidando de que
no la perdiesen de vista, no temía ya manifestarme su apasionada ternura
por medios y signos que yo solo había de entender. Era mía; pero no sé
qué voces del corazón me susurraban que mi victoria quedaría por algún
tiempo circunscrita al terreno de los principios, como la entrega de una
plaza psicológica.

«Volvamos al templete---me dijo con cierto donaire, en que vi algo de
travesura.---Se me ha olvidado una cosa.» Y adelantándose, antes de que
yo llegara la vi salir con el gran manojo de flores que apartado había
sin decirme para quién era. Mandó a la Teresona que armase un lucido
ramo. Paseamos de nuevo, y a mis preguntas contestó así, maravillándose
de mi torpeza. «¡Ingrato! ¡No adivinar para quién son estas
flores!\ldots»

Un rayo iluminó mi mente. «Ya\ldots{} de veras he sido torpe\ldots{} El
ramo es para la pobre Antoñita\ldots{}

---Que está bien cerca.

---Hermosa idea, y más hermosa si vamos los dos a llevárselo.

---Pepe---me dijo poniendo otra vez en la mirada toda su
ternura,---permítame que le eche en cara su torpeza, su\ldots{} ¿cómo
decírselo? No ha sido usted muy delicado. La persona en quien menos debe
usted pensar para que le acompañe al llevar esas flores soy yo.

---Pero de usted ha sido la idea de adornar con ellas el nicho.

---Mía fue la idea, creyendo que era idea suya\ldots{} ¿me entiende?

---Sí\ldots{} En todo tiene usted razón. Debo ir solo. Pero no ahora. Es
un poquito lejos, y no me esperará usted hasta que vuelva.

---Le prometo que sí le esperaré, si no se entretiene mucho. Es cerca.
Coge usted el paseo de las Acacias\ldots{}

---Lo cojo\ldots{} sí\ldots{} pero si con cogerlo bastara\ldots{}
Después de cogido tengo que andarlo todo.

---¿Y qué? Luego pasará el puente de San Isidro\ldots{}

---Si tuviera usted aquí su coche\ldots{}

---Vendrá a buscarme luego\ldots{} Pero en mi coche no debe usted ir,
criatura.

---Es verdad\ldots{} Bueno, amiga del alma. Voy, y cuando vuelva
encontraré aquí a su esposo que viene a buscarla.

---No vendrá, tontín; yo le aseguro que no vendrá.»

Díjome que su marido y ella andaban algo torcidos, por cuestiones
caseras de poca monta\ldots{} No era nada: genialidades de uno y otro. Y
como yo le manifestase grande anhelo de conocer la causa de aquellos
moños, me dijo: «Si usted es tan bueno y tan agradecido y tan caballero
que le lleva las flores a la pobre Antoñita, y las pone con muchísimo
respeto y cariño sobre su sepulcro, tenga por seguro que aquí le espero
y que le contaré\ldots{} vamos, eso, la gacetilla doméstica que desea
conocer\ldots{} Para usted no debo tener secretos.»

Francamente, esto de no tener secretos para mí me entusiasmó, la verdad,
me colmó de orgullo. Instándola a que reiterase su promesa, y cambiadas
las generales fórmulas de contrato, salí con mi hermoso ramillete,
deseando que en pujantes alas se me convirtiera. Tuve la suerte de
encontrar coche de alquiler apenas andado un tercio del paseo de las
Acacias, y a los quince minutos ya daba yo fondo en el cementerio.
Interneme de patio en patio; algunas personas enlutadas andaban tristes
y lentas por allí, cumplidas o por cumplir obligaciones semejantes a las
que yo llevaba; otras se entretenían en leer doloridos o rimbombantes
epitafios, y en mirar las coronas ya mustias del último Noviembre.

Llegué a donde iba: un guarda, cuyo auxilio reclamé y tuve mediante
propina, me trajo dos búcaros que para el adorno de los nichos allí se
facilitan; dividimos el ramo en dos, y puestos en su lugar, no tan alto
que necesitáramos escalera, quedó muy bonito, descollando por su
lucimiento en la descarnada tristeza del camposanto. La imagen de la
muerta, que ya navegaba con veloz carrera por el piélago de un inmenso
olvido, y casi traspasaba sus horizontes, revivió en mi mente: la vi
como si con los carnales ojos la viese. ¡La pobrecita gustaba tanto de
las flores! El cierre del nicho, sin letrero aún, no tenía más que un
número, tres guarismos que no decían nada; para mí eran un triste nombre
y un sentimiento no apagado todavía, pero ya muy débil y casi expirante,
como las luces que absorben con ansia de vivir su último aceite.
¡Infeliz Antonia! ¡Tan joven, y ya reducida a un signo de cantidad
pintorreado sobre un tabiquillo de yeso!\ldots{} Mirando la cifra, pensé
en la discordia conyugal de Eufrasia, y en volver pronto al Casino para
que mi amiga me la refiriese\ldots{} Pensé también que Antonia, si su
espíritu no estaba lejos de aquel depósito de su descompuesta humanidad,
se alegraría de ver las flores y el \emph{gitano} que se las ponía.

El guarda o sepulturero miraba mi obra con un guiño de ojos enteramente
escéptico y casi casi burlón. «Cuidará usted de que los ramos no se
caigan---le dije.---¿Cree usted que durarán mucho?» Y él, guiñando el
ojo no para el nicho, sino para mí: «Como durar, no sé\ldots{} Piense
que son flores\ldots{} Pero yo estaré al cuidado para que no las roben;
que aquí\ldots{} ya sabe\ldots{} anochecen los ramitos en un nicho y
amanecen en otro\ldots{} Vienen algunos llorando, y el que no trae
flores las toma de donde las hay\ldots{} Pero yo estaré con mucho
ojo\ldots{} Si alguien las quitara, yo las volveré a poner en su sitio.
A cada uno lo suyo\ldots{} Váyase tranquilo.» Me retiré, y al atravesar
el patio, volvime más de una vez a mirar si alguna enlutada de las que
por allí discurrían me quitaba las flores, mejor dicho, se las quitaba a
mi nicho, o sea el nicho de Antonia, para ponerlas en cualquier
enterramiento de muertos extraños\ldots{} Pero cuando pasé al otro
patio, mis reflexiones encamináronse por vía más generosa y alta, y
pensé así: «Dejemos el egoísmo a las puertas de esta morada de la
igualdad\ldots{} y las flores, como toda ofrenda\ldots{} sean para
todos.»

En quince minutos, arreando de firme, me llevó el coche al Casino; aún
era día claro cuando me vi de nuevo en presencia de Eufrasia, y dándole
cuenta de mi comisión, oí de su boca plácemes sinceros por mi
obediencia. Y yo: «Por mi parte cumplido está nuestro contrato; cumpla
usted ahora; refiérame\ldots» Y ella, riendo: «¿Pero de veras le
prometí\ldots?» «Prometió usted, con una fórmula agravante\ldots»
«¿Cuál, pobre niño?\ldots» «La declaración de que no debe tener secretos
conmigo\ldots» «¿Eso dije? ¿Está usted seguro?\ldots» «¡Eufrasia!»

---Bueno, Sr.~D. Pepe, mi amigo, mi protegido y mi criatura inocente: le
contaré la gacetilla\ldots{} Vámonos por aquí y demos la vuelta chica
del jardín, por las lilas\ldots{} Ha de saber usted que mi marido, desde
que mataron bárbaramente las turbas al pobre Fulgosio, está con la bilis
tan revuelta y con el genio tan amargado que no se le puede
sufrir\ldots{} Naturalmente, Fulgosio era un amigo muy querido: juntos
sirvieron en la facción, el pobre D. José como general, Saturno como
intendente\ldots{} Pues está el hombre poseído de un furor tan grande
contra las \emph{masas}, y contra el Progresismo y contra Bullwer, que a
ratos parece que pierde la razón\ldots{} Su odio más vivo es contra el
Socialismo, secta que dice ha salido del Infierno, o es el Infierno
mismo traído a la faz del mundo, y no hay, según él, penas ni castigos
bastante fuertes para los que propagan tal doctrina. Yo, por no
encalabrinarle más, le digo a todo que sí: por este lado no viene la
discordia. Pero hay otra cuestión, no política, sino particular,
planteada entre nosotros antes del 7 de Mayo, en la cual no estamos
conformes\ldots{} Por mucho que usted cavile, Pepito, no encontrará la
solución del acertijo. Óigala: Lorenzo Arrazola, Ministro de Gracia y
Justicia, que es amigo de Saturno y le debe favores, le habló, allá por
abril, de la concesión de un título de Castilla. A nuestro amigo el
Sr.~Clonard le pareció de perlas esta idea, porque, lo que dice, la
mejor recompensa para las personas que de otro campo han venido a
reconocer a Isabel II es darles acceso hasta lo que llaman \emph{gradas
del Trono}, por medio de la investidura de nobleza y grandeza de España,
y qué sé yo qué\ldots{} El Rey D. Francisco, a quien hablaron de ello
algunos de su tertulia, se mostró muy complacido, y dijo que se contara
con él\ldots{} A mi marido se le encendieron de tal modo las pajarillas
de la vanidad, que andaba demente con el Marquesado, descrismándose para
elegir el nombre de finca o lugar que había de ser el apodo
heráldico\ldots{}

«En fin, todos perdidos de la cabeza, menos yo, que me conservo serena,
y no quiero motes ni honores ni nada de eso\ldots{} al menos por
ahora\ldots{} Veo que usted se asombra, Pepe: sin duda no me conoce
bien. No soy vanidosa; me gustan las comodidades, la riqueza, que nos
hacen alegre y fácil la vida; me gusta poseer los bienes positivos,
vengan como vinieren; pero las apariencias chillonas no son de mi
devoción\ldots{} Además, yo no quiero lanzarme al mundo con un título
rimbombante. Es muy pronto para mí. Parecería una provocación, un
trágala\ldots{} Están muy frescas en la memoria de la gente ciertas
cosas que a mí me pasaron, y\ldots{} no quiero, no quiero que la malicia
me haga la autopsia, y empiece a sacar cosillas y a comparar, y a decir
esto y esto y esto\ldots{} Ya sé que otras se curarían poco de la
murmuración, y corriendo ellas el velo, creerían que todo estaba bien
tapadito. Yo no pienso así; sé que la sociedad es bastante desmemoriada;
pero yo no lo soy\ldots{} En principio, lo que se llama en principio,
Pepe, no rechazo el Marquesado, y para más adelante, no digo que no lo
admita, y me encasquete la corona y dé a muchas dentera, y a otras les
refriegue los hocicos con mi escudo; pero ahora no\ldots{} es muy
pronto. Esperaremos cuatro o cinco años. ¿No cree usted que soy
razonable?»

Díjele que la tengo por la misma razón, y que cada día encuentro en ella
nuevos motivos para admirarla y adorarla, amén de tenerla por eminente
maestra del vivir. Y ella siguió: «Pues aquí verá usted el porqué de las
desavenencias entre Saturno y yo de algún tiempo acá, y del horrible
altercado que tuvimos ayer. Díjome cosas que en verdad me
lastimaron\ldots{} me rasguñó en lo más delicado de mi alma\ldots{}
Luego, por la noche, vino a mí tan manso y tan tiernecito, que me dio
asco\ldots{} Para usted no tengo secretos\ldots{} Hoy no sabía qué
hacerme, ni en qué altarito colocarme. Yo me mantuve en mis
trece\ldots{} Es un hombre de una vulgaridad que no se cuenta en un
año\ldots{} Esta tarde le dije que iba al Sacramento, y del Sacramento a
las Comendadoras de Santiago, donde hay dos \emph{señoras de piso},
amigas mías, y de allí a las Góngoras; y en vez de andar esas
estaciones, me he venido aquí a rezar con mis rosas y mis lilas. Por
allá andará buscándome con el señor de Clonard\ldots{} Luego le diré que
he venido a La Latina y a Santa Isabel\ldots{} Tengo la buena costumbre
de variar el itinerario de mis devociones\ldots{} Así se me hace mi cruz
más ligerita\ldots»

Encantado la oí, y mi vanidad ante aquel espiritual divorcio se infló
hasta no caber dentro de mí. Entre las diversas expresiones enfáticas
que le dije, lo más presente en mi memoria es que me tengo por el más
feliz de los mortales, admirando el pórtico de la felicidad. «Es usted
un niño---me contestó ella con adorable acento al despedirme.---Por más
que presuma de hombre hecho, no es más que una criatura, criatura muy
esbelta y llena de atractivos, pero que todavía necesita crecer un
poquito y reforzarse del entendimiento, y endurecer las ideas\ldots»
Indicome, al fin, que partiese antes de que llegara su coche, que ya
estaba al caer, y me empujó hacia la puerta con un desenfado
gracioso\ldots{} «De aquello no se habla ya---me dijo.---Obedecerá el
niño a todo lo que se le mande. Adiós\ldots{} En casa de Serafín nos
veremos\ldots{} Adiós, adiós.»

Salí, mas no me alejé de la calle de Embajadores hasta que la vi pasar.
Ya sabía la muy pícara que yo rondaba. ¿Cómo no presumirlo? Y al pasar,
ya oscurecido, vi su rostro en la portezuela, y una mano que hacía el
gesto de azotar\ldots{} Y sus ojos negros también los vi, o me los
figuré rivales de la noche y de toda la oscuridad del mundo.

\hypertarget{xxviii}{%
\chapter{XXVIII}\label{xxviii}}

Sigo con mi historia de estos días, y de los hechos gratos paso a los
menos placenteros, de las flores a los abrojos, y de lo perfumado a lo
pestilente. ¿Qué cosa existe más fea y desagradable para nuestros
sentidos, tacto, vista, olfato, que el vernos privados de los precisos
dineros para las atenciones de la vida, ora sean éstas de las
elementales, ora de las artificiosas y superfluas que crea y fomenta
nuestra estúpida vanidad? Y no era lo peor que yo careciese de aquella
materia vivificante, sino que me apretasen los usureros para el pago de
lo que les debía, estrechándome con tal rigor de cerco militar, que se
creería que el cielo los desataba en mi persecución, como aquellos
vándalos que del Norte vinieron sobre estos infelices pueblos
mediterráneos. Mi insolvencia, más marcada cada día, les irritaba,
trocándoles en fieras. Contra su persecución no me valían ya ni
escondites, ni esquinazos, ni artes escurridizas de ningún género. Y
para colmo de infortunio, mi amigo Aransis, de quien yo ampararme solía
en estas guerras contra la cobranza, se hallaba en situación más
angustiosa, requerido y deshonrado ante tribunales, sin apoyo material
de su noble familia. ¿De la mía qué podía yo esperar? Nada más que
anatemas y malas caras. Mis hermanos no podían o no querían hacer nada
por mí. Sofía llegó a proponerme que huyera, que me embarcara, y no
volviese a parecer más por la Corte. Ya no había manera de enmendar
tantos yerros míos, ni de poner puertas al campo de mi disipación. Ya
serían inútiles todas las precauciones y disimulos para mantener en mi
madre la ignorancia de estos graves desórdenes. Si no lo sabía ya,
sabríalo mañana o la semana próxima. Ésta era para mí la más penosa de
las aprensiones, y el terror que mayormente me turbaba. Y no podía dudar
que algún indiscreto, o algún avieso amigo, le llevarían el cuento.
Cuantos afanes y desazones pudiera traerme mi endiablada situación,
parecíanme tolerables y llevaderos ante el conflicto inmenso de que mi
buena madre despertara del engañoso ensueño en que vivía. La muerte
sería su despertar\ldots{}

Pasaron días en ansiedad terrible y en continuo bochorno. Yo no visitaba
a nadie, no me presentaba de noche en ninguna casa conocida. Esperé
salir de las apreturas con nuevos dogales; mas aunque volaba por Madrid
en busca de un confiado judío que me atendiese, no pude encontrarle, y
llegué a creer que a todos los logreros crédulos y candorosos se los
había tragado la tierra. Y mientras esto ocurría, todo el mundo me
abandonaba; nadie iba en mi busca; huían de mí los amigos; las amigas no
me solicitaban. Sólo de Virginia y Valeria recibí, por Ramón Navarrete,
un recado afectuoso que me endulzó un poquito el alma sin aliviarme de
mi desazón. Segismunda iba de vez en cuando a mi casa, y como si yo
fuese San Esteban, que había de lograr la palma del martirio con
pedradas en el cráneo y el machacar de sesos, me decía: «Así estás
porque quieres, gran mentecato. Pronuncia una palabra\ldots{} muy
chiquita por cierto, la palabra más chica, sólo compuesta de dos letras,
y tendrás todo el dinero que necesites\ldots» Pero no me convencía, y
viendo cómo se enroscaban ante mí las serpientes de sus cabellos, y cómo
sonaban con metálico timbre las voces que salían de su cuadrada boca de
máscara griega, érame a cada instante más odiosa, y sus consejos me
sonaban a horrible sugestión de los demonios.

Pero de pronto, hallándome en el culminante punto de mi desesperación,
llegó a mí un consejo, un reclamo dulcísimo, una voz que me sacudió y
volvió del revés\ldots{} no puedo expresarlo de otro modo. Un rayo no me
partiera como me partió y anonadó un billetito de Eufrasia, escrito en
los términos más propios para destruirme y hacer de mis restos un hombre
nuevo\ldots{} El billete muy breve trazado con trémula mano, no decía
más que esto: «Niño mío, pobre náufrago, ¿te ahogas, y aún dudas?» ¡Me
tuteaba! El cariño encerrado en esta corta frase hizo explosión en
mí\ldots{} pudo más que mi conciencia y que todo lo del mundo\ldots{}
¡Me tuteaba! Teníame por suyo\ldots{} Salté de la silla, y empecé a dar
vueltas por la habitación, gritando: «No dudo, ya no dudo\ldots» Besé la
carta, y me sometí como un pájaro atontado a la fascinación de los
negros ojos, que en los trazos azules de la escritura me miraban\ldots{}
Movido de una valiente resolución salí a la puerta de mi cuarto, llamé a
la criada para decirle que avisase a mi hermano, a quien yo tenía que
comunicar algo muy urgente. Pero Agustín acababa de salir, y su mujer no
había entrado aún\ldots{} Sentime muy solo, y desconsoladísimo de no
poder comunicar a la familia mis trascendentales pensamientos.

Volví a encerrarme, y caí en profundas meditaciones. Me sentí filósofo,
me sentí \emph{pensador}, como ahora se dice, y me dio por descender con
mirada sutil hacia el fondo de las cosas. Y lo primero que en la
profundidad vi fue la pingüe fortuna de D. Feliciano de Emparán, que por
una combinación social de las más sencillas vendría pronto a mis manos
pecadoras, y si no venía para el libre dominio, vendría para el prudente
usufructo en una medida proporcionada a mis necesidades, apetitos y
larguezas. Según datos que han llegado a mí sin que yo los busque, el
ilustre señor disfruta un caudal diez veces, quizás veinte veces mayor
que lo heredado de sus padres, y éstos fueron ricos. Se cuenta que
Emparán \emph{retuvo}, el cómo no lo sé, una gran parte de los valores
públicos que poseían las monjas, y que anduvieron de mano en mano en la
catástrofe de la desamortización. Con estos papeles, D. Feliciano y
otros cuyos nombres suenan mucho, realizaron un negocio facilísimo, de
esos que no exigen rudo tr ni quemazón de cejas\ldots{} Bienes de
frailes compró Emparán por mano ajena, y bienes de aristócratas, que en
la continua liquidación del acervo histórico pasan, por pacto de retro,
o por venta al contado rabioso, de las manos que llevaron guantelete a
las desnudas y puercas manos de la usura. Del amigo Emparán son las
tierras del Condado de Tarfe, que ocupan casi media provincia; las
dehesas de Somolinos y de Doña Sancha, en las faldas de la sierra de
Gredos, y la vega de Santillán, bañada por el Tajo de arenas de oro.

Añádanse a esto las tierras patrimoniales en Azpeitia, y otras
adquiridas en el valle del Oria, en Durango, en Oñate, y se formará la
cabal cuenta\ldots{} ¡No, no, qué tonto yo! Falta una brillante partida.
¡Diecisiete casas en Madrid! De éstas, cuatro son de corredor, para
gente pobre, y como toda industria que explota la indigencia, producen
renta lucida. Entre las demás, las hay antiguas sin reforma, antiguas
pintorreadas que no logran rejuvenecerse, como los viejos que se tiñen,
y modernas de nueva planta, bien repartidas en cuartos bonitos para
empleados y pensionistas. ¡Y de todo esto voy yo a participar! Llueven
sobre mí estos bienes sin que yo haya hecho nada para merecerlos\ldots{}
Me tranquilizo recordando la idea que en la tarde del Casino, y antes de
aquella dichosa tarde, expresó Eufrasia con la serenidad y aplomo que
hacen de ella un oráculo infalible. Hela aquí: «Vivamos con todo el
bienestar posible; rodeémonos de comodidades, vengan de donde vinieren;
evitemos la penuria, las deudas; tengamos todo lo preciso para evitar
afanes; y en el seno de la opulencia bien ordenada, seamos modestos,
caritativos, religiosos y todo lo buenos que hay que ser\ldots»

El examen de la gran riqueza que yo había de disfrutar me llevó al
intento de inquirir las razones de que fuese yo el elegido para Coburgo
de la poderosa dinastía de Emparán. Habiendo tantos jóvenes de
excelentes condiciones para cargar con María Ignacia, ¿por qué se pensó
en mí, y en ello se puso tan tenaz empeño; en mí, que por mis ideas
desentono bastante de la noble familia; en mí, pobre, de muy dudosa
moralidad, paseante en corte, sin carrera ni oficio ni más patrimonio
que mi figura, mis modales finos, mi labia, mi saber ameno, hoy más
social que científico? Éste es un misterio que yo quería desentrañar, y
por Dios que lo he desentrañado, como verá el lector futuro, si tiene la
paciencia de seguirme en estas meditaciones.

Mi hermana Catalina\ldots{} lo diré con todo el respeto del
mundo\ldots{} Mi hermana Catalina es el demonio\ldots{} No quiere decir
esto que sea mala, ni que en su privada conducta y en sus relaciones con
la sociedad emplee infernales artes, ni que haya hecho pacto con el
\emph{Tartáreo Querub}, como suelen llamar los poetas a Lucifer, ni que
lleve consigo peste de azufre, ni nada de eso\ldots{} \emph{Demonio}
quiere decir el arte sumo de la astucia, de la trastienda y de la
diplomacia para lograr lo que nos proponemos; significa el empleo
habilísimo de medios espirituales para nuestros materiales fines. Es
indudable la comunicación, el visiteo y confraternidad de los Emparanes,
señor y señoras mayores, con Sor Catalina de los Desposorios, y con
otras monjas de La Latina que no conozco, y que son sin duda mujeres de
grandísimo talento para establecer y afianzar el dominio de unas almas
sobre otras, para someter, en suma, las voluntades seglares a las
voluntades religiosas. Y aquí debe de existir un factor desconocido, una
fuerza poderosa, que entre las monjas y los Emparanes actúa como
eficacísimo instrumento de captación, para que aquéllas cojan a éstos, y
los tengan en la garra y se los coman vivos cuando les venga en deseo.

Ahondando, ahondando, llego a ver en la idea de mi boda un caso inicial
de conciencia. Ha llegado a persuadirse D. Feliciano de que una gran
parte de sus bienes no son adquiridos cristianamente: cierto que no le
trajo a tal convencimiento la simple acción mujeril, y que en ello hay
de fijo obra de varón docto y que sabe su oficio. Pero si el docto varón
y las monjitas están en espiritual connivencia, como Dios manda, resulta
que, por la ley de predominio feminista, las franciscanas de la
Concepción son las amas, y las que llevan y traen a mi futuro suegro y a
las señoras mayores cogido y cogidas por una oreja. Veo también muy
claro que mi bendita hermana, unida en apretada piña con sus compañeras,
obtuvieron del opulento Emparán dones valiosos para su casa y Orden, y
entre las concesiones a que se ha visto obligado D. Feliciano, no ha
sido la más floja la mano de la niña para persona por la comunidad
designada. Sin duda, Catalina se ha hecho lenguas de mí, marcando y
enalteciendo mis cualidades, y haciendo ver quizás que el Cielo mismo me
designa para perpetuar, en mi coyunda con María Ignacia, la noble raza y
nombre de los Emparanes de Azpeitia. Fascinados éstos, míranme como el
mejor modelo de caballeros y de maridos en lo espiritual, y en lo físico
como el excelso tipo caballar para el cruzamiento y mejora de una casta
que en su vástago último aparece un poco y un mucho degenerada\ldots{}
Esto he pensado, este lógico aparato he construido para penetrar en la
sima profunda donde está la verdad, y creo haber dado con ella. ¿Lo que
he sacado de la hondura es la verdad, y verdad respiran las páginas que
acabo de escribir?\ldots{} Tú me lo dirás, ¡oh tiempo!, eterno hijo y
padre de ti mismo, que en lo que nos enseñas eres siempre el revelador
infalible.

\hypertarget{xxix}{%
\chapter{XXIX}\label{xxix}}

Entró mi hermano de la calle, y al punto que sentí sus pasos, le llamé y
le dije: «Agustín, cuando quieras, puedes visitar a los señores de
Emparán y pedirles para mí la mano de su hija María Ignacia. Mi
determinación, claramente revelada por la firmeza con que la expresé,
colmó de júbilo a mi hermano, que aturdido me dijo: `¡Ay, qué sorpresa
tan grata me das\ldots! Si te parece, voy ahora mismo. El llanto sobre
el difunto, Pepe\ldots{} ¡No vayas a arrepentirte!\ldots{} Sí, sí,
voy\ldots{} Me pongo la levita nueva, el sombrero nuevo\ldots{} Todo
nuevo\ldots{}'.»

Entraba en aquel punto Sofía, que de labios de su feliz consorte oyó la
noticia en el oscuro pasillo, y vino a mí con los brazos en cruz, y
antes que yo pudiera zafarme, me cogió y estrujó contra el colchón de su
exuberante pecho\ldots{} Sentí en mi cuello y rostro la fofa blandura,
el crujir de ballenas, y alguna de éstas me hizo daño. «¡Ay, mírame: se
me saltan las lágrimas, Pepillo! ¡Qué bueno eres! No podías menos de
rendirte a la razón, al justo medio de las cosas y al sentido
práctico\ldots{} Dispensa, hijo, que no te acompañe. Ahora mismo me
vuelvo a la calle para llevar la noticia a los de la familia, a todos
los amigos, todos, todos. Quiero que lo sepan, y que rabien\ldots{}
Alguno rabiará\ldots{} Ya andan diciendo que tal y qué sé yo\ldots{}
¿Pero no sabes una cosa? Ahora te lo digo a boca llena, porque si no te
lo digo reviento. Extendida está ya la real cédula del título de
Castilla que se concederá al Sr.~de Emparán. Será regalo de boda del
Gobierno a esa familia ilustre, firmísima columna del Trono y del
Altar\ldots{} Con que ya lo sabes: \emph{Marqués de Beramendi}, y de no
sé qué otra cosa muy sonada\ldots{} Pues hasta luego: no quiero que
nadie se me anticipe\ldots{} Ese pelmazo de Agustín, que va a pedir la
mano, no ha concluido de arreglarse\ldots{} Voy a peinarle un poco las
melenas, y a ponerle la levita bien ajustadita, para que no le haga
pliegues en la espalda\ldots{} ¡Ah!, se me olvidaba lo mejor, chiquillo.
El título no se le concede a D. Feliciano, sino a María Ignacia\ldots{}
Mira si la cosa es delicada\ldots{} Adiós, marquesito de Beramendi.»

Se fue, se fueron marido y mujer a espaciar en la calle su loco júbilo;
quedeme solo, y las meditaciones tornaron a posesionarse de mi cerebro,
presentándome las diversas fases del inmenso problema de mis nupcias.
Volví a preguntarme qué había hecho yo para merecer participación tan
lucida en aquella colosal riqueza. ¿Qué organismo social es éste,
fundado en la desigualdad y en la injusticia, que ciegamente reparte de
tan absurdo modo los bienes de la tierra? Retumba en mi mente, al pensar
en esto, el fragor de las tempestades que pavorosas estallan en toda
Europa. Mis conocimientos de las teorías o utopías socialistas reviven
en mí, y reconozco y declaro la usurpación que efectúo casándome con
Mariquita Ignacia. Yo, señorito holgazán inútil para todo; yo que no sé
trabajar ni aporto la menor cantidad de bienes a la familia humana, ¿con
qué derecho me apropio esa inmensa fortuna? Mas ahora entiendo que es
también muy dudoso el derecho de mi señor D. Feliciano a poseer lo que
posee. Por nacimiento se le dio lo que fue producto del tr de otra
generación, y por combinaciones mercantiles, con algo de políticas, ha
venido a sus manos lo que debe pertenecer a las clases indigentes, que
dejarían de serlo si recibieran lo que les corresponde, en buena ley de
Naturaleza\ldots{} Recapacitando en ello, me siento
\emph{San-Simoniano}, y afirmo que el mundo es del pueblo, de todos, y
que el derecho a los goces no es exclusivo de una clase privilegiada. La
riqueza pertenece a los \emph{trabajadores}, que la crean, la sostienen
y aquilatan, y todo el que en sus manos ávidas la retenga, al amparo de
un Estado despótico, detenta la propiedad, por no decir que la roba.

Comprendo el terror que causan estas ideas en la sociedad en que vivo.
Yo, que antes no me curaba del Socialismo y sólo me servía de él para
producir algún frívolo chiste en las conversaciones mundanas, ahora
tiemblo ante el problema, monstruo cejijunto, de grosera voz y manos
rapaces. Me pone carne de gallina la idea de que una súbita y despiadada
revolución venga a despojarme de todo esto que será mío, que ya casi en
principio lo es. A más de poseer bienes raíces y valores públicos,
tendré coches, caballos de silla (no me contento con menos de tres),
casas de campo, cotos para mis cacerías\ldots{} tendré para otros
recreos mil y mil superfluidades, de las cuales seré despojado por el
pueblo, por lo que Sofía con supremo desdén llama \emph{las masas}. Pero
bien podré yo, sigo discurriendo, prevenirme contra el desastre por
medio de un feliz arbitrio que mi riqueza me permitirá realizar.

El recuerdo de mis lecturas de Fourier y Considerant me sugiere la idea
de hacer un ensayo de la grande y nueva asociación humana dividida en
los elementales estamentos: capital, tr, inteligencia. Y sobre esta
sólida base estableceré un falansterio modelo, construido para la
existencia cómoda de los \emph{trabajadores} que en él han de habitar
por grupos o \emph{falanges}, conforme a las aptitudes y gustos de cada
uno. Por este medio me adelanto a la revolución, la inutilizo, le corto
las uñas, y\ldots{} ¡Qué tonterías digo! ¡Bonito es el genio de D.
Feliciano y bonito corte de \emph{fourieristas} el de las señoras
mayores para permitirme tales extravagancias! Y aunque me dejaran,
¿pensaría yo en ello después de cabalgar tan a gusto en el machito del
privilegio? ¡Qué delirios se me ocurren! De veras estoy loco. La
revolución vendrá\ldots{} La tormenta que vaga por Europa, de pueblo en
pueblo, descargando aquí centellas, allá granizo, en una parte y otra
eléctrico fluido que todo lo trastorna, ha de ser, andando el tiempo,
furioso torbellino que arrase el vano edificio de nuestra propiedad, sin
que contra él nos valgan falanges ni falansterios\ldots{} ¿Tardará
meses, años, lustros; tardará siglos?\ldots{} Que a mí no me coja es lo
que deseo, y que cuando estalle, ya estén leídas y dadas al olvido mis
deslavazadas Confesiones\ldots{} ¡Y con qué incongruencias nos sorprende
nuestro juguetón Destino! ¡Yo que quizás habría sido revolucionario, y
que sentí en mi alma vagos estímulos de rebeldía y protesta, ahora me
coloco entre las víctimas de la revolución, y ya no seré pueblo
justiciero, sino aristocracia justiciada, como enemigo del pobre y
ladrón de propiedad! ¡Yo que había mirado con tan tiernos ojos al dulce
clérigo Lamennais, viendo en él al apóstol del proletariado en nombre de
Cristo, primer pobre; yo que como él llamaba \emph{esclavitud moderna}
al viejo pauperismo, y pedía la redención de los menesterosos, víctimas
de un corto número de opresores y verdugos, ahora me paso con armas y
bagajes a esta minoría cruel y egoísta, y sentado en la mesa de Epulón,
arrojaré los huesos y piltrafas a la humanidad desheredada por inicuas
leyes\ldots!

De idea en idea, he venido a parar en que mi nueva familia querrá
rehacer mi personalidad en los viejos hábitos de sus devociones y de su
santurronería, así como en el continuo trato con clérigos y monjas. Eso
no: ya me defenderé hábilmente, y en último caso, mi externa
flexibilidad me permitirá compaginar las ideas con las obligaciones, que
si París valió una misa rezada, esta conquista mía vale misa cantada con
tres curas. Venga lo que viniere, ya no me arredro\ldots{} Me asalta el
recuerdo de las teorías de Owen, que hoy, con las de Fourier y las de
Saint-Simon, levantan en el mundo amenazadoras borrascas. Rechazo con
Owen todas las religiones, y establezco como fundamento moral de la
sociedad la Benevolencia. Mi riqueza me hace benévolo. Imitando al
filósofo inglés, erigiré una gran fábrica o manufactura a estilo de la
\emph{New Lanark}, y entre mis felices y bien alimentados obreros
practicaré todas las virtudes evangélicas\ldots{} Seré apóstol, seré el
Verbo de la Benevolencia universal, y daré un ejemplo a mis
contemporáneos y a las generaciones futuras para que sin dogma religioso
aguarden tranquilas las revoluciones que se avecinan, y las deshagan
como la sal en el agua\ldots{} Heme aquí, señores de la Posteridad, en
la mayor crisis de mi espíritu. ¡Yo que tan donosamente me burlé de la
llamada Economía Política, negándole títulos y honores de ciencia, ahora
ved cómo me vuelvo economista, económico, o como queráis llamarme!
¡Fatal evolución, radicales mudanzas del hombre dentro del curso de su
propia existencia, tan sólo por las misteriosas transfusiones del oro de
bolsillo a bolsillo!

\ldots¿Pero es verdad que yo soy rico, que lo seré dentro de algún
tiempo? Así parece. Pues bien: el mal camino, andarlo pronto. Con mi
conciencia hecha jirones ante mí, inútil despojo que para nada me sirve
ya, pienso que tendré coches, caballos de silla\ldots{} tres por lo
menos no hay quien me los quite\ldots{} montes para mis cacerías de
reses mayores, quintas para convidar a mis amigos; palacio en Madrid,
algún otro en provincias\ldots{} Compraré lindas estatuas y hermosas
pinturas que sustituyan a los abominables cuadros milagreros y feísimos
retratos de Pontífices, que adornan los salones de mi nueva
familia\ldots{} Y en cuanto a María Ignacia, la llevaré a París para que
los más hábiles corseteros del mundo me le arreglen aquel cuerpo
imposible, aunque tengan que amputar alguna parte de él y ponérsela
postiza; las modistas más hábiles harán para ella seráficos trajes y
sombreros olímpicos que la hermoseen, la corrijan, la\ldots{} ¡Qué
delirio! No puedo seguir.

\hypertarget{xxx}{%
\chapter{XXX}\label{xxx}}

\emph{6 de Junio}.---Al reanudar hoy el cuento de mi vida, veo que la
confesión última, con la cual debo empalmar la presente, es irrespetuosa
y depresiva para mi futura compañera. Pero, atento a que la sinceridad
resplandezca siempre en cuanto escribo, no borraré aquellos conceptos,
impresión fiel de lo que entonces pensaba y sentía. Distintas son hoy
mis impresiones, y puedo manifestar que en estos días no me ha parecido
mi novia tan desgraciada de figura como la describí en otra ocasión. Sea
porque le han puesto algún milagroso corsé, sea porque la naturaleza,
por influjo de amor, tiende a enmendar sus propias imperfecciones, ello
es que, viendo ayer a María Ignacia, antojóseme regularmente formada, y
casi casi un poquito esbelta; y aún me dan tentaciones de creer que se
le va corrigiendo la fealdad de la boca, o que se le reduce a un simple
defecto que fácilmente se disimula con la seriedad: no veo yo que sea la
risa el mejor adorno del rostro humano, y antes bien entiendo que la
mujer casada no tiene por qué enseñar los dientes.

Pues la causa de que la última confesión quedase interrumpida fue que
entraron como avalancha mis dos cuñadas, y Segismunda se precipitó a mí
para abrazarme, diciendo que quedaban olvidadas nuestras querellas y que
volvíamos a la cariñosa concordia entre hermanos, como mandan Dios, la
Sociedad, la familia, y no sé quién más. A poco llegó Agustín,
contándonos el buen acogimiento que habían dado los Emparanes a su
mensaje matrimonial. La escena fue conmovedora: el regocijo bailaba en
los ojos de D. Feliciano y de las señoras maduras. María Ignacia, cuando
entendió que yo la pedía, estuvo si cae o no cae con el accidente. «En
fin---dijo a su esposa,---para el domingo estamos todos convidados a
comer\ldots{} todos, y tú también, Segismunda\ldots{} la familia en
masa\ldots{} No faltaremos\ldots{} ¡Y qué casa, qué lujo, qué señorío a
la antigua usanza! Vengo encantado\ldots» Como un pavo cuando endereza
el moco y se hincha rastreando las alas, salió Agustín hacia su
habitación, y en apostura semejante, inflada como un globo, le siguió
Sofía, dejándome solo con Segismunda (cosa convenida entre las dos), que
al punto me dijo: «Ya puedes disponer, querido Pepe, de cuanto dinero
necesites para quitarte esa roña indecente de tus deudas\ldots{} Si
quieres evitarte la molestia de tratar con esos tíos marrulleros,
mándales a casa, y Gregorio se encargará de despacharles, recogiendo
todo tu papelorio. De buena has escapado, hijo. Ya ves cómo tenía yo
razón cuando te decía que ibas al abismo. Felizmente has hecho caso de
mis consejos, y ya estás salvo. Ahora, cuando te entreguemos tus
pagarés, nos firmas tú una obligación por la cantidad que resulte, y en
paz. Ya nos pagarás cuando gustes\ldots»

Parecíame bien discurrido el plan, y le di las gracias por su diligencia
y el cuidado de mis asuntos. Y ella, sentándose junto a mí en el modesto
canapé de Vitoria: «Pues ahora, ya que eres tú el grande, o lo serás, y
nosotros chiquitos, obligado estás a mirar por tus hermanos. Tu posición
de millonario y de marqués todo te lo facilita\ldots{} Óyeme con
atención un rato, querido Pepe. Ya ves que vamos subiendo, subiendo, no
tanto como subirás tú; pero tampoco nos arrastraremos por la tierra.
Agustín es el que no saldrá ya de la condición de empleado, y lo más a
que podrá aspirar es a una plaza de director general en Hacienda\ldots{}
que es lo mismo que nada. Gregorio y yo\ldots{} no digamos que somos
ricos, pero vamos en camino de serlo si la Providencia sigue ayudándonos
como hasta aquí. La semana pasada hemos comprado un terreno muy grande
más allá de la Era del Mico, pagándolo como fanegadas de pan llevar, y
dentro de algunos años, si Madrid crece y crece, como dicen que crecerá
cuando haya \emph{ferros-carriles}, lo venderemos a tanto el pie\ldots{}
Fuera de esto, es posible que nos quedemos con una finca muy buena en la
Vega de Añover\ldots{} Nos sale por una bicoca, y es tal que, poniéndole
riego, será, según dicen, el Potosí del espárrago y la California del
melón\ldots{} Bueno, Pepe: vete un día por casa y verás qué muebles
antiguos y modernos tengo allí, y qué espejos con marco de ébano, y qué
tapices de Santa Bárbara\ldots{} Nos hemos quedado con todo ello por un
pedazo de pan, como quien dice. Te enseñaré además un magnífico collar
de diamantes gordos montados en plata, y un par de esmeraldas
espléndidas, procedentes de la casa de Ceriñola\ldots{} Pues bien: a mí
también se me suben los humos a la cabeza, y aspiro ¿cómo no? a darme un
poco de lustre, no digo que hoy, no digo que mañana, porque es demasiado
pronto, sino dentro de un par de años, o de tres\ldots{} Eso lo dejo a
tu buen juicio\ldots{} No pretendo yo un título de Castilla, que eso me
parece mucho para mis cortas ambiciones; pero un titulito de esos que da
el Papa, y que cuestan poco dinero, sí que me convendrá, y tú, tú me lo
vas a conseguir.»

La sorpresa no me dejó expresarle ni conformidad ni reprobación. Debí de
estar un rato con los ojos muy abiertos, espantados, porque Segismunda,
sin acobardarse, prosiguió así: «¡No es para tanto asombro, vaya! Pues
qué, ¿no somos todos hijos de Dios? Tú, que pronto serás influyente y
poderoso, podrás hacer lo que te digo; y no te nos endioses ahora, ni
desprecies a los humildes. Cristeta me ha dicho que tú, con ponerle una
carta a tu amigo Antonelli, el Ministro del Papa, tendrás cuantos
títulos se te antoje pedirle, y aun es fácil que el mío te lo dé
\emph{libre de gastos}, lo que sería miel sobre hojuelas. La oportunidad
de la petición es cosa tuya\ldots{} Otra cosa: de esto no debe enterarse
Gregorio: quiero darle una sorpresa.»

No tardé en volver sobre mí, respirando de lleno el ambiente social que
tanto había contribuido a la evolución de mi conciencia y de mi
carácter, y benévolo y sonriente le dije: «Sí, sí, querida Segismunda:
lo que ambicionas paréceme muy razonable, y cuenta con que si de mí
depende la concesión del título, ya puedes empezar a usarlo. ¿Y qué,
piensas bautizar tu nobleza con el nombre de esa gran finca que pronto
será vuestra por pacto de retro, o por embargo?\ldots{} Sea por lo que
fuere, ¿fundarás en ella tu ejecutoria de nobleza pontificia?

---En ello he pensado---respondió cavilosa;---pero el título de
\emph{Condes de Titulcia}, que es el nombre del lugar próximo, no me
parece que suena bien\ldots{} ¿A ti cómo te suena?

---¡Titulcia, Titulcia!\ldots{} En efecto: como sonido es algo semejante
al de la moneda falsa, o que tiene hoja\ldots{} Suena también a título
de sainete.

---Eso digo yo\ldots{} Pues verás: devanándome los sesos, he inventado
este otro título: \emph{Condes de la Vera de Tajo}.

---¡Oh!, es admirable, como invención de tu caletre. Segismunda, tú
pitarás, tú serás Condesa, y por mi parte, espero a que me señales el
momento oportuno para escribir a Roma y empezar mis gestiones\ldots{}

---Ya contaba yo contigo. Nadie como tú ha podido apreciar mis esfuerzos
para engrandecer a la familia, y labrarnos una vida de comodidades: así
lo hace todo el que sabe y puede\ldots{} Gracias a mí, no es Gregorio un
triste empleado, y mis hijos unos pobres lambiones\ldots{} Ya ves qué
flaca me estoy quedando de tanto como discurro para marcarle a Gregorio
cada día lo que debe hacer\ldots{} Y estas noches me ha quitado el sueño
eso del maldito Socialismo, de que los periódicos hablan como si fuera
el fin del mundo. Dice Gregorio que ese tremendo huracán que anda
retumbando por las naciones quedará en agua de cerrajas; pero yo que
pienso, yo que examino las cosas, veo que ello trae miga, y muy mala
intención, Pepe, muy mala intención. ¡Vaya con la tecla de que todo ha
de ser para todos, y de que se deben repartir por igual los bienes de la
tierra! Ello será justo, pero imposible. ¿No crees tú lo mismo? ¿Quién
es el guapo que nos quite lo que hemos ganado con el sudor de nuestra
frente para dárselo a tanto vagabundo y a tanto perdido piojoso? ¿Y
habrá por esto una revolución muy grande, la sublevación de los pobres
contra los ricos, de los muchos contra los pocos? Tú que lo has
estudiado en los libros, me dirás si debo tener mucho miedo, o
tranquilizarme pensando que la catástrofe vendrá, sí, pero vendrá cuando
los que hoy vivimos estemos ya gozando de Dios.

Díjele que por lo que he sacado de mis estudios y de la observación de
lo presente, la revolución ha de venir; pero tardará un rato. Entre
tanto, debemos vivir lo mejor que podamos, y criar a los hijos, el que
los tenga, en la devoción de la buena vida, y enseñarles a que no
humillen al pobre y a que le den cariñosamente las sobras de nuestras
mesas, para que comiendo se curen de la manía de arrebatarnos lo que
poseemos.

«Me parece muy bien---dijo Segismunda:---fomentemos también la religión,
de la que nace la conformidad del pobre con la pobreza. ¿Para qué
pagamos tanto clérigo, y tanto obispo y tanto capellán, si no es para
que enseñen a los míseros la resignación, y les hagan ver que mientras
más sufran aquí, más fácilmente ganarán el Cielo?

---Justo; y entre tanto ganemos nosotros la tierra\ldots{}

---Que es lo más próximo\ldots{} y lo más seguro.»

Poco más hablamos, y se fue, dejándome en poder de Agustín y Sofía, que
con el convite en la grandiosa casa de Emparán estaban como chiquillos
con zapatos nuevos. Me consultaron si el frac de mi hermano sería
bastante de moda para una solemnidad tan extraordinaria, y si Sofía
haría mal papel llevando el vestido color de níspero con \emph{frunces}
y adorno de galones de seda. Respondile que mis presuntos suegros y las
señoras mayores saben conciliar la opulencia noble con la llaneza, y no
reparan en cortes de fracs ni en colorines de vestidos, con lo que
quedaron tan satisfechos.

\emph{8 de Junio}.---He vuelto al mundo, he reanudado mis relaciones. En
ningún semblante he visto el menor rasgo de irónica burla por mi
casamiento. He oído muchos plácemes. Alguien me ha mirado con asombro,
alguien con envidia. Sólo en las caras de Virginia y Valeria encuentro
una sombra de lástima mezclada de tristeza. No me hablan de mi boda, y
aun noto en ellas algo como supremo esfuerzo de discreción tocante a
este suceso. No pronuncian palabra alguna que suene a casorio, noviazgo,
ni cosa tal. Pero su seriedad me causa pena; creería yo que me estiman
menos, o que me miran como una amistad perdida para siempre. Ya no
revolotean junto a mí, ya no me marean dulcemente con risueñas chanzas;
ya soy para ellas un viejo\ldots{} Anoche, en sueños, las he visto huir
de mí, enlazadas de la mano, sin volver atrás los ojos, dejándome en una
especie de dorada sepultura, amortajado en hielo\ldots{}

Muchos días pasaron sin ver a Eufrasia, y la primera vez que a su lado
me encontré después de la dulce entrevista del Casino, no pudo hablarme
con confianza por estar presentes el Sr.~de Roa, Cristeta y a ratos Don
Saturno, que entraba y salía estorbándonos toda comunicación. Sólo pudo
decirme que está contenta de mí, y que no me aparto de sus pensamientos.
¿Cuándo podré verla? Respondió a esto que al Casino no volvería\ldots{}
y que\ldots{} ¡ay!, que acelerase mi boda todo lo que pudiese. Retireme
sin comprender bien la intrincada psicología de aquella mujer, mas con
esperanza de entenderla y desentrañarla pronto, algún día\ldots{} Desde
la sala próxima, volviéndome para mirarla, vi que en mí clavaba sus
negros ojos, y en ellos se me reveló su soberano talento, su apasionado
corazón\ldots{} y su profunda inmoralidad\ldots{}

Eran sus ojos el signo de los tiempos.

\hypertarget{xxxi}{%
\chapter{XXXI}\label{xxxi}}

\emph{12 de Junio}.---Ayer empezó el día con un tremendo disgusto.
Presentóseme muy de mañana una mujer desgreñada y con aspecto de loca;
rodeábala un enjambre de chiquillos de diferente edad, rotos y sucios,
mocosos y famélicos. Era la esposa del buen Cuadrado, y a contarme venía
un infortunio que para ella es como si todo el firmamento con estrellas
grandes y chicas, y el Sol y la Luna, se le hubiese caído encima. El
pobre D. Faustino, que movido del hambre más que del furor político,
tuvo platónica participación en la trifulca de mayo, llevando recadillos
y órdenes al cuartel del Hospicio, residencia del regimiento de España,
había sido preso y llevado a San Francisco. Véase cómo: Una mañanita se
presentó la policía en la casa, y sin más que un \emph{véngase usted con
nosotros}, se le llevaron\ldots{} Creyó la pobre mujer que pronto le
soltarían, como a tantos otros, por no poder probarles nada, y así se lo
decía él cuando la infeliz mujer iba a llevarle la comida. «Pero ayer,
¡Cristo Padre!---prosiguió ella,---va una servidora al cuartel y dicen:
`¿Cuadrado? Ya está en camino para el embarque'. ¡A Filipinas, Señor! Ya
me le llevan, ya se fue, ya no volverá\ldots» Y al decir esto la madre,
rompieron los pequeñuelos en tan aflictivo coro de llantos y chillidos,
que yo me vi precisado a llorar también.

Les consolé y socorrí, les aseguré que yo cuidaría de mantenerlos hasta
que el buen Cuadrado volviese, y corrí a Gobernación con ansia de
impedir iniquidad tan grande. Pero ya era tarde: ya no había medio de
tirar de la cuerda para detenerla y soltar de sus nudos un solo cuerpo
de los que a la proscripción conducía. Narváez era inflexible, y
acordadas las deportaciones, se tapaba el rostro la clemencia, pues en
todos aquellos que el Estado maldecía, echándoles de casa, estaba bien
manifiesta la culpabilidad revolucionaria. ¿Qué sería de un país sin
\emph{Orden Público}? ¿Y cómo se asegura el \emph{Orden Público} sino
desprendiendo y arrojando fuera todos los miembros o partes corruptas de
la enferma Nación?

¡Qué triste mañana, y qué atrevidos pensamientos en ella me asaltaron!
Los escribiré otro día. Ahora doy la preferencia a la carta de mi madre,
que encontré al volver a casa, y que fielmente, sin variar coma ni
punto, traslado a mis Confesiones:

«Hijo queridísimo, ya lo sé; ya estoy enterada. ¡Alabanzas mil al Señor!
Por lo que Catalina me dice, entiendo que de algún tiempo acá se le
aparecían en sueños unos ángeles que de ti le hablaban, y juntamente le
anunciaron maravillosas determinaciones del Cielo\ldots{} Que el Señor
lo ha dispuesto a su gusto y para sus altos fines, bien a la vista
está\ldots{} Digo que aquellos ángeles, y ángeles fueron aunque Catalina
por modestia no los nombre, le comunicaron la voluntad de Dios, y ella
procedió con arreglo al divino mandato\ldots{} Perdona que no vaya esto
muy bien hilado, porque la sorpresa y el contento, hijo mío, me
desconciertan todo el sentido, y tanto quiero escribir, que saltan las
palabras unas encima de otras, y no sé si escribo lo que pienso, o si
pienso escribir lo que no escribo\ldots{} Pues sí, Catalina oyó a los
ángeles, y aun creo que los vio en corporal figura cuando rezaba, y al
punto se dio a combinar y resolver que de la soledad de tus estudios
pasaras a los afanes y obligaciones de hombre casado\ldots{} Las
noticias que me da tu hermana de las virtudes de esa familia, que tiene
en su sala las imágenes de todos los Pontífices, me han hecho
llorar\ldots{} ¡Ay qué familia, y qué señores y señoras tan santos! Ello
ha sido que tú fuiste a esa casa movido del ansia de tus lecturas, y en
son de consultar libros antiguos y cuadros de Papas, y allí os visteis y
os conocisteis tú y la virginal Ignacia, de quien tendré la honra de ser
madre\ldots{} ¡Oh delicias mías, oh alegría de mi vejez, oh inefable don
del Espíritu Santo!\ldots{} Pues os visteis, y en uno y otro se encendió
un amor casto, como el de los serafines. ¡Cuán grande será tu mérito,
hijo mío, que sólo con mirarte entendieron el Sr.~D. Feliciano y esas
señoras graves que eras el niño enviado por Dios para hacer feliz
coyunda con la niña! ¡Y cuán altas y nobles serán las prendas de Ignacia
cuando tú, sólo con verla una vez, la diputaste por esposa que el Cielo
te designaba! ¡Ay!, vuelvo a llorar de alegría. Mis lagrimones caen
sobre el papel; pero sigo escribiéndote, y digo que no necesito que tú y
Catalina me ponderéis la belleza de Ignacia para que yo la vea tal cual
es realmente, la más hermosa criatura puesta por Dios en el mundo, con
la inocencia pintada en su rostro angélico, los ojos como luceros, la
boca como la misma pureza entre rosas y jazmines, y el cuerpo tan
gallardo que no hay palmeras ni juncos que se le puedan comparar\ldots{}
¡Ay, qué abrazos te doy con el pensamiento, y a ella también, a los dos,
a los dos, para que juntos recibáis los cariños de vuestra amorosa
madre!\ldots{} Como el Señor no ha de querer, pienso yo, que seas tan
sólo esposo putativo (que a sus fines no convendrá estado tan perfecto),
ya estoy viendo la caterva de graciosos nietecillos\ldots{} No, no puedo
seguir: los extremos de alegría me han llevado a soltar la pluma y a dar
por la estancia no sé cuántos paseos y aun algunos brincos. Me recojo
por si alguien entra y cree que me he vuelto loca.

»Tu padre, de la sorpresa de este notición, se ha quedado como lelo, y
tu hermano quiere ir a la boda. De los tantísimos millones que dicen vas
a poseer, nada quiero saber yo, porque eso me importa un bledo. Ya sé
que todo lo has de emplear en servicio de Dios, conforme al ejemplo que
te dan los padres de la bendita Ignacia\ldots{} Ya sé que como no tienes
vicios, ni hábitos de lujo, ni gustas de vanidades, todos esos tesoros
serán empleados en obras de religión, y ya estoy viendo el suntuoso
convento que construirás para la \emph{Concepción francisca} en Madrid.
El pajarito que todo me lo cuenta y que jamás me engaña, me dice que
harás otro convento de la misma Orden aquí, trayéndonos de priora a tu
hermana, y otro en Atienza. Buena falta hace allí una casa religiosa de
mucha santidad, que está el pueblo muy perdido\ldots{} Y ya estoy viendo
que con esos ríos de oro que entran en tu casa, se acabarán los pobres
en Madrid, pues tu mujer y tú, mis queridos hijos, no daréis descanso a
las manos en la limosna\ldots{} y tanto tenéis, que os sobrará para los
pobres de Atienza, donde por las malas cosechas están los labradores
muertos de hambre y no saben a quién volverse\ldots{} Pienso yo que, por
muchos pobres que salgan, no habrá número bastante para dar abasto a
vuestra caridad. ¿Verdad, hijo mío, que así es?\ldots{} Ahora sí que
podrá decir tu padre que se acabó el Socialismo; y por cierto que cada
mañana y cada noche me ha de dar matraca con el Socialismo dichoso. Yo
no le temo ya\ldots{} Vivan mis hijos, a quienes Dios concede tanta
riqueza para que alivien las miserias de la Humanidad, para que les
quiten de la cabeza a los pobres esa mala idea de revolucionarse por el
tuyo y mío.

»Cuento con que, recibidas las bendiciones, os vendréis a pasar el
verano en Atienza, que es tierra de mucha frescura. Allá iré yo a
prepararos la casa, y por de pronto voy a poneros unos juegos de sábanas
de hilo que la misma Reina y el Rey no los tienen mejores en su real
cama. Casaos; venid pronto, hijos míos\ldots{} No tardéis, por si me
mata tanta alegría. Yo me pasaré el resto de mi vida dando gracias a
Dios por el inmenso beneficio que te ha hecho; venid, venid. Véate yo, y
muérame después, que para nada sirvo ya en el mundo\ldots{} No sigo; no
puedo más: los lagrimones han mojado todo el papel. Recibe con ellos
para ti y para María Ignacia el amantísimo corazón de tu
madre.---\emph{Librada.»}

\flushright{Madrid, Marzo-Abril de 1902.}

~

\bigskip
\bigskip
\begin{center}
\textsc{fin de las tormentas del 48}
\end{center}

\end{document}
