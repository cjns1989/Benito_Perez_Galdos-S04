\PassOptionsToPackage{unicode=true}{hyperref} % options for packages loaded elsewhere
\PassOptionsToPackage{hyphens}{url}
%
\documentclass[oneside,8pt,spanish,]{extbook} % cjns1989 - 27112019 - added the oneside option: so that the text jumps left & right when reading on a tablet/ereader
\usepackage{lmodern}
\usepackage{amssymb,amsmath}
\usepackage{ifxetex,ifluatex}
\usepackage{fixltx2e} % provides \textsubscript
\ifnum 0\ifxetex 1\fi\ifluatex 1\fi=0 % if pdftex
  \usepackage[T1]{fontenc}
  \usepackage[utf8]{inputenc}
  \usepackage{textcomp} % provides euro and other symbols
\else % if luatex or xelatex
  \usepackage{unicode-math}
  \defaultfontfeatures{Ligatures=TeX,Scale=MatchLowercase}
%   \setmainfont[]{EBGaramond-Regular}
    \setmainfont[Numbers={OldStyle,Proportional}]{EBGaramond-Regular}      % cjns1989 - 20191129 - old style numbers 
\fi
% use upquote if available, for straight quotes in verbatim environments
\IfFileExists{upquote.sty}{\usepackage{upquote}}{}
% use microtype if available
\IfFileExists{microtype.sty}{%
\usepackage[]{microtype}
\UseMicrotypeSet[protrusion]{basicmath} % disable protrusion for tt fonts
}{}
\usepackage{hyperref}
\hypersetup{
            pdftitle={NARVÁEZ},
            pdfauthor={Benito Pérez Galdós},
            pdfborder={0 0 0},
            breaklinks=true}
\urlstyle{same}  % don't use monospace font for urls
\usepackage[papersize={4.80 in, 6.40  in},left=.5 in,right=.5 in]{geometry}
\setlength{\emergencystretch}{3em}  % prevent overfull lines
\providecommand{\tightlist}{%
  \setlength{\itemsep}{0pt}\setlength{\parskip}{0pt}}
\setcounter{secnumdepth}{0}

% set default figure placement to htbp
\makeatletter
\def\fps@figure{htbp}
\makeatother

\usepackage{ragged2e}
\usepackage{epigraph}
\renewcommand{\textflush}{flushepinormal}

\usepackage{indentfirst}

\usepackage{fancyhdr}
\pagestyle{fancy}
\fancyhf{}
\fancyhead[R]{\thepage}
\renewcommand{\headrulewidth}{0pt}
\usepackage{quoting}
\usepackage{ragged2e}

\newlength\mylen
\settowidth\mylen{……………….}

\usepackage{stackengine}
\usepackage{graphicx}
\def\asterism{\par\vspace{1em}{\centering\scalebox{.9}{%
  \stackon[-0.6pt]{\bfseries*~*}{\bfseries*}}\par}\vspace{.8em}\par}

 \usepackage{titlesec}
 \titleformat{\chapter}[display]
  {\normalfont\bfseries\filcenter}{}{0pt}{\Large}
 \titleformat{\section}[display]
  {\normalfont\bfseries\filcenter}{}{0pt}{\Large}
 \titleformat{\subsection}[display]
  {\normalfont\bfseries\filcenter}{}{0pt}{\Large}

\setcounter{secnumdepth}{1}
\ifnum 0\ifxetex 1\fi\ifluatex 1\fi=0 % if pdftex
  \usepackage[shorthands=off,main=spanish]{babel}
\else
  % load polyglossia as late as possible as it *could* call bidi if RTL lang (e.g. Hebrew or Arabic)
%   \usepackage{polyglossia}
%   \setmainlanguage[]{spanish}
%   \usepackage[french]{babel} % cjns1989 - 1.43 version of polyglossia on this system does not allow disabling the autospacing feature
\fi

\title{NARVÁEZ}
\author{Benito Pérez Galdós}
\date{}

\begin{document}
\maketitle

\hypertarget{i}{%
\chapter{I}\label{i}}

\textbf{Atienza}, \emph{Octubre}.---Dirijo hacia ti mi rostro y mi
pensamiento, consoladora Posteridad, y te llevo la ofrenda de mi vida
presente para que la guardes en el arca de la futura, donde renazca con
toda la verdad que pongo en mis Confesiones. No escribo estas para los
vivos, sino para los que han de nacer; me despojo de todo artificio,
cierro los ojos a toda mentira, a las vanas imágenes del mundo que me
rodea, y no veo ante mí más que el luminoso concierto de otras vidas
mejores, aleccionadas por nuestra experiencia y sabiamente instruidas en
la social doctrina que a nosotros nos falta; veo la regeneración humana
levantada sobre las ruinas de nuestros engaños, construida con los
dolores que al presente padecemos y con el material de tantos yerros y
equivocaciones\ldots{} Asáltame, no obstante, el temor de que la
enmienda social no sea tan pronta como ha soñado nuestra desdicha, de
que se perpetúen los errores aun después de conocidos, y de que al
aparecer estas Memorias en edad distante, encuentren personas y cosas en
la propia hechura y calidad de lo que refiero; que si la Historia,
mirada de hoy para lo pasado, nos presenta la continuidad monótona de
los mismos crímenes y tonterías, vista de hoy para lo futuro, no ha de
ofrecernos mejoría visible de nuestro ser, sino tan sólo alteraciones de
forma en la maldad y ridiculez de los hombres, como si estos pusieran
todo su empeño en amenizar el Carnaval de la existencia con la variación
y novedad pintoresca de sus disfraces morales, literarios y políticos.

Esto pienso, esto temo, esto discurro; mas no me arredro ante la
sospecha de que los futuros nada puedan o nada quieran aprender de mí,
por no sentirse peores que yo, o estimarse incapaces de mejora; que en
último caso, no habrán de negarme que mis defectos son el abolengo de
los suyos, y mis faltas semilla de las que ellos estarán cometiendo
cuando me lean, muy satisfechos de ver que los predecesores no les
llevamos ventaja en la virtud, y de que en vanidades y simplezas allá se
van los presentes con los pretéritos. Sin meterme, pues, a discernir si
mis amigos de la Posteridad son más tontos que yo, o por el contrario
más despiertos, sigo poniendo en el papel el traslado fiel de mis actos
y de mis intenciones, historiador y crítico anatómico de mí mismo. Y lo
primero que tengo que hacer en esta nueva salida de mi conciencia al
campo de la confesión, es explicar a la Posteridad el por qué de la gran
laguna de mis apuntes, suspensos desde el último Junio hasta los días de
Octubre en que renacen o despiertan de un largo sueño. No vean en este
paréntesis una voluntad perezosa, sino más bien atareada en demasía y
solicitada de mil externos incidentes, y añadan, para mi completa
disculpa, estorbos materiales de mi trabajo, como verán por lo que sin
pérdida de tiempo voy a contarles.

Es el caso que los señores de Emparán, hostigados sin duda por mi
bendita hermana Sor Catalina de los Desposorios, querían apresurar los
míos con María Ignacia, apretándoles a ello, o impaciencias de la niña,
que anhelaba la dulce coyunda, o el recelo de que yo me volviese atrás,
renegando a deshora del consentimiento que di. Esta segunda hipótesis,
como explicación de tales prisas, debe atribuirse a la desconfiada monja
antes que a los Emparanes, cuya voluntad había yo ganado con mis
demostraciones de afecto. La verdadera razón del precipitado
acontecimiento no debió ser otra que un dictamen de los principales
doctores de Madrid acerca de los nerviosos achaquillos de mi futura,
pues según oí, opinaron unánimes que la niña no entraría en caja
mientras no tomase la medicina que llamamos marido. Ved por qué móviles
farmacéuticos me llevaron una mañana de fines de Julio al convento de la
Encarnación, en cuya sacristía entramos libres María Ignacia y yo, y
esclavos salimos el uno del otro, enlazados por una moral cadena que en
toda nuestra vida no podíamos romper. No describiré la ceremonia, poco
aparatosa en verdad, conforme al gusto de mi nueva familia, que era
también el mío: una vez que nos dimos el sí, y significamos con la unión
de las manos el venturoso empalme de las existencias, recibidas las
bendiciones, oída la Epístola y cuanto quiso endilgarnos el curita que
nos casó, fuimos en coche a La Latina, a recibir los plácemes de mi
hermana y de otras monjas muy reverendas, de quienes hablaré en su día.
Allí se nos sirvió un chocolate espléndido con bollos y bizcotelas entre
jazmines, agua de limón en cristalinos vasos, alternados con búcaros de
claveles y rosas, todo ello tan delicioso que nos daba la falsa visión
de un desayuno en la Corte Celestial. La vanagloria de mi hermana se
traslucía en el rayo ardiente de sus ojos, que por los huequecillos de
la doble reja nos flechaban, y las otras monjas no parecían menos ufanas
de la victoria que habían ganado. «¡Ay, hermano mío---me dijo Catalina,
embellecida por el júbilo,---bendito sea el Señor, que me ha dejado ver
este gran día! No dejaré de alabar su misericordia mientras la vida me
dure. ¡Feliz tú, feliz tu esposa, que parecéis nacidos y cortados para
constituir una santa pareja, y realizar en la tierra los fines más
puros! Obra de Dios, no nuestra, es este matrimonio; como obra de Dios,
sus frutos serán divinamente humanos y humanamente divinos.» Oímos
atentos y conmovidos esta corta homilía mi mujer y yo, y metimos mano
por segunda vez a las bizcotelas y bollos, dejando las bandejas poco
menos que limpias, y apuramos los vasos de limón, que con el calor de
aquel día y el sofoco de la ceremonia, nuestra sed no acababa de
aplacarse.

Del convento fuimos a casa, y a las doce se sirvió la comida, a la que
asistieron como quince personas, los carlistones amigos de la casa,
Conde de Cleonard, Roa, Sureda; Doña Genara representando la rama de
Baraona, y por mi familia mis dos hermanos con sus respectivas esposas,
las cuales de la infladura de la satisfacción no cabían dentro de sí
mismas. Tampoco referiré pormenores de la comida, larga y agobiante por
causa del calor, y abrevio mi relato para llegar al más importante
suceso, que fue la libre partida, a primera hora de la noche, en viaje
de novios, con el fin de llevar nuestra luna de miel a la soledad y
frescura de Atienza. En silla particular de posta, adquirida
espléndidamente por D. Feliciano, salimos con dos servidores, la
doncella Calixta para cuidar de mi esposa, y el criado Francisco, en
calidad de mayordomo y asistente de ambos para todo servicio de viaje y
de casa, hombre excelente, de fidelidad y diligencia bien probadas.
Magnífico era el coche, los criados selectos, y para completar tan buen
avío llevaba yo un bolso con surtido abundante de monedas de oro y
plata, y Francisco un cinto con doscientas onzas, como para hacer boca,
pues la cartera de viaje contenía libramientos para cobrar en
Guadalajara o Zaragoza (en previsión de viaje más extenso) cuantas
cantidades pudiéramos necesitar.

No acabaría si a relatar me pusiera el trámite sin fin de las despedidas
y del besuqueo con que agobiaron a mi esposa su madre y la innumerable
caterva de sus amantes tías, de la rama de Baraona y de Emparán, y
Genara y las demás amigas, y las criadas todas; si describiera el
silencioso lagrimeo de D. Feliciano y los tiernos adioses de los íntimos
de la casa, y de los parientes, entre los cuales no eran mis hermanos y
cuñadas los menos hiperbólicos en las demostraciones. Creí que aquello
no tenía fin, pues terminada una ronda de besos que restallaban en las
mejillas de María Ignacia, empezaba otra ronda, y entre tantas babas,
pucheros y suspiros, se repetían sin cesar las recomendaciones de que
escribiéramos, de que nos cuidáramos, de que nos guardásemos del relente
al apuntar del alba, y los votos ardientes por nuestra felicidad\ldots{}
También a mí me tocó parte de aquellas efusiones, y hasta sobras del
amante besuqueo; sentí regado mi rostro por el llanto de las señoras
mayores, y la impresión de sus labios en mi frente y mejillas. Fue
precisa la autoridad de D. Feliciano para poner término a los adioses, y
hubimos de arrancar a mi mujer de los brazos de Doña Visita, que allí
quedó medio desmayada. A estrujones nos metieron en el carruaje, y este
arrancó por la calle de Alcalá en dirección de la Puerta del mismo
nombre, cuyo arco central franqueamos ya de noche; y cuando nos vimos
fuera, Ignacia, y yo respiramos cual si nos sintiéramos libres de un
peso y ligaduras oprimentes. En aquel punto fue común y acorde en los
dos la primera sensación de vivir el uno para el otro, para nosotros
mismos y para nadie más; por primera vez advertí en mi esposa la
satisfacción de hallarse en mi compañía sin más testigos que los
criados, y bajo el yugo de mi exclusiva autoridad. Con la vaga ternura
de sus miradas, más que con sus balbucientes razones, me decía que para
ella era yo toda su familia, y que el amor nuestro reducía los demás
afectos a secundaria condición.

No habíamos llegado a las Ventas del Espíritu Santo, cuando me pareció
advertir que la memoria de los amados padres y tías se iba desvaneciendo
a cada vuelta de las ruedas del coche, y que la pobre niña entraba en la
vida nueva con ganas de gustarla, y de morar apaciblemente en el campo
florido del matrimonio, desligada ya de la protección paterna,
innecesaria. A mí convergían todos los estímulos de su voluntad y los
vuelos tímidos de su imaginación juvenil: yo era su centro de atracción
y de gravedad; a mí volaba y en mí caía, respondiendo a mis pensamientos
con la sumisión de los suyos\ldots{} La presencia de los criados llegó a
sernos de una molestia intolerable, por lo cual resolví que no en
Guadalajara, sino en Alcalá hiciéramos la primera paradita, que había de
ser etapa capital en la existencia de Ignacia, esposa mía desde aquel
descanso en calurosa noche\ldots{} Habíamos pasado la divisoria que nos
transportaba en alegre vuelo a valles muy distantes de aquel en que se
meció la inocencia de la señorita de Emparán, y aunque para mí los
valles pasados y los venideros no diferían grandemente en ciertos
órdenes, no dejé de notar en mi ser algo grande y bello, imponente
armonía de satisfacciones y responsabilidades.

El calor nos impedía mayor celeridad en nuestro viaje: caminábamos en
las horas frescas de la madrugada y en las primeras de la noche. Por mi
gusto habría ordenado que anduviera nuestro vehículo más aprisa; pero mi
mujer no mostraba deseos de llegar pronto: hacíala dichosa el vivir
errante, y se encariñaba con la repetición de etapas y paraditas, aunque
fuese en mesones incómodos o en poblachos míseros, como las que hicimos,
por gusto de ella y al cabo también mío, en la Venta de Meco, en
Hontanar, en Sopetrán, y en un solitario y umbroso bosque junto a las
Casas de Galindo, y a la vera del manso Henares. Debo decir también que
cuando pernoctamos en Alcalá y aun un poquito antes, María Ignacia dio
en mostrarme zonas desconocidas de su espíritu, como si dormidas
facultades fuesen con el nuevo estado despertando en ella. Era como una
planta mustia que súbitamente reverdece y echa flores, sin que antes se
viera muestra de botones ni capullos en sus deslucidas ramas.
Sorprendiome mi mujer con rasgos de ternura primero, de ingenio después,
que no creí pudieran brotar de su ser imperfecto, o que tal me parecía.
Y lo más extraño fue que sus propias facciones sin encanto lo adquirían
gradualmente, por virtud de la inesperada presencia de ciertas donosuras
del entendimiento. Fue para mí criatura vuelta a criar, o mujer que en
forma de mariposa salía del caparachón del gusano. ¿Sería duradera esta
ilusión de un recién casado? Aún no es tiempo de contestarme a la
pregunta que entonces me hice.

Siempre que nos hallábamos solos, dábame Ignacia muestras felices de
aquel su renacimiento a la gracia, y tal poder tenía su mudanza
espiritual, que hasta en su fea boca se me antojó iniciada una
metamorfosis, obra milagrosa del Arte y la Naturaleza. Era, sin duda, el
momentáneo influjo de la exaltación matrimoñesca en sus verdores
iniciales, y debía yo temer de la severa realidad la pronta remisión de
las cosas a su verdadero punto. Díjome una noche Ignacia: «Cuando vean
mis papás lo buena que estoy, no lo van a creer. Ya pensaba yo meses ha
que casándome contigo no serían menester más medicinas. Pero aunque así
lo creía, me daba vergüenza decirlo. Esto de la vergüenza fue mi mayor
tormento desde que te conocí, Pepe mío\ldots{} Delante de ti estaba yo
tan vergonzosa, que ni a mirarte a mi gusto me atrevía\ldots{} ¡Vaya una
estupidez! Y cuando me quedaba sola, echábame las manos al pelo y me
arañaba la cara, diciéndome: «Por esta vergüenza maldita va a creer Pepe
que soy una bestia\ldots» Y no lo soy, ya lo has visto\ldots{} Aquí
tienes la causa de los arrechuchos que me daban. Todo era pensar en ti,
y rabiar de verme tan mal formada, y por lo mal formada,
vergonzosa\ldots{} Yo te quería, Pepe, y le pedí a Dios muchas veces que
te murieras antes que casarte con otra.»

Y otra noche: «De ti me habló una mañana Sor Catalina, y con lo que me
dijo quedé tan enamorada, que sin haberte visto nunca, te conocía ya y
estuve pensando en ti todo aquel día. Por la noche tuve un fuerte ataque
y pegué muchos gritos, y no podían sujetarme. No era más que las ganas
de verte y de tenerte a mi lado\ldots{} Pues aunque nunca te había
visto, ni sabía que existieras hasta que Sor Catalina me habló de ti, ya
éramos antiguos conocidos, Pepe, pues yo me imaginaba que vendría un
hombre muy fino y muy guapo a ser mi marido, y que me haría muchas
fiestas, y que yo me abrasaría de amor por él\ldots{} A solas conmigo,
no tenía yo vergüenza, y sin hablar, decía todo lo que se me antojaba.»

Y otra noche: «Cuando nos visitaste por primera vez, la impresión que
recibí fue de que eras como un ángel con levita, corbata, y lo demás que
vestís los hombres\ldots{} Por la noche no hacía más que llorar, llorar,
y a nadie quería decir el motivo de lo afligidísima que estaba. Pero mi
tía Josefa, que es la que me adivina cuanto pienso, se acostó conmigo,
me arrulló como a un niño, y dándome golpecitos en la espalda, me decía:
«No llores, boba, que con él te casarás, quiera o no quiera.» Por lo
visto, tú no querías, Pepe. Ya sé la razón: tu delicadeza, tus
escrúpulos de caballero por ser yo más rica que tú. Bien me lo dio a
entender la Madre Catalina una tarde, pintándote como el dechado de la
caballerosidad, con lo que mi amor por ti fue ya locura. Una noche mordí
las almohadas y las desgarré con mis dientes\ldots{} Otra me tiré al
suelo, y descalza, a obscuras, anduve a gatas por mi alcoba buscando un
botón de tu chaleco que se te cayó el día de tu primera comida en casa.
Yo lo había recogido sin que nadie me viera, y lo puse debajo de mi
almohada. Con las vueltas que di, sin poder dormir, se me cayó\ldots{}
Habías de verme como una cuadrúpeda buscando el botón\ldots{} Pues mira,
lo encontré: en un relicario lo guardo\ldots{} Lo encontré hozando en el
suelo como los cochinos\ldots{} lo descubrí por el olor, o no sé por
qué\ldots{} Ya ves cuánto te quería\ldots{} Yo confiaba en las promesas
de tu hermana, que siempre me decía: «Dios lo hará, Dios lo hará.» Y
acertó la santa señora, porque Dios lo hizo, y ahora te tengo bien
cogidito\ldots{} y ya no te me escapas, Pepillo; ya no te me escapas,
ratón mío\ldots{} que tu gata tiene las uñas muy listas y\ldots{} aunque
juegue contigo, no creas que te me vas, no\ldots{} porque te cazo, te
cojo, te aprieto, te como, te trago\ldots»

\hypertarget{ii}{%
\chapter{II}\label{ii}}

El camino carretero por donde veníamos, que es el de Guadalajara a Soria
por Almazán, aún no concluido, se nos acabó en Rebollosa de Jadraque, y
con él la comodidad del coche. Mandamos este a Sigüenza; de aquí
salieron a nuestro encuentro, prevenidos del itinerario, mi padre y mi
hermano Ramón con buenas caballerías, y en ellas continuamos el viaje
hasta la gran Atienza, donde ya estaba instalada mi madre con dos
semanas de antelación preparando el formidable avío de nuestro
alojamiento. Triunfal como entrada de reyes fue la nuestra en la
\emph{muy noble y muy leal} villa, en tiempos remotos tan despierta y
gloriosa, ogaño pobre, olvidada y dormilona. A distancia de más de media
legua por el camino de Angón, salieron a recibirnos multitud de jinetes
en asnos, mulas y rocines, enjaezados con sobrejalmas y pretales de
borlones rojos, precedidos del tamborilero y dulzainero, que oprimían
los lomos de unas poderosas burras blancas. En medio de la gallarda
procesión vi el estandarte de la \emph{Hermandad de los Recueros}, y al
término de ella se me aparecieron el que venía como \emph{Prioste} y
otros dos que hacían de \emph{secretario} y \emph{seise}, a su lado un
cura, que hacía el abad, de luenga capa los paisanos, el cura con
balandrán, los cuatro caballeros en lucidos alazanes. Y apenas llegó
cerca de nosotros la interesante cuadrilla, empezó un griterío de
aclamaciones y plácemes cariñosos, mezclados con vítores o simplemente
berridos de júbilo. Al punto comprendí que los vecinos de Atienza, en
obsequio mío y de mi esposa, reproducían la carnavalesca y tradicional
procesión llamada \emph{la Caballada}, con que la \emph{Hermandad de los
Recueros} conmemora, el día de Pentecostés, un hecho culminante de la
historia de Atienza. A la de España tengo que recurrir para dar una idea
del origen de esta venerable fiesta que ya cuenta siete siglos y medio
de antigüedad.

Menor de edad el Infante D. Alfonso, que luego fue el VIII de su nombre,
vencedor en las Navas, anduvo de mano en mano, cogido y soltado, entre
guerras y alteraciones sangrientas, por los señores feudales que se
disputaban su tutela. Ya le tenía D. Gutierre de Castro, a quien el Rey
Don Sancho había designado para la regencia, ya los Laras y otros tales,
hasta que su tío Don Fernando, Rey de León, entró por Castilla, y
apoderándose del chiquillo Rey, consiguió que las Cortes de Soria
confirmaran a su favor la entrega de Alfonsito y de las rentas reales.
Hecho esto, recluye al niño en el castillo de San Esteban de Gormaz y se
va para su reino. No contentos los señores de Castilla, o
\emph{ricos-omes}, que venían a ser algo semejantes, por el poder y la
audacia, a nuestros \emph{hombres públicos}, sacaron al reyecito de
donde estaba y lo depositaron en el castillo de Atienza, que se tenía
entonces por de los más seguros del reino\ldots{} Pero luego vino otro
bando de \emph{ricos-omes}, y no conformes con el encierro del Rey niño,
idearon robarlo y llevárselo a Ávila, empresa no fácil, porque el Rey de
León, sabedor de aquellas feudales discordias, avanzaba con su aguerrido
ejército, y ya venía tan cerca que casi se sentían los pasos de los
honderos de su vanguardia. ¿Qué hicieron los \emph{ricos-omes}? Pues
confabularse con los arrieros de la villa, \emph{recueros}, o
conductores de recuas, afamados por su robustez, ligereza y osadía, y
organizar una caravana, en la cual, clandestinamente, vestido de
arrierito, fue bravamente conducido y salvado, pasando ante las barbas
de las tropas leonesas, el niño que andando los años había de ser Don
Alfonso VIII, el de las Navas de Tolosa.

Y en cuanto cogió el cetro, quiso premiar la bizarría y tesón de los
arrieros de Atienza concediéndoles el privilegio de llamarse caballeros,
y el de constituirse en Hermandad o Cofradía para practicar entre sí la
caridad y ayudarse en los trabajos de la vida. Desgastada por el tiempo,
llega esta Hermandad a nuestros días, y anualmente, en el de
Pentecostés, celebra su hazaña con un como simulacro de ella, a la que
se da el nombre de \emph{la Caballada}, y empieza en procesión para
concluir en jolgorio y comistrajes al uso moderno. Con la idea de
obsequiarnos a mi mujer y a mí (pienso que por sugestión de mi madre)
organizaron la nueva salida de \emph{la Caballada} de este año, la cual
sorprendió y divirtió grandemente a María Ignacia. Para que comprendiese
la significación de aquel lindo espectáculo, le di la explicación
histórica que aquí reproduzco. Más que por mi propio contento, por la
sorpresa y alborozo de mi mujer agradecí la delicada invención de
agasajo tan pintoresco, y a las aclamaciones con que nos recibían
contesté con vivas a la Hermandad, al glorioso pendón y a todos los
recueros presentes, herederos de la hidalguía de los pasados.

En la falda oriental de un cerro coronado por gigantesco castillo en
ruinas, el más insolente guerrero de piedra que cabe imaginar, está
edificada la Muy Noble y Leal villa realenga. Sus casas son feas y
caducas, rodeadas de un misterio vivo; sus calles irregulares invitan al
sonambulismo; en sus ruinas se aposenta el alma de los tiempos muertos.
Dos órdenes de murallas la cercan, quiero decir que la cercaban, porque
de la exterior sólo quedan algunos bastiones y los cubos. Y de las
puertas que antaño daban paso desde el campo al primer recinto y de este
al segundo, permanecen dos en lo exterior y dentro no sé cuántas, que no
me he parado a contarlas. Por la que llaman de Antequera hicimos nuestra
entrada con cabalgata y pendón, y si bullicio hubo fuera, mayor fue
dentro, con la añadidura de los chiquillos de ambos sexos y de las
mujeres, que por todas las ventanas y ventanuchos de la carrera asomaban
sus rostros, y lanzaban exclamaciones de sorpresa y alegría. La comitiva
recorrió toda la calle Real hasta la plaza del Mercado, y entrando luego
por el arco de San Juan a la plaza donde está la iglesia de este nombre
y la casa de mi madre, llegamos al término del viaje y de la ovación. El
cura D. Juan de Taracena, que en \emph{la Caballada} venía como
\emph{abad}, y el \emph{Prioste} D. Ventura Miedes, habíanse adelantado
hasta mi casa para prevenir a mi madre. Apenas llegamos a la plaza,
acudió el cura a tenerme el estribo, y antes que el compás de mis
piernas se desembarazara de la silla, me cogió el hombre en sus
atléticos brazos, y con violento apretón privome de resuello. Fue la
primera vez en mi vida que me oí llamar Marqués, confundidos en familiar
lenguaje la llaneza y el cumplimiento. «Ven aquí, Pepillo, hijo
mío\ldots{} ¡Qué guapo estás y que caballerete! Bendiga Dios al
Excelentísimo Sr.~Marqués de Beramendi.»

Pasé de unos brazos a otros. En aquel vértigo, dando y recibiendo
saludos, perdí de vista a mi mujer. Después me contó que, apenas bajada
del caballo por mi hermano Ramón, llegáronse a ella unas mujeres con
blancos delantales, y cogiéndola en brazos sin pronunciar palabra, la
llevaron como en volandas adentro y por las escaleras arriba. Fue como
un paso milagroso, de santo arrebatado al cielo por manos de serafines.
Como recibe Dios a los bienaventurados, así la recibió mi madre, y
puesta Ignacia en un cómodo sillón, cual una imagen en sus andas,
encargáronle que no se diera la molestia de ningún movimiento y le
trajeron una taza de caldo. Tomándolo estaba cuando yo subí por mi pie,
seguido del cura, del Alcalde D. Manuel Salado y otras eximias
personalidades del pueblo, y mi madre me cogió por su cuenta para
besarme amorosa y decirme tiernas palabras\ldots{} El júbilo de la santa
señora me inspiraba cierta inquietud: la fuerza del contento, a su
cuerpo da a pasmosa agilidad, a su rostro arrebatos de color, a su
mirada un centelleo vivo, a su boca una continua tentación a la
risa\ldots{} Temiendo que diese con su alegría en los límites de la
locura, la incité al reposo; pero no me hacía caso. Alarmado la veía yo
entrar y salir por esta y la otra puerta con un vertiginoso tráfago de
menesteres, órdenes que dar, necesidades a que atender, inconvenientes
que prevenir. Y era que en la crítica ocasión de nuestra llegada,
habíamos de obsequiar a los ilustres \emph{recueros} organizadores de la
cabalgata. Felizmente abreviaron ellos la recepción, y repitiendo sus
bienvenidas y ofrecimientos, tocaron a retirada, después de poner en la
ventana de mi casa el histórico pendón de la Hermandad, en señal de que
se me nombraba \emph{Prioste} por todo el año corriente.

Ya sola con nosotros, mi madre enseñó a Ignacia los aposentos que había
de ocupar. Inauditos refinamientos de comodidad en nuestra alcoba y
gabinete encontramos, con escrupuloso aseo y tal profusión de finísimos
lienzos de cama y tocador, tal bruñido de caobas y nogales, tan
ingeniosa precaución contra moscas, mosquitos, hormigas y otros
bicharracos, que maravillados nos recogimos en aquel rincón de un
paraíso casero\ldots{} Así empezó la vida ordinaria en mi casa, y así
transcurrieron plácidos los días y las semanas, sin ningún cuidado por
mi parte, pues todos los ponía sobre sí mi buena madre, disponiendo las
suculentas comidas y la constante añadidura de golosinas, dedicadas
singularmente a lisonjear el paladar de mi esposa. En esta veía mi madre
un ser bajado del Cielo y de sobrenatural delicadeza. «¿Pero qué hija es
esta tan divina que me has traído, Pepe?---me dijo una tarde
encontrándonos solos.---¿Ha existido jamás hermosura como la suya?
¿Dónde se han visto ojos tan dulces, igualitos a los del Cordero Pascual
que tenemos en el Sagrario de la Parroquia, ni piel más fina, en cuya
comparación el raso parecería estameña, ni boca más graciosa, ni
cabellos más lucidos, verdaderas hebritas de oro de Arabia? Cuando tu
mujer se ríe, paréceme que todo el cielo se rasga dejando ver los
espacios de la bienaventuranza. ¿Ha visto nadie encías más encarnadas
que las de María Ignacia? ¿Y qué me dices de aquel cuerpo tan gordito
por arriba como por abajo, que no parece sino una de esas nubes en forma
de almohadón que se ven en los cuadros de gloria, y en ellos juegan los
angelitos y dan vueltas de carnero?\ldots{} No, no hay otra más bella en
toda la redondez del mundo, hijo mío, y ahora comprendo que te
enamorases de ella como un bobo, así me lo decía tu hermana, quedándote
en los huesos de tanto penar y discurrir por si te la daban o no te la
daban.»

Hablome también aquel día y los siguientes de la urgencia de poner
nuestros cinco sentidos, y aún eran pocos, en el cuidado de la sucesión.
Tanto tenía Ignacia de ángel como de niña, y mirada por ambos aspectos,
observábala mi madre juguetona, gustosa de ingenuas travesuras, y de
correr y brincar cuando salíamos de paseo. No encajaba esto propiamente
en la gravedad de una señora casada, según mi madre, la cual, mirando
siempre al enigma interesante de la sucesión, intentaba sujetar a su
nuera al martirio de una quietud solemne y expectante. «Hija de mi
alma---solía decirle,---no pises tan fuerte\ldots{} Anda con pausa,
sentando bien el pie, y no cargues el cuerpo a un lado ni a otro, sino
al centro\ldots» «Ángel, no abras la puerta tan de golpe\ldots{} ya ves:
ahora, con el batiente te has dado en los pechos, y parecía que la llave
se te clavaba en la boca del estómago\ldots» «Oye, no te rías así,
desaforadamente, sino poquito a poco, evitando la carcajada, que te hace
estremecer el hipocondrio, y podría sobrevenir una relajación. A Pepe le
encargo que no diga cosas de mucha gracia que te hagan romper en
risotadas, sino soserías de mediano chiste, para que te rías
moderadamente, que de otro modo la risa podría ser causa de un
fracasito\ldots» «Créeme, Ignacia, cada vez que te veo dar brinquitos,
cuando vamos de paseo, se me sube toda la sangre a la cabeza\ldots»
Tenemos una huerta muy amena y lozana, a corta distancia de la villa, no
lejos de la histórica ermita de la Estrella, y allí solemos merendar a
la vuelta del paseo. A propósito de esto, decía mi madre: «Si esta tarde
tomamos chocolate en la huerta, con D. Juan, D. Ventura y D. Manuel, no
te pongas a correr como una chicuela, ni a columpiarte en las ramas del
nogal, que esos señores se asustan de verte tan volatinera, me lo han
dicho, y también temen que sobrevenga el fracaso\ldots{} Yo te encargo
mucho que al sentarte en el ruedo tomes una postura circunspecta y de
peso, derechita, aplomándote bien sobre el asiento sin hacer
contorsiones ni cargar sobre los vacíos. Si sientes calor, abanícate con
pausa y compás lento, como se estila entre señoras; si no, posas las
manos una sobre otra y ambas sobre el vientre\ldots{} Hágote esta
advertencia, porque ayer te movías en la silla como si tuvieras azogue
en todo el cuerpo, y te abanicabas con furor, y hasta me pareció que te
reías del pobre D. Buenaventura cuando nos contaba lo del celtíbero y lo
del romano y lo del maldito agareno que armaban sus guerras en esta
villa. Más que mil libros sabe el hombre, y aunque le entendemos como si
nos hablara en griego, no podemos negarle nuestra veneración.

Previo el acordado signo de inteligencia con Ignacia, yo daba la razón a
mi madre en cuanto decía, para no turbar su \emph{sancta simplicitas},
don del cielo que a mis ojos la elevaba sobre toda la miseria humana.
Conforme conmigo, a su suegra tributaba mi mujer el homenaje de una
filial obediencia, y así vivíamos en admirable paz, gozosos,
descansados, dejándonos querer, y abdicando toda nuestra voluntad en la
de aquel ser angélico y providente que no vivía más que para nuestro
bien. Tales miramientos y cuidados, que más bien eran mimos, gastaba en
el trato de su hija, que no permitía que se levantase para tomar el
desayuno, y había de servírselo en la cama ella misma, dándole el
chocolate sorbo a sorbo, y metiéndole en la boca el bizcocho mojado,
como a los niños, con rigurosa medida de los bocadillos y de las tomas;
todo ello entreverado de frasecillas tiernas, a media lengua, como si,
más que con la hija, hablase con el nieto que según ella pronto había de
venir al mundo. Y a mí solía decirme muy seria: «Ya empiezan los
antojitos, y si no estoy equivocada, también hay mareos\ldots» «¡Pero,
mamá---le contestaba yo,---si todavía\ldots» Pero como no había razones
que de su infundado convencimiento la apeasen, tanto Ignacia como yo
dejábamos que su alma se adormeciera en aquel dulce ensueño.

Por mi padre, no menos inocente que mi madre, si bien eran de orden
distinto sus candideces, venían a mí noticias de Madrid y los dejos de
aquel mundo tumultuoso así en lo político como en lo social. Moderado
acérrimo, el buen señor ponía sobre su cabeza, después de Narváez, al
gran Sartorius que a todos nos protegía, y suscrito al \emph{Heraldo} se
lo leía enterito desde el artículo de fondo hasta el pie de imprenta
final, sin omitir los anuncios y el folletín, que era en aquellos días
\emph{Las Memorias de un Médico}, por Alejandro Dumas. Terminado el gran
atracón de lectura, extractaba mentalmente lo más interesante para
ponerme al tanto de los sucesos, y lo hacía por el método y plan de
aquel famoso periódico, que dividía todo su material en secciones bajo
la denominación de \emph{Partes: Parte Política, Parte Oficial, Parte
Religiosa, Parte Industrial}, y por último la gacetilla, noticias de
orden privado, y cuchufletas, que eran la \emph{Parte Indiferente}.

Dando a cada suceso su verdadero valor informativo, que con el tiempo
debía ser histórico, mi padre me contaba las incidencias del grave
pleito que teníamos con la Inglaterra, por haberse atrevido Narváez a
dar los pasaportes al inquieto y entrometido Embajador Bullwer; y
repetía trozos del \emph{Times} (pronunciado como lo escribimos), y los
discursos que sobre el caso oyó la Cámara de los Comunes, de la propia
boca de Lord Palmerston y de D'Israeli y del afamado Sir Roberto Peel
(pronunciado también como se escribe). También me daba cuenta del
inaudito chorreo de firmas que diariamente se agregaban a la exposición
dirigida a Su Majestad, pidiéndole que siguiera Narváez atizando palos a
roso y velloso, único medio de atajar la revolución que de las naciones
europeas quería metérsenos aquí; luego me hacía un resumen de las
críticas literarias de Cañete y de Navarrete, sobre esta y la otra
función dramática, y por fin, concediendo un modesto lugar a la
\emph{Parte Indiferente}, me refería que habían llegado Mister Price y
su hijo al Circo de Paúl, y que Macallister y su esposa maravillaban con
sus artes diabólicas al público de San Sebastián. Esta parte del
periódico solía ser más que ninguna otra del agrado de Ignacia, y yo
mismo encontraba en ella noticias que, referidas como cosa baladí
resultaban a mis ojos como sucesos de inaudita gravedad; por ejemplo:
leyó mi padre que en un pueblo de Soria se había descubierto el
estupendo caso de que todos los mozos útiles y robustos, de ocho años
acá, daban en la flor de cortarse la primera falange del dedo índice de
la mano derecha con el santo fin de eludir el servicio militar. ¡Qué
cosa más tremenda! ¡Brutal crimen contra la patria! ¿Qué país era este?
\emph{¿Quam rempublicam habemus? ¿In qua urbe vivimus?} Sin quererlo
imitaba yo a Cicerón en la iracundia de mis anatemas contra un pueblo
que de tal modo delata su desquiciamiento moral y político. Donde así se
debilita el sentimiento patrio, ¿qué puede resultar más que un engaño de
nación, un artificial organismo sin eficacia más que para la intriga y
los intereses bastardos? Esto de los \emph{intereses bastardos} fue
dicho por mi padre, que usaba para todo este modo de señalar el egoísmo
de nuestros políticos. Yo iba más allá, y con frase más enérgica marcaba
la ineptitud de la raza para las ideas modernas.

Lo que no nos decía \emph{El Heraldo} (que los papeles sólo nos dan la
corteza y rara vez la miga del pan público) lo sabíamos por cartas que
mi hermano Ramón recibía de Agustín. Las discordias entre los moderados
de más viso no dejaban a Narváez entregarse con desahogo al ejercicio de
su dictadura paternal, y por otra parte siempre estaba el hombre con la
pulga en el oído, temiendo que en Palacio le armaran la zancadilla. El
Rey no le quiere, la Reina Madre tampoco, y alrededor de Sus Majestades
bullen enemigos encubiertos del \emph{Espadón de Loja}. Las últimas
noticias de desavenencias entre los políticos eran que los acusadores de
Salamanca extremaban la guerra contra el simpático capitalista, y que
Pidal y Escosura se tiraban los trastos a la cabeza. Decíase que Pidal
trabajaba con O'Donnell para que viniese a ser la espada
\emph{moderada}, quitando de en medio a D. Ramón por atrabiliario y un
poquito populachero. Y como la inquietud de los demagogos y anárquicos
era cada día mayor, Narváez no cesaba en los envíos de deportados a
Filipinas, sistema expurgatorio que mi padre juzgaba de segura eficacia.
«No hay otro medio---nos decía con dogmático acento.---Si el cuerpo
humano no se limpia de malos humores y de los \emph{elementos} de toda
indigestión más que con las tomas de buenas purgas que acarreen para
fuera lo que sobra y perjudica, el cuerpo social no entra en caja de
otra manera, hijos míos. Y el buen resultado de estos limpiones tan bien
administrados por Sartorius y Narváez es doble, porque purgamos a
España, y a las islas Filipinas las beneficiamos\ldots{} pues.»

\hypertarget{iii}{%
\chapter{III}\label{iii}}

Llamados por las obligaciones de su oficina regresaron padre y hermano a
Sigüenza. La compañía de mi madre colmaba todos los anhelos de nuestro
corazón, y como sociedad, bastante teníamos con los amigos que nos
visitaban, descollando en nuestro afecto el Sr.~D. Buenaventura Miedes,
erudito investigador de las antigüedades atenzanas. Por su extremada
bondad, por la pureza de su alma candorosa, le perdonábamos la pesadez e
inoportunidad de sus históricas lecciones, y llevábamos con paciencia
las prolijas noticias que nos daba de la antigua \emph{Tutia}, capital
de \emph{los afamados Thicios}. Todo esto, así como las guerras de
Sertorio, la traición de Perpenna, la muerte alevosa que este dio al
arrogante tribuno militar, nos tenía sin cuidado. Una tarde entera de
las de la huerta, nos tuvo con las ansias del fastidio contándonos la
batalla que riñeron el dicho Sertorio y un tal Metelo en las
inmediaciones de Sigüenza. Luego nos habló del monte llamado Alto Rey, y
del hondo valle que al pie de esta eminencia y frente a nuestro Castillo
se abre, desde la cuenca del Henares a la del Duero. «Esta
angostura---nos dijo,---es el pasadizo habitual de la Historia de
España. Iberos y romanos, castellanos y agarenos han entrado y salido
por él en sus invasiones y continuas guerras. Por allí pasó Almanzor
cuando vino a encontrar la muerte en Medinaceli; por allí pasó el Cid
cuando despedido del Rey emprendió la gloriosa campaña que nos cuenta y
canta el Romancero; por allí todos los Alfonsos; por allí en nuestro
siglo el General Hugo; por allí el Empecinado; por allí Cabrera\ldots»

Sólo mi madre ponía en aquellas rancias historias una deferente
atención, que no por manifestarse con la fijeza de los ojos y la
benévola sonrisa era menos inconsciente. Oyéndole otra tarde repetir el
nombre de Sertorio, preguntó mi madre si el \emph{caballero romano} de
este nombre era o pudo ser antecesor de nuestro contemporáneo D. Luis
Sartorius, Conde de San Luis, pues la semejanza de ambos términos hacía
creer que fueran un solo apellido alterado por el tiempo. Acudí yo
pronto a desvanecer lo que juzgaba disparate; pero el eruditísimo
Miedes, que como buen caballero no quería que el corto saber histórico
de mi madre quedase desairado, tomó la palabra y salió por este hábil
registro: «No diré yo que los Sartorius de Sevilla vengan del romano
Quinto Sertorio; pero tampoco lo negaré, pues sabido es que la larga
permanencia de este en España dejó sin duda semilla en toda la región
Tarraconense y aun en la Lusitana y Bética\ldots{} No obstante, con
permiso de mi señora Doña Librada, me atreveré a poner en cuarentena
toda etimología romana de apellidos españoles, pues aun a la del mismo
Diego Porcellos, poblador de Burgos, que según el Cronicón Emilianense
era el apellido señorial más antiguo, le ha negado la moderna crítica el
abolengo romano, y demostrado está que no viene de \emph{procella}, como
quien dice, \emph{tempestad}; ni de \emph{porcelli}, reunión o
ayuntamiento de animalitos de la vista baja, con perdón; ni tampoco se
debe buscar su origen en el Monasterio de \emph{Porcellis}, en
territorio de Oca, como asientan Sandoval y Berganza; ni en el señorío
de \emph{Porciles}, perteneciente a la mitra de Burgos, según el libro
Becerro, resultando que ni por una parte ni por otra se puede probar que
fuera romano el tal \emph{Porcellos}, cuyo verdadero nombre castellano
fue \emph{Didacus Roderici}, que es como decir Diego Rodríguez\ldots{}
Búsquese el origen de nuestros apellidos en los troncos góticos o
germánicos y sarracenos, por donde se ve que los Bustos de Lara vienen
de los \emph{Gustioz}, \emph{Gudestios} o \emph{Gudesteos}; los González
de \emph{Gundisalvos}; los Suárez de \emph{Suero}, y estos del arábigo
\emph{Azur}\ldots» Aprovechamos mi mujer y yo la llegada del correo para
huir graciosamente de la desencadenada sabiduría del buen Miedes; pero
mi pobre madre, que en paciencia y bondad se deja tamañitos a todos los
santos del Cielo, aguantó sin pestañear el chubasco, que aún duró media
hora, más bien más que menos.

En la dulce uniformidad de aquella existencia, sucediéndose placenteras
las horas, sólo un hecho me sorprendía y maravillaba, y era el despertar
de Ignacia, el paso de su timidez a las solturas de un nuevo carácter, y
la resplandeciente aurora de su inteligencia, como un \emph{fiat lux}
pronunciado por el dios Himeneo. Mientras se trató de que nos casáramos,
en lo que, según dije, no hubo poca violencia de mi parte, ni la más
leve muestra vi del fruto que después había de admirar en ella. ¡Y yo,
en aquellos días tristes, ufano de conocer el mundo y la humanidad, me
equivocaba como un tonto, suponiendo en mi prometida las cualidades
negativas de una bestia que a su fealdad unía la supina estolidez! ¿Cómo
no percibí, cómo no adiviné las facultades de Ignacia, escondidas bajo
tan desairadas apariencias? Era que la educación encogida, con tanto
mimo y tanto arrumaco doméstico y religioso, había guardado en envoltura
de sobrepuestas vitelas aquellos tesoros, poniéndole sellos tan firmes
que no pudiera romperlos más que el matrimonio, cariño y confianza de
marido. Arrancado el sello por un amor que a los demás amores se
sobreponía, descubriéronse las escondidas joyas, y una tras otra iban
saliendo del forrado y pegoteado estuche.

La mujer que antes me había parecido despojada de todo encanto era la
misma bondad; los chispazos de razón fueron bien pronto un luminoso rayo
que todo lo encendía y alumbraba. Discurría sobre lo divino y lo humano
con un sentido que era mi mayor gozo; y descubriendo cada día nuevas
aptitudes, expresaba las ideas con donaire, que el uso iba trocando en
gracia exquisita. Pero lo más admirable en ella, lo que mayormente me
cautivaba era su templada voluntad, procurando en todo caso acordarse
con la mía y con la de mi madre, la ausencia completa de gazmoñerías,
impertinencias y salidas de tono, y el sentido de corrección unido
siempre a la ternura conyugal y filial. Desgraciadamente, a la
transformación espiritual no podía corresponder la física, y María
Ignacia en rostro y talle no podía desmentirse a sí propia. Un poco
había enflaquecido y el desaire de su cuerpo era menos notorio; en su
rostro, los ojos habían ganado en viveza, o al menos a mí me lo parecía;
la boca no tenía enmienda, por más que yo, influido de la buena voluntad
en contados momentos, la creyese menos desapacible. Diré también,
completando el elogio de mi cara mitad, que Ignacia tenía conciencia de
su falta de encantos naturales, y que resignada y tranquila sobre este
punto, no pretendía con afeites o violentos artificios disimular sus
defectos. Era una fea que no presumía de guapa ni reclamaba los honores
de tal; la sencillez y la naturalidad sin pretensiones dábanle un cierto
encanto que por momentos podía sustituir a los que el Cielo no quiso
concederle.

Adivino la pregunta que me hacen los que esto lean, y acudo a
contestarla. Sí: yo amaba a Ignacia, y mejor será que hable en presente
asegurando que le tengo amor, sin meterme en un profundo análisis de
este sentimiento, que podría resultarme estimación cariñosa. Sea lo que
quiera, mi consorte me inspira un entrañable afecto, que ha de crecer y
arraigarse con el trato. La obra de Sor Catalina de los Desposorios ha
resultado más dichosa de lo que yo creía. ¿Sabéis en qué conozco que amo
a mi mujer? Pues en que ahora me sabe muy mal la suposición de que se
hubiera casado con otro. Este otro, que no existe, pero que bien pudo
existir a poco que yo persistiera en mis escrúpulos, es un ente de
comparación, o una equis que me sirve para demostrar la realidad del
bien que disfruto. Y no entiendo por bienes exclusivamente las
materiales riquezas, sino ella, mi esposa, en quien veo un apoyo moral,
inapreciable refugio del espíritu si el Destino me depara, como presumo
y temo, grandes tribulaciones y naufragios.

La templanza del estío en aquel clima convidábanos a pasear por el
campo, y este era el mayor deleite de María Ignacia, que sabía
satisfacer su gusto sin contravenir las prescripciones de mi madre en lo
tocante a brincos y carreras. Largas caminatas hacíamos por los
contornos del pueblo, por las vegas estrechas o las lomas de sembradura
y pastos, por las sierras calvas o arbolados montes. Mi madre nos
acompañaba hasta donde le parecía, aguardándonos con Úrsula, su criada
predilecta, en cualquier paraje visible donde pudiéramos reunirnos
fácilmente. Solían ir con nosotros los chicos del confitero (D. Casimiro
Gutiérrez del Amo), alguna vez Tomasita la del Fiel de Fechos, casi
siempre Calixta, la criada que trajimos de Madrid, y Rosarito Salado, la
hija mayor del Alcalde, gran peatona, de extremada agilidad para escalar
peñas y trepar a los árboles. Admirábamos la hermosura del campo y
montañas; platicábamos con toda persona que al encuentro nos salía,
mendigos inclusive; visitábamos casas, casitas y chozas; hacíamos
paradas en medio de los rebaños, vadeábamos arroyos, saltábamos cercas;
tomábamos el tiento a la vida campesina, que es la vida madre de todas
las demás que componen la nacional existencia. ¡Mundo harto diferente
del de las ciudades, pero no menos instructivo! En él recibimos
enseñanzas más profundas que las que nos ofrece la sociedad formada; en
él nos preparamos para el conocimiento sintético de la humana vida. ¡El
campo, el monte, el río, la cabaña! No es sólo la égloga lo que en tan
amplios términos se encuentra, sino también el poema inmenso de la lucha
por el vivir con mayores esfuerzos aquí que en las ciudades, y el cuadro
integral de nuestra raza, más enlazada con la Historia que con la
Civilización, enorme cantera de virtudes y de rutinas que componen el
ser inmenso de esta nacionalidad.

Divagando en fáciles charlas, nos acomodábamos a las cortas luces de los
que iban en nuestra compañía, y si algo aprendían ellos de nosotros, yo
no extraía poca substancia de sus pintorescos relatos y de sus ingenuas
observaciones. Monte arriba, o por tortuosos senderos faldeando las
colinas, hablábamos de animales, de cosechas, de brujas, de milagros, de
pobres y ricos, de personas, anécdotas y chismajos del pueblo, o de
astronomía popular, sacándole a relucir a la luna y a las estrellas toda
su historia secular y romántica. Una tarde que volviendo del camino de
Naharros, entrábamos por junto al Salvador y la Corredera, nos paramos a
contemplar la mole del Castillo y su ingente pedestal de roca, inmensa
hipérbole del esfuerzo humano trabajando en audaz porfía con la
Naturaleza. Rosarito Salado, que siempre iba delantera, nos dijo que por
la cuesta empedrada, más arriba de la Trinidad, iba D. Ventura Miedes.
Propuso la Rosarito que subiéramos en su seguimiento; pero María Ignacia
se negó a ello recordando que mi madre nos tenía muy encomendado que no
fuéramos nunca al Castillo, porque entre sus ruinas andan demonios
maléficos, o genios burlones, amén de alimañas terrestres de lo más
dañino\ldots{} Vimos al sabio; con la mirada le seguimos en su marcha
fatigosa, y por el \emph{Arco de Guerra} tomamos la dirección de nuestra
casa.

Era D. Ventura Miedes de alta estatura que rara vez se veía derecha, sin
ningún aire ni garbo; vestía en invierno y verano un cumplido levitón
que le hacía más enjuto, y en sus andares iba siempre tan desaplomado
como si fuera movido del viento más que de su propia voluntad. Sus pies
grandísimos calzaba con zapatos de paño, en que se marcaban tales
protuberancias que parecían dos sacos negros llenos de avellanas y
nueces.

A la siguiente tarde, visitando las ruinas de San Antón, también le
vimos subir al Castillo. Como el viento fresco que venía de Monte Rey
agitaba sus faldones, y las desigualdades del piso le obligaban a hacer
balancín de sus brazos, se me representó cual un árbol escueto, de la
familia de los chopos, que descalzando del suelo sus raíces se lanzase a
correr, perseguido de Céfiro y Abrego burlones. ¡Pobre Miedes! Según mi
madre, no había hombre más completo, de corazón más puro, de procederes
más intachables. Poseedor, en mejores tiempos, de unas tierras de labor
y prados, tuvo y gozó el bienestar que da una medianía decorosa; pero la
pasión de los libros, en que empleaba lo más de su hacienda, llegando a
vender una finca para comprar papel impreso, su despego del trabajo
agrícola, y sobre tantos yerros la mala cabeza y devaneos de su mujer,
ya difunta, y de su hijo único, profesor de todos los vicios, le habían
traído a la miseria mal tapada con sutilezas de la dignidad y disimulos
ingeniosos. Vivía solo con su biblioteca y una criada viejísima, a quien
llamaban \emph{la Ranera}, que guisaba para los dos y barría toda la
casa menos la librería, donde es fama que jamás entraron escobas. La
edad del erudito señor andaba ya al ras de los setenta. Según oí, se
había conservado con ágiles disposiciones hasta bien pasados los
sesenta; pero ya iba de capa caída y daba tumbos con los pies y la
cabeza, la cual, de tanto cavilar en romanos y celtíberos, perdía
notoriamente su aplomo y gravedad.

Otra tarde que también le vimos (y era la tercera vez) camino del
Castillo, mi madre no le quitó los ojos hasta que le vio perderse entre
los muros, como el aguilucho que penetra en su nido, y a poco nos dijo
suspirando: «A mí, que le conozco bien, no me hará creer D. Buenaventura
que todas esas visitas al Castillo, mañana y tarde, son para deletrear
los garabatos, en lengua romana o arábiga, de aquellas piedras más
viejas que el pecar. Todo lo que allí escribieron los antiguos, lo tiene
el buen señor bien sabido de memoria. Va sin duda por la querencia de
alguna familia de menesterosos que se ha refugiado entre las ruinas,
porque habéis de saber, hijos míos, que no ha nacido hombre más
cristiano ni más caritativo que este señor de Miedes. En pobreza y falta
de medios pocos le ganan. Pues ahí le tenéis buscando miserables con
quienes partir el pedazo de pan que Dios le concede.

---Así es sin duda---dijo María Ignacia.---Ayer me contó la Prisca que
le vio subir muy de mañana con un manojo de cebollas y la mitad de un
pan de cuatro libras. Pobres habrá en el Castillo, y si usted nos da
licencia, allá iremos Pepe y yo a conocerles y a llevarles algo para que
coman y vivan. Mala cosa es la necesidad, y no tiene perdón de Dios el
que conociéndola no acude a remediarla.

\hypertarget{iv}{%
\chapter{IV}\label{iv}}

---Andaos con pulso en esto, queridos hijos---díjonos mi madre,---que si
os inflama el espíritu de caridad, bien podéis satisfaceros mandando
vuestra limosna con persona de casa. Pero no subáis: yo no he subido
nunca, que desde niña me infundieron miedo al Castillo, y jamás, en mi
larga vida, lo he podido desechar. ¿Llamáis a esto superstición? Dadle
el nombre que gustéis: yo lo llamo respeto a la costumbre, y
persistencia en los sentimientos que en mi niñez me inculcaron. Harto sé
que es pecado creer en brujas y en apariciones de duendes o trasgos;
pero no me negaréis que el Espíritu Maligno existe, y que hay Infierno,
y por consiguiente diablo y diablillos que andan siempre en el
ministerio de tentarnos y hacernos todo el mal que pueden\ldots{} Y no
me digáis que lo que hace D. Buenaventura podéis hacerlo vosotros, pues
con eso no estoy conforme. Es el amigo Miedes muy descuidado, no sólo en
las ideas, sino en su persona y vestimenta, como habéis visto, y con tal
de socorrer a una cuadrilla de vagabundos, no repara en que sean gitanos
piojosos o ladrones disfrazados de mendigos. ¿Qué le importan a él las
porquerías y el mal olor? Me ha contado \emph{la Ranera} que una vez,
volviendo de pasar la tarde entre unos húngaros caldereros, trajo el
buen señor tal carga de miseria, que para limpiarle y mondarle el cuerpo
fue menester ponerle en cueros vivos y sahumar toda la ropa. ¿Pues quién
os asegura que los tales inquilinos del Castillo no son una partida de
bandoleros, que se hacen los pobrecicos para merodear durante la noche y
quizás para asesinar al que cojan descuidado? No, no; no subáis allá,
que yo, por de pronto, trataré de sonsacar al sabio para que me cuente
el motivo de tantas subidas y bajadas, llevando provisiones de boca y
trayendo\ldots{} sabe Dios lo que traerá.»

Interrogado al día siguiente, Miedes nos contestó con evasivas que
aumentaron nuestra curiosidad. Lo que mi madre principalmente daba por
averiguado era que el erudito de Atienza padecía miseria horrorosa, que
ya no cabía dentro de los decorosos engaños. Para remediarle sin ofensa
y proveerle de víveres, mi madre se valía de mil artificios. Con
pretextos más o menos ingeniosos, allá iba el criado casi todas las
mañanas llevando al anticuario, para que lo probase y diera su opinión,
bien la cesta de patatas nuevas, bien la ristra de cebollas, el montón
de judías o la media docena de frescas lechugas, todo de nuestra feraz
huerta. Con estos regalitos y otros que en forma no menos delicada le
hacía el Cura, se apañaba el pobre y reparaba las faltas de su menguada
despensa.

Invitado a cenar con nosotros el Cura Don Juan Taracena, nos dio
explicación de las antiguas y de las nuevas candideces caritativas del
Sr.~de Miedes, refiriéndolo con risas y comentarios humorísticos que
revelaban así la compasión por el anticuario, como la estima en que
tenía sus buenos sentimientos. «Es un sabio tonto---nos dijo,---y un
alma de Dios, en la cual se juntan la erudición pasmosa y una
simplicidad digna del Limbo. Desde que le conozco, y de ello hará
treinta años largos, le he visto dominar todas las ciencias históricas y
proteger a todos los perdidos. Su mujer le salió rana, y pez el hijo
único que tuvo, el cual desde temprana edad despuntó por su vagancia y
malos instintos. El dinero de Miedes, antes que suyo era del primero que
lo había menester, y con tanto descuido lo daba, que era como si se
dejase robar o si se estafara a sí mismo. Regalaba hoy un puñado de
duros al primer farsante que pasaba por el pueblo, y mañana le veíamos
remendando sus propios zapatos. Delante de mí cambió una excelente mula
por dos tomos del \emph{Cronicón del Obispo de Tuy}. En cierta ocasión
hipotecó el prado de Huérmeces para socorrer a unos parientes pobres,
que a los dos meses le pusieron pleito; y cuando su mujer, que se había
fugado con Boceguillas, fue a parar abandonada y enferma al hospital de
Cogolludo, ¿qué hizo el hombre? Pues ir en su busca y socorrerla y
traerla a casa.

---Eso es caridad---dijo prontamente mi madre,---y con perdón, no hay
que vituperarlo.

---Caridad es, sí señora, y soy el primero en alabar el rasgo; pero
fíjense en una cosa: para todos los gastos del viaje a Cogolludo y
retorno, y el costerío de médicos y medicinas, vendió el sabio por poco
más de un pedazo de pan sus tierras de Cincovillas. ¿Y todo para qué?
Para que la Bibiana se pusiese buena. Buena que estuvo la condenada, le
faltó tiempo para fugarse con el barbero de Zorita de los Canes\ldots{}
¿Y Miedes? Pues emborronando una resma de papel para demostrar\ldots{}
allá lo mandó a la Academia de la Historia\ldots{} para demostrar que el
llamado \emph{García Eneco}, yerno de \emph{Isur} o \emph{Suero}, y
muerto en la batalla de Albelda, no es Íñigo Arista, primer caudillo de
los navarros, sino\ldots{} qué sé yo, el demonio coronado. Para no
cansar a ustedes, ¿saben de qué gentuza se nos apiada hoy D. Ventura?
¡Ay! estos son otros Sueros, otros celtíberos o de la familia del propio
Túbal, el primer vecino de España. ¿Se acuerda usted, Doña Librada, de
aquel Jerónimo Ansúrez, que llegó acá de la parte de Sacedón hará diez o
más años, tomó en renta las tierras de los Garcías del Amo en
Alpedroches, y unas veces por poca suerte y asolación de sequías y
pedriscos, otras por mal arreglo, vino a la ruina, y anduvo en justicia,
los hijos se le desmandaron, y uno de ellos dio muerte al molinero de
Palmaces?

---¡Ah! sí, ya me acuerdo\ldots{} ¡Ansúrez! Llamábanle \emph{el
alforjero}, que este es el mote que aquí damos a los de
Alpedroches\ldots{} Ya recuerdo\ldots{} Y el hombre tenía lo que llaman
ilustración, o un atisbo de ella. Se expresaba con donaire y daba gusto
oírle.

---Como que le criaron los benedictinos de Lupiana, y hasta su poco de
latín burdo sabía. ¿Recuerda la señora que tuvimos que echar un guante
los pudientes para reunirle con qué salir de aquí? Pues esta calamidad
de familia fue a caer en el Burgo de Osma, donde no tuvo más suerte o
mejor conducta que en Atienza. Uno de los hijos mató a un sanguijuelero,
y otro descalabró al alcalde de Quintanas Rubias. Echados del Burgo, se
perdieron de vista por algún tiempo. Dispersáronse los hijos como para
asolar toda la tierra: uno de ellos dicen que se mutiló el dedo índice
para esquivar el servicio del Rey; volvieron algunos junto al
padre\ldots{} Por fin, según entiendo, después de vagar en tierras de
Soria y de Teruel, o pidiendo limosna, o quizás tomándola antes que se
la den, han recalado por aquí.

---¿Y esos son---dijo mi madre tan sorprendida como alarmada,---los
nuevos amigos del bendito Miedes?\ldots{} ¿Y esa es la pandilla que
visita y la miseria que socorre?\ldots{} ¿Ansúrez\ldots?

---El mismo que viste y calza\ldots{} Miento, que según me ha dicho el
sabio, van todos ellos un poco ligeros de ropa.

---Pues debemos vestirles y calzarles---dijo Ignacia,---para que cuando
entre el frío no les coja en tal desamparo. ¡Pobrecitos!

---Ya sabrá nuestro Alcalde---indiqué yo,---qué clase de huéspedes
tenemos, y procurará darles pasaporte. Sean como quiera, vagos de
oficio, apóstoles de la religión del \emph{dolce farniente} o ladrones
en cuadrilla, no se van de aquí sin que yo los vea.»

Sobre esto se discutió largamente, opinando mi madre por que no subiera
yo al Castillo, a menos que me acompañase con la Guardia civil el señor
Cura, para que su presencia ahuyentase y confundiese cualquier invisible
maleficio que por allí anduviera. Defendió María Ignacia con calor la
visita, y resumió graciosamente el Cura las diferentes manifestaciones
proponiendo ir todos, menos mi madre, a quien contaríamos lo que
viésemos, en la seguridad de que ni rastro de demonios o duendes
habíamos de encontrar en aquellas alturas. Sin negar que existiesen
demonios, aseguró el buen Taracena que él no los había visto nunca, como
no fueran tales los que en forma humana vemos por el mundo, con cara y
hábitos de perversos egoístas, embusteros, crueles, hipócritas, matones
y aficionados a lo ajeno. Para estos no había más exorcismo que la ley,
y a falta de esta la sanción religiosa, que a cada cual en la otra vida
designa su merecido según sus obras. Es el Cura de San Juan de Atienza
un excelente hombre, puntual y correctísimo en las funciones de su
ministerio, buen maestro en cosas del mundo y en el conocimiento de toda
flaqueza, sin que se le pueda poner tacha más que por los pecadillos de
hablar sin freno, de comer con demasiado gusto y abundancia, y de beber
intrépidamente en solemnes casos. Siendo yo niño y él grandullón, me
quería, y con amenos cuentos, a veces sucios, nunca deshonestos, me
divertía; ahora me considera, y gran devoción tiene por mí. Aunque nada
me dice, yo le descubro la ambición de una canonjía de dignidad en la
catedral de Sigüenza. Ya veremos\ldots{}

Y a la mañana siguiente muy temprano, cuando yo no había salido aún de
mi cuarto, sentí discretos golpes de nudillos en la puerta, y a poco una
voz comedida y grave que decía: «Sr.~D. José, si la señora Marquesa está
con usted en este camarín, no pretendo entrar, ¡Dios me libre!; pero si
está usted solo en sus lavatorios de caballero, le suplico que, aunque
se halle en paños menores me franquee el paso, que es muy urgente, pero
mucho, lo que tengo que decirle.»

No conocí la voz de Miedes hasta la mitad de la oración suplicante, y
antes de que sonaran los últimos vocablos abrí la puerta, y doblándose
penetró en mi cuarto la estirada figura del sabio de Atienza. Con menos
pureza de frase que la que comúnmente usaba, turbado y presuroso, me
pidió que interpusiese mi valimiento con el Alcalde D. Manuel Salado
para que este no arrojara del Castillo al infeliz padre y más infelices
hijos que entre aquellos muros se albergaban, y que le quitase de la
cabeza la cruel idea de mandarles a Guadalajara por etapas entre estos
cuadrilleros a la moderna que llamamos guardias civiles\ldots{} Como yo
me mostrase muy dispuesto a secundar sus humanitarios propósitos, díjome
con cierto temblor del habla que los tales no podían ser calificados de
malhechores ni tampoco de personas recomendables, y que su exacta
calificación no será fácil mientras no se admita con carta de naturaleza
regular la clase y matrícula de \emph{delincuentes honrados}, o sea de
los que por designio de la Fatalidad, o por impulso de las hondas
necesidades no satisfechas, hambre y sed, o por diversos móviles nacidos
de las mismas leyes que nos protegen, así como de las que nos oprimen,
se ven lanzados a una o más acciones\ldots{} maléficas, o con
apariencias de maldad, conservando en sus almas la buena intención y el
principio fundamental de la virtud\ldots{}

No copio más que lo esencial de la retahíla que me endilgó el cuitado
Miedes, acariciando los botones del levitín que yo acababa de ceñirme, y
añado que la cabeza de mi amigo ilustre me pareció enteramente
trastornada. Con todo ello se redobló mi curiosidad. Mi mujer, no menos
interesada que yo en el asunto, vistiose prontamente en el cuarto
próximo y salió a saludar al sabio; invitámosle a desayuno; recogimos a
Taracena, que en el comedor nos esperaba ya charlando con mi madre;
echonos esta su bendición, y subimos a la Trinidad para emprender de
allí la marcha hacia el Castillo. Por el empinado sendero, explicaba D.
Juan a mi mujer la importancia de aquella feudal fortaleza y atalaya,
las ventajas de su emplazamiento frente a la angostura o pasadizo que
comunica las dos Castillas; y D. Ventura, que a cada paso que dábamos me
parecía más dislocado del cerebro, me anticipó la presentación de las
ilustres personas que íbamos a visitar: «\ldots{} Este Ansúrez, Jerónimo
en lenguaje cristiano, por distintos motes conocido: \emph{el alforjero}
en Alpedroches, \emph{hidalgo} en Bustares, \emph{bragado} en Atienza,
respeño en Hiendelaencina, hombre aquí y acullá digno de estudio, no
tiene, como verá usted, nada de vulgar. Por algún tiempo le diputé
sucesor de aquel famoso \emph{Abo l'Assur}, o \emph{Al Ebn Asshaver},
que de ambos modos lo designan las historias, señor de las ciudades de
Nájera y Viguera, en los confines de Castilla y Navarra\ldots{} pariente
próximo de \emph{Abo l'Alondar} \emph{(hijo del Victorioso)}, a quien se
atribuye la destrucción de la antigua \emph{Centóbriga}, que algunos
llaman \emph{Contrebia}\ldots»

Por piadosa cortesía, que siempre debemos a los dañados del juicio, le
manifesté mi sorpresa de que se hallaran tan dejadas de la mano de Dios
personas de altísimo abolengo; y él me contestó: «No presume este buen
hombre de linajudo. La investigación de su progenie es cosa mía\ldots{}
cosa enteramente mía, Sr.~D. José\ldots» Y parándome luego en lo peor de
la cuesta, cuando ya María Ignacia y el cura se aproximaban a las
ingentes ruinas, el trastornado investigador de la Historia bajó la voz
para decirme con misterioso acento: «Dando vueltas en el magín a esta
pícara idea, he venido a rectificar mi primera opinión, y cayendo del
burro de mis preocupaciones arábigas, opino y sustento que estos Ansúrez
no tienen nada que ver con el caballero \emph{Abo Assur}, ni con ningún
otro de casta agarena, y que su abolengo es celtíbero, pura y
castizamente celtíbero, como lo acredita el nombre, que derivo del
\emph{Zuria} o \emph{Zuri}, digamos \emph{Jaun Zuri (el señor blanco)},
tronco y fundamento de los afamados vascones.» Di algunos pasos hacia
arriba; pero Miedes me detuvo, clavó en mis botones la crispada garra, y
mirándome con ojos centelleantes, acabó su lección en esta extraña
forma: «Es indudablemente el \emph{Zuria} celtíbero, conservado al
través de los siglos en su prístino vigor de raza. Demuestro, como dos y
tres son cinco\ldots{} sí, D. José querido, lo demuestro, y veamos si
hay un guapo que me desmienta\ldots{} demuestro, digo, y ello es tan
claro como la luz del día, que este \emph{Zuria} viene de aquella rama o
familia céltica que del Monte Taurus o de la Paphlagonia nos mandó el
Oriente y se estableció en esta región, que andando los siglos vino a
llamarse \emph{Algaria}, en labios del moderno vulgo \emph{Alcarria}. La
tal rama céltica, que Strabón y Appiano llaman \emph{Kimris}, y Diodoro
de Sicilia \emph{Cimmerianos}, era sin duda la más hermosa, la más
inteligente; y no falta quien sostenga que estas tribus, a su paso por
el Ática, engendraron a los Titanes y a los dioses Saturno, Rea y
Júpiter, de quienes salió todo el paganismo; como también se dice, y yo
no he de negarlo, que de los mismos proceden los hebreos y
caldeos\ldots{} Que en el curso de tantos siglos y con tantas
alteraciones y mudanzas se mantiene pura esta soberana raza, la más
bella, Sr.~D. José; la mejor construida en estéticas proporciones,
Sr.~D. José, la que mejor personifica la dignidad humana, la indómita
raza que no consiente yugo de tiranos, Sr.~D. José, bien a la vista
está; y usted podrá, ¡carambo! apreciar por sí mismo estas verdades, que
no desmentirá\ldots{} verdades que no consiento sean contradichas,
porque aquí está Ventura Miedes para sostenerlas en todo terreno, Sr.~D.
José\ldots{} para imponerlas y hacerlas tragar a los incrédulos y
testarudos\ldots{} Lo dice Ventura Miedes, y basta, basta\ldots»

Pensé que me arrancaba los botones. Ya comenzaba a serme molesto el tal
sabio, y hube de apartarle para seguir mi camino. En esto, mi mujer y el
Cura, que habían traspasado ya el arco de entrada al Castillo, salieron,
Ignacia de prisa y ceñuda, Taracena con calma y jovial. Advertí en mi
esposa una palidez y expresión de susto que me alarmaron, y no dudé que
había visto algo muy desagradable. Antes que yo pudiera interrogarla, me
dijo: «No entres, Pepe\ldots{} Mamá tenía razón\ldots{} Hay demonios.»

\hypertarget{v}{%
\chapter{V}\label{v}}

La franca risa con que el buen párroco acogió estas turbadas
expresiones, me tranquilizó. «No hagas caso, Pepito: la señora Marquesa
se asusta de la majestad del lugar, de la imponente elevación de los
muros. En cuanto a los habitantes, nada tienen de terroríficos. Entra y
verás.

---Fue la primera impresión---dijo Ignacia agarrándome el
brazo.---Entraré contigo si quieres; pero mejor fuera no haber venido.

---¡Qué tontería! Sean lo que quieran, ¿nos van a comer? Entremos, y
vaya por delante de \emph{cicerone} el Sr.~de Miedes.»

Salvamos el boquete abierto en el adarve, pasamos junto al cubo, que
enhiesto y amenazador se mantiene, desafiando el cielo, subimos la
escalera que conduce al interior de la torre del Homenaje, de la cual
sólo queda un cascarón informe, y bajo una bóveda festoneada de
hierbatos, encaramos con la familia errante, que allí tenía su aposento.
Adelantose a recibirnos el padre o cabeza de la pequeña tribu, Jerónimo
Ansúrez, el cual, con cortesía solemne, muy de caballero, nos dio los
buenos días. Era un viejo hermosísimo, de barba corta como de quien
abandona por muchos días el cuidado de afeitarse, expresivo de ojos,
aguileño de nariz, la cabeza gallardamente alzada sobre los hombros, el
cuerpo airoso y gentil, fácil en los movimientos, noble en las
actitudes, vestido de paño pardo con no pocos remiendos, que parecían
heráldicos dibujos. Quedeme absorto mirándole, y por estar tan fija en
él mi atención, tardé en hacerme cargo de las otras figuras. Eran sus
hijos, tres en pie, dos tumbados. Al extender la vista por el círculo
que formaban no lejos de su padre, vi entre ellos a una mujer, que
subyugó mis ojos. Era la mujer más hermosa que yo había visto en mi
vida. Ni en Italia ni en España se me apareció jamás hermosura que con
aquella pudiera compararse\ldots{} Perfección tal de rostro y formas no
se hallara más que en la Grecia de Fidias. Diría que me pareció
cariátide; pero su temprana juventud no acusaba la necesaria robustez
para sostener arquitrabes con su linda cabeza\ldots{} La vi arrimada a
un trozo de muro, a la izquierda; era la figura más distante de la de su
padre. Apoyaba el codo derecho en una piedra, en la mano la barbilla.
Cruzados los pies desnudos, cargaba sobre el izquierdo el peso del
cuerpo esbeltísimo, incomparable en todas sus partes y líneas, de
absoluta proporción en todos sus bultos.

«Es mi hija Lucila---dijo el padre señalándola, y ella mirándonos con
curiosidad un tanto desdeñosa, no hizo ni un movimiento de cabeza ni
pronunció palabra alguna.

---Este es el hijo segundo---dijo Miedes designando a un muchachón
fornido, guapo, de tez tostada, que altanero nos contemplaba.---Su
nombre es \emph{Didaco} o \emph{Yago}, aunque vulgarmente lo llaman
Diego. Y este otro es \emph{Egidio}, Gil que decimos ahora.»

El tal \emph{Egidio}, jovenzuelo muy parecido a su hermana, se adelantó
a besarnos la mano. Junto a él vimos al que Miedes llamó \emph{Ruy}, un
chiquillo como de diez años, lindísimo, curtido del sol, medio desnudo,
con una piel cruzada en la cintura que le asemejaba al San Juan Bautista
de la iconografía corriente. Los dos restantes eran yacentes estatuas:
el uno dormía, el otro acababa de despertar y con soñolientos ojos nos
miraba.

«Y a estos dos gandules---preguntó Taracena riendo,---¿qué nombre les da
el amigo Miedes? ¡Ah! ya me acuerdo: el tagarote grande es
\emph{Gundisalvo}, y el otro \emph{Leguntio}. Dígame, Ansúrez: ¿ese
Leoncio ha cumplido los catorce años?

---Los cumplirá dos días después de la Virgen de Septiembre. Es el que
sigue a Gil, y Gil sigue a Lucila, que ya cumplió los diez y nueve.

---¿Y cuál es el que se cortó el dedo para escaparse del servicio del
Rey?

---Es ése que duerme, mi tercer hijo, Gonzalo: al mayor, que se llama
como yo, lo tenemos en Ceuta, por un \emph{achaque}\ldots{}

---¿Llama usted achaques a los crímenes?

---Por una mala querencia, señor. Acciones hay malas que son nacidas del
mucho querer.

---Como el querer de aquel galeote que se enamoró de la cesta de ropa. Y
dígame: este \emph{Gundisalvo}, o Gonzalo, ¿es el que domestica cuervos
y les enseña el habla, igualándolos a los loros?

---No lo tome a risa. Dos cuervos educó en el Burgo, que hablaban griego
y latín\ldots{}

---Vamos, que ayudarían a misa.

---Mejor que muchos cristianos. Uno se vendió y a Francia lo llevaron:
el otro me lo robó un sanguijuelero.»

Nos sentarnos, y sacando cigarrillos, a todos les di, y fumaron el padre
y los hijos mayores. Mi mujer, que de mi brazo se colgó pesándome en
algunos momentos, no desplegaba los labios, y Miedes hablaba en voz
queda con la moza Lucila, cuyo timbre de voz hasta mí llegaba como dulce
y lejana música. Interrogado Ansúrez por el Cura y por mí acerca de las
desdichas que le habían traído a tal pobreza y desamparo, se sentó en
una piedra, y con gran sencillez de lenguaje, ni jactancioso ni servil,
sino en un punto de sinceridad grave, nos dijo: «Yo, señores míos, soy
un hombre de buen natural, ni de los que van para santos, ni de los que
merecen condenarse; bueno cuando me ponen en condición de serlo, malo
cuando me obligan a volver por mi interés; mas no tanto que puedan los
más tirarme la piedra. El mundo es malo de por sí, y esta nuestra tierra
de España tan sembrada y rodeada está de males, que no puede vivir en
ella quien no se deje poner trabas en manos y pies, dogales en el
pescuezo, que al modo de cordeles son las tantísimas leyes con que nos
aprieta el maldito Gobierno, y lazos los arbitrios en que nos cogen para
comernos tantos sayones que llamamos jefe político, alcalde, obispo,
escribano, procurador síndico, repartidor de derramas, cura párroco,
fiel de fechos, guardia civil, ejecutor y toda la taifa que mangonea por
arriba y por abajo, sin que uno se pueda zafar\ldots{} Yo, aquí donde me
ven, no soy de los más legos, que los benitos de Lupiana me enseñaron
lectura y escritura, y me apacentaron el entendimiento con libros que en
mí dejaron alguna ciencia, aunque corta\ldots{} Pero sin saber cómo pasé
de aquel vivir a otro, y me metí a labrador, lo cual fue, pueden
creérmelo, como meterme en el laberinto de la perdición y en el infierno
de la miseria. Quien dice labranza dice palos, hambre, contribución,
apremios, multas, papel sellado, embargo, pobreza y deshonra\ldots{}
Pues aunque labrador, digo que no soy lerdo, y que si no me falta
paciencia, condición primera del que se pone a dar azadonazos en la
tierra mirando siempre para el cielo, me sobra lo que llamamos orgullo,
o como se dice, apersonamiento, que es el hipo de no dejarse atropellar,
ni permitir que a uno le popen y atosiguen. Labrar la tierra es cosa
dura, ¡ay!\ldots{} ¡con doscientos y el portero!\ldots{} y por labrarla
de la peor suerte, con trabajo propio en tierras ajenas, salta en cada
momento la cuestión de las cuestiones, aquella que ya trae revueltos a
los hombres desde que los hijos de Adán, o sus nietos y biznietos,
dieron en sembrar la primera semilla: la cuestión del tuyo y mío, o del
averiguar si siendo mío el sudor, mía, verbigracia, la idea, y míos los
miedos del ábrego y del pedrisco, han de ser tuyos los terrones abiertos
y la planta y el fruto\ldots{} Pues yo, que sé trabajar como el primero,
que en el libro de la tierra y del cielo estrellado leo sin equivocarme,
no he podido trabajar nunca sin que a cada vuelta me salieran la Partida
tal, el Fuero cual, el fisco por este lado, la escribanía por otro, las
ordenanzas, los reglamentos, las premáticas, el amo de la tierra, el amo
del agua, el amo del aire, el amo de la respiración, y tantos amos del
Infierno, que no puede uno moverse, pues de añadidura viene el sacerdote
con sus condenaciones, y delante de todos el guardia civil, que se echa
el fusil a la cara\ldots{} y si uno chista, cátate muerto. ¿Quién vive
así? Yo he sido honrado, luego tentado a no serlo. Me han perseguido, me
han atropellado, me han quitado lo mío y lo que tomaba para que los
tomadores de lo mío me pagaran con lo suyo\ldots{} me han metido en
cárceles, me han puesto en escritura con papeles, y aquí estoy valiendo
menos que la tinta que gastaron en contar mis desavíos; he perdido en
una semana lo que en seis años gané; he recibido palos y los he dado con
más gana de romper cabezas que de guardar la mía, y, por fin, llego a la
vejez cansado de la lucha y sin otro provecho que las amarguras,
rabietas y achuchones\ldots{}

»Yo he mirado siempre por mis hijos, y ellos, si bien me quieren, mal me
asisten, porque han heredado mi orgullosa condición, y son tales que no
sufren dueño, de lo que resulta que descalabraron a mucha gente, y a más
de cuatro hicieron sangre, pues cada cual tiene su honor, que no de otra
manera que con sangría debe lavarse si es manchado. Mis hijos son
bravos, sufridos, y de mucho ingenio para todo; sólo que no ha nacido
quien los meta en cintura, porque yo, que hacerlo podría, he olvidado el
modo de ordenar a los demás, no sabiendo ya cómo a mí propio me ordene.
Somos todos indómitos, y aborrecemos leyes, y renegamos del arreglo que
han traído al mundo los reyes por un lado, los patriotas por otro, con
malditas constituciones que de nada sirven, y libertad que a nadie
liberta, religión que a nadie redime, castigos que no enmiendan a nadie,
civilización que no instruye, y libros que no se sabe lo que son, porque
este los alaba y el otro los vitupera. Por encima, un Dios que mira y
calla y no suelta mosca, y por debajo un Diablo que si uno quiere
venderse a él, no da ni para zapatos: tacaño el de arriba, tacaño el de
abajo, y los hombres que están en medio, más tacaños todavía\ldots{} Y
si con lo dicho les basta para conocerme, no se hable más, y socórranme,
librándome de que la Guardia civil nos fusile, o de que un juez de manga
estrecha nos meta en el pudridero de una cárcel\ldots{} El señor
Marqués, que es poderoso, hable con el Alcalde para que nos dé un
salvoconducto con que podamos llegar a Madrid, pueblo grande y revuelto,
donde hallaremos algún modo de vivir ni mas honrado ni más deshonrado
que los muchos que por allí hay. Oíganlo, señores, y sean compasivos, y
no nos tengan por peores que los tantísimos que andan por campos y
ciudades amparados de leyes, vestidos de doctrinas, y con todos esos
atalajes de honradez que han inventado los muchos para comer a costa de
los pocos, o los pocos que supieron hacer su granjería de la necedad de
los muchos.»

La primera impresión de este discursillo fue que teníamos que
habérnoslas con un pillete de finísimo sentido y trastienda. María
Ignacia le oyó absorta, yo con el agrado que comúnmente producen las
bellezas del arte popular, Taracena con burlonas risas. Miedes, sentado
a distancia, la cabeza entre las manos, parecía hondamente abstraído.
Preguntado si era viudo, Ansúrez nos dijo: «Viudo tres veces. Mi primera
mujer era manchega, aragonesa la segunda, las dos de muy buen
ver\ldots{}

---¿Y la tercera?

---Hermosa si las hubo\ldots{} valenciana\ldots{} Con esta no estuve
casado por bendiciones, sino por nuestro arrimo y conveniencia natural.
De Dios están gozando las tres\ldots{} Mucha ley me tenían, ¡con
doscientos y el portero!

---¿Y qué nos cuenta el amigo Ansúrez de esta hija tan guapa, de esta
Lucila ---preguntó el Cura,---a quien el Sr.~Miedes llamará
\emph{Lucinda}, \emph{Lucania o Lucinelda?}

---Esta hija mía---replicó Ansúrez mirándola cariñoso,---ha venido a
estas miserias por lo mucho que quiere a su padre: ¿verdad, Lucihuela?»

Con miradas no más contestó la hermosa, conservando su gravedad de
estatua. Los chistes, no de muy buen gusto, con que Taracena ponderó el
contraste entre tan admirable belleza y la ruindad de su vestimenta (que
sólo consistía en una vieja falda y en una envoltura de trapo para el
cuerpo), no merecieron de ella ni fugaz sonrisa. Pensé que a todos nos
despreciaba profundamente.

«Aquí donde la ven los señores, sabe expresarse como las personas finas;
sólo que es muy vergonzosa, y su mal pelaje le aumenta la cortedad. En
una de las peores borrascas que me ha traído mi mala suerte, la puse a
servir. Hallándose en Molina de Aragón, la vio una señora de Zaragoza, y
tanto gustó de ella y de su buen modo, que se la llevó consigo, y en su
casa la tuvo, tratada y vestida como una damisela, no sin que también le
dieran la enseñanza de leer, escribir y algo de cuentas, coser, bordar y
otras filigranas\ldots{} Pero como para mi generación no hay manera de
torcer el maldito sino con que todos venimos al mundo, la dama
protectora de Lucila cerró la pestaña, y los herederos, que no gustaban
de intrusos, plantaron a mi niña en la calle sin más que lo puesto y un
cestito con vituallas para dos días. Anduvo la pobre de puerta en puerta
en busca de acomodo, y ya porque lo hallara muy malo, ya porque el que
halló pecaba de bueno en demasía, ello fue que mi honrada niña corrió
por montes y laderas en busca de padre y hermanos, y después de andar
todos tomando lenguas, ella por nosotros, nosotros por ella, nos
juntamos en la gran ciudad de Tarazona, y de ella hemos venido en
luengos meses partiendo nuestra miseria, como los señores nos ven\ldots»

Al llegar a este punto de su historia, hizo Ansúrez como que se secaba
una lágrima, y Lucila miró para la otra parte de las ruinas; mas no
advertí que llorase. Pensé que no gustaba de vernos, sintiéndose quizás
ofendida de nuestra curiosidad reparona, y deseando la soledad como el
más preciado ambiente de su salvaje belleza. De improviso levantose mi
mujer, y cogiéndome el brazo, con notoria inquietud y turbación me dijo:
«Vámonos, Pepe; no quiero estar más aquí.»

No la insté a consentir que permaneciéramos un ratito más interrogando a
los Ansúrez, porque la vi con ardiente anhelo de retirarse. Tiraba de mi
brazo con fuerza, y sin darme tiempo más que para prometer a los
desgraciados que intercederíamos en favor suyo, me sacó de las ruinas
repitiendo: «Vámonos\ldots{} salgamos de aquí, si no quieres que me
ponga mala.»

\hypertarget{vi}{%
\chapter{VI}\label{vi}}

De mediano talante estuve toda la mañana, pues el grato efecto de la
visita al Castillo se me convirtió en amargura viendo a María Ignacia
muda y cavilosa, metida en sí, cual si una idea pesimista esclavizara su
pensamiento. Sagaz observadora mi madre, al pasar junto a nosotros,
murmuraba: «¡Cuando digo yo que hay demonios!» Con su sombría tristeza
efectuaba María Ignacia una violenta reversión a los días pasados; se
parecía más a mi novia que a mi mujer; creyérase que se le disipaba la
recién adquirida gracia, y que se extinguían los chispazos de
inteligencia, volviendo a imperar el mohín de niña vergonzosa y la
desapacible estolidez de los días en que se me propuso el casorio. De
sobremesa, se me antojó romper el silencio que mi mujer y yo
guardábamos, convencido de que callando no íbamos a ninguna parte, y de
que las explicaciones razonables disiparían aquella nube. Y antes de que
yo dijese lo que decir quería, me interrumpió Ignacia con esta
observación: «Guapísima es la hija de Ansúrez, ¿verdad? No creo que
exista en el mundo mujer más hermosa. ¿Qué dices tú, Pepe?

---Digo que es linda, sí; pero que con aquella suciedad y aquel vestir
harapiento\ldots{} Quita allá, mujer.

---O eres tonto verdadero, o tonto fingido, Pepe, y a mí no me haces
creer lo que has dicho. ¡Suciedad! Todo eso es música. No había de
tardar mucho en lavarse y ponerse como una patena cuando lo
necesitara\ldots{} Y a mí me parece que como la hemos visto luce más su
hermosura. Parece una estatua, un cuadro no sé si de la Virgen o de
alguna diosa muy al fresco y a la pata la llana\ldots{} Es la belleza en
estado natural, lo mismo que Dios la crió. ¿No eran así las mujeres de
la antigüedad, cuando nosotras no usábamos corsé, y ustedes los hombres
no conocían los pantalones, y andábamos todos con trajes largos, túnicas
o qué sé yo qué\ldots?»

Al quedarnos solos, prosiguió María Ignacia de este modo: «Te aseguro
que esa mujer me ha trastornado. ¡Qué quieres! empiezo a creer en el mal
de ojo. De veras te digo que me cambiaría por ella, comprometiéndome a
estar descalza toda la vida, mal cubierta de guiñapos indecentes,
vagabunda, sin casa ni hogar\ldots{} siempre que adoptaras tú la misma
vida, dejándote crecer las guedejas y cambiando tu condición de señorío
por el oficio de vender burros o de componer calderos. Con tal de tener
la cara de esa mujer y su cuerpo precioso, yo diría la buenaventura, y
tú y yo nos ejercitaríamos en robar lo que pudiéramos. Puedes creerme
que es verdad lo que digo. Dios, que ve los corazones sabe que no
miento, que no me hago la romántica\ldots{} Mujer y esposa, estimo la
hermosura como el mayor de los bienes: todo lo demás no vale nada.»

El tema era gracioso; pero aunque intenté glosarlo con todo el ingenio
de que yo podía disponer, no conseguí hacer reír a María Ignacia, ni
sacarla de su tenebrosa melancolía. Como había comido poco y estaba
necesitada de alimento y distracción, le propuse que fuésemos a dar un
paseo por el camino de Riofrío, llevándonos una buena merienda. Aprobó
mi madre este plan, y antes de las cuatro ya teníamos preparada una
cesta con diversidad de fiambres y golosinas, la cual fue por delante,
alternando en cargarla los chicos del confitero y Calixta; luego salimos
mi mujer y yo con Tomasa y Rosarito Salado. La tarde se presentó
calurosa, por lo que no andábamos muy aprisa, y requeríamos la sombra
que las encinas y castaños proyectaban sobre el sendero a la falda del
Padrón de Atienza. Media hora llevábamos de paseo, cuando advertí que de
la parte de los altos de Barahona venía una nube parda con visos
amarillos en sus rebordes desgreñados; avanzaba como fúnebre cortina que
sólo cubría parte del cielo, pues hacia el Oeste brillaba el sol. La
nube pareciome de las que traen mala intención, y esta sospecha fue
confirmada por el sonar lejano de truenos hacia el Este. Felizmente
llevábamos a prevención paraguas y sombrillas, y no faltaban por allí
casitas en que guarecernos en caso de aguacero. «Me alegraré de que
llueva,» dijo María Ignacia, que de su mal humor se consolaba con las
displicencias de la atmósfera, o en estas vio perfecta imagen del estado
de su espíritu. Que la nube nos estropearía la tarde quitándonos el
regocijo de la merienda, ya no podíamos dudarlo viendo los goterones que
nos mandaba el cielo, y que caían estampando en el camino redondeles
como piezas de dos cuartos. No tardó en deslumbrarnos un relámpago que
de lo más próximo de la nube venía, y con el trueno que a poco retumbó,
echonos el cielo una rociada de agua y viento que no nos dio tiempo a
buscar abrigo. Ruidos en lo alto anunciaban estragos mayores; la lluvia
era como un sin fin de látigos que nos azotaban. Rosarito se amparó tras
una peña; guarecidos mi mujer y yo bajo una encina, vimos que empezaban
a caer con las gotas confites de hielo, que tal parecía el granizo,
primero del tamaño de cañamones, luego como garbanzos. Las exhalaciones,
difundiendo en todo lo que alcanzaba la vista repentina claridad lívida,
nos deslumbraban. «¿Tienes miedo?» pregunté a mi mujer; y ella me
respondió: «Ninguno; que caigan las piedras como castañas es lo que
deseo.»

Sobrevino una clara, y quise aprovecharla para llegar hasta un caserío
que veíamos a tiro de fusil. Emprendida la marcha, ¡María Santísima!, y
cuando no habíamos andado un tercio del camino, estalló sobre nuestras
cabezas formidable estruendo, y fuimos azotados de lluvia y piedra, que
ya superaba el grandor de las avellanas. Apretamos el paso, defendiendo
nuestras cabezas de los coscorrones del cielo, y pudimos alcanzar la
casa más próxima en un momento verdaderamente angustioso, pues al llegar
al amparo del edificio, ya eran nueces lo que con estruendo y vibración
del aire caía\ldots{} Ante nosotros corrían los cerdos, las cabras,
ávidas de refugio; corría también Rosarito con las faldas por la cabeza;
y cuando llegamos jadeantes, apedreados y hechos una sopa, vimos que
bajo el ancho balcón de la casa unas veinte o treinta mujeres, algunas
con sus críos en brazos, puestas de rodillas en actitud luctuosa,
invocaban al cielo con lamentos desgarradores, mezclados de oraciones, y
con súplicas que en algunas bocas se trocaban en blasfemias. Nunca vi
espectáculo más lastimoso, ni oí voces que más hondamente me
sorprendieran y aterraran\ldots{} Como si el cielo, benigno en su
fiereza, hubiera esperado a que estuviésemos en salvo para descargar
sobre la tierra toda su ira, la terrible lapidación tomó fuerza
aterradora: las piedras, cayendo en espesa lluvia, eran ya como huevos,
y el suelo se vio pronto cubierto de aquel blanquísimo material.
Algunas, como proyectiles lanzados por furibunda mano, rebotaban al caer
y salpicaban en pedazos angulosos, estallando como rotos vidrios, y a la
caída sonaban como un chasquido de huesos o de bolas de billar. Al
compás de la furiosa pedrea crecía el gran vocerío de las mujeres,
roncas ya de tanto pedir misericordia. A la Virgen invocaban unas
creyéndola más compasiva, otras a San Roque, a San Antonio, o a la
Santísima Trinidad, que era lo más seguro, y alguna voz que empezó
rezando el Padrenuestro, lo acababa diciendo: «¡Señor, Señor, que esté
una trabajando todo el año para que venga una cochina nube de ese
cochino cielo a quitarle a una lo ganado!» Y por otra parte oíamos:
«Santos, ¿qué jacedes que esto consentides? Mala peste con vos y con el
cura que no echa las aconjuraciones\ldots» «Virgen del Pilar, acude
pronto acá y libranos\ldots» «San Roque, ¿a dónde vos metéis, santico,
que estos cielos dejáis a los demonios?» «Padre nuestro\ldots{} todo
perdido, todo arrasado\ldots{} venga a nos el tu reino\ldots{} mi
patatal que estaba como un verjel de Dios, y ahora\ldots{} el pan
nuestro\ldots{} Perdición, Señor, perdición y vengan rayos\ldots»
«Jesús, Jesús, ¿aónde estás metío, señor Jesús de la cruz a cuestas?»
«Tiran coces los ángeles, y aquí nos mandan los cascos del pavimento
celestial\ldots» «Virgen, para, para; ya no más\ldots{} que nos
morimos\ldots» «¿Quién da patás en el cielo, y quién descuaja los
afirmamentos y nos echa encima too este vridio?» «¡Malhaya quien
trabaja, malhaya quien trae criaturas al mundo! Santo Jesús, ¿no diz que
sodes Pastor? ¿Por qué matas tu ganado? ¡Trocarte has en labrador para
que no mandes truenos, ni esta encandilación de tufo de azufre, ni estos
cantos de dos libras!» «¿Qué pecado hicisteis, patatas mías; en qué
habedes faltado, judías, tomates y lechugas?» «Apóstoles y mártires,
¿qué enfado tenéis? Semos pobres, trabajamos para vivir, y nos dejáis en
los huesos. Pelados huesos, ya no tenéis sino hebras de carne, y estas
hebras los perros de la contribución vendrán a quitárnoslas. El niño no
saca de nuestros pechos más que amargura, y el marido, si no le dan
vino, quiere que seamos burras para el trabajo\ldots» «¡Malhaya el
mundo, malhaya el trabajo, ábranse las sepulturas!» «¡Justicia caiga
sobre los malos, no sobre los pobres, que meten su alma en la tierra!»
«Virgen pura, Madre nuestra, líbranos de todo mal perverso, quítanos el
rayo y la piedra, amén, y guarece nuestros campos, amén, amén, amén.»

En su consternación, no faltaron a la cortesía las espantadas mujeres, y
nos abrieron paso. El amo de la casa nos dio un buen acogimiento en el
lugar de más respeto, que era la cocina. Mi mujer contemplaba, por un
estrecho ventanucho, el tremendo caer de piedra, y se divertía viendo a
Rosarito y a los chicos correr en busca de los mayores guijarros de
hielo y traerlos para que les tomáramos el peso. Algunas mujeres se
recogieron junto a nosotros, enumerando con febril palabra los estragos
causados por el temporal en sus huertos y plantíos. «¿Pero será verdad
que lo habéis perdido todo?»---les decíamos. «Sí, señor Marqués y
Marquesa, todo perdido, todo arrasado. Trabajamos para la nube, que se
come nuestro sudor en tan intanto que se reza un credo. Lo mismo fue
hace tres años\ldots{} La contribución, que nos la pidan a tiros, como
el cielo nos afeita el campo a pedradas.» Por disposición de Ignacia,
Tomasa y Rosarito repartieron entre aquellos infelices el contenido de
la cesta, y fue muy interesante ver cómo en breve tiempo las bocas de
algunas mujeres y de los chicos dieron cuenta del copioso repuesto. El
generoso aldeano que nos albergaba mandó recado a casa, a fin de que
viniesen con socorro de vestidos para mudarnos. Despejose el cielo a las
seis, y salieron las labradoras a buscar a sus hombres y a medir el
aterrador destrozo de sus campos.

Vino a poco el Alcalde con el secretario Zafrilla y gente de mi casa
para conducirnos al pueblo, como si fuésemos náufragos o aeronautas
caídos de las nubes, y aunque en ello había más oficiosidad y adulación
que justificado servicio, lo agradecimos. Mudados de ropa y puestos en
camino, díjome Salado que, sabedor de nuestros caritativos sentimientos
en pro de los refugiados en el Castillo, había dispuesto que se les
dejase salir libremente, dispensados de los honores de la Guardia civil,
y socorridos por cuenta del Ayuntamiento hasta Guadalajara. A esto dijo
María Ignacia, reiterando su gratitud al Alcalde, que no bastaba
permitirles la salida, sino obligarles a que salieran, antes hoy que
mañana, pues tal gente vaga y sin oficio conocido no era el mejor
ejemplo para un pueblo tan honrado como Atienza. En ello convinimos
todos, y a este punto encontramos a Taracena presuroso, que también
quería coadyuvar a nuestro salvamento. Mi mujer se adelantó con el cura,
y Zafrilla con Rosarito, llevando de batidores a los expedicionarios de
menor cuantía, y Salado y yo, a retaguardia de la caravana, charlamos un
poco sobre la calidad y circunstancias que creíamos ver en los Ansúrez.
Según D. Manuel, el padre es inteligentísimo en toda labor agrícola, y
conocedor de \emph{cuanto hay en la Naturaleza}, hombre de bien,
\emph{en el fondo}, pero echado a perder por las desgracias, por su
descuido y falta de orden, y mayormente por la índole perversa de sus
hijos, que si eran malos \emph{de suyo}, la miseria los hacía peores. De
Lucila no dijo más sino lo que ya sabíamos, que era una magnífica
hembra. ¡Lástima que el padre no la vendiera! Venderíanla quizás sus
hermanos si pudiesen, o esperarían unos y otro a llegar a Madrid, lugar
de ricos compradores, que saben apreciar el ganado de calidad superior y
no regatean su precio. «¡Vaya una res, compadre!---decía un poquito
encandilado de ojos, parándose ante mí en mitad del camino.---Y puedo
dar fe de que si mucho le falta de ropa, otro tanto le sobra de orgullo.
No he visto mayor recato, ni menos tela en lo que debe taparse.

---Es que ahora viaja en calidad de estatua, y como tal estatua no
repugna el desnudo, ni se deja querer.

---Pues no es de mármol ni de talla, Don José mío, que ayer le pude
echar un pellizco y\ldots{} Por poco me pega\ldots{} Cuando llegue a
Madrid, si antes no la roban, tendrá que ver esa ninfa después de un
buen lavatorio.

---Yo me la figuro lavada y bien vestida, y\ldots{} me parece que
pierde, quiero decir que estará menos bella.

---¡No, por Dios, D. José\ldots! Yo me la imagino con ropa, y
francamente\ldots{}

---Vamos, le gustaría a usted ponerle ropa.

---Naturalmente, para quitársela.»

No pudimos seguir porque mi mujer retrocedía con Rosarito, llamándome.
Inquieto corrí hacia ella, entendiendo que se sentía mal. «¿De qué
hablabais?---me dijo colgándose de mi brazo.---¿Por qué se iban quedando
atrás y a cada ratito se paraban? Alcalde, ¿podrá decirme qué cosas de
tantísimo interés le contaba usted al marido mío?

---Señora---replicó Salado prontamente,---le hablaba de establecer en
Atienza una fábrica de jabón.

---¡Jabón! ¿Y a quién quieren lavar? ¡Valientes pillos están ustedes!
Vayan por delante y no se aparten mucho. Que yo los vea\ldots{} Y
cuidado con secretearse. Ya saben que por lejos que se pongan, yo todito
lo oigo\ldots{} y nada se me escapa, ¡cuidadito!»

\hypertarget{vii}{%
\chapter{VII}\label{vii}}

Es Salado un trucha de primera, si falto de autoridad y luces para el
gobierno de la ínsula concejil, sobrado de marrulleras habilidades para
los enredos de campanario y los empeños de su egoísmo. Servicial y
deferente con los poderosos y con todo el que ayudarle pueda en su
privanza política, guarda sus rigores de ley y sus asperezas de carácter
para los humildes sometidos a su vara, por una punta más dura que roble,
blanda por otra como junco. Nada teme de los de abajo, infeliz rebaño de
hombres sencillos, más embrutecidos por la miseria que por la
ignorancia, los cuales bajo el falso colorín de una Constitución que
proclama y ordena franquicias mentirosas, gimen en efectiva esclavitud.
Nada teme tampoco de los de arriba, con tal que en \emph{la votada}
saque el candidato que se le designó, y se constituya después en agente
o truchimán del diputado, del jefe político y del ministro, cualesquiera
que sean los caprichos contra la ley o antojos contra la justicia que
inspiren los mandatos de estas insolentes voluntades. Fuera de las
infamias propias del oficio, que pocos ven, porque los que trabajan y
sufren están ciegos, insensibles, y los que tienen luces y algún dinero
huyen de los pueblos para refugiarse en Madrid, donde lo espacioso de la
jaula garantiza relativamente la libertad y la dignidad cívica; fuera de
esto, digo, Salado puede figurar entre los hombres corrientes,
simpáticos, agradables, tan dispuestos para un fregado como para un
barrido. Casado y con hijos, es mejor padre que esposo, y mejor Alcalde
para sí que padre para el pueblo que administra.

Sigo contando. Cerca ya de la Puerta de Antequera, salió el sacristán de
San Gil, apodado el \emph{Né}, a contarnos la más lastimosa ocurrencia
entre las innumerables, cómicas y trágicas, que produjo el pedrisco.
Pasando por alto las gallinas y pollos ahogados, el cerdo que perdió el
uso de la palabra, quiere decirse del gruñido, la burra que en los
momentos de pánico se metió en la iglesia y no paró hasta la sacristía,
la desaparición de cabras, cabrones y carneros; omitiendo asimismo la
rotura del brazo de la \emph{Tía Mortifica}, las descalabraduras de
otras viejas, las caídas de ancianos y tullidos que por su calidad de
pordioseros representaban menos valor que los animales, puso el narrador
toda su labia en referirnos el grave estropicio de D. Ventura Miedes.
Bajaba el benéfico sabio de socorrer a los Ansúrez (y consta que les
llevó tres libras de peras y una botella de tostadillo), cuando fue
sorprendido del temporal, y si él apresuraba el paso para evitar la
lluvia y los coscorrones, más prisa se dieron las piedras en caer
furiosas, creciendo de volumen a cada segundo. Arrebatado de su cabeza
el sombrero por una racha, fue a parar a la veleta de la torre de la
Trinidad. Hallábase el pobre D. Ventura en lo más desamparado del cerro,
sin ver en derredor suyo árbol ni cueva, ni pedazo de muro en que
guarecerse, y en esto las piedras como huevos de gallina, de los de dos
yemas, le caían sobre el cráneo y las sienes, aporreándole sin ninguna
compasión. Una, mayor que las demás, como huevo de pava, le dio con
fuerza y se rompió en cascos de hielo; vino luego un canto que más bien
parecía ladrillo, y al tremendo golpe perdió el sentido D. Ventura, y
cayó rodando por el suelo hasta dar en un hoyo, donde aún el cielo
despiadado siguió apedreándole como los gentiles a San Esteban.

Presenciaron esto desde el pórtico de Santa María unos mendigos; mas no
pudiendo socorrerle, dieron voces, que con el estrépito de la granizada
oír no pudo ningún cristiano. Pasado había la tormenta y ya lucía el
arco iris, cuando fue descubierto el infeliz Miedes hecho un ovillo
entre montones de granizo, y le recogieron medio helado y casi difunto,
llevándole a su casa en una burra, a la manera de los sacos que van al
molino, la cabeza cayendo por un lado, los pies por otro. Visto y
examinado del médico D. Pascual Pareja, dijo este, según nos refirió el
\emph{Né}, que las abolladuras hechas en el casco por las piedras eran
de cuidado; pero que la mayor gravedad estaba en los propios sesos, de
la conmoción y el \emph{derramen}. Grande fue nuestra pena por el
accidente del anciano sin ventura. Ignacia me dijo: «Día que empezó tan
mal no había de concluir sino con esta sarta de calamidades horrorosas.»

Habríamos corrido a casa de Miedes si no estuviese muy cerrada ya la
noche y no sintiéramos tanta prisa de vernos junto a mi madre. En casa,
el fenómeno meteorológico no había causado ningún desperfecto grave.
Describiendo con pintoresco estilo la lluvia de piedra, mi madre nos
dijo que creyó ver la espadaña de San Juan volando por los aires y
estrellándose sobre nuestro techo. Cenamos, y María Ignacia, rendida del
cansancio, se durmió con sueño tranquilo, Por la mañana despertó gozosa,
poseída de un cierto ardor de beneficencia, y me propuso socorrer a las
víctimas del temporal. «¿Y de los del Castillo qué se sabe?---me dijo
risueña.---A esos no los parte un rayo. Si se van hoy, debemos
favorecerles, y fuera de aquí arréglense para vivir con las mañas que
usan; que llevando algún dinero serán mañas menos malas.» Pareciome esta
observación la propia sensatez, y sobre lo mismo hablábamos después del
desayuno, cuando nos avisaron que el Sr.~Ansúrez, a punto de partir,
quería despedirse de nosotros y darnos las gracias. No quisimos hacerle
esperar, y encontramos al \emph{celtibero, secundum} Miedes, con uno de
sus hijos en la cocina, donde ya mi madre nos había tomado la delantera,
llevando dos hogazas, un manojo de cebollas y un cesto de ciruelas, para
obsequiar a la trashumante familia. Por cierto que en aquella segunda
entrevista, hubo de parecerme aún más gallarda que en la primera la
figura del viejo Ansúrez, y su rostro más impregnado de exquisita
nobleza. Sus elegantes actitudes no desmerecían con la pobre vestimenta
del coleto burdo, el remendado calzón y las abarcas de cuero. Su afable
sonrisa, su despejada frente, sus cabellos blancos, todo el conjunto de
su vejez vigorosa me hacían el efecto de ver reproducidos en él los
caballeros de remotas edades, que seguramente no irían mejor vestidos,
ni hablarían con más entonada y cortés gravedad. Su hijo Gonzalo, que en
realidad veíamos por primera vez, pues en nuestra visita de la mañana
anterior dormía, era una hermosa figura juvenil, el rostro ennegrecido,
los ojos con llamas, la mano poderosa, el desplante galán y altanero.

«Queremos---dijo el padre sin extremar la inclinación del
cuerpo,---despedirnos de Sus Excelencias y ofrecernos para cuanto hayan
menester de nosotros en estas o quellotras tierras\ldots{} Manden lo que
gusten, que si por nuestra pobreza no podemos servirles en acordancia
con lo que son Sus Mercedes, válganos por lo chico del servicio lo
grande de la voluntad.»

A mi pregunta de si pensaba la tribu trasladarse a Madrid, contestó que
él trataba de mantener a toda la familia en un haz y llevarla por un
solo rumbo; pero que esto no sería fácil, y tendrían que dispersarse
tomando cada cual por los caminos a que le llevasen sus diferentes
querencias. «A todos mis hijos---prosiguió,---ha puesto el Señor mucha
sal en la mollera, tanto que del rebosamiento de tanta sal han venido
sus desafueros y las maldades de algunos. Y con la sal abundante les
puso el Señor inclinaciones fuertes, a cada cual para lo suyo. A
Gonzalo, que está presente, le tira la milicia, pero la milicia libre,
que no hallará mientras no salten otras guerras como las pasadas; a
Diego le tira la mar, de quien se enamoró en cuanto la vido en la salida
del Ebro por los Alfaques, y tanto es su amor de las aguas, que en ellas
se metería dentro de un zapato para ver toda tierra descubierta o por
descubrir; a Gil le llama el mando, la guapeza, y no es capitán de
bandoleros, porque eso no trae cuenta con tanta Guardia cívica que
tenemos ahora; a Leoncio le tira la cerrajería fina, o sea el amañar
armas de fuego, y llaves tan sutiles que con ellas no pueda cerrar y
abrir quien no tenga el secreto; y si de Rodriguillo no diré, por razón
de su corta edad, que está ya bien clara la inclinación, pienso que le
tira la música, o el arte de sacar coplas y de componer lo prosaico con
buena concordancia. Si unos irán con gusto a Madrid, otros quieren más
campo, más aire y espacios grandes. De mí digo que me tira Madrid,
porque habiendo padecido trabajos y agonías debajo del trillo, que con
esto comparo al Gobierno y Fisco que nos aplastan, antes que ser la
espiga que está debajo, quiero ponerme donde va el trillador, y ayudarle
a llevar la máquina, si me dejan. Créanme los señores excelentísimos:
mejor que ser la liebre guisada, es ser el cocinero que la guisa, ya que
no sea uno el rico que se la come. Feo y mal mirado es el oficio de
verdugo; pero vale más ser ejecutor de la justicia que ajusticiado.
Labrador fuí, y los mejores años de mi vida me los entretuvo y gastó el
amor de la tierra; mas desengañado ya y harto de fatigas sin fruto,
digo: «¡Adiós, tierra, con doscientos y el portero!» A mí me han molido,
me han zarandeado, y me han quitado una y mil veces lo que gané con mi
sudor. Déjenme ahora maldecir y renegar del diezmo, de la primicia, del
voto de Santiago, del apremio, del montonero, del embargo, de la mano
muerta, de la mano viva. ¡Arre allá por cepas! Más vale saber que haber.
Váyanse al demonio el alcalde, el jefe político, el regidor decano, el
síndico personero, el agente de apremios, el recaudador, el fiel de
fechos, el escribano, el alguacil, el del fielato, el pontonero, y
cuantos tienen autoridad del Ministro para abajo. Pues ahora quiero yo
vengarme, o como se dice, ponerme encima, y ya que mis espaldas saben a
lo que saben los golpes, sepa también mi mano a qué sabe tener el palo,
y con el palo licencia para pegar de firme.

---Comprendido, Sr.~Ansúrez---dijo mi madre risueña:---lo que usted
quiere ahora es un destinito. Vaya, vaya: es tonto y pide para las
ánimas.

---Destino tendrá---afirmó María Ignacia, que encontraba graciosas las
cuitas y las ambiciones del buen Ansúrez.---Y si, como dicen, es usted
leído y escribido, bien podrá entrar en una oficina.

---Más que oficinante, me gustaría ser guarda de Sitios Reales,
administrador de un pósito\ldots{} verbigracia, o almacenero de los
tabacos de Su Majestad.

---Vaya, vaya---dijo mi madre;---aquí viene bien lo de \emph{aún no
ensillades y ya cabalgades}. Pepe, ya puedes recomendarle\ldots»

Preguntado si tenía relaciones en la Corte, o si en su larga vida había
hecho conocimiento con alguna persona de viso, que ahora le pudiera
favorecer, contestó que su estrechez y desgracia no le han traído más
que conocimiento de gente miserable, pues por algo se dice: \emph{en
cama angosta y en luengo camino no hallarás amigos}.

En este punto de la sabrosa conversación, precipitose mi mujer con esta
pregunta: «Ya sabemos que a uno de sus hijos le tira el mar, a este la
milicia, al otro la música, a usted le tira Madrid; ¿y a su hija Lucila,
qué le tira?

---Mi hija tira al monte, quiero decir, a las grandezas---replicó el
viejo,---como si de padre y madre coronados hubiera nacido esa criatura;
y aunque Sus Mercedes la ven tan extremada en el trajín pobre,
vistiéndose por la moda de las imágenes, es que gusta de pintar la
grandeza con la rematada pobreza, por aquello de parezco nada para serlo
todo\ldots{} Tiene buen natural, eso sí, y a compasiva no le gana ni
Santa Leocadia\ldots{} Pero yo quisiera que si vamos a Madrid,
encontráramos para Lucila un buen recogimiento al lado de señoras
maduras y sentadas que la enseñaran la gobernación de casa humilde, y le
quitaran de la cabeza la idea de que vuelven al mundo las hembras guapas
de la idolatría\ldots{} no sé explicarme\ldots{}

---Lo entendemos muy bien---observó mi madre.---Esa niña de usted, según
me dicen, es como si viniera de gentiles, o nos quisiera traer la moda
del tiempo en que eran vivas las estatuas\ldots{} ¡Buena pécora será la
muchacha si no la curan de esa manía!\ldots{} Pero mis hijos le darán a
usted cartas de recomendación para que en Madrid halle donde colocarla
honestamente.»

Esta idea sugirió a mi mujer el propósito de formular las
recomendaciones inmediatamente, ansiosa de mirar por la errante familia.
Sus nervios disparados no admitían espera, y que quieras que no, tiró de
mí y arriba me llevó para que escribiera las cartas. «¿Pero a quién he
de escribir, mujer?\ldots{}

---A tu familia, a tus amigos, a Eufrasia, a tu hermana Catalina\ldots{}

---Creo---le respondí,---que recomendándola a mi hermana no será preciso
molestar a nadie. Lo que no haga Catalina no lo hará ni el propio
Narváez.»

Obediente al caprichoso estímulo de María Ignacia, forma de un recelo
que locamente la inquietaba, cogí la pluma y empecé la carta. Mi mujer
miraba por encima de mi hombro lo que yo escribía; y viéndome indeciso
en los términos de recomendación, me apuntó resoluciones y fines
concretos: «Diles claramente, y encárgales con gran interés, que la
metan monja.

---Pero, mujer, falta que tenga vocación.

---La vocación se hace\ldots{} ¡Qué tonto eres! Monja, monja, que no hay
como la disciplina del claustro para domar a estas que dan en la flor de
vestirse por los figurines del Paraíso Terrenal. Así evitará su
perdición y la de muchos hombres. Ponlo, ponlo bien claro\ldots{} Que
nos interesamos por esa joven; que deseamos su ingreso en un convento de
regla muy estrecha\ldots{}

---¡Pero si no tiene dote, y ya sabes que sin dote es difícil\ldots!

---Yo la dotaré. Ponlo clarito: eso hace mucha fuerza.»

\hypertarget{viii}{%
\chapter{VIII}\label{viii}}

Pues Señor, escribí la carta conforme al deseo de mi mujer, y cuando
bajamos y la dimos al interesado, Taracena, que a la sazón llegaba,
vaticinó al viejo Ansúrez y a su hijo dichas y grandes medros por
nuestra protección. «No han tenido poca suerte en caer acá---les
dijo,---y en la coyuntura de hallar en Atienza a los señores Marqueses.
Digan que les ha venido Dios a ver, porque de estas gangas caen pocas.

---Ya lo sabemos, y yo doy gracias a Dios por esta bienandanza---replicó
Ansúrez;---que después de tantas perrerías de la suerte, alguna vez
habíamos de pelechar. Y la dicha será completa si Su Excelencia pone en
la carta, a más de lo tocante a la hija, alguna buena exhortación para
los señores que podrían colocarme.

---Pepe, hijo mío---dijo mi madre,---puesto ya en eso del recomendar,
escríbele a Sartorius, o al propio D. Ramón Narváez.»

A esto observó María Ignacia que si mi hermana tomaba bajo su santa
protección al buen Ansúrez, no necesitaba este de nadie, pues los mismos
San Luis y Narváez con todo su poder de relumbrón, quedan hoy muy por
bajo de Sor Catalina y de las otras monjas sus compañeras, las cuales, a
la calladita, llevan su influjo a todos los ramos, y a la mismísima
Superintendencia de Palacio y Sitios Reales.

Oyó esto con viva satisfacción el padre de la tribu, y D. Juan Taracena,
dándole una palmadita en la rodilla, le dijo: «\emph{Alforjero} te
llaman, no porque las haces, sino porque las llevas; \emph{bragado},
porque no hay quien te tosa; \emph{hidalgo}, porque lo pareces. Tú te
abrirás camino, y como las monjas interesen por ti a Narváez, cuéntate
colocado.» Y volviéndose a nosotros, agregó: «¡Quién sabe si el Espadón,
con ese ojo certero que tiene para descubrir aptitudes, encontrará en
este viejo ladino y fuerte el auxiliar de sus grandes ideas!

---Señor clérigo, no se burle de estos pobres---murmuró Ansúrez con
humildad que no debía de ser muy sincera.

---¿Qué idea tiene usted de Narváez?---le pregunté yo.---¿Cree que si se
presenta al General con carta o recadito de mi hermana, pidiéndole un
destino, le recibirá bien, o te dará un sofión, que bien podría ser un
par de palos?

---Señor---replicó Jerónimo prontamente,---creo que me dará los
palos\ldots{} y después de los palos el destino que se le pida.

---Vamos, que no le falta penetración. ¿Ha visto a Narváez alguna vez?

---No, señor; pero por lo que oí contar de ese sujeto, tocante a sus
guerras y a la política, he venido a conocer que el hombre es fuerte y
bueno, que pega y favorece.

---¡Sopla, sopla, que vivo te lo doy!---dijo el Cura sacudiéndose los
dedos como quien se ha quemado.---Pues no afila poco el tío. Basta de
examen y démosle la borla de doctor \emph{in utroque}. Váyase pronto a
Madrid, \emph{alforjero}, que si no le falta alguna cualidad de las que
son precisas para vivir entre gentes, pronto encontrará su
acomodo\ldots{} Y como los hijos salgan al papá, no es floja la plaga
que va a caer sobre la Administración Pública.

---Y si conforme llega cansado y viejo---observó mi madre,---llegara en
la flor de la edad, lo que es este se metía en el bolsillo a todo el
Madrid pretendiente.

---No me hagan mofa, señora y caballeros. No es sino que por luengos
años estudié en el mejor libro del mundo, que es la tierra. Sé cómo
viene el fruto, y cómo se pierde; sé que una cosa es sembrarlo y otra
comerlo, y de dónde salen las manos que cogen lo que no sembraron. Pues
con estas lecciones y experiencias, y con la continua desgracia, que a
los más torpes nos hace abrir el ojo, acaba uno por saber más que
Merlín.»

Como le echase mi madre un sermoncillo cariñoso, haciéndole ver que no
hallaría la fortuna fuera de los caminos de la virtud, de la honradez y
del santo temor de Dios, el patriarca celtíbero se sacudió las moscas
con esta donosa frase: «Yo quiero ser honrado; siempre lo he querido;
pero ¿quién es el guapo\ldots{} a ver, que salga ese guapo\ldots{} que
ajusta y acorda el querer con el poder? Y yo digo también a los señores:
el que de Vuestras Excelencias, grande o chico, sepa y pueda vivir entre
tantísimas leyes divinas y humanas sin poner el dedo en la trampa de
alguna de ellas para escaparse, que me tire todas las piedras que
encuentre encima de la haz de la tierra.

---Yo se las tiraría---dijo mi madre con profunda convicción,---si la
doctrina cristiana que profeso, sin trampa, entiéndalo, no me prohibiera
descalabrar a mis semejantes.»

Reímos todos esta sincera y valiente salida; riose también Ansúrez, y
despidiéndose muy agradecido del bien que le habíamos hecho (añadidos a
la carta y hortalizas algunos dineros), salió de casa con su hijo. Según
mi criado Francisco, que acompañó a la tribu hasta la salida del pueblo,
partieron todos antes de mediodía\ldots{} Acabando nosotros de comer,
vino el Alcalde con el triste cuento de que el bonísimo Miedes iba de
mal en peor, por lo cual el médico había mandado que le sacramentaran.
Si sobrevenía la muerte, cosa muy de temer en su edad y con aquel
endiablado achaque cerebral, cogiérale prevenido y bien aligerado para
el final viaje. Por deseo de ver y consolar al pobre señor, y suponiendo
además que carecería de lo más necesario, resolvimos visitarle mi mujer
y yo. Mi madre, que es la misma previsión y no pierde ripio para sus
actos de caridad, nos advirtió que despacharía por delante, y así lo
hizo, un buen codillo de jamón y obra de dos libras de carne, porque el
puchero que tendría puesto \emph{la Ranera} habría de dar caldos de los
que sirven para bautismo de cristianos.

Vivía el buen Miedes en el barrio más pobre, más excéntrico y solitario
de Atienza, en antigua y fea casa del primer recinto, apoyada en el muro
de base celtíbera, romana o agarena. La distancia no larga que la
separaba de nuestra vivienda, nos pareció enorme por la desigualdad de
rasantes y el empedrado inicuo, reproducción exacta de los pavimentos
del Purgatorio. En la soledad lúgubre de aquella parte de la villa, las
casas son como tumbas abiertas, deshabitadas de muertos, y que se
arriman unas a otras para no desplomarse. Preguntando a unos niños que
pasaban comiéndose el pan de la merienda, dimos con la morada del sabio.
Un zaguán largo y estrecho, de empedrado piso con hoyos, conducía de la
puerta a la cocina, dando ingreso por izquierda y derecha a diferentes
estancias, la cuadra con pesebre vacío, el camarín de \emph{la Ranera},
y algo más que no vimos: una escalera de palo sin pintar, de color
sienoso, como teas que piden lumbre, y festoneada de telarañas, conducía
desde el zaguán al salón alto, que era en una pieza biblioteca y alcoba,
separadas hasta media pared por tabique de mal juntas tablas que nunca
vieron pintura, y sí papeles pegados, suciedades de moscas y otros
bichos. Imposible describir el desorden de aquel local, émulo del Caos
la víspera de la Creación. Los libros debían de ser semovientes, y en el
silencio de la noche se pondrían todos en marcha, subiéndose y bajándose
de estantes a mesas y del techo al suelo, como ratones sabios o
cucarachas eruditas que salieran a pastar polvo. Los grandes estaban
sobre los chicos, y algunos abiertos yacían hojas abajo sobre el suelo,
mientras otros, hojas arriba, aleteaban subidos a increíbles alturas. No
podíamos explicarnos cómo andaba el tintero con sus plumas de ave,
acompañado de una pantufla, por los huecos de un estante vacío, mientras
se arrastraba por el suelo el velón, entre dos tomos de las
\emph{Antigüedades de Berganza} con las hojas manchadas de aceite.

El otro departamento, dormitorio del sabio, era como trastienda o
sacristía de la biblioteca, llena también de libros, que asomaban en
montones desiguales por debajo de la cama, o servían apilados para
colocar objetos pertinentes al servicio de alcoba. Allí vimos, entre las
polvorientas masas de papel, un cuadro de pintada talla que me pareció
pieza de mérito, un monetario, algunos trozos de cemento romano, y
pedazos de mármol con inscripciones y garabatos ininteligibles. Y allí
vimos también, como gusano dentro de su capullo, al gran D. Ventura,
tendido en el lecho debajo de una colcha que en su juventud fue blanca
rameada de rojo, la cabeza casi invisible de los vendajes que la
oprimían, los brazos fuera, vestidos de amarillenta lana, todo él con
aspecto tan fúnebre, que al echarle la vista creímos que estaba ya
muerto. Tras de nosotros entró \emph{la Ranera}, señora de edad muy
alta, con pañuelo negro liado a la cabeza, saya y jubón de estameña, los
pies en abarcas, la cara como pergamino, los ojos, pitañosos de su
natural, en aquella ocasión ribeteados del grandísimo duelo por la
inminente defunción de su amo; y después de mirar al demacrado D.
Ventura, que no remuzgaba ni se daba cuenta de nuestra visita, nos dijo
sin recatarse de bajar la voz, como es usual etiqueta ante moribundos:
«Muy malo está el pobrecito, y el rostril lo tiene ya como un terrón de
tierra. Dende que cayó, no se le han vuelto a encajar en su sitio los
sesos, que con los porretazos de la piedra se le desengonzaron, y ni
come ni duerme, ni habla cosa denguna con juicio.

---¿Pero qué dice el Médico, señora \emph{Ranera}; qué ha recetado? ¿Y
usted qué dispone?

---¿Qué ha de recetar D. Pascual más que traerle la Majestad? Y tocante
a comida, ¿para qué enciendo lumbre, si ya no le hace falta más que el
pan del cielo, y este lo trae el Cura? Pues yo, que todo lo presupongo,
vengo ahora de comprarle la mortaja, y no encontré más que una en casa
del \emph{Pocho}; pero tan corta, que no le llegará ni tan siquiera al
tobillo, según es mi señor de larguirucho\ldots{} A pegarle voy un
pedazo de estameña que tengo, del mesmo color franciscano, de una saya
de mi difunta güela, y con ello quedará mi cadáver bien adecentado de
pie y pierna.»

En esto, y antes que pudiéramos expresar a la maldita vieja el horror
que nos producía, despertó D. Ventura, o más bien se recobró un tanto de
la somnolencia febril, y revolviendo en torno sus miradas, sin mover la
cabeza, dijo con apagada voz de lo profundo: «Llevo lo menos seis días
durmiendo, y ahora con tanto dormir no veo claro, ni me ayuda el
discurso. Dime, \emph{Ranera}: ¿quién son estas venerables personas que
han entrado y me están mirando?

---Válgame Dios; ¿pero no conoce a los señores Marqueses?\ldots{} Y
ahora entra el señor Cura, que no podía venir en mejor coyuntura. Vea,
señor, que no está ya para más visitas que la de D. Juan, ni para
requilorios de comistraje y golosinas. Déjese de vanidades, y piense en
lo que más le importa, que es la salvación. Apañado está si despide al
Cura con cuatro bufidos, como esta mañana\ldots»

Sin atender a lo que \emph{la Ranera} decía, más bien como si no lo
escuchara, volviose Miedes hacia el párroco, moviendo todo el cuerpo
dentro de las sábanas como si intentara levantarse, y animándose de
mirada y gesto, soltó la voz a estas peregrinas razones: «Curángano, ya
te dije que no tenías para qué venir acá. Soy celtíbero: ¿no sabes que
soy celtíbero, de la familia de los \emph{Pelendones celtiberorum}, que
dijo el amigo Plinio, o más bien de los \emph{Turdimogos}, que vivían de
la parte del valle de Valdivielso?\ldots{} ¿Y no es sabido que por el
lado materno vengo del propio Cáucaso\ldots{} y que mi abuela era de la
familia de los \emph{Istolacios}?\ldots{} Soy Miedes, que es lo mismo
que decir \emph{Cuerno}\ldots{} pero este cuerno no es otro que el
símbolo de la inmortalidad\ldots{} ¿Qué vienes tú a buscar aquí,
curángano de Atienza, que es como decir \emph{Tutia}? Yo nací en
\emph{Numancia}: digo, en \emph{Comphloenta}\ldots{} tampoco; digo, en
Quintanilla de Tres Barrios, que es un pago de San Esteban de
Gormaz\ldots{} Yo no soy de tu Iglesia, pues soy celtíbero\ldots{}
Vete\ldots{} Que te vayas\ldots{} Señores Marqueses, llévenselo, si no
quieren que le tire a la cabeza esta sagrada pantufla\ldots»

Tratamos de sosegarle con cariñosas expresiones, y de traer a vías de
razón su descarriado entendimiento: todo inútil. Con el Cura y con
\emph{la Ranera} no quería cuentas. Yo, a fuerza de perífrasis, logré de
él alguna docilidad de pensamiento haciéndole comprender que no perdía
nada con prepararse, sin que ello significara peligro de muerte, y
cogiéndome la mano con la suya pegajosa y fría, me dijo: «D. José mío:
porque usted no se enfade, me confesaré; pero que me traigan un druida,
porque si no me traen un druida, ya ve usted que no puede ser\ldots{} Es
mucho cuento. Yo digo que cada uno vive y muere al son de sus
creencias\ldots{} Yo adoro al Dios desconocido, y le tributo mis
homenajes en el plenilunio\ldots{} Tú, Juanillo Taracena, a quien he
conocido mocoso y descalzo, con el calzón agujereado por las rodillas,
trayendo leña y carbón del monte, tú no eres druida, tú no has cogido el
muérdago\ldots{} ¿Qué tengo yo que ver contigo ni con tu negra
hopalanda?»

Opinó Taracena que no debíamos insistir. «Es un santo---nos dijo,---y si
Dios le ha privado de juicio en esta hora última, será porque le tiene
ya por suyo. Dejémosle, y si del descanso sale un ratito lúcido, le
traeré fácilmente a la razón.» Para ver si llevándole el genio se le
despejaba la cabeza, le aseguró que él, sacerdote cristiano, era también
druida, y que practicaba el rito celta en los plenilunios o fiestas de
guardar. Después le habló de sus amigos los vagabundos Ansúrez, lo que
fue gran despropósito, porque con este recuerdo y encadenamiento de
ideas nuevas con otras rancias y arraigadas en el meollo del sabio, se
disparó más y acabó de quitar el freno a sus furibundos disparates. «Tú,
pastor Taracena---dijo con gran desvarío de miradas, trabamiento de
lengua y agitación de manos,---me declaras la guerra, porque me has
visto perdidamente enamorado de la hermosa \emph{Illipulicia}, hija del
rey \emph{Zuria} o \emph{Zuri}, que a mi parecer es familia que ha
venido de la \emph{Troade}, vulgarmente Troya, destruida por los
griegos\ldots{} \emph{Teucro} engendró a \emph{Tros}, y \emph{Tros}
engendró a \emph{Ilo}, fundador de aquel pueblo, al que dio el nombre de
\emph{Ilium}. De allí procede esta preciosa niña, quien de sus abuelos
tomó el dulce nombre de \emph{Illipulicia}, que es como decir
\emph{Estrella del Reino}. A esa divina estrella insultaste tú,
clerizonte, diciéndonos que no se había lavado desde que a nado pasó el
río Scamandro para venir aquí. Tú sí que no te has lavado, sucio, desde
que te echaron el agua del Bautismo\ldots{} Pues el bellaco de nuestro
alcalde te dijo: «¡Juan, vaya una hembra! ¡Y es de la casta fina de amas
de cura!» Tú te echaste a reír como un sátiro, y yo que oí estas
infamias, resolví amar a \emph{Illipulicia} y hacerla dueña de mi
albedrío para defenderla contra vuestras artes seductoras\ldots{}
Atreveos, disolutos; acercaos, viciosos. Rabiad, rabiad, que vuestra no
ha de ser, aunque vengáis con todas las redes y anzuelos
infernales\ldots{} Los cuernos del dios Ibero la protegen\ldots{} y el
cuerno sacro soy yo, yo, Buenaventura Miedes. \emph{Illipulicia} es la
virginal sacerdotisa, la diosa casta, en quien está representada el alma
ibera, el alma española\ldots{} Ella es mi dama, o como quien dice, mi
inspiración, o llámese musa, y siendo ella el alma hispana y yo el
historiador, engendraremos la verdadera Historia, que aún no ha salido a
luz. Y como la Historia es la figura y trazas del pueblo, ved a
\emph{Illipulicia} en la forma de pueblo más gallarda\ldots{} Sabed que
todo pueblo es descalzo, y que la Historia es más bella cuanto más
desnuda, y cuanto menos etiqueta de ropas ponemos sobre su
cuerpo\ldots{} Con que, vedme aquí enamorado de ella, y rejuvenecido con
este amor. Rabiad, vejetes caducos, de verme tornado a la mocedad
florida\ldots{} Soy un joven lozano y fresco\ldots»

Por señas me indicó Ignacia que no podía resistir más tiempo ni aquella
atmósfera nauseabunda, ni el espectáculo de tanta miseria unida a tan
lastimosos extravíos de la razón. Salimos a respirar aire puro, y
paseamos por las calles visitando y admirando una vez más las
incomparables iglesias románicas de la villa, reliquias espléndidas y
tristes que nos hablan poético lenguaje. Ya conocíamos las bellezas de
Santa María y la Trinidad: empleamos la tarde en explorar los mutilados
restos de San Bartolomé y de San Gil, no sin que amargara nuestros goces
el melancólico recuerdo de D. Ventura, porque de él habíamos aprendido a
entender y saborear el divino arte de aquellas piedras.

\hypertarget{ix}{%
\chapter{IX}\label{ix}}

Al pasar de nuevo por la casa de Miedes, vimos en la puerta a la tía
\emph{Ranera}, dentro de un círculo formado por otras vejanconas y unos
arrapiezos de la vecindad. Con diligente afán cosía en la mortaja el
pedazo de estameña que faltaba. «Está igual o pior---nos dijo,---y tan
disparado del caletre, que discurre lo mesmo que un molino de viento. El
médico ha prenosticado que si le repite el arrebato de pintarla de
galán, poniéndose negro del golpe de sangre en la cabeza, en él se
quedará como si le retorcieran el pescuezo\ldots{} Ya ven los señores
que me estoy dando priesa, y para tenerlo todo aparejado y que no digan,
también he traído las velas\ldots{} ¡Pobre señor! era el primer
cristiano de la cristiandad, más bueno que San José bendito\ldots{}
¡Vaya por lo que le ha dado ahora, al cabo de los años!\ldots{} ¡Por
enamorarse de la que llama la \emph{princesa Filipolida}, que según
dicen es una puerca, y viste a la similitú de las gitanas! Dios le lleve
a su gloria, que bien se la merece, y perdónele aquesta ventolera, por
no ser pecado, sino locura. No: no peca un hombre para quien fue siempre
más amoroso el pergamino de los libros que el pellejo fino de mujeres, y
a la suya propia, la Bibiana Conejo, que de Dios goza, no le decía jamás
cosa denguna, aunque era tan limpia que se lavaba las manos con jabón de
olor\ldots{} así le trascendían a claveles\ldots{} ¡Y el que despreció a
la que tan bien golía, como que se mudaba los bajos cada semana, y de
camisa siempre que bajaba a la villa, que entonces vivían en Bochones,
ahora se trastorna por una que anda como la Madalena, hermana de unos
tales vagamundos\ldots{} que según dicen, no se puede entrar a ellos,
porque el fetor de cuadra da en la nariz!\ldots{} ¡Lo que una vede,
Señor! Y era tan simple mi amo y tan arrebatado de su caridad, que toda
la despensa de casa, donde siempre hubo de cuanto Dios crió,
verbigracia, cebollas, pan y vinagre, iba a parar al Castillo, y aquí
están estas mis encías con telarañas para dar testimonio de las hambres
que pasé\ldots{} Pero, al fin, esos diablos de los infiernos se han ido
ya, y mi Don Ventura subirá esta noche al Cielo, donde le darán su
puesto entre la sinfinidá de arcángeles. Váyanse ya tranquilos los
señores a su casa, y díganle a Doña Librada que mi amo es concluido.
Ahora quedaba porfiando que ha de volverse mozo, y entre el albéitar y
D. Juan el cura no lo podían asujetar\ldots{} Luego entrará en la
agonía, y por mucho que tire no ha de pasar de las diez de la noche.
Vaya por él y su descanso este Padrenuestro\ldots{} «Padre
nuestro\ldots» Rezaron todos, viejas y chiquillos, y mi mujer y yo nos
retiramos angustiados ante tan aterrador ejemplo de la miseria humana. A
la mañana siguiente, supimos que el buen Miedes había expirado al filo
de media noche. Fuimos a misa todos los de casa, y mi madre dispuso
costearle el entierro y funeral.

Difícil me será explicar la pena que sentí en los días siguientes, no sé
qué vacío en mi alma, como si la desaparición del sabio me afectara más
de lo que lógicamente correspondía, un desconsuelo de lo pasado
fugitivo, un temor de lo futuro incógnito. Mi mujer, restablecida en su
equilibrio nervioso, ocupábase con mi madre en formar lista y
presupuesto de las limosnas que habíamos de repartir en el pueblo y sus
arrabales, como tributo reclamado a nuestra sobrante riqueza por la
necesitada humanidad, con lo que satisfacían nuestros corazones un
generoso anhelo y se cumplía la ley de nivelación económica, o al menos
poníamos de nuestra parte la intención de cumplirla. Intacto estaba el
repuesto de onzas que habíamos traído de Madrid, y ante tales tesoros
lanzábase mi madre con grande espíritu a los más atrevidos cálculos de
caridad, reflejando en su rostro todos los esplendores de la
Bienaventuranza. «Gracias doy a Dios---nos dijo una mañana la santa
señora, viendo a mi mujer muy afanada en escribir los listines de
limosnas,---por este favor inmenso de veros socorrer delante de mí tanta
miseria, y os juro que no gozaría más si lo hiciera yo misma con mi
hacienda propia. No hay vida más ejemplar que la del que cultiva los
campos, porque toda ella es sacrificio y paciencia, de que no tenéis
idea los ricos que vivís y triunfáis en las ciudades. Mala es hoy la
condición del labrador rico, agobiado de contribuciones y gabelas, y
expuesto a que se lo coman, al menor descuido, los viles usureros; pero
la del labrador pobre, que apenas saca para el sostén de su familia y
animales, es mucho peor, como que vive de milagro; y nada quiero deciros
de los que no poseyendo más que sus cuerpos se atienen a un jornal,
cuando lo hay, que estos son como esclavos propiamente.»

La idea que expresó María Ignacia de socorrer a los que habían perdido
sus cosechas por el pedrisco, entusiasmó a mi madre, hasta el punto de
saltársele las lágrimas. «Bendito sea tu corazón piadoso, hija mía, y el
tino que tienes para todo---le dijo.---No podías pensar cosa más
acertada\ldots{} Poned, pues, en la lista a los infelices que en aquella
calamidad perdieron su esquilmo; pero no debéis olvidar a otros tan
desventurados como aquellos, o más, si me apuran; que si malo fue el
pedrisco que presenciasteis y que quitó la vida a nuestro pobre D.
Ventura, peor fue la horrible seca de este año, la cual asoló tanto, que
muchos no pueden llevar a las eras más que un puñado de espigas. Yo que
les conozco a todos, os diré cómo habéis de hacer la distribución, para
que no queden desigualados en el beneficio y sea el socorro conforme a
necesidad. A los que perdieron sus patatales y el sembrado de judías y
menudencias, les asignaréis doblón de a cuatro, o doblón de a ocho,
según tengan más o menos familia de hijos y animales\ldots{} De todo
este contingente puedo yo daros razón\ldots{} Y a los que no trillan,
por causa de la sequía, ni un tercio de su cosecha, les señalaréis a
onza por barba. ¡Ay, hijos míos, no conocéis del campo más que las galas
con que se viste por estos meses! Quedaos por acá y veréis la cara que
pone cuando se desnuda de todas las alegrías verdes y se recoge para
preparar las fatigas del año próximo. Ya habéis visto que el invierno
asoma el hocico por los altos de Sierra Pela. Los hogares ya quieren
lumbre, y los cuerpos echan mano de cualquier trapajo para abrigarse.
Pues imaginad qué días esperan a esa pobre gente que no tiene trigo para
pan, ni patatas, ni dinero con que proveerse de ello. Dios que no
abandona a sus criaturas, si mandó sequía y granizo para probar la
conformidad de estos pobres esclavos del terruño, os mandó luego a
vosotros, hijos míos, para traer el remedio, y seréis el uno el arco
iris que aparece después del Diluvio, la otra la paloma que viene con el
ramo de oliva en el piquito.»

Paloma y arco iris nos pusimos a formar la nueva estadística con los
datos que nos daba mi madre. Otra tarde nos dijo: «También en el pueblo
tenéis dónde emplear lo mucho que os queda, pues los telares están
parados, y los abarqueros y curtidores no saben de dónde sacar una
hogaza. La miseria proviene de estas modas malditas que traen ahora
trastornados a los pueblos, y de las muchas telas que aquí llegan,
falsas como Judas, tejidas como telarañas, pero lucidas a la vista, y
baratas, eso sí, con una baratura que desvanece a los tontos y aburre a
nuestros tejedores. ¡Vaya unos lienzos indecentes que nos traen, y unas
estameñas y unos tartanes que mirados al trasluz, parecen cedazos! Pues
los montereros también andan de capa caída. Ahora salen estos brutos con
la tecla de que las monteras de pellejo, para diario, no son elegantes,
y algunos se cubren las chollas con esos buñuelos de paño que vienen de
las Provincias\ldots{} Y habéis de ver a las chicas vistiendo ya por la
moda de Madrid, con esas indianas de a dos reales la vara, y esos
pañuelos de listas que hasta parece que no visten, sino que
desnudan\ldots»

Como allí nos sobraba el dinero, y no temíamos ulteriores escaseces,
pues mi próvido suegro ya nos anunciaba nueva remesa, abrimos
gallardamente la mano, y fuimos como benéfico rocío que derramó algún
consuelo sobre las entristecidas almas. Mas era tal el ardor que ponía
mi buena madre en aquellas empresas de caridad, que mientras más
dábamos, mayores larguezas nos pedía, como si el ejercicio del bien
llevase a su noble alma del entusiasmo a la embriaguez. «Ya podía tu
padre---dijo a María Ignacia,---mandaros un par de mulas cargadas de
onzas para que os decidáis a edificar aquí el convento de monjitas de
que me habló Catalina en sus cartas. Tan apagada está la cristiandad en
este pueblo, que nos hace falta un instituto religioso que avive el
fuego de la fe. ¡Ay, qué bien nos vendría un convento para la enseñanza
de niñas, donde estuvieran desde los cinco años hasta que saliesen para
casarse, aprendiendo todas las labores, y bien guardaditas del melindre
de novios, cartitas, bailoteo y demás perdición! Andan las muchachas
aquí tan desenvueltas, que esto parece un rincón de Madrid, y las de
buen palmito no piensan más que en retratarse cuando recala por Atienza
alguno de esos que traen maquinilla del garrotipo, con las que sacan
unos retratos que se miran a contraluz para ver lo blanco negro y lo
negro blanco. Y mocosas hay que hasta llegan a decir que les gusta el
café, y lo toman si se lo dan. Otras\ldots{} tú las conoces\ldots{} han
aprendido a ponerse el peinado de tirabuzones, que es una indecencia,
con aquellos mechones colgando; y algunas\ldots{} pongo por caso, las de
Cuadra y las de Aparicio\ldots{} mandan traer de Madrid corsés como el
tuyo, de los que sacan el pecho\ldots{} cosa impropia de solteras. Este
pueblo no es conocido. Me acuerdo de la villa de mi juventud, y me
parece que han pasado siglos, o que la humanidad se nos ha vuelto loca.»

Con estas cosas y la satisfacción de hacer el bien a tanto desvalido,
íbamos pasando los días de Atienza, que ya comenzaban a ser un poquito
enojosos. Expirante Septiembre, se descolgaba de la sierra, por las
tardes, un vientecillo enteramente soriano; crecían las noches;
descargaban a menudo copiosas lluvias que nos privaban del paseo, y
pronto nos haría la nieve sus primeras visitas. Preparados estaban ya
los hogares, limpias las chimeneas y apilada la leña que pronto
habríamos de quemar si no buscábamos mejor otoño en tierra templada. La
casa patrimonial, donde tan alegres habían transcurrido los días y las
semanas, ya se llenaba de una vaga tristeza, que hacía más obscuros sus
anchos aposentos, más bajas las techumbres, que casi se ponían a la
altura de nuestras cabezas, más negro el maderamen de las pesadas
puertas. Por los resquicios de las tuertas ventanas, avaras de luz, se
colaba con insolencia el aire frío; a media tarde teníamos que subir a
tientas para no tropezar en la escalera; los cortinajes nuevos con que
mi madre había decorado nuestro aposento, se trocaban en fúnebres
colgaduras, y las imágenes de Vírgenes y Santos nos ponían el ceño
adusto, o se asombraban de vernos allí.

Hube de fijarme entonces en un accidente de mi casa que en todo el
verano no mereció mi atención, y era el ruido, o más bien concierto de
ruidos que hacían las diferentes puertas del vetusto edificio al ser
abiertas o cerradas. Cada noche observaba yo un nuevo rumor o musical
concepto, ya como lastimero quejido, ya como frase de angustia o
sorpresa, y aplicando el oído y la imaginación, concluía por dar un
significado verbal a sones tan extraños. Por entretenernos en algo en
las lentas noches, comuniqué mis observaciones a Ignacia, y apoderada
esta de lo que tanto era artificio de la mente como realidad sonante,
\emph{oyó} más que yo, y compuso todo un poema con los ruidos de las
viejísimas tablas de mi casa solariega. «La puerta del comedor, siempre
que entra alguien, dice: \emph{«¡ay, ay, ay!, ¿cuándo os cansaréis de
abrirme?}\ldots» y la de la despensa: \emph{«Dejadme morir
cerrada\ldots»} Pues fíjate en los peldaños de la escalera cuando sube
Úrsula, que es de libras\ldots{} Dicen: \emph{«Muero porque no muero.»}
Y cuando baja Prisca, que corre como una rata, hablan en lenguaje
familiar. Yo lo oigo así: \emph{«Pues aquí venimos los frailes gilitos
vendiendo cabriiitos\ldots»} Pon atención y oirás lo mismo que oigo
yo\ldots{}

«Pepe, Pepe---me dijo Ignacia una noche cuando desperté del primer
sueño,---fíjate en ese ventanón que han dejado abierto en el desván. El
viento lo mueve, y al abrirse canta el primer verso de la jota\ldots{}
atiende y oirás: \emph{«Hay en el mundo una España\ldots»} luego se
cierra con un golpe, pum, al cual sigue un ruido muy suave, algo así
como el de las chupadas de un niño cuando coge la teta.» Puestos a oír,
oíamos verdaderas maravillas. La puerta del comedor hablaba en griego y
en latín, y decía cosas de la misa para echarse después a reír con
alguna frase desgarrada, más propia de boca de manola que de una
venerable puerta de casa ilustre; la que comunica el comedor con la
pieza donde están los armarios de ropa decía: \emph{«Madre, unos ojuelos
vi,»} y los armarios remedaban rezos de monjas, ronquidos de durmientes,
pregones como el \emph{«¡De Jarama, vivos!»} que tanto habíamos oído en
Madrid\ldots{}

Llegamos a componer el completo inventario de estos domésticos ruidos,
con música y letra; y como alguna noche nos molestase tanta música, nos
atrevimos a decir a mi madre que mandara untar de aceite los mohosos
goznes para que callasen, o fueran más silenciosas las parlantes y
cantantes puertas. Pero ella, sonriendo con la dulce severidad que
empleaba siempre que se veía en el caso de negarse a darnos gusto, nos
dijo: «Por Dios, hijos míos, no me pidáis que suprima los ruiditos de mi
casa, que si ella no me cantara con el son de sus puertas y el
estribillo de sus gonces, me parecería que pasaba de casa viva a casa
muerta. Con esos ruidos melancólicos, que me cuentan cosas del presente
y del pasado, me crié, y con ellos quisiera morirme. En ellos oigo la
voz de mis padres y de mis hermanos, la de mi tío Anselmo, corregidor
que fue de Guadalajara. Amigo íntimo del Empecinado y de D. Vicente
Sardina, nos refería las palizas que estos daban al General Hugo.
También me traen a la memoria esos murmullos la voz de mi abuela, cuando
a mí y a mi hermana nos contaba las fiestas que dieron en el Retiro por
el casorio de Doña Bárbara con Fernando VI; la voz de mi padre ¡ay! una
tarde, cuando, sentaditas mi madre y yo en este mismo sitio desgranando
judías, entró y muy afligido nos dijo que le habían cortado la cabeza al
Rey de Francia. Esto fue el año 93: la noticia de tal atrocidad llegó a
nuestra villa el día de San Blas: ya veis si tengo memoria\ldots{} Con
que, no matéis los ruidos, y dejadme mi casa como está\ldots{} No seáis,
por Dios, tan modernos.»

\hypertarget{x}{%
\chapter{X}\label{x}}

El testamento de Miedes, otorgado en Sigüenza veinte años ha, carecía de
interés por la desaparición de los bienes raíces. Los consistentes en
papel impreso y escrito pasaban a ser propiedad del Seminario de San
Bartolomé de Sigüenza, y el ajuar de casa, ropa y trebejos, que en buena
tasación no valdrían arriba de ochenta reales, se adjudicaba
íntegramente a la señora Laureana de La Toba, conocida por \emph{la
Ranera}. Habiéndome dicho un día D. Juan Taracena, testamentario con el
confitero Gutiérrez del Amo y D. Cosme Aparicio, que en el revoltijo de
la biblioteca se había encontrado un cajón de papeles escritos de puño y
letra del erudito atenzano, me picó el deseo de echar la vista sobre
ellos, y accedí a la invitación del señor Cura para examinarlos juntos,
y rebuscar algunos destellos de inteligencia dentro de aquel caos. Y
aquí viene a pelo la explicación de que lleve la fecha de Octubre esta
parte de mis Confesiones, toda en una pieza, después del largo silencio
de cuatro meses en que suspendida tuve mi comunicación con la
Posteridad. Lo poco que escribí desde la petición de mano hasta el día
de mi casamiento, pareciome tan falto de interés y sobrado de
fastidiosas declamaciones tocantes a la dignidad humana
\emph{sacrificada en aras del positivismo}, que lo rompí para no causar
risa y tedio a mis futuros lectores\ldots{} Entré por el aro del
matrimonio agenciado por mi hermana; nos vinimos a esta villa mi mujer y
yo, y pronto advertí la imposibilidad de escribir mis reservados
pensamientos, porque mi esposa y mi madre no me dejaron ni un instante
en la soledad necesaria para tal desahogo. Han pasado los meses en
espera de una ocasión dichosa, la cual no ha venido hasta que, sin
recelo de María Ignacia, he podido recluirme en la caverna del viejo
Miedes con el pretexto muy razonable de la compulsa y escrutinio de sus
descabalados papelotes.

En tres mañanas de recogimiento y aplicación, he podido emborronar toda
esta parte de los días de Atienza, que a mi parecer no será de las que
menos ilustren y amenicen la historia de mi vida, en contacto con la
vida y alma españolas. Ni mi mujer ni mi madre se sorprenden de que pase
aquí mañanas enteras, y aun les parece poco cuando a la hora de comer
les doy cuenta de los peregrinos borrones en prosa y verso que D. Juan,
revolviendo lo pasado, mientras yo escribo para lo futuro, ha podido
descubrir en este maremágnum: un Discurso de tesis escolástica (Alcalá,
1801), una epístola en ripiosos tercetos \emph{Contra el vicio de hablar
y vestir a la francesa} (1823), un extenso alegato refutando las
crónicas que atribuyen la fundación de León al Rey egipcio
\emph{Mercurio Trismegisto} (muy señor mío), y por fin una serie de
cartas que D. Ventura, por comezón monomaníaca, escribía desde su
solitaria cueva a todo personaje que descollaba en la celebridad militar
y política. Había carta a Espartero, al Marqués de Miraflores, a
Olózaga, a Martínez de la Rosa, a Mendizábal y a Narváez, y era
particularidad de todas ellas que, principiadas con gran esmero de letra
y profusión de atrevidos pensamientos, ninguna estaba concluida y, por
tanto, ninguna había ido a su destino. Graciosísima entre todas era la
que empezó a escribir para Narváez, con fecha reciente. Tanto gusto tuve
de su lectura que Taracena me la regaló, y aquí transcribo un párrafo de
ella muy interesante: «En vos, Señor, saludan las presentes kalendas al
esclarecido descendiente de aquellos \emph{Turdetanos} que en el Sur de
nuestra Península renovaron la ciencia de los famosos \emph{Túrdulos},
compañeros de nuestro común padre Túbal. La historia que de Vuecencia se
ha de escribir notará la concordancia del su carácter con el etimológico
sentido de la palabra \emph{Túrdulo}, que se compone de \emph{Thur}
(buey) y de \emph{Duluth} (exaltado). Reconociendo en Vuecencia el
primer \emph{túrdulo} del Reino, yo le proclamo Buey, que es lo mismo
que decir fuerte, y Exaltado, que suena lo mismo que \emph{liberal}, de
donde sale la especiosa síntesis de Vuecencia, o sea el ayuntamiento y
consorcio de los atributos de \emph{Fuerza} y \emph{Libertad}\ldots»

La soledad de Atienza se alegró estos días con la llegada de los
maranchoneros\ldots{} Son estos habitantes del no lejano pueblo de
Maranchón, que desde tiempo inmemorial viene consagrado a la recría y
tráfico de mulas. Ahora recuerdo que el gran Miedes veía en los
maranchoneros una tribu cántabra, de carácter nómada, que se internó en
el país de los \emph{Antrigones} y \emph{Vardulios}, y les enseñaba el
comercio y la trashumación de ganados. Ello es que recorren hoy ambas
Castillas con su mular rebaño, y por su continua movilidad, por su
hábito mercantil y su conocimiento de tan distintas regiones, son una
familia, por no decir raza, muy despierta, y tan ágil de pensamiento
como de músculos. Alegran a los pueblos y los sacan de su somnolencia,
soliviantan a las muchachas, dan vida a los negocios y propagan las
fórmulas del crédito: es costumbre en ellos vender al fiado las mulas,
sin más requisito que un pagaré cuya cobranza se hace después en
estipuladas fechas; traen las noticias antes que los ordinarios, y son
los que difunden por Castilla los dichos y modismos nuevos de origen
matritense o andaluz. Su traje es airoso, con tendencias al empleo de
colorines, y con carreras de moneditas de plata, por botones, en los
chalecos; calzan borceguíes; usan sombrero ancho o montera de piel;
adornan sus mulitas con rojos borlones en las cabezadas y pretales, y
les cuelgan cascabeles para que al entrar en los pueblos anuncien y
repiqueteen bien la errante mercancía.

Todo Atienza se echó a la calle a la llegada de los maranchoneros con
ciento y pico de mulas preciosas, bravas, de limpio pelo y finísimos
cabos, y mientras les daban pienso, empezaron los más listos y
charlatanes a dar y tomar lenguas para colocar algunos pares. En mi casa
estuvieron dos, sobrino y tío, que a mi madre conocían; mas no iban por
el negocio de mulas, sino por llevarnos memorias y regalos de mi hermana
Librada y de su familia. (Si no lo he dicho antes, ahora digo que mi
hermana mayor, casada en Atienza con un rico propietario, primo nuestro,
había trasladado su residencia, en Abril de este año, a Selas, y de aquí
a Maranchón, por el satisfactorio motivo de haber heredado mi primo
tierras muy extensas en aquellos dos pueblos.) Obsequiados los
mensajeros con vino blanco y roscones, de que gustaban mucho, se enredó
la conversación, y al referirnos pormenores de su granjería y episodios
de sus viajes, vino a resultar que inesperadamente, sin que precediera
curiosidad ni pregunta nuestra, tuvimos noticia de la cuadrilla o tribu
de los Ansúrez.

Entre otros cuentos o aventuras refirieron los tales que en una venta
cerca de Trijueque habían topado con los vagabundos, entrando en
pláticas y tratos con ellos, porque el Jerónimo les propuso comprarles
una mula de las ancianas, no para comerciar, sino para andar en ella, no
llegando a entenderse porque parecía insegura la fianza. Vista y
examinada la linda moza que los Ansúrez llevaban, propusieron los
marchantes tomarla a cambio, no de una mula, sino de dos, a escoger, y
con algún dinero encima si así fuese menester para igualar, y de esto
vino una pendencia con palos recíprocos, teniendo que salir más que de
prisa los \emph{agitanados} para que no acabara en sangre la
función\ldots{} Después volvieron a encontrarse en Taracena,
\emph{resultando} que la moza se había comprado zapatos en Valdenoches,
y algún trapo con que más honestamente se tapaba. Esquivaron los de
Maranchón nuevas disputas; pero la casualidad les hizo presenciar la que
tuvieron los Ansúrez entre sí, unos hijos con otros y algunos con el
padre, saliendo de la refriega la hermanita con un chichón en la frente;
y a consecuencia de este gran cisco se separaron, tirando cada cual por
su lado, como huyendo unos de otros, con intención de no volver a
juntarse nunca. Uno de los hijos tiró hacia Brihuega, otro se metió por
el camino que conduce a Pastrana y al paso para Cuenca y Reino de
Valencia, el tercero subió hacia el lugar de Talamanca, como para
correrse a Segovia; el cuarto dijo que se quedaría en Guadalajara, y el
chiquitín, con la hija guapa y el padre anciano dijeron que derechamente
se iban a Madrid. La dispersión de la tribu, contada con tanta sencillez
por los traficantes de mulas, me hacía el efecto de las emigraciones de
los hijos de algún patriarca, tal como la fábula o la Historia nos las
transmiten, y la salida de cada cual para fundar pueblos y difundir
ideas al Norte y al Sur, hacia donde nace o se pone el sol. Estaba sin
duda mi cerebro bajo el influjo de las ideas de Miedes, y en todo veía
éxodos de razas, familias dispersas, y viajes que traen la civilización
o van en pos de ella.

Y como persisto en no ocultar nada de lo que siento, séame o no
favorable, diré que desde que oí a los muleros, no se apartó de mi
pensamiento la imagen de la hija de Ansúrez. «¿Qué apuestas a que te
adivino lo que estás pensando?---me dijo Ignacia por la noche, ya solos
en nuestra alcoba. Y yo me eché a temblar, porque en efecto, mi mujer de
algunos días acá me adivina los pensamientos con sólo mirarme, y a veces
sin este requisito, por pura infiltración del rayo de sus ojos al través
de mi frente, o por misteriosa lectura de signos que trazan sin quererlo
mis manos, mis pasos, mi sombra sobre las paredes o el suelo. Antes que
acabara de responderle con una donosa evasiva, me dijo: «¡Mentiroso!
estás pensando en Lucila, o digamos \emph{Illipulicia}, como la llamaba
su enamorado caballero D. Ventura.» Negué; di nuevo giro a nuestro
coloquio; mas era verdad que en Lucila pensaba, llevando muy a mal que
descompusiese su escultural figura imponiendo a sus libres pies el
suplicio y la fealdad de estas horribles invenciones de los zapateros.
Por mi gusto habríale comprado en Guadalajara, en Cogolludo o donde la
encontrase, túnica y manto de finísima franela blanca, con las cuales
prendas y un delgadísimo camisolín de batista cubriese y guardase
honestamente toda su persona, sin añadidura de corsé, ni faja, ni
cinturón, ni canesú, ni medias, ni cosa alguna más que lo dicho,
privándola asimismo de toda suerte de alhajas o accesorios, que siempre
habían de interceptar alguna parte o pedacito de su soberana belleza, y
de distraer los ojos que en contemplarla se embelesaban. Sólo en su
cabeza consentiría un aro de metal, oro puro sin ornato ni piedras
preciosas, que sujetase su espléndida cabellera, recogida y arrollada en
una sola onda. Guardaba yo esta imagen en el más recóndito espacio de mi
pensamiento, bien sujeta de mis disimulos para que no se me escapase, y
le tributaba culto espiritual, castísimo, haciéndome la cuenta, como el
loco Miedes, de que en tal figura amo el alma de un pueblo y la
\emph{historia de las cosas vivas}.

El invierno nos arroja de Atienza. Echo muy de menos la sociedad, mis
amigos, la política, el fácil y pronto conocimiento de cuanto pasa en el
mundo. Ya resuenan lúgubremente en los empedrados de la antigua
\emph{Tutia} las herraduras de las caballerías que suben y bajan por
estas empinadas calles y carreras; ya se me hace fúnebre como el
\emph{Dies iræ} el ladrido de los perros en largas noches, y hasta el
matutino canto de los gallos me suena como una invitación a que tomemos
el portante. Y de los ruidos del maderamen de la casa no digamos: ellos
son de tal modo tristes, que harían regocijadas las \emph{Noches} de
Young y de Cadalso\ldots{} Ya me inspiran profunda antipatía los señores
y damas del pueblo, que con su apéndice de niñas emperejiladas a estilo
de Madrid redoblan ahora sus fastidiosas visitas, sin duda porque no
tienen a dónde ir. No puedo soportar a las de Aparicio; las del
Confitero me amargan, y las del Médico me enferman. D. Lucas de la
Cuadra se me ha sentado en la boca del estómago, y D. Manuel Salado en
la coronilla\ldots{} Ya los pórticos románicos se desdicen de todas
aquellas donosuras poéticas que nos habían cantado, y el alto Castillo
se reviste de una fiereza tal, que no nos atrevemos a mirarle cara a
cara. Si al pronto las nieves nos alegran la vista, no tardamos en
asustarnos de su blancura irónica, que deslíe y absorbe los colores de
la campiña, mata todo sonido y borra todo signo vital. Vientos glaciales
bajan del Alto Rey y quieren barrernos. La vida se reconcentra en las
cocinas, como en el orden vegetal desciende a las raíces la savia, y
junto al fuego se agrupa toda la bárbara inocencia y la marrullera
ignorancia de la humanidad campestre.

Madrid nos llama y Atienza nos despide, pues mi propia madre, que no se
cansa de tenernos a su lado ni de prodigarnos su inextinguible cariño,
reconoce que es hora de que ella torne a Sigüenza y nosotros a la Villa
y Corte, con todas las precauciones imaginables y cien más, y aún es
poco, porque\ldots{} hace días anduvieron ella y María Ignacia en
secreteos, y según parece, ya no hay dudas respecto a lo que más
deseamos todos, esposo y padres\ldots{} ¡Ay, Dios mío! El temor de un
fracaso, que ahora no sería imaginario como en los días de nuestra
llegada, inspira a mi señora madre las más audaces previsiones y los
planes más peregrinos respecto a viaje, método y pausas con que debemos
realizarlo, estructura y acomodos del coche, limpieza y monda de piedras
en todos los caminos que hemos de recorrer\ldots{} Pronto a partir,
precisado me veo a poner fin a estas páginas trazadas al descuido y como
a hurtadillas en la polvorosa madriguera del erudito atenzano. ¿Pluma de
estas Confesiones, cuándo volveré a cogerte?\ldots{} Adiós, Atienza,
ruina gloriosa, hospitalaria; adiós, santa madre mía; adiós, \emph{Noble
Hermandad de los Recueros}, que me hicisteis vuestro \emph{Prioste};
adiós, amigos míos, curas de San Juan, San Gil y la Trinidad; adiós,
Teresita Salado, Tomasa y chiquillos que alegrabais nuestras tardes;
adiós, paz y recreo del campo, simplicidad de costumbres; adiós, sombra
del grande y misterioso Miedes, el de la locura graciosa y sublime, el
soñador celtíbero, enamorado de la más bella representación del alma
hispana; adiós, en fin, imagen de la errante Lucila, mentira de la
realidad y verdad casi desnuda que pasaste como un relámpago de
hermosura entre el polvo de los deshechos terrones\ldots{} adiós, adiós,
adiós\ldots{} Ved aquí las últimas plumadas, las últimas sin remedio,
porque tengo que sellar y empaquetar cuidadosamente estos papeles para
llevármelos bien guardaditos\ldots{} No más, no más\ldots{} Hasta que
Dios quiera.

\hypertarget{xi}{%
\chapter{XI}\label{xi}}

\textbf{Madrid}, \emph{22 de Noviembre}.---Me parece mentira que puedo
consagrar un rato al desahogo de estas Confesiones, en lugar seguro,
lejos de la inspección y vigilancia de mi mujer, de mis suegros y de
toda la ilustre familia con quien vivo, tratado como príncipe, regalado
hasta el mimo, pero sin libertad. No debo quejarme, pues los bienes que
Dios derrama generoso sobre mí aligeran la cadena de oro que arrastro,
reduciéndola, fuera de contadas ocasiones, al peso y tensión de un
cabello. No me quejo; voy muy a gusto en este gallardo machito: en mi
casa me aman, y tienen de mí la más alta idea; en sociedad me veo
rodeado de consideraciones; el respeto me sigue, la admiración me
acompaña, y el dorado vulgo me rinde homenajes que en mi vida de célibe
nunca pude soñar. A mi nombre va unida, con el flamante título que
ostento, la idea de sensatez; pertenezco a \emph{las clases
conservadoras}; soy una faceta del inmenso diamante que resplandece en
la cimera del Estado y que se llama \emph{principio de autoridad}: en mí
se unen felizmente dos naturalezas, pues soy \emph{elemento joven}, que
es como decir inteligencia, y \emph{elemento de orden}, que es como
decir riqueza, poder, influjo. Váyanse, pues, unas libertades por otras,
que algo se puede sacrificar de la doméstica para gozar la pública, la
que nos autoriza para campar con nuestra caprichosa voluntad por encima
de la cuitada multitud, a quien nunca falta Rey que la ahorque ni Papa
que la excomulgue.

Desde que regresamos de Atienza, toda tentativa de confesión escrita
hallaba en la curiosidad de los míos insuperable obstáculo: ¿pues qué
había yo de escribir que mi mujer no atisbase, receloso fiscal de mis
pensamientos? Ausente mi amigo Aransis, no tenía yo quien me diese
seguro asilo, que bien puedo llamar confesonario; ahora que vuelve
Guillermo a Madrid, a su casa me voy y en su cuarto me meto, y en su
papel escribo\ldots{} Sepan los que en futura edad me leyeren que amo a
Ignacia con plácida ternura, y que estoy muy contento de haberla hecho
mi esposa. El afecto que le doy débilmente corresponde, así debo
declararlo, al exaltado amor que ella tiene por mí, y a la ofrenda que
constantemente me hace de su sinceridad, pues todo me lo revela y
confía, desde las cosas más importantes a las más menudas, y no hay
repliegue de su conciencia ni secreto de su mente que no ponga ante mí.
Su inteligencia descubre y ostenta de día en día nuevos tesoros. Con sus
padres es la niña encogida y vergonzosa de siempre, petrificada en las
ñoñerías tradicionales de la casa; para mí es la mujer de libre
pensamiento, la mujer de ideas propias que en el sagrario matrimonial
rompe el cascarón en que la criaron, y conservando hacia la familia las
fórmulas de un pasivo respeto, sólo en el esposo pone su alma entera.

Padre seré de los hijos que Ignacia quiera darme, y como es bueno que me
ejercite en las paternales obligaciones, de la Patria quieren hacerme
venturoso papá. Me ha llamado Sartorius para decirme con cortesana
franqueza que, por mi posición independiente y mis dotes intelectuales,
\emph{estoy llamado} a representar un distrito en el futuro Congreso.
¡Paso a los hombres de arraigo; atrás los vividores! Este lema de
regeneración política me parece muy bello, y no vacilo en poner al
servicio del país todo mi arraigo, que espero ha de aumentarme Dios.
Aunque las elecciones generales para nuevas Cortes no han de ser hasta
el año próximo, el previsor Conde me pregunta si llegado el caso podría
yo disponer en Sigüenza de los necesarios \emph{elementos} para el
triunfo. Le contesto que no me faltan allí parentela y amigos; pero
desconfío del éxito si vuelve a presentarse, como presumo, el señor
Conde de Fabraquer. Por lo que me aseguró el alcalde de Atienza, D.
Manuel Salado, con Fabraquer no será posible la lucha, a menos que el
Gobierno no haga un verdadero desmoche y tabla rasa\ldots{} Hablamos en
seguida de Brihuega, donde \emph{toda la fuerza} es de D. Luis María
Pastor; de Almazán, donde probablemente luchará, y no han de faltarle
medios y buenas armas, el Sr.~Ramírez de Arellano, funcionario de Gracia
y Justicia; y por fin echamos una miradita a Molina de Aragón, donde la
desventaja de tener enfrente a un antagonista tan formidable como D.
Fernando Urries, se compensara con el apoyo que ha de darme mi cuñado y
primo, gran propietario en Selas y Maranchón, y a poco que me ayude el
Gobierno\ldots{} Pensó en ello un instante Sartorius, y después me dijo:
«Ya lo resolveremos de aquí a las elecciones generales, que serán el
invierno próximo\ldots{} y por mi gusto no se convocarían nuevas Cortes
hasta el 50\ldots{} De todos modos tenemos tiempo\ldots{} Pero usted no
debe estar ocioso, amigo mío. Cada día se nota más en esas malditas
Cortes la falta de personas de arraigo\ldots{} Las complacencias de los
Gobiernos con los que hacen de la política un oficio, van desmoronando
el Régimen\ldots{} Yo veré si le sacamos a usted en alguna elección
parcial\ldots»

Volví, por indicación del amable Ministro, a los cuatro días; pero nada
de mi presunta paternidad política pudimos hablar, porque las graves
noticias llegadas de Roma arrebataban la atención de los hombres más o
menos \emph{arraigados}, no dejando espacio para tratar de personales
asuntillos. A pesar de esto, debo confesar ingenuamente que si en la
concurrida recepción o tertulia de Sartorius, a horas altas de la noche,
aparecí asociado al general asombro y pena que ocasionan los graves
sucesos de Italia, sentí en mi interior el hielo de la desafección a
todo lo que no trajera ligamentos o enlace con mi propio bienestar. En
verdad digo que lo ocurrido en Roma me inspira un cuidado muy relativo,
y no ha de quitarme porción ninguna del sosiego de mis días ni del sueño
de mis noches. Pero, como todos me creen muy entendedor de cosas y
personas romanas, no cesaron aquella noche de interrogarme acerca de los
antecedentes y móviles de los aterradores acontecimientos; contesté
conforme a mi conocimiento personal, y añadiendo a lo que ignoro alguna
ingeniosa gala de mi fantasía, satisfice la curiosidad y escuchado fui
como un oráculo.

Acerca del Marqués de Azeglio, propagandista de las ideas liberales bajo
la bandera papal, y del partido llamado \emph{Joven Italia}, que
proclamaba las dos grandes ideas \emph{Libertad} y \emph{Unidad}; acerca
del grande y austero revolucionario Mazzini, que a su fin va sin reparar
en los medios, hombre de robusta inteligencia, de formidable voluntad,
frío, despiadado, cerrado a todo sentimiento que no sea el de un
patriotismo fanático, a la romana, mezcla imponente de Catón y Sila, les
di prolijos informes que a mi parecer se aproximaban bastante a la
verdad. Las concesiones de Pío IX a los revolucionarios, que aparecían
en las calles de Roma ennegrecidos aún con el tizne de las logias, yo
las había presenciado; y también vi que el Papa, otorgando al pueblo
cuanto este pedía, llegó al límite de la generosidad. El pueblo,
desvanecido por las ideas de Balbo y Gioberti, y por la predicación del
Marqués de Azeglio, pedía más cuanto más obtenía. Mastai Ferretti
concedió el Ministerio laico, y Constitución y Cámaras. La moda de las
Constituciones llegó a invadir la morada de la inmutable Iglesia. Contra
la \emph{Joven Italia} y los revolucionarios alzaba fuerte antemural el
Imperio austriaco, poseedor de las más bellas regiones del Norte de
Italia; contra el Austria armaba sus huestes Carlos Alberto, Rey de
Cerdeña. ¿Ante cuál de estos dos poderes se inclinaría San
Pedro?\ldots{} Diles una explicación sucinta de las dos ideas
fundamentales que la Historia expresa con los términos rutinarios de
\emph{güelfos} y \emph{gibelinos}, y les referí que en los postreros
días de mi estancia en Roma yo había visto al Papa indeciso (perdonad,
yo le veía en la opinión que me rodeaba, dándome la perspectiva general
de las cosas), y, por fin, inclinado a no romper con el Imperio. Si
Julio II gritó «fuera los bárbaros,» Pío IX creyó sin duda comprometer
su tiara si los bárbaros, entiéndase austriacos, negaban su apoyo al
débil Estado romano y a la Barca del Pescador.

Incansable en organizar las demostraciones patrioteras, a la calle
lanzaba Mazzini las multitudes, con cuyo vocerío halagaba y amedrentaba
al Pontífice, el cual, harto de vanos ruidos y agobiado bajo la
pesadísima responsabilidad de la Iglesia que llevaba sobre sus hombros,
gritó un día en el balcón del Quirinal: «No puedo, no debo, no quiero.»
Con esto, y con la Encíclica en que desmintió el Pontífice su política
del 46 y 47, se desligó de la \emph{Joven Italia}: deshecha como el humo
la popularidad de Mastai Ferretti, el sentimiento popular le acusó de
defección a la causa de la patria. Lanzado a la resistencia, Su Santidad
nombró Ministro al Conde de Rossi.

A una me interrogaron acerca de este desgraciado personaje, y aunque yo
no le conocía más que de verle en la calle cuando era Embajador de
Francia, hice de él pintura física y moral con los elementos de la
opinión oída o sentida, que casi siempre han sido los más eficaces
medios de la Historia. Rossi era un hombre pálido y pensativo, poco
elegante y un tanto displicente, gran jurisconsulto y expositor de
ciencia jurídica\ldots{} Ministro papal (esto no lo alcancé yo, pero
hablé de ello como si lo hubiera visto), desplegó una energía que había
de ser insuficiente contra la hinchada onda de la revolución.

«¿Conoce usted el palacio de la Cancillería, en cuya escalera ha sido
asesinado Rossi?---me preguntan con el intenso interés trágico que
despierta el lugar de un crimen. Y yo impávido, bien asistido de mis
luminosos recuerdos, les describo todo el barrio, la \emph{via
Pellegrini}, el \emph{Campo di Fiori}; encaro con la majestuosa fachada
de la Cancillería, trazada por Bramante; traspaso el monumental pórtico,
obra de Fontana; entro en el bello patio, y torciendo a mano izquierda,
señalo el arranque de la escalera, en cuyos primeros peldaños ha
perecido a manos de la demagogia desmandada el Ministro de Pío IX. Luego
me lanzo de nuevo a la calle, y con mi fácil vena descriptiva les guío
hacia las construcciones heteróclitas entremezcladas con los vestigios
del \emph{Teatro de Pompeyo}, ¡donde fue asesinado César!\ldots{} y
admiran la coincidencia, que no está en las personas, ni en la calidad o
móviles del delito, quedando sólo reducida a la vecindad de lugares
trágicos. En pueblos tan pletóricos de Historia como aquel, las
tragedias se tocan, y juntas están las piedras en que sucumbieron
mártires o afilaron sus cuchillas los verdugos.

\emph{1.º de Diciembre}.---Según las noticias de Roma que nos llegan por
los correos de Francia, Rossi fue víctima de su temeraria confianza o de
su indomable valentía. Más altanero que precavido, despreció los avisos
que se le dieron de que las logias habían decretado su muerte. Entró
solo, sin miedo ni precaución, en la Cancillería, rompiendo por entre
una multitud enconada y bullanguera. Al poner el pie en el primer
peldaño recibió un garrotazo en el costado derecho. Volviose, y en el
mismo instante, por la izquierda, una furibunda mano armada de cuchillo
le cortó la yugular. Muerto el Ministro, la autoridad temporal del
Pontífice era una vana sombra. El siguiente día, 16 de Noviembre, trajo
el desenfreno de las muchedumbres, las gesticulaciones del patriotismo
epiléptico frente al Quirinal, la ansiedad de Pío IX, el ir y venir de
comisiones pidiendo y negando\ldots{} Las noticias de hoy confirman que
Su Santidad huyó de Roma. ¿En qué forma? ¿Disfrazado de aldeano como
Juan XXII escapando del Concilio de Constanza, o de mercader como
Clemente VII escabulléndose por entre las tropas españolas?

\emph{3 de Diciembre}.---Por referencias de nuestra Embajada se sabe que
Mastai Ferretti salió del Quirinal vestido de simple cura, y en
velocísima carrera de coche se plantó en Albano. Allí le tomó de su
cuenta el Ministro bávaro, conde de Spaur, que viajaba con su señora y
familia menuda. Con el carácter de ayo de los niños salvó Pío IX
felizmente la distancia entre Albano y la frontera de Nápoles\ldots{} Ya
le tenemos en Gaeta, que ha venido a ser la provisional Sede y metrópoli
del mundo católico. En Roma imperan Mazzini, Sterbini, Cicerovacchio, el
Príncipe Canino, que es un Bonaparte encenagado en la demagogia, y les
sigue y hace coro la ronca turba insaciable. Grandes acontecimientos se
preparan en el mundo. Arde Italia. El caballeresco Carlos Alberto reúne
la más florida milicia lombarda y piamontesa para marchar contra
Austria\ldots{} ¿Qué pasará? ¿En qué pararán estas colosales trifulcas,
que comparadas con nuestras revoluciones de campanario no nos parecen
menos grandes que los combates de Dioses y Héroes en los cantos de
Homero, o las peleas de arcángeles en las estrofas de Milton?\ldots{} No
lo sé, ni en verdad me importa mucho. Rueden los tronos; vacile, ya que
rodar no pueda, la inmortal tiara; sobre las monarquías deshechas alcen
su imperio efímeras o vigorosas repúblicas. Nada de esto alterará la paz
del hombre árbol, que ve resueltos los problemas de su nutrición
vegetal, y siente bien asegurado el suelo entre sus hondas raíces. Mi
optimismo me asegura que las tempestades europeas no se correrán a
España, porque aquí tenemos la Providencia de un D. Ramón María Narváez
que con el ten con ten de su fiereza y gracias andaluzas, tigre cuando
se ofrece, gato zalamero si es menester, maneja, gobierna y conduce a
este díscolo Reino, y en él asegura el bienestar de los que lo han
adquirido, o están en el trajín de su adquisición. Vívame mil años mi
\emph{Espadón de Loja}, y durmamos tranquilos los que juntamente somos
usufructuarios y sostenedores del orden social.

\hypertarget{xii}{%
\chapter{XII}\label{xii}}

\emph{16 de Marzo de 1849}.---De tal modo absorben mi espíritu el
cuidado de mi cara mitad y el problema de la sucesión, que ha de
resolver María Ignacia, según los cálculos más discretos, en fines de
Mayo o principios de Junio, que no hay espacio en mi pensamiento para
suceso alguno de orden distinto, así privado como público. ¿Qué me
importan las alteraciones de Francia, de Roma o de Hungría, ni las
malandanzas del Estado español, ante este inmenso enigma del embarazo,
cuyo término y desenlace feliz esperamos con el alma en un hilo? ¿Qué
puede afectarme ese lejano enredo de la República Romana, ni las
diabluras de los Mazzinis, Caninos y Garibaldis? ¿Ni qué atención puedo
prestar a los entusiasmos de mi cuñada Sofía por Luis Napoleón,
Presidente de la República Francesa, o por Manin, desgraciado Dux de la
de Venecia? Y cuando mi hermano Gregorio me da irresistibles matracas
por el desconcierto de la Hacienda española, ¿qué he de hacer más que
abrir la oreja derecha para que salga lo que por la izquierda entró? Ya
comprenderéis que de la guerra intestina que arde en Cataluña hago tanto
caso como de las nubes de antaño, que lo mismo es para mí Cabrera que un
monigote de papel, y que los movimientos de Pavía, de Concha o de
Córdova en persecución de los facciosos no mueven mi curiosidad. Entre o
salga Montemolín, lo mismo me da, por no decir que ahí me las den todas.

No me cansaré de afirmar que son cada día más vivos y puros mis afectos
hacia la compañera de mi vida, y que esta ha llegado a seducirme y
enamorarme con sólo el talismán de sus anímicas dotes. Diré también que
mis suegros y toda la familia me quieren entrañablemente, viendo y
comprobando con diarios ejemplos que \emph{hago feliz} a la niña. Cuido
mucho de no dar pretexto al menor disgusto de mis papás políticos,
atento siempre a mi completa identificación con ellos y a fundirme en
las ideas y rutinas del mundo Emparánico, sin hipocresía ni violencia.
Sólo en los comienzos de mi asimilación me causaron enojo las extremadas
santurronerías a que las señoras mayores me sometieron, y se me hacía
muy largo el tiempo consagrado, sobre la diaria misa, a Triduos,
Cuarenta Horas, o visitas a las monjas del Sacramento, de la Latina y de
Santo Domingo el Real; pero a ello me fui acostumbrando con graduales
abdicaciones del albedrío, hasta llegar a cierta somnolencia que se
compadece con las materiales ventajas de mi posición. Por el bienestar
que me rodea y las comodidades que disfruto, doy gracias a Dios y a mi
hermana Catalina, sintiendo mucho no poder dárselas más que con el
pensamiento, pues desde que volví de Atienza no he visto a la bendita
religiosa, que ahora está rigiendo la comunidad Concepcionista
Franciscana de Talavera de la Reina. Ved aquí por qué no la he nombrado
en esta parte de mis Confesiones. De veras me ha dolido no encontrarla
en Madrid, no sólo porque estoy privado de sus consejos amorosos, sino
porque su ausencia me tiene ignorante de si recibió y acogió a los
Ansúrez, recomendados por mi carta. Nada sé de esta gente, nada del
noble patriarca de la tribu, nada de la sin par Lucila, y pienso que,
desamparados aquí, se han corrido a tierras distantes.

Volviendo a mi nueva familia y al fenómeno de mi adaptación social, diré
que fue para mí un poquito duro, en los primeros días, el trato de las
personas que frecuentaban mi casa en las veladas de invierno. Poca
substancia, o más bien ninguna, sacaba yo de la conversación de los
respetables señores carlinos o convenidos de Vergara, a los que no creo
ofender si digo de ellos que su desenfrenado absolutismo me daba de cara
como un mal olor de boca. A los que ya he dado a conocer tendré que
añadir alguno, si Dios me da salud y tiempo, que ostentando traje
militar o civil, trae olor de curas y tipo de la Bóveda de San Ginés.
Pero con todos estos tufos y apariencias desagradables, yo voy
apechugando con ellos, y ya no me causan la menor molestia ni sus
personas anticuadas ni sus estrafalarios discursos. A todo se hace el
hombre en las diferentes situaciones a que le lleva su Destino, y por
algo dice la filosofía popular: \emph{No con quien naces, sino con quien
paces}. En realidad yo pacía exclusivamente con mi mujer, y de este
nuestro pastar reservado en el íntimo campo conyugal, nació el que yo me
adaptase fácilmente a la vida Emparánica, como se verá por lo que voy a
referir ahora.

Me lanzo a descubrir y delatar lo más secreto de mis conversaciones con
María Ignacia. Ya en los días de Atienza, cuando nos quedábamos solos,
se me quejaba de la pesadez insulsa del rosario que mi madre nos hacía
rezar con ella todas las noches. Claro es que estas opiniones eran sólo
para mí, y ante mi madre nada decía que pudiera disgustarla. En Madrid
me manifestó las propias ideas, y una noche llegó a decirme: «El rosario
me sirve a mí para pensar en mis cosas. No hay nada más propio que esta
taravilla para meterse una en sí misma. Ya tengo yo mi lengua bien
acostumbrada a rezárselo ella sola, y la dejo ir al compás de la
cancamurria de los demás. Dentro de mí, yo solita pienso, y si viene a
pelo, le pido a Dios con palabras mías lo que quiero pedirle\ldots{}
¡Vaya, que si dijese yo estas cosas a mis tías, creerían que me he
vuelto loca! Pues hace tiempo que pienso así; pero a nadie lo he dicho,
porque la vergüenza me sellaba la boca. Como entre nosotros no hay
vergüenza, todos mis pensamientos son tuyos.

Y en la noche de un día consagrado a religioso bureo, con misa solemne
por la mañana, por la tarde manifiesto y procesión, y como fin de
fiesta, fastidiosa charla mística del Sr.~Sureda con nuestras reverendas
tías, María Ignacia, cuando estuvimos donde nadie pudiera oírnos, me
dijo: «Con muchos días como este, pronto se hace una volteriana, aunque
yo, la verdad, no he leído a ese Voltaire ni falta que me hace. Oye,
Pepe: ¿no te parece que sobre todas las estupideces humanas está la de
adorar a esos santos de palo, más sacrílegos aún cuando los visten
ridículamente? ¿No crees que un pueblo que adora esas figuras y en ellas
pone toda su fe, no tiene verdadera religión, aunque los curas lo
arreglen diciendo que es un símbolo lo que nos mandan adorar entre
velas? Yo te aseguro que no siento devoción delante de ninguna imagen,
como no sea la de Jesucristo, y que si yo tuviera que arreglar el mundo,
mi primer acto sería condenar al fuego a toda esa caterva de santos de
bulto, empezando por los que llevan ropa.

---Lo mismo pienso---le respondí.---Pero nosotros, que tenemos nuestro
entendimiento limpio de esos desvaríos, hemos de disimularlo, y hacer
como que no discurrimos, ni vemos más allá de las narices del
Sr.~Sureda, o de tu tía Josefa\ldots{} Seamos cautos, mujer mía, que
nada cuesta decir a todo \emph{amén}, y vivir en santa paz con la
familia.»

Y una noche, recordando lo que desentonadamente se habló en nuestra
tertulia de la situación del Papa, y de las tropas que mandaremos a
Italia para restablecerle en su trono, mi mujer se dejó decir: «Ya ese
bendito Conde de Cleonard me tenía estomagada con que la Iglesia debe
ser maestra de la vida en todos los órdenes, con que los liberales están
condenados, con que debemos traernos para acá al Papa, y hacerle cabeza
de nuestra nación\ldots{} Pues yo digo que si es Vicario de Jesucristo,
¿para qué necesita fusiles y cañones? Jesucristo no tuvo artilleros, ni
le hacían falta para nada\ldots{} Y también digo que no tuvo
embajadores, ni ministros de Hacienda, ni cobraba dinero por bulas o
dispensas, ni gastaba esos lujos\ldots{} como que nunca se puso zapatos.
¿Lo entiendes tú, Pepe? Me dirás que no, y que tus dudas son iguales a
las mías\ldots{} Pero tienes razón, hijito: callémonos y hagámonos los
tontos, que así nadie se mete con nosotros, y vivimos tan tranquilos.»

El escepticismo de mi cara esposa no se estacionaba: era esencialmente
progresivo, como se verá por los conceptos formulados hará unos veinte
días: «Esto de que hemos de confesar y comulgar todos los meses me
parece un abuso de nuestra paciencia, Pepe. ¿No crees lo mismo? Bueno
que me hagan confesar a mí; pero tú, que eres hombre, ¿por qué has de
arrodillarte tan a menudo delante de un sacerdote para contarle lo que
has hecho? ¡Pues buena tendrías el alma si a cada treinta días te la
llenaras de nuevos pecados! Con confesar una vez al año, o dos, vamos,
bastaría, pienso yo. Claro es que salimos del paso muy lindamente. Yo de
algún tiempo acá no le digo al cura más que lo que me parece. Ya te
conté los disparates que me preguntó el de las Descalzas. Desde entonces
hago mi composición y no me apuro por nada. ¿Y tú cómo te las arreglas
con D. Sinforoso? ¿Es preguntón; es de los que se pasan de listos y
quieren saber, a más de los pecados cometidos, los pecados probables, y
se meten en lo que no les importa?\ldots{} Verdad que tú ya sabrás
desenvolverte. A buena parte van. Yo digo que la mujer casada no debe
confesarse más que con su marido, si este no es un pillete, como hay
muchos. A ti te digo yo todo lo que pienso; tú me dices a mí parte de lo
que discurres, porque un hombre, naturalmente, debe tener alguna más
libertad de pensar, y así somos felices, y nos entendemos a maravilla.»

\emph{30 de Marzo}.---Suspendo aquí los desenfados de María Ignacia,
para dar sitio al estupendo notición de hoy. En Novara, gran batalla
entre piamonteses y austriacos, vencedores estos, viéndose precisado
Carlos Alberto a salir de estampía, previa abdicación en su hijo Víctor
Manuel. No caben en sí de contento los de mi tertulia Emparánica, y mi
hermano Agustín ya ve asegurada la paz del mundo y el orden social con
este triunfo del Imperio\ldots{} Ni ante la rota de Novara, que ha sido
el humo en que se desvanecen las esperanzas unitarias de los italianos,
entran en razón los descamisados y descalzonados de Roma, que siguen
adorando a esa tarasca ebria de su República. El Papa, muy obsequiado
del \emph{Rey Piísimo} (Fernando II), continúa en Gaeta esperando que
las tropas francesas y españolas le devuelvan sus Estados, hoy en poder
de todos los demonios. Estos no van con exorcismos ni anatemas, y es
menester gran cantidad de pólvora y balas para conseguir arrojarlos del
santo cuerpo en que se han metido.

«¿No has reparado---me dijo anoche Ignacia,---que en casa no quieren a
Narváez? Lo habrás notado sin duda. Ello está bien a la vista. Siempre
que hablas de él, para elogiarle, naturalmente, o callan o salen con
alguna cuchufleta\ldots{} y que el Sureda las dice del peor gusto. Luego
papá y las tías no pierden ripio para ponerle faltas: que si es un
cascarrabias, que si no guarda la religión, que si no mira más que por
sí, que si todo lo arregla con andaluzadas, que si debajo de la capa de
moderado es un liberal tremendo, que si ha dicho o no ha dicho del
Nuncio una frase muy fea\ldots{} y no pude enterarme, porque entre sí
los hombres la pronunciaron muy en secreto, y unos se indignaban, otros
se reían\ldots{} En fin, Pepe, que no le quieren en casa, desengáñate.
¿Sabes la que soltó esta noche D. Serafín Cleonard? Pues que la Reina ha
perdido el miedo a Narváez; pero que le mantiene en el poder por meterle
miedo a su marido D. Francisco y tenerle siempre en jaque\ldots{} Mi tía
Josefa, que, como sabes, está muy al tanto de lo que pasa en el cuarto
del Rey, se echó a reír y dijo: «Ya no le temen. ¿Qué han de temerle, si
el tigre va saliendo gato? Preparado está ya el cascabel que han de
ponerle.

---¿Y no añadió quién es el guapo que se lo pondrá?

---Se lo calló la muy ladina. Si mañana se les va la lengua un poquito
más\ldots{} seré toda orejas, para grabarlo bien en mi memoria y poder
contártelo.»

\hypertarget{xiii}{%
\chapter{XIII}\label{xiii}}

\emph{17 de Mayo}.---No me preguntéis nada de cosas públicas, ni aun de
la expedición militar que ha salido ya para Italia. Todo lo ignoro, y lo
que traen a mi oído derecho los amigos cuenteros y parlanchines, o el
bullicio de las calles, no tardo en arrojarlo por el izquierdo hasta
dejar mi caletre vacío de cuanto no pertenezca a mis personales
intereses y cuidados. He tenido a mi mujer muy malita. ¡Qué días, qué
cinco semanas de mortal ansiedad! En mi sobresalto y tribulación temí
que no sólo perdiéramos el fruto, sino el árbol. Gracias a Dios, vimos
felizmente resuelto el infarto de la garganta y cuello con alarmantes
manifestaciones de erisipela\ldots{} Dejadme que respire. Ya la tenemos
completamente bien: el mundo recobra su alegría. Yo le digo a María
Ignacia que Dios está resueltamente de nuestra parte; ella se ríe y me
contesta, barajando la fe con el escepticismo: «Acá para entre los dos,
Pepe, yo pienso que Dios me ha de conceder\ldots{} ya sabes qué\ldots{}
el \emph{tener} felizmente a nuestro hijo, pues ya que me negó tantas
cosas buenas que otras poseen, esta me la tiene que dar. Si no, no sería
justo\ldots{} Aunque\ldots{} vete a saber si es justo. Yo voy creyendo
que no lo es, y que su principal atributo es la injusticia, al menos lo
que por tal tenemos de tejas abajo, y que es quizás\ldots{} la sublime
esencia de la justicia. En fin, chico, lo que quiera Dios ha de ser, y,
como dice tu madre, venga lo que viniere, siempre tendremos que dar
gracias.»

Así en la enfermedad como en la convalecencia y franca mejoría, se
redoblaron los mimos que a María Ignacia prodigamos todos, y por mi
parte, a más de renovar ante ella la declaración y juramento de
fidelidad que como esposo le debo, le sometí y entregué mi lícita
libertad, que tal fue el compromiso de alejarme sistemáticamente de todo
lugar donde pudiera presentárseme ocasión pecaminosa. Con ello no hago,
en realidad, gran sacrificio, porque de tal modo embarga mi voluntad el
indescifrado misterio de la sucesión, que al presente nada me solicita
fuera de mi casa, y me sorprendo de encontrarme desalentado y glacial
ante personas que el año anterior me sacaban fácilmente de quicio. Desde
mi regreso de Atienza, he visto más de una vez a Eufrasia, en su casa,
en las ajenas, en el teatro, en la calle. En nuestras primeras
entrevistas, encareció sin ironía mis virtudes, incitándome a persistir
en ellas. En Febrero último, un casual incidente nos aproximó y puso en
soledad con tan tentadoras circunstancias, que el no desmandarme habría
sido, más que honradez, santidad. Por fortuna, la presteza con que
acudió la manchega a la corrección de mi atrevimiento, nos salvó a los
dos, acreditando su virtud más que la mía. Desde entonces nos hemos
visto poco y sin ocasión de largas explicaderas. Me han dicho que en su
casa, donde politiqueaban el año anterior los disidentes de la situación
moderada, cabildean ahora los enemigos más obscuros del régimen. No sé
qué hay de verdad en esto, ni me importa.

De Virginia y Valeria debo decir que cada una tiene de novio a un
capitán\ldots{} Por extraordinario efecto de reflexión de lo femenino a
lo masculino, los dos novios me parecen un capitán solo. Ya no bromean
conmigo las dos chiquillas, ni yo, respetándome y respetándolas, me
permito jugar con ellas a los amorcitos. Sé lo que debo a la sociedad, a
los amigos y a mí propio: siento en mí la saludable invasión anímica de
la sensatez; como árbol magnífico que soy, plantado en el suelo de la
patria, me duelen las raíces al menor movimiento de mi tronco\ldots{}
Noto en mí un sentimiento nuevo, la alegría de la corrección, porque
nace entre las vanaglorias de una vida llena de ventajas y dulzuras del
orden material. En la cúspide de mi sensatez, pirámide que tiene por
base mi sólida posición, afirmo de nuevo que la renuncia que hice a
María Ignacia de mi asistencia a reuniones mundanas, no es en realidad
un sacrificio muy meritorio, pues en muchos casos no iba yo a ciertas
casas más que a medir la longitud y latitud de mi aburrimiento. Tan sólo
echo de menos la tertulia de María Buschental, cenáculo de hombres
presidido por una mujer encantadora, de sutil ingenio. Allí van mis
mejores amigos; allí se habla de lo divino y lo humano con deliciosa
libertad, y se lleva puntual cuenta y razón de las flaquezas cortesanas
que ofrecen interés por andar en ellas los poderosos, pues las flaquezas
de los pequeños a nadie interesan; allí se hace la exacta crítica de las
cosas públicas, harto más sincera que la de los periódicos, porque las
causas y móviles de los hechos, comúnmente reseñados con falaz criterio
por la Prensa, salen de las bocas vestidos y armados de la refulgente
verdad\ldots{} Espero que en cuanto rebasemos la formidable línea de la
sucesión, recabaré de mi bendita esposa que, a cambio de otras
concesiones, me dé de alta en el amenísimo conciliábulo de la calle del
Príncipe. Por hoy, me resigno a no tener más sitio de esparcimiento y
charla que el Teatro de Oriente (convertido en Congreso, mientras se
concluye la nueva \emph{Cámara de los Comunes}), aunque allí, como dice
Salamanca, tiene uno la \emph{desdicha de encontrar siempre a todas las
personas que le cargan}.

\emph{29 de Mayo}.---Pongo en conocimiento de la Posteridad un
importante suceso. Ayer estuvo en casa mi amigo Eduardo San Román con
esta comisión: «Vengo de parte del General Narváez a llevarte a su
presencia\ldots{} No te asustes: desea conocerte.» Sorpresa y confusión:
esta sube de punto cuando agrega el simpático emisario que no se trata
de concederme audiencia, por otra parte no solicitada, ni de una
entrevista ceremoniosa: será una simple presentación de confianza, por
la mañana, cuando el General, no vestido aún, o a medio vestir y quizás
tomando chocolate, recibe a sus amigos más íntimos. Francamente, no
entraba en mi cabeza que con tan primitivas formas de llaneza me llamase
y recibiese D. Ramón a mí, para él desconocido, o apenas conocido de
nombre. Llegué a creer que San Román me daba una broma; pero con tal
seriedad insistió en su mensaje, que hube de tenerlo por verídico.
Pensando que me hallaba en vísperas de una singular emergencia, me dije:
«¿Qué es esto? ¿Para qué me querrá el dueño y árbitro de los destinos de
la Nación?\ldots{} No puede ser para ofrecerme un acta en elección
parcial, que de esto se ocupa Sartorius\ldots{} Para reñirme no ha de
ser, porque en nada le ofendí, y no soy su subordinado\ldots{} ni para
darme las gracias, porque ningún servicio me debe\ldots» En fin, pronto
saldría de confusiones. Convine con Eduardo en que nos reuniríamos en
casa, por hoy, a la hora que él designara.

Por la noche, mi mujer y yo apuramos hipótesis y conjeturas para dar con
el quid de tan extraña cita, y en el giro de nuestra charla, hablamos de
mi presunto introductor San Román, en quien reconozco a uno de mis
mejores amigos. Soldado de pluma más que de espada, sus notables
escritos de Arte Militar le han valido el entorchado de plata. Es quizás
el brigadier más joven del ejército, y en política no anda ciertamente a
retaguardia: D. Ramón le ha hecho diputado por Loja, su pueblo, que es
como hacerle de la familia\ldots{} La tenaz adhesión de nuestro
pensamiento a la persona del guerrero de Arlabán, nos llevó a recordar
la carta inédita, inconcluida y sin curso del pobre Miedes, que de
Atienza trajimos y conservamos como oro en paño en recuerdo de nuestro
bondadoso y trastornado amigo.

«Mira tú---dije a María Ignacia,---que sería muy gracioso entrar yo a la
presencia de Narváez saludándole con el dictado de \emph{Buey liberal},
que según Miedes es la fórmula sintética de su carácter.

---Gracioso sería, sí\ldots{} ¡Lo que tardaría el hombre en tirarte por
las escaleras abajo!

---Como no dispusiera que me agregaran a la primera cuerda que salga
para Filipinas\ldots»

Bromas aparte, no llegué sin temor, esta mañana, a la Inspección de
Milicias, morada del General cuando es Ministro Presidente. La idea que
todos los españoles, con razón o sin ella, han formado de la fiereza del
personaje, justificaba mi vago recelo, que San Román cuidó de disipar
asegurándome que no debía temer ningún arranque iracundo, porque el
león, no tan fiero como se le pinta, sólo echa el zarpazo a los
subalternos que no cumplen su deber. Entramos, y en una estancia nada
elegante, que más bien parecía cuerpo de guardia, vi que hacían antesala
unas cinco o seis personas, algunas de las cuales conocía yo. Eran D.
Juan Gaya, Administrador de la Imprenta Nacional y Director de la
\emph{Gaceta}, mi jefe un año ha, hoy Diputado por la Seo de Urgel
(¡Cielos, apiadaos del inocente Cuadrado, mi compañero de oficina!); el
corpulentísimo D. José María Mora, Diputado por un distrito de Alicante
y oficial en Gobernación, y el de tenebroso entrecejo y desapacible
rostro Don Claudio Moyano, Rector de la Universidad. Además vi a uno que
me pareció periodista, cara que conozco mucho, mas el nombre se me ha
ido de la memoria\ldots{} Mientras yo saludaba a mi antiguo jefe en la
\emph{Gaceta}, y le proponía que trabajásemos juntos para traer de su
destierro al sin ventura Cuadrado, desapareció Eduardo San Román. Al
poco rato le vi volver con un ayudante, y ambos me llevaron afuera, como
quien desanda lo andado, y luego me condujeron por un pasillo con
dobleces que no parecía sino un rompecabezas. Al término de esta
caminata, entramos en un aposento grande, todo claridad, donde lo
primero que vi ¡Dios me valga!, fue la propia persona del \emph{Túrdulo}
D. Ramón Narváez en mangas de camisa. Entrar yo por aquella puerta y
salir él de otra frontera, con vivo paso, mirar fiero y arranque
impetuoso, que me dio la impresión de un toro saliendo del toril, fue
todo uno. Quedeme parado a pocos pasos de la puerta sin saber qué hacer,
ni a dónde volverme, ni a quién saludar. Por un momento dudé que fuera
el Duque de Valencia quien de tal modo me recibía. Mis introductores, no
menos perplejos que yo, se pararon también en firme junto a mí, a punto
que el General, en medio de la estancia, gritaba como quien da la voz de
mando en lo más comprometido de una batalla: «¡Bodegaaa!

---Mi General---dijo el ayudante,---yo le llamaré.

---En el pasillo se cruzó con nosotros cuando entrábamos,» balbució San
Román, señalando al ayudante la dirección que tomar debía.

Narváez, gritando nuevamente «¡Bodega!» reforzaba su exclamación con el
repique de una campanilla que cogió de la mesa y agitaba en su mano.
Después se volvió hacia mí, y secamente, sin dar espacio al saludo que
inicié, me dijo: «Dispense usted, \emph{pollo.»} Al poco rato, como si
la presencia de un extraño calmase su furia, aplacó los gritos, y no
hacía más que sacudir la campana, diciendo por lo bajo: «Este Bodega me
va a quitar a mí la vida.» De pronto entró el ayudante, y tras él un
criado como de cincuenta años con un servicio de chocolate. Lo mismo fue
verlo Narváez que le tiró la campanilla con toda la fuerza de su brazo,
diciendo: «Ahora te lo tomas tú, arrastrado\ldots{} que ya con tu
cachaza me has quitado la gana\ldots{} ¡Si me tienes podrida la
paciencia!\ldots{} Que te lo lleves, te digo\ldots{} ¡Qué no lo tomo,
ea, que no lo tomo!»

Cayó la campanilla a los pies del criado, el cual, imperturbable, como
si creyera en conciencia que de su enrabiscado señor no debiera hacer
más caso que de un niño, dio con el pie al proyectil que este le había
lanzado, y siguió su camino rodeando la pieza hasta dejar el servicio en
una mesa próxima a la ventana. Yo había oído hablar del famoso Bodega,
del viejo soldado, compañero y servidor del General en la guerra, y
ahora su ayuda de cámara y mayordomo; pero no le había visto nunca.
Encontrele alguna semejanza con el gran Miedes, la cual, si muy vaga en
la fisonomía, más acentuada en la traza y estatura, salva la diferencia
de edad, era exactísima en los pies, grandes, juanetudos, como los del
sabio celtíbero, marcando bajo el paño de los zapatos bultos como
nueces. Pues el fiel servidor, mudo y flemático, sin precipitarse en sus
movimientos, luego que dejó el chocolate en la mesa, cogió el chaleco, y
alzándolo en ambas manos, hizo un movimiento semejante al del
banderillero cuando cita al toro y le muestra los palillos que ha de
clavarle. Narváez arrojó sobre su asistente una mirada de indignación, y
llegándose a él dio media vuelta y se dejó meter los brazos por los
agujeros de aquella prenda. Luego se abrochó de prisa, y antes que
Bodega trajera la levita le echó otra rociada: «Te digo que te lleves
ese menjurje. He dicho que no lo tomo ya. Llévatelo, o te lo tiro a la
cabeza.» Bodega, sin la menor alteración en su rostro, que parecía de
palo, puso a su amo la levita; el General, volviéndole la espalda, se la
ajustó con un nervioso estirón del paño sobre la cintura; luego palpó y
aseguró su peluquín, que con los berrinches parecía desviarse un poco.
Retirose Bodega con la tranquilidad del justo, sin cuidarse de obedecer
a su señor en lo de llevarse el desayuno, y el Duque, al verle salir, le
flechó de nuevo con una mirada de odio; después dirigió otra de desdén
al chocolate; por último, volviéndose a mí, me señaló un sofá, a punto
que él también se sentaba, y me dijo: «Dispense, \emph{pollo}, que le
reciba con esta confianza\ldots{} Voy a decirle con qué objeto me he
tomado la libertad de llamarle\ldots{}

\hypertarget{xiv}{%
\chapter{XIV}\label{xiv}}

---Mi General---le respondí,---estoy siempre a sus órdenes. No podía
usted hacerme honor más grande que tratarme con esta confianza\ldots{}

---Pues, verá\ldots{}

---Tome usted su chocolate, mi General---le dije creyendo corresponder a
su franqueza.---Por mí no se prive\ldots»

Me interrumpió con un gesto impaciente que traduje de este modo: «No se
ocupe usted de lo que no le importa. Yo tomaré o no tomaré el chocolate
conforme a mi santa voluntad; usted oiga y calle.» Así lo hice. No sin
grande estupor oí estas palabras, que reproduzco suprimiendo el ligero
ceceo andaluz con que el Dictador las pronunciaba: «Pues quería decir a
usted lo siguiente: en su casa, en la casa de los señores De Emparán se
conspira de un modo descarado contra mí\ldots{} No, no me lo niegue. Con
usted no va nada. Tengo de usted la mejor idea: ya sé que es sensato,
muy sensato, y que entre las ideas del Marqués de Beramendi y las de su
suegro\ldots{} hay un abismo\ldots{} Lo que no quita que usted aparente
amoldarse\ldots{} Naturalmente, es esposo de su hija\ldots{} ¡Si me hago
cargo!\ldots{} Es posible también que delante del yerno no se permitan
decir todo lo que sienten, ni dejar traslucir sus intenciones. Yo lo sé
todo, y si no lo sé todo, sé mucho, lo bastante para no dejarme
sorprender. Mi objeto al llamarle no es pedirle que me cuente lo que se
habla en su casa. Ni yo acostumbro apelar a esos medios, ni usted, que
es un joven pundonoroso, de gran talento, según me dicen, se había de
prestar a un espionaje de tal naturaleza\ldots{} No, no: mi objeto es
tan sólo decirle que haga entender a su familia que Narváez no está
ignorante de lo que se trama contra él, y que se halla dispuesto a
\emph{meter mano} a todo el que perturbe, sin distinción de pobres y
ricos. Es gran injusticia mandar a Filipinas a tanto infeliz
descamisado, y dejar aquí a los revoltosos de buena posición, que pelean
contra \emph{lo existente}\ldots{} con armas que no son el trabuco
naranjero, y se hacen fuertes en barricadas\ldots{} que no son las de
las calles. Aquí donde usted me ve, soy yo más liberal que nadie, y si
me apuran, más demócrata que la Virgen Democracia. Ni temo a los de
abajo ni adulo a los de arriba\ldots{} Si los que pintan el diablo en la
casa de Emparán son \emph{carlinos}, enhorabuena: que salgan al campo,
que den la cara. Yo he visto de cerca las caras de Zumalacárregui, de
González Moreno, de Don Basilio, de otros muchos guerreros muy
respetables, y no me dan asco. Ellos luchaban en su campo, yo en el mío;
ellos se mataban por su Rey, yo por mi Reina. Éramos rivales nobles.
Ganamos nosotros la partida. Por zancas o barrancas, quedaron los
facciosos debajo; nosotros encima\ldots{} Pues ahora los convenidos de
Vergara, y los clérigos de capa corta que allí tuvieron su desengaño,
quieren suplantarnos y abolir el Régimen, y traernos el carlismo sin D.
Carlos, o el absolutismo con Isabel, y esto no hemos de tolerarlo,
¡carape!\ldots{} Como no hemos de consentir que los que tronaron contra
la desamortización, sean ahora los que quieran echar abajo \emph{lo
existente}\ldots{} No será tan malo el árbol cuando a su sombra hicieron
sus pacotillas estos ricachones que ahora se gastan el dinero en
escapularios, y que me acusan de que no miro por la Religión\ldots{}
Hable usted de esto con su señor papá político, y con otros que en pocos
años se han llenado de millones. Si es tan malo el Régimen, que se lo
cuenten a los que por ese mismo sistema político, ¡ahí duele! fueron
\emph{Comisionados del crédito público}, y se encargaron de recoger el
papel-moneda de los conventos\ldots{} ¿Dónde está ese papel? Yo no digo
nada: hable usted con los que dicen que se ha convertido en ladrillos y
estos en casas\ldots»

Aprovechando el primer descanso que tomó el orador, dije que si en mi
casa se hablaba mal del Gobierno, común achaque de toda casa de Madrid,
cualquiera que fuese la procedencia de sus ladrillos, no debía ello
tomarse como efectiva conjura, sino como desahogo natural de las almas
españolas; a lo que me contestó el Duque con un suspiro que de su pecho
salía como avergonzado, por no ser aquel pecho de los que albergan la
resignación, o el sentimiento de una radical impotencia contra fatales
obstáculos. Después miró un instante al suelo, y me dijo que aunque la
intriga no tuviese su principal centro en mi casa, allí debía él dar un
toque de atención en esta forma: «Cuidado, caballeros, que tengo abierto
el registro para Filipinas\ldots» En esto apareció de nuevo Bodega, y su
amo le interpeló en el tono más suave: «Bodega, hijo, ¿qué haces que no
te llevas ese chocolate maldito? No lo tomo\ldots{} Oye otra cosa:
sírvenos el almuerzo a las doce en punto. Este señor almuerza hoy
conmigo.» Cuando yo le daba las gracias por tanta fineza, entró el
ayudante, al cual preguntó su jefe si había más personas en la antesala.
«Acaba de entrar D. Pedro Egaña; hace un rato llegaron el Sr.~Sagasti y
D. Pascual Madoz.

---Que pasen a esa sala los que aguardaban y los recién venidos: los
despacharé a todos de una estocada---dijo el Duque abriendo la puerta
que a la estancia próxima conducía.---Bodega, no hay prisa para el
almuerzo, porque hoy no tengo que ir a Palacio: de aquí me iré al
Senado.»

Y con severidad tutelar, tranquilo y apacible, como quien ejerce
paternalmente la autoridad doméstica, el gran Bodega recogió el
servicio, diciendo: «Buena memoria nos dé Dios. Si no va mi General a
Palacio, bien sabe que le espera en su casa el Sr.~D. Luis Mayans. ¿No
quedaron en eso?

---¡Oh! sí: tienes razón\ldots{} Almorzaremos a las doce en punto.»

Pasando el Duque a la sala de audiencias, quedamos allí el ayudante y yo
con San Román, el cual, mientras hablamos Narváez y yo lo que referido
queda, había permanecido en discreto apartamiento, leyendo no sé si
\emph{La España} o \emph{El Heraldo}, a la claridad del balcón. Luego
que estuvimos solos, vino Eduardo a mí para darme instrucciones acerca
de la actitud que debo observar ante el General en las incidencias
probables de un largo coloquio. «Si te trata con confianza, guárdate
mucho de hacer lo mismo con él; si te da alguna broma, aguántala sin que
se te pase por el magín la idea de devolvérsela, aun siendo de las más
inocentes. No tolera confianzas de nadie, como no sea de Bodega, y en
cuanto a bromas, no ha nacido todavía quien se las dé. Es un hombre
bonísimo, pero de un amor propio que no le cabe en el alma. Admite que
se le contradiga en ideas; pero no quiere oír cosa alguna por donde a él
se le figure que queda en ridículo a sus propios ojos. Nada de chistes,
Pepe, alusivos a lo que ha hecho, o pueda hacer y acontecer. Cuanto al
General se refiera, sea dicho en el tono más serio.»

Terció el simpático ayudante en la conversación para añadir nuevas
advertencias a las expresadas por San Román, lo que yo agradecí mucho,
porque con tales maestros no había medio de desbarrar. «Fíjese usted
también en esto: de las caricaturas que le sacan en los periódicos
callejeros, no tiene usted que hacer mención ni aun para reprobarlas, ni
tampoco hablar de los papeles satíricos, ni reírles las gracias. Los
muñecos y las sátiras más o menos chistosas o indecentes, le sacan de
quicio\ldots{} Dé la prensa en general, aun de la moderada, hable usted
con poca estima.

---Es un gran corazón y una gran inteligencia---dijo San Román;---pero
inteligencia y corazón no se manifiestan más que con arranques,
prontitudes, explosiones. Si mantuviera sus facultades en un medio
constante de potencia afectiva y reflexiva, no habría hombre de Estado
que se le igualara.

---Es todo inspiración, todo inspiración.

---Lanza el gran bufido, y cuanto mayor sea este, más pronto vuelve el
hombre al estado de calma y prudencia. Créelo: si a todos los que ha
mandado fusilar, pudiera resucitarlos, lo haría de buena gana\ldots{} Si
es duro en los hechos, en la palabra suele ser muy inconveniente\ldots{}
pero su furor pasa pronto.

---Le hemos visto pedir perdón a muchos que le oyeron cosas terribles,
cogidos de las solapas.

---Las personas a quienes más ha protegido y protege, digo yo que son
las hechuras del arrepentimiento. Recibieron algún apabullo, les salpicó
a la cara el espumarajo de la ira del león\ldots{} Pero luego ha venido
el león mismo a limpiarlo, concluyendo por colmar de beneficios al
ofendido.

---La principal regla de conducta es no tomarse con él ni la más ligera
confianza.

---Una mañana estuvo aquí un diputado andaluz, que es hombre
graciosísimo. Fue en las Cortes pasadas. De su nombre no me acuerdo; de
su cara sí: alto, moreno, con patillas de \emph{boca de jacha}, dientes
muy blancos, y un decir ameno, con chiste en cada frase, y los ademanes
tan sueltos y desahogados que ellos bastaran para hacer reír. Narváez se
divirtió oyéndole contar cosas de la tierra: aquel día ceceaba como en
su mocedad. El pobre granadino, viendo a su paisano tan gozoso y
bromista, se fue del seguro y cometió la pifia de ponerle la mano en el
hombro. Sentir la mano del andaluz en su hombro fue para D. Ramón como
sentir la picadura de una víbora. Volviose, cogió con violencia la
insolente mano, y echando lumbre por los ojos, le dio un fuerte estirón
hacia abajo, diciendo: «¡Esa mano en los calzones!» Quedose el otro de
una pieza. No volvió a soltar chistes, ni D. Ramón se los hubiera reído
aunque a chorros los echara. Pasado algún tiempo, el tal se trocó de
amigo en furioso enemigo de Narváez, y escribió sus chirigotas en
\emph{La Postdata}\ldots{} Al fin se hizo progresista: ha estado en un
tris que le mandemos a Filipinas.»

Antes que San Román concluyera, oímos la voz del General en la sala
próxima. Reñía con D. Pedro Egaña y con D. Pascual Madoz, que también es
hombre de malas pulgas. Luego supimos por el ayudante que los Sres.
Gaya, Mora, Sagasti y Moyano se habían retirado después de oír alguna
palabra, ni agria ni dulce, del \emph{Espadón}. Este toreaba por lo fino
a D. Pedro Egaña, que venía con pretensiones vascongadas, y a Don
Pascual Madoz, que solicitaba privilegios para Cataluña. Era un caso de
incompatibilidad irreductible entre los intereses catalanes y los
vascos. Llamado por el Duque, pasó el ayudante a la sala de audiencias
para hacerse cargo de todo el papelorio que dejaban los dos pedigüeños
de gollerías, y al abrirse la puerta oímos a Narváez que gritaba: «¿Pero
esto es España o la ermita de \emph{San Jarando} que hay en mi tierra,
donde cada sacristán no pide más que para su santico? Ea, caballeros, yo
estoy aquí para mirar por el Padre Eterno, que es la Nación, y no por
los santos catalanes o vascongados\ldots» Les despidió con buena sombra,
y si Egaña partió cejijunto, conteniendo su enfado dentro de la
cortesía, D. Pascual, que es muy nervioso, chillón, rudo, francote, como
cuarterón de catalán y aragonés, y de aragonés y navarro, salió con la
peluca bermeja un tanto descompuesta y erizada, diciendo: «General, es
usted atroz, y a este paso iremos\ldots{} a donde no queremos ir.»

Terminadas las audiencias, creímos que nadie quedaba en la sala; pero el
periodista que vi al entrar, y que según dicho del ayudante se había
retirado, apareció de nuevo como un duende, no sé si por secreta
puertecilla o surgiendo de los pliegues de un cortinón. Con forzada
sonrisa y pruritos de ligereza que eran disimulo y atenuantes de su
miedo, adelantose en seguimiento del General que a nuestro lado volvía.
Infeliz esclavo de las duras necesidades de su oficio, se arriesgaba,
con peligro de la existencia, a quitarle motas o pulgas al león.
Volviose este con el movimiento rápido que a sus arranques de ira o de
generosidad precedía, y tocado por suerte de la segunda más que de la
primera, dijo al intruso en el tono con que imitaba la paciencia: «Pero,
condenado Santanita, ¿cuándo concluirá usted de freírme la sangre?

---Mi General---dijo con ceceo andaluz el llamado Santana,
tranquilizándose,---es usted más bueno que el pan y más dadivoso que San
Antonio bendito. ¿Qué le cuesta decirme con palabra y media lo que está
pidiendo con tanta necesidad mi \emph{Carta autógrafa} de esta noche?

---¡Si no hay nada, si no tengo nada que decirle!

---Mi General, yo le voy conociendo ya, y sé que cuando más regatea más
da, y que si al principio le niega a uno hasta la sal del bautismo,
luego le entrega su corazón, ese corazón más grande que la Puerta de
Alcalá\ldots{}

---Basta, Santana\ldots---replicó D. Ramón, en plena expresión de
benevolencia.---Ahora no puedo entretenerme. Véngase esta noche antes de
comer, a la salida del Congreso\ldots{} no, no: de diez a once, y
hablaremos.

---¿Pero no podré llevarme ahora un par de rengloncitos, como quien
dice, nada?\ldots{} La expedición ha llegado a Gaeta. ¿Se sabe ya si
Córdova ha conferenciado con el Papa?\ldots{} ¿Cuándo empezamos las
operaciones?\ldots{} \emph{¿Atacaremos} a Garibaldi antes que lleguen
los refuerzos?\ldots{}

---Que vuelva esta noche, ¡jinojo!---dijo Narváez como con ganas de
enfadarse \emph{una chispita}, pues con la mayor presteza pasaba de un
extremo a otro de la gama humoral.---Esta noche, y no moler, amigo. Ya
sabe que le quiero bien, por trabajador y honrado, y que le distingo
entre tanto holgazán trapisondista.

---A la orden, mi General---murmuró el otro despidiéndose con militar
saludo y saliendo como un cohete.

\hypertarget{xv}{%
\chapter{XV}\label{xv}}

---Este Santana me gusta---nos dijo Narváez cuando nos sentábamos a la
mesa.---Es hombre de gran mérito; es un inventor que adivina alguna cosa
que no se ve y que él quiere descubrir; confía en sí mismo; no tiene
capital: él lo creará con cuatro pedazos de papel y una piedra
litográfica\ldots{} y con la paciencia de todo el mundo, ¡carape!, pues
el maldito pone a contribución a cuantos podemos darle alguna noticia, y
hasta que no aflojamos la mosca no nos deja en paz\ldots{} Pero con eso
y con todo, este hombre es una voluntad, y merece que se le
proteja\ldots{} Le conozco desde que empezó. Me ha dado algunas
jaquecas\ldots»

Luego me contó San Román este pasaje delicioso de las relaciones de
Narváez con Santana. «En los primeros días de la \emph{Autógrafa}, se le
fue la mano al periodista apreciando ciertos actos del General. Este, al
leer el periódico bufaba como un gato. «Si encuentro en la calle a ese
catatintas, le deshago---me dijo. Y una tarde quiso la mala suerte del
periodista que, viniendo él por la calle Mayor fuésemos por la misma
calle y acera, en dirección contraria, el General y yo\ldots{} Santana,
con ojo de lince, le vio desde lejos y se pasó a la acera de Platerías;
Narváez, que también tiene buen ojo, le sorprendió el movimiento y se
fue a él como un ave de presa, y antes que pudiera escabullirse le
agarró por las solapas y\ldots{} yo no sé las perrerías que le dijo. El
otro daba sus excusas\ldots{} Realmente, el agravio era insignificante,
de esos que se hacen un día y otro a los hombres políticos,
censurándoles con más o menos equidad sin lastimar su honra. Seguimos
calle adelante, sin que yo me permitiese hacerle ninguna observación
sobre la aspereza de su genio, porque le vi sofocadísimo, y tardaba más
que de costumbre en recobrar la calma. Por la noche, aquí, le noté
bastante aplanado, taciturno, contestando poco y mal a los hombres
políticos que vinieron a verle. Hasta con su íntimo amigo, el granadino
D. Miguel Roda, estuvo muy avinagrado. A la mañana siguiente le encontré
en la misma disposición de espíritu; a Bodega tan pronto le llenaba de
improperios como le llamaba hijo\ldots{} Bien se veía que un pesar le
agobiaba; pero como es hombre de arranques, y los de sinceridad son
quizás los más hermosos que tiene, así como no se le pudre en el cuerpo
ningún resquemor por agravio recibido, tampoco se le quedan dentro las
espinillas de los disparates que hace. Soltando un terno volviose a mí
de repente y me dijo: «¡Qué me traigan a ese Santana!\ldots{} Eduardito,
hazme el favor de traérmele. Ayer, ya lo viste, le atropellé
estúpidamente\ldots{} No había motivo\ldots{} Estuve muy duro\ldots{}
¡Un hombre que se gana la vida sin pedir a nadie más que
noticias!\ldots{} Este le mete a uno los dedos en la boca, jamás en los
bolsillos. Quiero hacer algo por él, y demostrarle que Narváez no es
rencoroso. Dispondré que se suscriban a la \emph{Carta autógrafa} todas
las Direcciones Generales, a más de los Ministerios\ldots{} y se
recomendará la suscripción a todos los jefes políticos y a los cuerpos
del Ejército\ldots» Con que ya ves si el hombre es de buen natural. Esto
pasó tal como te lo cuento.» Era en verdad un rasgo que descubría la
integridad del carácter, una línea que era toda la figura.

Durante el almuerzo, del que participaron también San Román y el
ayudante, nada nos dijo el Duque digno de que yo lo mencione. El hábito
del gobierno le había curado de sus resabios expansivos, y comúnmente,
como alguna cuestión picante no excitara su nativa franqueza, nada decía
que debiera reservarse. De los diversos asuntos políticos o
internacionales que estaban, como suele decirse, sobre el tapete, apenas
habló; ocupose más de nosotros que de sí mismo, pidiéndonos noticia de
la sociedad que frecuentamos, y distinguiéndome a mí con sus finezas. No
sé si debo contar como tal la insistencia en darme la denominación de
\emph{pollo}, que me pareció de notoria impropiedad, pues aunque soy
joven efectivo, por razón de mi estado y circunstancias no pertenezco a
la juventud suelta y de cascos ligeros designada vulgarmente con aquel
término gallináceo. Este se aplica hoy sin ton ni son, y significa
frivolidad, corbatas de colorines, primeros pasos en cualquier carrera;
significa infatigabilidad en el baile, lanzándose a la moderna
\emph{polka} con vértigo y furor, audacia en los amores, atreviéndose
con las damas de alto copete, alegría decidora, jactancia de los
triunfos cuando los hay, resignación en las calabazas; significa el
desprecio del romanticismo y la repugnancia de venenos y puñales. El
llamar \emph{pollos} a los muchachos es uso moderno, y data del 46; lo
inventó, que invento es la novísima aplicación de las cosas, así
vocablos como fuerzas naturales, una dama muy linda, en una reunión
aristocrática, no sé si en casa de Montúfar o de Montijo, o de Santa
Cruz (averígüenlo los eruditos). Oía esta señora las arrebatadas
declaraciones de un jovenzuelo tan elegante como atrevido, y aunque las
oía con agrado, hubo de contestarlas con una negativa graciosa. El
mancebo, que no era bastante fino para guardarse el \emph{no} sin más
explicaciones, pidió a la dama razón de su desvío, y ella, tomando el
brazo de un señor maduro (cuarenta años), le dijo: «¿Por qué? Porque es
usted todavía \emph{demasiado pollo.»} La frase fue de las que caen en
terreno fértil: hizo fortuna, sin duda como flor nacida en tales labios,
y no tardó en extenderse rápidamente al lenguaje común. Bautizados por
la hermosa dama, nombre de \emph{pollos} tuvieron ya para \emph{in
æternum} todos los jovencitos bien vestidos y arrogantes que buscan
dotes o pretenden los favores de mujeres hechas, más o menos casadas,
bien o mal avenidas con sus esposos. Ha llegado a tener un uso constante
y amaneradísimo la palabreja: a mí me llamaron \emph{pollo} desde que
vine de Italia hasta que me casé. Después del cambio radical de mi
posición, nadie me ha llamado así más que Narváez, del cual me ha dicho
San Román que aplica el mote a muchos que ya gallean. Para él son
todavía \emph{pollos} Cumbres Altas y Pepe Casasola.

Otro toque del General. A mitad del almuerzo noté que no le parecía
bastante bueno el vino que bebíamos. «Tráenos el borgoña del año 4,»
dijo a Bodega que hacía de maestresala, tan imperturbable, metódico y
puntual en estas funciones como en todas las demás de su omnímodo
servicio. Sin mirar a su amo, ni alterar ningún rasgo de su fisonomía,
que era siempre de palo, Bodega contestó: «El borgoña se guarda para las
comidas de etiqueta.» Yo temblé; no me atreví a mirar al Duque, creí que
ya volaba un plato desde la mano del anfitrión a la cabeza del criado;
pero no cruzó los aires más que esta frase con que el General nos
explicaba su mansedumbre, después de mirar compasivamente al gran
Bodega: «A este bruto hay que matarlo o dejarlo.»

Servido el café, mandó poner junto al balcón una mesita, y me hizo señas
de que allí nos apartáramos para tomarlo juntos y solos. «Vaya---pensé
yo,---ahora me dirá lo que resta, pues ya no tengo duda de que hay
segunda parte.» En efecto: no tardó el hombre en explicarse. Ved aquí
cómo: «Pues hay conspiración, \emph{pollo}, por más que usted no se
entere bien de lo que se habla en su casa. ¿No va usted por la de
Socobio, Saturnino? ¿No frecuenta usted la de Socobio, Serafín, que hoy
vive en las habitaciones altas de Palacio?» Díjele que muy rara vez voy
yo a esas casas, y siempre de visita, acompañado de mi mujer, a lo que
él replicó: «Pues en este mal negocio anda, como portadora de recaditos
y de instrucciones, una señora que\ldots{} no es ofensa,
\emph{pollo}\ldots{} una señora que, según públicos rumores, ha tenido y
tiene amistades íntimas con usted.» Al oír esto me turbé un poco. Si se
refería el General a Eufrasia, podía ser verdad que esta señora
conspirase; mas no lo es que tenga conmigo las concomitancias \emph{de
hecho} que el vulgo supone.

«¿Qué señora es esa, mi General? Creo que a usted le han informado mal.

---La de Terry, hijo\ldots{} ¡Si es más conocida que la ruda!\ldots{}
Pero ¿se hace usted el novicio, o cree que yo lo soy?\ldots{}

---Yo le juro que\ldots{}

---¿Pero es de veras?\ldots{} Vamos, ahora que es usted hombre de
arraigo no quiere ponerse a la altura de su reputación.»

Le conté ingenuamente el caso, mi amor por Eufrasia, mis largas esperas,
y por fin, mi retirada honesta al campo de la fidelidad conyugal. No me
creía. Riendo me dijo: «¡Pamplinoso!\ldots{} Pues quien lleva el alza y
baja de estos enredos me había asegurado que no era usted solo\ldots{}
porque esa no está por exclusivismos, ¿sabe usted?\ldots{} Es de las
\emph{de ancha base}, como el Ministerio que quiere Pacheco, donde
entran todos\ldots{} Otra: también oí que se jacta de haber hecho la
boda de usted.

---No es cierto, mi General---respondí, molesto de tener que dar tales
explicaciones.

---Ahora resulta que este \emph{pollo} cándido y honesto no se entera de
nada. ¿No sabe tampoco que Eufrasia y una tal Rafaelita, hija de uno que
fue jefe político en tiempo de Espartero, son los correos de gabinete
que llevan a la casa de Socobio y al palacio de usted las órdenes de
otra casa más grande?

---No lo sabía, mi General.

---¿Y también ignora que esta y otras andan ahora continuamente entre
curas?

---He observado en esa, como en otras amigas mías, un furor de moda
religiosa, y demasiada querencia de los altares, sacristías y
confesonarios.

---La manchega y su editor responsable, Socobio, confiesan ahora con el
Padre Fulgencio.

---Sé que el escolapio es muy amigo de esa familia.

---Pues siento mucho que no se haya usted arreglado con esa señora, pues
de usted pensaba valerme para hacer entender, tanto a \emph{la}
Eufrasia, como a \emph{la} Rafaela\ldots»

Detúvose y lanzó un terno de los garrafales acompañado del destello
iracundo de sus ojos, y seguido de esta explosión: «Como me llamo
Narváez, que no quisiera morirme sin coger un barco viejo, de los más
viejos que tenemos en los arsenales, y llenarlo de estas beatas\ldots{}
y mandarlo bien abarrotado de ellas\ldots{} ¿Qué Canarias ni qué
Filipinas?\ldots¡a las islas Marianas!»

Dando un golpetazo en la mesilla, levantose repitiendo: «¡A las islas
Marianas!» Recorrió una y otra vez la estancia, corajudo, apretando las
mandíbulas y mascando el cigarro, y sus labios escupían el nombre de
aquel remoto archipiélago: «Marianas\ldots{} Islas Marianas\ldots»

Pasado lo más vivo del arrechucho, volvió a mi lado y prosiguió así:
«¿Tienen algo que echarme en cara como jefe de un Gobierno que está
obligado, como todos, a mirar por los intereses eclesiásticos? Hablo de
intereses, porque de Fe y de Principios no hay que hablar, que católicos
el que más y el que menos somos todos aquí. ¿No he mandado un ejército a
Italia para restaurar a Pío IX en sus Estados, que le birlaron los
demagogos de Roma? ¿No estoy dispuesto, luego que el Papa recobre su
Silla y en ella esté bien seguro, a tratar con él del nuevo Concordato,
cediendo en todo, y haciéndolo a gusto de nuestras reverendas beatas, y
de nuestros venerables obispos, y de nuestros convenidos de Vergara, y
de nuestros apreciabilísimos compradores de bienes del Clero?\ldots{} No
me digan a mí que estos quieren el Régimen: en esa intriga no hay más
que Carlismo, Montemolinismo\ldots{} Parece que aquí todos están
locos\ldots{} locos los de abajo, locos los de arriba y los de más
arriba\ldots{} Créalo usted: a veces, metido yo en mí mismo, me
pregunto: ¿Pero seré yo solo el cuerdo entre tanto tocado, y mi papel
aquí es el de rector de un manicomio?\ldots{} ¡España y los españoles!
¡Vaya una tropa, compadre! Aquí, el Gobierno no halla día seguro; aquí
es imposible acostarse sin pensar: ¿qué absurdo, qué disparate nos caerá
mañana? Y se da usted a discurrir cosas raras, y nunca acierta. Mil
veces me digo yo: ¿tendrán razón los \emph{anárquicos?} ¡Porque mire
usted que tenemos cosas, carape! El que inventó el llamar \emph{cosas de
España} a todos los desatinos que da de sí esta Nación, ya supo lo que
decía\ldots{} Y aquí no se puede gobernar porque nadie está en su
puesto, nadie en su obligación y en su papel, sino todo el mundo en el
papel de los demás. Como que hay quien conspira contra sí mismo, sí, no
lo dude usted, quien se entretiene en destruir su propia casa\ldots{}
labrada, Dios sabe cómo, con esfuerzos\ldots{} que me río yo\ldots! ¡Ay,
\emph{pollo!} usted no es militar, usted no ha hecho la guerra,
peleándose con otros españoles por \emph{un sí} y \emph{un no}; usted no
se ha metido hasta la cintura en ríos de sangre. ¿Y todo para qué? Para
que, a la vuelta de algunos años de lucha y de otros tantos de celebrar
la victoria con himnos y luminarias, nos encontremos como el primer
día\ldots{} ni más ni menos que el primer día, creyendo, como antes se
creyó, que puede venir el Zancarrón, y que aquí no ha pasado
nada\ldots{} Lo que digo: todos locos\ldots»

Comprendí que el General, en esta familiar y quizás indiscreta expansión
de su ánimo, sólo mostraba una mínima parte de su pensamiento. Oyéndole
por primera vez en mi vida, parecíame ver en todo su desarrollo la
procesión que le andaba por dentro. Acordeme de un concepto enigmático
de Miedes, que así dice con enrevesado estilo: «Gobernáis atado de pies
y manos, con ligaduras palatinas, y os estorba el paso y el gesto la
polvorienta madeja de supersticiones, o de místicos escrúpulos que
descienden de la altura como telarañas de los tiempos\ldots» Esta
monserga del sabio atenzano, que copio de memoria sin responder de la
exactitud de su fraseología, ya no me parece tan estrafalaria.

«Dispénseme usted, \emph{pollo}, que le haya molestado---me dijo
después.---Y admitiendo que su dominio sobre esa viborilla de la Socobio
no es como creí, bien podrá valerse de algún medio, como su pretendiente
y adorador que fue, para persuadirla de que ella y su amiga la Milagro
corren el riesgo de salir un día codo con codo entre guardias
civiles\ldots{} No es broma, no\ldots{} Yo soy capaz de eso\ldots{} Que
me busquen el genio y verán\ldots{} Las contemplaciones tienen un
límite. O gobierno como se debe gobernar, o me voy a mi casa. Tener fama
de duro y no serlo es gran tontería. Exigirme que lleve a todo el mundo
derecho, ir yo más derecho que nadie, y que se me tuerzan los que a
todos deben darnos ejemplo, es fuerte cosa\ldots» Algo más entre dientes
dijo que no pude entender. Hállase, sin duda, estos días atormentado por
la tenaz aprensión de que no le permiten desplegar alguno de sus
capitales atributos. O no le dejan ser \emph{thur}, que es como decir
\emph{buey} (fuerte), o no le dejan ser \emph{duluth} (liberal), o le
estorban sistemáticamente para dar al mundo la feliz combinación de
ambas cualidades. Saco de la entrevista la impresión de que es un hombre
de tanta voluntad como inteligencia; pero le falta el resorte que hace
mover concertadamente estas dos preciosas y fundamentales piezas del
mecanismo anímico.

¿Y cómo puedo yo explicarme que viéndome aquel día D. Ramón por primera
vez, dejara traslucir ante mí una parte, siquier pequeña, de sus
amarguras políticas? Lo explico y razono por mi insignificancia, porque
nunca fue, según mil veces oí, tan hábil en disimular sus agravios como
expresivo en arrojarlos a la cara del primero que le sale. Tratando
conmigo de un negocio de espionaje, sin quererlo, abandonándose a la
sinceridad, se le fue un poco la mano, y como el velo que tapaba el
asunto privado estaba unido por invisible alfiler al velo del público
asunto, vi más de lo que el General quería que viese\ldots{} Si no
hubiera nombrado al Padre Fulgencio, nuestra conversación no habría
salido de los términos de la gacetilla; pero en un descuido de su boca
andaluza, movida siempre de la imaginación y harto abundante en amarga
saliva, escupió al fraile (a quien sin duda no podía tragar), y desde
aquel momento lo que sólo había sido gacetilla fue Historia\ldots{}
Historia no fría y colada como la que pasa a los libros, sino viva y
caliente como la sangre de nuestras venas.

\hypertarget{xvi}{%
\chapter{XVI}\label{xvi}}

\emph{31 de Mayo}.---Asistido de mi excelente memoria pude contarle a
María Ignacia los varios incidentes y dichos de mi conferencia con
Narváez. No se contuvo mi mujer en el asombro que tan interesante visita
debía de causarle, sino que se divirtió grandemente oyéndome referir los
pasajes cómicos, y se rió con ellos como en la representación de un
gracioso sainete. «Por lo que cuentas---me dijo,---pienso, como tú, que
le falta un resorte, y es lástima que un hombre de tan buenas prendas no
las tenga completas y bien ordenadas. Pero se me ocurre una cosa, Pepe.
Dios le negó a D. Ramón el resorte o clavija para concertar la voluntad
con la inteligencia; pero le ha concedido a Bodega, que viene a ser como
clavija suplente, que hace las veces de la que falta. Me parece a mí que
España estaría gobernada con perfección si el Duque fuera ejecutor de lo
que pensara y dispusiese el Bodega\ldots{} ¿No crees tú lo mismo?»

Hablamos aquella noche y al siguiente día de lo que Narváez llamaba
conspiración en casa de Emparán, y convinimos en que, si no formal
conjura, hay un exceso de comidillas que pueden ocasionar algún
disgusto. Me ha dicho Ignacia que delante de ella suspenden la
conversación o varían de tema. Como en mi presencia no se habla tampoco
de Narváez y sus Ministros, resultamos mi mujer y yo en una especie de
aislamiento político dentro de la familia. Don Feliciano, en puridad,
parece curarse poco de las hablillas de sus amigotes, o no les da
importancia real, como hombre que llegado al colmo de sus ambiciones,
bien cubierto el riñón, vive persuadido de que con unos y con otros
siempre ha de estar a flote. Que personalmente no patrocina aventuras,
bien a la vista está. Es absolutista furibundo, cimentado en el pedernal
de la religión, más que por la pura fe, por la tenaz creencia de que las
artes de Gobierno se derivan del dogma, y de que la potestad civil y la
divina son dos brazos de un solo cuerpo. A pesar de esto, no se lleva
mal con \emph{lo existente}, ni apetece variaciones que podrían traernos
un estado peor. Su gran riqueza es la consejera de su inestabilidad, y
le inspira el prudente sistema de poner toda cuestión política en manos
de Dios. «A lo que el Señor disponga debemos atenernos---es su
lema.---Ni se mueve la hoja en el árbol sin la voluntad celeste, ni los
titulados gobernantes disponen cosa alguna que no venga de lo alto.»
Esta filosofía, adoptada por mi ilustre suegro en la plenitud de sus
materiales provechos, es de lo más práctico que han ideado los hombres.

Por picar en todo, de Eufrasia charlamos mi mujer y yo. Indudablemente,
la conjura que trae tan desasosegado al bueno de Don Ramón es la de casa
de Socobio, no la de la nuestra. Por algo que María Ignacia ha oído a su
tía Josefa, hemos podido traslucir que los hilos de alguna tramoya
palaciega pasan por los dedos de la dama moruna y rematan en su
conciliábulo, viniendo sólo al nuestro alguna ramificación secundaria.
No puedo menos de abominar del politiqueo de las mujeres, sacando a
relucir el ejemplo de mi cuñada Sofía y de otras de igual laya, que con
sus hombrunas aficiones dan a todos de cara y sirven de fácil asunto a
los escritores satíricos. Dijo a esto mi sabia esposa que no es Eufrasia
una marisabidilla o politicómana a estilo de Sofía, pues su talento la
preserva de caer en tal ridiculez. Las intrigüelas de la Socobio no la
privan del encanto femenino, ni su natural instinto de toda elegancia la
permite incurrir en afectaciones que destruyen la gracia. Y acabó
exhortándome (fórmula donosa del mandato) a que me abstuviese de
acercarme a la tal sirena (monstruo medio mujer, mitad merluza), pues
corro el peligro de que sus cantos armoniosos y pérfidos me arrastren a
algún escollo del que no pueda salir, o tengan que sacarme sabe Dios
cómo.

\emph{3 de Junio}.---Por accidente natural de lo que llamo cacerías de
hechos y pesca de personas, vino a caer anoche en nuestras manos el
Padre Fulgencio, por todos muy nombrado, de pocos conocido. Veréis lo
que pasó. Fui a Gobernación a visitar a Sartorius. Por la noche, una vez
solos, le faltó tiempo a mi cara esposa para decirme: «¿No sabes,
Pepillo, quién ha estado aquí esta tarde? Pues el Padre Fulgencio. No lo
tomes a broma: el celebérrimo escolapio, confesor de monjas, confesor de
reyes\ldots{} Asómbrate, chico: dijo que sentía tanto no verte\ldots{}
que la fama de tu talento le ha despertado la curiosidad, y que desea
echar un párrafo contigo. Mis tías no sabían qué hacerle. Por poco le
ponen un cirio a cada lado del sillón donde estaba sentadito\ldots{}
Antes que se me olvide: tantas flores quiso echarme el hombre, que ya me
apestaba. Que soy modelo de esposas, modelo de hijas y modelo de no sé
qué. Le consta que Dios se ocupa mucho de mí, y que tiene muy bien
arregladitas todas las cosas para mi felicidad\ldots{} Ha dispuesto Su
Divina Majestad que yo te dé sin fin de hijos, y que todos ellos sean
muy buenos, pero muy buenos, alguno santo. Ya ves qué gloria para ti y
para mí\ldots{} Pues te aseguro que nos hemos equivocado de medio a
medio, chico, y la idea que teníamos del Padre no concuerda ni poco ni
mucho con la realidad. Recordarás que nos lo figurábamos como uno de
esos frailachos sin educación, puercos, zafiotes, de esos que hablando
contigo, a lo mejor te sueltan un eructo, sin más precaución que ponerse
la mano en la boca en el momento de darlo a la luz. Ni es tampoco viejo,
sino así, entre-joven; ni es sucio, Pepe; antes bien, me ha parecido que
se rocía la sotana con aguas olorosas\ldots{} Como lo oyes: no te rías.
Su rostro es más bien guapo que feo, dentro del tipo de guapeza propio
de curas, que es muy distinto de la hermosura de hombres\ldots{} ya me
entiendes. Los ojos son negros y listos, la tez bastante morena, y el
habla\ldots{} ¡ay, hijo! el habla fue lo que más me sorprendió, pues
nosotros nos lo figurábamos con una voz muy bronca, como de castellano
cerril o vizcainote medio salvaje, y resulta que es andaluz, que cecea
un poquito, y con su miajita de gracia y aquel. No habló más que de
temas de religión pura, sin mezcla de política, y de personas
religiosas. ¡Ah!\ldots{} se me olvidaba lo mejor: mis tías le
preguntaron por tu hermana\ldots{} Sabrás que de Talavera tratan de
mandárnosla otra vez acá, porque no le prueba aquel clima, ni las
franciscanas de Madrid se pueden pasar sin su dulce compañera. Vuelven
todas las palomas dispersas a juntarse en su nido\ldots{} ¡Ay! si yo
fuera Reina, si yo fuera Narváez y Bodega reunidos, ¿sabes lo que haría?
Plantar en la calle a todas las monjas, y suprimir la vida de claustro.
La que quiera dedicarse a rezar por los pecadores, que rece en su casa.
¡Mira que llamarlas esposas de Jesucristo! ¡Qué indecencia! ¿Cuándo tuvo
el Redentor esposas, ni mentó para nada estos casorios? ¿Ni qué falta le
hacen a Dios estos coros de Vírgenes flatulentas, aburridas y
desaseadas?\ldots{} ¡Ay, si mis tías me oyeran! Creerían que me he
vuelto loca\ldots{} Pues algún día, cuando yo acabe de perder la
vergüenza, pues hasta hoy no la he perdido más que para ti, les diré que
el Señor no puede estar conforme con tanta virginidad, ni estimar a las
doncellas más que a las casadas. ¡A dónde iría a parar la Humanidad si
todas nos quedásemos para vestir imágenes! ¿Nacen o no nacen las
criaturas? Pues si nacemos, claro es que tiene que haber madres, ¡y lo
que es madres vírgenes\ldots! No se sabe más que de una, María
Santísima\ldots{} Con que, sin mamás y papás, ¿cómo ha de haber mundo y
personas?\ldots{} Pero dejemos esto, y sigo contándote que el Padre
Fulgencio tomó chocolate, no sin hacer antes muchos repulgos con su
boquita, los cuales no acabaron hasta que entró mi tía Josefa con la
jícara y bollos, diciendo: «Hágalo por penitencia, Padre, y si es
exceso, cárguelo a nuestra cuenta.» Bueno: pues ni la más ligera alusión
a las cosas de que hemos hablado nosotros, hizo el escolapio,
acreditándose así de hombre ladino. Si yo no hubiera estado presente,
¡sabe Dios\ldots! En resumidas cuentas, el D. Fulgencio no me resultó
antipático. El será un peine, como dicen que dijo Narváez en casa de la
Generala Córdova; pero lo que es en visita, nadie verá en él más que un
pobre gaznápiro correctito, bien criado, insignificante. Se fue a las
seis, repitiendo sus plácemes y cucamonas al despedirse de mí.»

La visita del famoso escolapio solo sirvió para que María Ignacia
conociera su facha, modos y habla dengosa. De lo interno, nada.
«Fue---me dijo, expresando gráficamente lo incompleto de su
observación,---como si me presentaran un libro de Historia escrito en
lengua desconocida y con estampas. No comprendí nada del texto.
Contentéme con ver los monigotes.»

\emph{4 de Junio}.---A mí viene mi nunca bastante ensalzado suegro, y me
manifiesta que seré pronto diputado en elección parcial. Aunque harto
estaba yo de saber lo que se urdía, híceme de nuevas, para que el señor
de Emparán pudiera darse el lustre de su protección y de mi
agradecimiento. Desde Abril venía mi hermano Agustín trabajando a la
calladita con el Conde de San Luis este negocio, y elegida entre las dos
vacantes la de Tolosa, no necesitó más el Gobierno para ver en mí una
firmísima columna del Régimen. A fines de Mayo, sólo faltaba el
\emph{exequatur} de los cacicones, diputados por Vizcaya, Guipúzcoa y
Álava, que poseedores de toda influencia en las tres provincias, tienen
hecho un pacto fraternal con visos de masónico, por el cual mandan ellos
solos dentro de aquel país, con cierta independencia del mangoneo
ministerial. Para obtener el pase o conformidad de estos reyezuelos de
taifa, solicitó mi hermano la mediación de mi suegro, según este me dijo
al referirme las dificultades vencidas. Habló, pues con D. Pedro Egaña y
D.

Francisco Hormaeche, con el médico Sánchez Toca y D. Fermín Lasala, que
representan los distritos de Vitoria, Guernica, Vergara y San Sebastián
respectivamente, y si en los dos últimos halló excelentes disposiciones
en favor mío, los primeros se le pusieron de uñas, y hubo de sacar el
Cristo de su amistad y de su \emph{arraigo} en Guipúzcoa para que me
tragasen y digiriesen. Debo advertir que tanto el Sr.~Egaña como el
Sr.~Hormaeche son cabezas de pedernal, y tan extremadamente celosos de
la conveniencia y franquicias de aquellos pueblos, que a todo las
anteponen, y sólo a la defensa de esta particularidad española se
consagran. Por esto, más que de diputados tienen, según la gente dice,
traza de \emph{embajadores}, que como tales proceden, y como tales
cobran. Mi buen padre político cuida mucho de hacerme comprender que su
noble país me acepta, no por mi nombre, que allí nada significa, sino
por el nombre adyecticio que me ha dado mi matrimonio, y por el sonoro
título vasco de Beramendi.

Mi mujer y yo, que en las noches pasadas divagamos acerca de este
asunto, riéndonos de las Cortes, de los electores de Tolosa, y de los
discursos que tengo que pronunciar defendiendo los fueros, acabamos de
ponernos en solfa con esta metamorfosis de mi nombre en el pensamiento
tolosano, pues no soy quien soy, sino un yerno, al que se pega la
etiqueta de un marquesado. Nos hace muchísima gracia lo que anoche mismo
nos contó San Román. Preguntado Narváez por el candidato nuevo, y no
acordándose de mi apellido, salió del paso así: «¿Candidato por Tolosa?
El \emph{pollo} de Emparán.»

\hypertarget{xvii}{%
\chapter{XVII}\label{xvii}}

\emph{8 de Junio}.---Obligado a reflejar en estos papeles, con mis
particulares andanzas, algo de lo que anda o corre en tomo mío, diré que
la expedición que hemos mandado a Italia en socorro del Soberano
Pontífice continúa moviendo la opinión y dando mucho que hablar.
Considérase afortunado todo aquel madrileño que puede mostrar una carta
de Reina, de Estébanez Calderón, de Lersundi o de Arteche, describiendo
la marcialidad y gallardía de las tropas en el acto de recibir la papal
bendición, y manifestando las ganas que tienen de batirse y acá volver
cargaditos de laureles. Sobre este particular, mi buena madre ha escrito
a María Ignacia lo que a la letra copio, reflejo del popular
sentimiento: «Y de la Cruzada que habéis mandado a Italia para reponer
al Papa en su Silla, no te digo más sino que me pasé la tarde
lloriqueando; tal efecto me hizo el relato que trae el periódico de la
bendición de Su Santidad a las tropas, cosa grande, hija, cosa sublime,
que a todos los españoles debe llenarnos de satisfacción y júbilo. ¿Qué
más podían ambicionar nuestros militares? Me los figuro locos de
alegría, deseando que les den la voz de fuego y de ataque, para no dejar
títere con cabeza, y dar cuenta de toda esa caterva de anárquicos,
infieles y \emph{repúblicos} que le han usurpado al Pontífice su bendito
reino. Digo yo que si los soldados españoles han sido y son de suyo
valientes, como hijos, hermanos y sobrinos del Cid Campeador, y no han
menester de bendiciones del Papa para vencer a todo el mundo, ahora que
les cae tan de cerca y como de primera mano el rocío celestial, su
arranque y bríos serán tales que no habrá poder humano que les haga
frente. El cartaginés y el romano, el celtíbero, el godo y el sarraceno
de que nos hablaba el pobrecillo Miedes, que de Dios goce, serían ahora
niños de teta delante de nuestra milicia. Pienso que cuando esta leas,
querida hija, habrán llegado a Madrid noticias de alguna tremenda
batalla en que no queden ni los rabos de los Garibaldis y
Mazzinis\ldots{} Ya estoy viendo al gran Pío entrando triunfalmente en
Roma en brazos de los Córdovas y Lersundis, que ahora son los caballeros
o paladines de Dios\ldots{} Hemos de consagrar, hijita del alma, nuestro
sufragio y nuestras oraciones a los pobrecitos que han de morir, pues
muertes habrá, que ellas son inseparable calamidad de las guerras. Y no
es bien que nos metamos en averiguaciones del por qué permite Dios
peleas sanguinarias entre los hombres, pudiendo arreglar las cosas con
sólo su querer. Tratándose ahora de poner en su Silla al que es Vicario
del mismo Dios, parecía natural que Dios, en este caso juez y parte,
dispusiese hacer polvo a los malos sin sacrificar la vida de los buenos.
Pero ¡ay! la semejanza de esta campaña por la Fe con las comunes
querellas entre naciones, más debe maravillarnos que confundirnos, pues
lo que hay es que Dios abandona su causa a los humanos, y es grande
orgullo que sea España la que ahora pelea por Él\ldots{} Ya estoy
viendo, hija mía, los beneficios que van a llover sobre nuestra Nación
por esta Cruzada. En premio de haber salido a su defensa, el Señor nos
dará la paz en todo lo que resta de siglo, y si me apuras, por el que
viene; y a nuestra Reina piadosa colmará de venturas, y al Rey muy pío
otro tanto, y les concederá numerosa y masculina sucesión para dicha del
Reino; y entre todos los Ministros y magnates que habéis dispuesto la
Cruzada repartirá felicidades, buenas cosechas, suerte en los negocios y
demás cosas buenas.

Hija muy amada, ya espero todos los días la noticia de tu alumbramiento,
y lo veo tan feliz que más no puede ser. Dios y la Santísima Virgen te
asistirán. Y como Pepe me ha dicho que me mandará la noticia por el
telégrafo del Gobierno, no hago más que mirar a la torre que tenemos en
el alto de Baides a ver si hace alguna garatusa con las bolas\ldots{} Yo
no lo entiendo; pero como el telegrafista D. León Preciado me ha
prometido que me comunicará la noticia tan pronto como llegue, en él
descanso, y no hago más que pedir a Dios que te dé un buen cuarto de
hora. Supongo que en estos días estarás muy molesta\ldots{} Llévalo con
paciencia, niña mía, y no dudes de la completa felicidad del suceso.
Verás como no me equivoco en lo que te anuncié, y para que no lo olvides
y cobres ánimo, te lo repito: Tendrás hijo varón, tan robusto y sanote
que si te descuidas la emprenderá contigo a bofetadas a poquito de
nacer. Será tan guapo que las muchachas, en su día, se volverán locas
por él, y sacará todo el talento de su padre, y todita tu bondad, tu
prudencia y tu gracia. Apúntalo, hija, para que veas que acierta y no se
equivoca en un solo punto de estas adivinanzas vuestra amante
madre---\emph{Librada.»}

\emph{12 de Junio}.---Agustín y D. Feliciano me notifican que ya
parieron los de Tolosa el embuchado de mi elección. Me imagino los
terribles incidentes del acto, tantas firmas en el Ayuntamiento como
colegios electorales componen el venturoso distrito, descanso de las
urnas, que no habrán tenido que indigestarse de papeletas; algunos
vasitos de \emph{sagardúa} empinados a mi salud por los muñidores
electorales de cada barrio, y luego un acta más limpia que la cosa más
limpia del mundo, la cual es, según el gracioso marqués de Albaida, mi
amigo, \emph{el bolsillo de los contribuyentes}. Aunque tengo bien
aprendida mi lección política, me advierte Agustín que estoy obligado a
votar siempre con el Gobierno, salvo en alguna cuestión vascongada que
pudiera surgir, y en caso de disidencia, votar con Sartorius, como fiel
parroquiano de su iglesia\ldots{} No puedo seguir. Me llaman de mi casa.
Ya me figuro\ldots{} Abandono mi \emph{confesonario}, la biblioteca del
Congreso\ldots{}

\emph{15 de Junio}.---El día 12, a las tres de la tarde, salió mi mujer
de su cuidado con felicidad y presteza, que parecieron maravillosas al
propio Corral. Según este, que presidió el acto en nombre de Esculapio,
y mi suegra, que al mismo llevaba su conocimiento práctico y el maternal
cariño, no se ha visto alumbramiento más fácil y espontáneo, ni
primeriza más valiente, ni criatura más desahogada que la que Dios me ha
dado por hijo. Sus primeros berridos revelaron un carácter impetuoso,
dominante, que no admite objeciones a su potente albedrío. Mi suegra
observó que cuando lo fajaban después de lavarlo, daba manotazos como un
atleta del circo, y que su robustez es lo mismo que la de un aguador. Mi
mujer dice que es muy pillo, y que le da unos tremendos estrujones con
\emph{aquellas manazas}. No necesito contarle a la Posteridad mi
satisfacción, mi orgullo, mi gratitud a Dios, omnipotente y próvido; ni
afirmar que se centuplica el cariño a mi mujer por los extraordinarios
bienes que me ha traído, entre ellos la inefable dicha de ser padre,
cabeza de familia, dicha que las redondea y resume todas, así las
espirituales como las del orden social, así las que tienen su raíz en el
corazón como las que extienden por todo el ancho campo de la vida sus
lozanas ramificaciones.

Tres días he permanecido junto a María Ignacia sin separarme de ella un
instante, platicando del chiquillo y de lo bravo y jacarandoso que
viene. Bien quisiera criarlo, y asegura que le sobra lozanía para ello;
pero los abuelos y yo entregamos el heredero de Emparán a la opulenta
ubre de una de las dos amas alcarreñas enviadas por mi madre. No debe
exponerse mi esposa a los peligros y pejigueras de la lactancia, ni ello
estaría, como dice mi suegro, \emph{en armonía con su posición}\ldots{}

Si hoy he tenido que abandonar mi grato puesto de honor y de alegría
junto a María Ignacia, débese al enfadoso deber de jurar mi cargo en
este maldito Teatro Congreso. Tres días ha, me estrené de padre de
familia; hoy me estreno de padre de la patria. Una vez prestado, con la
debida solemnidad, de rodillas, la mano sobre los Santos Evangelios, el
juramento que confirmaba mi investidura, pasé a sentarme en los escaños,
prestando voluble atención al rezo perezoso con que aquellos señores,
mis compadres de la patria, en corto número allí reunidos, examinaban y
discutían los Aranceles de Aduanas; y fue tal mi embeleso ante tan
entretenido asunto, que habría caído en profundo sopor si no escapara
del salón, buscando mayor amenidad en el de Conferencias, ancho
vestíbulo de lo que ha de ser teatro. Allí me encontré a mi caro amigo
Federico Vahey, diputado por Vélez-Málaga, el hombre de mejor sombra de
este Congreso, el que con sus oportunidades y agudezas ameniza las
soñolientas páginas del \emph{Diario de las Sesiones}; y sentándome con
él en un diván excéntrico, pasamos revista al nutrido personal de
periodistas y diputados que allí bullía. Después de apurar graciosos
comentarios de aquel vano tumulto, y de trazar con fácil palabra
retratos breves de este y el otro, díjome Vahey que lleva una exacta
estadística de los representantes del país que gastan peluca, los cuales
no son menos de diez y siete. Con disimulo me los designa en los grupos
próximos, sin cuidado en los distantes, para que yo aprecie la variedad
de color y estilo de aquellos capilares artefactos, que tapan calvas
venerables. La primera peluca que me hace notar es la de Pascual Madoz,
rubia y con ricitos, como las que las beatas suelen poner a San Rafael o
al Ángel de la Guarda; veo y examino después la del Sr.~Maresch y Ros,
diputado por Barcelona, excelente persona, de notoria honradez y trato
muy afable, mas de un gusto marcadamente catalán en la disposición de
sus pelos postizos. Muy bien hecha y ajustada, hasta parecer cabellera
de verdad, es la falsa de Martínez Davalillo, representante de Santa
Coloma de Farnés; pero no puedo decir lo mismo de la del Sr.~D. Joaquín
López Mora, de un gris polvoroso, y con bucles que parecen serpientes;
ni merece mejor crítica la del Sr.~Ruiz Cermeño, representante de
Arévalo, que parece de hojas secas. Pero después de bien vistas y
examinadas todas, asignamos el primer premio de fealdad a las que
ostentan los dos hermanos Ainat y Funes, el uno diputado por Pego, el
otro no sé por dónde, las cuales, sobre ser mayores que el natural,
imitan en su bermeja color tirando a rucia, las greñas del león viejo
del Retiro. Ved aquí en lo que nos entreteníamos dos descuidados padres
de la patria, novel el uno, corrido y desengañado el otro.

No quise volverme a casa sin echar otra ojeada al Salón de Sesiones, por
ver a qué alturas andaba la divertidísima cuestión de Aranceles. Ante
una docena de diputados soñolientos, hablaba un orador de alta estatura,
ya viejo, de bella fisonomía y cabellos blancos naturales, vestido con
luenga levita de corte inglés, muy elegante, la palabra tan pronto
atropellada como premiosa, el gesto vivo, tendiendo con facilidad a
descomponerse. Era Mendizábal.

En el momento de mi entrada en el Salón, decía: «Yo, señores, repitiendo
lo que ayer tuve el honor de manifestar al Sr.~Infante, soy partidario
del libre comercio; pero no desconozco que en espera de tiempos mejores,
hemos de conceder a nuestra industria una protección prudente\ldots»
Después se metió en un laberinto de cifras, en el cual no pude seguirle.
Entendí que hacía estudio comparativo de la fabricación algodonera en
Inglaterra y en Cataluña. En el Banco Negro, o de los Ministros, sólo
estaba el Sr.~Mon, con benévolo cansancio, mirando al orador, y
denegando alguna vez con signos de cabeza, o con un sonreír bonachón. En
el banco de la Comisión, había dos individuos, el señor Amblard y otro
que no conozco (me parece que era el Sr.~Barzanallana, pero no puedo
asegurarlo), ambos de bruces en el respaldo delantero, o sea el
Ministerial, en actitud de hastío. Entre los diputados que escuchaban al
orador vi a Gonzalo Morón, que a todo atiende, de todo habla y en todo
ha de lucir su ingenio fecundo; Sánchez Silva, que no pierde ripio en
las cuestiones de Hacienda; Madoz, que entró poco antes que yo, y D.
Alejandro Oliván. Los demás, como el gotoso Sr.~Álvaro, director de
Aduanas, y el Sr.~Canga Argüelles, que, según creo, es director de
Fincas del Estado, dormían una siestecita o escribían en sus pupitres.
Detúveme un rato, atraído de la familiar sencillez de aquel cuadro que
me pareció interesante, y no pude menos de contemplar con tanta tristeza
como admiración al hombre de voluntad atlética, que expresaba su
pensamiento rodeado de un silencio tedioso y de una desatención lúgubre,
ante unas cuantas personas que representaban a la generación heredera de
la suya\ldots{} Por fin, oí decir a Mendizábal tras un leve suspiro: «Y
no sigo, señores diputados, porque el Congreso está fatigado, con razón
fatigado de este interminable debate\ldots{} y yo también lo estoy.»
Recogiendo con ambas manos los largos faldones de su levita, se dobló
despacio para sentarse. Como entonces le veía yo por primera vez en mi
vida, me pareció que buscaba el descanso como todo aquel que cree haber
hecho grandes cosas.

El Vicepresidente, Conde de Vistahermosa, a quien faltaba poco para
descabezar un sueñecico, levantó la sesión.

\emph{20 de Junio}.---Ayer volví al Congreso porque era día de Secciones
y querían meterme en una comisión de importancia. Fuera de este motivo,
relacionado con mis altos deberes, vine por el gustillo de oír a
Olózaga, que hablaba por primera vez después de su vuelta de la
emigración, y aunque el asunto en que había de intervenir era la enojosa
y nunca terminada cuestión de Aranceles, se creyó que de esto tomaría
pie para un discurso político de sensación y bullanga. Hubo, pues, plena
entrada y concurso de gente política o de afición, y las tribunas, que
aquí son palcos, se habían llenado dos horas antes de la hora
reglamentaria. Ya después de las cinco empezó el célebre agitador
progresista su discurso, que como retórica parlamentaria me pareció
admirable, oración capciosa en que los derechos de Aduanas eran un
pérfido artificio combinado con arte sagaz para producir gran cisma y
confusión en la inquieta mayoría. Gracias que el Gobierno anduvo listo y
acudió con remedios oportunos a componer el cotarro. Terminado todo con
menos rebullicio de lo que se esperaba, no pude consagrar el resto de la
tarde al recreo de mi confesión, porque se me atravesó inopinadamente
una eventualidad que no sé si llamar feliz o adversa, y que debió de ser
obra de un diablillo chancero, a juzgar por la extraña mezcolanza de
sorpresa, sobresalto y alegría que ante ella sentí. No había concluido
D. Salustiano su perorata, cuando un ujier me entregó un papelito
enviado desde las tribunas. Era de una señora que me suplicaba subiese a
verla antes de que terminara la sesión. Leer la esquela, alzar la vista
hacia el palco frontero y ver a Eufrasia, que en aquel instante me
miraba risueña, llevándose a la mejilla su abanico cerrado, fue todo
uno. No había escape. ¿Cómo eludir, sin pecado de grosería, un reclamo
tan halagüeño? Pensé que algún asunto más importante para ella que para
mí quería comunicarme la señora de Socobio, y con esta idea tomé la
resolución de acceder a su ruego; así, en cuanto Olózaga se sentó,
levantéme yo, y al palco me fui derecho. Salió a mi encuentro la dama, y
en el antepalco, que es de los mayores en este soberbio edificio
teatral, fuí recibido sin ceremonia, ambos en pie porque no teníamos
donde sentarnos. Como las demás señoras no se habían movido de su sitio,
atentas a la respuesta que daban a Olózaga los oradores de la comisión,
pudimos hablar lo que fielmente copio:

«Ante todo, amigo mío, abra usted de par en par su alma para recibir mis
enhorabuenas; ábrala mucho, porque si no, no caben. Ya es usted padre;
asegurada está la sucesión de su casa y familia\ldots{} Créalo: he
tenido un alegrón muy grande. Ya sé que la madre y el niño siguen muy
bien: él como un ternero, ella como una excelente vaca. Ya tiene usted
todo lo que deseaba: un hogar feliz, una posición independiente\ldots{}
Con lo que no estoy conforme, es con que me le hayan metido en política,
trayéndole a esta farsa del Congreso. Porque esto es una mascarada, y si
no sirve usted para dar bromas, vale más que se largue de aquí.»

Díjele que yo tomaba la política a beneficio de inventario, o con un
simple fin decorativo; que mi hermano Agustín y Sartorius me habían dado
la investidura, propiamente así llamada porque era como ponerse un
vestido elegante, o un lucido uniforme social. A esto respondió con
gracia:

«El traje ha de resultar molesto para quien se lo pone sin la mira de
hacer el papelón. Esto es muy bueno para los que buscan el negocio; pero
los que ya lo tienen hecho no vienen aquí más que a servir de
comparsas\ldots{} Vamos, no me mire usted tanto: creeré que estoy hecha
una visión.

---Es todo lo contrario. La encuentro a usted guapísima.

---Un poquito flaca.

---Propiamente flaca no: con tendencias a la estabilidad de formas, y a
no engordar\ldots{} En el rostro no hallo variación: solamente los ojos
me parecen más grandes, más soñadores\ldots{} o soñolientos\ldots{}

---Pensé que iba usted a decir que estoy ojerosa. Eso no: duermo
perfectamente, y no lloro nunca ni tengo por qué.»

Reparé en su traje elegantísimo, de batista de Escocia chaconada, con
fino dibujo verde musgo sobre fondo blanco; el sombrero de paja gruesa
de Italia, con lazos y flores de tafetán de los mismos tonos. El
ajustado cuerpo en forma de blusa marcaba su inverosímil talle gentil,
unión de las abultadas zonas del seno y caderas.

«Ya habrá usted comprendido---prosiguió---que no te he llamado
exclusivamente para darle mis parabienes. Tenemos que hablar un
poquito\ldots{} pero aquí no puede ser. Cuando se levante la sesión,
véngase a dar conmigo una vuelta por la Castellana. Mi coche está en esa
calle por donde se sube a la parroquia de Santiago. Allí le
espero\ldots{} Y ahora, no se entretenga más. Ya suena la campana
llamando a votación\ldots{} También aquí tengo yo que ser su maestra,
instruyéndole en las obligaciones parlamentarias. Ese cencerro convoca a
todo el ganado de la mayoría para que vote lo que manda el Gobierno.
Vaya usted, corra, y lleve preparado el \emph{sí} o el \emph{no}, según
lo que sea\ldots{} Con que ¿le espero en mi coche?»

Mirando cara a cara el peligro y sobresaltado de la atracción que sobre
mí sentía, contesté que daríamos la vuelta en la Castellana\ldots{} una
sola vuelta, todo lo más dos\ldots{} Media hora después navegaba yo en
el coche, y por cierto que al entrar en él iba ya un poquito mareado.

\hypertarget{xviii}{%
\chapter{XVIII}\label{xviii}}

Sépase ante todo que no íbamos solos Eufrasia y yo. Nos acompañaba una
vieja muy compuesta, hermosura en ruinas, que tuvo su apogeo y esplendor
en los años medios de Fernando VII, camarista que fue de la Reina Doña
Isabel de Braganza. Perteneciente a la aristocracia mercenaria, de
creación palatina, ostenta el deslucido título de Condesa o Baronesa (no
estoy bien seguro) de San Roque, de San Víctor, o de no sé qué santo. En
la duda, la designaré provisionalmente por el primer bienaventurado que
se me ocurra. Es mujer histórica y de historia, hoy mandada recoger por
la subida cuenta de sus años, aunque todavía colea en la vida social.
Entiendo que tiene un hijo y un yerno en la regia servidumbre.

«Ya sé---me dijo Eufrasia en el rápido avance del coche por la calle del
Arenal,---que Rafaela y yo estamos amenazadas de salir, codo con codo,
en la primera cuerda para Filipinas.»

Soltaron ambas la risa, y yo agregué, siguiendo la broma: «A donde van
usted y su amiga es a las islas Marianas\ldots{} ¿Pero cómo lo saben si
yo a nadie lo he dicho?

---Lo sabemos---replicó la veterana beldad,---porque el fantasmón no lo
dijo a usted solo. Por Pepe Villavieja me mandó un recado para que yo lo
pusiera en conocimiento de las interesadas\ldots{} No hicimos caso: nos
reímos\ldots{}

---Tan bien le resulta a ese espantajo---observó Eufrasia,---el meter
miedo a los hombres, que cree poder amedrentar fácilmente a las mujeres.
¡A buena parte viene!\ldots{} ¿Pero qué ha de hacer él más que estar a
la defensiva, muy al cuidado de su pelleja? ¿Con que a las islas
Marianas nada menos? ¿Está él bien seguro de que no le embarcarán para
allá con viento fresco? Si en aquellas islas hay caribes, ¡qué buen
maestro se pierden para perfeccionarse en la barbarie!

---¿Pero es verdad que conspiramos, amiga mía? Yo no lo creí. Pensé que
se trataba de una intrigüela\ldots{} no política.

---Puede usted tranquilizar a su amigo, asegurándole que se han
suspendido los trabajos, y que no hemos de volver a las andadas hasta
que no se sepa cómo va el negocio de Italia.

---Hasta que no veamos---dijo la San Víctor,---si \emph{Fernandito} pega
o no pega.

---Yo todo lo temo de esta gente y de su \emph{mala pata}---declaró mi
amiga.---Al refrán que reza \emph{Por todas partes se va a Roma}, debe
añadírsele: \emph{menos por Gaeta}.

---Pero explíqueme, Eufrasia---dije yo riendo de verla tan
\emph{oposicionista},---¿qué motivos, qué razones\ldots{} porque alguna
razón habrá\ldots{} la han traído a la enemistad de Narváez? Antes no
pensaba usted así\ldots{} ¿Ha recibido D. Saturno algún agravio del
Presidente del Consejo?»

Mordisqueando el abanico, la moruna miraba hacia la calle con evidente
ira, más bien rabia. Durante una pausa breve, la San Blas y yo nos
miramos, como interrogándonos sobre cuál de los dos hablaría primero, y
sobre lo que debíamos decir para poner airoso término a la pausa. Rompió
por fin el silencio la marchita beldad con esta familiar explicación:
«Usted, Sr.~de Fajardo, merece toda confianza, y como está en
antecedentes\ldots{} me consta por la misma Eufrasia que está en
antecedentes\ldots{} yo me permito responder por mi amiga, para que esta
pobre no se vea en la precisión de recordar\ldots{} ciertas infamias.
Narváez es hombre muy deslenguado. No respeta ni categorías ni
reputaciones, y poniéndose a soltar chascarrillos, no se detiene ante
ningún reparo. Hablando de esta una noche en casa de Santa Coloma,
refirió no sé qué incidentes, de esos que los hombres poco delicados se
confían unos a otros, escenas o casos de la vida que el tuno de Terry
hubo de relatarle viajando por el extranjero\ldots{} cosas
reservadísimas que contadas con descaro y mala intención\ldots{}
resultan\ldots{}

---¡Mentiras, fábulas absurdas!---dijo Eufrasia pálida y balbuciente y
completando la información de su amiga.---Cuando me trajeron el cuento,
no sentía más que una cosa: no poder volverme hombre.

---Pues hay más, Sr.~de Fajardo---prosiguió la otra.---Al Presidente del
Consejo se le podrán perdonar las botaratadas de lenguaje, que quien
trata con políticos es natural que alguna vez se desboque; pero al
caballero no se le perdona que sin venir al caso ridiculice a personas
de arraigo, apartadas de estas miserias de la vida pública. Ya sabe
usted que se trató de conceder a Saturnino un título de Castilla. Esta
no quería; pensaba que era subir demasiado pronto. Pero el pobre
Saturno, que de algún tiempo acá venía sonando con el Marquesado, no era
tan modesto en sus ambiciones. El asunto iba por buenos caminos.
Arrazola estaba conforme; el Rey se interesaba en ello. Un día, en el
mismísimo Palacio Real, preguntó a Narváez el Duque de Gor qué título se
pensaba dar a Saturnino, y el \emph{Espadón}, como si dijera una cosa
muy seria, respondió: «Le haremos \emph{Marqués de Capricornio.»} Ya ve
usted qué grosero insulto.

---Tanta grosería y bajeza---dijo Eufrasia,---me han hecho mudar de
parecer respecto a esa gracia y a su oportunidad. Ahora, viendo en qué
manos está la Nación, lo que antes creí prematuro ya me parece tardío.
Seremos Marqueses. Esta Sociedad no merece la modestia. Donde ya no hay
ninguna virtud, donde todo se ha pisoteado, y por si algo faltaba, ya
pisotean de firme, la mayor de las tonterías es tener delicadeza y
escrúpulos. Coronas que fueron de oro han venido a ser de papel dorado,
y las de papel se han hecho de oro. Respetar lo pasado, mirarlo mucho,
ya para amarlo, ya para temerlo, es cosa que ahora no se usa. Pues
vivamos en lo presente, y coloquémonos donde sea más fácil pisotear que
ser pisoteado.»

Causáronme pena este pesimismo y el nuevo ser psicológico de mi amiga.
Yo no comprendía por qué rápida evolución, la que hace un año me daba
prácticos consejos del vivir manso, cauteloso y positivo, esquivando las
pasiones, se dejaba contaminar de las más violentas. Sobre esto dije
algo, a lo que me respondió imperturbable: «Las pasiones vienen cuando
tenemos arreglada la vida. Si por acaso llegan antes, se encuentran la
puerta cerrada, por estar una en los afanes de dentro\ldots{} Y como al
encontrar cerrado se marchan las pasiones, de aquí que pasen por
virtuosos los que no lo son. Va una mujer tan tranquila, y a lo mejor
alguien le da con el pie; entonces se acuerda de que es víbora, de que
puede serlo, y lo es.»

Admirando su ingenio, díjele que todo aquel reconcomio contra Narváez
podía muy bien carecer de fundamento, como nacido de hablillas y
dicharachos de los desocupados. ¿Quién le aseguraba que eran del propio
Duque las malvadas referencias de Terry, y la grosería del título de
\emph{Capricornio?}

«¡Ay!---exclamó Eufrasia;---como si yo misma lo hubiera escuchado,
segura estoy de que esas infamias salieron de aquella boca, manchada con
tantas blasfemias y palabrotas de cuartel. Usted, por lo visto, se ha
dejado deslumbrar por el brillo falso de ese soldadote, y ha creído la
leyendita que propalan los adulones que le rodean. ¡Oh, Narváez, león
que lleva dentro un cordero! ¿No es eso? Un hombre que en sus arranques
instintivos de mal humor atropella sin reparo al más pacífico, y luego
le pide perdón y le hace favores, y le da chocolate de Astorga. Ese es
el tipo que quieren darnos en aleluyas, corazón sensible que cuando se
irrita ruge, y cuando se aplaca es lo mismo que un niño\ldots{} ¿No es
esta la leyenda? ¿Apostamos a que usted es de los que la ponen en
circulación y la reparten de oreja en oreja para que corra?»

Respondí que la tal leyenda, bosquejo biográfico del natural trazado por
los contemporáneos, me parecía lo más próximo a la verdad, y que por
ella, pues no hay mejor modelo, fijarán los historiadores futuros la
figura de Narváez. Eufrasia sonrió, recreándose en la fuerza de los
argumentos que en contra de la leyenda cree poseer, y reclamada la
atención de su amiga y la mía nos dijo: «Pues aquí me tienen ustedes con
voz y autoridad de Historia para echar abajo esa mentira novelesca. Lo
que voy a contar, yo lo he sentido muy de cerca, y mi padre, que vivo
está, y otros señores manchegos muy respetables, pueden dar de ello
testimonio. El año 38 pasó este caballero por un pueblo de la Mancha que
se llama Calzada de Calatrava\ldots{} Iba en persecución del carlista
Gómez\ldots{} ya sabe usted, la famosa expedición de Gómez\ldots{} De
aquel pueblo al mío, donde yo estaba con mis padres, no hay más
distancia que dos leguas o poco más. Yo era entonces una mozuela: me
acuerdo de aquellos sucedidos como si fueran de ayer, y la impresión de
terror que dejaron en mí no se borrará nunca; que si espanto causaban
allí los facciosos con sus crueldades y saqueos, no daba menos que
sentir este maldito que los perseguía en nombre de la Reina, pues unos y
otros llegaban, asolaban y partían como una legión de demonios. Era en
el mes de Agosto; llegó Narváez tal como ayer, y hoy mandó fusilar, con
juicio sumarísimo, al último Prior de la Orden de Calatrava, D.
Valeriano Torrubia, a un rico propietario de la misma ciudad y a una
mujer. ¿Creerán ustedes que este hecho brutal era escarmiento de
facciosos porque las víctimas habían dado apoyo al cabecilla Gómez? Pues
están muy equivocados, y si la Historia se escribe así, maldita sea mil
veces. El delito del pobre D. Valeriano era estar emparentado con la
familia de Espartero, y ser, como este, hijo de Granátula, que sólo
dista de la Calzada una hora de camino. Para condenarlo, así como a sus
compañeros, en la sumaria hecha de mogollón sin más objeto que cubrir el
expediente, se alegó la entrega de un fuerte, realizada siete meses
antes, al paso de Cabrera, después de una reñida acción en que
perecieron trescientos y pico de liberales. Oigan ustedes a mi padre. Mi
madre, que era Torrubia y tenía parentesco con el Prior, diría, si
viviera, que ninguno de aquellos infelices era carlista ni tuvo arte ni
parte en la entrega del fuerte. Todo esto, si no lo he presenciado, lo
he sentido en derredor mío, expresado con gritos de dolor que eran
gritos de verdad. No son referencias lejanas desfiguradas por el tiempo
y la distancia, sino hechos que palpitaban a mi lado, entre mi familia y
mis convecinos, y que siguieron estampados en la memoria de todos los
que entonces vivíamos en la Mancha.

«Pues oigan más. La única persona, entre las principales de la Calzada,
que pudo intervenir en la entrega del fuerte, fue un cura llamado
Vadillo. ¿Por qué, pregunto yo, este hombre de la leyenda, este cordero
con garras de león no fusiló a Vadillo y sí a los otros, que nunca se
significaron como carlinos? ¿Por qué no quiso escuchar, ni recibir
siquiera, al hermano de Espartero, canónigo de Ciudad Real, que acudió a
pedir clemencia, y llevaba, según dicen, órdenes de que se suspendiera
la ejecución? Porque, sépanlo ustedes y sépalo el mundo todo, lo que
menos le importaba a este tío era perseguir carlistas y alentar
liberales; su pasión dominante era el odio a Espartero, y la envidia de
los triunfos y de los increíbles adelantos de mi paisano; su móvil, la
idea de ser como él, poderoso y popular; su fin, destruir todo lo que
significase adhesión a Espartero, partido de Espartero, familia de
Espartero\ldots{} Esto, que aquí no se vio nunca, lo vimos claro todos
los que allá vivíamos: yo respiré estas ideas, y de su verdad no puedo
dudar\ldots{} Ahora viene la segunda parte de mi cuento, y aunque para
mí esta parte es tan verdadera como la que acabo de referir, no me
atrevo a darla como Historia. Vamos, que también traigo yo mi poquitín
de leyenda para colgársela al \emph{Espadoncito andaluz}. La noche antes
del fusilamiento, la pasó D. Ramón en compañía de una guapísima
mujer\ldots{} La conocí: había sido mi amiguita; tenía tres años más que
yo\ldots{} Fue público y notorio que el cura Vadillo no era extraño a
las amistades de la buena moza con el General. Si un día entregó un
fuerte a Cabrera, otro día le entregaba otro fuerte a Narváez; sólo que
este castillo, aunque muy bonito como mujer, no valía nada como
fortificación\ldots{} Cierto es lo que digo de esas amistades: lo que
presento como leyenda, usted, Pepe, puede ponerlo en claro si se atreve
a preguntárselo a Narváez\ldots{} o a Bodega, que debe saberlo lo mismo
que su amo. Pregunte usted a cualquiera de los dos si es cierto que en
la noche de marras vacilaba el General entre el rigor y la clemencia, y
que Rufina Campos le pidió que fusilara sin piedad, ofreciendo su cuerpo
en pago de la orden; si es verdad que en su impaciencia por concluir
aquel negocio de las muertes, le hizo coger la pluma y le llevó la mano
para que firmara\ldots{} Este es un punto que yo no me atrevo a sacar de
la Fábula para llevarlo a la Historia: lo cuento como me lo contaron, y
no respondo de ello.

Lo que no tiene duda, amigo mío, es que en Calzada de Calatrava había
por aquel tiempo una fuerte discordia entre dos bandos que se habían
formado, y ardían en rencores con más fuego de pasioncillas locales que
de ideas políticas, y que uno de estos bandos se valió del tremendo
Narváez para desbaratar al otro. Pescaron al \emph{Espadón} echándole
por cebo la carne fresca de Rufina Campos. Con que ahí tienen los
señores Narvaístas una vela que encenderle a su ídolo, el borrego con
zarpa de león, que más valdría decir de hiena, por la propiedad de las
cosas históricas\ldots{} ¡Y este hombre quiere que ahora nos dobleguemos
ante su Orden y ante su \emph{Principio de autoridad}, él, que siempre
fue díscolo y revolucionario, él, que no hizo más que pisotear su tan
cacareado \emph{Principio!} ¿Cómo se ha de respetar a quien nada
respetó? ¿Cómo ha de sofocar las conspiraciones quien toda su vida se la
pasó conspirando? Si los sublevados victoriosos del 40 llamaban
insurrectos a los vencidos, y estos a su vez, triunfantes el 43,
llamaron rebeldes a los del 40, ¿qué nombre hemos de dar a todos más que
el de bandidos? No se asombre usted, Pepe, ni me ponga la carita
burlona, que sus burlas y su estupefacción no son más que una máscara
con que tapa un escepticismo tan negro como el mío. Yo no creo en estos
hombres, Pepe, ni usted tampoco. La Historia de España, mientras hubo
guerra, es una Historia que pone los pelos de punta; pero la que en la
paz escriben ahora estos danzantes, no se pone los pelos de ninguna
manera, porque es una historia calva, que gasta peluca. Yo, qué quiere
usted que le diga, entre una y otra, prefiero la primera\ldots{} me
repugnan los pelos postizos.»

Esta idea nos dio pie para reír, dejando incontestada la graciosa sátira
contra los hombres públicos, y sin comentario el terrible cuento
manchego.

\hypertarget{xix}{%
\chapter{XIX}\label{xix}}

Recorriendo la Castellana, cuando ya la tarde caía, deploraba yo que la
presencia de la beldad vetusta me privase de hablar con Eufrasia
libremente. Perdóneme mi cara esposa; yo me sentía de improviso
arrastrado fuera de la existencia regular, al influjo de aquella mujer,
que si fue mi tentadora en tiempos libres, cuando con piadosa mano hacia
las pacíficas venturas materiales me guiaba, ahora, por diverso estilo,
me trastorna y enciende con los atrevimientos de su voluntad sin freno.
Lo único de que yo hablarle podía delante de la señora mayor, era la
conspiración de ópera cómica en que ponía todos los donaires y sutilezas
de su entendimiento, y sobre ello le pedí más explicaciones, que sólo a
medias quiso darme. «Conténtese usted, por ahora, con lo que le
dije\ldots{} y es que por el momento hay tregua\ldots{} ¡Pues no
faltaría más sino que yo le revelara a un enemigo nuestros planes!
Bastante haré, el día en que se den los pasaportes al Ministerio
Narváez-Bodega, y se haga limpia general de \emph{hombres públicos},
bastante haré, digo, con librarle a usted de que le lleven a las
Marianas, a tomar los aires que me recetaron a mí\ldots{} Esté, pues,
tranquilo\ldots{} Y no le digo que se venga a conspirar a mi campo,
porque con el Marqués de Beramendi no hay que contar ya para nada.
Hombre acaudalado y padre de familia, sus ambiciones deben limitarse a
cuidar hijos, que los tendrá en gran número, sin que pueda en ningún
caso dudar que son suyos\ldots{} ¿Le parece que es ésta poca ventaja en
los tiempos que corren?

---Es usted mala, Eufrasia, y pensando bien por el lado mío, arroja por
otros lados su sátira cruel.

---¿Pero no le he dicho que soy víbora, Pepe? Entre morder y ser
mordida, con veneno, ¿qué es preferible?\ldots{} Y en resumidas cuentas,
el ser satírica no es lo peor que puede ser una mujer\ldots{} Porque yo
muerda un poco, no se escandalizará usted, Pepe.

---Pero creeré que no está en carácter, y que pierde parte de su encanto
con esas mordeduras. ¿Recuerda usted lo que significa en griego su
bonito nombre? \emph{Eufrasia}.

---Ya me lo dijo usted en otra ocasión: significa \emph{Alegría}.

---Pues eso ha de ser usted siempre: \emph{Alegría}, la alegría del
mundo, de la sociedad\ldots{}

---¡Ay, Pepito, Pepito\ldots{} a buenas horas!\ldots{} En otro tiempo
pude pensar que sería eso\ldots{} ¡Pero hoy, después de tantas penas y
de tanto luchar!\ldots{} Además, mi condición alegre se va saliendo de
mí a medida que va entrando la hipocresía.

---¡Hipócrita\ldots{} también se declara hipócrita!

---Me declaro práctica, maestra en filosofía marrullera, con arreglo a
la época y al país en que vivimos. ¡Y usted me desconoce, y usted me
niega, Pepe, usted que es mi mejor discípulo!\ldots»

En esto, echábase encima la noche, y una contingencia venturosa vino a
conjurarse contra mi virtud y a favorecerme en mis desatinados estímulos
de perdición. La Condesa o Baronesa de San Lucas, de San Gil o de no sé
qué santo, dijo a su amiga que, llegada la hora de recogerse, diese
orden al cochero de dejarla en su casa, Costanilla de la
Veterinaria\ldots{} ¡Con cuánto gozo sentí el traqueteo de las ruedas,
corriendo presurosas, descontando los segundos que faltaban para que
sola conmigo se quedase la moruna! El ansiado instante llegó al fin, y
con él reverdecieron mis antiguas cualidades de audacia y desparpajo.
Mis primeros conceptos, reforzados con ademanes que centuplicaban su
expresión, fueron para darle a entender que mi ciencia de hipocresía era
una vana fórmula, mientras no la justificara con faltas positivas y
delitos categóricos que\ldots{}

«¡Eh, eh!---me dijo más serena que yo.---¡Mucho cuidado, señor
pollo\ldots{} con espolones! Estese quieto, y no se me desmande tampoco
de palabra. Tome ejemplo de mí. Es hora de que yo vuelva a mi casa, y
usted forzosamente ha de irse a la suya, donde le esperan su mujer y su
hijo. A los disparates que me ha dicho contestaré muy poco; pero ello
será tal que habrá de agradecérmelo. ¿Quiere usted que seamos amigos,
que empecemos otro curso de amistad? Pues para hablar de eso, para
discutir si puede ser o no, si usted y yo merecemos el beneficio de esa
amistad\ldots{} quizás no lo merezca usted, quizás sea yo quien no lo
merece\ldots{} pues digo que para tratar de esto, es menester que nos
veamos otro día, o que nos escribamos. ¿Qué prefiere?

---Las dos cosas. ¿Va usted por las tardes al Casino de Embajadores?

---¡Ay, qué chiquillo!\ldots{} Basta: yo escribiré a usted.

---¿Al Congreso?

---Al Congreso. Y usted tomará las precauciones debidas para que no le
lleven las cartas a su casa.

---¿Y yo a dónde contesto?

---Déjeme que lo piense.

---¡Ay, qué pensadora se nos ha vuelto!

---Hijo, me llamo \emph{Alegría}, no me llamo \emph{Locura}. ¡Pues si yo
no pensara, qué sería de mí! Pensando, pensando, he llegado a donde
estoy. Si mucho he discurrido para subir, no tendré que discurrir menos
para no caerme.»

La extraordinaria donosura con que lo dijo desató en mí con mayor fuerza
los en mal hora resucitados ímpetus amorosos o de aventureros
amoríos\ldots{} Pero no me dio tiempo la dama moruna para la debida
manifestación, puramente verbal, de lo que yo sentía, y tirando del
cordón avisó al cochero para que parase\ldots{} Estábamos en la calle
del Arco de Santa María. «Bájate prontito, y no seas loco---me dijo
endulzando con el tuteo el amargor y crudeza de la
expulsión.---Obedéceme sin chistar, y te escribiré al Congreso.» ¿Qué
había de hacer yo más que resignarme? Triste cosa era quedarme a pie de
un modo tan brusco, aunque mi desairada situación fuese la más conforme
con los buenos principios\ldots{} Pero lo más singular de aquel paso, no
sé si comienzo fin o empalme de livianas empresas, fue que al
desaparecer de mi vista el coche de la moruna, se apagó en mi
pensamiento la ilusión que con tan vivo centelleo me había turbado.
Cierto que a una caída más o menos hipócrita quedaba no sólo expuesto,
sino comprometido, por ley caballeresca no muy ajustada a la eterna ley
moral; pero en medio de los velados desórdenes de un extravío de esta
naturaleza, no creo que deje de conservar intangible y puro el bien de
mi casa, ni la paz que allí me rodea. Si contemplando a Eufrasia y
oyendo su gracioso divagar de política, pude repetir para mis adentros
el verso de Leopardi \emph{E il naufragar m'e dolce in questo mare},
caminito de mi casa, y acercándome a este refugio bien templado, me
dije: «En ese mar bonito y placentero, podré pasearme sin que nadie me
vea; pero nunca naufragaré.»

Firme en estas ideas, y comprendiendo cuán penoso y desairado sería para
mí que María Ignacia tuviese conocimiento de mi paseo con la Socobio,
por soplo de algún paseante que me hubiera visto, eché por la calle de
en medio, y se lo conté yo con franqueza relativamente honrada. Claro es
que no le conté todo porque no era preciso; y cuidé de advertir que nos
acompañó \emph{en todo el paseo} la respetable señora Condesa o Baronesa
de San Juan Nepomuceno. Con gran sorpresa mía, no pareció mi mujer
enojada de aquel incidente. Tuve la suerte de cogerla en un momento en
que las expansiones de su grande alegría no daban a su alma tiempo ni
espacio para el recelo. Nuestro niño revela una resolución firmísima de
vivir, y aptitudes colosales para proveerse de medios de vida. Mama de
una manera insolente, bárbara, y se apodera de la teta con muy mala
educación. El ama es robusta, inagotable, y además, de buen natural.
Todas estas bienandanzas se reflejan en el alma de mi esposa, y ayudan a
su restablecimiento, franco, rápido y seguro. No quiere María Ignacia
abrir en su espíritu ningún hueco por donde entre la tristeza; no quiere
más que afianzarse en la posesión de sus felicidades, que estima bien
ganadas. Dios le concede lo que merecía.

Viéndola tan bien dispuesta, me permití ampliar un poquito las
referencias de mi paseo romántico, y ella con gran sentido me dijo:
«Procura no volver más, y si otra vez te invita, busca una manera
delicada de zafarte sin caer en grosería\ldots{} La verdad, esa
intriganta me ha tenido por algún tiempo en ascuas; pero esas ascuas ya
no me queman\ldots{} ¿En qué me fundo para sentirlo así? No lo sé; en
algo que se nos revela por el corazón, por las ideas y el cavilar de una
misma. Yo no creo en angelitos que vienen con recados a la oreja, como
es uso y manía de monjas; pero sí creo que Dios nos baraja los
pensamientos para que con ellos sepamos la verdad de las cosas nuestras,
de lo que nos llega a lo vivo, Pepe. Como te digo, las ascuas en que
estuve por esa maldita manchega, ya no me queman\ldots{} No viene el mal
por ese lado. O no habrá más ascuas, o cree que vendrán de otra parte.
Pero de ninguna parte vendrán, ¿verdad, marido mío?»

\emph{23 de Junio}.---Viendo crecer de día en día la estimación en que
mi suegro y toda la familia me tienen, siento en mí la autoridad; me
lanzo a platicar con el Sr.~D. Feliciano del delicado asunto de las
habladurías de su tertulia, pues sin que yo vea en ello, como Narváez,
el escándalo de una conspiración, pienso que tales enredos no armonizan
con la respetabilidad de la casa. Presentada exquisitamente la cuestión,
mi ilustre padre político concuerda conmigo, y alabando mi prudencia y
sensatez, se arranca con estas sesudas consideraciones: «Yo me encargo
de llamar al orden a estos mis amigos, y de hacerles comprender que, si
vienen mudanzas hondas en la política, no quiero que salgan de mi
casa\ldots{} Tengamos en cuenta que eres diputado, y ministerial de
añadidura, y que si algo ocurre y te ves en el caso de tomar la palabra
en el Congreso para defender la situación, no es bien que te acusen de
jugar con dos cartas\ldots{} Puedes decirle al señor Presidente del
Consejo, si de esto vuelve a hablarte, que si algunos sujetos graves, y
otros que no lo son, le tienden algún lazo para que se enrede y caiga,
los hilos no pasan por mi mano. Yo, bien lo sabe él, no soy partidario
del Parlamentarismo, ni creo en este Régimen de estira y afloja; pero
respeto lo \emph{existente}, por el hecho de ser \emph{existente}, que
no es poco. También nosotros tenemos nuestros \emph{hechos consumados},
como ahora se dice, dignos de todo respeto. ¿Qué sería de la Sociedad si
cada cual no permaneciera en los puestos adquiridos? El disputar los
puestos es lo que da alas al funesto Socialismo, y lo que fomenta la
Demagogia, ese \emph{virus}, Pepe, ese maldito \emph{virus} que hace
estragos en todo el mundo. Ya que la República Romana, centro de
ladrones y asesinos, está a punto de caer arrasada por nuestras tropas,
vean ahora estos gobiernos de poner aquí un poco de orden, y de refrenar
a tanto periodicucho, y de hacer entender a los del \emph{Progreso} que
se despidan del poder para siempre\ldots»

Conforme con todo lo sustancial de esta arenga me manifesté, añadiendo
que las clases pudientes \emph{somos} las llamadas a conducir el rebaño
social. Pero me recaté de expresar la idea que al oír a mi suegro me
andaba por el magín, esto es: que todos los pudientes, cuál más, cuál
menos, llevamos dentro el demagogo, y si me apuran, el socialista, que
son dos clases de \emph{virus}, de donde resulta que no habrá orden
verdadero hasta que no nos metan en cintura\ldots{} o nos metamos
nosotros mismos.

Esto pensaba, y ansioso de distracción, di con mi cuerpo en el Congreso,
donde me aburrí soberanamente; por la noche, previo el asentimiento de
María Ignacia, con quien yo consultaba siempre mis visitas nocturnas, me
fui a casa de María Buschental, donde encontré algunos amigos de mi
época de soltero, y otros con quienes había hecho conocimiento en las
Cortes: Escosura, Tassara, Borrego, Carriquiri. Departimos de cosas
sociales y políticas con la libertad que es el fresco ambiente de
aquella morada neutral de las opiniones, y si he de decir verdad,
también allí, entre tan amenos narradores y comentaristas, me sentí,
como quien dice, a dos dedos del hastío. Hallábame en un estado
particular de mi alma, sensación de ansiedad y de vacío, dolencia que de
tarde en tarde y sin ninguna inmediata razón ni causa conocida suele
acometerme, y que por lo común, lo mismo que viene se va, dejándome un
leve rastro de tristeza. Ni aun María Buschental, cuyo trato y gracias
amables con puntaditas maliciosas fueron y son siempre el antídoto de
las murrias, logró desvanecer las mías. Por último, confabulados ella y
mi amigo Escosura, aplicaron solapadamente a mi melancolía el
tratamiento de las bromas, sin excusar las del género más agresivo, y
hube de oír sátiras crueles en que no salía yo muy bien librado.

Según María, yo penaba por la Socobio, mujer corrida y de mucha
trastienda, maestra y grande erudita en todos los artes de amor. Según
Patricio, yo no he tenido con ella más que triunfos pasajeros,
regateados, y felicidades suspendidas de improviso para precipitarme a
la desesperación\ldots{} Yo negué, declarando que no hay tales triunfos
ni los he solicitado. Reían a carcajadas, y sin duda todo lo que dijeron
lo creían como artículo de fe. Así es el mundo: en la crónica social,
disfrutaba yo injustamente reputación de glorias y fracasos, como los
falsos héroes que con apócrifas grandezas usurpan un lugar en la
Historia. Así lo dije a la dama y a mi maleante amigo, añadiendo no sé
qué frivolidades para seguir la broma, y algún chiste, que no me salió,
francamente, pues no estaba yo para chistes. Por fin, agarrándome a la
primera coyuntura que se me presentó, me despedí cuando empezaban la
animación y el interés dramático en el gracioso mentidero de María
Buschental.

Deseaba yo verme en la calle y respirar aire menos impuro que el de un
salón. Sentía vivísimo anhelo de llegar a mi casa, de ver a mi mujer y a
mi hijo, y buscar mi solaz y recreo en la felicidad que nadie podía
disputarme. Sinceramente y sin la menor afectación, me reí de la
historia que mis amigos me colgaban, y ahondando con miradas atentas en
todo mi ser, por una parte y otra, advertí que la moruna no me
interesaba ni poco ni mucho, que la fascinación de sus gracias es
pasajera. Mas no porque observase todo esto, y de mi observación o
descubrimiento me alegrase, se mitigaba mi tristeza. «Es el pícaro
trastorno de nervios, o del cerebro, quizás desfallecimiento del
espíritu---me dije,---ese vacío, esa expectación inexplicable\ldots{}
Voy corriendo a mi casa, y allí se me quitará.»

Sentí detrás de mí una voz que me llamaba, y me estremecí cual si sonara
un disparo en mis oídos\ldots{} Era mi amigo, el pintor Genaro
Villaamil, que al salir del café de la Iberia, me vio pasar, y corrió en
mi seguimiento. Algunas noches solemos retirarnos juntos, pues somos
casi vecinos. Vive en el Postigo de San Martín. Hablome de no sé
qué\ldots{} algo de la expedición de Italia, de la Fuoco, de su peinado,
no menos famoso que sus pies\ldots{} Yo le oía sin ninguna atención, y
deseaba que me dejara solo. Parecíame que teniendo que oírle y
contestarle, por urbanidad, tardaría más en llegar a mi casa.

Íbamos por la calle del Arenal, él, más corto de piernas que yo,
acelerando su andar para seguirme, cuando una mujer pasó frente a
nosotros como a diez pasos de distancia\ldots{} Cruzaba de la acera de
San Martín a la de San Ginés, y nosotros íbamos ya muy cerca de la
iglesia de este nombre. La mujer que vimos se paró un instante ante mí y
me miró fijamente. Yo la vi a la claridad de la luna que inundaba la
calle, la vi, la miré y la reconocí\ldots{} Era Lucila\ldots{} Siguió la
moza su camino. ¡Cielos! entraba en la iglesia. Atravesó el patio, y
antes de llegar a la puerta volvió a detenerse y a mirarme. Antes dudara
de mi existencia que dudar que aquella mujer era Lucila, la hermosura
salvaje que descubrí en el castillo de Atienza, la sacerdotisa, la musa
histórica del gran Miedes, la perfecta hermosura, la ideal hembra, con
quien ninguna de las de nuestra edad y raza puede ser comparada\ldots{}
Mi amigo Villaamil, apretándome el brazo, exclamó con entusiasmo de
artista y de varón: «¡Qué mujer, Pepe! Nunca vi figura igual.» Habíamos
entrado en el patio; yo me abalancé hacia la puerta de la iglesia,
engañado por la ilusión de que Lucila me esperaba en aquella
penumbra\ldots{} Nada vi: la soberana imagen habíase apagado en la
cavidad del templo, como luz devorada por el vacío.

\hypertarget{xx}{%
\chapter{XX}\label{xx}}

La impresión que de aquella imagen quedó en mi retina y en mi mente fue
tan viva, que puedo describirla como si aún la tuviera delante. La que
en su cuerpo y rostro es la perfección misma, cifra y conjunto de
proporcionadas partes armónicas, vestía como las hijas del pueblo más
elegantes, entre manola y señorita, la falda sin vuelos, de medio paso,
un pañuelo por los hombros. No llevaba mantilla; el peinado, de lo más
sencillo, gracioso y coquetón que puede imaginarse\ldots{} Con ardiente
curiosidad y anhelo me metí en la iglesia, Genaro detrás de mí, y apenas
dimos algunos pasos hacia la capilla en que veíamos claridad, bultos, y
oíamos murmullo de rezos, la poca gente que allí había salió perezosa,
arrastrando los pies. El rosario, novena o lo que fuese había terminado.
Las luces se apagaban: el sacristán pasó junto a mí con un manojo de
llaves. En la vaga sombra, difícilmente se conocían las personas que
iban hacia las puertas\ldots{} Busqué inútilmente entre ellas a la que,
tan descuidada en su devoción, llegaba en las postrimerías del piadoso
acto\ldots{} Pero pensé que situándome en la salida no podía
escapárseme. A un tiempo, Villaamil y yo nos hicimos cargo de una grave
dificultad estratégica. San Ginés tiene dos entradas, y por consiguiente
dos salidas. Yo hubiera querido dividirme y vigilar ambas puertas.
«Usted mire por la calle del Arenal---me dijo el pintor con rápida
previsión militar;---yo miraré por la plazuela.» Así lo hicimos.

Vi salir a pocos hombres, en los que no me fijé, y mayor número de
mujeres que observé atentamente, cerciorándome de que todas eran viejas,
y las que no lo eran, no daban lugar a confusión a causa de su
ostensible fealdad. Por mi puesto de guardia, puedo jurarlo, no salió la
mujer de las soberanas proporciones. Cuando terminada la requisa, y
expulsado yo por el sacristán, me reuní en la plazuela con mi amigo,
este me comunicó que por su puerta no había salido la moza, podía
jurarlo. Mi desconsuelo y ansiedad fueron tales que no acerté con
ninguna explicación del caso, y sin el testimonio del pintor habríalo
tenido por un caso de alucinación. «Para mí, querido Pepe---me dijo
Villaamil,---esa mujer no ha salido\ldots» «¿Cómo que no ha salido? ¿Es
acaso alguna efigie que pernocta en los altares?» «Si no es efigie
sagrada, merece serlo. Ahora me confirmo en que no fue engaño lo que
creí ver. La moza, al entrar en la iglesia, avanzó derechamente hacia la
sacristía.» Un rato estuvimos discutiendo este enrevesado punto: ¿Tiene
la sacristía comunicación directa con la calle? Hicimos reconocimiento
topográfico, dando la vuelta a la parroquia por el arco y pasadizo.
Sostenía Villaamil que por una puertecilla que hay en la plazuela, muy
cerca del arco, había visto salir varios bultos; pero la distancia y el
sombrajo que allí hacen los muros le impidió distinguir si eran clérigos
o mujeres. La portezuela por donde se \emph{desvanecieron} estos
fantasmas estaba cerrada a piedra y barro. El balcón estrecho y las
desiguales ventanas que a cierta altura vimos nos indicaban que hay allí
una habitación aneja a la parroquia. ¿Será la vivienda del párroco?
Villaamil declaró con firmeza que a la mañana siguiente lo averiguaría.
Mis deseos eran averiguarlo al punto. De pronto, como quien encuentra la
solución de un problema obscuro, Genaro me dijo: «Oiga usted, Pepe: ¿se
habrá metido en la bóveda, en la célebre bóveda de los disciplinantes?»
«¿Y dónde está la bóveda?» «Viene a caer aquí debajo, y su entrada es
por la capilla del Cristo, donde estaban rezando cuando entramos\ldots»
«¿Y esa bóveda tiene luego salida por alguna parte?» «Dicen unos que
sale a las Descalzas Reales, otros que a San Felipe el Real; pero esto
me parece fábula\ldots»

Propúsome el pintor interrogar al sereno, pero a ello me negué, no por
falta de ganas: deseaba emprender solo mis investigaciones. La
intervención de Villaamil en un asunto que yo consideraba enteramente
mío me molestaba. Todo intruso que me disputara mi absoluto derecho a
descubrir a Lucila era ya mi enemigo. Fingiendo un poco le hice creer
que sólo un interés caprichoso y pasajero me había movido, y me le llevé
hacia la calle del Arenal, para dejarle en su casa antes de entrar yo en
la mía. Por el camino le hablé de todo menos de aquel misterioso
hallazgo y pérdida de la mujer bonita; pero él, sin poder apartar de lo
que vimos su potente imaginación de artista, exclamaba: «¡Qué cuadro! Es
la primera vez que veo en Madrid un asunto poético y una composición
prodigiosa\ldots{} La mujer furtiva es lo de menos\ldots{} ¡Pero la
plazuela iluminada por la luna, el arco de San Ginés, donde se alcanza a
ver el farolillo del sereno\ldots{} luz rojiza\ldots{} los desiguales
edificios, la disposición irregular de las casas y tejados\ldots! Es un
cuadro, Pepe, un soberbio cuadro\ldots» No tuve yo tranquilidad al
quedarme solo, y abrasado de celos precoces, no podía desechar el temor
de que Villaamil se me anticipara en la busca y rastreo de la mayor
belleza del mundo.

Entré en mi casa en una situación de ánimo que no permitía otro disimulo
que el darme por enfermo y necesitado de soledad y descanso. Mi mujer,
con tierna solicitud, dispuso que me trajeran tacitas de tila y de té.
No podía yo resistir su mirada penetrante, y cerraba los ojos con
afectación de dolor de cabeza, que no tardó en ser efectivo. Varias
veces he preguntado a María Ignacia si hablo yo en sueños, y me ha dicho
que no, que tan sólo doy grandes suspiros. Esto me tranquiliza, pues
tendría muy poca gracia que durmiendo nombrase yo a Lucila, o por ella
preguntase a imaginarios guardianes\ldots{} La noche fue malísima, y los
ratos de insomnio me atormentaban menos que los breves letargos con
angustiosa opresión y terrores. Ni un momento dejé de sentir la
presencia vigilante y cariñosa de mi mujer. Su ternura me incomodaba; le
mandé que se recogiese, afirmando que me sentía bien y que mi desazón
había pasado.

\emph{Otro día de Junio}.---Pienso que he perdido la razón, o que llevo
dentro de mí un ser nuevo, invasor intruso que ha desalojado mi antiguo
ser. No me conozco. Dudo si la continua presencia de Lucila en mi alma
es un suplicio intolerable, o un bien necesario que me ocasionaría la
muerte si desapareciese. Ninguna mujer se ha posesionado de mi
pensamiento y de mi voluntad con tan absorbente tiranía. Soy suyo, y por
mía la tengo desde el principio al fin del mundo. Porque desde su
emergencia en el castillo, fue para mí la ideal mujer, la perfección del
tipo, y ante ella no puede haber otra, ni la hubo ni la habrá. ¿Esto que
escribo es locura? Así lo pienso; pero una vez escrito no será tachado
por mi mano. Quiero manifestarme cual soy en el momento presente, y si
deliro ¿qué razón hay para que me obstine en aparecer discreto y sesudo,
tal y como mi señor suegro me ve, o quiere verme, representándome a su
imagen y semejanza? Salgan al papel mis desatinos, si lo son, en espera
de que el tiempo los convierta en concertadas razones.

La inutilidad de las diligencias que hoy he practicado en San Ginés y
contornos, me ha traído a un abatimiento lúgubre. Ni sacristanes y
monaguillos, ni el sereno, ni el celador del barrio, ni los tenderos
vecinos saben nada de semejante mujer\ldots{} He recorrido las calles
próximas, he dado vuelta a toda la manzana. Recordando que Lucila
apareció por el lado de San Martín, he reconocido también las calles de
Capellanes, Tahona de las Descalzas y otras, con la esperanza de
encontrarme al patriarca Ansúrez, o al hermanito pequeño; pero ningún
rostro de la familia celtíbera he topado en mi divagación por este
barrio. En casa logro componer mi pálido semblante, para que ni aun mi
mujercita, con su milagrosa perspicacia, entre en el sagrado de mis
pensamientos. Voy al Congreso, que es donde más solo puedo sentirme, y
huyendo de los amigos que en el Salón de conferencias y pasillos me
agobian con su enfadosa charla, busco un refugio en mi asiento de los
escaños rojos, y me sumerjo en las narcóticas aguas de la discusión de
Aranceles. Me creo dentro de una redoma, y mi atención es como la del
pececillo colorado que nada en redondo mirando el cristal que lo
aprisiona. Veo al cetrino Nicolás Rivero, al fornido Pidal, a Cantero
chiquitín, a Moreno López elegante, a Negrete proceroso, y oyendo el
run-run de un orador, para mí desconocido, cierro los párpados; el sueño
me rinde\ldots{} Al volver en mí me siento demagogo, me descubro
anárquico; no encuentro palabras bastante expresivas para calificar el
horripilante desenfreno y audacia de las ideas que se congestionan en mi
mente. Porque la somnolencia no acabe de aplanarme, huyo del
Teatro-Congreso, y me voy de paseo por la calle Mayor y Carrera de San
Jerónimo sin parar hasta el Retiro, donde encuentro amigos, algunos
diputados; hablo con ellos; sigo, empalmo con otros; vuelvo a charlar,
tomo y dejo, y lo mismo acompañado que solo, continúo sintiendo en mí el
llamear ardiente de las fieras pasiones revolucionarias. Los sombreros
de copa que cubren el cráneo de tanto señor y señorete me producen
indecible antipatía, y nada sería para mí tan sabroso como emplear mi
bastón en el apabullo de todos los tubos de felpa que me salgan al paso.
¿Hay nada más imbécil que la invención de esta ridícula tapadera de
nuestras cabezas?\ldots{} En mi negro humor, hasta las señoras se me
hacen odiosas y soberanamente grotescas, con sus modas de París y el
artificio vano de su exótica finura.

Sí, sí, debo de estar enfermo: esta noche, de las cenizas de la hoguera
en que prendí fuego a toda la sociedad de mi clase, ha surgido mi grande
amor al pueblo. Todo lo que no sea pueblo no es más que una comparsería
indecente, figuras de un carnaval que a lo chocarrero llama elegante, y
a las pesadas bromas da el nombre de cultura. Los días del vivir actual,
esto que con tanto énfasis llamamos \emph{nuestro siglo, nuestra época},
¿qué es más que un lapso de tiempo alquilado para fiestas? El plazo de
alquiler a su fin se aproxima, y en ese momento del quitar de caretas,
volveremos todos a ser pueblo, o no seremos nada\ldots{} Amo a Lucila
porque amo al pueblo: estos dos amores no son más que uno\ldots{}
Presumo que voy al mayor desconcierto de mi razón, y dejo la
pluma\ldots{}

Vuelvo a tomarla, después de una pausa de dos horas, y declaro que veré
con grandísimo gozo los disturbios y convulsiones que tanto temen
nuestros \emph{hombres públicos}. La tan maldecida República Romana
tiene todas mis simpatías, y los Mazzinis y Garibaldis son mis
ídolos\ldots{} Lleno estoy del condenado \emph{virus} que es la
desesperación de mi suegro ilustre, y con este veneno apaciento mis
ideas, con él mis deseos de que nuestras tropas, impotentes para reponer
a Pío IX en su eterna Silla, tengan que traérsele para acá, de que
húngaros y austriacos hagan polvo a los Radecskys y Metterniches, de que
todos los pueblos ardan y todas las artificiales categorías sucumban, de
que Francia sea inmensa barricada donde alcen su haraposa bandera los
socialistas, comunistas y falansterianos del mundo entero\ldots{} Ya
veis que voy de mal en peor\ldots{} Me siento insufrible: vuelvo a dejar
la pluma\ldots{} Suspendo esta confesión; pero conste que soy demagogo,
furiosamente demagogo\ldots{}

\emph{Otro día de Junio}.---Hoy, gracias a Dios, en mi alma turbada se
van apagando los incendios revolucionarios. No obstante, oyendo al
Sr.~de Emparán, que me ha dado matraca horrible con la carta filosófica
remitida por Donoso Cortés desde Berlín, y publicada estos días por
\emph{El Heraldo}, he sentido en mí un vivo anhelo de que lo maten, no a
Donoso Cortés, sino a mi suegro (a los dos no fuera malo), de que vengan
al Gobierno las hordas socialistas y le arrebaten cuanto posee, sus
riquezas todas, raíces, valores públicos, \emph{etcétera}, no dejándole
más que la camisa, y esto por el aquél de la decencia. ¿Qué?\ldots{}
¿qué tenéis que decirme? Ya entiendo: que Emparán en la miseria sería yo
miserable, reducido a la extrema necesidad de pedir limosna. ¿Y qué?
¿Pensáis que esto me arredra? Pues bien: seré mendigo, andaré descalzo,
gozando en la total ruina de los zapateros y en el acabamiento de todo
sastre. ¿No iban descalzos y muy ligeritos de ropa los iberos y celtas,
y eran felices, y se gobernaban admirablemente y vivían luengos
años?\ldots{} Si por algo, fijaos bien, rectifico esta idea destructora,
y dejo a la remota Posteridad el despojo y aniquilamiento de mi padre
político, es porque me aterra pensar que mi mujer y mi hijo anden
también descalzos y en paños menores por esos mundos. No: sálvense de la
catástrofe estos caros objetos, y si para ello es indispensable el
indulto del Sr.~de Emparán, recojo todo mi \emph{virus}, y perdonado
queda en este renglón. Para quien no tendré misericordia es para Donoso
Cortés, que en su famosa carta berlinesa me ha estomagado con sus
ñoñerías filosófico-ultramontanas. ¿Hay elocuencia más vacía ni retórica
más insustancial? Desde que ha sabido que Narváez le odia cordialmente y
se jacta de no haberle leído nunca, se aviva y enciende más mi cariño al
\emph{Espadón}, y voy creyendo que es el único grande hombre entre tanto
necio hablador y tanto acebuche barnizado. Sostuve esta tarde una viva
disputa en el Casino, defendiendo rabiosamente a Narváez, y abominando
de los que con desdeñoso humorismo llama la \emph{cáfila de
abogados}\ldots{} Éntrame ardiente anhelo de ver al Duque, y de platicar
con él de los diversos temas que hoy mueven las lenguas de nuestros
\emph{hombres públicos} y de nuestras \emph{mujeres}\ldots{}
\emph{privadas} (guarda, Pablo). De mañana no paso sin que yo me encare
con el \emph{buey liberal}, o en su defecto, con Bodega, que en este
momento de la Historia mía y de España también merece mi afectuoso
respeto. Él es pueblo, como yo, pueblo que resplandece en las alturas.

\hypertarget{xxi}{%
\chapter{XXI}\label{xxi}}

\emph{Primeros de Julio}.---Han pasado algunos días, no sé cuántos:
llevo mal ahora la cuenta del tiempo\ldots{} En este paréntesis corto de
mis Confesiones, mi pensamiento no ha estado libre de alternativas y
mudanzas. Sufrí recrudescencias de mi rabia demagógica, y he visto luego
que esta formidable pasión o dolencia remitía, dejándome volver a mi
normal estado de sensatez. Conviéneme declarar que ni en mis delirios ni
en mis sedaciones me ha faltado el cariño a mi mujer y a mi chiquillo,
sentimiento de un orden reposado, compuesto de deber y amor, y que ha
llegado a parecerme armonizable con mis ensueños. Cuando disponga de más
reposo, explicaré la filosofía que pongo en práctica para socorrerme con
ese cómodo sincretismo\ldots{} Lo más urgente ahora es que traslade al
papel un suceso mío, que no por mío precisamente, sino por suceso en sí
propio importante, debe ser comunicado a la indagadora Posteridad. Ello
es que al cabo quiso Eufrasia que se \emph{cumplieran las profecías}:
así llamo a las promesas de ella, y a las malignas suposiciones del
vulgo. Una carta que al Congreso me escribió, la respuesta mía, una
breve entrevista después en el paseo, determinaron lo que por lo visto
deseaba ella más que yo en aquel día, no muy lejano del presente.
Cogiome en tal estado espasmódico y cerebral, que mi primer impulso fue
no acudir al dulce reclamo. Después lo pensé mejor, y entendí que el
Acaso me deparaba quizás un grande alivio de mis murrias; deparábame
asimismo el gusto de dar la razón al penseque mundano, y de convertir el
cronicón apócrifo en historia verídica, espejo de la vida real. Me
molestaba la mentira ¡y era tan fácil trocarla en verdad!

Diome la verdad mi amiga una tarde en el Casino de Embajadores\ldots{}
Perdonad que me interrumpa para deciros otra vez, y van dos, que me
carga Donoso Cortés, y que ya estoy ahíto de la indigesta carta
filosófica que nos enjaretó desde Berlín. Infinitas veces se ha tragado
su lectura mi papá político, y algunos párrafos quedaron impresos en su
memoria como el Padrenuestro. Creeré que lo aprendió en viernes. Esta
mañana lo repetía en tono triunfal: «Si se me preguntara mi opinión
particular sobre el eclecticismo, diría que es una rama seca y deshojada
del árbol del racionalismo. Del racionalismo ha salido el spinozismo, el
volterianismo, el kantismo, el hegelianismo y el cousinismo, doctrinas
de perdición\ldots{} La sociedad europea se muere: sus extremidades
están frías, su corazón lo estará dentro de poco. ¿Sabéis por qué se
muere?» A esta pregunta que mi suegro hacía con entonación propia, como
si fuera de su cosecha, contestábamos al unísono mi mujer y yo: «No
señor: no sabemos nada.» Y él, hinchándose de vana elocuencia, como lo
estaban sus bolsillos de copiosos caudales, se contestaba: «Muere porque
la sociedad había sido hecha por Dios para alimentarse de la substancia
católica, y médicos empíricos le han dado por alimento la sustancia
racionalista\ldots»

Pero lo que más a mi señor suegro, reventando de rico, seduce y
entusiasma, es aquel pasaje sentimental en que nuestro rutilante orador
nos revela que hemos venido al mundo para llorar y padecer. La cosa
resulta clarísima y se demuestra con un ejemplo. «La vida es una
expiación---decía D. Feliciano con semblante fúnebre al repetir uno de
los trozos más enfáticos de la carta;---la tierra es un valle de
lágrimas. Si no queréis alzar la vista a los Cielos, ponedla en la cuna
del niño sin pecado\ldots{} ¿Qué hace el niño privado aún de
pensamiento, de razón y hasta de voluntad? Pues llorar\ldots» Argumento
incontestable: si el niño, que todavía es un ángel, llora, nosotros que
estamos llenos de pecados, ¿qué fin y destino tenemos más que hacer
pucheros en todo el curso de nuestra vida? Observaba yo que mi ilustre
suegro, con tanto recomendar el llanto a las personas mayores, se
abstenía personalmente de toda demostración de duelo, y nos decía, más
regañón que dolorido: «Esta es la verdad, la doctrina pura. Aprended,
aprended aquí.»

Perdónenme la digresión. Sigo contando. Quedamos en que fui a la calle
de Embajadores. Ya comprenderéis que de tan delicado asunto sólo debo
hablar lo preciso para establecer la debida coordinación lógica entre
las diversas partes de estas confidencias. Me permito saltar de la
primera a la segunda entrevista con Eufrasia, que fue ayer, y añado que
las alegrías de estos reservados encuentros dejan en mí un sedimento
amargo, y que no han apagado, no, el volcán que suscitó en mi mente la
fatal aparición de la salvaje Lucila. Os diré con confianza que los
halagos de la moruna, con ser en determinadas ocasiones de
extraordinaria intensidad sensitiva, me traen el hielo en inmediata
concatenación con el fuego, cual si fuesen eslabones que forman un
toisón de alternados metales. En sus encantos, a poco de gustarlos, no
me ha sido difícil ver el desabrimiento de las cosas de serie, que traen
de atrás su principio y continúan repitiéndose en la igualdad de sus
casos y consecuencias. Yo me sentía sucesor de alguien y predecesor de
otro u otros, y si mi herencia me parecía triste, más lástima que
envidia sentía de mis presuntos herederos.

\emph{Otro día de Julio}.---A la tercera vez, con más empeño que en la
primera y segunda, trato de indagar el móvil y fin de aquella
conspiración de zarzuela en que la moruna entretiene sus ocios. La
reciente intimidad no tiene bastante poder para quebrantar el secreto.
Eufrasia elude las preguntas, cambia de conversación, niega cuando se ve
estrechada; acaba por afirmar que todo concluyó, que fue una broma,
chismorreo de damas locuaces, que no saben cómo pasar el rato. Mis
coloquios en tan cercana disposición me permiten observar que es
recelosa, sagaz y reservada, que las pasiones no ahogan jamás su
discernimiento, que poniendo en sus empresas toda la perseverancia del
mundo, sabe esperar. Yo no me recato de confesarle mis simpatías por la
demagogia, sin descubrir el secreto psicológico de esta novedad, y ella
me alienta, declarándose también un poquito revolucionaria, sin precisar
ideas.

Permitidme que en una nueva digresión afirme otra vez, y van ciento, que
me encocoran lo indecible el Sr.~Donoso, Marqués de Valdegamas, y su
ciencia relamida. Si me ofrecéis recibo lo tomaré, y sigo en mi
cantinela\ldots{} Es que a diferentes horas, en las situaciones más
diferentes, invade mi alma el desdén de estas retóricas vacías. Ese buen
señor que a mis contemporáneos entusiasma, a mí me revienta: no puedo
remediarlo\ldots{} Y a propósito, para que no me acuséis de
inoportunidad: Eufrasia, tomando pie de no sé qué apreciación mía, me ha
dicho, mientras se arreglaba el desordenado cabello: «¿Verdad que es
hermosa la carta de Donoso Cortés?» Yo troné contra el ídolo de las
damas y de los grillos parlamentarios, y mi amiga lo defendió con
grandes hipérboles, repitiendo algunas de sus vaciedades más rotundas:
«Luzbel no es el rival, es el esclavo del Altísimo.»

---Bueno, ¿y qué? Concedo que no es el rival, sino el esclavo\ldots{} ¿Y
qué?

---Que el mal no es obra de Satanás: «el mal que \emph{el ángel rebelde}
infunde o inspira, no lo inspira y no lo infunde sino permitiéndolo el
Señor, y el Señor no lo permite sino para castigar a los impíos, o para
purificar a los justos con el hierro candente de las
tribulaciones\ldots» Así lo parla el maestro\ldots{}

---Eso va con nosotros: falta saber si somos impíos y merecemos azotes,
o justos que seremos purificados.---No seas tonto. Eso lo dice por las
revoluciones\ldots{}

---¿Qué más revolución que nosotros?

---No hables en plural: tú eres demagogo.

---Y tú descamisada\ldots{}

---¡Ay, qué pillo!\ldots{} El descamisado, el \emph{sans culotte} eres
tú\ldots{} Las palabras de \emph{Quiquiriquí} sobre el Sr.~de Luzbel no
van con nosotros. Es que algunos han dicho que la revolución de Febrero
del año pasado en Francia, la que echó del trono a Luis Felipe, fue un
castigo, y que después vendría la misericordia de Dios. Pues no es eso:
Donoso Cortés, con ese talentazo que no le cabe en la cabeza, ve las
cosas claras y dice que no habrá misericordia\ldots{} «Los que vivan
verán asombrados que la revolución de Febrero no fue más que una
amenaza, y que ahora viene el castigo\ldots»

---¡Ya escampa! Pongámonos en salvo.

---No te burles. Vendrá un cambiazo muy gordo que nos libre de tanto
pillo.

---Y en ese cambiazo trabajas tú y otras, a cencerros tapados\ldots{}
Destruiréis todo lo actual, y pondréis al frente de la Administración un
Ministerio de niños llorones presidido por \emph{Quiquiriquí}.

Soltó al oír esto una risa franca, fresca, sonora, expresión de abandono
y travesura.

«Déjame que cierre así la discusión---me dijo.---Mi nombre es
\emph{Alegría}\ldots» Y acabó por confesarme que también a ella le
revuelven el estómago los sermones de Valdegamas, y que si los celebra y
repite es por seguir la corriente; que toda aquella hinchazón
insubstancial no sirve para nada, ni traerá la más pequeña mudanza de
las cosas públicas. El mundo, según Eufrasia, se gobierna por pasiones,
no por ideas, y estas no influyen sino cuando son apasionadas. No echo
yo en saco roto esta sentencia, que me parece de un profundo sentido en
los tiempos que corren. Tiene la moruna mucho talento. Así lo declaro, y
ella con candoroso orgullo me dice: «¿Pues qué eres tú\ldots? Si yo
fuera Reina haría de España una gran Nación. Yo sabría ser mujer y
soberana, sin que la soberana y la mujer se estorbasen la una a la otra.
Yo poseería y practicaría el arte más difícil, que es el de escoger
hombres más o menos públicos, y en cada puesto estaría el sujeto apto
para desempeñarlo\ldots{} Yo los examinaría bien, y hasta que no
estuviera bien segura de sus cualidades no les daría el rango\ldots{}
Créete que yo haría una Reina admirable, como Isabel de Inglaterra, o
Catalina de Rusia; pero con la condición de ser soberana absolutamente
absoluta, porque de otro modo no respondería del acierto. ¿Libertad? No
habría más libertad que la mía. ¿Religión? La mía, y que fuera yo mi
propio Papa. ¿Ejército? Yo Generalísima. ¿Marina? Yo Almirantísima.
¿Gobierno? Yo Ministrísima\ldots{} Verías tú qué bien andaba todo. Yo y
el Pueblo, y entre este y yo un cierto número de lacayos instruidos que
sirvieran fielmente al Pueblo en mi nombre.» Preguntada por mí acerca
del lugar que a su esposo daría en este absolutísimo gobierno mujeril,
me contestó que en su Reino decretaría el cese de todos los maridos que
no fueran padres, y que a D. Saturno, por gratitud, le nombraría
Inspector General de Matrimonios, para divorciar a los que no tuviesen
prole\ldots{} Yo, como padre que soy bien acreditado, tendría un puesto
de importancia en la Nación\ldots{}

Con estas y otras tonterías pasamos el rato. El ingenio de esta mujer me
divierte\ldots{} pero el vacío de mi alma continúa sin llenar. Termina
la moruna diciéndome que se va a la Granja, donde está la Corte, y me
incita a que vaya también yo con mi familia\ldots{} Si María Ignacia y
sus padres desean lo mismo, ¿por qué no acabo de resolverme? ¿Qué
interés o querencia me amarran a Madrid? Respondo que sí, que no y qué
sé yo.

\emph{Otro día de Julio}.---Hoy, después de dos infructuosas tentativas,
he logrado satisfacer mi vivo deseo de hablar con Narváez, de quien
tenía yo las mejores ausencias, pues supe no ha mucho que en casa del
Duque de San Carlos me alabó y encareció infinitamente más de lo que yo
merezco. Antes de pasar a la presencia del \emph{Espadón} tocome un poco
de antesala, la cual se me hizo corta por la agradable compañía de mi
amigo y compañero de Congreso, Eusebio Calonge, el más joven quizás de
los mariscales de Campo. ¿De qué habíamos de hablar sino de la
expedición a Italia, general comidilla en estos días? Marchitas las
ilusiones de los que vieron en el envío de tropas a Gaeta un principio
de históricas hazañas militares, ¿qué hacían allí los españoles? Recibir
la bendición del Papa, ocupar a Terracina, y gastar su ardimiento en
marchas y contramarchas.

«El veto del General francés, cerrándonos el camino de Roma---me dijo
Calonge,---nos ha puesto en situación muy desairada. La expedición queda
reducida a un acto diplomático, y únicamente con ese carácter se la
puede defender hasta cierto punto. Mi opinión es que los actos
diplomáticos de un ejército sólo son eficaces después de actos
verdaderamente militares. La fuerza que pega duro es la fuerza que puede
negociar\ldots» Pareciome de perlas esta observación de mi amigo, que
revelaba la viveza de su entendimiento, y algo más habríamos divagado
sobre aquel asunto, si no nos interrumpiera D. Juan Bravo Murillo, que
salía de hablar con Narváez. Tocaba su vez a Calonge, que según me dijo
despacharía en cinco minutos. No llegaron a tantos los que empleamos D.
Juan y yo en recíprocas salutaciones. No he tenido ocasión de decir que
el ilustre extremeño y hombre público es antigua relación de los
Emparanes, y ha dirigido como letrado en ocasiones diversas, y en una
muy reciente, los asuntos de la casa. D. Feliciano le estima como amigo,
y le mira como a un santo en la religión de la jurisprudencia. Nada teme
mi suegro del rigor de las leyes teniendo en sus altares a San Juan
Bravo Murillo.

«¡Dichosos los ojos\ldots!---exclamó Narváez al recibirme;---y conste
que ya no le llamo \emph{pollo}. Por muchas razones merece usted el
empleo inmediato\ldots»

Hablamos de todo, de Eufrasia, de mi familia, de mi hijo, de los
Emparanes, de los Socobios, de todo menos de la \emph{campaña de
Italia}, punto delicadísimo que no me atreví a tocar, sabedor de lo
aburrido que anda mi hombre con este frustrado intento de intervención
gloriosa. En su tono, en su mirada, descubro la calma que ha sucedido a
su recelo de las conjuras, y siempre que la conversación recae en cosa
referente a mi persona, sus elogios me colman de gratitud, no inferior a
mi confusión, pues ignoro en qué funda el alto concepto que de mí ha
formado. Háblame de que desea utilizar mis dotes, esas dotes que con
increíble benevolencia y engaño llama extraordinarias, y cuando pienso
que su idea es ofrecerme un puesto diplomático, sale por un registro que
me causa tanta sorpresa como disgusto. ¿Sabéis a qué quiere aplicar el
Duque las facultades mías, que estima o parece estimar desmedidamente?
Pues a las funciones de un cargo palatino. La independencia que disfruto
me permite tomar a risa la propuesta de mi jefe y amigo, y manifestarle
que podrá hacer de mí lo que quiera, pero jamás hará un palaciego. Él se
ríe también; al despedirme me da palmaditas, repite en forma humorística
su pensamiento de vestirme de gentilhombre, sumiller de corps o cosa
tal, y con toda seriedad me dice: «Yo miro este asunto por el lado mío,
por el lado de la conveniencia oficial, y sostengo que es necesidad
imperiosa del Estado tener en aquella casa un personal inteligente,
instruido, que posea las buenas formas y las ideas liberales\ldots{} Ya
ve usted si es difícil\ldots{} digamos imposible. Adiós; que vuelva
usted pronto por aquí, y aunque no quiera hablaremos de lo mismo\ldots»
Salí: la idea del General, descartando radicalmente de ella mi persona,
pareciome idea luminosa y madura, de hombre de mundo, de hombre de
Estado.

Al anochecer, camino de mi casa, no falté a la estación que dos veces al
día, una por lo menos, hago en San Ginés, por la querencia misteriosa de
los lugares donde, visto una vez el paso de la felicidad, creemos que
allí nos está esperando para pasar de nuevo. Es aquel mi sitio de
peregrinación, y a él acudo por devota costumbre, o por impensado rumbo
de mis andares. No diré que hayan sido absolutamente infructuosas mis
pesquisas en la parroquia y sus aledaños, porque si ningún conocimiento
positivo ha venido a saciar la sed que me devora, creo haber descubierto
hilos menudos que a otro más grande, y finalmente al ovillo de esta sin
igual aventura, pueden conducirme. Desengañado de sacristanes y monagos,
así como de vecinos y porteras, me dediqué al trato de pobres de ambos
sexos que piden en aquel santo lugar. Repartiendo sin tasa calderilla y
algo de plata, he adquirido en tan mísera república relaciones muy
útiles\ldots{} Pero anoche encontré la puerta cerrada; la turba
mendicante se había retirado de sus puestos, faltándome hasta el más
fiel y consecuente amigo, que esperarme suele a deshora en la
escalerilla del patio por la calle del Arenal. De los hilos tenues,
imperceptibles casi, que este hilandero de chismes ha puesto en mi mano,
no quiero ni debo hablar mientras no sepa si han de conducirme a la
esperanza o a mayor desesperación.

\hypertarget{xxii}{%
\chapter{XXII}\label{xxii}}

\emph{16 de Julio}.---Decididamente nos vamos a la Granja. Habría yo
preferido pasar en Atienza los rigores del verano, por disfrutar de
mayor sosiego y dar a mi madre el gustazo de tenernos en su compañía.
Estos eran también los deseos y planes de María Ignacia; pero el unánime
voto de todo el señorío Emparánico en favor del Real Sitio de San
Ildefonso se impone a nuestra voluntad. Punto final en las discusiones,
y comienzo de los fastidiosos preparativos\ldots{} Mi mujer, o ignora en
absoluto mi devaneo con Eufrasia, o lo considera superficial y sin
importancia, aplicando al caso una filosofía suya, soberana,
elevadísima, que en rigor no puede admitirse más que estableciendo ley
conyugal distinta para cada sexo\ldots{} Cuido de rodear mi falta de
cuantas precauciones pueden preservarla del conocimiento y aun de la
sospecha de esta familia; pero creo difícil mantener la ignorancia más
allá de los temporales límites que encierran todo humano artificio.

Deseaba yo una ocasión de ver a Eufrasia antes de su partida, y hablarle
de estos temores, apelando a su buen discernimiento para que, mientras
dure la jornada en el Real Sitio, encerremos en mayor tapujo nuestras
intimidades, o las encubramos con la soberana hipocresía de suspenderlas
efectivamente. De fijo accederá, porque, como gran maestra de la vida,
es cautelosa, ve y entiende toda realidad, y en sus programas, según me
ha dicho mil veces, \emph{figura en primer término} la conservación de
mi prestigio y buena fama en la familia. La ocasión que yo buscaba se me
ha presentado esta tarde. Habiendo ido con mi señor suegro a visitar a
Bravo Murillo (para consultarle un pleito Emparánico entablado en el
Consejo Real), tuve el gusto de toparme allí con Don Saturno del Socobio
y su morisca esposa, que se despedían del extremeño, con quien están
todos los Socobios del mundo en buena amistad social y jurídica.

Pero antes de que yo refiera esta visita y las entretenidas pláticas que
en casa del insigne letrado y ministro tuvimos, oblígame el orden del
relato a contar alguna meditación mía muy interesante; que las
meditaciones, y aun los volubles escarceos de la mente, son materia o
documentación utilísima de la historia de un hombre, más o menos sincero
confesor de sí mismo. Es, pues, el caso que al despertar esta tarde de
la siestecilla con que suelo pagar mi tributo a los ardores veraniegos,
sentí en mi alma un bienestar hondo, cual si de ella, con la virtud de
aquel descanso, se desprendiera un formidable peso que la oprimía.
Sentíame no ya aliviado, sino totalmente restablecido de lo que yo
llamaba el \emph{mal de Lucila}, la monomanía, la horrenda pasión de
ánimo que encadenó mi pensamiento y todo mi ser a la imagen más soñada
que vista de aquella mujer. Y la súbita extinción de mi mal, habíamela
traído\ldots{} ¿A que no lo adivináis? Pues una idea, que al despertar
apareció posesionada de mi mente, y encendida dentro de ella como
vivísima luz, semejante por su potencia a las que en los faros alumbran
el paso de las naves. La idea que me iluminaba, única, despidiendo rayos
en mi cerebro, era esta: la enfermedad que yo he padecido no es más que
\emph{una efusión estética}.

«Mujer---dije a la mía, que en el momento de mi despertar se me apareció
con el chiquillo en brazos,---¿no sabes que ahora caigo en que soy un
artista sin arte\ldots{} un hombre que crece, vive y toma puesto en la
vida social fuera de su vocación? En mí has de ver un artista inmenso,
escultor, pintor, músico tal vez\ldots{} quiero decir que yo he debido
ser ese gran creador de arte, y por no serlo, me pongo malísimo, y hasta
parece que se me va el santo al Cielo.»

Echose a reír mi digna esposa, y sin dejar de zarandear en sus brazos al
crío, me contestó: «¡Pero, bobito, si eso que me dices no es idea
tuya!\ldots{} ¡Si eso te lo dije yo anoche cuando te acostabas! Y te lo
repetí no sé si dos o tres veces hasta que te quedaste dormidito. ¿Ya no
te acuerdas?

---Sí: algo voy recordando. Me hablaste de eso; pero no dijiste el
nombre del mal que tuve. El nombre de lo que padecemos es muy
importante, y creo yo que el hecho solo de saber ese nombre nos cura.
Esto que padecí se llama \emph{efusión estética}.

---No me vengas a mí con terminachos. Yo no sé más sino que no te
conviene estar ocioso. Tu mamá te conocía bien cuando te recomendaba que
escribieras la \emph{Historia del Papado}, y aun creía la pobre que la
estabas escribiendo. Yo soñé noches pasadas que habías hecho una
catedral tan magnífica, que las de Toledo y León parecían al lado de la
tuya buñuelos de piedra\ldots{} Y otra noche pensé, esto no fue sueño,
que si llegas a dedicarte a la estatuaria, habrías hecho
maravillas\ldots{} De todo entiendes, y sobre cada cosa discurres con
tanto tino que se queda una tonta oyéndote\ldots{} Más de una vez te
dije que has sido muy desgraciado, Pepe, porque primero quisieron
hacerte clérigo y te mandaron a Roma, donde no te encaminaron por el
lado del arte, sino por el de desempolvar bibliotecas; luego viniste
aquí, te dieron un empleo; nadie se cuidó de ver para qué servías; te
lanzaste al mundo; te hiciste señorito elegante; y por fin, sin que
lucharas por la vida, ni por el arte, ni por nada, te viste en buena
posición y casado con una fea\ldots{} ¡Ya lo creo que estarás enfermo,
Pepe! Y has de ir de mal en peor como no busques ahora otro rumbo, y te
ocupes en algo que sea boca de volcán por donde arrojes todo lo que
tienes dentro del alma.»

Respondile que cuanto me decía era exactísimo, menos que yo me hubiese
casado con una fea, y quien así lo afirmara mentía bellacamente. Varió
con rápido giro María Ignacia la conversación, diciéndome que su padre
me esperaba ya para ir a la visita del Sr.~Bravo Murillo. Vestime de
prisa y corriendo; a los veinte minutos ya estábamos en la calle suegro
y yerno. Por el camino iba yo pensando en mi enfermedad, la cual, al
paso por San Ginés, no me pareció radicalmente curada\ldots{} ¿Podría
creer al menos en una mejoría profunda y franca, precursora del perfecto
equilibrio? La idea que al despertar de mi siesta me trajo conciencia
luminosa de curación, había sufrido alguna mudanza, como el lento correr
de una veleta, y observándola me dije: «No era \emph{efusión estética},
sino \emph{efusión popular.»} Oyendo las campanudas majaderías que D.
Feliciano me echó por el camino, tocantes al Principio de Autoridad y a
las medidas que debían adoptarse contra el tremendo \emph{virus}, me
sentí otra vez dañado profundamente, y el síntoma denunciador de mi
recaída no era otro que un vivo afán de que reventara mi suegro, o de
que un alzamiento de las turbas le hiciese total liquidación de vida y
hacienda. En este morboso anhelo mío no entraba para nada la idea de
herencia: mi furor revolucionario contra el Sr.~de Emparán era
esencialmente desinteresado y justiciero\ldots{}

Adelante. Antes de que yo tuviese el honor de conocer a D. Juan Bravo
Murillo, me contó mi suegro que este grave señor se desayuna con media
docena de chorizos crudos y medio cuartillo de Valdepeñas. Pensaba yo
que quien con tan grosero y bárbaro comistraje se prepara el cuerpo para
los trabajos matutinos, no podía ser una inteligencia sutil, de
penetrantes destellos. Mas luego, viéndole, oyéndole y tratándole,
reconocí en él cualidades de hombre entero, sesudo, tenaz, de viril
discernimiento sin fantasía, que me reconciliaron con aquel hábito suyo
de la ingestión de chorizos cuando los demás tomamos café o chocolate.
La persona de D. Juan no puede ser más extremeña: como político es
compacto, duro, consistente; como orador, macizo, aplastante, pesado, de
una claridad pasmosa en los asuntos de ley escrita. Al jurisperito le
tengo por excelente, al político por uno de los más vulgares, hombre
aferrado a ideas viejas, y hecho a las rutinas como a los embutidos de
su país. La extremeña virtud de la voluntad le sirve para enranciarse
más cada día, y es lástima que tal virtud se aplique a convertir en
actos el pensar retrógrado y los sentimientos absolutistas. Menos
austero de lo que parece, goza no obstante fama de honrado, y lo es. Ha
podido ser millonario, y su fortuna, según dicen, no pasa de moderada,
en el sentido general. No escandaliza con su lujo, y su vanidad se
reduce a vestir bien: usa levitas de buen paño de Sedán bien cortadas,
guantes amarillos, botas de charol, y fuma puros de a cuarta, del mejor
habano. En sociedad es afable, muy distante de la zalamería; en la
Administración todo lo severo que puede ser aquí un Ministro, tratante
en favor y credenciales.

Encontramos la sala de D. Juan llena de gente, y a él recibiendo
plácemes por su recobrada salud. Había tenido un ataquillo de
\emph{grippe}, la enfermedad que ahora está de moda, y restablecido ya,
sus amigos políticos, sus clientes y una caterva de extremeños acudían a
felicitarle. Diputados vi unos doce, y al poco rato, con los que en pos
de mí llegaron, la cifra pasó de veinte. Allí estaba Cándido Nocedal,
que a mi parecer se pasa de listo, de fácil y seductora palabra,
progresista el 40, el 44 moderado de la fracción Puritana, en la cual
permanece; allí también Carriquiri, hombre rico y por lo tanto ameno,
alegre y de afable trato; allí D. Cristóbal Campoy, auditor de Guerra en
el ejército de D. Carlos, hoy moderado de los de peso, que andando se
tambalea como un santo que llevan en procesión; allí Don Félix Martín,
el diputado labrador, el \emph{villano de Illescas}, como suelen
llamarle, alto, moreno, con gruesos anteojos, y un levitón que debiera
ser de paño pardo para que el hombre estuviese más en carácter; allí Don
Santiago Negrete, diputado por Llerena, corpulento, cetrino, de voz
atronadora; allí los extremeños Ayala y Fernández Daza, este de figura
juvenil y semblante risueño; allí, en fin, D. Joaquín Compani, \emph{el
ingenuo del Congreso}, o hablando en francés, \emph{l'enfant terrible},
porque las verdades se le salen de la boca sin que pueda la discreción
contenerlas, hombre de una franqueza sublime, orador altísono y de voz
cavernosa, que se ha hecho célebre por haber soltado la bomba de que
sólo hay en España \emph{dos elementos de gobierno: el cansancio de los
pueblos} y \emph{la empleomanía}. Naturalmente, tal afirmación fue
terror y escándalo de los que viven dentro de la ficción y el
convencionalismo; pero no se arredró \emph{el ingenuo}, y sin pararse en
pelillos hizo brava defensa de la empleomanía, y sostuvo que es \emph{un
hecho} contra el cual nada pueden los declamadores, porque escaseando en
España los medios de vivir, hay que reconocer a los españoles el derecho
al presupuesto.

Ofrecidos mis respetos a D. Juan, dejéle con D. Feliciano hablando del
asunto contencioso, y pasé a saludar a mis amigos de la Cámara. Entró en
seguida D. Joaquín Rodríguez Leal, diputado extremeño, independiente,
progresista, amigo particular de Bravo Murillo, y tras él el Marqués de
Torreorgaz, menguadito de talla, de buen humor, contento de la vida,
como hombre adinerado. Este representante del país no deja transcurrir
ninguna legislatura sin presentar y apoyar una proposición de ley
declarando la absoluta incompatibilidad del cargo de diputado en los
empleos, honores y obvenciones. ¡Qué si quieres! Es un soñador, el
hombre de lo imposible, y D. Juan Bravo Murillo, según cuentan, ha
sudado más de una vez la gota gorda contestando a tales utopías. Son
amigos y paisanos, y no riñen más que en el Congreso. Llegaron luego
otros extremeños desconocidos, dos de ellos con sus respectivas señoras,
de la tierra de Hernán Cortés y Pizarro, y por fin hizo triunfal entrada
el matrimonio Socobio, D. Saturno risueño, claudicante, envejecido;
Eufrasia elegantísima, dominando desde el primer instante con su
desenvoltura graciosa toda la reunión. No fueron pocas las alabanzas que
D. Juan le tributó por su hermosura, y los piropos con que le rindió
pleitesía como dueño de la casa y admirador respetuoso del bello sexo.
Las extremeñas damas allí presentes, que aún vestían por la última moda
de Badajoz, o por las retrasadas de Madrid, no quitaban los ojos de la
vestimenta y accesorios de la manchega, reparando todo lo que llevaba.

Iniciamos la conversación por el tema fácil de los insufribles calores y
de lo bien que sienta un viajecito a la Granja en esta canicular
estación, y D. Juan saca uno de sus tópicos predilectos, que es
\emph{traer aguas a Madrid}. Asegura que el abastecernos de tan precioso
\emph{elemento de vida} se impone, cueste lo que costare, para que la
capital de las Españas no sea un pueblo sediento y sucio. A renglón
seguido se entabla una interesante porfía sobre la calidad de los cuatro
viajes que surten esta capital, y se marcan bandos o partidos, pues si
el uno defiende el sabor del Bajo Abroñigal o la Castellana, no falta
quien pondere la delgadez del Abroñigal Alto y la Alcubilla. D. Juan,
que ha estudiado detenidamente el asunto, nos dice que Madrid se
despoblará si continúa bebiendo por la primitiva medición de reales, que
se dividen en cuartillos y estos en pajas. La pobreza de aguas de la
Corte se evidencia con sólo decir que corren en ella, cuando corren,
treinta y tres fuentes, en las cuales hay ochocientos y pico de
aguadores que distribuyen en todo el vecindario trescientos treinta y
siete reales de líquido potable. Pero D. Juan presentará a las Cortes un
proyecto de ley para traer acá el Lozoya, sacándolo enterito de su lecho
y derramándolo por nuestras calles, plazas, paseos y jardines. Oyeron
esto los presentes como un cuento de hadas. La pintura que hizo Bravo
Murillo de los espléndidos chorros de agua que su proyecto realizado
habría de verter sobre Madrid, cautivó de tal modo al auditorio, que no
sólo se nos refrescaban las imaginaciones, sino también los cuerpos.

\hypertarget{xxiii}{%
\chapter{XXIII}\label{xxiii}}

Pero el marrullero y pesadísimo D. Saturno, que anda de algún tiempo acá
medio trastornado con la manía de antiparlamentarismo, y consagra sus
estrechas facultades y su holgado tiempo a proveerse de razones, datos y
copiosas estadísticas que demuestren la inutilidad o más bien el
perjuicio de las llamadas Cortes, ora sean Constituyentes, ora
Ordinarias, echó sobre el proyecto del Lozoya no diré un jarro de agua,
sino cántaros de fuego, asegurando que de la Representación Nacional no
puede salir traída de aguas ni de ninguna cosa buena, sino traída de
barullo, confusión, corruptelas e inmoralidad.

«Y no lo tome a mala parte, D Juan, que contra usted no voy, porque
usted no ha inventado el Parlamentarismo, ni en él\ldots{} las cosas
claras\ldots{} se encuentra muy a gusto, por más que lo calle, vamos,
que no pueda decirlo\ldots{} ¡Pero qué bien gobernaríamos sin Cortes, D.
Juan, y qué derecho andaría todo el mundo!

---Eso habría que verlo\ldots{}

---Muy pronto se dice; pero en la práctica\ldots{}

---No está el mal en las Cortes, sino en el maldito Reglamento.---Por mi
parte, que las supriman.»

Estas y otras observaciones que como granizada caían sobre la opinión de
D. Saturno, salieron de los grupos en que estaban Torreorgaz, Negrete,
Compani, Campoy, D. Félix Martín y Carriquiri.

«Si me dejan meter baza, señores---indicó la moruna,---les diré que mi
marido no condena el Parlamentarismo en principio\ldots{}

---¡Oh, sí! en principio, en principio y en fin. Es malo, malo \emph{per
se}---vociferó Socobio,---y en ningún caso puede ser bueno. No hagan
ustedes caso de mi mujer, que está un poco tocada, y transige, transige
con el mal, por aquella falsa teoría de que se puede consentir un mal
relativo para evitar un mal absoluto.

---Bueno---prosiguió Eufrasia, sin hacer gran caso del
orador:---reneguemos del Parlamentarismo en principio y en postre, pues
todo lo que conocemos de él es ruin y corrompido\ldots{} Se puede
demostrar que las Cortes actuales no son más que un Régimen de comedia,
porque los procuradores de los pueblos o distritos no los representan
más que en el nombre; todos salen elegidos por obra y gracia del
Gobierno, que primero los trae y luego los paga\ldots{} Señores, no hay
que ofenderse\ldots{} Cuando quieran se saca la cuenta parlamentaria, y
se demuestra que de los trescientos y tantos señores que dicen \emph{sí}
y \emph{no}, los más son funcionarios, y por tanto cobran\ldots{} Todo
es engañifa\ldots{} No hay farsa más repugnante que esta de las
Cámaras\ldots{}

---¡Señora, por Dios\ldots!

---¡Señora\ldots{} por decirlo usted, puede pasar\ldots{} Pero\ldots{}

---¡Señora\ldots!

---¡Si nadie tiene por qué ofenderse! ¡Oído!---exclamó D. Saturno,
echándose mano al bolsillo de la levita.---Soy el litigante monomaníaco,
y digo como él: «¿Hablaba usted de mi pleito? Aquí traigo los papeles.»
Yo, señores, soy un hombre muy práctico, y de mucha paciencia. Soy un
hombre, señores, que cuando digo una cosa la pruebo, y\ldots{} aquí
traigo los papeles. Llevo ya algunos meses recogiendo datos, y formando
mi estadística\ldots{} Voy siempre prevenido, señores. Papel canta.
Contra la realidad, contra los números, no hay aquello de \emph{tal y
qué sé yo}\ldots{} Esto es indiscutible\ldots{} Si el Sr.~D. Juan me lo
permite, y estos caballeros me honran con su atención, les leeré mi
cuadro sinóptico.»

Sacó un doblado papelote, y mientras con solemne pausa lo desplegaba, su
mujer dijo: «No es necesario leerlo. Hartos están de saber los señores
del margen, que si se exceptúan tres o cuatro próceres, como Berwick,
Bedmar y Vistahermosa, media docena de propietarios ricos, y otra media
de fabricantes, los cuales, entre paréntesis, vienen al Congreso
engañados y para dar a \emph{la reunión} algún viso de independencia;
exceptuando esos poquitos, todos, todos cobran sueldo en una forma o en
otra.

---Señora, yo no sé lo que es un sueldo---dijo respetuoso el
\emph{Villano de Illescas}.

---¡Sr.~Martín, feliz garbanzo que no figura en esta olla!

---¿Y yo, señora?---preguntó risueño Rodríguez Leal, rico hacendado de
Badajoz.

---Tampoco usted cobra\ldots{} directamente; pero se le da su
partija\ldots{} no se ofenda\ldots{} en empleítos para repartir en casa.
Que levante el dedo el \emph{independiente} que no lleva tras de sí una
cáfila de primos, sobrinos o cuñados, que piden y toman destino.

---Señora, ¿pero se ha de hilar tan delgado que\ldots?

---Saturno---prosiguió la dama,---para que se convenzan de que el
Congreso no es más que una legión asalariada, léeles tu estadística.

---Que la lea, que la lea.»

Y D. Juan Bravo Murillo se volvió para mí, que a su lado estaba,
diciéndome risueño: «¿Para qué endilgarnos el mamotreto? \emph{Peor es
meneallo}.

---En el trabajo que ha hecho mi marido con escrupuloso esmero y
paciencia, se ve lo que todos cobran, y también\ldots{} aunque sea mala
comparación\ldots{} el plato donde comen.»

Breve silencio. Entra pomposo y risueño en la sala D. Nicolás Hurtado,
diputado por Zafra, el cual, después de saludar al señor Ministro, se
encara con Eufrasia y le dice graciosamente: «Amiga mía, ya está usted
con la cantinela de si comemos o no comemos\ldots{} Deje usted vivir a
todo el mundo, criatura, que estando bien comidos, mejor podremos
admirar y festejar a usted\ldots{}

---Gracias, D. Nicolás\ldots{} Siéntese a mi lado, y vote conmigo.

---Sí lo haré. Ya sabe usted que no cobro.

---Así consta en el decreto de su nombramiento\ldots{} No podía ser de
otro modo para poder estar sujeto a reelección\ldots{} Pero en nuestro
delicioso país para todo tenemos trampa; y así, por bajo cuerda,
mediante un solapado artificio, percibe usted\ldots{}

---Veinticuatro mil reales como Oficial Primero en la Sección de lo
Contencioso del Ministerio de Hacienda---dijo D. Saturno impávido.---Y
no hay que asustarse, Nicolás, que aquí no nos ponemos colorados por
estas cosas.

---Explicaré a ustedes\ldots» rezongó el señor Hurtado, llevándose la
mano a las gafas.

Por lo bajo le dijo la moruna no sé qué conceptos afables y donosos, que
le redujeron a prudente mutismo, y siguió lo que podremos llamar
información alimenticio-parlamentaria. El ingenuo Compani,
\emph{l'enfant terrible} del Congreso, afirmó que por sí no cobraba;
pero que entre parentela y amigos tiene como unos treinta chupones sobre
su conciencia, sin que por esto abomine del Parlamentarismo, porque la
vida moderna requiere un nutrido presupuesto para dar de comer a los que
carecen de bienes de fortuna, y no son hábiles para ninguna industria,
ni aun siquiera para la de pescadores de caña.

«Allá voy, allá voy---dijo D. Saturno impaciente.---En mi Cuadro
Sinóptico figuran veintinueve sanguijuelas parlamentarias que chupan por
Gobernación.

---Hombre, me parecen muchos para un solo Ministerio---observó
Carriquiri.

---Papeles hablan, y numeritos cantan---dijo Socobio.---Y si hay un
guapo que se atreva a rectificarme lo que tengo escrito, aquí le
espero\ldots{} Adelante. Por Gracia y Justicia cobran treinta y dos
padres de la patria, comprendidos jueces, oidores y empleados del
Ministerio.

---No puede ser.

---Se le ha ido a usted la mano en la estadística, amigo D. Saturno.

---Pues yo aseguro que los de Gobernación me parecen pocos---afirmó la
moruna.---¿A que me pongo yo a contar y saco más?

---¡No por Dios!

---Verán\ldots{} el Sr.~D. Ricardo de Federico, \emph{treinta mil}
reales; el Sr.~Fernández Espino, \emph{treinta mil}; \emph{cincuenta
mil} el Sr.~Gaya, director de la \emph{Gaceta}; el Sr.~D. José Juan
Navarro, \emph{cuarenta mil}; el Sr.~Ruiz Cermeño,
\emph{cuarenta}\ldots{}

---Basta.

---Collantes, \emph{cincuenta mil}; D. Joaquín Cezar, \emph{cuarenta};
Álvaro, Anduaga\ldots{} Bueno, señores: me callo. Saturno, échanos los
de Gracia y Justicia.

---Bastará decir que son treinta y dos.

---Se te ha olvidado agregar a D. Manuel Ortiz de Zúñiga, que ahora se
\emph{nutre}\ldots{} por la Comisión de Códigos.

---No se olvida nada. Ahora van los de Hacienda, que son ¡ay!
veinticuatro, y con cada sueldazo que da miedo.

---Pero en esa lista estarán comprendidos los ex-ministros que disfrutan
su cesantía---indicó el Sr.~Campoi.

---No están incluidos---replicó Socobio.---Esos componen otra serie de
comilones. Constan también aquí los ex-ministros que no perciben
cesantía, \emph{rara avis}, los señores Mendizábal, Cantero\ldots{}

---Ya que estoy en el uso de la palabra---dijo el ex-carlista
Campoi,---protesto de que se me haya metido entre los que manducan en
Gobernación. Yo no cobro más que en el concepto de Jefe político cesante
de Granada, a donde fui sacrificando mi salud, sacrificando mi
tranquilidad, y sacrificando mis ideas. Si no tuviera que contender con
una bella y distinguida señora, yo sostendría\ldots{} Pero vale más que
renuncie a la palabra y\ldots{} He dicho.

---Sigamos. Adelante, D. Saturnino.

---En Instrucción Pública tenemos quince; en Guerra, veintidós; en
Marina, ocho; en el Consejo Real\ldots{} tantos como Consejeros\ldots{}
Señores, esto da grima. ¿Qué Parlamento es este, ni qué Representación
Nacional, ni qué niño muerto? Pues vean más: Empleados en Palacio, seis;
en Estado, nueve.»

Nocedal, Carriquiri, Negrete y el mismo D. Juan sonreían entre burlones
y melancólicos, como si juntamente vieran la extensión del mal y la
imposibilidad de remediarlo. Las damas extremeñas, del antiguo tipo de
señoras, calladitas y vergonzosas, no hacían más que sonreír, abanicarse
con pausado ritmo, y apoyar las exclamaciones de los más próximos con
algún término de su cortísimo vocabulario social, con un
\emph{¡enteramente!}\ldots{} \emph{¡qué cosa!}.. \emph{¡es muy
extraño!}\ldots{} Si antes admiraron y repararon el atavío de la bella
manchega, cuando la oyeron despotricar con tan picante y hombruno
desenfado, no volvían de su asombro, y la diputaban mujer de poco seso,
contaminada de la chocarrería francesa.

Antes se trocarían en caudalosos ríos los viajes de Madrid, inundando
las calles de la Villa y Corte; trocáranse los aguadores en marineros y
los coches en góndolas; antes el calor africano que sentíamos, en
celliscas y hielos de Diciembre se convirtiera, que renunciar D. Saturno
a la cumplida explanación de sus estadísticas ante cada uno de los
grupos en particular, y luego persona por persona, mostrando las notas y
comprobantes que sobre sí llevaba, y deteniéndose a convencer con mayor
esfuerzo de razones a D. Juan Bravo Murillo, que oía, suspiraba, y
moviendo la pesada cabeza decía que había que verlo, que una cosa es
predicar y otra dar trigo\ldots{} Opinaba lo mismo Emparán, fiel eco del
eximio letrado y político, y detrás repetía lo propio el coro de
Carriquiri, Campoi, Negrete y otros. Torreorgaz pretendía convencer a D.
Nicolás Hurtado de que si cuajara su salvador proyecto de
incompatibilidad absoluta, el Parlamento sería lo que debe ser, y D.
Nicolás Hurtado fruncía el entrecejo, acabando por afirmar que con
Parlamento libre \emph{iríamos a la Convención}, sí señor\ldots{}
\emph{¡y a los horrores del 93!} El ingenuo Compani, a quien nadie hacía
caso, explicaba a las señoras su plan de reglamentación de la
empleomanía, y Nocedal, siempre ferviente devoto de las mujeres
graciosas y bonitas, se fue derecho a Eufrasia diciendo que a Saturno se
le había olvidado la estadística más interesante, la de los diputados
maridos, la de los viudos con enredo, o solteros en estado de merecer.
Al lado de cada cifra de sueldo debe ponerse: «¿Quién es ella?

---Cándido---replicó la moruna,---no tome usted a risa nuestro Cuadro
Sinóptico, que es un monumento de sinceridad. Hay que decir las cosas
claras, para que pueblo y reyes y hombres públicos abran los ojos y
vean. Y no me diga usted que algunos pocos, muchos si se quiere, no
figuran en nómina. Esos que parecen estar curados de empleomanía,
padecen de otro mal mayor, lo que llama Sánchez Toca la
\emph{empleopesía}, o furor de apandar destinos para fomentar la
vagancia de provincias enteras. Hable usted de esto a los hidrópicos de
credenciales, a los Mones y Pidales y Canga-Argüelles, a D. Fernando
Muñoz, a los Collantes, a Sartorius, al mismo D. Juan, a Venavides, con
ser puritano, y verá usted que el Régimen es una farsa, un engaña-bobos.

---Crea usted, señora, que yo no defiendo el Régimen, ni lo creo
perfecto; pero tal como es, con él hemos de seguir mientras no nos
descubran otro mejor. Esos que no llamaré lunares, sino verrugas y
lamparones que afean el bello rostro del Régimen, son inherentes a toda
innovación, y se irán corrigiendo con el tiempo. Como decía D. Juan
Nicasio, dentro de unos trescientos años se habrá completado la
educación del país, y las espinas de hoy serán entonces rosas y
claveles. No todas las cosas del mundo son como la mujer, que en el
principio fue bella, y bella y seductora es hoy\ldots{} como la muestra.

---Gracias, Candidito.

---Pero la mujer es obra de Dios, mientras que el Parlamento es obra de
los hombres: por eso es tan imperfecto\ldots{}

---Pues suprimirlo.

---Mejor será corregirlo. ¿Cuánto mal no se ha dicho de las mujeres? Y
buenas o malas, tuertas o derechas, sin ellas no podemos vivir. ¿Qué
defecto ve usted en el Parlamento? ¿Que en él se habla demasiado?

---Eso no es defecto, porque yo\ldots{} ya ve usted si hablo sin ton ni
son, y digo mil disparates\ldots{} ¿pero eso qué? Yo siempre estoy
dentro de la legalidad. Soy quizás demasiado rigorista en mis actos,
aunque en la palabra parezca un poquito casquivana.

---Usted no parece más que una belleza superior, y por eso tiene algún
derecho a no ser tan rigorista\ldots{} Así como hay bulas para difuntos,
haylas para las mujeres que unen a la belleza el ingenio.

---¿Bula yo? No la quiero ni me hace falta. La bula es dispensa de algo,
y yo, cumpliendo, como cumplo, mis deberes, no necesito\ldots{}

---Quiero decir\ldots{} ¿No sabe usted que el justo peca siete veces?

---Yo ni siete ni ninguna, Cándido; y por justa me tengo.»

\hypertarget{xxiv}{%
\chapter{XXIV}\label{xxiv}}

Desfilaban los visitantes; mas D. Saturno embistió al Ministro y a mi
suegro con su salmodia de moscardón, sin darles respiro, de lo que me
alegré mucho, porque así pudimos tener Eufrasia y yo algunos apartes, y
comunicarnos las respectivas instrucciones y consignas. Muy contenta de
que fuese yo a la Granja con mi familia, me dijo: «Allí no hay que
pensar en tonterías. Virtud a todo trance, y edificación completa.
Déjalo de mi cuidado, y verás que bien me arreglo para que tú en tu
terreno y yo en el mío edifiquemos con nuestra conducta intachable. Ya
nos veremos allá, en el teatro, en los jardines, y hablaremos\ldots{}
pero poquito y con la mayor cautela. Hasta la Granja, Pepe\ldots{} ¡Ay!
¿no ves? Mi Saturno se ceba en el pobre D. Juan y en D. Feliciano.» En
efecto: miré con disimulo las caras de las víctimas, y vi que a D. Juan
lo había volteado ya dos veces, recogiéndolo para despedirlo de nuevo.
Rogué a mi amiga que echase un capote, y así lo hizo, librando de la
cogida feroz a tan respetables señores. Poco después de esto, marido y
mujer salieron, y quedándonos solos con D. Juan mi suegro y yo,
escuchamos las observaciones que el extremeño nos hizo acerca de la cosa
pública. No ve claro\ldots{} El verano, políticamente hablando, viene
cargado de nubarrones. Los grupos disidentes de Venavides, González
Brabo, Ríos Rosas, ayudados de Gonzalo Morón y Bermúdez de Castro, dan
mucha guerra. La mayoría va sacando los pies de las alforjas, y no hay
ya destinos con que amansarla y sostener en ella esa satisfacción
interior que es el nervio y alma de todo ejército\ldots{} Las actuales
Cortes envejecen ya, y están minadas por las malas pasiones. Hay que
traer nuevas Cortes el año próximo\ldots{} ¿Pero quién puede hacer
cálculos para un año más, en este país de lo imprevisto? Teme que las
tempestades que se anunciaron no ha mucho estallen ogaño\ldots{} Los
revolucionarios no desmayan; la sociedad, apenas curada de una fiebre,
se inficiona de otra\ldots{} Y esto ¿qué es? Es, a su juicio, que el
pueblo español no quiere curarse de su principal defecto, la
exageración.

Oyendo esto, mi suegro echaba lumbre por los ojos, señal de la
conformidad de sus ideas con las que expresaba D. Juan. El cual,
vanaglorioso como si acabara de descubrir un mundo, continuó así: «Sí,
amigos míos, la exageración es lo que nos pierde a los españoles. Aquí
el religioso cree que no lo es si no le damos la Inquisición, y el
filósofo no ha de parar hasta la impiedad y el descreimiento; el militar
quiere guerras para su medro personal, y el civil revoluciones para
desarmar al ejército; el negociante no está contento si no alcanza
ganancias locas por la usura y el monopolio; el hombre público no piensa
más que en acaparar toda la influencia, dejando a los contrarios en
seco. En todo la exageración, el fanatismo\ldots{} Si Dios quisiera
hacer de España un gran pueblo, nos haría lo que no somos,
sensatos\ldots{} Pero búsquenme en esta Nación la sensatez. ¿Dónde está?
En ninguna parte. No veo sensatez en los partidos; no la veo en la
Prensa; no hay sensatez en el Gobierno\ldots{} no hay sensatez,
digámoslo aquí en confianza, ni en la Familia Real\ldots{} ¿Y cómo le
decimos al pueblo bajo que sea sensato si los que andamos por las
alturas no lo somos?\ldots{} En fin, amigos míos, buenas tardes\ldots{}
Es un poco insensato tanto charlar\ldots{} Ya saben que me tienen
siempre a sus órdenes.»

En la calle, oyendo repetir a Emparán la muletilla de la sensatez, con
hipérboles harto empalagosas, me sentí repentinamente recaído en mi
demagógica dolencia, y se me representó como el más gustoso espectáculo
la ejecución de mi suegro, en garrote vil, haciendo artístico juego con
D. Juan, en dos lados del mismo patíbulo, y ambos echando un palmo de
lengua con muchísima sensatez\ldots{} En casa, el mal me acometió con
mayor furia, y del exterminio general no exceptuaba yo más que a mi cara
esposa y a mi hijo. Como no quería salir de Madrid sin despedirme de
Narváez, a quien debo tantas atenciones, me fui a la Presidencia: no
estaba. Dejé recado a Bodega; volví más de una vez, y al fin, a media
noche, antes de retirarme al descanso, el General me hizo la distinción
de recibirme a mí solo, entre tantos postulantes de audiencia, y tuve el
gusto de platicar con él, viéndole en zapatillas, sin peluca, con
holgado traje de \emph{nankín}.

«Yo también iré a la Granja---me dijo,---pero lo menos posible\ldots{}
Allí no va uno más que a ver cosas desagradables\ldots{} Hay que decir a
todo \emph{amén}, repudriéndose uno por dentro. Esta vida de Gobierno es
muy perra. Aquí el gobernante está siempre vendido, porque cuando no hay
revoluciones hay intrigas, y estas salen de donde menos debieran salir;
cuando no le atacan a uno de frente o por el costado, le minan el
terreno\ldots» Aquí se detuvo, creyendo sin duda que había dicho
demasiado. Pareciome que se esforzaba en desechar tristezas, y que
buscaba temas susceptibles de charla jovial. De pronto me sorprendió con
esta familiar salida: «Bien, \emph{pollo}, bien. ¿Sabe usted que ahora
me dan ganas de volver a llamarle \emph{pollo}?\ldots{} No sé si es
porque le veo más joven, o me siento yo más viejo\ldots{} Antes que se
me olvide: lo que me dijo usted hace días se ha confirmado plenamente.
Ya no conspiran en casa de Emparán, ni tampoco en las de Socobio. Toda
esa gente arrimada a la cola es muy cuca: no quiere comprometerse. ¿Sabe
usted dónde se reúnen ahora los zorros? En la Escuela Pía de San Antón.
Creen que cuando toquen a escurrir el bulto los salvará el lugar
sagrado. No me conocen. La suerte de ellos es que ya no les hago caso.
Sí, hijo: me les he metido en el bolsillo. Nada temo por ese lado. En
Aranjuez hablé con Su Majestad\ldots{} Ella, naturalmente, me dio la
razón, y con la razón la seguridad de que no tendremos un disgusto. La
Reina es un ángel; pero\ldots{} no está averiguado que los ángeles
sirvan para ceñir la corona en una Monarquía constitucional\ldots{} Pero
en fin, es buena, y como ella pueda hacer el bien, crea usted que lo
hace\ldots{} No falta sino que pueda hacerlo, que la dejen\ldots{} que
no se atraviese alguna influencia mala\ldots{} y vaya usted a responder
de que no habrá malas influencias en ese maldito Palacio donde entra y
sale todo el que quiere\ldots{} En fin, de esto no puedo decirle a usted
más.»

Charlamos un poco de política, expresé mi recelo de que no pudiera
gobernar más tiempo con las actuales Cortes, y él, expansivo y
desdeñoso, me contestó que con estas y con otras es muy difícil el
gobierno\ldots{} Le informé de la Estadística de D. Saturno, y no le
pareció mal; que las verdades suelen decirlas los niños y los tontos. De
lo que hablamos deduje su desprecio del Parlamento, mecanismo que hacía
funcionar sin conocer bien su objeto, pues los que lo pusieron en sus
manos no le habían demostrado para qué servía, y los que hoy le ayudan a
moverlo no están de ello muy bien enterados. ¡El Parlamento! Funcionando
por sí, no permitiría gobernar; funcionando a fuerza de mercedes, no
sirve para nada. Tal como tenemos hoy el Régimen, no es otra cosa que el
absolutismo adornado de guirindolas liberales\ldots{} Así lo manifesté
al General, correspondiendo a la franqueza que me daba y pedía; y él,
después de una pausa en que su mente parecía perderse en penosas
vacilaciones, me dijo: «Yo quiero poner muy alto el Principio de
autoridad, porque sin eso, ya usted lo ve, no hay país posible; pero al
propio tiempo quiero ser liberal, muy liberal, más liberal que nadie.»

Iba yo a contestarle, viendo clara una gallardísima respuesta; pero a
las primeras palabras se me fue el santo al cielo; se evaporaron mis
ideas y me llené de confusión. Yo no sabía cómo puede un gobernante ser
liberal, muy liberal; yo ignoraba lo que es Libertad\ldots{} «¿Pero qué
es Libertad, mi General?---le pregunté por disimular mi turbación. Y él
me respondió: «Pues Libertad\ldots{} Ello es, es\ldots{} Yo lo siento,
pero la definición no me sale, no doy con ella. Dígame usted ahora qué
entiende por Principio de autoridad\ldots» «¡Ah!---repliqué yo más
confuso a cada instante.---Principio de autoridad es pura y simplemente
el aforismo de \emph{quien manda manda}\ldots{} Ahora el porqué del
mando, el origen de la autoridad, yo no lo veo claro. Usted recibe la
facultad de mandarnos a todos; la Reina, que hoy le da a usted el
bastón, ya sea garrote o junquillo, mañana se lo quita. ¿Por
qué?\ldots{} ¿Porque el Espíritu Santo inspira a los Reyes? No: no
creamos eso. ¿Es la Soberana la suma sabiduría, como dicen los Mensajes
a la Corona? No.~A Su Majestad no la inspira el Espíritu Santo, sino la
opinión, que puede equivocarse. Y esa opinión ¿cómo llega a Su Majestad?
Puede llegar por boca de leales consejeros; pero puede llegar, y llega
también, por boca de una monja histérica, o de un fraile, o de un criado
de Palacio. En fin, que la autoridad viene\ldots{} del aire, como la
salud y las enfermedades, y usted es un continuo enfermo que está
esperando siempre que un soplo lo mate o que otro lo resucite.

\emph{---Pollo}, no se guasee usted conmigo---me dijo Narváez nada
colérico, antes bien inclinado a las bromas.---Quedamos en que usted
sabe menos que yo del Principio de autoridad, y de quien lo trae y lo
lleva. Bueno: explíqueme ahora en qué consiste la Libertad\ldots{}
porque yo soy liberal, quiero serlo.

---Quiere serlo\ldots{} adora la Libertad. Yo también amo algo que no
poseo\ldots{} que ni siquiera sé dónde está. Precisamente eso nos
distingue de los tontos a usted y a mí, General: que amamos lo que no
entendemos.

---Con muchísimo salero se está burlando de mí este ángel. Y digo que se
burla, porque\ldots{} me habían asegurado que tiene usted mucho talento;
que desde su más tierna infancia no hizo más que tragar libros y
librotes, y que en Roma todas las bibliotecas eran pocas para usted. Eso
me habían dicho y lo creí; pero ahora, a los que me trajeron la copla
del niño Beramendi, o Fajardo, tengo que decirles que me devuelvan el
dinero\ldots{} porque resulta que usted sabe de estas cosas lo mismo que
yo, total, nada; que en usted, como en mí, todo es un sentimiento, un
deseo, una soñación y nada más. ¿Bastará con eso? Porque, oiga
\emph{pollo}, aquí en confianza: yo he sondeado a Sartorius, a Bravo
Murillo, a todas las eminencias del \emph{moderantismo}, para que me
expliquen bien esto de la Libertad y de la Autoridad y del Régimen, y la
verdad, \emph{camará}, no me han sacado de mis dudas. Dígame: en estas
cosas ¿habrá que decir lo de aquel sabio: \emph{sólo sé que no sé nada?}

---Sí, mi General, al menos por lo que a mí toca. Cierto que yo almacené
infinidad de textos en mi caletre; pero aunque algo conservo de aquel
fárrago, no me sirve para responder a su pregunta. El punto que me
consulta es de acción, y yo en cosas de acción estoy poco fuerte. Todos
los problemas de la vida me los han dado resueltos. Hablando en plata,
soy un hombre de inspiración que no tiene arte en que ejercitarla. Usted
me lleva a mí gran ventaja, porque tiene inspiración y arte, el arte de
Gobierno.

---Y según eso, yo debo dejarme llevar de la inspiración, o hablando en
oro, hacer mi santa voluntad.

---La santa voluntad de un hombre de gran entendimiento, como el que me
escucha, no puede ser otra que salvar al país de un cataclismo\ldots{}
Si me lo permite, General, me atreveré a preguntarle\ldots{}

---Atrévase: ya ve que soy muy llano. Me ha cogido en \emph{la hora del
pavo}.

---¿Cree usted, como Bravo Murillo, que esto se va poniendo mal, que por
debilidades de todos, la política ¿cómo diré\ldots? fundamental, lleva
una dirección torcida?

---Sí señor, así lo creo.

---Y esta dirección torcida de la política fundamental ¿quién puede
enmendarla, estableciendo la dirección derecha?

---Sólo hay en España un hombre capaz de hacer eso.

---¿Quién es? ¿se puede saber?

---O ese hombre no existe, o es Narváez.

---Pues conociendo usted, mi General, mejor que nadie, la torcedura de
que hablo, ¡ánimo y a ello!»

Se levantó como por un resorte, y se lanzó a dar paseos por la estancia
marcando enérgicamente el paso militar. Luego se paró ante mí, y tomando
la actitud de gallo insolente, provocativo, de indómito coraje, me dijo:
«¡Carape, Pepito, que me está usted buscando el genio! ¿Se atreve a
dudar que puedo\ldots?

---¡A ello, mi General!

---¿Va usted pronto a la Granja?

---Mañana, si no me manda otra cosa.

---¿Conoce usted de cerca la Corte? ¿No? Pues es preciso que la
conozca---dijo reanudando el paseo casi a paso de carga.---Dígame,
\emph{niño del mérito}: ¿no le convendría ser Gentilhombre de Su
Majestad?

---Soy harto subversivo para servir en Palacio.

---Vamos, como yo. Tampoco serviría en la Corte por nada de este mundo.
Primero sería sereno del barrio, salvaguardia, rebuscador de colillas.
Veo que somos igualmente demagogos, o demócratas, hablando en oro con
diamantes\ldots{} Oiga usted, \emph{joven} (nueva parada brusca ante mí
con tiesura de gallo): yo haré que le presenten a la Reina\ldots{} ¡Verá
usted qué agradable, qué simpática!\ldots{} ¡Oh, si con un gran corazón
se gobernara\ldots!

---Accedo a la presentación\ldots{} Y al Rey ¿por qué no? Deseo
conocerle.

---Muy agradable también\ldots{} a primera vista, muy
inteligente\ldots{} Le cautivará a usted. Pero\ldots{} ya sabe que ese
buen señor y yo andamos algo esquinados. Por hoy, no puedo decirle a
usted más\ldots{} Pues bien: conocerá usted la Corte de cerca, la verá
por dentro y por debajo, y cuando haya leído ese libro al derecho y al
revés, convendrá conmigo en que dentro de lo humano no hay nada más
difícil que\ldots{}

---¿Que qué?

---Basta. Pasemos a otro asunto---dijo con rápido giro del pensamiento,
volviendo a sentarse junto a mí.---Ahora me contestará el simpático
Beramendi a una pregunta un poquito escabrosa\ldots{} Ya comprenderá que
este cura no se asusta de nada.

---Ni yo.

---Lo que hablemos no sale de aquí.

Reiterada mi disposición a la confianza, me interrogó respecto a
Eufrasia. ¿Insistía yo en negar mis amorosas relaciones con ella? ¿Desde
mi última negativa no había ocurrido novedades que\ldots? No le dejé
concluir. A un hombre que con tanta llaneza me trataba, no podía yo
negarle la verdad. Apenas se la di, me permití agregar: «General,
aprovecho este momento de espontaneidad para pedir a usted un favor, una
merced\ldots{} No es para mí\ldots{}

---Ya la adivino: me pide usted el título de Castilla para esa ave fría
de Socobio. Bueno, \emph{pollo}. Yo hablaré con la Reina y con Arrazola,
y cuando volvamos a Madrid se hará\ldots{} La razón de haber detenido
ese asunto es que\ldots{} vamos; bastaba que fuera recomendación de D.
Francisco para que yo le diera carpetazo. Pero ahora, hijo mío, mediando
usted\ldots{} las cosas varían\ldots{}

---En este caso, señor Duque, más que en otro alguno, le conviene a
usted ser generoso.

---Y ya que hablamos de ese diablo de mujer---me dijo sonriendo con
picardía,---de confianza en confianza llegaré hasta preguntarle a usted
si es celoso.

---No, mi General; no tengo ese defecto.

---Vamos, que es usted de una pasta angelical. Tendrá usted otro enredo
que le interese más. Bien, \emph{pollo}. El mundo es de los
\emph{pollos}.

---¿Y por qué me hace usted, mi General, esa pregunta de los celos?
¿Puedo saberlo?»

Bien porque de improviso terminase la \emph{hora del pavo}, bien porque
calculadamente quisiera mostrarme el lado áspero de su carácter, ello es
que le vi cambiar de fisonomía y de tono. El bueno y jovial amigo se
retiraba dejando el puesto al hombre autoritario y de inseguro genio.
\emph{«Camará}---me dijo acudiendo a coger despachos y cartas que le
traía Bodega,---no tarde usted en irse a la Granja\ldots{} Es la
una\ldots{} Descansar\ldots{} Le conviene conocer de cerca la
Corte\ldots{} Será usted presentado a la Reina\ldots{} Vaya, con Dios.»

\hypertarget{xxv}{%
\chapter{XXV}\label{xxv}}

\textbf{San Ildefonso}, \emph{Agosto}.---El General Gobernador del Real
Sitio, permitiéndome escribir estas páginas en su oficina de la Casa de
Canónigos, ha venido a ser el Mecenas de mis Confesiones, y a su
graciosa protección deberá la Posteridad el conocimiento de mis
singulares aventuras o desventuras (que de todo hay) en esta veraniega
Corte de las Españas; y sabrá lo que he pensado y visto, extrañas ideas,
excelsas personas.

Sean las primeras líneas de esta crónica para consignar que mi hijo
continúa famoso vividor y mamón impertérrito, anunciando con su precoz
robustez los grandes arrestos de una existencia fuerte y emprendedora.
Su madre goza de perfecta salud; come con apetito, y se recrea en
observar cómo se nutre y vigoriza; no pierde ocasión de hacerme notar la
dureza de sus carnes y el apretado tejido de sus músculos, diciéndome
mientras yo apruebo y admiro: «¿Te parece, Pepillo, que estoy bien
dispuesta para mi oficio de madre? Ya sabes que mi gloria es tener
muchos hijos y poder criarlos gordos y sanos, y educarlos después para
que sean hombres de mérito, o mujeres de su casa. Es mi ambición y no
tengo otra. Ahora, tú verás\ldots» No necesito decir cuánto me agradan
estos proyectos de hacerme patriarca, y por mi parte estoy decidido a no
poner limitación a la numerosa tribu que mi esposa me anuncia. Aumenta
mi gozo el ver que María Ignacia no vigila mis actos, cual si no dudase
de mi honradez conyugal, o se viese plenamente compensada de cualquier
disgusto con las garantías de no interrumpir la serie prolífica que
ambiciona. Sin duda se dice: «Dame hijos y llámame tonta.» Pero yo me
guardo muy bien de llamarla tonta. Su inteligencia es cada día más alta,
y quizás por tanta elevación y sutileza, ha dejado de estar a mi
alcance. Pido a Dios que mi hijo se parezca más a mi mujer que a mí.

Pues señor\ldots{} a los cuatro días justos de mi estancia en este Real
Sitio fuí presentado al Rey, a la salida de la Colegiata, por el Marqués
de Malpica. No hubo en la presentación más que los cumplimientos de
ritual; pero dentro de ellos supo D. Francisco mostrarme excepcional
afabilidad, seguro indicio de que mi persona no le era desconocida. Al
siguiente día recibí la visita del gentilhombre, D. Juan Quiroga, quien
me señaló hora para tener el honor de ser recibido por Su Majestad. A
fin de que esto vaya con el mejor método, debo empezar por dar
conocimiento del Gentilhombre, hermano de la religiosa francisca Sor
María de los Dolores Rafaela Patrocinio, comúnmente nombrada \emph{Sor
Patrocinio}, quien con la celebridad que adquiriendo va, paréceme que
llegará al futuro siglo antes que estas páginas en que por primera vez
escribo su nombre. No la he visto nunca; tan sólo sé de ella lo que la
fama con el resonar de estupendos milagros nos cuenta un día y otro; por
lo cual no es ocasión todavía de que a mis Memorias la traiga, como hago
ahora con su hermano, a quien tuve por persona noble, juzgándole por su
apostura, tono y modales.

No se compadece la nobleza del aspecto con el origen y crianza del
Sr.~Quiroga, de quien se cuenta que tuvo niñez mísera y juventud harto
trabajosa, pues el hombre se formó y educó en un modestísimo
establecimiento de bebidas del Paseo de la Virgen del Puerto, donde,
para estímulo del despacho, había el pasatiempo de juegos de envite,
como el cané y el famoso de las tres cartas para descubrir el as de
oros; y tan buena organización tuvo la casa, según dicen, en este
enredillo, que los viandantes salían de allí muy ligeros de todo lo que
llevaban. Pues ved de qué bajas capas ha salido este hombre, y admirad
conmigo que haya sabido disimular y poner en olvido su ruin escuela,
tomando aspecto, lenguaje y modos tan finos que ello parece milagro. Sin
duda lo es, si no de la virtud, de la ambición, anímica y social fuerza
capaz no sólo de mover las montañas, sino de purificar las charcas
cenagosas, y hacer de un Rinconete un Don Quijote. Este ha dado quince y
raya, por la trayectoria de su transformación, a los Godoyes y Muñoces,
y si bien se eleva mucho menos, es su mérito mayor, porque se ha elevado
de más bajo. Y hay más: si de los milagros de su bendita hermana dudan
los incrédulos, y aun algunos teólogos, de los de éste nadie puede decir
lo mismo. En fin, que el hombre me agradó mucho, y sin esfuerzo le
ofrecí mi amistad a cambio de la suya.

Pero si grato fue el emisario del Rey Francisco, mayor encanto tuvo este
para mí, contribuyendo no poco a mi satisfacción la sorpresa, porque me
habían hecho formar del esposo de Isabel idea muy distante y muy
distinta de la realidad. Juzgando por los pareceres del vulgo, que se
forman sabe Dios cómo, creía yo encontrarme con un señor desabrido y
chillón, de escasa cultura, ideas pobres y encogidas maneras, y no le vi
conforme al anticipado retrato, al menos en lo esencial; pues si bien no
suena su voz con el timbre más robusto, en finura de trato, extensión de
conocimientos comunes para poder hablar superficialmente con todo el
mundo, y arte Real de desplegar toda la amabilidad compatible con la
etiqueta, creo que no hay en la familia quien pueda superarle. Me agradó
la pureza de su pronunciación castellana; de rostro le encontré
demasiado bonito, con perjuicio de la gravedad varonil; de cuerpo algo
menguado en la mitad inferior. A la conciencia de estos defectillos
atribuyo la timidez que en él he creído advertir: la vencerá cuando en
la conciencia de su posición se afirme. ¡Cuidado que está fuerte el
hombre en literatura italiana! Tengo por cierto que hubo de prepararse
para mi visita, la cual creyó que debía constar de dos materias
principales: mi manuscrito de Roma, que ha leído, y algo de literatura y
artes de aquella tierra. Juicios muy atinados, del patrón selecto, le oí
sobre pintura y escultura, sobre los Médicis, sobre León X y Julio II; y
españolizando su erudición me habló del Marqués de Pescara y Victoria
Colonna, de la Campaña del Garellano, del grande Osuna, del pintor
Ribera, y de otros asuntos y personas en que los nombres de Italia y
España suenan juntos en dulce armonía. De la presente expedición en
auxilio del Pontífice\ldots{} se calló muy buenas cosas\ldots{}

Y por fin le tocó la vez al manuscrito de mis romanas aventuras. Yo,
francamente, quizás por haber transcurrido tanto tiempo desde que perdí
mis papeles, no me ruboricé oyendo elogiar aquella joya. Si no tuviera
la mejor idea de la discreción de Su Majestad, habría podido creer que
se burlaba de mí. Entre col y col no dejó de tirarme alguna china,
siempre con bastante delicadeza, por la malicia y poca vergüenza que
revelo en algunos pasajes de mi autobiografía\ldots{} Hasta aquí, fuera
de lo hiperbólico de las alabanzas y de lo atenuado de las censuras, no
había nada de particular. Lo extraordinario, lo que suscitó en mí tanta
sorpresa como admiración, por el poder adivinatorio que en D. Francisco
revelaba, fue que me hablase de la continuación de mis Memorias, escrita
en Madrid en Febrero y Marzo del año anterior, parte que no se me ha
perdido, y bien guardada está en mi poder, y yo bien seguro de que por
nadie ha sido leída.

«Será interesante, en esa Segunda Parte---me dijo sonriendo con aires de
agudeza,---aquel pasaje del baile de Villahermosa, en que se le aparece
bajo el disfraz de una \emph{ciociara} la propia Barberina, y le embroma
a usted de lo lindo diciéndole que es gallega recriada en Tordehúmos.
Principia usted creyendo que es Barberina, y luego ve en la máscara una
dama incógnita que le ha robado su manuscrito y quiere divertirse un
rato a costa del autor\ldots{} Es graciosísimo, convenga usted en que es
saladísimo. La falsa italiana se divirtió todo lo que quiso, y luego se
le escapó a usted metiéndose en un coche con sus criadas\ldots{}

---Señor---respondí con todo el descaro del mundo,---si Vuestra Majestad
conoce esa parte de mi historia, la habrá leído en el manuscrito de la
máscara, no en el mío.

---Yo no digo que lo haya leído, señor Marqués; digo que será
interesante escrito por usted\ldots{} La escena de Villahermosa se hizo
pública. ¿Cómo? Lo ignoro. Lo que sí sé es que la primera lectora de su
manuscrito de Italia fue una ilustrada monjita\ldots{} A propósito,
Marqués, puedo dar a usted una noticia que seguramente le será muy
grata\ldots{} Su señora hermana, Sor Catalina de los Desposorios, a
quien usted no ha visto desde el año pasado, volverá este otoño al lado
de las religiosas de la Concepción Francisca, que están ahora en el
convento de Jesús.»

Siguiéndole, pues así me lo ordenaba la cortesía, en el repentino
quiebro que dio a la conversación, hube de mostrarme muy gozoso de que
mi hermana volviese a Madrid, de que se juntara prontito con las otras
monjas franciscanas y milagreras, no sé si descalzas, calzadas o por
calzar. El bondadoso Príncipe quiso halagarme en el orgullo de linaje,
tributando a mi señora hermana elogios que sin duda merecía, y que yo
escuché con bien acentuadas muestras de gratitud. «Es Sor Catalina de
los Desposorios---dijo D. Francisco gravemente, marcando con la cabeza
cada palabra encomiástica,---una religiosa eminentísima, por sus
virtudes, por su talento, verdadera gloria de la Orden Franciscana; y yo
creo que, si no fuese tan modesta, luciría más, mucho más\ldots{} Pero
si con la modestia de Sor Catalina, insigne escritora que no quiere
escribir, pierde mucho la Orden, con la misma virtud gana mucho ella en
su alma, y\ldots{} váyase lo uno por lo otro.»

No sabiendo cómo corresponder a estos encomios, declaré que el alma es
lo primero; glosé con afectados conceptos la idea excelsa que el Rey
tiene de mi hermana, y sospechando que la visita pasaba de las
dimensiones convenientes, pedí la venia para retirarme. El Rey no me
retuvo, y saludándome afectuoso, después de poner en mi mano el
manuscrito, me dijo: «Isabel también lo ha leído, y desea conocer a
usted.» Respondí que ansío ofrecer mis respetos a la Reina: sólo aguardo
que se me conceda la audiencia solicitada\ldots{} Cortesías, un sonreír
ceremonioso, y afuera, Pepe\ldots{} La verdad, no salí descontento, con
mejor opinión de la Majestad Consorte que la que al entrar llevaba, y
con mis recobrados papeles bajo el brazo. Milagro me parece que haya
vuelto a mí lo que Sofía sigilosamente me sustrajo, ahora restituido a
su dueño por este discreto y piadoso varón.

Sigo mi cuento. En la Granja he podido añadir a mis buenas relaciones de
Madrid otras muy agradables. Cuento entre mis amistades, \emph{pollos},
hombres maduros de ambas aristocracias, y damas y señoritas o
\emph{pollas} de la más alta distinción. Los amigos que más trato son
Pepe Ruiz de Arana, Enrique Galve (Alba) y Juanito Arcicollar (Santa
Cruz). Los corros que en los jardines se forman son las más risueñas
tertulias que cabe imaginar, encanto de los ojos y del oído, cual si los
arriates de flores se animaran, cobrando el don de mirada y el don de
palique, entre los murmullos y risotadas del agua de las fuentes
mitológicas. Allí se juntan, formando lindos grupos de matronas y
ninfas, la Marquesa de Santa Cruz, las Duquesas de Gor y de San Carlos,
la Princesa de Anglona, y entre ellas, diseminadas por su propia
ligereza versátil, Carmen, Pepa, Luisa, Encarnación, Rosario, Jacoba,
Cristina, Joaquina y otras, retoños lindísimos de las casas de Malpica,
Gor, Santiago, Santa Cruz, que pronto formarán nuevas ramas frondosas
del árbol de la Grandeza\ldots{} En rancho aparte se reúne la
aristocracia nueva, producto de la riqueza, de la audacia mercantil o de
la usura; mas no veo un extremado prurito de separación entre estos dos
firmamentos sociales que pretenden destacarse sobre el vulgo. Hay
tangencias y aun inmersiones de unas masas en otras. Yo mismo entro y
salgo de esfera en esfera, y llevo y traigo ideas de aquí para allá,
confundiendo, hibridizando las clases. Mi amiga Eufrasia ha compuesto
hábilmente su círculo, atrayendo a no pocos ancianos y \emph{pollos} de
ilustre nombre, mientras D. Saturno, infatigable en su proselitismo
antiliberal y antiparlamentario, se infiltra en los corros
aristocráticos, y busca y halla catecúmenas para su iglesia entre las
matronas de Malpica o de Santa Coloma.

Paso ratos entretenidos en estas tertulias \emph{au grand air}, bajo los
olmos y tilos de los incomparables jardines. Pero no puedo arrastrar a
mi mujer a que participe de mi distracción; ha tomado el hábito y el
gusto del vivir obscuro y retraído, y no hay quien la saque de su
estuche, o del capullo que ha labrado con las atenciones del niño y su
propia timidez. A mis instancias para que no se retraiga en absoluto de
la vida social, responde que no le hacen falta corros, ni le interesa
saber cómo se viste Fulanita o se peina Doña Mengana: de lo que en los
jardines se hable y se murmure se enterará \emph{cuando yo se lo
cuente}. D. Feliciano y su esposa sí frecuentan la sociedad jardinesca,
arrimándose a la gente de sangre azul, entre la cual tienen no poca
simpatía por la noble ranciedad de sus caracteres. A excepción de Doña
Josefa, inseparable de María Ignacia en sus caseras afecciones y
menesteres, las damas maduras se han quedado en Madrid a las inmediatas
órdenes de Genara Baraona, consagradas al visiteo de monjas, vestidero
de imágenes, y al trajín de hermandades caritativas o de pura devoción
santurrónica.

Tenemos en el teatro compañía modesta de ópera; en la Colegiata
funciones religiosas de gran lucimiento. Pero las más divertidas fiestas
de la jornada son las cacerías en Riofrío, paseos a Balsaín, en coche o
caballo, y las excursiones borricales a la Boca del Asno, Chorro Grande,
Silla del Rey, y otros agrestes y pintorescos lugares. En el descanso y
merienda de una de estas caminatas fuí presentado a Su Majestad, que me
agració con amables atenciones, riñéndome blandamente por no haber ido a
visitarla. Excuseme como pude, y aunque la culpa no era mía, sino de
ella, culpable me declaré, y prometí enmendar pronto mi descuido. No he
visto mujer más atractiva que Isabel II, ni que posea más finas redes
para cautivar los ánimos. Pienso que una gran parte de sus encantos los
debe a la conciencia de su posición, al libre uso de la palabra para
anticipar su pensamiento al de los demás, lo que ayuda ciertamente a la
adquisición de majestad o aire soberano. Pero no hay duda que ella ha
sabido crearse una realeza suya, en perfecta armonía con sus azules ojos
picarescos y con su nariz respingada, realeza que toca por un extremo
con la dignidad atávica, y por otro con no sé qué desgaire plebeyo, todo
gracejo y donosura. Es la síntesis del españolismo, y el producto de las
más brillantes épocas históricas. Manos diferentes han contribuido a
formar esta interesante majestad. No es difícil ver en tal obra la mano
de Fernando III, de Felipe IV, quizás la de otros reyes y princesas de
la sucesiva y cruzada serie, manos austriacas y borbónicas, y si hay
manos de poetas castizos, digamos que la última pasada se la dio D.
Ramón de la Cruz.

Fue tan extraño, tan inaudito lo que me pasó en las entrevistas o
audiencias que se ha dignado concederme la Reina, que para contarlo con
el debido respeto de la Historia general y de la de mi vida, necesito
tomar resuello, y preparar bien mi espíritu para que no me falte la
sinceridad, ni el adecuado lenguaje de esta virtud.

\hypertarget{xxvi}{%
\chapter{XXVI}\label{xxvi}}

La tarde de la merienda, a la vuelta de la Boca del Asno, Su Majestad,
pasado un rato después de los saludos de ceremonia, y cuando yo pensé
que no se acordaba ni del santo de mi nombre, se volvió de repente a mí
y me dijo: «Pero tú, Beramendi, que tan bien sabes escribir las
\emph{cosas que pasan}\ldots{} y con tanta naturalidad, que parece que
las estamos viviendo, ¿por qué no escribes esto que ahora ocurre con la
Lola Montes?» Por aquellos días traían los periódicos el proceso que a
nuestra célebre compatriota le formaban por bigamia. Afortunadamente, yo
había leído el caso, y pude contestar a Su Majestad con dominio del
asunto. «Señora, para escribir eso---le dije,---necesitaría conocerlo
por mí mismo, y esto no es fácil; la propia Montes no habría de contarme
toda la verdad\ldots» «Pues yo declaro---añadió la Reina,---que me ha
hecho gracia el desahogo de esa mujer para casarse con el teniente
Heald, estando casada con otro. Vamos, que daría yo cualquier cosa por
oír lo que dice el teniente, que, según cuentan, es una criatura\ldots{}
¡Y qué monísimo estará llorando por su Lolita, que el otro reclama! Lo
que es mujer de talento, vaya si lo es. ¿Y qué me dices de la que le
armó al Rey de Baviera? Ello será una barbaridad; pero a mí me agrada,
no puedo remediarlo, que sea española la que ha hecho tantas
diabluras\ldots{} Anímate, anímate a escribirlo, y desde ahora te
aseguro que si lo imprimes lo leeré con muchísimo gusto.» Respondí que
si la señora tenía gran empeño en que tal historia escribiese, la
obedecería; pero que yo, no sé si por mi suerte o mi desgracia, no me
dedico a las letras, ni paso de un simple aficionado sin pretensiones.
Díjome Su Majestad que no fuera tan modesto, y ya no se habló más del
asunto, porque quien variaba la conversación a su antojo, picando aquí y
allá, se puso a bromear con la Marquesa de Sevilla la Nueva sobre la
mayor o menor gallardía de los buches en que cabalgaban los señorones de
su cortesano acompañamiento. La verdad, no estaba yo satisfecho de
aquella mi primera conversación con Isabel II, porque si su idea fue
plantear un tema literario, no había estado muy atinada en la elección,
y además, yo no había sabido darle un airoso giro.

Sigo contando. Llegó el deseado instante de ser recibido por Su
Majestad, y al referir la audiencia, tengo que condolerme otra vez de mi
mala suerte, porque si desgraciado fuí en la presentación, al aire
libre, peor anduve en la visita entre paredes, llegando al extremo de
turbarme y no saber qué decir. Pues señor: hice mi antesalita, no muy
larga, y cuando el Gentilhombre me condujo hasta la puerta de la cámara,
iba yo un tanto perplejo y sobresaltado. La Reina estaba en pie. Junto a
la mesa central hojeaba un álbum que me pareció de paisajes de Italia. A
mi reverencia correspondió con una sonrisa, dejando con desdén el álbum;
sentose, señalándome una silla frontera, y me miró. Creí que su mirada
medía mi talla, y que sus ojos penetraban en los míos. Vestía un traje
blanco con motitas, muy ligero y elegante. Advertí sus formas llenas,
redondas, contenidas dentro de la más perfecta esbeltez. «¿Qué te
parece---me dijo,---la vida en el Real Sitio? ¿Verdad que es un poco
triste?\ldots{} ¿Sabes que han venido a invitarme para que vaya a Madrid
a ver una lucha de fieras? ¿La has visto tú?» Contestele que todo se
reduce a echar a pelear un toro con un tigre, y a poner un rinoceronte
gordo delante de un león flaco. Opinaba yo que Su Majestad no se
divertiría mucho en este ejercicio. «No sé si determinarme a ir a ver
eso---prosiguió en un tonillo de dubitación tediosa.---Mamá y el Rey
quieren ir\ldots{} Ya les he dicho que vayan ellos\ldots{} ¿Y tú estás
contento aquí?\ldots{} Lo dudo: ¡en Madrid os divertís tanto los
jóvenes! Madrid es muy bonito, y a mí me gusta mucho. ¡Qué poco vale la
ópera que acá tenemos! Anoche fui a oír el \emph{Macbeth}, y
francamente, me indigné viendo la facha con que entran los espectros de
Banquo y Duncan en el banquete. Yo recordaba los gigantones del
Corpus\ldots{} Y luego, \emph{lady Macbeth} con su ronquera en el
brindis y los tambaleos que hace para soltar la voz, me parecía que
brindaba con Peleón\ldots{} Aquí es gran tontería traer
espectáculos\ldots{} Paseos, excursiones, cacerías, son lo más
propio\ldots{} Y las cacerías no creas que me hacen a mí mucha gracia.
No me gusta matar ni ver matar a un pobrecito conejo, que sale a
buscarse la vida por el campo\ldots{} ¿Te gusta a ti la caza? Dicen que
es imagen de la guerra. Una y otra me son antipáticas; y para que veas
si tengo yo desgracia: desde muy niña no oigo hablar más que de guerras.
¡Guerras por mí, que es lo que más me duele!\ldots{} y luego
revoluciones y trapisondas\ldots»

A este gracioso divagar de la Soberana contesté con generalidades o
conceptos comunes. Poco lucida era la conversación, sin nada en que se
revelara la grandeza de la persona con quien yo tenía el honor de
hablar. En una de las transiciones que Su Majestad hacía para variar los
asuntos, noté más viveza en el cambio de tonalidad; vi en su rostro una
inflexión penosa; por un instante vaciló, dejando una palabra para tomar
otra. Sin duda quería Isabel hablarme de algo cuya forma verbal no
afluía fácilmente de sus labios como los anteriores temas, que venían a
ser gacetillas ennoblecidas por la palabra Real. Por fin, poniendo cara
compasiva, y agraciándome con una sonrisa bondadosa que a mi parecer a
la de los ángeles igualaba, me dijo: «Mira, Beramendi, de tu asunto me
ocuparé con muchísimo interés. Hoy no puedo decirte nada concreto, no
puedo\ldots{} vamos, que no puedo. Pero cree que no habrá para mí mayor
gusto que complacerte. Quisiera contentar a todos, y que nadie tuviese
en España ningún\ldots{} vamos, ninguna pretensión que yo no pudiera
satisfacer\ldots{} ¡Pero hay tantos, tantos que a mí vienen, y yo\ldots!
¡Pobre de mí! no puedo ser tan buena como quiero\ldots»

Yo no sabía qué decir; no comprendía ni palabra. ¿Qué asunto mío era
aquel en que no podía complacerme? Por mi desgracia no caí en la cuenta
de que Su Majestad era víctima de un error, y relacioné sus
manifestaciones con el ridículo plan de mi suegro de obtener para mí un
cargo en Palacio. Algo de esto me había dicho también Narváez; yo no
hice caso. La Reina, obcecada, remató mi confusión con estos conceptos,
un poco menos obscuros que los anteriores: «Narváez me habló; me habló
Santa Coloma por encargo de tu suegro. A ti te digo lo que a ellos
dije\ldots{} que lo haré más adelante. Siento un deseo vivísimo de
complacerte, como a todo el mundo\ldots{} Ten un poco de paciencia, y
aguárdate un mes, dos meses\ldots»

A decirle iba que no tengo ningun interés en ocupar un puesto palatino;
pero por no desautorizar a Narváez ni a mi suegro me callé. Estas
discreciones ridículas, en la conversación con Reyes, le comprometen a
uno tanto como las indiscreciones más estúpidas\ldots{} Me limité a
indicar: «No se inquiete Vuestra Majestad por mí. ¡Si para mí es
igual!\ldots» Y ella, gozosa de oírme tan poco impaciente, se levantó en
son de despedida, y como quien pronuncia la última palabra de un asunto
fastidioso, me dijo: «Bueno, Beramendi: queda de mi cuidado\ldots{} Yo
no lo olvido. Será mi mayor gusto\ldots{} Adiós, Marqués\ldots{} Confía
en tu Reina\ldots»

Le besé la mano y salí aturdido, no sin los resquemores que nos ocasiona
la sospecha de haber cometido falta grave de cortesía, por mal entender
de las cosas. Aquel \emph{confía en tu Reina} quedó estampado en mi
mente con letras de fuego. No se apartaba de mí la idea de que entre la
Reina y yo \emph{se cernía}\ldots{} no puedo expresarlo de otro
modo\ldots{} un error formidable, y de que fue gran torpeza mía no
disiparlo sobre el terreno. Toda la tarde estuve en esta ansiedad,
discurriendo de qué medios valerme para salir de tan cruel
incertidumbre. Pero a nadie osaba comunicar mi recelo, por la ridiculez
que el caso entrañaba. Figúrese ahora el pío lector de la Posteridad (si
he de merecer ¡vive Dios! el honor de que la Posteridad me lea), cuál
sería mi asombro cuando aquella misma noche, acabadito de comer, recibí
la visita del Gentilhombre Marqués de Iturbieta, que en mi busca venía
de parte de Su Majestad para llevarme inmediatamente a su presencia, ¡a
la presencia de Su Majestad!\ldots{}

Hubo de decírmelo tres veces para que me persuadiese de que no soñaba.
«Pero esta no es hora de audiencia---le dije; y el amable señor sólo
contestaba dándome prisa para que me vistiera y me fuese con él. Así lo
hice, y al cuarto de hora, sin más que una breve antesala, me vi delante
de Isabel II, que venía del comedor, elegantísima, descotada con cierta
demasía generosa muy de moda hoy, y harto apropiada a la estación
canicular\ldots{} Cuando la vi venir hacia mí, sonriente; cuando alargó
su mano hacia la mía, como si quisiera sacarme a bailar, vi en ella una
figura ideal, vi a la Reina\ldots{} harto distinta de la otra Reina que
había visto por la mañana, y oí un acento que no me pareció el mismo
que, algunas horas antes, pronunciaba las cláusulas vulgarísimas de un
coloquio entre señorita pobre y caballero simple. Me dejó atónito y como
embelesado con estas sus primeras palabras: «Si no hubieras venido, me
habrías hecho pasar una mala noche; tal disgusto tenía yo por la
barbaridad que hice esta tarde\ldots{} Cuando caí en ello no tenía
consuelo\ldots{} ¡Pero qué habrás pensado de mí!\ldots{} Puedes creer
que es la primera vez en mi vida que esto me pasa\ldots{}

---Señora---le dije,---no es para que Vuestra Majestad se
disguste\ldots{}

---Pero tú, tonto, ¿por qué no me advertiste\ldots{} que estaba yo
tocando el violón?»

La familiaridad de la frase me hizo reír\ldots{} «No he tenido
sosiego---prosiguió,---hasta que decidí mandarte llamar, para suplicarte
que me perdones\ldots{}

---¡Señora\ldots{} perdonar!»

Indicándome que me sentara, se sentó ella de través en una silla,
apoyando el codo en el respaldo de la misma. «Sí, perdonarme,
porque\ldots{} ¡vaya, que estuve torpísima!\ldots{} ¡Confundir una
persona con otra!\ldots{} Nunca me había pasado cosa semejante. Lo único
que como Reina me han enseñado es el conocimiento de las personas, no
confundirlas, no hacer trueques de nombres ni de fisonomías. En este
arte he sido siempre muy segura. ¡Cómo que no sé otra cosa!\ldots{} Pues
hoy\ldots{} ¿Pero dónde tenía yo mi cabeza, Señor?»

Decía esto Su Majestad, firme el brazo en la silla, cogiéndose con la
mano derecha el pendiente de la oreja del mismo lado. Y luego, con
soberana modestia de gran persona, prosiguió: «Te explicaré en qué
consistió el error. Pero antes has de perdonarme.

---Señora, por Dios, no tengo por qué perdonar ofensa que no ha
existido.

---¿Qué no? Vas a verlo\ldots{} Pues como recibo a tanta gente, como me
hablan de este y el otro, como vienen a mí cada día centenares de
recomendaciones, no es extraño que alguna vez confunda nombres\ldots{}
asuntos. Las caras no las he confundido nunca: por esto me ha causado
tanto enojo la torpeza de hoy. Vamos, que esta tarde, cuando me hicieron
comprender mi equivocación\ldots{} me hubiera pegado\ldots{} Porque es
gran desatino confundir tu cara con la de\ldots{} Dispénsame que calle
este nombre. El milagro puedes saber; el santo no hay para qué.

---Puede Vuestra Majestad callar también el milagro. Yo no necesito
explicaciones\ldots{}

---No, no está mal que lo sepas. Figúrate\ldots{} Estoy asediada de
peticiones\ldots{} Naturalmente, todo el que algo necesita, acude a mí.
Soy la dispensadora de mercedes y gracias, soy la Reina que desea serlo,
haciendo felices a todos los españoles, lo que es un poquito
difícil\ldots{} pero, en fin, se hace lo que se puede\ldots{} Y como yo,
si en mí consistiera, a ninguno de los que piden le dejaría ir con las
manos vacías, resulta que\ldots{} En una palabra, un hijo de un Grande
de España que va a contraer matrimonio, no el Grande de España, sino el
pequeño hijo del Grande, me hizo saber hace días que para sostener el
lustre de su nombre le hace falta\ldots{} una friolera\ldots{} treinta
mil duros\ldots{} Mayores cantidades que ésas he dado yo sin ton ni
son\ldots{} Por ahí corre un cuento acerca de mí\ldots{} ¿no lo has oído
tú? Pues te lo voy a contar; porque aunque parece cuento, no lo es; es
Historia\ldots{} sólo que estas cosas no pasan a la Historia\ldots{} Aún
no era yo mayor de edad, cuando un desgraciado caballero, hijo de un
servidor muy leal de mi padre y de mi madre, vino a decirme que se veía
en grande aprieto, que le ejecutaban, le deshonraban y qué sé yo
qué\ldots{} Vamos, que le hacían falta veinte mil duros\ldots{} El
lloraba pidiéndomelos, y yo lloraba también, más que de pena, de la
alegría que me daba el poder remediar tamaña desgracia\ldots{} ¿Qué
creerás que hice? pues llamar a D. Martín de los Heros, que era entonces
mi Intendente, y decirle con la mayor naturalidad del mundo: «Heros,
tráeme ahora mismo veinte mil duros\ldots» El pobrecito D. Martín, que
era más bueno que San José, me miraba y suspiraba, y no decía nada; no
se atrevía\ldots{} Como que nadie se ha cuidado de advertirme las cosas,
ni de instruirme, por lo cual yo ignoraba todo, y principalmente las
cantidades. Tanto sabía yo lo que son veinte mil duros, como lo que son
veinte mil moscas. D. Martín ¿qué hizo? Pues se fue a la Intendencia, y
mientras yo estaba de paseo, hizo subir veinte mil duros, en duros ¿eh?,
y me los puso sobre la mesa, así, muy apiladitos. ¡Jesús de mi alma! ¡yo
que vuelvo del paseo con mi hermana, y me veo aquel catafalco de dinero,
aquello que parecía un monte de plata\ldots! Llamo y entra D. Martín,
que me acechaba en la cámara próxima. «Intendente, ¿qué es esto?» Y él
muy serio: «Señora, esto es lo que Vuestra Majestad me ha pedido, veinte
mil pesos.» ¡Ave María Purísima! ¡qué miedo me entró!\ldots{} «¿Pero es
tanto? ¿Pero veinte mil duros son tantísimos duros? No, no es esto lo
que yo pedía. Es que no me han enseñado ni siquiera el mucho y poco de
las cosas. No, no, Martín: no hay que darle tanto a ese perdido, que
según dicen, maltrata a su mujer\ldots» ¿Qué te parece? Pues aquella
lección se me ha quedado muy presente, y no fue lección perdida. Por
fin, el donativo se redujo a cinco mil duros, y aún me parece que me
corrí demasiado.

---La bondad de una Reina justo es que no esté contenida dentro de la
prudencia.

---Pero todo tiene un límite, no convenía que me criaran en las
\emph{Mil y una noches}.

---Por lo visto, ni con la lección de Don Martín se ha curado Vuestra
Majestad de su esplendidez\ldots{} El caso de ahora\ldots{}

---El caso de ahora se inició con petición de treinta mil duros; pero yo
los reduje a quince\ldots{} Lo tremendo es haber confundido al
peticionario contigo, \emph{quid pro quo} muy extraño, pues no os
parecéis más que en el título; en las fisonomías, nada. El tiene cara de
tonto, y tú de todo lo contrario.

---Señora, ¿cómo agradeceré yo distinción tan grande?

---Pues perdonando mi simpleza y no hablando con nadie de este asunto.
¡Cuidado si estuve torpe y ciega! Y ello fue porque ayer me hablaron del
otro, me anunciaron su visita para hoy, y yo me preparé de razones para
entretenerle. Al hablarte de tu suegro me refería\ldots{} al que va a
ser suegro del otro, ¿me entiendes? De ti ya sé que eres casado. Y a
propósito: tráeme a tu mujer; deseo conocerla. Entiendo que es muy feliz
contigo.

---Señora, si así lo dijeron a Vuestra Majestad, será cierto\ldots{}
pero yo no lo aseguro.

---Pues yo no lo inventé. Alguien me lo ha dicho.

---Señora, no siempre se dice la verdad a los Reyes.

---Según eso, no es verdad que hagas feliz a tu mujer. Es muy buena.
¿También en eso me han engañado?

---En esto sí que han dicho a Vuestra Majestad una verdad como un
templo. Mi mujer es un ángel.»

---¡Un ángel! Así llaman a todas las mujeres sufridas. que llevan con
paciencia las trastadas de sus maridos\ldots{} Yo concibo que la mujer
modelo sea un demonio. Beramendi\ldots»

Al decir esto, la Reina se levantó. Yo hice lo mismo, creyendo que se me
daba señal de retirada. «No, no---me dijo con la mayor delicia de su voz
y toda la nobleza de su alma.---Quédate un rato\ldots{} Te invito a una
pequeña \emph{soirée}\ldots{} de provincias. Estamos solas mi madre y
yo, con el Rey y algunos amigos.»

\hypertarget{xxvii}{%
\chapter{XXVII}\label{xxvii}}

La señora Posteridad se hará cargo de mi satisfacción y gratitud por
tantas bondades. Retirose Su Majestad, y a poco entraron en la sala
donde yo estaba, el pianista Guelbenzu, amigo mío; la dama de servicio,
Condesa de Sevilla la Nueva, y Bravo Murillo, Ministro de jornada.
Pasamos a un salón próximo, donde volví a ver a Isabel II, acompañada
del Rey y de la Reina Madre, con D. Fernando Muñoz y dos o tres figuras
palatinas. Amabilidad ceremoniosa y fría merecí del Rey, que algo me
dijo, sonriendo, del \emph{quid pro quo} motivo de mi presencia en
Palacio. Doña María Cristina, a quien me presentó su hija, acogiome con
notoria sequedad, y en su mirada recelosa leí estos o parecidos
pensamientos: «¿Quién será este pájaro?\ldots{} ¿A qué vendrá éste
aquí?\ldots» Don Fernando Muñoz me hizo varias preguntas con acompasada
rigidez, propia de un examen, y luego me habló de Roma y sus monumentos,
con erudición fresca, reciente, aprendida de los \emph{cicerones}.

Mientras escuchaba yo al Duque, la Reina, no lejos de mí, hablaba con
Guelbenzu de programas musicales. «Esta noche no canto---le
decía.---Tengo la voz tomada\ldots» La vi acercarse a un espejo Psiquis,
arrimado al ángulo del salón, y contemplarse un instante, componiendo
con sutil mano los bandós que rodean sus orejas, y recogiendo un poco el
escote que se abría demasiado. Después vino a mí; reparé su andar
ligero, los pies chicos con zapatitos blancos que sacudían los bordes de
estas faldas en forma de campana que ahora se usan\ldots{} Yo me condolí
de mi desgracia, pues desgracia era, y de las más grandes, que Su
Majestad no se dignara cantar aquella noche; y ella me dijo: «Pues mira,
no pierdes nada con no oírme, porque canto muy mal. Además, estoy
perdida de la voz. En los jardines me enfrié esta tarde. Oiremos a
Guelbenzu solo, y todos vamos ganando.» Bruscamente, saltando de un
asunto a otro, como el pájaro que aletea de rama en rama, me dijo:
«Beramendi, ¿no tienes tú ninguna Gran Cruz?\ldots{} ¿que no? Pues es
preciso que tengas una, la que quieras\ldots» Me incliné. D. Fernando
Muñoz, que no se movía de mi lado como si montara una guardia, quiso
introducir otro tema de conversación; pero no le resultó el juego, y la
Reina, sin parar mientes en su padrastro morganático, continuó así: «El
25 tengo Besamanos, por ser los días de mi hermana. Vendrá Narváez, y le
diré lo de tu Gran Cruz. Ya sé que Narváez es amigo tuyo\ldots{} Pero di
una cosa: ¿puedes tú aguantarle? Cuidado, que de Narváez no puedo decir
nada que no sea para colmarle de elogios, como militar valiente, como
hombre de gobierno; ¡pero qué genio, Señor!\ldots{} En su casa no te
sufre más que Bodega, que debe de ser un santo.

---El genio fuerte del General---dijo Muñoz,---tiene su razón de ser.
Con blanduras no hay modo de gobernar a este país.

---Ciertamente---indiqué yo.---Y también puede asegurarse que el General
no es todo asperezas. En más de una ocasión le he visto cariñoso,
amabilísimo\ldots{}

---Esas ocasiones habrán sido pocas para su mujer---afirmó la
Reina.---La pobre Duquesa de Valencia no gusta de vivir en Madrid. Su
marido la trata peor que a los progresistas. Pero, en fin, el hombre
vale mucho, y se le pueden perdonar las rabietas por el talento que
tiene, y aquella firmeza de carácter\ldots{} Por cierto que a ti te
aprecia, te quiere: me lo dijo. Y a propósito, Beramendi: ¿es cierto que
estás escribiendo la Historia del Papado? A mí me lo han dicho.

---Algo de esto oí yo también---apuntó D. Fernando Muñoz por no estar
silencioso.

Respondí que, en efecto, había pensado escribir esa Historia, pero que
las dificultades del asunto me habían hecho desistir\ldots{}

«Pues es lástima, porque ahí tendrías campo ancho donde lucirte. ¡Y que
no harías poco servicio a la Religión! Al Santo Padre le había de gustar
muchísimo que escribieras las Vidas de todos sus antecesores desde San
Pedro\ldots»

El movimiento de las figuras que componían la reunión era determinado
por la Reina, que pasaba de grupo en grupo. Dirigiéndose a Bravo
Murillo, me libró de la guardia del Duque de Riánsares, que allá se fue
también, y la razón de esto voy a decirla al instante. En estos días ha
corrido la voz de que abandona D. Alejandro Mon el Ministerio de
Hacienda, y que le sustituye Bravo Murillo. Descontentísimo del
asturiano está el Sr.~Muñoz, porque aquel se ha cansado de colocarle la
interminable cáfila de parientes y demás indígenas de Tarancón, y en
cuanto vio que la Reina hablaba con el Ministro de Instrucción y
Comercio, acudió a olfatear si es cierto lo del cambio ministerial.
Cierto debe de ser a juzgar por el interés del diálogo que en aquel
grupo observé, mediando principalmente la Reina Madre. En uno de estos
pases y renovación de los corrillos, vine a encontrarme junto a D.
Francisco y la Camarista. Díjome el Rey: «Es preciso hacer tocar a
Guelbenzu las sonatas de ese Beethoven\ldots{} Oirá usted la mejor
música que se ha escrito en el mundo.» Intervino la dama para revelarnos
que como \emph{Los Puritanos} no hay nada\ldots{} Sonó el piano: no me
fijé en lo que tocó el maestro, ni puedo apreciar el tiempo que duró la
tocata. Sólo sé que un ratito estuve en pie junto a la Reina sentada, y
que ella me dijo: «Es natural que no estés alegre, a pesar de la buena
música\ldots{} Comprendo que tienes tu pensamiento lejos de aquí\ldots{}
No creas, por ello te aplaudo. Eres consecuente\ldots» Contesté que nada
echaba de menos, ni lamentaba ausencias; y ella prosiguió: «A propósito,
Marqués, o sin venir a cuento, si quieres: esta tarde he visto a la
moruna y he hablado con ella. Es una mujer interesantísima.» Me
disculpé, negué: vano empeño mío. Levantose Su Majestad, y dando yo
algunos pasos en pos de ella, pude recibir de sus labios esta donosa
prueba de confianza, que me encantó: «Lo sé todo, como dicen en esa
pieza de cuyo título no me acuerdo; lo sé todo, Marqués; te alabo el
gusto.» No me dio tiempo a contestarle, pues era como la mariposa, que
apenas pica en una flor, en busca de otra vuela.

Minutos después, la Reina Madre me preguntaba si conocía yo Nápoles, y
Bravo Murillo se condolió de que yo hubiera desistido de escribir la
Historia de toditos los Papas, obra que sería, sin duda, de las más
edificantes. Ya me iba cargando a mí tanta insistencia sobre un
propósito que nunca tuve; mas como no podía contestar con una grosería,
hube de aguantar la mecha y decir que sí, que no y qué sé yo.
Fácilmente, las conversaciones con personas Reales le llevan a uno a las
mayores hipocresías del pensamiento, y a las más chabacanas formas del
lenguaje. Sólo la Reina con su libre iniciativa y su arte delicioso para
revestir de gracia la etiqueta, rompía la entonada vulgaridad del hablar
palatino. Ya muy avanzada la reunión, en pie los dos, me dijo que no se
contenta con darme a mí la Gran Cruz, sino que también dará a María
Ignacia la banda de María Luisa. Su deseo es recompensar a las personas
que lo merecen, y yo soy de los primeros, no sólo por mi adhesión a la
Real familia, sino por mi inteligencia de escritor, pues si no he podido
escribir aún la Historia del Papado (¡otra vez!), la escribiré, que
viene a ser lo mismo. «Tengo la convicción---añadió,---de que eres de
los buenos, de los seguros, y la independencia que disfrutas garantiza
tu lealtad. Me dijo Narváez que tu suegro era partidario de mi primo
Montemolín, y que tú le has quitado de la cabeza esa debilidad,
ganándole para mi causa. Te lo agradezco mucho. La verdad es que Dios me
ha traído al mundo con bendición, pues bendición es el sin número de
personas honradas que me han defendido, me defienden y me defenderán en
lo que me quede de reinado. He sido muy dichosa\ldots{} Tú calcula los
miles de hombres que se han dejado matar por mí, y los que aún harán lo
mismo cuando llegue el caso, que ojalá no llegue\ldots{} Por eso quiero
yo tanto al pueblo español, y, créelo, estoy siempre pensando en
él\ldots{} ¡Qué pueblo tan bueno! ¿verdad? Él me adora y yo lo adoro a
él\ldots{} Muchas veces, cuando estoy solita, cierro los ojos y procuro
borrar de mi memoria las caras que comúnmente veo, toda esta gente de
Palacio, y los Ministros y Generales\ldots{} Pues lo hago para
representarme el pueblo, de quien sale todo, los pobrecitos españoles
esparcidos por tantas villas, aldeas, valles y montes. Ellos son los que
sostienen este trono mío, y me amparan con sus haciendas y sus vidas. Y
yo digo: «Por fuerza pensarán en mí, como yo pienso en ellos, y al
nombrarme dirán: \emph{nuestra Reina}, como yo digo: \emph{mi
Pueblo\ldots»}

A tan nobles palabras contesté con las más expresivas de gratitud y amor
que se me ocurrían, y pensé que Su Majestad y yo nos parecemos: padece
la \emph{efusión popular}.

«Por mi parte hago lo que puedo para que mi pueblo sea feliz---declaró
Isabel contestando a un concepto mío.---¡Y cuidado si es difícil esto de
la felicidad de un pueblo! Porque uno viene y te dice una cosa, y luego
entra otro y te dice otra cosa, y por aquí salta una capital gritando
\emph{tal y que sé yo}, y por allá otra grita lo contrario. Ya ves que
no es fácil percibir la verdad en medio de esta grillera. Nunca sabe una
si acierta o no acierta. ¿De quién hacer caso, a quién oír? Porque esto
no se estudia, y aunque yo me aprendiera de memoria cuanto dicen los
libros sobre los modos de gobernar, no adelantaría nada. No queda más
que la inspiración, y pedir a Dios que me dirija, que me ponga las cosas
bien claras, de modo que yo las pueda resolver. De Dios viene todo lo
bueno\ldots{} Dios, que ha permitido los sacrificios que este pueblo ha
hecho por mí, me iluminará para que yo no resulte una ingrata.

---Seguramente, la inspiración del Cielo debe guiar a todo Soberano---le
dije permitiéndome aconsejarle sin lisonja.---Pero cuide mucho Vuestra
Majestad de ver de dónde viene, y quién se la trae. Porque entre muchas
inspiraciones verdaderamente celestiales, podría venir alguna que no lo
fuese\ldots{}

---¡Oh, no! ya tengo yo cuidado---replicó.---Las personas que traen la
inspiración de arriba, muy pronto se conocen\ldots{} Mi sistema es
ponerme en brazos de la Providencia. ¿Quién ha sacado adelante mi causa
y este trono mío más que la Providencia? Pues Dios no abandona a Isabel
II, Dios quiere a Isabel II.

---Sin duda\ldots»

Con mucho salero se echó a reír Su Majestad, repitiendo la popular frase
\emph{Fíate de la Virgen y no corras}, y luego añadió: «No: yo no me
entrego a una confianza ciega, ni espero de Dios que vaya diciéndome
todo lo que tengo que hacer\ldots{} Algo ha de discurrir una por
sí\ldots{} yo cavilo también un poquito\ldots{} Verdad que me canso
pronto. ¡Es tan fácil y tan cómodo no pensar nada!\ldots{} Pues sí, yo
pienso\ldots{} Y a donde no llega la razón, llega el sentimiento: ¿no
opinas tú lo mismo? Sentimos una cosa\ldots{} Pues aquello es lo mejor.

---No siempre, señora.

---Sentimos, y\ldots{} Sí, sintiendo acertamos.

---Se corre el riesgo, por ese camino, de sentir y pensar algo que luego
a Dios no le parece bien. Y Dios se vuelve y dice: ¡pero si no es eso lo
que yo te inspiré!\ldots{}

---¡Ay! en lo que Dios inspira no nos equivocamos\ldots{} No hay guía
como nuestro corazón.

---No es mala guía; pero que vaya con él la razón---le contesté
hablándole como a una niña.---Así lo quiere Dios, y si no lo hacemos se
incomoda y nos pega.

---¡Ah!\ldots{} Dios es muy bueno\ldots{} bueno con los buenos, se
entiende, que no tienen malas entrañas. Es soberanamente bondadoso, y se
enfada menos de lo que dicen. Esas voces de los enfados de Dios las
hacen correr los malos, que temen el castigo.

---Nadie como Vuestra Majestad puede asegurar que Dios es bueno\ldots{}
Pero por lo mismo que ha sido tan pródigo con la Reina de España, no
debe la Reina de España pedirle demasiado.

---Vaya, explícame bien eso. ¿Qué has querido decir? Te autorizo para
que me hables con la mayor franqueza.

---Pues diré que Vuestra Majestad tiene un gran corazón, y en él
inmensos tesoros de bondad, de generosidad y ternura que no deben ser
derrochados. No olvide Isabel II la lección de D. Martín de los Heros, y
antes de regalar veinte mil duros de corazón, fíjese bien en el bulto
que hacen apilados estos veinte mil duros de corazón, y asústese ahora,
como se asustó entonces, y rebaje, rebaje, y no dé más que cinco
mil\ldots{} y mejor si los reduce a reales\ldots{} Señora, yo me permito
abusar de la autorización de franqueza que mi Reina me ha dado, y digo
mil disparates, que Vuestra Majestad se dignará perdonarme.

---No, no---dijo Isabel revistiendo de gravedad su picaresco
rostro.---Has hablado como un libro, como hablará la Historia de los
Papas cuando la escribas.»

Un nuevo movimiento de las figuras de la reunión puso fin a este sabroso
diálogo. Volví a encontrarme junto al Rey, mejor dicho, vino él hacia
mí, y me dijo: «¿Y por qué no se decide usted a darnos una Historia de
España verdad? Está por escribir\ldots{} Todo lo que va de siglo es
interesantísimo, y pues no parece fácil superar a Toreno en la guerra de
la Independencia, el historiador que tal emprenda debe empezar en el 14,
cuando mi tío volvió a España\ldots{} Una Historia imparcial, que se
aparte del criterio extremado de las facciones; una relación verídica,
escrita con talento, revisada por personas peritas, y autorizada por la
Iglesia, crea usted que sería una gran cosa. Y la publicación de esa
obra, no faltará quien la patrocine.» Contesté reconociendo la
importancia de un trabajo tan considerable, y la cortedad de mis fuerzas
para realizarlo\ldots{} Arrimose a la sazón la Reina a los que de ello
hablábamos, y éramos ya más de dos, por inopinado crecimiento del grupo,
y nos dijo: «¿Hablan de escribir la Historia de Isabel II? Sí,
Beramendi, sí\ldots{} Yo subvenciono esa obra.

---Es pronto---afirmó el Rey con gran sentido:---no ha de ir el
historiador por delante del Reinado, sino detrás\ldots{}

---¿Y por qué no han de ir juntos, cogiditos de la mano?---indicó la
Reina.

---Porque la Historia verde sabe mal, como la fruta. Hay que dejarla
madurar en el árbol.

---¿De modo---dijo Su Majestad haciendo reír a todos con su donosa
ocurrencia,---que aún estamos verdes? Más vale así\ldots{} Pues yo deseo
que pronto hablen y escriban de mí, por supuesto que escriban bien,
elogiándome mucho y poniéndome en las nubes\ldots{} Yo aspiro a que de
mi Reinado se cuenten maravillas.

---Los pueblos más felices---dijo Montesquieu por boca del Rey,---son
aquellos cuya Historia es fastidiosa.

---Pues yo no quiero---afirmó la Reina,---que al leer mi Reinado bostece
la gente\ldots{} ¡Historia fastidiosa! Eso ni deleita ni enseña.

---La de España---indicó María Cristina, melancólica,---es y será
siempre un folletín.

---Mamá, eso es tener mala idea de los españoles.

---Tengo la que ellos me han dado---replicó la ex-Gobernadora.

---Los españoles son buenos, valientes, honrados, caballeros---declaró
Isabel;---en general, se entiende, porque ¡también hay cada
pillo\ldots!»

Encontrándonos de nuevo frente a frente, me dijo: «¿No crees tú que la
Crónica mía, la de mi Reinado será bella?

---Bella será\ldots{} ¿pero quién asegura que no será también triste?

---¿Por qué?\ldots{} Me asustas\ldots{} Yo no ceso de pensar en mi
Historia, y me la represento como una matrona gallardísima\ldots{}

---Sí, con un laurel en la mano y un león a los pies. Esa es la Historia
oficial, académica y mentirosa. La que merece ser escrita es la del Ser
Español, la del Alma Española, en la cual van confundidos pueblo y
corona, súbditos y reyes\ldots{}

---¡Oh, sí!\ldots{} así debe ser.

---Y esa Historia me la represento yo como una diosa, mujer real y al
propio tiempo divina, de perfecta hermosura\ldots{}

---Vestidita por la moda griega, con túnica muy ceñida, que marque bien
las formas. Así representa el Arte todo lo ideal, así el ser de las
cosas, así el alma de los pueblos\ldots{} Esa figura que tú ves, como
española castiza, será morena.

---Tostada del sol, de este sol de España, que no es un sol cualquiera.

---Y la verás esbeltísima, con poca ropa, descalza\ldots{} no diré que
sucia, sino empolvada\ldots{} naturalmente, de andar por estos caminos y
vericuetos del demonio, por tanta sierra, por tanto páramo\ldots{} País
grandioso el nuestro, pero empolvado\ldots{}

---¡Oh, qué bien lo expresa Vuestra Majestad!»

Al decir yo esto, sentí turbación angustiosa. Hallábame solo, apartado
en un ángulo de la sala. Me asaltó la duda de que la Reina me hubiese
ayudado, dialogando conmigo, a la descripción de la bella figura que veo
y siento\ldots{} Pronto adquirí la certidumbre de que yo me lo había
pensado y dicho solo\ldots{} Cuando dije a Su Majestad que la Historia
de su Reinado podría ser triste, ella no pronunció más que estas
palabras: «¿Por qué?\ldots{} ¡Me asustas!» y se alejó de mí, solicitada
su atención de los otros grupos. Lo demás que \emph{hablamos}, lo hablé
para mí, súbitamente atacado del \emph{mal de Lucila}, de la efusión que
llamo \emph{estética y popular}.

Llegó el instante final. La Reina y demás personas \emph{augustas} nos
hicieron reverencia y se retiraron. Los que no somos augustos nos fuimos
a la calle. En la escalera de Palacio, resplandeciente en la obscuridad
de los jardines, llevaba conmigo la imagen de aquella ideal princesa
Illipulicia, soñada por el celtíbero Miedes. Toda la noche me la pasé en
este delirio\ldots{} Mi cerebro era una linterna mágica. Reproducía en
serie circular la plataforma del Castillo de Atienza, el patio de San
Ginés, un cielo turbio, un suelo árido, una estancia del Alcázar
Real\ldots{} Isabel, vestida de manola, me decía que escribiese su
Historia; Lucila callaba siempre, imagen y representación del inmenso
enigma.

\hypertarget{xxviii}{%
\chapter{XXVIII}\label{xxviii}}

\textbf{San Ildefonso}, \emph{Septiembre}.---El 25 de Agosto, día de San
Luis Rey de Francia, a los pocos de mi doble entrevista con la Reina,
fue para mí memorable, por la aglomeración y enracimado de sucesos que
voy a enumerar. Asistí al Besamanos; vi a Narváez y a Sartorius; vi a D.
Saturno con un resplandeciente uniforme no sé de qué, cubierto el pecho
de cruces y cintajos de variados colorines; en los dorados salones tuve
el honor de ser presentado al Nuncio de Su Santidad, monseñor Brunelli,
y al Embajador de Austria, un caballero muy guapo vestido de magiar; y
en fin, terminada la ceremonia palatina, bajé al parque con toda la
Corte, y corrieron las fuentes en presencia de Su Majestad, soberana
pastora de aquella Arcadia de abanicos. Mi mujer también paseó por los
jardines, y juntos disfrutamos de aquel lindo espectáculo de las aguas
amaestradas y sacadas a bailar sobre el verdor de los parterres y
arboledas. En el teatro, donde cantaron \emph{Don Pasquale} por
despedida, vi a Eufrasia, que con misterio de ópera cómica me dijo que
se hablaba sotto voce de mis \emph{frecuentes visititas} a Palacio. No
le hice caso: yo no había vuelto allá desde la \emph{soirée} que he
descrito.

A Narváez le vi al anochecer en la Casa de Canónigos, y me dijo\ldots{}
¿qué me dijo? Ya no me acuerdo\ldots{} No sé cómo tengo mi cabeza. De
dos semanas acá, mi aturdimiento y mis distracciones graves suscitan
alarmas de mi cara esposa, que inquieta por mi salud me somete a
cariñosos interrogatorios acerca de cuanto hago y dejo de hacer, de
cuanto hablo, pienso y sueño. «No es nada, mujer---le contesto yo, que a
todo antepongo su tranquilidad;---no es más que\ldots{} eso que padezco,
y que me ataca de vez en cuando, la \emph{efusión}\ldots{} ¿de qué?, la
\emph{efusión de lo ideal}, de lo desconocido, de lo que debiendo
existir no existe. Volvemos a lo mismo: yo debí dedicarme a un arte, y
en él habría sido maestro\ldots{} Pero no tengo arte, y mis facultades
funcionan en el vacío\ldots{} No me hagas reír, mujer. ¿Qué dices, que
el ser padre es un arte?\ldots{} ¿padre de muchos hijos\ldots? Bueno,
mujer. Lo admito, si en ello te empeñas\ldots{} Pero ese arte, como la
historia de un reinado que empieza, está todavía verde.»

Ahora me acuerdo de lo que me dijo Narváez. Fue de lo más
insignificante, y en realidad no merece ser transcrito. «Yo me vuelvo a
Madrid, y dentro de unos días saldré para las aguas de
Puertollano\ldots{} Aquí nada tiene usted ya que hacer. Pronto se irá la
Corte. Se le van a usted la \emph{Marquesa de Capricornio} y los demás
enredillos que tiene el \emph{pollo} aquí\ldots{} A mi regreso de la
Mancha espero encontrarle a usted en los Madriles\ldots» En efecto,
pasados algunos días, desapareció la Corte; partió Eufrasia sin
despedirse de mí, y el Real Sitio, árboles y flores, aguas transparentes
y sutiles aires, se adormecían lentamente en una soledad dulce y fresca.
Contenta de esta soledad, mi mujer desea permanecer hasta fin de
Septiembre, y del mismo parecer son sus padres. Yo lo apruebo. Deseo el
descanso.

\textbf{Madrid}, \emph{Octubre}.---Ya estamos aquí. Escribo en el
Congreso. Nada digno de mención nos ocurrió en la Granja después de la
partida de la Corte, como no sea la tranquilidad que disfruté, la íntima
vida que hice con mi mujer, consagrándole yo todos los instantes de mi
vida, y las feroces mañas que va sacando mi hijo, las cuales manifiesta
tirándome del bigote hasta hacerme llorar\ldots{}

La traviesa, la diabólica Eufrasia no ha vuelto a llevarme a la isla de
Paphos \emph{(Casino de la Reina)}. La he visto poco y de prisa,
coincidiendo en visitas, o encontrándonos en el Prado, y no he podido
hablar con ella detenidamente de cosa alguna. Sus ojos, que ni en las
ocasiones de mayor disimulo dejan de ser elocuentes, me dicen que se
halla en grave crisis de ambición o de amor. El anuncio que le hice de
la pronta concesión del título, no produjo en ella la grata sorpresa que
yo esperaba. «¿Y hemos de agradecerlo al \emph{Espadón}?---me
dijo.---Pues que nos titulen \emph{Marqueses de la Ingratitud.»}

Y voy con el asunto que, a mi entender, merece aquí preferente lugar,
por el grande espacio que ocupa en mi espíritu noche y día. Ya dije que
entre los pobres pedigüeños de la parroquia de San Ginés, hay uno con
quien entablé relaciones policíacas, socolor de caridad, tocantes al
descubrimiento de la hermosura celtíbera vista y evaporada en la puerta
de aquella sacra mansión. Mi amigo, que me ha resultado también
celtíbero de los llamados \emph{Ilergetes}, consagró su vida al negocio
de sanguijuelas en tierras de Teruel\ldots{} Es hombre muy corrido;
peleó por D. Carlos en la partida del Serrador, y establecido por fin en
Madrid como herbolista, ha venido por sucesivas desgracias comerciales y
domésticas a la mísera condición presente. Conserva el hombre agilidad
de piernas y lucidez del entendimiento, lo que no es poca ventaja para
el trabajo diplomático que yo le encomendé; pero tales partes pierden
mucho de su energía por la deplorable ruina de otras: uno de los brazos,
envuelto en amarillas bayetas, no funciona; el cuello se le tuerce del
lado izquierdo, los ojos son como fuentes, y la lengua y boca sufren de
un \emph{paralís} que desfigura su sintaxis y su pronunciación, pues por
causa de tal dolencia compone los conceptos al revés, y suele comerse
las primeras sílabas de las palabras más importantes. Con todos estos
inconvenientes, el pobre \emph{Gambito}, que tal es su nombre o su
apodo, me sirve bien, añadiendo a sus incompletas facultades una
voluntad y una diligencia increíbles.

Antes de irme a la Granja, díjome que la hermosa mujer había vuelto, sin
hacer más que llegarse a la sacristía con una carta\ldots{} ¿Para quién?
Para un capellán, que habría estado en la iglesia, sino estuviera en el
cementerio: había fallecido dos días antes\ldots{} Desconsolada se fue
la moza llevándose la carta. ¿De quién era esta? Gambito no lo sabía ni
pudo averiguarlo entonces. A mi regreso de la Granja, estimulado el
hombre por mis donativos, y en espera de mayor recompensa, me da cuenta
de sus minuciosas pesquisas en Agosto y Septiembre, y de ellas resulta
una luz desigual, que tan pronto esclarece el asunto como lo rodea de
mayores tinieblas. Con mi feliz memoria reproduzco textualmente el
informe, componiendo a mi modo la sintaxis, y supliendo las sílabas
comidas: «El \emph{Surez Jeromo} entró servicio de \emph{Colapios} (los
Escolapios) señores Padres de \emph{Tafe} (Getafe), y la su hija, que la
llaman \emph{Cigüela} (Lucihuela), moró en una casa de Madres
\emph{Colapias} donde se arrecogen hijas de Padres, o hijas de
cualsiquiera Madres putativas\ldots» Para que yo descifrara lo restante
de esta jerga hubo de repetirlo una y otra vez, y aún así no pude llegar
a la interpretación exacta. Toda la paciencia del mundo no basta para
poner en claro los trazos de este borrado palimpsesto. Creo haber sacado
en limpio que Lucihuela estuvo unos días en el convento de Jesús, y que
después pasó al servicio de un señor que Gambito llama \emph{Taja}
(ignoro el verdadero nombre, al que creo falta una sílaba),
administrador de los lavaderos del \emph{Pío Infante Don Cisco}
(traduzco: lavaderos del Príncipe Pío, pertenecientes al infante Don
Francisco)\ldots{}

Débil luz, resplandor vago, ¿a dónde me llevas?

\textbf{Madrid}, \emph{20 de Octubre}.---Ayer reventó sobre Madrid una
bomba. Pienso que su estruendo formidable es público ruido de los que
han de llegar a la Posteridad sin que yo los transmita; pero ahí van por
mi cuenta noticias de cómo fue la explosión y de las cóleras y risas que
produjo, refiriendo después el desarrollo de suceso tan extraordinario
hasta su inaudita solución. Desde el jueves por la noche empezaron a
correr voces de crisis, suponiendo en esta los caracteres más
extraños\ldots{} Oílo yo en casa de María Buschental; mas no le di
crédito, y aun me permití negarlo autorizadamente. Por la tarde había yo
visto al Duque de Valencia en su casa, y nada le oí que pudiera ser
vaticinio de cambio de Gobierno. Pero las afirmaciones que hice no
acallaban los rumores, que a cada instante venían más densos y con más
visos de verdad, de esa verdad inverosímil que aquí gastamos. «Hay
crisis---dijo Carriquiri, entrando a media noche;---la crisis más
absurda y más\ldots{} demagógica que puede imaginarse\ldots{} Nada: que
a D. Ramón, sin decirle oste ni moste, le ponen la cuenta en la mano y
le señalan la puerta.» Llegó luego Tassara y nos contó que la primera
noticia de este gatuperio la tuvo Molins, Ministro de Marina, el cual,
comiendo en su casa, recibió un pliego de la Reina, incluyéndole carta
que le había escrito su marido, en la cual este le decía en substancia:
«Narváez y compinches son unos tales y unos cuales, y para que no acaben
de perder a la Nación, hay que sustituirles inmediatamente por estos
caballeros muy dignos cuyos nombres van en la adjunta lista.»

---¿Quiénes son?

---No recuerdo más que al Conde de Cleonard y al Sr.~Cea Bermúdez, Conde
de Colombí\ldots{} La lista ha sido inspirada por personas que traen
recados del Altísimo.

---Esto es ignominioso.

---Esto es simplemente cómico y no puede prevalecer. ¿Y el Duque?

---Al llegar a su casa se encontró con una comunicación semejante a la
que recibió Roca de Togores.»

Puso fin a la confusión Andrés Borrego, refiriendo que aquella misma
tarde (lo sabía de la mejor tinta), habiendo tenido Narváez un soplo de
lo que se tramaba, fue a Palacio y habló a la Reina: «Señora, esto se ha
dicho, esto se susurra\ldots» Y la Reina le contestó riendo: «No hagas
caso. Son patrañas que salen del cuarto de \emph{ese}\ldots» Oyendo
esto, muchos negábamos que pudiera ser verdad; otros lo confirmaban,
algunos callaban, mordiéndose las uñas. «Es forzoso---dijo no recuerdo
quién,---abrirle a la opinión unas tragaderas del tamaño de esta casa.
Según se van poniendo las cosas, todo es posible, todo puede suceder, y
no hay bola, por disparatada que sea, que no entrañe la verdad\ldots» Y
otro: «La historia de España se nos está volviendo folletín.» Y otro:
«Eso no lo inventa usted. Es frase de doña María Cristina\ldots» «Pero
la Reina Madre habló del folletín sin calificarlo, y ahora debemos decir
\emph{folletín malo»}\ldots{} «No, \emph{folletín tonto»}. Y todos
concluían por llevarse las manos a la cabeza, exclamando: «¡Señores,
cómo estará Narváez! Será cosa de alquilar balcones\ldots»

Participando de esta curiosidad, y con medios de satisfacerla, me fui a
la Presidencia. Al bajar presuroso por la calle de Alcalá, me encontré a
San Román que llevaba la misma dirección y objeto que yo, y hablando del
suceso de la noche, entramos en la gruta de la fiera, a quien suponíamos
en el paroxismo del furor. Un ayudante nos dijo en la puerta que el
General estaba en el palacio de la Reina Madre, y que le aguardaban
muchos señores en el salón, ávidos de saber la verdad o mentira de una
crisis que parece comedia. Subimos. Entre los que allí esperaban el
\emph{parto de la Fatalidad} (así lo dijo uno de los presentes, creo que
Bermúdez de Castro), vi a Sartorius y a D. José Zaragoza, Jefe político
de Madrid, el cual hacía rudo contraste con el Ministro, pues si este es
la propia distinción y delicadeza, la sangre fría y comedimiento en
todas las ocasiones, el diputado por Ciudad Real, cenceño, rudo, de faz
temerosa y mirada fulgurante, parece cortado para la acción vehemente y
repentina. Otros había en la sala, entre ellos mi hermano Agustín,
comentando lo que ignoraban o arrojando bilis sobre lo que sabían; a
cada instante entraban más caras de estupefacción, de impaciencia, de
ira\ldots{} Por fin, como todo llega en este mundo, vimos que la mampara
roja se abrió con chirrido estridente, por la violencia del golpe que la
empujara, y entró Narváez con paso y tiesura de gallo, y sin quitarse el
sombrero echó una fulmínea mirada en redondo, diciendo: «Señores, ya lo
ven ustedes: esto no tiene nombre\ldots{} Sí, sí; lo tiene: es una
canallada\ldots{} ¡Ni entre gitanos, señores; ni entre gitanos!

---¿Qué dice la Reina Madre?---preguntó San Luis, que más que anatemas y
desvergüenzas, deseaba hechos para someterlos a un frío examen.

---Doña María Cristina\ldots---contestó el de Loja, ya en el colmo de la
fiereza y de la amargura.---Pues nada, señores: que todos son unos. La
Reina Madre no sabe nada; dice que no tiene arte ni parte\ldots{} y yo
no sé si creerlo\ldots{} no creo nada.

---Yo pongo mi mano en el fuego---declaró Sartorius con cierta
solemnidad,---por la inocencia de la Reina Cristina en este asunto.

Algo más expresó no sé quién en defensa de la ex-Gobernadora.

«Mi General---dijo con acentos de club el Jefe Político,---bien claro
está que la voluntad de Isabel II ha sido secuestrada. Esto es una
intriga, y la primera víctima de la intriga es Su Majestad. O no
servimos para nada, o debemos echar el cuerpo adelante para amparar a la
Reina.

---¡Sacar el cuerpo, yo! Lo he sacado ya mil y mil veces. ¡Si mi cuerpo
¡ajo! es una criba, de los balazos que ha recibido ¡ajo! defendiendo el
trono liberal!\ldots{} Y ya ven el pago\ldots{} El Gobierno, señores, ha
presentado su dimisión. No podía hacer otra cosa sin faltar a la
decencia\ldots{} ¡y a la vergüenza, ¡ajo!\ldots{} Ceder a esto es
declarar que la vergüenza se ha concluido en España.»

Insistió Zaragoza en que esta crisis no es más que una infame celada.
«Corramos a Palacio---gritó con destemplada voz,---rompamos los lazos
pérfidos que oprimen a Su Majestad.

---El que tenga la cara endurecida para los bofetones y quiera ir a
Palacio, que vaya---dijo Narváez sin mirar a nadie, paseándose, la vista
arrastrada por el suelo.---Yo no me expongo a que un mequetrefe con
medias coloradas, o un fantasmón cargado de veneras, me mande salir a la
calle\ldots{} Vámonos a nuestras casas, y que se arreglen como puedan.

---Mi General---le dijo enfáticamente Don José María Mora.---Usted tiene
a su lado la mayoría de las Cortes; usted tiene el Ejército\ldots{}

---Yo no soy ya jefe del Ejército\ldots{} Lo es el general Cleonard, que
a estas horas habrá jurado en manos de la Reina\ldots{} ¿Pero no se han
enterado todavía, ajo?»

Soltó esta bomba gritando en medio de la sala con gesto de ira y
menosprecio, y a sus palabras sucedió un silencio de consternación. Casi
todos los presentes, hasta que oyeron aquella declaración fatídica,
conservaban un resto de esperanza; algunos, ciegos optimistas, creían
que habría componenda, bien porque Narváez hubiese amedrentado a Isabel,
bien porque esta pudiera librarse a tiempo del encantamento que
aprisionaba su soberano albedrío\ldots{} La noticia, dada por el propio
\emph{Espadón}, de que Cleonard juraba, y era ya sin duda Presidente y
Ministro de la Guerra, abatió grandemente los ánimos.

«Pues si es así---murmuró mi hermano Agustín,---digo que esa señora está
loca.

---Encantada, señores, o hechizada como \emph{el} Carlos II.

---El hechizado aquí soy yo\ldots{} y después sacado a bailar---dijo
Narváez pasando de la cólera al sarcasmo.---¿Pues no querían que
refrendara yo los decretos? Todos están locos allá\ldots{} ¡A fe que
tengo yo cara de zurcidor de estos\ldots{} líos! Molins ha ido a Palacio
a ejercer de escribano\ldots{}

---Mi General---declaró el impetuoso Don José Zaragoza avanzando al
centro de la sala,---el Jefe Político de Madrid sabe dónde se ha tramado
este \emph{maquiavelismo}. Ya no tengo por qué guardar secreto. En la
Escuela Pía de San Antón se reunieron esta tarde los que serán
compañeros del Sr.~Cleonard en el flamante Ministerio, y los que han
engañado a nuestra querida Soberana. Los conozco a todos; sé cuanto allí
pasó y cuantos disparates allí se hablaron. Había en la reunión hombres
que quieren ser públicos, y mujeres que lo fueron. Al anochecer
trasladáronse todos en coches al convento de Jesús a recibir
órdenes\ldots{} Lo mismo se hizo hace ocho días; pero la Monja que da la
consigna les dijo entonces: «Aún es pronto, hijos míos. Esperad hasta
que yo os avise. La Reina no cede. Ya cederá\ldots» Hoy, la impostora
les ha dicho que todo estaba hecho, y locos de contento se han ido al
cuarto del Rey, el cual los presentó a su augusta esposa. La
Reina\ldots{} me consta, señores, y lo aseguro como si lo hubiera
visto\ldots{} nuestra amada Soberana habló con ellos un momento\ldots{}
les despidió diciendo a los nuevos gobernantes que mañana jurarán, y
luego rompió a llorar\ldots{} Pues bien, mi General: conozco a todos los
que andan en esta intriga, y tengo notas bien claras de sus
domicilios\ldots{} Con media palabra que se me diga, voy y los prendo a
todos antes que sea de día, sin distinción de sexo, calidad ni estado,
sin reparar en uniformes ni en faldas, ni en hábitos ni en
sobrepellices, y mañana, es decir, hoy, antes de las ocho salen para
Leganés, y de Leganés, por la tarde para donde se disponga, sea Cádiz,
sea Cartagena, que no faltará un cachucho en un puerto o en otro, que
los lleve a tomar los aires de Filipinas\ldots{} Esto haré, si el Jefe
lo manda, y respondo de que no es atropello, sino justicia.»

Pausa. El murmullo que resonó en la sala demostraba cuán feliz y
oportuno pareció a todos los presentes este atrevido plan policíaco.

\hypertarget{xxix}{%
\chapter{XXIX}\label{xxix}}

No tardó en llegar Molins, próximas ya las tres de la madrugada. Es este
un caballero tan acompasado en la vida social como en la política, como
en la literaria. Sus actitudes son como sus versos; sus actos como sus
discursos, y su traje como toda su correcta y atildadísima persona. Su
estatura es aventajada, su talle esbelto, su rostro grave, abundante el
cabello en cabeza y barba, la dentadura perfecta, todo suyo y de
intachable limpieza. En el trato cautiva, en la oratoria instruye más
que arrebata, en la conversación corriente se oye y se le oye con
agrado. Aunque allí le esperaban como agua de Mayo, ansiosos de conocer
lo ocurrido en la refrendación, el Ministro de Marina no se precipitó a
narrar el acto: es hombre que en nada se precipita. Venía de uniforme,
el peinado sentadísimo, sin que un solo pelo se desmandara; traía cara
melancólica, como de quien sabe apreciar serenamente el punto y ocasión
en que los sucesos particulares revisten la suficiente gravedad para
convertirse en históricos. Ama con caballeresco ardor, de índole
política, a nuestra excelsa Soberana y al Principio que representa, y
cree en la ficción Constitucional-Monárquico-Parlamentaria, como se cree
en los Misterios dogmáticos, sin entender ni jota de ellos.

Con elegancia narrativa dio cuenta Molins de su cometido, y la serenidad
y pulcritud de su palabra fueron como bálsamo que aplacaba la irritación
de que los oyentes estaban poseídos. El hecho que refirió habría
carecido totalmente de interés si el cuentadante no hubiera marcado muy
bien en el relato la nota patética, que acrecía su valor histórico. La
Reina, en todo el tiempo que duraron los trámites, \emph{no cesaba de
llorar}, y a la conclusión, su dolor parecía no tener consuelo.

Maravillados escucharon todos esta relación, y la crítica del suceso
adquirió un tinte compasivo. No quedaba duda de que circunstancias y
resortes misteriosos, que los de fuera no podían penetrar, constreñían a
Isabel II a cambiar de Gobierno.

«¡La Reina está secuestrada!---gritaron algunos; y otros:---¡Salvemos a
la Reina!»

Y Ruiz Cermeño, diputado por Arévalo, con calma y agudeza, como hombre
que se precia de penetrar hasta el fondo de las cosas, nos dijo a los
que le rodeábamos: «Esto es un golpe de Estado, un verdadero golpe de
Estado.» Mi hermano Agustín, que tan hondamente se afana por el porvenir
de esta Nación, no dejaba de expresar sus temores: «¡Pero el Régimen,
Señor\ldots! ¿A dónde va a parar el Régimen con estas cosas?\ldots{} Y
ahora precisamente, cuando el Régimen iba como una seda\ldots»

Lo que contó Molins del llanto amargo de Isabel fue desconsuelo y
aflicción de todos, menos de Narváez, el cual, irguiéndose más bravo,
echando por aquella boca terno sobre terno, hizo estas terribles
manifestaciones: «Dejarla que llore\ldots{} Ríos de sangre han corrido
por causa de ella\ldots{} Y ahora nos quiere pagar con lágrimas\ldots{}
No queremos lágrimas, sino justicia, razón y formalidad. Se reina con
juicio, no con lloriqueos\ldots{} Ella se ha metido en este
pantano\ldots{} Pues vea cómo sale. Que la saquen los angelitos, o esa
beata de las llagas asquerosas\ldots{} Nosotros, señores, a nuestras
casas, a ver pasar la mojiganga Cleonard-Colombi. (\emph{Risas}.) Usted,
amigo Zaragoza, ¿qué ha dicho de prender y de encarcelar? De eso se
cuidará el que le suceda, que a estas horas estará usted
destituido\ldots{} y habrán nombrado a un escolapio, o al demandadero de
las monjas. (\emph{Carcajadas}.) El que sea recibirá órdenes de prender
a todos los que estamos aquí, a mí el primero\ldots{} En mi casa me
encontrarán. (\emph{Rumores}.) Con que, caballeros, a dimitir todo el
que tenga posición para ello\ldots{} Arrojarle las posiciones a la cara,
para que vea lo que somos. Que el Gobierno encuentre vacantes la
multitud de plazas que necesita para monagos, cornudos y demás
patulea\ldots{} La orden del día es esta: ¡vergüenza, dimisiones!»

\emph{Conticuere omnes}, y empezó el desfile. Vi salir cariacontecidos a
Esteban Collantes y a D. José María Mora, al corpulento D. Ramón López
Vázquez y al gracioso Vahey, al narigudo Martínez Almagro y al elegante
Lillo. Disponíame yo a partir con mi hermano, cuando me indicó San Román
que me quedara de los últimos, pues el General tenía que hablarme. No
tuve necesidad de aguardar al día, porque Narváez me cogió por un brazo
y llevándome aparte me dijo: «Váyase usted, Beramendi, que es muy tarde.
Mañana charlaremos. Si entre tanto ve usted a esa\ldots{} (y lo soltó
redondo), dígale que le cortaré las orejas\ldots{} cuando la coja, que
algún día será.»

\textbf{Madrid}, \emph{22 de Octubre}.---El viernes 19 fue día grande en
Madrid por lo divertido y fecundo en sorpresas. Desde muy temprano se
estacionaban grupos frente al Principal, signo infalible de jarana o de
expectación, y de doce a una, ya los cafés hervían de gente ociosa, que
es la más numerosa gente de esta capital. Desiertas, según oí, estaban
las oficinas; un sentimiento de ansiosa interinidad lanzaba a los
funcionarios a la calle y a todo sitio donde corrieran auténticas
noticias, y aquí y allá los poseedores del presupuesto encontraban la
nube de famélicos cesantes. En el tiempo que llevamos de Régimen, el
pánico de unos y las esperanzas de otros, confundiéndose, han creado un
mundo de necesidades que ha sido y es en España la principal inspiración
de los poetas cómicos. Hay una rama de la literatura contemporánea
consagrada exclusivamente al \emph{turrón} y a los hambrientos, sátira
en que se moteja a los que comen, y se ridiculiza a los que piden pan,
revelándose el poeta tan necesitado como los \emph{lambiones} que
describe.

En grupos y corrillos se habla del nuevo Ministerio con desprecio y
asombro, y menudean las preguntas maleantes: «¿Pero ese Armesto quién
es?» «¿Pueden ustedes decirme quién es ese Manresa?» Entre miles que no
saben responder a estas preguntas, sale alguno que tiene vagas noticias
de los improvisados \emph{hombres públicos}. «Pues ese D. Vicente
Armesto es empleado supernumerario en el Tribunal de Cuentas, con el
sueldo de veinte mil reales\ldots{}

---¡Vaya una carrerita, señores!\ldots{} ¿Y es por ventura yerno,
sobrino, hermano de leche de alguno de Palacio, o tiene que ver con
monjas?

---Es cuñado del general Cleonard\ldots{} o concuñado, que para el caso
es lo mismo\ldots{} Vaya, señores; yo convido a café y copas al que me
diga quién es Colombi.

---Y yo obsequio con un almuerzo al que me demuestre con datos\ldots{}
ha de ser con datos\ldots{} que Manresa es alguien.

---Hombre, no hay que confundir a Colombi con Manresa, pues de este no
se ha podido averiguar sino que no le conoce ni su familia, mientras que
Colombi es nuestro embajador en Lisboa, y \emph{al parecer} hermano del
Sr.~Cea Bermúdez, de reaccionaria memoria\ldots{} He oído, no respondo
de ello, que ese Sr.~Colombi es persona respetable y que no aceptará el
cargo\ldots{} En cuanto a Manresa, por aquí andaba uno que aseguró
conocerle. Es murciano, auditor de Guerra de la categoría de
capitán\ldots{} y está procesado porque de palabra faltó al tribunal, se
ignora cómo y cuándo.»

Las voces más absurdas y los dicharachos más irrespetuosos animaban los
corrillos de la Carrera de San Jerónimo y calle de Sevilla. «Por más que
me digan, yo sostengo que ese Padre Fulgencio es un mito. No creo en
Padres ni Madres que quitan y ponen Ministros\ldots» «Existe un
\emph{Pae} Fulgencio; pero hay quien dice que es el \emph{Pae} Cirilo,
que se ha cambiado el nombre\ldots» «Todo esto, créanme, es obra de un
tal Isidrito, que fue cerero y hoy la persona de mayor metimiento en la
Concepción Francisca. Todos los días toma café con ese Manresa en los
Dos Amigos, y por las noches lleva los cirios benditos a Palacio, para
encender a la Virgen del Olvido que tiene el Rey en su cámara\ldots» «No
hay que tomar a broma lo de las llagas, que quien las ha visto de cerca
me asegura que son de ley, y que la monja tiene pasadas de parte a parte
las palmas de las manos. Las enseña poniéndose en un escabel con los
brazos en cruz; pero la del costado, por donde se le ve el corazón, la
enseña echándose boca arriba y quedándose en éxtasis\ldots» «Dicen que
el primer decreto de Manresa será para nombrar Obispo al \emph{Pae}
Fulgencio, dándole la mitra de \emph{Aunque os pese}, diócesis de la
calle de la Justa\ldots» «Hombre, no: es calle de las Beatas.»

Por la tarde, no se hablaba más que de las dimisiones que todo el
personal de algún viso arrojaba a la cabeza de los nuevos Consejeros.
Dimitía el Capitán General de Madrid, Conde de Mirasol; el Gobernador
Militar, el Jefe político, el Alcalde corregidor y las Secretarías en
masa de Gobernación y Gracia y Justicia. Al anochecer, decían los
guasones que Armesto no admitía la cartera de Hacienda, y que en su
lugar se nombraba a un bollero ambulante de la Plaza de Toros, llamado
Maza. Corrió el rumor de que el Tribunal Supremo en peso dimitía; que
será nombrado Capitán General de Madrid el General Villarreal, convenido
de Vergara, y Jefe político el Sr.~Ferreira Caamaño. A este señor le
conozco: es diputado a Cortes por un distrito de Galicia, y habla con
gran violencia dando manotazos. Ha sido juez de primera instancia, jefe
político, y hoy está furioso porque el Gobierno no es bastante
reaccionario\ldots{} A costa del Sr.~Balboa, a quien llaman \emph{Don
Trinidad}, corren y circulan enormes chirigotas. Su Excelencia, al tomar
posesión, dijo a los pocos empleados que concurrieron, que él es muy
liberal y que respetará todas las libertades, menos la de imprenta, y
luego preguntó cómo se extendían los reales decretos. Cierra la noche
con una atmósfera tan densa contra el nuevo Gabinete, del cual hacen
descarada burla hasta los chicos de las calles, que hay ya quien
profetiza la vuelta de Narváez antes de veinticuatro horas.

Al entrar en mi casa encuentro un billete de Eufrasia, escrito con todo
el ingenioso disimulo que acostumbra, fingida letra y firma varonil,
diciéndome que tiene que hablarme y que me espera en \emph{Gobernación}
a las nueve de la noche. Según la antigua clave de nuestra criminal
correspondencia, artificio vigente en el verano último, Gobernación
quiere decir la iglesia de San José, como \emph{Gracia y Justicia} es
San Sebastián, y \emph{Hacienda} San Ginés. Las iglesias que no tienen
más que una puerta se designan con nombres de Direcciones Generales; por
ejemplo: \emph{Aduanas} es el Oratorio del Olivar, \emph{Rentas
Estancadas} las Niñas de Leganés\ldots{} La hora que se indica de noche
se entiende siempre de la mañana\ldots{} Fui y esperé su salida por la
calle de las Torres, sitio muy del caso para figurar un encuentro
fortuito, y conferenciar brevemente sobre cualquier asunto, o ponernos
de acuerdo para fijar día y hora de bajar al \emph{Casino}. Generalmente
no eran largos mis plantones, porque a tantas cualidades de tacto y
agudeza, Eufrasia añadía la preciosa puntualidad. Extrañome anteayer su
tardanza, y ya me cansaba de dar vueltas arriba y abajo, cuando me veo
venir presurosa por la calle de la Reina con rumbo hacia mí, a Rafaela
Milagro, vestida del trapillo de andar por iglesias, armada de ridículo
y de un par de libros devotos. Requiriéndome con mirada expresiva para
que a su encuentro avanzara, nos pusimos al habla en la citada calle,
después en la de San Jorge, donde de sus labios oí lo que a la letra
copio, previa la advertencia de que Rafaela y Eufrasia se comunican y
guardan recíprocamente sus secretos con escrupulosa fidelidad: «Pues no
puede venir, Pepe, y por eso vengo yo\ldots{} Me manda que venga\ldots{}
para decirle que no la espere y contarle lo que ha pasado\ldots{} ¡Ay,
hijo! una zaragata horrorosa\ldots{} que si nos descuidamos saldrá en
los papeles, y aumentará el escándalo de esta maldita crisis\ldots{}
Esos señores han faltado, Pepe; se han portado cochinamente, pues harto
les consta que si no es por Eufrasia no cogen el Gobierno\ldots{} Han
sido unos puercos\ldots{} Aguarde que le cuente. Era cosa
convenida\ldots{} si antes no lo supo, sépalo usted ahora\ldots{} que
Saturno sería Ministro de Gracia y Justicia. ¡Qué más natural! ¡Con lo
que él sabe de cosas de clero y curia! Y de que así fue tratado
solemnemente, pueden dar testimonio el señor Cleonard,
\emph{Quiroguilla}, Rodón, y otros que no nombro. Pues dan la lista a la
Reina, y nos encontramos de Ministro de Gracia y Justicia a ese Manresa.
Para mí fue como un escopetazo. Eufrasia se voló\ldots{} Había que
oírla. Nos echamos la mantilla, corrimos al convento de Jesús\ldots{}
«Hija, no se ha podido evitar---le dijeron.---El Sr.~Manresa ha sido
impuesto por quien puede\ldots{} Su nombramiento vino de arriba\ldots{}
Y Eufrasia contestó con salero: «Por eso parece un pájaro que se ha
caído del nido\ldots{} Pues del nido no me caigo yo, y esta me la
pagan\ldots» «Hija, tenga paciencia, otra vez será.»

»Salimos de allí más furiosas que entramos. Eufrasia mandó recado al
Padre Fulgencio llamándole a su casa, y al mediodía\ldots{} pim\ldots{}
el Padre\ldots{} Venía temblando, y entró haciendo mil
zalamerías\ldots{} Que lo sentía tanto, que era resolución
superior\ldots{} que al Sr.~Manresa no se le podían negar
condiciones\ldots{} en fin, que él lo arreglaría esta misma tarde, pues
como gran amigo y capellán de Saturno, contaba con él para el
Ministerio\ldots{} El arreglo, Pepe, vea usted lo que era. Parece que
ayer el Sr.~Armesto le hacía \emph{fu} a la cartera de Hacienda,
abroncado por las perrerías que le dicen los periódicos. Pues si en
efecto no aceptaba, Hacienda ría para Saturno. Eufrasia, hinchadas las
narices, y con ese imperio que tiene, le dice: «Váyase usted ahora
mismo, y antes de la noche me lo trae arreglado en esa forma. Si así no
lo hace, usted y los demás que nos han dado este bofetón, se acordarán
de mí.» ¡Ay, Dios mío, qué cosas pasan! Pues llega el escolapio al
anochecer, sudando como un pollo, y con el resuello tan corto como el
que se está ahogando\ldots{}

---¿Y no traía el arreglo?

---¡Qué arreglo ni qué ocho cuartos! Lo que traía era un miedo
fenomenal. Verá usted\ldots{} Que lo sentía muchísimo; que había tenido
un gran disgusto; que desde luego contara Saturno con la cartera en la
primera crisis parcial; pero que hoy por hoy no podía ser\ldots{} porque
los de arriba\ldots{} siempre los de arriba, habían dispuesto que en
caso de no admitir el Sr.~Armesto, fuera Ministro el Sr.~Maza.

---¿Maza? Por eso anoche se hablaba de un bollero\ldots{}

---No sé si es o no bollero; lo indudable es que a Saturno le han dado
el pastel de gato. ¿Verdad que han sido unos grandísimos puercos? Pues
considere usted ahora cómo se pondría nuestra amiga\ldots{} usted que la
conoce\ldots{} cuando el Padre vino con aquellas tintinimarras. Tormenta
mayor no he visto nunca. Primero, se quedó lívida\ldots{} yo pensé que
le daba algo\ldots{} después soltó la risa, una risa sarcástica, como
esas de las cómicas en el teatro, cuando fingen que se vuelven
locas\ldots{} yo creí que enloquecía de verdad\ldots{} después se encaró
con el escolapio\ldots{} Cristeta, que también estaba presente, y yo
creímos que le pegaba\ldots{} A dos dedos estuvieron sus manos de la
cara del pobre señor\ldots{} Y disparándose en gritos, ¡Dios mío, Dios
mío, qué cosas salieron por aquella boca!\ldots{} Cristeta y yo
aterradas, Saturno gritándole que callase, y ella, mientras más la
amonestaba el marido, más descompuesta y furiosa\ldots{}

---¿Y el Padre?

---De todos colores, mirando por dónde podría escabullirse\ldots{}
Querido Pepe, no me atrevo a repetir los horrores que oímos, y que el
desventurado D. Fulgencio soportó con humildad evangélica\ldots{} Pero
lo más gracioso fue la escena final\ldots{} Salió escapado el escolapio
corriéndose del gabinete a la sala; pero con el azoramiento de la huida
se le olvidó el sombrero de teja; volvía por él\ldots{} ¿Qué hizo
Eufrasia? Agarró el sombrero que estaba en una silla, lo tiró en el
suelo, y bailó sobre él un zapateado, dejándolo como usted puede
suponer. Después lo arrojó a los pies del clérigo, diciéndole: «Váyase
usted pronto de mi casa, mal caballero y peor sacerdote, y no se le
ocurra volver a poner las patas en ella\ldots»

---Y ustedes acudirían a calmarla\ldots{}

---Calle usted, hijo; tuvimos que acudir a Saturno, que nos dio el gran
susto. ¡Vaya un soponcio! A fuerza de refregones, logramos volverle en
sí; pero luego se nos puso gravemente enfermo, y a media noche tuvo un
vómito de sangre\ldots{} El pobrecito me parece que no la cuenta\ldots{}
¡Lo que usted oye!\ldots{} La leona, que de otra manera no puedo
llamarla, está consternadísima. Me dijo: «Rafaela, vete a San José por
la calle de las Torres, y entérale de la situación\ldots» Esta mañana
Saturno ha pedido confesarse.

---¿Pero tan grave está?

---Y no es para menos, Pepe. A cualquiera le doy yo este desengaño.
¡Pues no estaba poco consentido en que sería Ministro! Y sobre el
disgusto, el escándalo\ldots{} El pobrecito ha pedido los
Sacramentos\ldots{} Y aquí me tiene usted con el encargo de buscarle
confesor\ldots{} porque no hemos de llevarle el suyo, que era el dichoso
Fulgencio\ldots{} Ahora, una vez informado usted de estas trapisondas,
entraré en San José, y si no encuentro al padre Morales, iré a Monserrat
en busca del padre Claret\ldots{} Vaya, Pepe, adiós. Le diré que le he
visto a usted tan bueno y tan guapo. Dígame: ¿cree que este maldito
Ministerio durará mucho?

---Muchísimo: según mis informes, tendrá una vida muy larga\ldots{} lo
menos de veinticuatro horas.

---¿Es de verdad? ¡Oh, qué noticia le llevo a la pobre Eufrasia! Aunque
resulte falsa, se consolará con ella\ldots{} Adiós, hijo, adiós.»

\hypertarget{xxx}{%
\chapter{XXX}\label{xxx}}

Página histórica me pareció el verídico cuento traído por Rafaela, y
pensando en él y en la profunda lección que entraña, me fui a correr por
Madrid en busca de las novedades que diera de sí el día, las cuales se
me antojó que habían de ser gordas y buenas. No me equivoqué. Menudeaban
las dimisiones; los valores públicos, que el viernes coadyuvaron no poco
a la rechifla del nuevo Gabinete, bajándose dos enteros, seguirían
descendiendo el sábado, según opinión de todos los agentes y bolsistas
que encontré por las calles. Engrosaban los grupos. Contáronme los
empleados de la Secretaría de Gobernación que D. Trinidad no resolvía
nada, y asombrado de recibir dimisiones, se pasaba el tiempo
enterándose, con infantiles preguntas, de las funciones más elementales
de su cargo. En Hacienda, supe que había tomado la cartera el
Sr.~Armesto, vencidos sus escrúpulos, y en Guerra funcionaba ya el
Sr.~Cleonard, determinando\ldots{} que no podía ni sabía resolver nada.
Por la tarde, cruzando Narváez a pie la Puerta del Sol, fue aclamado por
la multitud. Así se contó en la redacción de \emph{El Heraldo}. No
presencié yo el caso; mis noticias fueron que no hubo aclamación, sino
un respetuoso saludar del público y frases de simpatía. Me lo figuro con
su andar de gallo arrogante, por entre el gentío, recibiendo las
demostraciones afectuosas, y contestándolas no más que con un ligero
movimiento de cabeza, tieso y avinagrado, que así es Narváez ante las
tropas y ante el pueblo.

Por la tarde no falté a su casa, en la calle de Isabel la Católica o de
la Inquisición. Entré y salí, con estos o los otros amigos. Se
acentuaban los rumores de que volvía \emph{El Espadón}. ¿Pero cuándo?
Los más impacientes concedían al nuevo Ministerio ocho días de
existencia. La generalidad opinaba que se le dejaría vivir un mes,
siquiera por decoro de la Prerrogativa regia, pues esta quedará muy mal
parada si los Gobiernos que nombra no hacen más que jurar y dimitir.
Podrá Su Majestad hacer un desatino, mas no es bien que lo confiese, y
todo monárquico fiel debe ayudar a la Reina al disimulo de sus torpezas
políticas. Esto se decía, esto se pensaba. A las cuatro de la tarde
supimos unos cuantos a ciencia cierta, o poco menos, que se planteaba la
contra-crisis aquella misma noche del sábado\ldots{} A las cinco,
repercutían los destemplados acordes de una murga en la calle de
Valverde, donde vive el Sr.~Armesto, y una vez que los felicitantes
atronaron bien la calle, retirándose mustios y sin blanca, porque el
señor Ministro no se hallaba en su domicilio, corriéronse con las
propias intenciones concertistas a la calle Ancha de Peligros, donde
reside, en humilde casa de huéspedes, el Sr.~Manresa, y hasta el
obscurecer escucharon los vecinos el horrible estrépito de clarinetes y
trompas. Mientras el Ministerio recibía estas demostraciones harto
equívocas del entusiasmo popular, corría de mano en mano por Madrid un
soneto de pie forzado, creación repentina de un ingenio muy chusco. Sólo
recuerdo ahora, mientras esto escribo, el primer cuarteto, que dice así:

\small
\newlength\mlenb
\settowidth\mlenb{\quad Temo que el cetro se convierta en báculo,}
\begin{center}
\parbox{\mlenb}{\textit{\quad Temo que el cetro se convierta en báculo,  \\
                Y el Estado, hoy caduco, muera ético,                    \\
                Si otro escolapio en ademán ascético                     \\
                Logra ser del Rey cónyuge el oráculo…}}                \\
\end{center}
\normalsize

No recuerdo bien lo demás. Me procuraré copia de los catorce versos.

A las siete, todo Madrid sabía ya que el Ministerio Cleonard-Manresa, o
\emph{Fulgencio-Patrocinio}, que de las dos maneras se decía, apenas
nacido estaba dando las boqueadas\ldots{} Es muy tarde: yo me duermo.

\textbf{Madrid}, \emph{23 de Octubre}.---Continúo el relato fiel de
estos inauditos sucesos, refiriéndome a la tarde del 21, con lo cual
pego la hebra en el mismo punto en que la rompí. Pues serían las siete
cuando determiné visitar a Eufrasia, compadecido del desdichado D.
Saturno, y anhelando saber si era su enfermedad tan grave como burlesca
fue la sofoquina que la motivó. Llegueme, pues, a la calle de
Fuencarral, frente a la capillita del Arco de Santa María, y subí al
principal de la histórica morada que perteneció al Duque de Montellano.
Al abrirme la puerta, un criado puso en mi conocimiento que el señor se
había tranquilizado después de la confesión, que hizo con grandísima
piedad a las once de la mañana\ldots{} Al mediodía se le dio un
sopicaldo, que no devolvió como se temía, y en aquel momento acababa de
coger el sueño. La señora y Doña Cristeta estaban en la sala con la
Condesa y otras visitas\ldots{} Ya me disponía yo a retirarme, informado
de lo que quise saber, cuando apareció Cristeta, que atisbando desde el
pasillo había conocido mi voz. «Pase, pase, Pepe---me dijo.---Viene
usted que ni bajado del Cielo para sacarnos de estas dudas. ¿Pero es
cierto lo que nos cuenta el amigo Campoi? ¿que corren rumores\ldots{}
vamos, que todo se deshace como la sal en el agua?»

En la sala encontré a Eufrasia, arrebujada en un luengo manto, pálida y
echando lumbre de sus negros ojos; a la veterana beldad, su amiga, cuyo
título de Condesa o Baronesa de no sé qué santo no quiere albergarse en
mi memoria; al respetable auditor que fue del ejército carlino y hoy
diputado por Vera, D. Cristóbal Campoi, acompañado de su señora, y a
otra pareja de dama y caballero que no conocí. Brevemente satisfice la
curiosidad de todos dando cuenta de lo que sabía, y extendiendo la
papeleta de defunción del enteco y llagado Ministerio
Cleonard-Patrocinio-Fulgencio.

«¿De modo---dijo Eufrasia sin reír, más bien lúgubre, como enfermo de
fiebre que se ve obligado a romper el silencio,---de modo que ha sido
como un relámpago?\ldots{} Bien se le puede llamar \emph{El Ministerio
Relámpago.»} Ved aquí el origen de una denominación que aquella noche y
al siguiente día cundió con asombrosa rapidez, y de ella se apoderaron
todas las bocas de Madrid. Renegando de una criatura, en cuyo engendro
había tenido eficaz participación, Eufrasia le administró el agua de
socorro, dándole apropiado nombre, y diciendo al verle expirar: «Es un
fenómeno. No podía vivir. \emph{Relámpagos} al Cielo.» Celebraron los
visitantes la ocurrencia del nombre, y hallándose a medio despejar la
sala, llevome la moruna al gabinete próximo, donde a solas pudimos
hablar un instante. La pulsé: su piel abrasaba. Diome rápida noticia de
su dolencia: sentíase febril en grado sumo; mas el desasosiego nervioso
no le consentía permanecer acostada. Todo su anhelo era ver gente, oír
noticias, enterarse del espantoso ridículo de los Ministros nuevos, y
sólo así se calmaba la sed de su espíritu, ávido de venganza. «Siéntate
un rato, y cuéntame, cuéntame\ldots{} Ante todo: ¿conoces el soneto?
Esta tarde me lo trajo Navarrete. Es graciosísimo\ldots{} ¡Ah! entre las
burbujas del chiste palpitan verdades históricas que andando el tiempo
darán mucho que hablar. Se me ha grabado en el pensamiento el segundo
cuarteto, que dice:

\small
\newlength\mlena
\settowidth\mlena{\quad Mas no a hipócrita Sor, que con emético}
\begin{center}
\parbox{\mlena}{\textit{\quad Venero a Dios, venero al tabernáculo;   \\
                Mas no a hipócrita Sor, que con emético               \\
                Llagas remeda, a cuyo humor herpético                 \\
                Fue quizá el torpe vicio receptáculo.}}               \\
\end{center}
\normalsize

---Sigue, acaba\ldots{} he olvidado los tercetos.

---Yo también. Lo recordaba todo; pero\ldots{} no sé\ldots{} la fiebre
me ha borrado de la memoria el final\ldots{} Dejemos el soneto.
Cuéntame, cuéntame\ldots»

Lo que yo pudiera contarle, al dominio público pertenecía ya. Mayor
interés había de tener lo que ella, como partícipe más o menos esencial
en la conspiración, podía traer al acervo de la Historia, o a los
archivos anecdóticos que guardan quizá la más interesante documentación
de los pueblos. A esto me dijo: «Desengañada y herida, me revuelvo como
mujer contra los que me han traído a esta ridícula situación\ldots{}
Ellos, con apariencia de hombres, se asemejan a nosotras por la viveza
de sus odios ocultos, por el delirio de sus ambiciones disimuladas, y
por el arte de fraguar en la obscuridad las intrigas\ldots{} Todos somos
\emph{unas}\ldots{} La amargura de mi desengaño se me ha derramado por
todo el cuerpo y el alma, y no me consuelo más que con la idea de
abandonar lo que fue mi partido, y pasarme con armas y bagajes al que
quise combatir. Esto es de mujer, y yo soy mujer entera, sin mezcla, de
una pieza en mis odios como en mis cariños. No sé si cuando vengan las
represalias de Narváez, que las gasta pesadas, me tocará alguna china.
Si así fuere, me pongo en tus manos para que me evites cualquier
molestia\ldots»

Sin temor de prometer lo que no podría cumplir, la tranquilicé sobre
este punto, dándole seguridades categóricas de que su nombre no figurará
para nada, en caso de formación de procesos. Y ella prosiguió: «Así lo
harás, Pepe, y yo te lo agradeceré en el alma\ldots{} Ahora no estoy
para largas conversaciones, porque el hablar mucho y vivo me pone los
nervios como cuerdas de violín. Ni podemos entretenernos demasiado,
porque vendrán más visitas, y yo tengo que recibirlas o retirarme. Una
sola cosa te diré esta noche para que los vencedores la tengan en cuenta
y es\ldots{} que me gustaría ver que sentaban la mano de firme.

---La sentarán\ldots{} y duro; todo lo que se pueda sin herir en las
partes más vivas de la Nación, naturalmente.

---¡Ay, ay, ay! Pepe. No harán nada, no perseguirán a nadie.

---¿Lo crees tú?\ldots{} Así será, cuando lo asegura la que podría ser
historiadora de esta intriga, si quisiera.

---¡Historiadora yo!---dijo tristemente, sin poder atajar su
locuacidad.---¡Quién pudiera serlo! Si piensas que yo conozco la
conspiración y sus resortes, estás equivocado. Conozco algo; pero los
móviles hondos, que determinan hechos positivos, han sido y son un
misterio para mí\ldots{} Y vas a ver el misterio más impenetrable, Pepe.
Pon toda tu atención en esto: la Reina se resistió una vez y otra al
cambio de Ministerio que le proponía el Rey. No tragaba a Cleonard y sus
cofrades ni aun envueltos en la confitura religiosa. Y era tal su
resistencia que perdimos toda esperanza. ¿Cómo es que de la noche a la
mañana consiente la niña en despedir a Narváez de mala manera?\ldots{}
Fíjate en esto, Pepe\ldots{} ¿Y cómo es que a su consentimiento
acompañan lloros y suspiros?

---Los lloriqueos parecen indicar que no está contenta de lo que hace.

---O que forzada se ve a determinar lo que no quiere. Yo, que algo
entiendo de cosas palatinas, no me explico este cambio más que por el
miedo. ¿Y cómo han logrado infundirle ese pánico que la pone atadita de
pies y manos a merced de los intrigantes? Voy a decírtelo\ldots{} y
perdóneme Dios esta sospecha, esta\ldots{} inspiración. Para mí, se
apoderaron de un secreto de la Reina, y con este secreto, cogido como un
puñal, la han amenazado, le han dicho: «O eres nuestra o mueres.»

---¿Creerás que entre los infinitos disparates que corren en bocas de la
gente no ha faltado ese?

---Y vosotros los sensatos, los que todo lo veis recortado y medidito,
habréis creído que esos disparates son obra de imaginaciones locas, y un
plagio de los melodramas tremebundos, traducidos del francés.

---Yo ni afirmo ni niego\ldots{} En eso como en todo, el misterio
existe; ¿pero quién es el guapo que lo descifra?

---El guapo, la guapa sería yo, si me dejaran, si me dieran medios de
indagación.

---Aun con tales medios no te lanzarías a poner tu mano en lo más
delicado del asunto.

---Ya\ldots{} tú eres de los que creen que estos misterios son como los
del dogma\ldots{} Se les mira de lejos, se les adora, y es locura
intentar comprenderlos y desentrañarlos.»

Tan exaltada la vi, que para sosegarla hube de emplear este
razonamiento: «Pero dime una cosa, Eufrasia, y apelo a tu conciencia:
¿antes de que esos pícaros le birlaran a tu marido la cartera prometida,
pensabas eso mismo?

---No: entonces no pensaba nada malo de los que eran mis amigos. Todo me
parecía bien. Te abro mi conciencia: estos horrores los he pensado
después, cuando he sido chasqueada vilmente.

---No estás serena. ¿Cómo has de juzgar la maldad de otros, no estando
tú libre de maldad?\ldots{} Pero sea lo que quiera, y dejando a un lado
tu conciencia, respóndeme: la captación infame del secreto, ¿a quién la
atribuyes? Tu lógica infernal\ldots{} seguimos en el melodrama\ldots{}
tu lógica, como aguja imantada por los demonios, ¿señala un punto fijo?
¿Es Fulgencio, es la Monja?

---No: no puedo fijarme en nadie, y ahora que tengo conciencia, menos.
La iniciativa puede haber sido de esos, no lo sé: la ejecución ha sido
de otros. ¿Quién\ldots{} quiénes? Cualquiera lo sabe. Cristeta, que ha
vivido largo tiempo en Palacio, dice que aquello es un mundo, un mar, un
convento\ldots{} ¡Ya ves si será difícil\ldots! En fin, Pepe, tú que tan
en gracia le has caído a Narváez, puedes decirle que no se entretenga en
cazar moscas, esto es, en prender Manresas, Armestos y Balboas, pobres
títeres que no valen el hilo que los mueve\ldots»

Con arrogante voz y ademán, en pie, actuando de ideal dictadora,
completó así su pensamiento: «Que prendan a Fulgencio y le registren
bien la celda\ldots{} que prendan a la Monja y la registren\ldots{} sin
respetar ni celda, ni ropas, ni relicarios, ni altaritos, ni
llagas\ldots{}

---Con todo eso, amiga mía, más fácil será encontrar una aguja en un
pajar que la verdad en un monasterio.

---Que prendan a Rodón, Secretario del Rey\ldots{}

---¿No será más culpable su Gentilhombre, el hermano de la Monja?

---Quiroga, que no tiene más ambición que la de las cruces y cintajos,
no es hombre de travesura\ldots{} Pero nada se pierde con ponerlo a la
sombra\ldots{} El primero a quien deben echar mano es un señor Taja,
administrador de las huertas y lavaderos del Príncipe Pío, posesión Real
cedida en usufructo al Infante D. Francisco\ldots{}

---¿Has dicho Taja? ¿No faltará a ese apellido la primera sílaba? ¿No es
Re-Taja, Mor-taja?

---No\ldots{} Taja no más. Y para que la redada sea completa, caigan
también el hermano de ese señor y su mujer, ujier él, si no estoy
equivocada, azafata ella: viven en los altos de Palacio.

---Esos nombres, esos Tajas masculinos y femeninos---dije yo redoblando
la atención que en la dictadora ponía,---no son desconocidos para mí: en
mi mente están días ha, relacionados con otro asunto, que no pertenece a
la Historia de España; aunque sí, puede que sea de lo más nacional, de
lo más histórico\ldots{} Dime: ¿no es criado, o subalterno de ese Taja
que sirve al Infante, un viejo llamado Ansúrez, de aspecto noble\ldots?

---No sé su nombre; pero he visto al anciano gallardo, de barba blanca y
figura señoril. Dos veces me ha traído cartas del Taja, y por conducto
de él he mandado la contestación.

---¿Y tú sabes\ldots{} haz memoria, rebaña bien en tus recuerdos\ldots{}
sabes algo de una hija de ese viejo noble, guapísima, de extraordinaria
belleza?

---Algo de una moza muy linda oí\ldots{} ¿a quién?\ldots{} a
Fulgencio\ldots{} quizás al propio Taja\ldots{} pero no puedo
asegurarlo. Novicia fue según creo, antes de servir a los Tajas\ldots{}
O me engaño mucho, o algo me dijeron de que por segunda vez volvió al
convento\ldots{} ¿Sabes quién puede darte noticia de esa familia de
padres nobles barbudos y de hijas como estatuas? Pues tu hermana
Catalina.

---¿Y dónde está mi hermana Catalina?

---No sé: si estuviese en Madrid, ella sería, y no te ofendas, una de
las primeras que yo señalaría a los corchetes del Sr.~Zaragoza\ldots{}

---¡Estás loca!\ldots{} ¡Mi hermana!

---Sí, sí: no me vuelvo atrás de lo dicho\ldots{} Si te asustas de
oírme, culpa a mi calentura, que con el mucho hablar se me enciende más
y acaba por trastornarme.

---Y a mí. Me has pegado tu fiebre.

---Pues vete\ldots{} Yo estoy atroz\ldots{} los dos deliramos. Empiezo a
ver visiones.

---Yo también\ldots{} Veo la historia interna de los pueblos, la
historia verdad, representada en una mujer vestida de ninfa, de
diosa\ldots{} no diré que sucia, sino empolvada, de andar por estos
caminos de la vida española, secos, tortuosos, ásperos\ldots{}

---Pepe mío, si has de ponerte malito, vete a tu casa, que bastantes
enfermos tengo yo en la mía.

---Sí, me voy\ldots{} Adiós\ldots{} duerme\ldots{}

---Adiós\ldots{} No olvides mi encargo. Prender, registrar bien\ldots»

Salí: hasta que pude respirar el aire fresco, calle adelante, no me
sentí sereno, en disposición de apreciar las cosas en su sentido y
aspecto real. «Taja, Taja, Taja\ldots» Esto repetía yo, y las dos
sílabas pronunciadas por mi boca, me sonaban como un idioma de
salvajes\ldots{} Ya veía más claro en el asunto que periódicamente me
enfermaba con penosísimas efusiones\ldots{} Ya la fugitiva imagen de
Illipulicia no burlaba mi persecución; ni le valdrían sus disfraces,
manola gallarda o franciscana monja, para perderse en las tinieblas.
Cerca venía ya, y con ella se juntaba, sin confundirse, otra ideal
figura, la majestuosa y gentil Reina, próvida de todos sus tesoros,
enamorada del bien y de su pueblo\ldots{} Las dos andaban hacia mí, sin
que yo pudiera decir cuál venía delante y cuál detrás, cuál de las dos
guiaba y cuál se dejaba conducir.

Deliré aquella noche\ldots{} así me lo dijo mi mujer\ldots{} Pero antes
que os hable de mi delirio, dejadme que acabe el cuento histórico.

\hypertarget{xxxi}{%
\chapter{XXXI}\label{xxxi}}

Si recibió la vida el \emph{Gabinete Relámpago} en la Cámara del Rey, el
golpe de muerte se lo dio María Cristina en su propio palacio, donde
tuvo con Isabel II una larga encerrona. ¿Qué le diría? Lo adivino. El
meollo del extenso sermón de la Reina Madre no pudo ser más que este:
«Hija querida, se puede hacer todo\ldots{} todo precisamente no, pero
bastante sí; se puede hacer mucho. Lo que no puede de ningún modo
hacerse es lo que has hecho.» Grabadas en mi mente la mirada y la
sonrisa, el rostro hechicero de Su Majestad; grabado también en mí su
pensamiento por la honda estampación de sus facciones; metido su
carácter dentro de mi ser, y sintiendo lo que ella siente, expresaré la
idea de que Isabel II, sin conocimiento del Régimen, que nadie le ha
enseñado; sin conocimiento del pueblo que rige, más que por las vagas
impresiones que llegan hasta ella, hizo lo que hizo movida del miedo y
sabiendo que hacía un disparate. La calidad, la intensidad de aquel
miedo es lo que no llego a penetrar todavía; pero he de poder poco, o yo
conoceré ese estímulo de las regias acciones\ldots{} La madre ha debido
de decirle: «¿Por qué antes de cometer esa barbaridad no hablaste
conmigo y con el mismo Narváez? Entre los dos habríamos hallado un medio
de sacarte del conflicto.» Seguramente, Isabel, más fuerte en el sentir
que en el razonar, no responde a su madre, y con infantil silencio, los
ojos bajos, da a entender que reconoce su error y espera un buen consejo
para enmendarlo. La madre (hablo como si lo oyera) le dice: «Hija mía, a
grandes males, grandes remedios. Faltas nacidas de inmensas tonterías
son más difíciles de corregir que las que nacen de un error del
entendimiento. Pero hay que hacer frente a ellas, y corregirlas sin
reparar en sacrificios del amor propio y aun de la misma dignidad. Hasta
la dignidad debe ponerse a un ladito para componer estas roturas\ldots{}
Fuera miedo: vete pronto a Palacio; llamas a Narváez y le encargas de
formar el Ministerio lo mismo que estaba, o como él quiera. Por hacer un
poco de papelón, él se negará\ldots{} se pondrá unos moños de este
tamaño\ldots{} Te dirá que el poder le fatiga\ldots{} ¡y sin el poder no
puede vivir!; te dirá que llames a otros \emph{hombres}; que él no tiene
inconveniente en apoyar a esos \emph{hombres} por servirte\ldots{} ¡y lo
que hará es rabiar como un perro si llamas a otros! No; por hoy no hay
aquí más \emph{hombres} que él y su cuadrilla\ldots{} Más adelante se
verá\ldots{} Tú no hagas caso de los escrúpulos que ha de sacar: son
fingidos y mentirosos\ldots{} Hará la comedia de despreciar lo que más
desea. Tú te aguantas, insistes, haciéndole creer que le tienes por
necesario\ldots{} y nada. Verás cómo Narváez te desenreda esta gran
madeja que has enredado tú\ldots{} Ánimo, hija mía, y a Palacio\ldots{}
Yo iré contigo y estaré al cuidado de ti, no sea que desbarres otra
vez\ldots»

Los que agazapados en la Mayordomía Mayor vimos a Narváez entrar en
Palacio, no dudábamos de que saldría Presidente del Consejo, por más que
la conferencia con Isabel, larga como la Cuaresma, pudo despertar en los
más impacientes algún recelo. A las diez llegó Sartorius, llamado para
el refrendo, llevando de secretario particular a mi hermano Agustín, y
poco después vimos pasar la desconsolada figura del Conde de Cleonard.
Expliconos mi hermano la tramitación que había de llevar a la
\emph{Gaceta} las formas legales e históricas. Cleonard daría la
estocada a su propio Ministro de la Gobernación, D. Trinidad Balboa;
entregaría después los trastos al Conde de San Luis, y este, con la
simple puntilla, remataba prontamente a todo el intruso, llagado y
relampagueante Ministerio, restablecida la íntegra cuadrilla del
\emph{diestro} de Loja. Lo que no nos contó Agustín, que no pudo
presenciarlo, y sí el Gentilhombre, Marqués de Torralba, testigo de la
escena, fue la cruel expresión que Narváez, rara vez comedido en la
victoria, arrojó a la cara del vencido D. Serafín María de Matta, Conde
de Cleonard, cuando este se retiraba de la Cámara regia: «Ahora, váyase
usted a descansar de sus fatigas.» No eran flojas las que debió de pasar
el hombre, llevado a tales trotes por monjas y clérigos, él, maduro ya,
militar de valía, más distinguido en la técnica que en guerreras
campañas, persona, en fin, merecedora de respeto.

Todo quedó, pues, enmendado en la noche del 20 al 21, y al feísimo
desperfecto político se le puso un parche, o se le echó un zurcido, para
que los tiempos futuros no lo conozcan; intento inútil, pues aunque
buena zurcidora es la reina Cristina y no tiene Narváez malas agujas,
entre todos no han podido disimular el desgarrón ni esconder sus
hilachas\ldots{} No eran aún las doce cuando me fui a la Presidencia,
donde Narváez recibía plácemes por su nuevo triunfo, y humaradas de
incienso de los aduladores, que en aquella dichosa ocasión
horrorosamente se multiplicaban. El Presidente, Sartorius y D. José
Zaragoza estaban encerrados. Por mi hermano supe que serían reducidas a
prisión aquella misma noche las siguientes personas: Sor Patrocinio, el
Padre Fulgencio, el Sr. Rodón, Secretario del Rey; el señor Quiroga y
otros, y que se efectuarían no pocos registros domiciliarios en casas
muy principales. Impaciente por hablar con mi D. Ramón, busqué y hallé
un medio de romper la consigna, llegándome a donde los ejecutores de la
ley estaban con las manos en la masa, ávidos de castigo, de venganza, de
sentar en los huesos de todo culpable, o que lo pareciera, los nudos más
duros del garrote de la autoridad. De la mente de Narváez salía
centelleando el famoso \emph{Principio}; con ráfagas de él forjaba San
Luis los rayos, y Zaragoza, juntándolos en haces y probándoles las
puntas, se relamía de gusto y pedía más, siempre más\ldots{}

Con palabra rápida y festiva conté al \emph{Espadón} el saladísimo
chasco de D. Saturno y el trágico furor de mi amiga, la rociada de
improperios con que obsequió al escolapio, y por fin, el donoso
zapateado que bailó sobre el sombrero de teja. Las carcajadas del
General retumbaron con tal estruendo, que creí oírlas repetidas por todo
el edificio, y si no se echó a reír también la cercana Cibeles, poco
debió de faltarle. Puesto a referir, le informé del arrepentimiento de
la moruna, del ardor vengativo con que viene a nuestro partido, y de sus
opiniones acerca del obscuro resorte empleado para vencer y anonadar la
entereza de la Reina. Si todo lo oyó Narváez con regocijo, esta última
referencia le movió a fruncir el ceño y a soltar de sus ojos una
centella de ira, que me hizo temblar. Sobre cuanto dije hizo
observaciones muy vivas; mas sobre aquello puso la losa de su silencio,
y sobre la losa trazó un rayo\ldots{}

«Amigo Zaragoza---dijo Narváez transmitiendo al Jefe Político las ideas
que le sugerí tocantes a prisiones.---Agregue usted a la lista esos
Tajas\ldots{} el que administra la posesión del Príncipe Pío\ldots{}

---Ya está---replicó Zaragoza;---pero se trata de otros Tajas, de un
matrimonio que vive en Palacio\ldots{} ¿No es eso?

---Justamente\ldots{} Y no estará de más, Don José---indiqué yo,---que
sea buscado, cogido, interrogado, un tal Jerónimo Ansúrez, viejo de
aspecto noble, que tiene una hija muy guapa\ldots{}

---Este \emph{pollo}---dijo D. Ramón con salero,---quiere que la policía
se ponga al servicio de sus galanteos, y que le haga una leva de todas
las mozas de buen trapío.»

Apuntados los Tajas y los Ansúrez por la mano del Jefe Político, que
rasgaba el delgado papel añadiendo nombres a la preciosa lista, volvió
el General al recuerdo de Eufrasia y de su furibundo rompimiento con los
del \emph{Relámpago}. «Esa diabla no será molestada en lo más
mínimo---me dijo.---No me pesa tenerla por aliada, pues es más viva que
la pólvora\ldots{} Y del título, ¿qué?\ldots{} Por mi parte, pasado
algún tiempo, no habrá inconveniente en concedérselo.»

A mi casa me fui caviloso y con fiebre, que sin duda me había comunicado
la morisca, y mi mujer me encontró mal, tan mal como en la famosa noche
del encuentro de Lucila en San Ginés. Dormí con frecuentes intervalos de
insomnio angustioso, y no sé si deliraba más dormido que despierto.
Respetando mi turbación en los ratos de desvelo, María Ignacia no me
interrogaba; pero viéndola yo, al apuntar el día, dar vueltas junto a mí
con maternal cariño, más atento a mi sosiego que al suyo, la llamé a mi
lado y le dije: «No es nada, chiquilla: es eso que padezco, la efusión
de lo ideal\ldots{} y todo proviene de que hay un arte que yo debí
cultivar y no cultivo\ldots{}

---El arte que echas de menos será el estudio de lenguas antiguas o
salvajes, porque toda la noche has estudiado conjugando los verbos
caribes, que dicen: \emph{Taja}, \emph{taja}, \emph{taja}.

No, mujer. No pienso yo en lenguas sabias; ni el arte mío perdido es la
escultura, ni la música, ni la poesía: es la Historia interna y viva de
los pueblos\ldots{} Esa Historia no puedo escribirla\ldots{} Para
conocer sus elementos necesito vivirla, ¿entiendes? vivirla en el pueblo
y junto al trono mismo. ¿Y cómo he de estudiar yo la palpitación
nacional en esos dos extremos que abarcan toda la vida de una
raza\ldots? ¿No ves que es imposible? El ideal de esa Historia me
fascina, me atrae\ldots{} ¿pero cómo apoderarme de él? Por eso estoy
enfermo: mi mal es la perfecta conciencia de una misión, llámala
aptitud, que no puedo cumplir\ldots» Tuve bastante tino para contenerme
y callar en el momento de sentir el chispazo de una idea que podría
lastimarla. La idea era esta: «El hombre que no lucha por un ideal, el
hombre a quien le dan todo hecho, en la flor de los años, y que se
encuentra en plena posesión de los goces materiales sin haberlos
conquistado por sí, es hombre perdido, es hombre muerto, inútil para
todo fin grande.» Callé. Ignacia me dijo:

«Pues todo eso de la Historia interna, de arriba y de abajo, lo vamos
conociendo sin andar a vueltas con ideales y fantasías. Nos basta con
tener oídos y ojos.

---¿Qué has de ver ni oír tú, pobrecilla, ni yo, ni nadie?\ldots{} ¡El
vivir del pueblo, el vivir de los reyes! ¿Quién lo ha podido penetrar y
menos escribir?

---Pues bien al tanto estamos de lo que pasa estos días. ¿Qué ha sido
ello? Que nuestra simpática Reina, engañada por esos señores que venían
a casa, y por otros, quiso cambiar de Gobierno. Luego llegó la Madre y
le dijo: «Isabel, eso está mal hecho.» La pobrecita no sabe todavía el
oficio; pero ya lo irá aprendiendo\ldots{} En fin, que ello ha tenido un
buen arreglo, como en las comedias.

---Me confirmo en que sólo conoces la superficial apariencia, la
vestidura de las cosas. Debajo está el ser vivo, que ni tú ni yo
conocemos. Es lo histórico inédito, que dejaría de serlo si yo pudiera
cultivar mi arte.

---¡Qué tonto! No hay más que lo que se ve. ¿Qué hablas ahí del fondo de
las cosas, y de seres vivos que se ocultan? Todo se reduce a que esos
caballeros querían mandar, disponer de los destinos públicos para sus
paniaguados, y no pudieron valerse de otro resorte que el que les dio la
influencia del Rey.

---Si lo sucedido fuese tan vulgar no valdría la pena de contarlo. Hay
algo más.

---Hay, ya lo sé, que estos tales son los carlistas derrotados, el
eterno Pretendiente absolutista, que no ceja. Lo desarman en los campos
de batalla, y acá se viene y trata de infiltrarse\ldots{} Lo que no
consiguió con la guerra lo intenta con el milagro. Ya ves: ha empezado
por procurarse una monja con llagas\ldots{} ¡Vaya una porquería!

---¿Y por qué tiene poder esa monja?

---Porque es una embaucadora lista, y hace creer a muchos, mentira
parece, que está inspirada por Dios.

---Si hace creer eso no es una mujer adocenada.

---Tienes razón: vulgar no es. Talento muy sutil se necesita, y un gran
saber de cosas místicas, para engañar con su falsa santidad al Rey y a
la Reina\ldots{} Y yo digo: ¿me engañaría también a mí si se lo
propusiera? Me da miedo pensarlo\ldots{} No, no, a mí no me engañaba.
Aunque parezco tonta, no lo soy: ¿verdad, Pepe? En esta cabeza mía no
entran tales paparruchas. ¡Ay, Virgen del Carmen, si me oyeran mis
padres y mis tías\ldots!

---Tus tías y tus padres viven de ficciones; tú, si no posees la verdad,
la vislumbras, ves el camino por donde a ella se va\ldots{}

---Veo que los caminos de esa gente codiciosa y milagrera no son los de
Dios.»

Al oír estas palabras de mi mujer, vinieron a mi memoria (¡oh misterioso
contacto de las ideas en nuestra mente!) los dos tercetos del soneto que
corría por Madrid, y con cierto júbilo hube de recitarlos.

\small
\newlength\mlenc
\settowidth\mlenc{\quad ¿Cuestión de religión lo que es de clínica?}
\begin{center}
\parbox{\mlenc}{\textit{\quad ¿Cuestión de religión lo que es de clínica? \\
                        ¿Y darnos leyes desde el torno? ¡Cáscaras!        \\
                        Esto no se tolera ni en el Bósforo.               \\
                        Mas si la farsa demasiado cínica                  \\
                        Se repite, caerán todas las máscaras,             \\
                        Y arderá España entera como un fósforo.}}         \\
\end{center}
\normalsize

---Cálmate, Pepe, y suprime por ahora los versos---me dijo María Ignacia
arropándome cariñosa.---Tienes fiebre.

\hypertarget{xxxii}{%
\chapter{XXXII}\label{xxxii}}

\emph{24 de Octubre}.---Muy tarde me levanté el 21, y antes de salir de
casa, me informaron de que el Gobierno funcionaba con perfecta
regularidad, y de que se habían efectuado las prisiones. A Balboa le
mandaban a Ceuta, en posta; al Secretario del Rey le despachaban para
Oviedo; a Quiroga, para Ronda. El efímero Presidente del Consejo no
había sido preso, pero sí separado de la Dirección del Colegio Superior
Militar. Los cuitados Manresa y Armesto, padecieron tan sólo el sustillo
de una detención, después de la cual se les mandó a casa\ldots{} Del
Padre Fulgencio supe que se le había llevado al Gobierno civil, mientras
la policía le registraba minuciosamente la celda. Luego me enteré de que
se le encontró un cajoncito con bastante dinero en oro y billetes del
Banco, y un retrato suyo vestido \emph{mismamente} de obispo, con
báculo, mitra y pectoral, en actitud de dar la bendición. El revoltoso
clérigo se daba el solitario gusto de anticipar, por medio de una mala
pintura, su elevación al episcopado, que era el ensueño de su vida y la
meta de sus ambiciones. Se decía que le mandaban a la casa que los
Escolapios tienen en Archidona.

Si en estos escarmientos iban de prisa las autoridades, aún no habían
podido poner la mano sobre la venerada y llagada Monja, por estar metida
en clausura. Narváez, que tan valiente parece, y realmente lo es frente
a demagogos, progresistas radicales y conspiradores del estado laico,
anda con pies de plomo allí donde puede tropezar con el fuero de la
Iglesia. Su famoso Principio de autoridad, fulminante espada contra los
perturbadores del orden en las calles o en la tribuna, se convierte en
caña frente a la obscura facción fortificada en conventos, sacristías o
beaterios\ldots{} Más fácil era, pues, tomar las formidables alturas de
Arlabán que forzar los enmohecidos cerrojos del claustro de Jesús. Puedo
dar fe, por haberlo presenciado, de la confusión y rabia de D. José
Zaragoza, que por temperamento habría cumplimentado en un santiamén las
órdenes de apoderarse de la Monja, y por disciplina no podía salirse del
estrecho camino de la legalidad eclesiástica. El hombre bufaba\ldots{}
era un gato, a quien se ordenaba que se pusiese guantes para cazar el
ratón\ldots{} Sartorius, aún más que Narváez, quería que, tratándose de
contener y escarmentar a personas religiosas, se procediera con la
corrección más exquisita. Los que en todas sus campañas por el Orden
eran incorrectos, autoritarios, y no reconocían obstáculo ni miramiento,
en aquella empresa contra sus mayores enemigos procedían con tanta
parsimonia como delicadeza, de lo que resultaba que el gran Principio
era burlado y escarnecido por los delincuentes, y estos a la postre
resultaban los verdaderos poseedores de la Autoridad.

Acordado el destierro de Patrocinio, no era dable llegar hasta ella sin
que el Ordinario permitiera la violación de clausura, y el Ordinario no
podía disponerlo sin previo consentimiento del Vicario de la Orden. He
aquí, pues, a mi Jefe Político, mordiendo los guantes que aprisionaban
sus rapantes uñas, y corriendo a contarle sus cuitas a D. Ramón, que
soltaba todos los registros de su cólera blasfemante, sin resolverse a
embestir como de ordinario suele. Ante la majestad religiosa, la de la
ley se achicaba y sucumbía. Desesperado y reconociendo su impotencia, el
\emph{Espadón} clamaba: «Tráiganme todos los ejércitos carlistas, y me
batiré con ellos; pero no me pongan frente a monjas, protegidas por
vicarios.» En suma, no era ni \emph{Buey} ni \emph{Liberal}, y por no
determinarse a ser ambas cosas, o siquiera una, ha dejado tan incompleto
y deslucido su papel histórico.

Mientras esto se resolvía, en el transcurso de las horas del 21, me fui
en busca de mi buen Gambito, el pobre de San Ginés, y le encontré, sí,
pero con tal turbación en la descompuesta máquina de sus nervios, y tan
avanzado en su tartamudez, que me vi negro para comprender lo que
decirme quería: \emph{«Ñor, Cigüela\ldots{} vento\ldots{} sus\ldots{}
llagas.»} Me determino a traducir que Lucila está en el convento de
Jesús; pero no sé si debo creer que también tiene llagas, o que
simplemente está donde las hay para edificación de los creyentes.
Gambito vuelve a tomar la palabra, o el tartamudeo, y continúa
esclareciendo mis dudas, o aumentando mi turbación: \emph{«Santismas
llagas, ñor\ldots{} Güela convento\ldots{} Sor y Sores\ldots{} Taja
preso\ldots»} Si de esta horrible jerga sale una verdad, la presencia de
Illipulicia en el claustro de Jesús, no he perdido el tiempo, ni es tan
imperfecto el órgano de información que en mi provecho explora lo
desconocido\ldots{}

Por la tarde, hablé con Zaragoza, que ya parecía loco, de la
contrariedad que le causaba su infructuosa cacería monjil. Narváez, a
quien vi después, ponía el grito en el Cielo descargando su verbosidad
injuriosa sobre toda la Corte celestial. Avanzada ya la noche, se obtuvo
el consentimiento del Vicario; pero\ldots{} A cada paso por tan
escabrosa senda, tropezaban los aburridos gobernantes con una nueva
dificultad. Exigía el Vicario que se le presentase una orden del
Nuncio\ldots{} Ved al pobre Zaragoza camino de la Nunciatura, con medio
palmo de lengua fuera. Ya Narváez, en el paroxismo de la rabia, hablaba
de fusilar al primer magnate religioso que se le pusiera por delante.
Bien sabían ellos que el \emph{Espadón} no haría nada\ldots{} Dejaría de
ser poder si lo hiciese\ldots{} Por fin, trajo Zaragoza el
consentimiento del Nuncio; pero\ldots{}

Pero no haría nada mientras el señor Ministro de Gracia y Justicia no le
dirigiese una comunicación exponiendo los motivos en que se fundaba el
Gobierno para quebrantar la clausura\ldots{} Narváez alcanzó el techo
con las manos, y se desahogó en sucias imprecaciones, no sólo contra el
Nuncio, sino contra la madre de tan venerable señor, contra el padre,
los abuelos y toda la familia\ldots{} Ya iba comprendiendo que su
autoridad en aquel caso era irrisoria, y que las limitaciones del poder
que representaba ponían a este bajo las sandalias de poderes más altos.
No hubo más remedio que correr al domicilio de Arrazola, sacarle del
lecho, y hacerle extender de prisa y corriendo la comunicación que había
de ser llave de la voluntad de Monseñor Brunelli, para que éste abriese
la del Vicario, y el Vicario la del Ordinario, y este descorriera sin
violencia los claustrales cerrojos.

A la madrugada del 22, toda la tramitación jurídico-eclesiástica parecía
terminada, y Zaragoza fue al convento decidido a romper las puertas si
se le oponían nuevos obstáculos. Pedile permiso para acompañarle,
disfrazado de corchete, en la interesantísima diligencia que a efectuar
iba, y me dijo que no necesitaba ningún disfraz ni disimulo de mi
persona; que bien podía ir en su compañía como empleado de la Jefatura,
y que si era mi deseo sacar del convento monja o novicia, podía sin
temor hacerlo, pues ya le tenían tan frita la sangre las señoras
franciscanas, que se permitiría la venganza de no mirar por ellas si
\emph{tocaban a violar}, o si alguien promovía la desbandada del místico
rebaño. En la plazuela de Jesús había gran gentío esperando la función
sabrosa y gratuita: hombres de ideas exaltadas, restos de los disueltos
clubs, manolas y \emph{mozos crúos}, el público de las ejecuciones de
pena de muerte y de todo espectáculo callejero. Supimos que antes de
llegar el Jefe Político no faltó quien propusiera quemar el monasterio:
corría entre la multitud el notición de que Patrocinio había intentado
envenenar a la Reina con unas rosquillas, y en este y el otro grupo se
repetían los versos:

¿Cuestión de religión lo que es de clínica, y darnos leyes desde el
torno? ¡Cáscaras!\ldots{}

\small
\newlength\mlend
\settowidth\mlend{\quad ¿Cuestión de religión lo que es de clínica,}
\begin{center}
\parbox{\mlend}{\textit{\quad ¿Cuestión de religión lo que es de clínica? \\
                        ¿Y darnos leyes desde el torno? ¡Cáscaras!…}}   \\
\end{center}
\normalsize

Media hora larga transcurrió antes de que se nos franqueara la puerta
mayor del convento de Jesús. Un clérigo casi enano entraba y salía, y
habría estado saliendo y entrando hasta el amanecer si Zaragoza no
pronunciara, como pronunció, y con toda energía, la última palabra de la
tramitación y de los pretextos y largas para ganar tiempo. Penetramos al
fin, Zaragoza bufando, yo con una emoción que fue de las más intensas
que he sentido en mi vida\ldots{} Pasamos a un ancho recinto donde
estaba el torno. A la voz de trueno del Jefe Político abriose otra
puerta cuyos goznes gimieron; a lo largo de un obscuro pasadizo llegamos
al claustro, donde vimos a toda la comunidad en fila, alumbrada por
faroles que tenían unas monjas, por cirios en manos de otras. Era un
hermoso cuadro de ópera seria, extremadamente seria. No faltaba más que
el canto. Dijo la primera palabra Zaragoza con voz que empezó un tanto
brusca y acabó por ser comedida\ldots{} Siguió un corto silencio,
durante el cual busqué con ansiosa mirada la imagen de Lucila entre los
fantasmas de azul y blanco que componían el coro. No la vi; volví a
recorrer de un extremo a otro la fila\ldots{} Mas no había claridad
suficiente para el examen de tantos rostros, y alguno de estos, situado
en último término, ocultaba sus facciones en la penumbra. La que
claramente vi, por ser la que más descollaba, fue la famosa Patrocinio,
cuyo semblante iluminaban los cirios próximos. Era de extraordinaria
blancura, y afectaba o tenía serenidad grande. En verdad que la Monja de
las llagas me pareció hermosa, y su grave continente, su mirar
penetrante y la tenue sonrisa plácida con que acentuaba la mirada, eran
el exterior emblema de un soberano poder político y social. Sus manos
con guantes blanquísimos parecían de mármol: en ellas sostenía una
imagen pequeña, la Virgen del Olvido, como ofreciéndola en adoración a
los que profanábamos la santa casa.

Oí la voz de Zaragoza, dirigiéndose a la \emph{Sor} con gran mesura; mas
sin atender a lo que decía, eché mis ojos a lo largo de la fila buscando
lo que más me interesaba, y en esto vi al extremo izquierdo unos ojos
negros, que me turbaron y estremecieron. No me miraban a mí, sino a la
llagada Monja con supremo interés fraternal. Era mi hermana
Catalina\ldots{} En contestación a lo que Zaragoza le dijo, la de las
llagas pronunció alguna frase mística que no entendí: tanta unción y
misterio quiso poner en ella. Si en efecto era una embaucadora,
prodigioso arte desplegaba para el dominio de los que caían bajo su mano
milagrera\ldots{} Busqué de nuevo a mi hermana, y la vi andar con lento
paso hacia el centro de lo que llamo coro, por delante de la primera
fila de religiosas. Sor Patrocinio, que a cada instante descollaba más
por su estupenda blancura, por su serenidad y el perfecto histrionismo
de sus actitudes hieráticas, dio un paso hacia mi hermana diciéndole:
«Hija mía, salgamos.»

Acudieron a besarle las enguantadas manos todas las monjas, y en este
desfile pude examinarlas a gusto, rostro por rostro, sin que ninguno se
me escapara. No vi a Lucila: alguna vi que podía ser ella desfigurada de
cara y talle por el hábito y la toca; mas no era fácil
comprobarlo\ldots{} Miré de nuevo\ldots{} No la vi; no estaba: casi,
casi tenía de ello completa certidumbre. Mi hermana pasó muy cerca de mí
sin verme: no concedía el don de su mirada a ninguno de los que
presenciábamos el acto. Salieron las dos, y Zaragoza, que iba detrás, me
cogió de un brazo para llevarme consigo, lo que sentí mucho, porque me
habría gustado quedarme un poco más, apurando mi examen de monjiles
rostros. Salimos. Vi que Patrocinio y mi hermana entraron en un coche de
posta que aguardaba en la calle; que tras ellas entraba también un
clérigo, al cual yo no había visto hasta aquel instante, y tras el
clérigo un seglar, que era, sin duda, delegado de policía. El coche
partió por la calle del Fúcar. Luego supe que las dos monjas con su
Virgen del Olvido iban camino de Badajoz.

Entre la satisfacción y el desconsuelo se compartía mi alma. Si había yo
visto un hermoso cuadro de la vida española, faltábame ver el corazón y
la interna fibra de aquel extraño asunto. «¡Y pensar---me dijo Zaragoza
sombrío, cuando nos retirábamos,---pensar que ni con estos rigores ni
con todos los de la Inquisición, si los empleáramos, llegaríamos a
conocer la verdad\ldots! quiero decir, el resorte principal, el nervio
de este negocio.»

Callé meditabundo. Sin saber de dónde venían, yo sentía esperanzas que
aleteaban cerca de mí. La verdad estaba próxima: yo la descubriría
pronto, yo encontraría la representación viva del alma española. Lucila
se acercaba. «No ceso de pensar en esa verdad que se nos oculta,» me
dijo Zaragoza: y yo a él: «Pienso en lo mismo, Don José\ldots{} y espero
llegar a ella, descubrirla, dominarla, poseerla\ldots» Amanecía.

\flushright{Santander (San Quintín), Julio-Agosto de 1902.}

~

\bigskip
\bigskip
\begin{center}
\textsc{fin de narváez}
\end{center}

\end{document}
