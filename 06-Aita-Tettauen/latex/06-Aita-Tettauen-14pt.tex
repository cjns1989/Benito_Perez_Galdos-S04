\PassOptionsToPackage{unicode=true}{hyperref} % options for packages loaded elsewhere
\PassOptionsToPackage{hyphens}{url}
%
\documentclass[oneside,14pt,spanish,]{extbook} % cjns1989 - 27112019 - added the oneside option: so that the text jumps left & right when reading on a tablet/ereader
\usepackage{lmodern}
\usepackage{amssymb,amsmath}
\usepackage{ifxetex,ifluatex}
\usepackage{fixltx2e} % provides \textsubscript
\ifnum 0\ifxetex 1\fi\ifluatex 1\fi=0 % if pdftex
  \usepackage[T1]{fontenc}
  \usepackage[utf8]{inputenc}
  \usepackage{textcomp} % provides euro and other symbols
\else % if luatex or xelatex
  \usepackage{unicode-math}
  \defaultfontfeatures{Ligatures=TeX,Scale=MatchLowercase}
%   \setmainfont[]{EBGaramond-Regular}
    \setmainfont[Numbers={OldStyle,Proportional}]{EBGaramond-Regular}      % cjns1989 - 20191129 - old style numbers 
\fi
% use upquote if available, for straight quotes in verbatim environments
\IfFileExists{upquote.sty}{\usepackage{upquote}}{}
% use microtype if available
\IfFileExists{microtype.sty}{%
\usepackage[]{microtype}
\UseMicrotypeSet[protrusion]{basicmath} % disable protrusion for tt fonts
}{}
\usepackage{hyperref}
\hypersetup{
            pdftitle={AITA TETTAUEN},
            pdfauthor={Benito Pérez Galdós},
            pdfborder={0 0 0},
            breaklinks=true}
\urlstyle{same}  % don't use monospace font for urls
\usepackage[papersize={4.80 in, 6.40  in},left=.5 in,right=.5 in]{geometry}
\setlength{\emergencystretch}{3em}  % prevent overfull lines
\providecommand{\tightlist}{%
  \setlength{\itemsep}{0pt}\setlength{\parskip}{0pt}}
\setcounter{secnumdepth}{0}

% set default figure placement to htbp
\makeatletter
\def\fps@figure{htbp}
\makeatother

\usepackage{ragged2e}
\usepackage{epigraph}
\renewcommand{\textflush}{flushepinormal}   \usepackage{indentfirst}   \usepackage{fancyhdr}
\pagestyle{fancy}
\fancyhf{}
\fancyhead[R]{\thepage}
\renewcommand{\headrulewidth}{0pt}
\usepackage{quoting}
\usepackage{ragged2e}   \newlength\mylen

\settowidth\mylen{...................}

\usepackage{stackengine}
\usepackage{graphicx}
\def\asterism{\par\vspace{1em}{\centering\scalebox{.9}{%
  \stackon[-0.6pt]{\bfseries*~*}{\bfseries*}}\par}\vspace{.8em}\par}

 \usepackage{titlesec}
 \titleformat{\chapter}[display]
  {\normalfont\bfseries\filcenter}{}{0pt}{\Large}
 \titleformat{\section}[display]
  {\normalfont\bfseries\filcenter}{}{0pt}{\Large}
 \titleformat{\subsection}[display]
  {\normalfont\bfseries\filcenter}{}{0pt}{\Large}

\setcounter{secnumdepth}{1}
\ifnum 0\ifxetex 1\fi\ifluatex 1\fi=0 % if pdftex
  \usepackage[shorthands=off,main=spanish]{babel}
\else
  % load polyglossia as late as possible as it *could* call bidi if RTL lang (e.g. Hebrew or Arabic)
%   \usepackage{polyglossia}
%   \setmainlanguage[]{spanish}
%   \usepackage[french]{babel} % cjns1989 - 1.43 version of polyglossia on this system does not allow disabling the autospacing feature
\fi

\title{AITA TETTAUEN}
\author{Benito Pérez Galdós}
\date{}

\begin{document}
\maketitle

\hypertarget{primera-parte}{%
\chapter{PRIMERA PARTE}\label{primera-parte}}

\begin{flushright}
\textbf{Madrid, Octubre-Noviembre de 1859.}
\end{flushright}

\hypertarget{i}{%
\section{I}\label{i}}

Antes de que el mundo dejara de ser joven y antes de que la Historia
fuese mayor de edad, se pudo advertir y comprobar la decadencia y ruina
de todas las cosas humanas, y su derivación lenta desde lo sublime a lo
pequeño, desde lo bello a lo vulgar, cayendo las grandezas de hoy para
que en su lugar grandezas nuevas se levanten, y desvaneciéndose los
ideales más puros en la viciada atmósfera de la realidad. Decaen los
imperios, se desmedran las razas, los fuertes se debilitan y la
hermosura perece entre arrugas y canas\ldots{} Mas no suspende la vida
su eterna función, y con los caminos que descienden hacia la vejez, se
cruzan los caminos de la juventud que van hacia arriba. Siempre hay
imperios potentes, razas vigorosas, ideales y bellezas de virginal
frescura; que junto al sumidero de la muerte están los manantiales del
nacer continuo y fecundo\ldots{} En fin, echando por delante estas
retóricas, os dice el historiador que la hermosura de la sin par Lucila,
hija de Ansúrez, se deslucía y marchitaba, no bien cumplidos los treinta
años de su existencia.

Quien hubiera visto aquel primoroso renuevo del árbol celtíbero en la
edad de su primaveral desarrollo, cuando con ella volvían al mundo las
gracias y la donosura de la princesa Illipulicia, \emph{secundum}
Miedes, soberano arqueólogo; quien gozara del aspecto helénico, de la
estatuaria majestad de aquella figura transportada de la edad homérica y
emigrante de Troya, no la habría reconocido en la dama campesina de
1859, cuyo rostro y talle iban embutiendo sus líneas en la grasa
invasora, producto en aquel cuerpo, como en otros, de la vida regalona y
descuidada, del comer metódico, del matrimonio sin glorias ni afanes,
con cinco alumbramientos y el trajín de labradora rica, que más convida
al desgaire que a la compostura\ldots{} A poco de casarse, dio Lucila en
engordar, con gran regocijo de su esposo el buen Halconero, que a menudo
la pesaba (en el aparato que le servía para el romaneo de sus carneros,
destinados al Matadero de Madrid), y celebraba triunfante las libras que
en cada trimestre iba ganando aquel lozano cuerpo. ¡Adiós ideal; adiós,
leyenda; clásicas formas, adiós!

Cinco vástagos, reducidos a cuatro por muerte del segundo, componían la
prole de Halconero y Lucila en 1859. Sólo en las facciones del
primogénito, nacido en Diciembre del 52, se reprodujo la hermosura de la
madre; los otros tres, una niña y los dos varoncitos menores, sacaron
las narices romas y aplastadas, características de la raza de Halconero,
y no apuntaba en sus rostros un tipo de atávica belleza. Más que de la
gallarda familia de los \emph{autrigones}, según Ptolomeo, o
\emph{allotriges}, como los designaron Strabón y don Ventura Miedes,
parecían reproducción de los feos y rudos \emph{turmodigos}, que designa
Plinio como pobladores de la comarca llamada \emph{conventus cluniensis}
(hoy Coruña del Condado). El niño mayor, Vicente como su papá, sí que se
traía todos los rasgos étnicos de los \emph{autrigones}; y si viviera el
gran anticuario de Atienza, le diputaría por acabado tipo de la tribu de
los \emph{Segisamunculenses}, que habitaron en Osma, no lejos de la
ciudad donde hubo de ver la luz Jerónimo Ansúrez el Grande, en quien
revivió la más potente y hermosa casta de españoles. Por desgracia suya
y de la familia, el gallardo niño, que se criaba \emph{como un rollo de
manteca} hasta cumplir los tres años, desde esta edad dio en
encanijarse, sin que acertaran a combatir el raquitismo con sus
consumadas artes y buena voluntad el médico y boticario de la Villa del
Prado.

Cayendo y levantándose llegó Vicentito al 59, el rostro como de un
ángel, torcido y desaplomado el cuerpo, y así estaba cuando, de resultas
de la caída de un caballo (de cartón), se le formó un bulto en la
pierna, y este se resolvió en tumor, que hubieron de sajarle los
doctores del pueblo con éxito equívoco, pues luego se reprodujo con mala
traza y acerbos sufrimientos de la criatura. Afligidos los padres, y
temerosos de que su primogénito, si curaba, se les quedase cojo,
acordaron trasladarse a Madrid para emprender allí nuevo tratamiento con
asistencia de los mejores facultativos de la capital. Ved aquí la razón
de que en el verano y otoño del 59 les halláramos instalados en Madrid,
plazuela de la Concepción Jerónima, atentos marido y mujer a las
opiniones de diferentes médicos famosos, y a la probatura de variadas
preparaciones farmacéuticas.

El pobre niño, aunque mejoraba de la pierna, padecía en Madrid de
aplanamiento y opacidad del ánimo, sin duda por el trasplante desde el
ambiente campesino a la estrechez de una rinconada, en la cual ninguna
distracción hallaban sus ávidos ojos ni su despabilada mente. Densas
melancolías le asaltaron; perdió el apetito, y costaba Dios y ayuda
hacerle tomar las medicinas. Imposibilitado de andar, y sujeto a un
encierro y quietud tan contrarios a la viveza de la infancia, no podían
los padres proporcionar al enfermito más distracción que la que pudiera
gozar arrimado a los cristales de un angosto balcón. La plazuela,
abierta sólo por un lado, ofrecía la soledad inquietante de un recodo
traicionero. Las personas que por allí pasaban se podían contar, y eran
siempre las mismas: por la mañana, gentes piadosas, que acudían a las
pocas misas celebradas en la Concepción Jerónima; por la tarde, gentes
de viso en coche o a pie, visitantes del palacio del Duque de Rivas,
frontero a la casa donde habitaban los Halconero. El cascado cimbalillo
y las campanas de las monjas entristecían más aquel apartado lugar con
su tañer continuo, que marcaba diferentes horas del día y de la noche,
haciéndolas odiosas.

Todos se afligían de ver tan mustio al chiquillo; pero sólo su madre, la
persona más lista de la casa, dio en el \emph{quid} de los motivos de
aquella turbación, y propuso el remedio más adecuado, según consta en la
crónica coetánea que nos ha conservado algunos coloquios familiares
entre Lucila y Halconero. «La razón de la tristeza del pobre ángel y de
su desgana para todo ---dijo Lucila---no es otra que el apartamiento de
esta maldita casa en que nos hemos metido, pues aquí no puede distraerse
con lo que más le gusta y enamora, que es ver soldados. El Ejército es
su delirio: sueña con cazadores y se desvela pensando en los artilleros.
En el pueblo, con sólo repasar las aleluyas de tropa que le comprábamos,
aprendió a distinguir los uniformes de toditas las armas, y mi padre le
enseñó a conocer las insignias de grados y empleos\ldots{} capitán,
comandante, coronel, y de ahí para arriba. El día que entramos en Madrid
por la puerta y calle de Toledo, pasaron cuatro lanceros y un cabo, y el
pobre niño sacó medio cuerpo por la ventanilla\ldots{} Creímos que se
tiraba del coche\ldots{} Pues ahora, dime tú si puede estar contento el
hijo en esta plazuela encantada, por donde no pasa un soldado ni para un
remedio. El alma mía sufre y no se queja; es prudentito y aguanta su
tristeza y soledad, pensando que le engañábamos cuando le decíamos: «En
Madrid verás pasar batallones con música, escuadrones de caballería
tocando los clarines, y artillería con cañones y todo.» Y nada de esto
ha visto; ni podrá verlo en mucho tiempo, porque el médico nos dice que
tiene para rato, con la pierna estirada y sin movimiento\ldots{} Hazte
cargo de lo que te digo, Vicente, y considera que necesitamos levantarle
los espíritus al niño, para que el alma ayude al cuerpo, y los dos a la
medicina\ldots{} Al médico no le gusta que esté triste: bien nos lo ha
dicho\ldots{} Si mi consejo vale, salgamos pronto de este escondrijo, y
vámonos a donde encontremos luz, alegría\ldots{} y soldados. En la calle
Mayor, entre Platerías y la Almudena, ha visto mi padre hoy más de tres
y más de cuatro pisos segundos y terceros con papeles\ldots{} Esos
papeles nos están diciendo: «Lugareños, veníos acá.»

No necesitó el rico labrador que Lucila ampliara sus razonamientos, pues
con lo dicho quedó plenamente convencido. «Sí, mujer---fue su
respuesta:---has hablado como quien eres, y toda la razón está contigo.
Hemos de dar al niño satisfacciones de su gusto militar, para que se le
pongan los espíritus en aquel punto de alegría que ha de ayudar a las
potencias corporales\ldots{} Bien dijo quien dijo que alma lleva cuerpo,
y que los humores del físico se arreglan o descomponen según el
mandamiento de esa gobernadora que llevamos en donde nadie la ve hasta
que Dios nos la pide\ldots{} Sin saber lo que hacíamos, hemos metido al
niño en una cárcel\ldots; y a ti, que por estar al cuidado de las
criaturas poco o nada callejeas, tampoco te hace provecho esta vivienda.
Sólo con mirarte día tras día, y sin necesidad de ponerte en la romana,
veo que desde que estamos aquí has perdido tres libras, y mucho será que
no pierdas para fin de año mayor peso\ldots{} Tomaremos una de las casas
que ha visto tu padre en la calle Mayor, para que nuestro pobre
baldadito tenga un buen miradero en que recrearse con los militares que
van y vienen por allí, sueltos o en formación. Y a la cuenta que han de
ser muchos, porque, a lo que parece, la Reina ha determinado declararle
la guerra al Moro, por no sé qué tropelías, y hemos de tener en la Corte
movimiento de tropas; que en Madrid pienso yo que se juntarán las de
toda España para ir a esa guerra, debajo de las banderas de los
Católicos Reyes doña Isabel y don Francisco. ¡Qué regocijo para nosotros
ver que el niño se anima, y animándose suelta el maleficio de la
pierna!\ldots{} Todo ello por la virtud de su entusiasmo, oyendo el
redoblar de sin fin de tambores, y viendo pasar cientos de miles de
hombres a caballo con las banderas de los diferentes reinos de
España\ldots{} Y por cierto que no llego a comprender de quién saca
nuestro hijo tal afición a las armas, pues en tu familia, según me ha
dicho Jerónimo, no hubo guerreros, que se sepa, y en la mía lo mismo. Yo
apaleo las ramas de mi árbol genealógico, a ver si cae un militar, y no
encuentro más que a un don Pierres Jacques, francés de nación, al
servicio de España, primo segundo de mi abuela materna, el cual don
Pierres perdió un brazo en la defensa de Mahón, allá por los tiempos de
Maricastaña. Venga de donde viniere la devoción militar del niño, Dios
nos le conserve y nos le cure para que sea un buen soldado de su
patria\ldots{} que en este caso digo yo: «alférez te vean mis ojos, que
general, como tenerlo en la mano.»

Transcurrida una semana después de esta conversación, ya estaba la
familia en su nueva casa, calle Mayor, esquina a Milaneses, todos
contentos y Vicentito en sus glorias, pues raro era el día, que no veía
pasar un batallón de línea o de cazadores atronando la calle con su
vibrante música. Le encantaba la infantería, los de a caballo le
embelesaban y los artilleros le enloquecían. A poco de vivir allí,
pasándose las horas arrimadito al balcón, extendida la pierna sobre
cojines, sabía de milicia y de jerarquías militares casi tanto como la
guía de forasteros\ldots{} Y en esto ocurrió que un día de aquel mes y
año (Octubre de 1859) entraron de la calle Jerónimo Ansúrez y don
Vicente Halconero, este último con el rostro encendido por ráfagas de
entusiasmo que de los ojos le salían, la voz balbuciente: «Lucila, hijos
míos---exclamó plantado en medio de la sala,---declarada la
guerra\ldots{} la guerra\ldots{} de\ldots{} clarada en el Congre\ldots{}
¿no lo creéis?\ldots{} greso\ldots{} Congreso levántase O'Donnell y
dice: «Gue\ldots{} al Moro, guerra\ldots{} declarada por
O'Donnell\ldots» Tras de Halconero permanecía rígido y mudo Jerónimo
Ansúrez: su rostro castellano, de austera y noble hermosura, que podía
dar idea de la resurrección de Diego Porcellos, de Laín Calvo o del
caballeresco abad de Cardeña, expresaba un vago renacer de grandezas
atávicas.

\hypertarget{ii}{%
\section{II}\label{ii}}

Había sufrido el rico labrador de la Villa del Prado un ataque ligero de
parálisis, meses antes de lo que ahora se cuenta. Fue un aviso de su
naturaleza apoplética recomendándole que se moderase en el comer. Sujeto
a un régimen de sobriedad por su cara esposa, tasaba sus atracones en la
comida y particularmente en la cena, con lo que se le compuso aquel
desarreglo, quedándole sólo el achaque de tartamudear en los momentos de
viva emoción o de coraje, y la inseguridad de piernas\ldots{} La
prudente Lucila le recomendó aquella tarde (22 de Octubre, si no miente
la Historia) que no tomase tan a pecho la guerra que se anunciaba, pues
él no estaba para bromas, ni podían hacerle provecho los malos ratos que
suelen darse los patriotas por saber quién gana o pierde las batallas.
No podía someterse el buen señor a este criterio, porque las glorias de
su patria le importaban más que la vida, y prefería morir de un reventón
de gusto a vivir en la indiferencia de estas glorias ahora refrescadas.
Aquella noche, cenando y empinando más de lo determinado por la discreta
Lucila, se dejó decir que España entraría en Marruecos por una punta y
saldría por otra, no dejando títere ni moro con cabeza en todo el
imperio. Y no debían los españoles contentarse con hacer suya toda la
tierra de berberiscos, y abatir sus mezquitas y apandar sus tesoros,
sino que al volverse para acá victoriosos, debían dejarse caer como al
descuido sobre Gibraltar, y apoderarse de la inexpugnable plaza antes
que la Inglaterra pudiese traer acá sus navíos. Una vez dueños del
famoso peñasco, quedaría bien zurcido aquel jirón de la capa nacional, y
ya podíamos los españoles embozarnos muy a gusto en ella.

También en el viejo Ansúrez hervía la efusión patriótica; mas no eran
sus demostraciones tan infantiles como las de Halconero. Su espíritu
reflexivo, dotado de tanta claridad y agudeza que fácilmente penetraba
hasta la entraña de todas las cosas, ponía en el examen de la anunciada
guerra el sentido más puro de la realidad. «Buena será esta
campaña---decía,---y debemos alabar al señor de O'Donnell por la idea de
llevar nuestros soldados al África; que así echamos la vista y el rostro
fuera de este patio de Tócame Roque en que vivimos. ¡Con doscientos y el
portero, que ya nos apesta la política, siempre el mismo sainete
representado en los mismos corredores de vecindad! Bien, muy
bien\ldots{} Pero esta guerra será dura, y nos ha de costar trabajo
volver con provecho y gloria. No es el moro enemigo de poca cuenta, y en
su tierra cada hombre vale por cuatro\ldots{} Otra cosa les digo para
que se pongan en lo cierto al entender de guerras africanas, y es que el
moro y el español son más hermanos de lo que parece. Quiten un poco de
religión, quiten otro poco de lengua, y el parentesco y aire de familia
saltan a los ojos. ¿Qué es el moro más que un español mahometano? ¿Y
cuántos españoles vemos que son moros con disfraz de cristianos? En lo
del celo por las mujeres y en tenerlas al por mayor, allá se van unos
con otros; que aquí el que más y el que menos no se contenta con la
suya, y corre tras la del vecino. Los harenes de aquí se distinguen de
los de allá en que están abiertos, y así nuestras moras salen y entran
cuando les da la gana, y hacen su santo gusto. No hay cosa más fácil que
venir acá un moro, aprender el habla en poco tiempo y hacerse pasar por
español neto. Yo he conocido un moro de Larache, que aquí se llamaba
Pablo Torres, y ni el diablo conocía el engaño. Las caras y los modos de
accionar son los mismos acá y allá; y si se pudiera cambiar fácilmente
de lengua como de vestidos, vendría la confusión de pueblos\ldots{} Yo
he visto el parentesco muy cerca de mí. Mi segunda mujer, alpujarreña,
me tenía siempre la casa llena de sahumerios, y sabía poner el alcuzcuz.
Contábame que su madre se pintaba de amarillo las uñas, y que su padre
se sentaba siempre en el suelo con las piernas cruzadas. Era mi señora
suegra mujer humilde, y según me contaron, no se incomodaba porque su
marido, mi señor suegro, se regalase con otras dos mujeres de añadidura.
Con que ya ven\ldots{} Otros ejemplos sacaré si por lo que he dicho no
me confiesan que esta guerra que ahora emprendemos es un poquito guerra
civil\ldots{} Pero civil o de naciones, adelante con ella, y veamos otra
vez a Cristo vencedor de Mahoma. Yo digo\ldots{} oigan esto\ldots{} yo
digo que entre un vascongado que se deja matar por don Carlos y por la
Virgen, su Generalísima, y un andaluz de los que por la Libertad se
metieron con Torrijos en la trampa de González Moreno, hay más
diferencia que entre el malagueño y el berberisco que ahora van a
pelearse por una brizna de honor\ldots{} o por el viceversa de quítate
tú, Alcorán, para ponerme yo, Evangelio\ldots»

En este punto le interrumpió su hija, que con cierta inquietud veía las
frecuentes libaciones del celtíbero entre bocado y bocado de la cena.
«Padre---le dijo,---ha bebido usted más de la cuenta, y ya empieza a
desbarrar. Cierre el pico, y váyase a la cama.» Pudo más en Halconero el
efecto congestivo de la cena que el interés del tema de África, y
hundiendo en el pecho la barba y alargando los morros, atronó el comedor
con la cadencia de sus ronquidos. El niño Vicente, sentado junto a su
madre, se comía con los ojos al abuelo, y no perdía sílaba de las
extraordinarias opiniones de este sobre Moros y Cristianos. A todos les
levantó Lucila de la mesa, arreando con empujones a su marido, cargando
con ayuda de Jerónimo al chiquillo enfermo. Ya los otros dormían\ldots{}
No tardó Halconero en estirar su pesado cuerpo en el lecho matrimonial,
bramando con más fuerza y más desahogo de pulmones. Ansúrez se metió en
su cuarto. En el próximo a la alcoba principal, desnudaba Lucila a su
hijo enfermo para meterle en la cama, y el chiquillo, más despabilado
aquella noche que de costumbre, no paraba en su charla candorosa.
«Madre---decía,---y ahora, con esta guerra, ¿qué hará mi tío Gonzalo
Ansúrez, que se hizo moro antes de que yo naciera, mucho antes, y allá
vive como un príncipe? Tú me contaste que tiene palacio de mármol, y
muchas criadas moras que le arreglan la cama de seda y le sirven la
comida en platos de oro\ldots{} Tú me dijiste\ldots»

---Cállate, hijo mío: si te calientas ahora la cabeza, te desvelarás, y
tú y nosotros pasaremos mala noche.

---Tú me decías\ldots{} ¿ya no te acuerdas?\ldots{} Fue cuando estuve
tan malo, tan malo ¡ay!\ldots{} parecía que me metían en la carne clavos
ardiendo\ldots{} Para que tomara las medicinas, me decías: «Va a venir
tu tío Gonzalo el moro, y te traerá muchos regalos, un vestido verde
bordado de oro, espadas muy bonitas, y un caballo\ldots{} de carne.»
Dice mi abuelo que los caballos moros son los mejores del mundo\ldots{}
corren como el viento, y no les falta más que hablar para ser como las
personas\ldots{} Pues ni vino mi tío, ni me trajo el caballo, ni
nada\ldots{}

---Cállate\ldots{} que no podrás coger el sueño, y te entrará calentura.

---Y yo te pregunto ahora: si la Reina de España le declara la guerra al
Rey de los moros, ¿qué hará mi tío don Gonzalo? ¿Peleará con los de
allá, o se vendrá con los españoles? Contéstame pronto.

---Yo no sé nada\ldots{} Mañana lo averiguaremos.

---Porque si no pelea con los cristianos, ni es caballero ni
español\ldots{} ¿Cómo quieres tú que yo duerma, pensando que mi tío es
traidor a España?\ldots{} Tú sabrás si se hizo mahometano de verdad, o
de comedia, con el aquel de sonsacar los secretos de la morería y
contárselo todo al Gobierno español.

---¿Qué sé yo de eso? Ea, niño, a dormir.

---Pues dime que vendrá mi tío a tratar con la Reina del modo de
embestir a esos perros\ldots{} y a traerme el caballo\ldots{} Mira,
madre, armas no quiero, porque yo aquí no voy a matar a nadie\ldots{} El
caballo sí me hace falta\ldots{} porque la pierna se me va
curando\ldots{} En cuanto que pueda doblar la rodilla, cojo mi caballo,
me monto en él, y verás\ldots{} Te digo que lo manejaré como a los de
cartón, y para que sea manso y bueno, le daré terrones de azúcar y
alguna mantecada de Astorga\ldots{} Verás cómo lo hago brincar y correr.
Ya sé que tú y Nicasia os pondréis a chiflar de miedo cuando me veáis
metiéndole las espuelas para que corra más\ldots{} No tengáis cuidado,
que no me caeré\ldots{} Sé montar\ldots{} Soy un gran jinete, madre, un
gran jinete\ldots{}

Tanto hizo Lucila por sosegarle, poniendo una de cariño y otra de
autoridad, que el chiquillo se calló\ldots{} se durmió\ldots{} Mas no
fue su sueño tranquilo: a media noche daba voces\ldots{} reía,
suspiraba\ldots{} le dolía la pierna\ldots{} el caballo no quería
pararse, y corría por rápida pendiente hacia un despeñadero. Acudió su
madre a medio vestir, y no bastando sus caricias para calmarle, se
acostó con él. Sacudidas nerviosas interrumpían el sueño del pobre hijo.
Lucila no cesaba de pulsarle. «No tiene fiebre---se decía;---no es nada:
es tan sólo el talento, que por ser mayor de lo que corresponde a la
edad del niño, no le cabe en la cabeza\ldots»

La certera observación hecha por Vicentito respecto al caso de su tío
Gonzalo Ansúrez, quedó bien fija en el pensamiento de la madre. ¿Qué
partido tomaría, en la guerra de España con Marruecos, el español que
había renegado de su pueblo y de su fe, adoptando la religión y patria
berberiscas? De esto habló Lucila con su padre al siguiente día, y el
celtíbero no se mordió la lengua para contestarle: «Si tu hermano fuese
un lameplatos y un roemendrugos, tal vez se aprovecharía de la guerra
para decir \emph{yo pequé}, y arrimarse a los suyos. Pero Gonzalo es
allí hombre de riñón bien cubierto; vive considerado de grandes y
chicos, y el mismísimo señor Sultán le llama su amigo, toma de él
consejo, y le ha obsequiado con algunas cargas de dinero
contante\ldots{} En Tetuán se ha establecido, y su casa, si no la mejor,
no es de las peores del pueblo. Comercia en lanas, comercia en
almendras, y de un punto que llaman Tafilete le traen sus recuas de
camellos, un mes sí y otro no, pieles magníficas, de las que manda una
parte a Marsella, y otra parte allí se queda para ese calzado ancho y
suelto que llaman babuchas. Todo esto lo sé por aquel señor que de allá
vino el año pasado, y me trajo carta de mi hijo, acompañada de las cinco
onzas que te di para que me las guardaras. Era el mensajero un señor
llamado don Jacob Méndez, que los más de los años viene a España y la
recorre de punta a punta, comprando esmeraldas, que ahora están en alza,
y aljófar, perlitas menudas, que en la Morería tienen gran salida y
precio muy bueno. El tal me pareció hombre corriente y de mundo. Aunque
no hablamos palabra de religión, túvele por judío: su nombre, su rostro
afilado, su desconfianza y el comercio que traía, así me lo declaraban.
Se aposentó en la Posada del Peine; allí le vi dos o tres tardes, y me
refirió de mi hijo mil cosas que yo ignoraba, pues sólo dos veces tuve
con él correspondencia escrita. Lo que el señor don Jacob me contaba fue
para mí de grande admiración, y más que nada me agradó saber que Gonzalo
es hombre de cuenta, y que ha labrado su acomodo con el trabajo y el
buen cumplimiento comercial. Habla la lengua arábiga tan de corrido como
si la mamara con la leche. Y es al modo de literato, porque en romance
llano y en copias altas escribe cosas magníficas que suspenden. Es
querido y respetado de todos\ldots{} También tuvo sus quiebras el pobre
hijo mío, pues en un pueblo que llaman Alcázar-Quebir tomó partido por
un bando de dos o tres que se formaron en no sé qué revuelta, y su
cabeza estuvo a dos dedos de ser cortada. Milagro fue que escapara; pero
aquello se arregló cortando y salando otras cabezas, y con la paz volvió
Gonzalo a la querencia del señor Sultán, lloviendo sobre él riquezas y
honores\ldots{} De estas cosas y otras que sé tocantes a tu hermano, no
he hablado contigo todo lo que quisiera, porque rara vez te encuentro
sola, y delante de Halconero no nombro yo a mi Gonzalo por nada de este
mundo. Ya sabes que a tu marido le hace poca gracia tener un cuñado
mahometano, y dice que mayor deshonra no podría caer sobre mi familia.»

Requirió Lucila a Jerónimo para que le dijese el nombre arábigo que en
su vida musulmana usaba Gonzalo, y Ansúrez dijo que habiendo interrogado
sobre ello al buen don Jacob, este pronunció una gran retahíla de voces
que eran como si echase fuera el aliento para volverlo a tomar,
escupiendo sílabas, una por una, después de enjuagarse con ellas. «Como
yo no entendía nada de aquel murmullo---añadió Ansúrez sacando de su
bolsillo una mugrienta cartera, y de esta un papel,---le rogué a don
Jacob que me lo escribiera con letras castellanas, para ver de
aprendérmelo de memoria\ldots{} Aquí lo tienes. Por más que he trabajado
en retener estos terminajos, aún no puedo pronunciarlos de corrido. En
el largo rótulo se dice que Gonzalo se llama como Mahoma, que es hijo
mío, y que ha estado en la Meca, lo cual es tener como un divino
certificado de fiel creyente.»

Leyó Lucila en el papel este nombre de nombres, trazado con elegantes
rasgos que parecían de cálamo más que de pluma: \emph{Sidi El Hach
Mohammed Ben Sur El Nasiry}.

\hypertarget{iii}{%
\section{III}\label{iii}}

«Madrita, también a ti te gustan los militares\ldots{} no me digas que
no\ldots{} Bien conozco que te gustan, picarona\ldots{} No pasa tropa
formada, con música, sin que te asomes conmigo a verla\ldots» Esto decía
Vicentito a su madre, ambos en el balcón, viendo la cola de un
regimiento que desfilaba con marcial ritmo hacia el centro de la Villa.
Ya llegaba la banda a la casa de Cordero; ya la vanguardia de
chiquillos, fascinados por los graciosos aspavientos que hacía con su
bastón de porra el tambor mayor, se espaciaba en la Puerta del Sol; ya
la bandera iba más allá de Platerías, replegada, firme como una antena
en mar tranquilo; las últimas filas de la formación, semejante a un
inmenso anélido, pasaban bajo los balcones de Lucila. Esta respondió a
su hijo, acariciándole el cabello: «Miro a los soldados porque te gustan
a ti, tontín. Si no fueran tu delirio los soldados, yo ni los miraría
siquiera.»

---Estos soldados son los más guapos que he visto. Llevan uniformes
nuevos. Les he mirado el número, que es un 7.

---El 7 es \emph{África}.

\emph{---¿África} el \emph{7}? Y luego dices que no entiendes de tropa.
Si sabes todos los números de la infantería de Línea y de Cazadores,
¿por qué no me los enseñas?

---Conozco algunos\ldots{} muy pocos\ldots{} números sueltos que se
aprenden sin saber cómo.

---Yo no sé pasar de los primeros: 1, \emph{Inmemorial del Rey}; 2,
\emph{Reina}; 3, \emph{Príncipe}; 4, \emph{Princesa}; 5,
\emph{Infante}\ldots{} No has querido enseñarme más.

---Pues sigue la cuenta: 6, \emph{Saboya}; 7, \emph{África}\ldots{}

---Y más, más, madrita\ldots{} dímelos todos.

---No sé, no sé, hijo\ldots{} No te pongas pesado. ¿De dónde quieres que
sepa yo esas cosas?

---Un día\ldots{} bien me acuerdo\ldots{} pasaban Cazadores, y tú
dijiste: 11, \emph{Arapiles}.

---Sí\ldots{} ese número sé por casualidad\ldots{} por casualidad sé
otros, como 28, \emph{Luchana}.

---¿Cazadores?

---No, hijo: \emph{Luchana} es de Línea.

Insistió el gracioso chiquillo; pero su madre tuvo arte para poner punto
final a un tema que la mortificaba. Como la mañana estaba fresca, hízole
retirar del balcón, acomodándole en el sofá de Vitoria con blanda
colchoneta, donde pasaba las lentas horas. Se aproximaba la de la visita
del médico, que de día en día hacía más lisonjeros augurios\ldots{}
Llegó el doctor más pronto de lo que se esperaba, y mientras duró el
examen de la pierna y se hizo la cura, mortificando grandemente al pobre
chico, riñó a este y a la madre porque no se observaba la quietud
indispensable para la curación. Debía Vicentito moderarse en sus
entusiasmos militares y ecuestres, esperando mejores días para
entregarse a ellos. Lloriqueaba el enfermo, no tanto por el dolor de la
cura como por ver que se le tasaban los goces de su ardiente afición.
Halconero le consolaba con la promesa de traerle una colección de
\emph{vistas de batallas} que, puestas dentro de una caja negra, se
miraban por un cristal de aumento, y ello resultaba como si estuviese
uno en medio del campo de acción viendo pelear a moros con cristianos.
Era la campaña de los franceses en Argel, en láminas iluminadas, que
parecían la verdad misma, todo muy propio y con su color natural. Con
esto se fue sosegando el chico y resignándose a la quietud. Solo con su
madre otro día, al caer de la tarde, le dijo: «Me estaré quieto si tú
estás conmigo siempre, y me cuentas cosas, aunque no sean cosas de
militares. A ti te quiero más que a nadie, y todo lo que me dices me lo
creo, aunque sea mentira\ldots» Entretúvole Lucila con diversas
historias, mitad verídicas, mitad inventadas por ella: consejas de
animales, de cacerías de leones, de naufragios terribles, de islas que
salían del mar y se volvían a meter en él, de milagros estupendos y
apariciones de vírgenes en un árbol, en una peña, en una gruta\ldots{}

«Espérate un poco, madrita---dijo el chico con jovialidad
picaresca,---que tengo que hablarte de una cosa. Ahora me acuerdo, por
las apariciones que me estás contando. Hace tres noches, aquella noche
que saliste con padre a dar el pésame al señor de Centurión porque se le
murió su mujer\ldots{} Pues aquí se quedaron mi abuelo, don Bruno y
Juanito, el amigo que yo quiero más, porque lo que dice parece cantado.»

---Juanito Santiuste es un magnífico cantor de historias. ¡Lástima que
no vaya al Congreso!\ldots{} A veces llora una oyéndole: no se puede
remediar.

---Pues aquella noche habló de ti\ldots{} Dijo que tú eras, no sé
cuándo, la mujer más hermosa que había en el mundo\ldots{}

---¡Jesús, qué disparate!

---Que él no te había visto; pero que lo había oído\ldots{} que eras tan
guapa como la Virgen, y que en un castillo te apareciste\ldots{} sin
zapatos\ldots{} quiere decir, con pies como los de las estatuas, y que
los que te vieron aparecer se cayeron al suelo encandilados de ver tu
hermosura\ldots{}

---¡Jesús! Hijo mío, no hagas caso. Juanito quería burlarse de los que
le escuchaban.

---No, no, que lo decía muy serio, ¡vaya!\ldots{} Él no lo vio; pero se
lo contaron, y en Madrid está quien lo sabe\ldots{} ¿Fue milagro, madre?
Juanito dice que salió en papeles y hasta en un libro\ldots{} No me lo
niegues\ldots{} Explícame tú cómo te apareciste. ¿Venías del Cielo?
¿Bajaste volando? A mí no me niegues nada. Y si te apareciste por arte
del diablo, dímelo, que yo te guardaré el secreto.

---Chiquillo, no sé si enfadarme o reírme---respondió Lucila prefiriendo
la demostración de gozo.---Disparates sin pies ni cabeza es lo que os
contó Juan. Como que Juan es loco. ¿No lo has conocido? Dicen que tiene
mucho talento, y que repite todo lo que habla ese Castelar\ldots{}

---Es verdad, madrita. Loco parece Juan algunas veces. Aquel día, cuando
se puso en medio de la sala, y mirándote a ti, que entrabas de la compra
con mantón y dos cebollas en la mano, te soltó aquellos gritos
de\ldots{} ¿Cómo era, madre?

---«¡Virgen democracia, yo te saludo!» Nos moríamos de risa oyéndole, y
él, con nuestras risas, se dislocaba más.

La entrada de Jerónimo y de Leoncio Ansúrez, que venían de la calle,
desvió la conversación hacia puntos de mayor interés. La guerra
empezaría pronto. Ya se habían dado las órdenes para la movilización de
fuerzas, concentrando batallones en Cádiz, Málaga y Algeciras. El bueno
de Leoncio, aunque domesticado por las dulzuras de la familia, tiraba
siempre al monte de las aventuras guerreras, como genuino celtíbero, y
ya no pensaba más que en ir a campaña. Su habilidad de armero le
aseguraba la incorporación en cualquiera de los Cuerpos de Ejército o en
el Cuartel general. Un famoso general le estimaba por su destreza y
prontitud en la compostura de toda clase de armas de fuego. Seguiría,
pues, la formidable corriente que a todas las actividades españolas
arrastraba hacia la tierra berberisca. Lo único que le entorpecía la
voluntad era el desconsuelo de separarse de su mujer y de su hijo.
Quería que mientras él estuviera en África, Virginia y Lucila viviesen
juntas, acompañándose las mujeres y los niños, con lo que la soledad de
\emph{Mita} sería más llevadera. Desde luego accedió Lucila, y
Halconero, que a la sazón entró, dijo que su mayor gusto era dar
albergue a la mujer de Leoncio, mientras este anduviera en el servicio
de la patria. Todo español estaba obligado a prestar su ayuda al
glorioso ejército. También él \emph{se pondría las botas}, si no
estuviera tan viejo y achacoso. ¡Qué gusto plantarse en África, a la
zaga de la tropa, y allí, si no podía batirse, fregar las cacerolas del
rancho, ayudar a la colocación de tiendas, o dar el pienso a los
caballos!\ldots{} El hombre vibraba de entusiasmo, y no quería que se
hablase más que de guerra y de las indudables hazañas que, antes de
consumadas, ya andaban en lenguas de la gente. La opinión enloquecida
escribía la Historia antes que la engendrara el Tiempo.

Cuando acababan de cenar, entró Juanito Santiuste, habitante en casa
próxima, amigo de Halconero por la amistad de Leoncio. Solía concurrir a
la sobremesa del buen hidalgo campesino, y como por su trato se revelaba
excelente muchacho, ameno, decidor y cantor de ideales generosos,
Halconero y Lucila veían con gusto su compañía, y le celebraban las
gracias oratorias. Conviene decir, ante todo, que Santiuste, después de
mil peripecias en su romántica y azarosa vida, había vuelto a las
primitivas aficiones literarias. La realidad le hizo ver que no le
llamaba Dios por el camino de la herrería mecánica, y que mejor que
armas de fuego, construiría poemas, cuentos y artículos de periódico. El
mismo Leoncio, que le había tomado grande afecto, le empujó hacia el
sendero angosto de las letras, que entonces empalmaba con el ancho
camino de la política. Sucedió además que, cuando menos lo esperaba, le
cayó un destinillo como llovido del Cielo, que le permitía vivir sin
ahogos. Vieron algunos en esto la mano blanca, escondida, de Teresa
Villaescusa. Podía ser: sin duda fue ella la deidad bienhechora; mas no
dio la cara, y aparecía como protector el Marqués de Beramendi. Ello es
que a Juanito le vino Dios a ver: se proveyó de ropa decente; pudo
acomodarse en una casa de huéspedes de mediano trato; erigió sobre su
cabeza el sombrero de copa, prenda indispensable del empleado y
literato; frecuentó círculos donde jamás había puesto los pies, y, en
fin, tomó airecillos de importante personalidad. Bien merecía el pobre
salir de la tenebrosa obscuridad y miseria en que había vivido, y
espaciar en ambiente de cultura su corazón hermoso y su despejada
inteligencia. Era colaborador gratuito de más de un periódico, y en uno
solo cobraba por sus trabajos míseras cantidades, que a él le parecían
los tesoros de Creso; tan hecho estaba el hombre a la pobreza
degradante.

Apenas le vio entrar Halconero, le pidió noticias. Él, como periodista,
solía llevarlas frescas, y cuando no las tenía, las inventaba, llegando
a creer en conciencia que eran la verdad pura: «Ya tenemos plan de
campaña. Dividido el ejército de África en tres cuerpos, ya están
designados los generales que han de mandarlos. Estos son Echagüe, Ros de
Olano y Zabala. Pero hay más, hay más: se dice que irá también Prim.»

---¡Prim\ldots{} Prim!---repitieron con más curiosidad que asombro las
bocas de Halconero, de Ansúrez y de don Bruno Carrasco, que a la
sobremesa llegó minutos antes que Santiuste.

---Prim ha venido del extranjero a escape y le ha dicho a don Leopoldo:
«¿Pero qué es esto? ¿Yo, Prim, no mando tropas en África?» Y dice
O'Donnell: «Habéis llegado tarde, General. Los jefes de los tres cuerpos
de Ejército están ya nombrados.» Y Prim: «Bien. Pero si no hay cuerpo de
Ejército, habrá una brigada, un regimiento, un batallón, una compañía
que yo pueda mandar.» A esta manera de pedir no podía responder
O'Donnell más que creando una División de reserva para que al frente de
ella luzca el de Reus su bizarría\ldots{}

---¡Prim!\ldots{} ¡oh!---repitieron las bocas de todos, expresando con
dos monosílabos la admiración dubitativa del héroe inédito, cuya leyenda
estaba a medio formar.

Luego tomó Santiuste la flauta, y dijo: «¡Qué hermoso espectáculo el de
un pueblo que antes de ver realizadas las hazañas ya las da por hechas!
Lo que la Historia no ha escrito aún, lo ve la Fe con sus ojos vendados.
Creer ciegamente en el fin glorioso de la campaña, equivale a la
realidad de ese fin. Ved cómo las madres pobres de las aldeas no se
afligen de ver partir a sus hijos para el África. Oíd a los viejos, que,
como Horario, pronuncian el terrible \emph{¡que mueran!}\ldots{} si
muertos sellan con su sangre el honor de España. Ved cómo la Nación
entrega cuanto posee, para que nada falte al soldado. Aquí dan dinero,
allá provisiones, acullá las damas destejen con sus finos dedos las
telas\ldots{} quiero decir que sacan hilas para curar a los heridos.
Quién da caballos, quién mulas\ldots{} Los pueblos ricos dan zapatos;
los pobres, alpargatas. Los obispos empeñan la mitra, y los catedráticos
sacrifican parte de sus míseras pagas\ldots{} ¡Espectáculo admirable,
sublime, que nos consuela de las vulgaridades y miserias de la
política!»

El sagaz Ansúrez agregó a los toques de flauta estas prosaicas
observaciones: «Aún no sabemos lo que será O'Donnell como General en
Jefe del ejército de África: es de creer que sepa conducirlo y
acaudillarlo con la mayor ventaja nuestra y daño grande del enemigo.
Esto lo veremos. Lo que no tiene duda es que el buen señor se acredita
con esta guerra de político muy ladino, de los de vista larga, pues
levantando al país para la guerra y encendiendo el patriotismo, consigue
que todos los españoles, sin faltar uno, piensen una misma cosa, y
sientan lo mismo, como si un solo corazón existiera para tantos pechos,
y con una sola idea se alumbraran todos los caletres. ¿Les parece a
ustedes poco? Esto es lo más grande que se ha hecho en España desde que
yo nací, y me alegro, pues en mi larga vida no he visto más que
trifulcas entre españoles, guerra de sangre, de discursos, motines, y
persecuciones de estos contra los otros\ldots»

---El \emph{Progreso}---afirmó don Bruno Carrasco poniendo en la
declaración toda su seriedad de paquidermo,---ha plegado su bandera
política y ha enfundado sus agravios ante la declaración de guerra,
hecho que a todos los partidos impone un silencio patriótico y una
expectación patriótica\ldots{}

Puesta a un lado la flauta, cogió Santiuste el cornetín, y tocó estas
cláusulas vibrantes: «El ideal de la patria se sobrepone a todos los
ideales cuando el honor de la Nación está en peligro. Puede la Nación
vivir sin riquezas, sin paz, y aun privada de los bienes del progreso
puede vivir; pero sin honor nunca vivirá. O lava con sangre los ultrajes
hechos a su nombre y representación, o arrastrará una existencia de
vilipendio, despreciada de todo el mundo.»

Así siguió un rato; pero como no hiciera su música el efecto que
buscaba, soltó el cornetín, cogió la trompa, y soplando en ella con toda
su fuerza, produjo estos bélicos sonidos: «¡Qué gloria ver resucitado en
nuestra época el soldado de Castilla, el castellano Cid, verle junto a
nosotros y tocar con nuestra mano la suya, y poder abrazarle y
bendecirle en la realidad, no en libros y papeles! Reviven en la edad
presente las pasadas. Vemos en manos del valiente O'Donnell la cruz de
las Navas, y en las manos de los otros caudillos, la espada de Cortés,
el mandoble de Pizarro y el bastón glorioso del Gran Capitán. Las
sombras augustas del emperador Carlos V y del gran Cisneros, nos hablan
desde los negros muros de Túnez y de Orán. La epopeya, que habíamos
relegado al Romancero, vuelve a nosotros trayendo de la mano la figura
de aquella excelsa y santa Reina que elevó su espíritu más alto que
cuantos soberanos reinaron en esta tierra, la que al clavar la cruz en
los adarves de Granada, no creyó cumplida con tan grande hazaña su
histórica empresa, y con gallardo atrevimiento y ambición religiosa y
política nos señaló el África como remate y complemento del solar
español. Al volar desde este mundo al Cielo, donde la esperaba el premio
de sus virtudes, Isabel ordenó a sus herederos que arrebatasen a la
Media Luna el suelo mauritano, español suelo, y formasen el futuro reino
de España con los extremos de los dos continentes. El bravo mar que
entre ellos corre no los enemista y separa, sino, antes bien los une y
acaricia, besando ambas orillas con alternados ósculos, y cambiando
entre una y otra signos de paz y amor. Del Pirineo al Atlas, todo será
España.»

\hypertarget{iv}{%
\section{IV}\label{iv}}

Vibraban todos los presentes al son de estos roncos trompetazos. Lucila,
sin poder impedir que se le saltaran las lágrimas, decía: «Este Juan es
un loco, que dice tonterías bonitas.» Halconero, deshaciéndose en
entusiasmo que le mantenía rebelde al sueño, mandó traer Jerez para
festejar al trompista y regalarse todos. Cogiéndole un momento aparte,
Lucila dijo a Santiuste: «Hágame el favor, Juanito, de no contar estas
cosas tan rimbombéricas cuando esté mi niño delante. Yo quise acostarle;
pero cualquiera le arranca de aquí cuando viene usted con estas tocatas.
Mírele allí junto a su padre, comiéndosele a usted con los ojos\ldots{}
Se trastorna, se desvela, y luego las malas noches me tocan a mí: no es
usted quien las pasa. Ya tenemos jaqueca para toda la noche con lo que
usted ha dicho del Cid, de Cortés, de Pizarro y del Gran Capitán o del
Gran Teniente\ldots{} Buena la hemos hecho. Acostadito el niño y sin
poder dormir, empiezan las preguntas; y yo, que soy tan ignorante, me
veo negra para responderle. Con que hágame el favor de dejar la trompa
cuando esté aquí mi hijo; coja el flautín o la zambomba, y cuéntenos
algo que nos entretenga y nos haga reír.»

El buen Jerez prodigado por Halconero avivó los fuegos patrióticos de la
tertulia, cuidando el amo de la casa de ser el primero en las alegres
expansiones. Alborotadamente trataron de diversos puntos relacionados
con la guerra, y Carrasco y Santiuste afirmaron que Moros y Cristianos
son en alma y cuerpo diferentes, como el día y la noche. Ansúrez, cuya
natural capacidad ilustraba todas las cuestiones, sostuvo que las
apariencias de desemejanza las daba, más que la religión y el lenguaje,
el hecho de no existir en la Morería lo que aquí llamamos modas. El moro
no sabe lo que es esto. Sus armas, sus vestidos, sus hábitos, sus
alimentos, se perpetúan al través de los siglos, y lo mismo se eternizan
sus modos de sentir y de pensar. Aquí, por el contrario, tenemos la
continua mudanza en todo: modas en el vestir, modas de política, modas
de religión, modas de filosofía, modas de poesía. Ideas y artes sufren
los efectos del delirio de variedad\ldots{} Hoy se llevan estas
corbatas; mañana serán otras. Hoy se gobierna por este sistema; mañana
será por el contrario. Filósofos y sombrereros, poetas y peinadoras,
tienen su figurín distinto para cada quince años. Al otro lado del
Estrecho les dura un figurín, para todo, la friolera de diez o doce
siglos\ldots{} Y así, hemos dado en creer que esta permanencia es señal
de poca o ninguna civilización, lo cual no es justo, pues ni ellos son
bárbaros por no conocer las modas, ni nosotros civilizados por tenerlas
y seguirlas tan locamente. La civilización consiste en ser buenos,
humanos y tolerantes, en hacer buenas leyes y en cumplirlas\ldots{}

No expresó el agudo celtíbero estas ideas en la forma que aquí se les
da, sino con la frase seca, desnuda y categórica que usar solía. Las
presentes páginas sólo transmiten textualmente el final, que fue de este
modo: «Entre las cosas santas y buenas que nos recomendó Jesucristo al
fundar nuestra doctrina, yo no he podido encontrar nada que sea
recomendación de las modas. Dijo: «amaos los unos a los otros;» pero no
dijo: «sed veletas en el pensar y en el vestir, en el comer y en el
edificar.» Y aunque nada dijo de estas veleidades de los hombres,
entiendo que las condenó en el Desierto cuando el Demonio quiso
tentarle. Sabéis que le llevó a un alto, y mostrándole toda la tierra,
se la ofreció en dominio si le adoraba. Para mí que le dijo: «Ahí tienes
el mundo de las modas: adórame y será tuyo.» El Señor, a mi parecer,
contestó: «Vete al infierno tú y tus modas, y no tientes al Señor tu
Dios.»

Sin comprender la sutil argumentación del viejo Ansúrez, los amigos la
tomaron a chacota, y por divertida más que por razonable la
celebraron\ldots{} Y a otra cosa. Aunque Lucila llamaba disparates a las
huecas declamaciones del joven de la trompa, y se burlaba de él por
disimular su devoción de las cosas guerreras, se alegraba de verle
entrar, y no perdía sílaba de sus peroratas, exuberantes de elocuencia y
de histórica poesía. Clavijo, Santiago, los Alfonsos, el Cid, la cruz de
las Navas, la cruz del Cardenal Mendoza, la cruz de Lepanto y otras
famosas cruces; las torres de Granada, las carabelas de Colón, los
tercios de Flandes y demás estrofas sublimes del gran poema, conmovían
todo su ser, y le disparaban el corazón a un palpitar loco; de su pecho
irradiaba un calorcillo que encendía en su rostro matices de embriaguez
dulce. Cierto que procuraba repeler hacia adentro la emoción; pero no
siempre lo conseguía, pues la viveza y humedad de los ojos desmentían
las burlonas palabras.

Una noche, acostando a Vicente, después de curarle la pierna con amoroso
cuidado, el chiquillo le dijo: «Madrita, estoy enfadado contigo\ldots{}
pero muy enfadado\ldots»

---Yo te desenfadaré, si me dices pronto en qué ha podido ofenderte tu
madre.

---¡Zalamera! Estoy enfadado por tres cosas\ldots{} tres perradas me has
hecho\ldots{}

---¿La primera\ldots?

---Que le dices a Juanito que no nos cuente cosas de guerra\ldots{} para
que yo no me despabile\ldots{} Pues bien te gustan a ti las cosas de
guerra. ¿Crees que no te he visto llorando cuando Juan contaba lo que
hizo Hernán Cortés en la Habana\ldots{} o en otro punto de las Américas,
no sé\ldots? El hombre quemó sus navíos para que los hombres que iban
con él no pudieran volverse acá, y luego se metió, espada en mano, por
un río arriba, y conquistó un imperio de negros más grande que de aquí a
la Villa del Prado\ldots{} Luego te pregunto yo: «Madre, ¿quién era ese
Hernán Cortés?» Y tú me respondes: «Un vago, un perdido\ldots»

---Tiempo tienes de saber esas cosas, hijo del alma. Ahora estás
enfermito, y no conviene que te calientes la cabeza, ni que pierdas el
sueño. ¿Y de dónde sacas tú que soy yo guerrera? ¡Vaya una tontería! Yo
no estoy en el mundo más que para cuidar a tu padre, a ti y a tus
hermanitos, y las guerras de hoy, como las de tiempos pasados, me
importan un bledo. Naturalmente, una es española, y cuando tocan el
\emph{chin chin} de las glorias de esta tierra, el corazón baila un
poquito\ldots{} Segunda cosa\ldots{}

---Que tú, por llevarme la contraria, y porque se te ha metido en la
cabeza que yo no sé montar, has escrito al tío Gonzalo\ldots{} o será mi
abuelo el que ha escrito, no sé\ldots{} habéis escrito para que el tío
no me traiga el caballo que me prometió. ¡Y yo aquí con esta pierna
tiesa!\ldots{} Pues te digo que así no me curaré nunca. Ya puedo doblar
la rodilla sin que me duela mucho\ldots{} ¿Ves cómo la doblo? Yo te digo
que no me ha de costar trabajo apretar los muslos para agarrarme bien,
ni meter espuelas con gana para correr\ldots{} ¡hala!\ldots{} correr
como el viento.

---¡Ay, bobito mío\ldots{} pues no estás poco avispado con tu caballo
árabe!\ldots{} Espera, espera un poco. La semana que entra, dice el
médico que podrás andar con muletas\ldots{} Lo que hemos escrito a mi
hermano el moro, es que tenga preparado el caballo, y la silla, y todo,
para cuando se le avise\ldots{} Ahora, la tercera cosa.

---Pues\ldots{} no quería decírtelo\ldots{} pero te lo digo\ldots{} Ya
sabes que una noche contó Juanito que tú te apareciste en un castillo, y
que al verte aparecer, los que allí estaban se cayeron al suelo del
susto y de\ldots{} de\ldots{} de ver lo guapa que eras\ldots{} Eras como
la Virgen, o como otras vírgenes que hubo antes de la del Pilar y la del
Rosario\ldots{} Yo no sé\ldots{} Juanito te comparó con unas vírgenes,
santas o no sé qué\ldots{} Para que se vea si eres mala. ¡Aquellos que
estaban en el castillo te vieron aparecerte, y no quieres que te vea tu
hijo! Si tú te desapareces y vuelves a salir cuando te da la gana, ¿por
qué no lo haces delante de mí para que yo te vea? Todas las noches te
pido este favor, y tú te ríes y me mandas a paseo.

---Y ahora también me río, bobito, porque esas apariciones son cuentos y
desvaríos de Juan. Yo me aparezco\ldots{} cuando entro por esa puerta.
No he aprendido otra manera de hacer mi aparición.

---Bueno, bueno\ldots{} Sigo muy enfadado, madrita\ldots{} No creas que
me desenfado con tus besos, con tus carantoñas\ldots{} Y para que veas
si soy bueno, me voy a dormir\ldots{} No tendrás que chillarme, ni
decirme que te estoy martirizando\ldots{} Me dormiré ahora mismo\ldots{}
ya me estoy durmiendo\ldots{} y no soñaré nada, no quiero\ldots{} Dijo
don Bruno que mañana, mañana\ldots{} pasará mucha tropa\ldots{} mucha
tropa\ldots{} Salen para la guerra\ldots{} de aquí van a la
guerra\ldots{} Va el tío Leoncio\ldots{} esta tarde lo dijo\ldots{} Yo
me asomaré a ver la guerra\ldots{} la tropa que va a la guerra\ldots{}
pum, pum; chan, charanchán\ldots{}

Se durmió como un ángel, a quien Marte arrullara en sus brazos. No fue
tan dichosa Lucila, que padeció inquietud y desvelo hasta muy alta la
noche, mortificada por visiones y pensamientos lastimosos, y por el
desasosiego de su marido, con quien compartía el no muy ancho tálamo.
Daba vueltas sin cesar sobre sí mismo el buen don Vicente, llevándose
tras sí sábanas y mantas, con lo que quedaba desamparada de abrigo la
dama celtíbera. Y sobre tantas molestias, el rico labrador pronunciaba
frases incoherentes, cortadas por estruendosos regüeldos; cantaba el
himno de Riego y la Marcha fusilera, dejando oír entre estas músicas
alguna vaga modulación de alarido patriótico, como ecos lejanos de un
tumulto callejero.

Con paciencia sufría la esposa estas incomodidades, y en la cavidad
verdinegra del insomnio revolvía historias pasadas y presentes. La
mirada de su hijo, dulce y quejumbrosa, con que expresaba su ardimiento
militar cohibido por la cojera, permanecía estampada en la retina de la
madre. Eran los ojos de Vicentín negros como los de ella, luminosos,
bañados en esa tristeza cósmica que envuelve las estrellas, así en las
claras como en las obscuras noches. En los ojos del niño guerrero veía
Lucila algo como la regresión de un ideal que ella tenía por muerto y
desvanecido; ideal que salía de su tumba para volver a la realidad
viviente. También Lucila había sido guerrera, y la gallardía militar,
así en los hechos como en las personas, fue objeto de su culto. Llevose
el diablo estas aficiones; cambió el teatro de la vida de la joven
celtíbera, y desgarrada una decoración, pusieron otra que hizo olvidar
la pasada idolatría\ldots{} Pues ahora, un niño inocente, precoz,
enfermo, imposibilitado hasta de jugar con cosas guerreras, hacía que
por la decoración nueva se transparentasen las líneas y colores de la
antigua\ldots{}

Otra cosa: no eran estas reapariciones de lo pasado el único suplicio de
Lucila en sus horas de insomnio. Debe decirse con claridad que, desde su
casamiento, ningún hombre, fuera de su buen marido, cautivó su corazón.
Pero en mal hora vino el espiritual Santiuste a desmentir la regla
general. No le quería, no hacía ningún cálculo de amor referente a él;
pero posaba con harta frecuencia su pensamiento en la persona del
desgraciado joven, como un ave cansada de volar por los espacios altos
del deber. Por su cuñada Virginia conoció a Santiuste; por Leoncio supo
su miseria y desamparo, y la dignidad con que el muchacho soportaba
tantas desventuras. A menudo se decía: «¿Pero cómo se arreglará ese
hombre para vivir con tanto apuro?\ldots{} ¿Será verdad que le quería
una mujer del mundo llamada Teresa? Y si le quiso y le quiere, ¿cómo le
consiente tan destrozadito de ropa y tan vacío de alimento?»

El cambio de fortuna del cantor de la edad heroica colmó de satisfacción
a Lucila\ldots{} ¡Gracias a Dios que el pobre chico podía vivir, aunque
modestamente! ¡De buena gana le habría ella cosido y arreglado la ropa,
y regalado unas botas decentes para entrar con pie seguro en la nueva
vida! Si le gustaba por pobre desvalido, más le agradó por las bondades
de su corazón, que claramente en toda ocasión se manifestaban, y por la
rectitud inflexible que movía sus acciones. Su inteligencia y saber, su
facundia prodigiosa, descollaban en aquella sociedad vulgarísima como el
águila caudal entre humildes y rastreros patos. Y cuando, por la
declaración de guerra, desenfundó Santiuste la trompa y empezó a soltar
notas de epopeya, si todos le oían con admiración, Lucila se arrebataba
interiormente en un fuego de entusiasmo, que en su seno escondía con
violentos disimulos. El ideal guerrero tan pronto revivía en los ojos
del niño doliente, como en los labios de aquel otro niño grande que
jugaba con el Romancero.

Interrumpió estas cavilaciones de la celtíbera la claridad del día que
por las rendijas de la ventana se colaba, y ante ella puso la señora
término a su mental suplicio, y se lanzó del lecho, dejando al esposo en
postura de tranquilidad, panza arriba, estiradas las extremidades, y
echando de su abierta boca los ronquidos como el resoplar cadencioso de
una máquina de vapor. Vistiose a prisa la hija de Ansúrez, ávida de
lanzarse al trajín casero, que era como el organismo supletorio de su
ser moral\ldots{} Ya no pensaba más que en despertar a la muchacha,
sacándola a tirones de su camastro, y en encender lumbre. Luego
prepararía el desayuno de Jerónimo, que era el primero en dejar las
ociosas lanas; el de los niños, que aún dormían como pajaritos apegados
al calor del nido. Pronto llegaría el panadero\ldots{} Ya se sentían en
la escalera los pasos de plomo del aguador\ldots{} Empezaba el día, la
rutina normal y fácil, el conjunto de menudas obligaciones que, al modo
de tejidos de mimbres, forman el armadijo consistente de una existencia
mediocre, honrada, sin luchas.

\hypertarget{v}{%
\section{V}\label{v}}

Los niños menores, Pilarita y sus hermanillos Bonifacio y Manolo,
contagiados de los gustos del primogénito, despreciaban toda clase de
juguetes para consagrarse al militar juego, aprovechando el material de
guerra desechado por Vicente: cañones, tropa y oficialidad de cartón o
de estaño, banderolas, espadas de palo y morriones de papel. La niña,
desmintiendo su sexo apacible, era la más brava en las marchas, en las
escaramuzas y refriegas, que algún día le valieron solfas de Lucila en
semejante parte. Empezó figurándose cantinera, por algo que había oído a
su hermano mayor: aguardiente vendía en un cacharrito de lata, y
cigarros de papel torcidos por ella misma. Mas pronto se cansó de estos
femeninos menesteres de guerra, y arrollando a sus hermanos pequeños y
arrebatándoles espada y casco, se puso al frente de ellos, y les condujo
más de una vez a la victoria, o a nuevas solfas de la madre, que no
podía resistir tanta batahola y entorpecimientos en las habitaciones y
pasillos de la casa. Con sillas armaban plazas fuertes, bajo la
dirección técnica de Vicente, y en la última torre de ella se colocaba
Pilarita dando voces, atribuyéndose, no sólo entidad militar de plaza
sitiada, sino la divina entidad de Virgen del Pilar, y clamaba: «¡Yo no
quiero ser francesa\ldots{} francesa no\ldots{} Aragoneses,
defendisme\ldots!» Adoptaba Bonifacio para embestir la plaza el ariete
romano, y Manolo imitaba la artillería con los más fuertes zumbidos que
articular podía su gran boca. En el asalto eran tan fieros, que los
muros y bastiones se desplomaban, y entre el deshecho montón de sillas
caía la Pilarica con chichones en la frente\ldots{} Inmediatamente venía
la zurribanda, y con ella los gritos, ayes, lamentos y otras voces
guerreras.

«Por Dios, Vicente, no les azuces a estas diabluras. Ten juicio tú, ya
que ellos no pueden tenerlo. Y a esta mocosa la voy a mandar a la
escuela, para que allí me la sujeten y me le quiten sus mañas
hombrunas\ldots»

Entrado Noviembre, todo Madrid repetía en variedad de formas el juego de
guerra de los niños de Halconero. Los señores mayores, las damas de
viso, hombres y mujeres de las clases inferiores, procedían y hablaban,
poco más o menos, como los chiquillos que esgrimen espadas de caña en
medio de la calle y se agrandan la estatura con morriones de papel.
Guerra clamaban las verduleras; venganza y guerra los obispos. No había
español ni española que no sintiera en su alma el ultraje, y en su
propio rostro la bofetada que a España dio la kabila de Anyera,
profanando unas piedras y destruyendo nuestras garitas en el campo de
Ceuta.

El agravio no era de los que piden reparación de sangre. Fueron los
españoles a la guerra porque necesitaban gallear un poquito ante Europa,
y dar al sentimiento público, en el interior, un alimento sano y
reconstituyente. Demostró el general O'Donnell gran sagacidad política,
inventando aquel ingenioso saneamiento de la psicología española.
Imitador de Napoleón III, buscaba en la gloria militar un medio de
integración de la nacionalidad, un dogmatismo patrio que disciplinara
las almas y las hiciera más dóciles a la acción política. Con las
victorias de Crimea y de Italia fabricó Napoleón patriotismo más o menos
de ley, que hubo de servirle para consolidar su imperio. Francia nos
daba las modas del vestir, las modas del pensar y del sentir artístico:
nos hacía los ferrocarriles; nos ponía, con mano de niñera ilustrada, en
los andadores del progreso; de Francia trajimos también una remesa de
imperialismo casero y modestito, que refrescó nuestro ambiente y limpió
nuestra sangre viciada por las facciones.

Los partidos de oposición, deslumbrados por el espejismo histórico,
cayeron en el artificio. Olózaga y Calvo Asensio cantaron en el Congreso
las mismas odas que en sus púlpitos entonaban los obispos\ldots{} Decía
Calvo Asensio que \emph{el dedo de Dios nos marcaba el camino que
debíamos seguir para aniquilar al agareno.} Estas y otras elocuentes
pamplinas arrebataban al auditorio y encendían más la hoguera
patriótica. Un representante de la nobleza, ofreciendo al Trono el
concurso de sus iguales, decía, \emph{mutatis mutandis}, lo mismo que la
ínfima plebe en tabernas y mercados. Contra el pobre agareno iba el
furor de pobres y ricos, de Clero y Nobleza, de niños pequeños y niños
grandes. La Reina, al despedir a O'Donnell con frases de sincera
emoción, le echaba al cuello medallitas que tenía por milagrosas. Sentía
Isabel no ser hombre para coger un arma y acudir a tan santa guerra; y
era verdad lo que expresaba, pues nadie como ella sintió el intenso amor
de las aventuras españolas, mezcla de fe religiosa, de locura
caballeresca y de gallarda superstición. El efecto de unanimidad y de
embriaguez sintética estaba conseguido. Gran triunfo del irlandés, de
intención honrada y vista penetrante.

En cada mesa de cada café funcionaba un consejo de grandes tácticos y
peritos estrategas. Eran, por lo común, empleados de mediano sueldo,
retirados del ejército, o cesantes que llevaban su abnegación hasta el
punto de alabar al Gobierno, de posponer su hambre a las altas miras de
la patria y a la gloria del ejército. Allí se vio la grande generosidad
de este pueblo, que olvidaba sus miserias, resignándose a comer
entusiasmo y glorias, mal aderezadas con pan seco. Las madres ofrecían
todos sus hijos, y los viejos querían alargar su vida para presenciar
tantas victorias; los curas tocaban el clarín, y salpicaban de agua
bendita los roses de los soldados, incitándoles a no volver sin dejar
destruido el islamismo, arrasadas las mezquitas, y clavada la cruz en
todos los alcázares agarenos. Gentes había mal nutridas, que lloraban
oyendo hablar del próximo embarque de tropas, y darían su última pitanza
por que nada faltase a nuestros valientes soldados. Nunca habían visto
los nacidos un movimiento de opinión tan poderoso y unánime\ldots{} De
este sentimiento y convicciones salían tantos planes de guerra como
bocas había en cada círculo de café. «Es indudable que \emph{nosotros}
desembarcaremos en Malabatah, cerca de Tánger\ldots{} Tomamos Tánger, no
sin pérdidas, y en seguida vamos a ocupar el monte de las Monas\ldots»

Esto decía Leovigildo Rodríguez. Le cortaba la palabra Federico Nieto
\emph{(alias don Frenético)}, diciendo con airadas voces: «Cállese usted
y no extravíe la opinión. Tánger no puede ser el \emph{objetivo}\ldots{}
Mi primo Joaquín, que ha estado en Ceuta y conoce aquello palmo a palmo,
me ha dicho que todo lo que no sea tomar tierra en aquella plaza y subir
derechitos a lo que llaman Sierra Bullones, es andarse por las
ramas\ldots»

---¡Oh, eso no puede ser!---aseguró Agustín Fajardo, pasando su dedo por
la mesa como por un plano imaginario.---Fijarse bien, señores. Aquí está
Tánger\ldots{} aquí está Ceuta\ldots{} aquí Tetuán\ldots{} Unamos por
tres líneas estos tres puntos. Resulta un triángulo de lados
desiguales\ldots{} ¿El lado más corto cuál es? El que une a Tetuán con
Ceuta\ldots{} Pues mi teoría es esta: Otras naciones irán a su
\emph{objetivo} por el camino más largo. España debe ir siempre por el
más corto. Si no lo hiciera, no sería España\ldots{} Esta es mi teoría,
señores; es mi teoría.

Con estos desatinos fantásticos iba la gente alimentando la pasión
patriótica, que a todos sostenía en un cierto estado de iluminismo
alegre. Nadie dudaba del triunfo: el esplendor de nuestras armas traería
después bienes sin cuento, que cada cual se imaginaba conforme a sus
gustos y necesidades. El buen Halconero, que en patriótico fanatismo
daba quince y raya a todos los españoles, pensaba que después de la
guerra los laureles nos abrumarían. Probablemente, tras la campaña en
África vendrían otras marimorenas con diferentes naciones europeas o
asiáticas, y de este continuo pelear resultaría mucha, muchísima gloria
y poco dinero, porque los brazos abandonaban la cosecha del trigo por la
de laureles. ¿Pero qué importaba? Con tal de ver a España tosiendo
fuerte, escupiendo por el colmillo en el ruedo de las naciones europeas,
nos allanaríamos a sustentarnos con piruétanos y tagarninas.

Obligado el insigne paquidermo don Bruno Carrasco a tocar su pito en la
orquesta patriótica conforme a la tregua concedida por el
\emph{Progreso}, no podía saciarse de política, su comidilla sabrosa y
constante. Los temas desde la subida de O'Donnell hasta el Otoño del 59
habían pasado a la Historia. Ya Carrasco no podía poner en su púlpito
más que el paño de gala para cantar himnos al Ejército y al \emph{Dios
de las Batallas}. Era ya fiambre manido el asunto de los \emph{Cargos de
piedra}, y la acusación y proceso contra Esteban Collantes, farsa de
justicia que encubría el propósito de inutilizar a los moderados por la
difamación. No era culpable el ex-ministro de Fomento en el Gabinete
Sartorius: la culpa venía de arriba y de peticiones de dinero que el
Gobierno no podía desatender. Fue verdad que el valor de los ciento
treinta mil cargos de piedra se aplicó a objeto distinto de la
reparación de carreteras; cierto que la cantidad fue sustituida por otra
igual dada por Salamanca; indudable que don Agustín Esteban Collantes,
días antes de la caída de San Luis, ordenó que el milloncejo se
reintegrase a su primitivo destino; verdad fue que en el camino hacia la
casilla del presupuesto, se perdieron los cuartos, y que la
responsabilidad de tal extravío recaía exclusivamente sobre el Director
General de Obras Públicas, y que este trasladó a Londres su residencia.
Ruidoso escándalo trajo la grave acusación, una de las mayores torpezas
de la Unión Liberal, porque en el proceso salieron a relucir infinidad
de suciedades de nuestra administración, y nadie a la postre fue
castigado. El ex-ministro se defendió con maestría y sutileza grandes.
Inmensa labor fue, para el que se sentía inocente, demostrarlo sin
dirigir un solo golpe al punto delicado de donde procedía la infracción
de ley\ldots{}

Pues sobre este embrollo y sobre los incidentes del dramático proceso,
habló don Bruno tres meses, sin descanso de su lengua ni agotamiento de
su saliva. Él lo sabía todo: la inocencia de Collantes, la dudosa
conducta de Mora, el origen palatino de aquella irregularidad. Las
relaciones entre los partidos de gobierno quedaron rotas y envenenado el
ambiente político. Si no inventa O'Donnell la guerra de África, sabe
Dios lo que habría pasado. Fue la guerra un colosal sahumerio\ldots{}
Casi tanto como los \emph{Cargos de piedra}, sacó de quicio a don Bruno
la intentona republicana que estalló y fue sofocada en el curso del
estío. En aquella locura pereció el más loco de nuestros demócratas,
Sixto Cámara, joven, apuesto, de rostro interesante y algo místico.
Trató de sublevar a la guarnición de Olivenza: no pudo conseguirlo;
huyó, y perseguido por la Guardia Civil en los campos extremeños, murió
de calor y de sed. Místico fue el martirio de aquel visionario que
padeció la generosa demencia de querer implantar la República con tres
republicanos.

En los claros que dejaban estos asuntos de real importancia, subía don
Bruno a su púlpito para condenar los resellamientos y pasar revista a
los nuevos periódicos, \emph{La Discusión}, inspirado por Rivero;
\emph{El Estado}, dirigido por el poeta Campoamor; \emph{El Horizonte},
hechura de don Luis González Bravo, papel impulsivo y un tanto burlesco
con remembranzas de \emph{El Guirigay}\ldots{} De El Contemporáneo, el
periódico elegante, órgano de la fracción más europeizada del
moderantismo, hablaba pestes el buen don Bruno; odiaba con toda su alma
a los caballeros \emph{del guante blanco}, que derramaban sus luces en
aquel diario, dándole la nota de la distinción y del saborete inglés, a
los que llamaban \emph{Sincretismo} a la Unión Liberal, y a cada momento
empleaban términos tan estrambóticos como el \emph{Self-government} y el
\emph{Habeas Corpus}\ldots{} ¡Qué tendría que ver con la política el
\emph{Santísimo Corpus Christi}!

Una mañana de Noviembre, hallándose don Bruno y Halconero en casa de
este charlando de la movilización de tropas, entró jadeante Juanito
Santiuste con la noticia de que él, también él, ¡feliz mortal!,
iría\ldots{} «¿A dónde, hijo mío?» ¡A la guerra! Por el Marqués de
Beramendi, su amigo, había conseguido una plaza en la \emph{Sección
Volante de la Imprenta de Campaña}. Ya tenía preparado su equipaje, que
era de los más exiguos, y aquella misma tarde\ldots{} ¡Cielo santo,
Juanico a la guerra! ¡Y él también sería héroe, y a más de ser héroe,
tendría la gloria de ver tantas grandezas\ldots! Y andando el tiempo,
dentro de un siglo, sus inocentes biznietos dirían: «Mi abuelo estuvo en
la más alta acción, \emph{etcétera}\ldots» Fuese porque aquel día
estuviera don Vicente amagado de un nuevo ataque de su mal, fuese porque
la noticia de la partida del trovador colmara su exaltación, ello es que
el hombre rompió en llanto. Su trabada lengua decía: «Tú vas, Juan, y yo
no\ldots{} Yo inútil, yo\ldots{} trasto viejo\ldots{} tú gloria, yo
estropajo\ldots{} Abrázame\ldots{} te quiero\ldots{} ¡Viva España\ldots!
Hijos míos\ldots{} Lucila, venid\ldots{} ¡Que me traigan a
Donnell\ldots{} que me traigan a Prim!» Dichos estos y otros desatinos,
salió disparado por el pasillo, los brazos en alto, el andar tan
inseguro que daba encontronazos en los tabiques, rebotando de uno en
otro. Seguíanle todos asustados de aquel delirio. Al volver a la sala,
su rostro amoratado indicaba fuerte congestión; su voz, ya ronca y casi
ininteligible, repetía: «¡Prim\ldots{} ejército\ldots{} march\ldots!»
Para mayor duelo, los chicos menores, que aquel día tuvieron la humorada
de disfrazarse de moros, se habían ennegrecido la cara con tizne de la
cocina, y haciendo pucheros marchaban detrás de su padre, dando al
cuadro, con la mayor inocencia, un tono de trágica burla. Halconero,
girando sobre la pierna derecha que de improviso se le quedó como si
fuera de palo, cayó al suelo sin que Lucila ni los demás pudieran
contener la caída. Pesaba mucho: la palabra escapaba mugiendo de su boca
torcida, como escapan los habitantes de una casa que se desploma. Con
gran dificultad, entre Lucila, don Bruno y Santiuste, levantaron en vilo
el pesado cuerpo, y lo tendieron en la cama.

\hypertarget{vi}{%
\section{VI}\label{vi}}

El médico, llamado a toda prisa, no recetó más que la Extremaunción.
Acudieron a la casa Virginia y Leoncio; pero este, como Santiuste, no
tardó en salir, pues ambos debían prepararse para partir aquella misma
tarde. El niño cojo, que arrimado al balcón había presenciado el
accidente y caída de su padre, recibió tan fuerte impresión, que en
largo rato no pudo moverse ni pronunciar palabra. Los pequeños, que a la
cocina huyeron aterrorizados, mojaron con sus lágrimas el tizne, y
diluido este en las caras como pintura de acuarela, se convirtieron en
mulatos. En su aflicción y espanto encontró Lucila una ligera pausa para
salir a consolar a Vicente, que junto al balcón permanecía. «Tu padre
está malito\ldots{} pero no te asustes\ldots{} Ha sido un ahogo. Dios
querrá que se le pase pronto\ldots{} Me parece, hijo mío, que tú quieres
llorar y no puedes. Llora un poquito, sí; aunque\ldots{} ya te
digo\ldots{} tu padre está mejor\ldots{} Ya he mandado a la Nicasia que
te ponga tu silla en el comedor\ldots{} Me vuelvo al lado de tu padre;
pero ya saldré un ratito\ldots{} te haré compañía y te contaré
cosas\ldots{} Tus hermanos, que hoy están muy mañosos y pintados de
negro, se meterán contigo en el comedor. Tú cuidarás de que guarden
silencio\ldots{} Entretenles enseñándoles las vistas de batallas\ldots{}
Adiós, Vicente: llora un poquito\ldots{} no te importe llorar\ldots»
Volvió la madre a su obligación. Durante la breve ausencia, el enfermo
había recobrado el sentido, aunque sólo de una manera borrosa,
crepuscular, pronunciando palabras confusas. Don Bruno Carrasco a gritos
le interrogaba, creyendo que de este modo sería mejor entendido. Conoció
don Vicente a su mujer, y haciendo por cogerle una mano, intento que no
pudo realizar, le dijo: «Luci\ldots{} di\ldots{} dile a Prim que\ldots{}
que pase\ldots{} a Donnell que\ldots{} pase\ldots{} a Chagüe\ldots{}
pase\ldots» A esto siguieron mugidos, como una recriminación a su propio
cuerpo por aquella mala partida de no querer moverse\ldots{} Sólo el
brazo derecho tenía un resto de vida, estirándose y encogiéndose como el
alón de un ave moribunda.

Entró de la calle Jerónimo Ansúrez, que, ignorante del grave suceso,
tuvo más palabras para el estupor que para el remedio, y con penetración
clínica de hombre tan ducho en vidas como en muertes, juzgó desesperado
el caso. Ayudó a su hija en la aplicación de sinapismos, y viendo que a
las quemaduras de la mostaza no respondía ni con vibraciones de dolor
aquel madero que había sido cuerpo humano, propuso que, conforme al
dicho del médico, se mandara al diablo la Medicina y se llamase a la
Religión. Él mismo llevó el aviso a la parroquia, y a eso de la una
dieron la Extremaunción a don Vicente, pues para otros auxilios del alma
no tenía el enfermo la necesaria lucidez. No obstante, cuando sonaron en
la sala los pasos del sacerdote, la consternada Lucila creyó descubrir
en el moribundo una chispa de conocimiento\ldots{} Cariñosa atención
puso en aquellos mugidos, y hasta llegó a traducirlos libremente de este
modo: «Luci\ldots{} di\ldots{} dile a Dios que\ldots{} pase.»

Las tres serían cuando entregó a Dios su alma el bueno, el honrado, el
sencillo labrador don Vicente Halconero, que jamás hizo mal a nadie, y a
muchos bien sin tasa; varón de grande utilidad en la República, o por
mejor decir, en el Reino, porque no devoraba porción ninguna del Tesoro
Nacional, sino que creaba, con su labor de la tierra, nueva riqueza cada
año. No aumentaba la confusión de opiniones, sino que tendía con su
patriótica fe a simplificar las ideas, y a buscar la síntesis que
pudiera traer a nuestro país positivas grandezas. Su trabajo agrícola
era un beneficio para España, y otro su inocencia, virtud preciada
contra la invasión de maliciosos. Fecundaba la tierra, fecundaba el
ambiente.

Soltó Lucila las exclamaciones de su duelo con afluencia que del corazón
y del alma le salía. Era un poema de gratitud, tributo al hombre que la
sacó de la soledad triste, ignominiosa, y que, al dignificar su persona,
le dio paz, bienestar, honor, y cuanto pudiera ambicionar la mujer menos
humilde. Había sido Halconero el maravilloso príncipe de los cuentos
orientales, que ofrecen su mano y su reino a la niña despreciada,
víctima de las brutalidades de un genio maléfico. El buen caballero
labrador, que tenía por blasón su arado y podadera, y por leyenda el
\emph{Super omnia rura}, la hizo reina de su casa, de sus abundantes
cosechas, de sus ganados, que poblaban praderas y montes. En este trono,
al que subió la celtíbera como por milagro, quedaron borradas las
sombras de un pasado triste, y hasta los amargos dejos de sus desdichas
se extinguieron en tantas dulzuras. Luego vino su coronación de reina,
los hijos, las sagradas prendas de aquella unión bendita. Con los frutos
de ella, la casa labradora se perpetuaba y prometía mayores bienandanzas
en edades futuras\ldots{}

Por las notas agudas del llanto de Lucila, que hasta el comedor
llegaban, comprendió Vicentito que su padre no existía ya. Era un niño
de conocimiento y alcances superiores a su edad. Su misma dolencia, que
a forzosa quietud le sometía, daba mayor lucidez a su mente para las
cosas graves. La falta de ejercicio corporal, entorpeciendo la acción
del niño, permitía un precoz desarrollo de las facultades del
hombre\ldots{} Como se ha dicho, los ecos de la voz plañidera de su
madre, difundidos por la casa muda, dieron al chiquillo la idea y
sensación del gran infortunio de la familia: sintió el \emph{vacío de
padre}, la repentina ausencia de una suprema autoridad y
custodia\ldots{} Viéndole llorar, también lloraron sus hermanitos. Pero
él les dijo: «No lloremos todos a un tiempo, que haremos demasiado
ruido\ldots{} Si la madrita nos oye llorar, se pondrá más triste\ldots{}
No es más sino que el padre está malo\ldots{} pero ahora viene el médico
y se pondrá bueno.» Con estas y otras exhortaciones les hizo callar, y
él, sin limpiarse las lágrimas, dio algunas vueltas, con sus muletas, en
torno a la mesa del comedor, aún sin manteles ni preparativo alguno de
comida, aunque había pasado la hora. Después se sentó, estirando su
pierna sobre otra silla, y permaneció pensativo un buen rato, mientras
Pilarita y los pequeños, sentados en ruedo casi debajo de la mesa,
repasaban las vistas de batallas, agregándoles innumerables detalles, ya
con trazos de lápiz gordo, ya con la impresión de sus manos
puercas\ldots{} Entró en esto Nicasia llorosa. Vicente no le dijo nada,
ni necesitó que ella le contase lo ocurrido. Venía, por orden de la
señora, no más que a darles de comer, y a recomendarles que no hiciesen
ruido, y que fuesen aquel día los niños más buenos del mundo. Puesto un
mantel en media mesa, en un santiamén les dio de comer la moza,
sirviéndoles sopa fría, carne y garbanzos del cocido a medio hacer,
tortilla improvisada, como remedión, higos y nueces de postre. Vicentito
fue excesivamente parco con el comer. Entró Jerónimo Ansúrez con rostro
grave cuando aún no habían concluido, y a todos les acarició
diciéndoles: «¡Qué guapos son estos niños, y qué bien se portan hoy! Les
voy a traer almendras confitadas y unos candeleritos con velas de
colores, con su Virgen de la Paloma y todo. Luego vendrá Virginia con su
nene, y jugaréis a los altaricos.» Se fue a tratar del féretro y demás,
en una tienda de la Concepción Jerónima. Vicente se puso a repasar un
librillo de estampas de animales, y aún estaba en las primeras hojas,
cuando vio entrar a Juan Santiuste, de puntillas, la consternación
pintada en su rostro estatuario, que si era comúnmente fiel intérprete
de la alegría, mejor expresaba el dolor. Llegose derecho al cojito y le
estrechó las manos\ldots{} Se sentó a su lado\ldots{} No habló del padre
muerto, ni había para qué. Había venido Juan a ver cómo seguía don
Vicente. Los porteros confirmaron lo que él temía. Subió desolado.
Nicasia, enterándole en breves palabras de la muerte, le dijo: «Pase,
don Juan, al comedor: allí están los niños.»

No acertó el chico a decirle palabra. Dejándose acariciar de él, le
miraba con arrobamiento. Juan le pasó la mano por los cabellos negros,
sedosos, atusándoselos con gracia\ldots{} «Vicente---le dijo,---te
quiero tanto, que no siento irme a la guerra más que por no poder estar
contigo y verte todos los días.»

---¡Te vas a la guerra, Juan\ldots! Verás: antes quería yo que fueses a
la guerra, y hoy me da pena de que te vayas\ldots{} ¡Tanto tiempo sin
verte; tanto tiempo solo!\ldots{} ¿Y si cuando vengas me encuentras más
cojo que ahora? No: yo no quiero estar cojo.

Oyéndole sintió Santiuste un arrebato de amor tan grande por aquel niño
enfermo, prodigio de graciosa inteligencia, que no pudo reprimirse, y
cogiéndole la cabeza le besó con ardor en los cabellos, en la frente, en
las mejillas, y no paró en sus demostraciones hasta que el chiquillo
protestó con cariñosa queja: «¡Juan, que me ahogas!» Santiuste oprimía
contra su pecho la cabeza del niño, diciéndole: «No sabes cuánto te
quiero, hijo mío\ldots{} No te lo había dicho nunca\ldots{} Ahora te lo
digo, porque sí; porque quiero que lo sepas\ldots{} Eres muy bueno,
Vicente, y por bueno te quiero yo\ldots»

---Pues si me quieres---replicó el chico,---escríbeme de allí todo lo
que vaya pasando en la guerra, para que yo me entere. Escribes y le
mandas las cartas a mi madre, y ella y yo las leeremos juntos, y nos
acordaremos de ti. Mi madre también te quiere: se lo he conocido; te
quiere como si fueras mi hermano, y me parece que no le hace mucha
gracia que te vayas a la guerra. Podría cogerte una bala, y matarte o
dejarte derrengado\ldots{} o con la cara rota, sin tu guapeza natural.

---Ya cuidaré yo de que no me cojan balas; y en lo de escribiros cartas
a tu madre y a ti, estad tranquilos. Todo, todito lo que vaya pasando,
batallas, victorias, lo sabréis ella y tú tan pronto como el
Gobierno\ldots{} Déjame que te bese otra vez, criatura\ldots{} La idea
de que estaré tanto tiempo sin verte me vuelve loco\ldots{}

En el nuevo arrebato de su cariño ardiente, no pudo Santiuste contener
sus lágrimas; y viéndole llorar, Vicente también lloró. «Hoy estoy
triste, Juan---le dijo.---La verdad, no debieras marcharte\ldots{} voy a
quedarme muy solo\ldots{} Si no tienes prisa y esperas a que salga mi
madre, verás cómo ella te dice también que no te vayas\ldots» Acongojado
y con un nudo en la garganta, Santiuste no sabía qué decir\ldots{} «No,
no estaré hasta que tu madre venga---murmuró al fin, mirando con pavor a
la puerta:---tengo mucha prisa\ldots» La presencia de Lucila le infundía
miedo en aquella fúnebre ocasión. Verla y oírla era ordinariamente su
encanto; mas aquel día la imagen y la voz de la celtíbera debían ser
guardadas en arqueta de oro, de donde se sacarían a su debido
tiempo\ldots{} Tal era su temor de verla, que con súbito movimiento
cogió el sombrero para marcharse. Quiso detenerle Vicentillo. «¿Quieres
que llame a Nicasia para que le diga a madrita que estás aquí?»

---No, no, no---replicó Juan con mayor espanto;---madrita no puede venir
ahora\ldots{} Yo me voy\ldots{} Déjame darte muchos besos\ldots{} y
también a tus hermanitos\ldots{} Tú, Vicente, no te olvides de mí. ¡Mira
que te quiero mucho, y pensaré en ti a todas horas\ldots! En el corazón
me llevo tu cara, que es la cara de tu madre\ldots{} quiero decir, que
te pareces a ella\ldots{} Adiós\ldots{} Recibiréis cartas, y hoy os
contaré una batalla, mañana otra. No perderé ninguna, para que toda la
guerra quede bien referida. Hoy sale O'Donnell; yo también. Vamos juntos
a Cádiz, y allí nos embarcamos\ldots{} Ya te dije que Cádiz es puerto de
mar\ldots{}

---Tú, que sabes tanto, le dirás a O'Donnell lo que tiene que
hacer\ldots{} Y tú llevarás tu fusil\ldots{} Pongo que te encuentras por
delante a un moro: te matará si no le matas a él\ldots{}

---Naturalmente, allí me darán armas\ldots{} Y yo te aseguro que si
algún morazo se me pone a tiro, lo mando al otro mundo con un recadito
para Mahoma.

---Dice mi abuelo Jerónimo que los moros tienen su cielo separado del
nuestro, donde está Majoma con muchísimas mujeres, bailando y
divirtiéndose. ¿Será verdad eso, Juan?

«Debe de ser verdad\ldots{} Cuando yo vuelva te daré noticias de la
tierra y del cielo moro\ldots{} Adiós, niño mío; no puedo detenerme
más.» El temor de que Lucila entrase, singular ejemplo de delicadeza
llevada a un extremo increíble, le hacía temblar. Besó de nuevo al
chiquillo con ardiente ternura, repartió besos entre los demás, y salió
con pisar blando.

\hypertarget{vii}{%
\section{VII}\label{vii}}

Pero al bajar vio que subían el ataúd, y como era tan angosta la
escalera, hubo de volver hacia arriba y meterse en la casa, única manera
de dar paso al fúnebre cajón. En aquel instante, gran estrépito militar
venía de la calle, por la cual marchaba un batallón con música, y
bullicio y vítores de la gente. Favorecido de aquel estruendo, pudo
Santiuste escabullirse hacia el interior de la casa mortuoria, y volvió
a meterse en el comedor, después de cerciorarse por Nicasia de que los
chicos continuaban solos en aquella pieza. Fascinado Vicentito por la
bullanga marcial que atronaba la calle, creyó que su amigo Juan volvía
para echar con él otro parrafito de cosas de la guerra.

«¿Qué tropa es esa, Juan?»

---Cazadores de \emph{Ciudad-Rodrigo}, que van a la estación.

\emph{---Ciudad-Rodrigo}, número 9\ldots{} ¡Y no puedo asomarme!

---No, hijo mío; no te muevas de aquí. Verás a los cazadores de
\emph{Ciudad-Rodrigo} cuando vuelvan de África vencedores\ldots{} Estoy
aquí otra vez porque no he podido pasar\ldots{} Y me alegro de volver,
porque se me olvidó decirte que\ldots{} Vicente, dirás a tu madre que
siento mucho no despedirme de ella; que\ldots{}

---Que nos escribirás, que nos quieres\ldots{}

---Que siento no despedirme, Vicente: no le digas más que eso\ldots{}
por ahora. Y cuando llegue mi primera carta, le dirás\ldots{}
eso\ldots{} que os quiero mucho, que os llevo en el alma\ldots{} No, no
digas nada de esto\ldots{} Adiós, hijo mío\ldots{} Si me detengo más, me
quedo en tierra. Adiós. Otro beso, otro\ldots{}

Salió como un cohete, y no hallando obstáculo en la escalera, pronto
pisó la calle, donde no era fácil el tránsito por la muchedumbre que al
batallón aclamaba y en su marcha le seguía. Ventanas y balcones
rebosaban de gente: lo que esta no podía expresar con la boca, lo
expresaba con los pañuelos desplegados al viento. Subió Santiuste en
cuatro brincos a su casa, cerró la maletilla en que metido había todo su
ajuar, envolvió en un papel algunos objetos que en la maleta no cabían,
y acompañado de un chico de la patrona que se brindó \emph{por
patriotismo} a llevarle el equipaje, se metió por la Plaza Mayor, para
coger la calle de Atocha, que a la estación del mismo nombre debía
conducirle. Apretando el paso llegaron el viajero y su ayudante de carga
al crucero de Atocha, donde era tan grande el tropel de gente, que no
había medio de romperlo para pasar al embarcadero del ferrocarril. La
multitud no cabía en el suelo, y se extendía por alto: los chicos,
encaramados en la fuente de la Alcachofa y en los árboles; las mujeres
del pueblo, subidas al cerrillo de San Blas y al techo de la ermita.
Coches de lujo, con señoras y caballeros de la mejor sociedad, trataban
de navegar en la masa humana, que se movía como el mar, con oleaje de
estrujones y espuma de gritos. Era felizmente un mar alegre. Nadie se
quejaba de las apreturas: la molestia y el vaivén penoso eran motivo de
risa, de graciosos dicharachos. Poco terreno habían ganado Santiuste y
la compañía, abriéndose hueco a fuerza de vigorosos codazos, cuando
vieron un coche abierto en que venía O'Donnell con Posada Herrera y
Armero. Apenas se dibujó sobre las olas la figura del General, los vivas
a España, a O'Donnell y al Ejército formaron un ruido de huracán. Miles
de manos se agitaban por encima de las cabezas. Navegaba el coche con
suma dificultad, y el cochero entraba en familiar conferencia con la
multitud. «Pero dejen pasar\ldots{} No puedo ir por otro lado\ldots{}
Hagan el favor\ldots{} despejen.» Y una mujer del pueblo: «Atrás todo el
mundo. Pase, Leopoldo\ldots»

Con esfuerzo de brazos y suprema inspiración, Santiuste y su compañero
levantaron en alto, el uno la maletilla, el otro su envoltorio de
papeles, gritando: «Señores, que yo también voy a la guerra\ldots{}
déjenme paso\ldots» «¿Y a qué vais vosotros allá, lambiones?» Las burlas
y chirigotas que oyeron no les acobardaron: entre risas y algún trastazo
llegaron a poner la mano en la capota del coche del General, y con tal
arrimo, náugrafos asidos a una lancha, llegaron al puerto de la
estación. El gabancillo de Santiuste no salió de aquel mal paso sin
lastimosos desgarrones, y del envoltorio de papel, chafado y roto, se
escaparon una zapatilla, una pistola y un tintero de bolsillo.

En la plazoleta de la estación, vio Santiuste más coches, y en ellos
damas que lloraban y señores que hacían pucheros. La patriótica ternura
se desbordaba en todas las almas. Allí los vivas eran más cultos, y
nadie pedía orejas de moros, mas no era menor el estruendo. Entre mil
caras, distinguió Juan el interesante rostro de Teresa
Villaescusa\ldots{} También lloraba, pues aunque mala mujer, era una
furibunda patriota. Iría de cantinera si la dejaran.

Santiuste la vio, mas no fue visto de ella. Atendía la guapa mujer a un
señor viejo que en el coche la acompañaba, y que sin duda le decía: «No
es propio de las señoras llorar tanto por cosas de patriotismo, ni dar
vivas. Para dar vivas estamos los hombres, y para llorar, los niños y
las mujeres de pueblo. Las hembras que no son de pueblo, deben
entusiasmarse con dignidad, sin lágrimas ni voces descompuestas\ldots{}
Pon tú cara risueña, que es lo que te corresponde, y yo grito, como vas
a oír: «¡Viva España, viva la Reina!.»

Alelado, primero con la visión de Teresa, después entristecido por otras
añoranzas de mayor intensidad en su espíritu, Santiuste pudo sobreponer
fácilmente a estas flaquezas la grande ilusión de África: este manantial
de felicidad era entonces abundante y puro, y en él encontraba el alma
todos los consuelos que pudiera necesitar\ldots{} Despidiose de su
machacante el expedicionario, y penetró en la estación. Entre el barullo
que allí había, no tardó en encontrar amigos: el Marqués de Beramendi,
que le había proporcionado la dicha de acompañar al ejército en calidad
de cronista; Manolo Tarfe, el mayor entusiasta de O'Donnell, que a todos
embarcaba para la guerra y se quedaba en Madrid; el Capitán Navascués,
que iba en la escolta del General en Jefe; O'Lean, Gallo, Pulpis, y por
fin, Rinaldi, el prodigioso políglota a quien O'Donnell llevaba de
intérprete. Era Aníbal Rinaldi joven de lenguas, más bien niño, nacido
en Damasco, recriado en Granada; hablaba con perfección el árabe, su
idioma natal, y otros doce de añadidura. Con este simpático mozo trabó
amistad Santiuste, días antes de la partida, cautivado por su saber
filológico y por la dulzura y franqueza de su trato. Concertáronse para
ir juntos en uno de los coches destinados a intérpretes, cronistas y
demás \emph{elemento auxiliar}, y colocadas las maletas de uno y otro en
dos extremos del departamento, Santiuste ocupó su sitio. Tan nervioso
estaba, que temía que el tren partiera sin él si se entretenía en
despedidas y salutaciones. Los minutos que faltaban para la salida se le
hacían años en que todos los días fueran Cuaresma. Quería partir,
correr, volar\ldots{} Por fin, un clamoreo de vivas expresó la salida, y
el tren dio los primeros pasos, hiriendo la calzada de hierro con las
suelas del mismo metal.

«Gracias a Dios---dijo Santiuste a Rinaldi, sentado frente a él;---ya
partimos, ya vamos\ldots{} Será un sueño llegar al África; pero ya no lo
es salir de Madrid, y salir con O'Donnell. Si él llega, llegaremos
nosotros.»

---Dormiremos---dijo Aníbal requiriendo las blanduras del rincón junto a
la ventanilla.

---Yo no duermo---replicó Santiuste.---No quiero dormir. Temo soñar que
no he salido, que me he quedado en Madrid. Pasaré la noche mirando los
fantasmas del campo, el suelo de España que corre hacia atrás, como
formas yacentes y líneas acostadas\ldots{}

Bufaba el tren en las cortas pendientes, echando fuego por las
narices\ldots{} A lo largo de las planicies fáciles, se dormía en un
ritmo ternario, imitando el trote del Clavileño.

\hypertarget{segunda-parte}{%
\chapter{SEGUNDA PARTE}\label{segunda-parte}}

\begin{center}
\textbf{África.—De Ceuta al Valle de Tetuán: Noviembre              \\
        y Diciembre de 1859.—Enero de 1860.}                        \\ 
\end{center}

\hypertarget{i-1}{%
\section{I}\label{i-1}}

Seis días tardó de Madrid a Cádiz el Clavileño, que sólo era ferrocarril
hasta Tembleque; lo demás lo anduvo por caminos carreteros. El 14 se
embarcó O'Donnell en el vapor \emph{Vulcano} para hacer un
reconocimiento de la costa africana. En Cádiz esperaban orden de
embarque las tropas del Segundo Cuerpo al mando de Zabala, y allí quedó
también Santiuste, quien, si por una parte se alegró de aquel descanso
junto a sus buenas tías, por otra renegaba de la tardanza en pisar la
tierra berberisca, objeto de sus más ardientes ansias. Por fin, regresó
a Cádiz el General en Jefe, pasó revista a las tropas el 19, santo de S.
M., y a los pocos días partió con el Segundo Cuerpo, desembarcando en
Ceuta casi al mismo tiempo que lo hacía Prim con la división de Reserva,
procedente de San Roque y Algeciras. Dura fue la travesía por causa del
maldito Levante, que en los meses de \emph{erre} suele jugar con las
aguas del Estrecho, alborotándolas furiosamente. El pobre Santiuste, que
era el hombre menos marinero del mundo, pasó fatigas de muerte, tumbado
en la cubierta del vapor, sin más consuelo de aquel terrible sufrimiento
que lanzar maldiciones contra Neptuno y Eolo\ldots{} Llegó a sentirse
como un pellejo vacío que no podría jamás tenerse en pie\ldots{} Por
fin, oyó decir que ya se veía Ceuta. Transcurrido un lapso de tiempo que
a él le pareció de muchas horas, oyó decir que el vapor fondeaba. Los
tremendos balances no amenguaban por esto, y el pobre mareante,
incorporándose con supremo esfuerzo para mirar por encima de la borda,
vio el Hacho, vio la ciudad tendida en el istmo, como un gran telón que
por el cielo arriba se encaramaba, después se hundía en los abismos
profundos\ldots{}

Las maniobras y el barullo del desembarco diéronle algún aliento.
Deseaba ser de los primeros en tomar tierra; pero fue de los últimos.
Con dificultad podía tenerse en pie, y el uniforme que le habían dado
antes de salir de Cádiz le pesaba y estorbaba horriblemente, no
acertando ni a meter los botones en sus ojales respectivos para
conservar la dignidad de la persona y del traje; el ros se le perdió en
las fatigas del mareo: pusiéronle otro, que se le encasquetaba hasta las
orejas. Con tal facha, y viendo que cielo, mar, barco y tierra
continuaban en angustioso sube y baja delante de su vista, obligándole a
cerrar los ojos para reconstruir en su retina las líneas fijas del
Universo, fue llevado como en vilo hacia la escala, y de allí le bajaron
a un bote, que también se hundía y se encaramaba\ldots{} No pudo decir
lo que le pasó hasta sentirse arrojado como un fardo sobre los losetones
del muelle. Su amigo el Capitán Pulpis vino a darle ánimos. Sacó
Santiuste fuerzas de su extenuación, y evocando su dignidad y mirándose
el uniforme que vestía, se puso en pie, anduvo\ldots{} Entre soldados
que se reían de su facha y desaliento, llegó a un sitio donde le dieron
vino y pan. Habría preferido café, caldo, cualquier bebida caliente;
pero hubo de conformarse, pues no estaba el tiempo para pedir cotufas en
el golfo. Vio mujeres que, al paso de la tropa, le miraron compasivas.
La mirada de las hembras levantó un poco su espíritu y le entonó el
desmayado cuerpo.

Oyó salutaciones, clamor de vítores. Con decir \emph{¡viva la Reina!},
lo decían todo pueblo y soldados. Llegaba la hora del sacrificio por la
patria, y era indecoroso pensar en comer. Adelante, adelante. La
muchedumbre militar, en cuya retaguardia iba el mísero poeta y orador
Santiuste, marchaba por la población ante un abigarrado gentío. Vio
casas de desigual altura, unas con tejado, otras con azoteas; vio que
por encima de algunas tapias asomaban palmeras y naranjos\ldots{} vio
caras compungidas y caras risueñas\ldots{} Luego pasó por un conducto
obscuro y estrecho, semejante a los túneles del ferrocarril; pasó por un
puente levadizo, bajo el cual se extendía profundo foso vestido de
hierba; vio bastiones, plazas de armas con pirámides de balas negras
junto a los cañones verdes, inválidos; franqueó puertas rematadas con el
escudo nacional, y, por fin, vio campo, terreno inculto a derecha e
izquierda, lomas áridas con algunos grupos de chumberas o palmitos,
entre peñas, y ya no veía mujeres ni paisanos. La tropa, en cuyas filas
iba, avanzaba silenciosa: a lo lejos, a medida que el paisaje se abría,
divisó el cronista soldados de todas armas, en grupos, no en actitud de
combatir, sino de descanso; acémilas que volvían descargadas, camillas
que aún no transportaban heridos. De moros no veía Juan ni rastro por
ninguna parte.

Agradeció mucho el poeta militar que la masa de tropa, dentro de la cual
era como gota de agua en la ola movible, suspendiera su marcha,
alcanzado quizás el término de ella. Difícilmente se tenía ya en pie, y
necesitaba evocar toda su dignidad y todo su patriotismo para no
tumbarse a un lado del sendero. Algo le consoló ver que los soldados
reconocían los sitios en que debían armar sus tiendas, y observó con
gozo todos los indicios de esta función doméstica que aun en la vida de
campaña es indispensable. Oyó que aquel lugar se llamaba \emph{El
Otero}; le animó mucho el notar que los soldados, alegres y activos, no
se recataban para manifestar su horroroso apetito. Desde la salida de
Cádiz no había vuelto a ver a su amigo Rinaldi: le suponía junto al
General en Jefe, y a este se le figuraba en Ceuta, ordenando la
situación de las fuerzas en los puntos convenientes para comenzar la
campaña. La atenuación física desmedraba de tal modo las facultades
mentales de Santiuste, que apenas podía discurrir, y al intentarlo no
lograba traer a sus juicios la lógica fugitiva. No sabía en qué Cuerpo
de Ejército se encontraba, ni si era su Jefe Prim o Zabala.

El capitán Pulpis, única persona con quien hablar podía, pues los demás
no paraban mientes en él ni le hacían ningún caso, le dijo que más
adentro, fuera ya del campo neutral, había un caseretón llamado \emph{El
Serrallo}, que fácilmente ocupó Echagüe días antes. Rodeado aquel sitio
de cerros eminentes, en estos se levantaron fuertes. Atacaron los moros;
se les rechazó en cuantas embestidas dieron. Habíamos tenido pérdidas;
ellos muchas más\ldots{} Ya que pisaban territorio marroquí dos Cuerpos
de Ejército, y el Tercero no tardaría, pronto veríamos formidables
batallas\ldots{} Todo esto le hubiera parecido muy bien al amigo
Santiuste, si se encontrara en el estado de equilibrio fisiológico que
permite la fácil apreciación de los planes guerreros, pues los estómagos
vacíos obscurecen las facultades del alma, y esta no puede darse cuenta
de cosa alguna referente a la gloria y al patriotismo. Más que las
noticias de los encuentros, honrosamente sostenidos por Echagüe,
agradeció Santiuste que Pulpis le brindara el abrigo de la tienda,
acabada de armar por los soldados; allí esperaría la comida que les
diesen, la cual no había de ser mucha, pues las raciones venían escasas
por no poderse transportar desde Cádiz, Málaga y Algeciras todo lo
necesario.

Iba cayendo la tarde. El machacante de un sargento, de la compañía de
Pulpis, dio pan al extenuado cronista; este se reanimó; fue recobrando
su ser, desvirtuado por el mareo, el cansancio y el ayuno, y pudo
esperar, con relativa paciencia, la hora feliz en que repartieran algo
caliente y sabroso. Esto llegó al fin, y devorado fue sin que nadie
pusiese el menor reparo. Dio Santiuste gracias a Dios y a Pulpis por la
reparación de su cuerpo, que le devolvía gradualmente las luces y el
vigor del alma. Un poco de café, mal colado y caliente, iluminó más el
cerebro del héroe por fuerza, poniéndole en condiciones de enterarse de
todo, de apreciar los juicios que oía referentes a hechos y a personas.
Recostado en la parte de la tienda donde menos estorbo podía causar su
cuerpo, escuchó comentarios que los oficiales hacían de la situación y
objetivos del Ejército, y pudo entender que aún no se sabía con certeza
si iríamos sobre Tánger o sobre Tetuán. Dominaba entre los contendientes
la opinión de que lo segundo era difícil, y lo primero imposible.

El comandante don Luis de Castillejo, hombre de historia militar y
social muy cuajada de peripecias, y además despejadísimo, aseguró que si
el objetivo era Tetuán, el Ejército debió tomar tierra africana en la
desembocadura del Río Martín. Él conocía palmo a palmo toda la costa,
por haberla recorrido a pie o en lancha, en ocasiones dramáticas de su
vida. Además, había servido en Ceuta, en Alhucemas y en Chafarinas;
conocía también parte del territorio de Anyera, y podía resueltamente
asegurar que el mejor punto de desembarco para contener a los anyerinos
y expugnar a Tetuán era el Río Martín. ¿Cómo no lo vio así el General en
Jefe cuando salió en el \emph{Vulcano} a recorrer la costa? O no pudo
acercarse bastante por causa del ventarrón que aquel día reinaba, o los
técnicos que llevó consigo no pudieron asesorarle bien, por no haber
estudiado previamente la costa entre Cabo Negro y Cabo Mazari, ni las
débiles defensas que tienen los moros en la boca del río.

El sueño cerró las bocas de los oficiales, y Santiuste se adormeció
pensando en su compromiso de referir puntual y rectamente cuanto viese.
Su amigo y protector Beramendi le había dicho: «Hágase cuenta de que
escribe para mí solo, y sea esclavo de la verdad.» Ajustando sus ideas
al recuerdo de la voluntad del Marqués, se durmió con este propósito:
«Mañana escribiré que todavía no sabemos a dónde vamos\ldots{} que
quizás el Estado Mayor tampoco lo sabe\ldots{} que el desembarco en
Ceuta es un disparate estratégico\ldots»

Y despertando al toque de diana, que en el campamento sonaba como himno
religioso, pensó que si debía ser estrictamente sincero con el simpático
Fajardo, a su amiguito Vicente Halconero, hijo de Lucila, escribiría en
tonos de patriotismo infantil y sonrosado, así, por ejemplo: «Todo
admirable, todo conforme al ensueño\ldots{} los generales
acertadísimos\ldots{} los soldados alegres, deseando batirse, batiéndose
como leones\ldots{} como españoles bien comidos\ldots{} la pitanza
pronta en todo caso, y abundante\ldots{} los moros iracundos en el
ataque\ldots{} cayendo como moscas\ldots{} el país precioso, con oasis,
palmeras, camellos\ldots{} higos chumbos por todas partes\ldots{} las
mezquitas arrasadas por los nuestros\ldots{} la Cruz triunfante, y ¡viva
España!»

Medio repuesto ya del gran quebranto del viaje, salió Juan a pasear por
el campamento, y no fue poco su asombro al ver que, recorriendo un gran
espacio de terreno, no dejaba de ver tropas y más tropas. Queriendo
llegar al fin de aquel humano enjambre, siguió laderas abajo y laderas
arriba hasta dar en un cerro que llamaban del \emph{Renegado}. Desde
allí se veía el mar por una parte, por otra las alturas en que se
alzaban los fuertes que mandó levantar Echagüe. Internándose un poco,
vio \emph{el Serrallo}, construcción vieja, almenada, y en torno a ella
más tropas\ldots{} Aunque no conocía, como Vicentito, los números de los
Cuerpos, pudo apreciar, por la variedad de cifras, la muchedumbre de
aquellos. Cuarenta y un batallones, según alguien le dijo, ocupaban
aquel territorio. Los soldados, alegres y bulliciosos, deseaban que les
echaran moros para dar cuenta de ellos.

Volvió a su tienda el trovador, y se ocupó en escribir sus primeras
cartas, lo que hizo con la prolijidad y cuidado de un primerizo en tales
obligaciones. Aún conservaba el sentimiento de su deber, no turbado por
el cansancio; aún hervía en su mente la ilusión de grandezas épicas,
anunciadas por la voz inequívoca de los corazones, así como por la
profética voz de los vates políticos y literarios. Dio Santiuste, en sus
dos cartas, noticias desacordes: en una pintaba la realidad; en otra
dejaba correr su loca fantasía. Pero ya porque no tuviese costumbre de
poner la debida atención en las cosas prácticas, ya porque su cerebro no
estaba aún bien firme, equivocó los sobrescritos de los pliegos,
enviando a Beramendi la carta imaginativa, la real a Lucila y su
niño\ldots{} El cantor de glorias no se enteró del trueco hasta muchos
días después, cuando vio en un periódico las lindas parrafadas poéticas
que dirigió al adorado hijo de la celtíbera.

Ansiaba Santiuste ver moros, y presenciar una gallarda pelea. Poco hubo
de esperar para la satisfacción de su anhelo, porque a mediodía del 30
vomitó Sierra-Bullones gran morisma. Bajaban y se escondían entre
matorrales, rompiendo el fuego contra los españoles. Estos acudían hacia
ellos; daban el cuerpo los berberiscos con espantosa gritería; cundía el
fuego en extensión considerable. Desde la vertiente sur de la hondonada
del Serrallo, donde se hallaba Juan, no podía ver este sino una parte de
la acción. Subiendo un poco para ver mejor, sin cuidado de mayor riesgo,
encontrose a unos cuantos mirones junto a un peñasco guarnecido de
chumberas. Arrimose también allí. Un amigo le cogió por el brazo: era
Enrique Clavería, de Administración militar, jovenzuelo muy simpático,
hijo del Coronel de un regimiento que había quedado en la Península.
Santiuste y el joven Clavería, que también era un poco literato y
enjaretaba versos como todo buen español de veinte años, pusieron toda
su atención en el espectáculo que delante tenían. Vueltos de cara al
Oeste, por donde se columbraba la angostura llamada boquete de Anyera,
vieron que los moros salían por aquella parte como nube de moscas.
Admiraba el cronista su agilidad de saltamontes; las burdas chilabas,
del color de la tierra, les confundían con esta; se les veía perderse
entre matorrales y salir de ellos saltando, con rápida flexión de sus
zancas obscuras.

Todo lo que Santiuste ignoraba respecto a Cuerpos y personal del
Ejército, lo sabía Clavería. Este le designaba los movimientos, y qué
fuerzas los efectuaban. «¿Ves cómo se despliegan en línea? Allí está la
izquierda; la derecha nos la tapa esa loma, que no nos deja ver el
barranco del Infierno.»

---¿Y tu General dónde está?

---¿Echagüe? ¿Dónde ha de estar sino en el sitio de mayor peligro? Allí,
en la derecha le tienes: no podemos verlo. Fíjate ahora en el ala
izquierda\ldots{} Enfila tu vista por aquel pedazo de muralla con
dientes, que parece ruina de una mezquita\ldots{} ¿Ves de dónde sale
tanto humo? Pues allí está Lassausaye, ese inglés valiente como un gallo
de pelea\ldots{} Es de los que no retroceden así les parta un
rayo\ldots{}

---¿O'Donnell dónde está? Se habrá quedado en el Otero dando sus
disposiciones.

---¡Quia!\ldots{} le tienes aquí\ldots{} ¿Ves el Serrallo?\ldots{}
Enfílate por la torre del Este\ldots{} un poquito más allá\ldots{}

---Ya, ya veo\ldots{} distingo la escolta\ldots{} Ahora pica espuelas,
sube hacia la línea de combate. ¿Será que la cosa anda mal?

---El General en Jefe avanza\ldots{} Va en busca de Zabala. ¿No ves a
Zabala?\ldots{} Allí, junto a la loma que nos tapa la vista del ala
derecha.

Los otros mirones, que eran acemileros del Primer Cuerpo, y un médico
del Segundo, prorrumpieron en exclamaciones de júbilo al ver la gran
polvareda y el humazo que marcaban una tenacísima refriega en el ala
izquierda. Aseguró uno que veía moros sin cuento cayendo patas arriba;
otros, con bárbara temeridad, se aproximaban a los españoles, disparando
sus espingardas casi a boca de jarro. «Ese Lassausaye es de hielo por de
fuera, y por dentro todo fuego---exclamaban.---¡Bien por
\emph{Simancas}, bien por Las Navas!\ldots{} ¡Vaya una muestra de
cazadores!\ldots» Loco de entusiasmo, un acemilero se puso las manos en
la boca formando caracol, con el vano intento de llevar su voz a tanta
distancia, y con toda la fuerza de sus pulmones gritó: \emph{«Simancas,}
hijo mío, ¡bravo!\ldots{} Aquí está España mirándote\ldots{} ¡Bravo,
\emph{Simancas}, hijo!»

\hypertarget{ii-1}{%
\section{II}\label{ii-1}}

«¿Y \emph{Talavera?»} preguntaba el médico.

\emph{---Talavera} está con Echagüe\ldots{} allá\ldots{} detrás de la
loma. No podemos verlo\ldots{} Pero los tiros y el humo dicen que los
moros cargan por aquella parte.

En efecto, los moros se corrían hacia las alturas del \emph{Renegado}:
querían envolver a Echagüe. Pero allí tenían la peor de las posiciones,
por causa de los cantiles que precipitaban el suelo hacia la mar. Con
todo su valor insensato, nada lograron a la postre. \emph{Talavera} y
\emph{Borbón} les sacudieron de firme en todo el resto de la tarde, y al
fin, los que no pudieron ganar el monte se arrojaron por el cantil
abajo, para esconderse entre las peñas donde reventaban las olas. Ya
anochecía cuando Santiuste y los demás vieron regresar a O'Donnell con
Zabala hacia el Serrallo; después bajó Echagüe. Todos traían cara de
haber cumplido su deber con fruto. El llamado Dios de las Batallas les
había dado el éxito de cada día\ldots{} No fue ciertamente victoria sin
quebrantos, pues muertos quedaron siete oficiales y cuarenta y tres
soldados. Los heridos fueron doscientos sesenta, contándose entre ellos
tres jefes y catorce oficiales.

En marcha hacia su campamento, situado entre el Otero y la Veguilla, no
lejos del Cuartel general, Juan sintió el descenso de su entusiasmo, al
ver que en una camilla traían al pobre Pulpis gravemente herido. Metiose
con él en la tienda, decidido a ser el primero en asistirle, y pasó una
noche tristísima oyendo los lamentos del capitán, acribillado a balazos
y con una grave herida en la cabeza. Aunque el médico aseguró que no
había peligro de muerte, no se calmaba el afán de Santiuste ante el
lastimoso estado de su amigo, ni este se conformaba con que le enviaran,
como cuerpo inútil, a los hospitales de Ceuta, privándole de compartir
las glorias de \emph{Simancas} en los restantes lances de la
guerra\ldots{} Pero el descorazonamiento del cronista no llegó a las
frialdades más negras hasta la siguiente mañana, cuando le dio por
recorrer todo el lugar de la acción del 30. Los heridos que en las
tiendas de sanidad veían eran cientos, y a él le parecieron miles. Los
muertos que vio recoger y conducir a las sepulturas, formaban en su
mente fúnebre legión. Iba el capellán castrense de un lado para otro
echando responsos con militar presteza, y a su paso desaparecían bajo la
tierra tantos y tantos jóvenes que horas antes fueron vigorosos, sentían
intensamente la alegría de vivir, y se juzgaban mantenedores del honor
de su patria. Por esta caían en el hoyo, como los musulmanes perecían
también por el honor de la suya, juntándose debajo de la tierra los dos
honores, que en la descomposición de la carne quedarían reducidos a un
honor solo.

El noble corazón del orador y poeta sintió la misma lástima ante los
muertos berberiscos que ante los cristianos. Estos eran enterrados con
mayor respeto; los otros por simple ley de sanidad, para que no
corrompieran el aire. Vio en los moros caras muertas de pavorosa
hermosura. Muchos contraían los labios con sonrisa de burla o de orgullo
desdeñoso. Las cabezas rapadas, oprimidas por el lío de cuerdas de pelo
de camello, al modo de turbante, tenían el color de las calabazas de
peregrino; las manos, por fuera negras, amarillas por la palma, ofrecían
con sus crispados dedos las más extrañas formas\ldots{} las piernas
flacas y de color terroso, en algunos teñidas de sangre, mostraban, como
los brazos, inverosímiles contorsiones y posturas de una gimnasia
fantástica. Todo esto lo vio y pensó Juan, observando cómo los vivos se
desembarazaban de los muertos. Los cadáveres moros, que yacían no lejos
del mar, eran arrojados por el cantil abajo, y algunos quedaban con
medio cuerpo en el agua y medio en las rocas, para el equitativo reparto
entre aves y peces.

Empezó a soplar aquel día Levante furioso, que por la noche trajo
abundante lluvia. Vio Santiuste que el África se envolvía en nube de
tristeza, y que los vivos colores de su suelo se desleían en un medio
fangoso y opaco. Del mismo modo, en el alma del solitario joven se iba
marchitando y desluciendo la ilusión de guerra. Quizás, pensó, no había
visto aún bastante guerra para conocer y juzgar fríamente este aspecto
de la acción humana, tan antiguo como el mundo\ldots{} Quizás influía en
su ánimo el feísimo cariz del tiempo, la lluvia constante, la suciedad
del piso y la consiguiente inacción del Ejército, que además de
aburrirse, sufría escasez por no andar muy corriente el servicio de
bucólica. Las operaciones, en aquellos húmedos días, de suelo enfangado
y pardo cielo, no tuvieron importancia: redujéronse a tentativas
aisladas de los moros contra los fuertes que dominaban el
\emph{Serrallo}.

Trasladado a Ceuta el capitán Pulpis con todos los remiendos que en su
agujereado cuerpo pudo hacer la Facultad, quedó Juan más desconsolado y
triste. Asistir y curar al herido, charlar con él, más en broma que en
serio, cuando le veía en buena disposición mental, era inefable consuelo
para el alma de Santiuste, encendida siempre en fuego de amor al
prójimo\ldots{} Pero Dios, que miraba por el hombre bueno y piadoso, le
deparó, a cambio de la amistad perdida, otra de bastante precio; y fue
que a punto de ver partir a Pulpis, hizo conocimiento con un clérigo
castrense, llamado don Toribio Godino, el cual, desde las primeras
palabras, se le reveló como varón sencillísimo, de corazón generoso y
ameno trato.

Grandes coloquios tuvieron el cura y el desengañado poeta en aquellos
días de calma tediosa, arrimados al hueco menos frío de una tienda.
Franqueándose uno y otro, como si toda la vida se hubieran conocido,
resultó que el señor Godino era primo de doña Celia, la señora de
Centurión; que había sido muy amigo del coronel Villaescusa, padre de la
famosa Teresita; que a esta y a \emph{la Manuela}, su madre, las conocía
como si las hubiera\ldots{} dado a luz\ldots{} Peor era la madre que la
hija, pues esta tenía buen corazón, y si pecaba era por despojar a los
ricos para dar a los pobres. Gracias a ella don Toribio no se había
muerto de hambre en el invierno del 57, que fue de los más crudos.
Teresa robaba a los ángeles su figura y modos para meterse en líos de
caridad. Era un contrasentido, un disparate moral\ldots{} De confianza
en confianza, hizo don Toribio historia de los hechos culminantes de su
vida, ya bastante larga, pues andaba al ras de los setenta. En su
juventud había conocido y tratado a famosos clérigos, como Ruiz Padrón,
Muñoz Torrero y otros, de quienes se le pegó el tufillo liberal, que no
pudo echar fuera de sí en sucesivos años. Fue perseguido el 24 con tal
encarnizamiento, que si no se refugiara en Portugal, le habrían quitado
la vida. Repatriado en tiempos de la Regencia, vivió gracias a la
protección del señor Garelly, y de don Javier de Burgos, que si no le
estimaba mucho como sacerdote, apreciábale como latinista\ldots{}
Míseramente pudo sostenerse en curatos rurales, luchando con la
malquerencia de facciosos más o menos encubiertos. Siguió hasta el 50
amparado de la obscuridad, sin poder aspirar a mejor acomodo; pero en
aquella fecha se desencadenó contra él furioso viento de persecución,
sin saber de dónde venía, y obligado a trasladarse a Madrid, se le acusó
de masonismo y se le retiraron las licencias. Tales injusticias y
crueldades indujeron al don Toribio a ser poco discreto en la
manifestación de sus ideas, un tanto libres en todo lo que no
perteneciese al dogma. Siempre fue ortodoxo; mas no lo creían así sus
colegas, sin duda por ser hombre que al pie de la letra practicaba el
\emph{in dubiis libertas}. Por fin le vino Dios a ver en la persona del
General Zabala, el cual, apiadado de él y juzgándole sin prejuicios ni
malquerencias, le sacó de aquel anticipado Purgatorio y le trajo al
clero castrense, donde el pobre señor respiró viéndose rodeado de
compañeros buenos y tolerantes. En su ardiente gratitud, aplicaba al
digno General el \emph{Deus nobis haec otia fecit}, y se sentía capaz de
dar la vida, si necesario fuese, por la de su noble bienhechor.

En sucesivas conversaciones, cuando lo permitía el ocio del campamento,
Santiuste confió al buen clérigo algunos particulares de su vida; y una
tarde, viniendo a parar a sus recientes dudas o desfallecimientos en la
fe y devoción de la guerra, le dijo: «¿Cree usted, amigo don Toribio,
que existe el llamado \emph{Dios de las Batallas}? ¿Cree usted en esa
confusión del Marte pagano con nuestro Cristo Redentor, que jamás cogió
una espada? ¿Qué piensa usted de la Virgen, como dispensadora del
triunfo en las guerras, al modo de aquellas diosas que tomaban partido
por los griegos o por los troyanos? ¿Al Apóstol Santiago le tiene usted
por verdadero general de españoles y matador de moros? ¿Dónde está el
texto de Cristo en que dijera a sus discípulos: montad a caballo y
cortadme cabezas de los hijos de Agar?.»

Sonrió el castrense mirando al suelo, y rascándose la barba, no afeitada
en seis días, respondió de este modo a la consulta: «Hijo mío, nos hemos
encontrado esas tradiciones de fe, y tenemos que respetarlas sin
meternos en libros de Teologías. A mí, la verdad, no me caben en la
cabeza Dios guerrero, ni Jesucristo militar, ni Nuestra Señora con
bastón de Capitana Generala; pero eso pertenece al conjunto de creencias
y de actos sacramentales que me dan de comer. De todo ese conjunto como,
y el alimento es cosa capital, hijo mío; pues si yo observo los ayunos y
abstinencias que la Iglesia me manda, no estoy por pasar hambre todo el
año. Ya sabes que \emph{el abad de lo que canta yanta}. Yo canto todo lo
que sea preciso para un yantar moderado y sin gula. Y no te digo más,
que con lo dicho basta para que sepas la opinión de un capellán de tropa
que sabe cumplir sus deberes\ldots{} Y ya que de comer tratamos, sabrás
que nos esperan fatigas y no pocos ayunos, fuera de los días de rúbrica,
porque vamos hacia el Sur. ¿No lo sabías? Sí: de esta ratonera en que
estamos no podemos salir más que escabulléndonos por la costa. Ya tienes
al Tercer Cuerpo, que manda el general Ros, acampado en esa parte del
\emph{Tarajar}: ya han empezado allá las obras que han de proteger
nuestro camino. Hacia Río Martín vamos, de donde subiremos a Tetuán, si
Dios lo quiere, pues aunque no exista el \emph{de las Batallas}, Dios
hay que sobre todos los actos de los hombres impera, así moros como
cristianos\ldots»

Recluido Juan en el campamento del Otero, apenas se dio cuenta de la
acción del 12, en que Prim, con los regimientos del \emph{Príncipe, de
Granada}, cuatro compañías de \emph{Almansa}, cazadores de
\emph{Vergara} y otras fuerzas, acudió a la defensa del camino que
abrían los ingenieros junto al reducto del Príncipe Alfonso, para
franquear la marcha a lo largo de la costa. Atacaron los moros con
fiereza; pero pudo más Prim, que los destrozó y dispersó, secundado por
el coronel del \emph{Príncipe}, don Cándido Pieltaín, y el coronel de
\emph{Granada}, don Miguel Trillo\ldots{} En esta rápida y vigorosa
acción, murió el coronel de Artillería don Juan Molíns. Gran duelo hizo
todo el Ejército a este ilustrado y valiente militar\ldots{} La acción
del 15, parte por lo que pudo ver, parte por lo que le contaron, la
relató Santiuste en las dos cartas que escribió a Madrid, con corta
diferencia en el sentido y tono de una y otra. El nubarrón de moros que
descargó por el boquete de Anyera parecía como un diluvio de hombres.
Tras los de a pie, que no bajarían de catorce mil, se desgajaron de la
altura como unos mil de caballo, turbamulta vistosa, pintoresca, de
pasmosa agilidad y gallardía en sus movimientos. Se creyó, y luego quedó
plenamente confirmado, que al frente de los gallardos jinetes venía
Muley el Abbás, hermano del Emperador y caudillo de su Ejército.

El incansable inglés Lassausaye y el General Gasset, con fuerzas del
primer Cuerpo, reciben dignamente a toda aquella caterva; mientras
avanza O'Donnell hasta el centro de la línea de combate, el General
García desbarata la falange mora, haciéndola retirar hacia el mismo
boquete por donde había entrado en escena; hasta muy cerca de la bahía
de Benzú persigue Lassausaye a los jinetes, que huyen, con la fantástica
presteza que ponían en todos sus movimientos: se les ve como una nube de
saltamontes que levanta el vuelo\ldots{} Tendidos sobre el cuello de sus
veloces caballos, al viento los alquiceles blancos, parecían visiones de
hipogrifos que tornan a sus cuadras mitológicas, entre el cielo y la
tierra. ¡Hermosa y teatral acción, tan decisiva y brillante para los
españoles, que algunos pudieron creer reproducida la milagrosa
intervención del Apóstol Santiago! Por esto decía Santiuste en su carta
a Lucila y Vicentito: «No vimos a Santiago; pero allí estaba\ldots{} yo
sentí estremecido el suelo por las herraduras de su caballo.»

\hypertarget{iii-1}{%
\section{III}\label{iii-1}}

En las acciones del 20, 22 y 25 de Diciembre, repitieron los moros su
intento de interrumpir los trabajos del camino de Tetuán. Pero en el
espacio que mediaba desde el boquete de Anyera al campamento del Tercer
Cuerpo, no lejos de los bosques donde aquellos se guarecían, O'Donnell
puso doce piezas de montaña, y ocho de artillería rodada. Decir que los
pobres hijos de Mahoma fueron barridos, no expresa bien la rapidez
pavorosa de su fuga. Otros intentaron atacar el frente del Tercer
Cuerpo; pero Ros de Olano, que les aguardaba prevenido, mandó avanzar su
vanguardia, protegida por cuatro cañones de montaña, y no fue menester
más para que los enemigos tornaran con pie ligero a los altos de
Sierra-Bullones. Del reconocimiento que hizo Prim el día 22 en el camino
de Tetuán, habló también Santiuste en sus cartas, ateniéndose a lo que
le contaron, pues nada vio de aquel suceso. Ello fue que Prim batió y
dispersó a la caballería mora, en la entrada del Valle de los
Castillejos. Fiados en la ligereza de sus caballos, los árabes hacían
simulacro de retiradas, volaban hacia los montes, volvían de improviso
con veloz carrera y vocerío formidable\ldots{}

De la prodigiosa táctica de los jinetes berberiscos, que suplían la
fuerza con la agilidad, habló Santiuste a su amigo Vicentito Halconero,
añadiendo teorías militares impropias de la débil comprensión de un
niño. Pero el triste poeta no sabía lo que hacía: sin equivocarse en los
sobrescritos, trocaba los asuntos, transmitiendo a Beramendi relatos e
ideas infantiles, mientras al amado nieto de Ansúrez endilgaba las
consideraciones más sutiles que la campaña sugerían. A uno y otro amigo
les contó que O'Donnell había mandado repartir a la tropa castañas y
batatas, para que el 24 celebraran el Nacimiento del Niño Dios. Concedió
asimismo dos horas de esparcimiento, después del toque de retreta, para
que los soldados se divirtieran, recordando el bullicio y alegría de sus
hogares en tan memorable noche. Era quizás la primera vez que en la casa
misma del Islamismo sonaba el \emph{Gloria a Dios en las alturas},
transformado en rudas coplas por diez y ocho siglos de poesía cristiana.
Se permitió a los soldados que encendiesen hogueras; tocaron las
músicas, y el campamento español, en toda su largura, desde el Otero
hasta la Concepción, resplandecía con rojas luminarias, que lo mismo que
las alegres voces eran expresión del regocijo familiar. Reían, bailaban
y se divertían los pobres soldados a dos pasos de un enemigo feroz, y
sobre un terreno por conquistar.

Con forzado júbilo disimulaban los españoles la tristeza de la patria
ausente, y así, cuando las cornetas, a las diez en punto, tocaron a
silencio y se dio por terminada la huelga, los más divertidos cayeron en
opacas añoranzas. La Noche-Buena prosiguió dentro de las tiendas, ya en
meditaciones sobre la suerte que Dios nos depararía en Marruecos, ya en
apagados coloquios que traían a los labios de los combatientes nombres y
dichos de seres amados\ldots{} Y no bien apuntó el día, vinieron los
moros a despertar a los durmientes y a sacudir de su modorra a los
cavilosos. El tiroteo de las trincheras anunció batalla; el enemigo, que
creía habérselas con un Ejército embriagado, lo halló bien prevenido.
Toda la mañana se tirotearon españoles y marroquíes, empeñando hacia la
mitad del día combates encarnizados. Repetían los moros su táctica de
sorpresa y fingida retirada; mas el juego, descubierto por los de acá,
era completamente ineficaz\ldots{} Acababan desbandándose, sin ganar una
pulgada de terreno\ldots{} Escasas pérdidas tuvo España el día de la
Natividad; los moros cayeron en gran número, unos acribillados por las
bayonetas, otros despeñados en los cantiles. En su azorada fuga corrían
hacia el mar, y en las peñas o en medio de las olas encontraban los más
de aquellos infelices la muerte, los menos su salvación.

El día 29 de Diciembre, hallándose el trovador con ganas de sacudir la
inacción en que le tenían sus murrias, montó en el caballejo que le
habían destinado, y después de subir a las alturas para ver trincheras y
fortines, dirigiose al campamento del Tercer Cuerpo, donde tenía buenos
amigos, que no había visto desde que pisara el suelo africano. No era
mal jinete Juan, y su figura escueta, en un caballo de pocas carnes como
el que montaba, no carecía de donaire estético. Podía pasar por un Don
Quijote en la flor de su edad (veinticinco años), caballero en un
Rocinante desmedrado por la mala vida más que por los años\ldots{} Salió
mi hombre del Otero, y faldeando el cerro que divide las alturas del
Serrallo del arroyo de Anyera, se dirigió al campamento de la
\emph{Concepción} con ánimo de seguir adelante, para enterarse de las
obras del camino de Tetuán. El día era espléndido: un sol brillante
pintaba de oro y siena los montes; cielo y mar sonreían ante las
alegrías de la Naturaleza. Sintió el poeta en su alma como una
disipación de las nieblas que la envolvían, y esta claridad se le
convirtió en regocijo cuando vio venir por el cerro abajo a Leoncio
Ansúrez. Este le llamaba con fuertes voces, adelantándose a los soldados
con quienes venía\ldots{} Paró Juan su caballo al reconocer a su amigo;
hizo por abrazarle desde la altura de la silla; el armero le echó los
brazos a la cintura. ¡Qué feliz encuentro! No se habían visto desde que
llegaron al África. «¿Pero qué es de ti?\ldots{} ¿Cómo te prueba esto?
¿Estás contento? ¿Qué noticias tienes de Madrid?\ldots» Estas y otras
preguntas fueron el exordio de una conversación que de lo familiar pasó
a las cosas de interés militar y público.

«Dime, Juan, ¿te has batido?»

---Yo no, Leoncio. Mi misión aquí no es hacer la Historia, sino
contarla. Soy español de paz, por no decir moro de paz. ¿Y tú? No habrás
matado sólo conejos.

---He matado moros\ldots{} no creas que uno ni dos\ldots{}

---Como eres gran tirador, te habrán dejado meter baza\ldots{}

---Tú lo has dicho. Me arrimo a \emph{Cazadores de Baza}, que son mis
amigos\ldots{} ¡y qué quieres!\ldots{} doy gusto al dedo. Muchísimos
moros me deben el encontrarse ya en el paraíso del señor Mahoma\ldots{}
Por cierto que esos perros tienen amigos que les han traído armas
mejores que la espingarda\ldots{} mejores para ellos; para nosotros,
todo lo contrario. Mira.

---¿Qué es eso?

---Balas que he recogido en el campo de las acciones últimas. Veníamos
notando en sus tiros mayor alcance. El General me ha mandado recoger
balas, y aquí llevo las que he podido encontrar\ldots{} Por el hilo se
saca el ovillo, y por el proyectil el arma\ldots{} Yo digo y sostengo
que el nuevo armamento de algunos moros es el \emph{rifle inglés de
espiga}. Ya verá el General Ros, ya verá el General en Jefe, ya verá
España que hay aquí mano oculta\ldots{}

---El oro inglés, como solemos decir\ldots{}

---Pero no les vale, no\ldots{} En Tetuán hablaremos, señores ingleses.

---¿Crees tú que llegaremos a Tetuán?

---Como creo que llegamos a mi campamento\ldots{} Ya estamos en
él\ldots{} Entremos por allí, que es la puerta más próxima. Llamamos a
esa entrada la \emph{Puerta de Alcalá}.

Era el fortificado campamento como un pueblo con calles de tiendas, en
líneas cruzadas a escuadra. Gran animación había en la ciudad de lona.
Todo el vecindario estaba en las avenidas y calles, gozando de la
hermosura del día y del calorcillo del sol. Unos ponían a secar ropas
recién lavadas; otros se fregoteaban el cuerpo, desnudos de la cintura
arriba. En el barrio de provisiones humeaban los peroles sobre las
trébedes; en estos ardía la leña verde con alegre estallido. Más allá,
los caballos comían su ración en sacos colgados de su propio
cuello\ldots{} Monturas, camas, mantas, todo salía en busca del
beneficio del sol\ldots{}

Se apeó Santiuste, entregando su rocín a unos ordenanzas, amigos de
Leoncio, y dijo a este: «Quiero estirar mis piernas ateridas. Te
participo que no me voy de tu campamento sin ver a Perico Alarcón. Tú me
dirás dónde puedo encontrarle.» Respondiole Ansúrez que Alarcón, si no
estaba en su tienda, estaría en la del General o en la del Duque de Gor.
Siguieron andando, y en esto observaron que las alturas que dominaban la
costa, sobre la ensenada llamada Uad Arrial, estaban pobladas de
curiosos, oficiales en su mayor parte, vueltos hacia el mar, algunos
provistos de anteojos. ¿Qué pasaba en el mar? Corrieron hacia allá los
dos amigos, y antes de que llegaran a las alturas, voces alegres de los
que volvían les enteraron del caso. ¡Era la escuadra, la escuadra
española, que navegaba hacia el Sur para bombardear los fuertes moros de
Cabo Negro y Río Martín! Se veían perfectamente, sin anteojos, las
gallardas naves\ldots{} Por allí, por allí\ldots{} ¿Cuántos buques
son?\ldots{} ¡Seis, siete\ldots{} son nueve, entre vapores y de
vela!\ldots{} Ya se veía la nave delantera desaparecer tras la punta del
Cabo; ya iban entrando una tras otra en la ensenada de Río Martín;
pronto se oirían cañonazos\ldots{}

Pasó algún tiempo, y un silencio religioso se cernía como nube sobre los
grupos de mirones. Entre ellos estaba el General del Tercer Cuerpo, el
Coronel Duque de Gor, los Brigadieres Cervino y Mogrovejo: allí multitud
de jefes y oficiales; pero Alarcón no parecía. Después de mirar
detenidamente en todos los grupos, supieron, por referencias de un amigo
de Enrique Clavería, que el cronista del Tercer Cuerpo había ido al
Cuartel general, a que don Leopoldo le diera informes oficiales de aquel
movimiento de la escuadra, para poder escribir su próxima carta \emph{De
un testigo} con el debido conocimiento de las operaciones
proyectadas\ldots{} En esto sonó tiroteo próximo\ldots{} De improviso
todos los curiosos volvieron más que de prisa al campamento. Sonaron las
cornetas llamando a formación. Con rapidez eléctrica, los hombres
dispersos en las calles de la ciudad de lona se agruparon en haces
guerreros. Oyó Santiuste que gritaban: \emph{¡Baza, Baza!} Iban a salir
los Cazadores de este nombre para rechazar a los moros, que ya
zancajeaban dando alaridos de peña en peña. El enjambre corría no lejos
del campamento, extendiéndose por las alturas que descienden hasta el
mar, cerca ya de los Castillejos\ldots{} Sale \emph{Baza} con mágica
presteza; le siguen fuerzas de \emph{Llerena}, \emph{Granada} y
\emph{Zamora}\ldots{} El enemigo embiste a los soldados de
\emph{Vergara} que protegían los trabajos del camino\ldots{} Y cuando el
tiroteo es más sonoro, óyense los zambombazos de los barcos de guerra,
hacia el Sur, repercutiendo en los aires como truenos lejanos\ldots{}

Fascinado Leoncio por la marcha de los de \emph{Baza}, corrió tras
ellos, dejando solo a su amigo. Pensaba este retirarse, y cuando iba en
requerimiento de su caballo, que pastaba en un padrillo del Tarajar con
otros jamelgos y dos burros de los cantineros, vio venir a Perico
Alarcón presuroso, en dirección a su campamento. Los dos amigos se
reconocieron y gozosos se juntaron. No se habían visto desde Madrid;
anhelaban referirse mutuamente sus impresiones de la guerra\ldots{} Mas
la ocasión de charlar no era la más propicia, porque el uno quería
volverse a su campamento; el otro, ardiendo en curiosidad, se iba con el
alma y con los ojos hacia el camino de Tetuán, donde sonaba el vivo
tiroteo. «Déjame aquí, Pedro---dijo Santiuste, oponiendo su pesada
inercia a la viveza de su amigo.---Estoy enfermo. Vete tú, y si no
tardas en volver, te aguardaré donde me indiques.» No necesitó Alarcón
más licencia para salir disparado, diciendo a Juan que le esperase en
tal tienda de \emph{Ciudad-Rodrigo}, una de las más próximas al sitio
donde se separaron.

En cuanto estuvo solo Santiuste, dejó al Acaso que guiara su ambulación
incierta: lleváronle sus pasos ante una gran tienda, que al punto
reconoció como Hospital de Sangre, por el número de camillas que en su
interior desde fuera se veían y por los olores farmacéuticos envueltos
en exclamaciones de dolor que en la puerta recibían al visitante. Entró
Juan, a punto que sacaban en parihuelas un soldado muerto para llevarle
a enterrar. Tres heridos graves yacían sobre colchonetas, rígidos, en
posición supina, alguno de ellos con la cara tan cruzada de vendajes,
que no se le veían las facciones, y más parecía envoltorio que ser
humano. Hacia el fondo de la tienda, un oficial agonizaba: tenía puesto
el ros, desnudo el pecho de ropa, mas no de bizmas y vendajes, pues toda
la región torácica era una criba. Además, le habían amputado un brazo. A
una señal del médico, un auxiliar sanitario quitó el ros al moribundo y
le cubrió con sábana y manta hasta la boca. Los ojos tenía muy
abiertos\ldots{} El cura, después de mascullar latines para encomendar
el alma, rezaba en silencio\ldots{} Retirose el médico para arrimarse a
otros en quienes aún podía ser eficaz la ciencia. Aproximándose al
expirante, Juan le vio dar las boqueadas, con que pasó de la vida a la
muerte. El castrense dijo a Santiuste: «¡Lástima de chico! Es hijo del
coronel Gallo, y acabadito de salir de la Academia de Toledo le trajeron
a esta campaña.»

Acongojado estaba Juan ante el espectáculo de aquellos martirios; pero
no sabía salir del hospital. Viendo a un herido que en su delirar
ardiente cantaba coplas obscenas, a otro que se condolía de su suerte
con ahogados acentos, observándolos a todos, y el entrar y salir de
médicos o asistentes de Sanidad, se le pasaba el tiempo sin sentirlo.
Menos espanto le causaban aquellas lástimas que el horrible tiroteo, a
cada instante más nutrido y cercano\ldots{} Cuando ya la tarde declinaba
y los sirvientes del hospital encendieron velas, el ruido de tiros se
iba apagando, perdiéndose en invisibles lejanías. De pronto vio Juan que
llegaban a la tienda camillas con nuevas víctimas, en número tal, que
tuvo que echarse fuera para hacerles hueco. Heridos llegaron
silenciosos, que parecían muertos; otros blasfemaban, increpando al
cielo y a la tierra; algunos bromeaban, comentando su mala estrella con
picantes dicharachos\ldots{} La sangre derramada y las vidas en peligro,
de sí mismas se burlaban.

Fue y vino Santiuste un rato entre las tiendas próximas, viendo soldados
ilesos que en grupos alegres volvían al campamento, hasta que tuvo la
suerte de ser encontrado y detenido por Pedro Antonio de Alarcón, que
haciendo presa en su brazo le dijo: «Palomino atontado, ya te cogí:
pensé que te habías ido\ldots{} ¡Vaya un julepe que se han ganado los
moritos!\ldots{} Ven y te contaré. Esta noche la pasas conmigo.
Cenaremos juntos\ldots{} tengo provisiones muy ricas\ldots{} Ven\ldots{}
No chistes; no te me escapas\ldots{} Eres mi prisionero.»

\hypertarget{iv-1}{%
\section{IV}\label{iv-1}}

Aunque era de soldados la tienda de Perico Alarcón, ofrecía dentro de
sus paredes de lona refinamientos epicúreos. Dos velas podían lucir
colocadas en botellas vacías; había mesa de tijera, como un catre, para
comer; dos y hasta tres sillas del mismo sistema de abre y cierra. Las
latas que contuvieron sardinas o carne salada de buey hacían veces de
vajilla para servir diferentes manjares; las camas de dos dobleces eran
muy cómodas, con grupas de cabalgaduras por almohadas y buenas mantas de
abrigo. Del mástil que sustentaba todo el artificio de la tienda pendían
objetos de puro lujo en campaña: estuche de afeitarse, abrigos
impermeables, gorros para dormir, un saquito con castañas y nueces, la
máquina de daguerrotipo, un manojo de chorizos y otras cosas de uso
común en la vida. En una cesta, cariñosamente colocada entre dos camas,
se guardaban botellas de Jerez y algunas de \emph{champagne}, obsequio
del General del Tercer Cuerpo al amigo que ilustraba la guerra con sus
admirables narraciones y comentarios.

En el compañerismo más ecualitario descansaban allí vanos soldados y un
oficial, a más de Pedro Alarcón. Todo era común, la comida y los avíos
domésticos. Apenas entró el oficial, acostose rendido: no era para menos
la acción de aquella tarde, después de doce horas en el servicio de
trinchera. Se quitó el uniforme, quedándose con la camiseta de tartán
rojo y los calzones interiores de lo mismo; se lió a la cabeza un
pañuelo de hierbas; se comió un chorizo; luego bebió del café caliente
que de la hoguera próxima trajeron los soldados, y tartamudeando las
buenas noches se entregó a un sueño profundo. Alarcón y su amigo,
decididos a regalarse con una cena opípara, se sentaron junto a la mesa:
comieron carne de lata, huevos duros, almendras, pasas, y polvorones de
Ceuta. De todo partían con los soldados. A estos les tiraba más la
sociedad de sus compañeros que la de personas de superior clase, y se
fueron al amor de la hoguera, donde asaron batatas y se regalaron con
café y charla sabrosa, hasta que el sueño les llevó a la querencia de
sus camastros.

«No sabes, Perico, cuánto me alegro de verte---dijo Santiuste,---ni qué
ganas tenía de charlar contigo. Sólo con oírte me siento animado y se me
abre un poco esa puerta de la nutrición que llamamos apetito, y se me
cierra la de esos desvanes que llamamos melancolías.»

---Tú estás enfermo, Juan---contestó el otro;---tienes la malaria de los
campamentos, quizás nostalgia de personas y afectos que has dejado allá,
en esa Berbería bautizada que llamamos España. La malaria castrense es
achaque de los que no tienen costumbre de dormir al raso, o en estos
palacios de lona con pavimentos de tierra húmeda. Pero te aclimatarás, y
como no te dé el cólera, te harás una naturaleza militar y un temple
guerrero. No te creas: más \emph{confort} hay aquí que en las
buhardillas donde tú has vivido\ldots{} y por mi parte, juro en Dios y
en mi ánima que Granada la morisca y Madrid la cortesana han sido para
mí más esquivas en la cuestión de bucólica\ldots{} en ciertas épocas,
Juan, en ciertas épocas\ldots{} más esquivas, digo, que este campamento,
donde no sólo comemos gloria, sino longanizas, batatas de Málaga y hasta
jamón de Trevélez\ldots{} como lo oyes\ldots{} En fin, cuéntame, Juan,
cuéntame\ldots{}

---En pocas palabras te lo cuento todo, Perico. Estoy desilusionado de
la guerra. Te reirás de mí, acordándote de aquel entusiasmo mío que más
parecía locura\ldots{} Pues sí, en mi espíritu se han marchitado todas
aquellas flores que fueron mi encanto\ldots{} ya sabes\ldots{} Yo me
adornaba con ellas, yo me tragaba su aroma y lo echaba por los ojos, por
la boca\ldots{} Me servían para hacerme pasar por elocuente y para que
lloraran oyéndome las mujeres y los chiquillos\ldots{} Esas flores eran
el Cid, Fernán González, Toledo, Granada, Flandes, Ceriñola, Pavía, San
Quintín, Otumba\ldots{} Pues bien, Pedro: de esas flores no queda en mi
espíritu más que una hojarasca que huele a cosa rancia y
descompuesta\ldots{} Vine a esta guerra con ilusiones de amor. La guerra
era mi novia, y yo el novio compuesto y lleno de esperanzas. Imagínate
lo que habré sufrido al ver que mi amada se me vuelve fea y hombruna,
que sus azahares apestan tanto como su boca\ldots{} ¿Casarme yo con esa
visión?, ¡quia! En vez de decir \emph{sí}, he dicho \emph{no}, y he
vuelto la espalda. La guerra, vista en la realidad, se me ha hecho tan
odiosa como bella se me representaba cuando de ella me enamoré por las
lecturas\ldots{} ¡Ay!, querido Pedro, ese mundo vivido en los libros, en
páginas de verso y prosa, ¡cuán distinto es del mundo real! Es aquel un
mundo que parece haber nacido en los libros mismos, por virtud de los
caracteres de imprenta. Lo que ahora me parece sueño, ¿fue verdad alguna
vez? Voy creyendo que no\ldots{} ¿Y cómo me explico que siendo para mí
tan antipático y repulsivo el ver a hombres matando sin piedad a otros
hombres, me hayan encantado las carnicerías de Clavijo, Calatañazor y
las Navas de Tolosa? ¡Matar hombre a hombre! ¿Y yo adoré esto, y yo
rendí culto a tales brutalidades y las llamé glorias? ¡Glorias! ¿No es
verdad, amigo mío, que muchas palabras de constante uso no son más que
falsificaciones de las ideas? El lenguaje es el gran encubridor de las
corruptelas del sentido moral, que desvían a la humanidad de sus
verdaderos fines\ldots{} ¿Te ríes, Perico? ¿Me tienes por loco?

---¡Con cien mil de a caballo!, como diría Manolo Fernández y
González---replicó el granadino,---si no estás loco, lo pareces. Juraría
yo que tus facultades están alteradas por el no comer. Si te alimentaras
como yo, no padecerías esos desmayos del pensamiento\ldots{} Come más
carne, Juan: tengo otras dos latas\ldots{} y bebe de este Jerez que
limpia los cerebros mohosos\ldots{} Vamos a cuentas. Cierto que el
hombre no debe matar al hombre por el gusto de matarlo\ldots{} ¿Pero qué
harás tú, mi querido Santiuste, si viene alguno contra ti con
intenciones de quitarte la vida? ¿Te cruzarás de brazos?\ldots{} Digan
lo que quieran los primitivos legisladores de la humanidad, nos vemos
obligados a matar a los que quieren ser nuestros matadores\ldots{} Muy
bonito, muy bonito es eso de no derramar sangre humana. Pero los
hombres, por ley natural, se han congregado en familias; las familias en
pueblos; los pueblos en naciones, y estas tienen sus territorios, sus
intereses\ldots{} Surge la lucha por los dones de la Naturaleza, la
lucha por los caminos de la tierra o del mar, ¿y cómo se han de ver y
sentenciar estos pleitos, señor Don Pacífico? ¿Por asambleas de
filósofos?\ldots{} Me maravilla que tú, que das ahora en no creer en la
guerra ni en la gloria militar, creas en la Edad de oro. Bueno:
pongámonos en la Edad de oro. Figurémonos que no hay \emph{tuyo} y
\emph{mío}, que comemos bellotas y nos vestimos de verdes
lampazos\ldots{} ¡Muy bonito, señor, muy bonito! Pero un día, en pleno
éxtasis paradisiaco, dos hombres de mal genio o dos grupos de hombres se
disputan el fruto de una encina o el chorro de una fuente. Pues ya
tienes en planta la guerra: o los hombres riñen, o dejan de ser hombres;
ya tienes un vencedor y un vencido. Adiós, Edad de oro\ldots{} El hombre
no se contenta con vivir de bellotas: inventa el pan, el vino, el
azúcar, y de invención en invención llega hasta el \emph{Pavo en
galantina con trufas, o el Pastel inglés con pasas de Corinto, ron de
Jamaica, canela de Ceilán y nuez moscada de Madagascar}. Figúrate tú las
guerras y conquistas que hay debajo de estos sabrosos ingredientes
alimenticios\ldots{}

---Ya sabía yo---dijo Santiuste triste, pero comiendo y bebiendo de lo
que Perico le ofrecía,---que ibas a tocar esa cuerda\ldots{} Es la única
que los cantores de la guerra tienen en su lira.

---También te digo que en principio, fíjate bien, en principio, creo que
la guerra es un mal, y que sería muy bueno que llegáramos a la paz
universal y perpetua\ldots{} Pero hay que esperar un poco, Juan. Cántame
esa canción de la paz dentro de veinticuatro siglos, y me tendrás
resueltamente a tu lado\ldots{} dentro de veinticuatro siglos; que no ha
de pasar menos tiempo de aquí a que los pueblos y las razas ventilen sus
diferencias en consejo de ancianos o en cátedras de filósofos\ldots{} La
Humanidad es joven. ¿Qué te crees tú?, ¿que es vieja? Está casi en la
infancia todavía\ldots{} Para verla en la mayor edad y en estado de
plena razón y juicio sereno, hemos de esperar hasta el siglo
\emph{Cuarenta y tres}, que es, como quien dice, pasado mañana por la
tarde.

---Pues en el Siglo nuestro, Perico, y sin necesidad de dar un brinco
hasta el \emph{Cuarenta y tres}, yo sostengo que la guerra es un juego
estúpido, contrario a la ley de Dios y a la misma Naturaleza. Yo te
aseguro que al ver en estos días el sinnúmero de muertos destrozados por
las balas, no he sentido más lástima de los españoles que de los moros.
Mi piedad borra las nacionalidades y el abolengo, que no son más que
artificios. Igual lástima he sentido de los españoles que de los
africanos, y si pudiera devolverles la vida, lo haría sin distinguir de
castas ni de nombres\ldots{} Y más te digo\ldots{} Creo que has sentido
tú lo mismo que yo: creo que en el moro muerto has visto el prójimo, el
hermano. Sin quererlo, tu piedad ingénita ha reconocido el gran
principio humanitario y la ley soberana que dice: «no matar.»

---Cierto, Juan, que llevamos dentro el principio; y que este principio
asoma la cabeza cuando menos lo pensamos, no lo puedo negar; pero luego
salen los hechos, la historia, el concepto de patria y de nación, y
aquel principio vuelve a meterse para dentro y se agazapa en el fondo
del alma, donde vivirá, esperando que pasen los veinticuatro
siglos\ldots{} Te confieso ingenuamente que ante los cadáveres moros veo
la Humanidad; pero ante los moros vivos, que brincando y aullando vienen
contra nosotros, veo las naciones, veo las razas, el Cristianismo y
Mahoma frente a frente\ldots{} Celebro, pues, con toda el alma que
nuestros soldados les maten, único medio de impedir que ellos nos maten
a nosotros\ldots{} Ahora tomemos café, Juan, y luego te voy a dar un
cigarro habano, que ha de saberte a gloria\ldots{}

---Eres aquí el poeta de la guerra. España trae artilleros para los
cañones, y poetas que conviertan en estrofas sonoras los hechos
militares, para fascinar al pueblo\ldots{} Porque en el fondo de todo
esto no hay más que un plan político: dar sonoridad, empaque y fuerza al
partido de O'Donnell. Yo respeto esa idea; pero digo y repito que no amo
la guerra, que me es odiosa, y me planto en el principio de no matar. Ya
sé que voy contra el pensar y el sentir de mi país\ldots{} ya sé que me
gano el desprecio o el desvío de cuantos me conocen. Perderé mis
amistades; seré un solitario, un extravagante, un loco\ldots{} Mi
destino lo quiere así. De dentro de mi alma ha salido este movimiento,
que al modo de terremoto ha trabucado mis ideas, poniendo arriba las que
estaban debajo. Me siento hombre distinto del hombre que yo era. ¿Debo
entristecerme o alegrarme?

---Ahora fumemos\ldots{} Pues te diré, querido Juan. No sé si tu
cataclismo debe alegrarte o entristecerte. Eso el tiempo te lo dirá. En
ti veo una cosa, y es que, a mi parecer, en este quiebro repentino que
das ahora, vas para San Francisco de Asís. Tienes mucho talento, Juan, y
un alma que quiere elevarse a las alturas. Antes de ahora te he dicho:
«Juan, en ti hay algo extraordinario que no sé lo que es. Ya veremos por
dónde sales.» Como tu maestro Castelar, tienes dentro un pedazo muy
grande de la divinidad. En Castelar esa divinidad es la elocuencia, un
poder de palabra que sube por encima de toda realidad y se mece en los
serenos espacios ideales\ldots{} Pues ahora veo que tú también te
remontas, y tengo que decirte lo mismo que al otro amigo del alma.
«Emilio---le he dicho, no una vez, sino cien;---Emilio, tú debes hacerte
cura. Serías un apóstol, un conquistador de pueblos y el catequizador
más grande que ha visto el mundo. Tu palabra, ineficaz para la política
por demasiado grandilocuente, sería el rayo del Evangelio\ldots» Pues lo
mismo te digo a ti: «Juan, hazte sacerdote\ldots{} serás el apóstol de
la paz y de los más bellos ideales humanos\ldots»

---No es eso, no es eso---dijo Santiuste dando golpes en la mesa,
mientras su boca chupaba con deleite el puro.---No me llama el
sacerdocio\ldots{} y si me llamara, no podría ir a él, por una
circunstancia\ldots{} ¡Pero si lo sabes, Perico; te lo he dicho mil
veces! Es que me aterra el celibato, no entro por el celibato\ldots{} Es
cuestión de temperamento, de sangre, y contra esto nada podemos\ldots{}
Conoces muy bien mis arrebatos y los terribles incendios que levanta en
mí el fuego de amor\ldots{} Mis pasiones son exaltadas, delirantes.
Divinizo a la mujer amada, y llego a creer que solos ella y yo existimos
en el Universo. Cuando estuve enamorado de la Villaescusa, mi vida era
un torbellino en que alternaban los goces celestiales con los suplicios
del Infierno\ldots{} En fin, ya te lo conté\ldots{} lo sabes
todo\ldots{}

---Pero aquello pasó.

---Pasó, es cierto\ldots{} Pero ¡ay Pedro Antonio!, después\ldots{} he
vuelto a enamorarme.

---¿Cuándo, Juan?

---No hace mucho. Otra vez ese estado de locura y candor, de pasión
ardiente, que anhela en un punto la gloria y el sacrificio.

---¡Vaya con Juan! ¿Y es, como Teresa, mujer de cabeza ligera?

---Todo lo contrario: cabeza bien firme.

---¿Casada?

---Casada\ldots{} digo, no\ldots{} es viuda\ldots{} Enviudó horas antes
de salir yo de Madrid.

---¿Hermosa?

---Su imagen entiendo yo que es única en el mundo.

---¡Con quinientos mil de a caballo, Juan!, eres el hombre de la suerte
si esa dama te corresponde.

---Entiendo que sí.

---¿Pero no lo sabes de seguro?\ldots{}

---Perico, nada más puedo decirte por hoy\ldots{} Dime tú ahora si tiene
sentido común que me recomiendes el sacerdocio, siendo yo como soy el
eterno enamorado\ldots{} Por mucho tiempo pensé que a ninguna mujer
podría yo amar como a Teresa\ldots{} y después\ldots{} aquí me tienes
loco otra vez\ldots{} Y algún día, ¡quién sabe!, si esta muere o me
retira su cariño, yo\ldots{} seguiré amando, enloqueciendo\ldots{} Mi
ternura es un filón inagotable. Ya ves que estoy incapacitado para la
vida religiosa que me recomiendas.

---No, no---gritó Alarcón con súbita idea conciliadora.---No hay la
incompatibilidad que crees, Santiuste. Eres místico, místico \emph{a
nativitate}\ldots{} Amor y misticismo van de la mano en el espíritu del
hombre. Yo veo en ti el apóstol que comienza su predicación elocuente
condenando el celibato, y estableciendo el amor de Dios\ldots{} el amor
divino sobre la base\ldots{}

---¿Del casamiento de los curas?

---No te rías, Juan. ¡Si estoy cansado de decírselo a Emilio!\ldots{}
«Emilio, tus discursos no son humanos; tu oratoria es el lenguaje de los
ángeles y el aliento del espíritu divino. Predica la fe, predica la paz,
el amor y la igualdad, y te llevarás detrás de ti a todas las gentes.
Todo el mundo americano será tuyo. Predica el nuevo verbo, que es la
Democracia, según Cristo, y la Democracia según Cristo no puede privar
al sacerdote de las dulzuras del amor humano\ldots» Con que ya ves,
Juan, si te resuelvo el problema. Cierto que serías un sacerdote
revolucionario; pero para eso has nacido tú, para las ideas que se
desbordan del vaso común en que todos bebemos, para las empresas
difíciles, no intentadas de otro alguno\ldots{} Apóstol de la paz, tu
camino es bien claro: fe, igualdad, amor.

\hypertarget{v-1}{%
\section{V}\label{v-1}}

Quedose meditabundo Santiuste, la barba en la palma de la mano, el mirar
fijo en las rayas de la mesa. Alarcón, retirado el cabo de vela ya
moribundo, erigió un cabo más grande, que casi era sargento, en la boca
de la botella. Quitose luego el ros; se lió un largo pañuelo en la
cabeza con muchas vueltas, quedando las orejas tapadas, y de un estuche
que a prevención tenía, sacó papeles, tintero y pluma. «Ha sonado la
hora---dijo a su amigo, poniéndole la mano en el hombro;---la hora del
descanso para ti; para mí, del cumplimiento del deber.»

---¿No duermes tú, Pedro?

---Échate en mi cama, Juan; arrópate bien y descansa, que buena falta te
hace. La paz poética duerme, la poesía militar vela. Tengo que escribir
esta noche mi carta de \emph{Un testigo}\ldots{}

---Pondrás en endechas de prosa las carnicerías de ayer y hoy\ldots{} Tú
eres el único para esto, Perico. Verdad que encuentras el lenguaje muy
acomodado a la expresión épica del valor castellano, y al impío
desprecio con que se mira a los pobres moros. Nuestra lengua es una hoja
bien afilada para cortar cabezas mahometanas, y un instrumento sonoro y
retumbante para dar al viento las fatuidades y jactancias
históricas\ldots{} Pero tú has descubierto y has empleado antes que
ningún escritor el arte de suavizar ese instrumento, tocándolo con
gracia inaudita. Tú sabes quitar a los sonidos épicos su vana hinchazón,
dándoles una elegancia incomparable, haciéndolos simpáticos a nuestros
oídos y acomodándolos a los nuevos modos de lenguaje\ldots{} Yo no podré
nunca imitarte en esto. He usado y abusado de la trompa, sin cuidarme de
atenuar la ronquera de su sonido, y ahora, en esta transformación de mis
ideas y en esta repugnancia de la épica militar, me he quedado sin
instrumento, pues aunque soplara la trompa, no sacaría de ella más que
lamentos desacordes. ¿Qué pito tocaré yo ahora? Esta es mi
confusión\ldots{} Entiendo que ya no hay pito ni flauta para mí.

Esto decía, despojándose para acostarse del ros, poncho y calzón
militar, que con tan poco garbo llevaba. Alarcón, poniendo sus cinco
sentidos en lo que escribía, sólo le contestó con medias palabras. Ambos
callaron. Cubierto ya de la manta y con más cansancio que sueño, Juan
contemplaba el rostro de su amigo, iluminado de lleno por la luz de la
próxima vela. Con las vueltas del pañuelo de colores en su cabeza,
Perico Alarcón era un perfecto agareno. Viéndole de perfil, la vivaz
mirada fija en el papel, ligeramente fruncido el ceño, apretando uno
contra otro los labios, Santiuste llegó a sentir la impresión de tener
delante a un vecino del Atlas. «Si no estuviera yo despierto---pensaba
parpadeando,---creería que uno de esos caballeros de zancas ágiles, de
airosa estampa y de rostro curtido, se había metido en esta tienda para
escribir en ella la relación épica de los combates, trabucando
irónicamente el patriotismo\ldots{} Así le sale historia de España lo
que debiera ser historia marroquí\ldots{} Perico, moro de Guadix, eres
un español al revés o un mahometano con bautismo\ldots{} Escribes a lo
castellano, y piensas y sientes a lo musulmán\ldots{} Musulmán
eres\ldots{} El cristiano soy yo.»

Se durmió repitiendo entre dientes \emph{el cristiano soy yo}. Toda la
noche anduvo esta afirmación revoloteando dentro del cerebro, como el
murciélago que al querer salir del recinto en que se ha refugiado, vuela
y choca en las paredes sin encontrar agujero que le conduzca al espacio
negro y libre. Paredes y bóvedas dolían cuando la idea chocaba en ellas,
buscando un escape que no podía encontrar\ldots{} Durmió al fin
Santiuste hasta muy entrada la mañana; Alarcón, que había trasnochado
por causa del trabajo, dejó el camastro a hora más avanzada. Las diez
serían cuando salió a despedir a su amigo. Ambos fueron a caballo hasta
el campamento del Segundo Cuerpo, donde se separaron, prometiéndose
pasar juntos la noche de San Silvestre, y celebrar con otra cenita el
paso del 59 al 60.

Pero en la mañana del 31, cuando fue Juan al Tercer Cuerpo en busca de
su amigo, enterose de que sufría una fuerte contusión, hallazgo de la
curiosidad en las refriegas del 30. No perdió Perico su buen humor por
aquel contratiempo, que si en un hombre de armas habría sido
insignificante, en el hombre de pluma era mucho más de lo que a sus
funciones correspondía. Un amigo de Alarcón, Carlos Navarro y Rodrigo,
escritor agregado al Cuartel General, le instaba para que se retirase a
Ceuta, donde el descanso y la esmerada asistencia le repondrían en un
periquete. No se avenía Pedro Antonio a separarse del Ejército, al cual
le unían su caldeada imaginación y su arrebato patriótico. Insistió
Navarro, y como al hablar de esto se fijara en el demacrado rostro de
Juan, que oía y callaba, le dijo: «También usted, Santiuste, mejor
estará en Ceuta que aquí\ldots{} Su cara me dice que no le prueban estos
aires guerreros\ldots» Replicó Juan que él no retrocedería, y que las
penalidades no le asustaban. Aunque sin entusiasmo militar, le fascinaba
el brío de tantos hombres tocados de la locura de hacerse daño. Quería
ver hasta dónde llegaba este delirio y la máxima extensión del mal que a
sí misma se causaba la humanidad, como si cifrara su orgullo en
desaparecer de la tierra\ldots{} Estas filosofías del trovador
desengañado provocaron a los tres a una enmarañada discusión de
principios y hechos. Como sucede siempre, de esta discusión no nació
ninguna luz, sino el propósito de comer juntos y pasar alegremente el
día. Nada digno de notarse ocurrió al expirar el año 59. Navarro se fue
al Cuartel General, y Alarcón y Santiuste quedaron en \emph{La
Concepción} aguardando los sucesos que en un gran saco repleto traía el
60, y que este empezó a lanzar al espacio histórico desde el primer día
de su existencia.

Sin esperar a que sonara la diana del 1.º de Enero, la Historia,
impaciente, empezó a moverse y hacer de las suyas, ganosa de marcar
aquel día con signo que lo distinguiera y perpetuara. Aún no apuntaba la
aurora, cuando don Juan Prim, designado para delantero y batidor en la
marcha de las tropas hacia Tetuán, pasó por la playa en aquella
dirección, llevando Ingenieros y Artillería, los cazadores de
\emph{Vergara}, el regimiento del \emph{Príncipe}, batallones de
\emph{Cuenca} y de \emph{Luchana}, con \emph{Húsares de la Princesa}. La
marcha era lenta y cuidadosa. Santiuste, que se había levantado a la
madrugada, bajó a la playa con Leoncio, y juntos siguieron a las tropas
de Prim. De una playa pasaban a otra, salvando un cerro divisorio, y así
dos o tres veces, costera y monte, hasta llegar a la vista de un valle
que recibió el nombre de \emph{Los Castillejos} por dos grupos de
carcomidas ruinas que en él no lejos del mar existían.

A una distancia que no podía llamarse prudente, vieron Leoncio y
Santiuste que los soldados de \emph{Vergara} y \emph{Príncipe}, mandados
por don Cándido Pieltaín, se posesionaron de las alturas próximas al
mar, echando de allí sin dificultad a los moros, y que \emph{Cuenca} se
encaramaba en un cerro, distante como dos tiros de fusil tierra adentro.
Por el camino que la vanguardia había recorrido desde el campo de
\emph{La Concepción}, vieron Leoncio y Juan que avanzaban más y más
tropas. Se las veía bordear la costa de playa en cerro, y en aquel sube
y baja con ondulaciones de culebra, la fila de hombres se perdía en los
descensos para reaparecer en las alturas.

Tanto Leoncio como Santiuste tenían amigos en la vanguardia mandada por
Prim. En \emph{Vergara} estaba el comandante Castillejo, de ambos
conocido; en \emph{Húsares de la Princesa} servía Vallabriga, a quien
Leoncio trataba en Madrid, y con varios oficiales del \emph{Príncipe}
había entablado relaciones Santiuste en el campamento del Otero. A uno
de estos oficiales, el teniente José Ferrer, gallego de buen humor, le
vio y habló repetidas veces, y se hicieron amigos, movidos quizás de la
disparidad de sus caracteres, porque todo lo que el gallego tenía de
bromista y gracioso, lo tenía el otro de taciturno y grave\ldots{}
Acercándose a los húsares, que formaban detrás del General, hablaron con
Vallabriga. Después fueron hacia donde estaba el \emph{Príncipe}. Ferrer
les dijo que no podían seguir las cosas tan por la buena. Como gallego
fino, desconfiaba de que durara el chiripón con que habían estrenado el
año, tomando aquellas posiciones como quien toma un cuarto
desalquilado\ldots{} Tanta felicidad era el mejor barrunto de un
disgusto muy gordo. Confirmó esta idea Leoncio, que con su prodigiosa
vista exploraba las próximas colinas y lejanos picachos, ya iluminados
por el sol naciente. «Por allá arriba me parece que distingo el nublado
de saltamontes\ldots{} ¡Jesús!, y por allí una nube, por más acá otra.
Se esconden en la montaña\ldots{} salen otra vez, vuelven a
esconderse\ldots{} Y aquí, por nuestro camino, viene el General en Jefe.
¿No veis su escolta? Ahora se para\ldots{} Aquí llega un ayudante con
órdenes.»

La orden era que bajase Prim al llano y se apoderara de un edificio al
modo de ermita llamado la \emph{Casa del Morabito}, y que la artillería
batiera los matorrales donde se ocultaban grandes masas de moros.
Sonaron las cornetas\ldots{} las filas de hombres y caballos se
estremecieron; aire de pelea circulaba por entre ellos, moviendo crines,
frunciendo bocas y apretando puños\ldots{} «¿Qué hacemos?» preguntó
Santiuste a su compañero. Y la respuesta fue: «Arrímate a mí; no temas
nada. Vamos a ver qué pasa. Sospecho que no será cosa mayor. Si disparo
mi carabina, tú la cargas, mientras yo hago fuego con mis pistolas. Si
fuese menester, dispararemos a un tiempo. Vamos detrás del
\emph{Príncipe}\ldots» Desaparecieron\ldots{} El torbellino los envolvió
en las ondulaciones de su cola: la cabeza era Prim.

La casa del condenado Morabito, ¡confúndale Alá!, quedó tomada en poco
tiempo. En razón inversa de la duración del combate estuvo su
intensidad. Las tropas, más que nunca despabiladas aquel día, pusieron
espacio cortísimo entre el pensamiento del jefe y el brazo que lo
ejecutaba: verdad que tuvieron el auxilio de las fuerzas sutiles de la
Marina, que en el momento más oportuno, aproximándose a la costa,
cañonearon de firme a la morería que bajaba de la montaña. Y entre
tanto, parte de la tripulación de los cuatro vapores y de los cañoneros
saltó a tierra, y carabina en mano se agregó a los soldados, ayudando a
poner en dispersión a las gavillas de infieles que defendían el valle de
los Castillejos. Pero con todo este buen resultado, más aparente que
real, ni Prim ni el General en Jefe, que junto a la casa del Morabito se
hallaba con su Estado Mayor, conceptuaron segura la posesión del valle,
porque en los manchones de arboleda se ocultaban aún centenares de
hombres, y otros no se retiraban de las alturas lejanas, como en espera
de fuerzas mayores para reconquistar lo perdido. Antes que O'Donnell se
lo mandara, Prim, al frente del \emph{Príncipe} y de \emph{Vergara},
corrió a desalojar el valle de aquellos inquilinos molestos que aún no
querían marcharse. Una, dos, tres cargas a la bayoneta con gradual
empuje, despejaron las alturas, y ya dictaba el General las órdenes para
que empezaran las obras de atrincheramiento del campo conquistado,
cuando por una hendidura de los montes de la izquierda brotó como un
chorro de infantes y jinetes árabes, y contra ellos cargaron dos
escuadrones de \emph{Húsares de la Princesa}, obligándoles a volver la
espalda.

Llevados de un ímpetu ardoroso, los húsares no se contentaron con
repeler a los musulmanes, sino que siguieron persiguiéndolos y
acuchillándolos por el mismo camino estrecho y tortuoso que llevaban en
su fuga; y corriendo tras ellos, en una de las revueltas vieron el campo
moro asentado entre cerros muy altos, blancas tiendas cónicas, y en
derredor de ellas gran gentío de peones y caballeros. Sin encomendarse a
Dios ni al diablo, los de la \emph{Princesa} seguían adelante con
guerrero furor, metiéndose de lleno en la trampa que los taimados hijos
de Mahoma les habían armado. Tras de los escuadrones lanzados a esta
temeraria aventura, acudieron los demás, anhelosos de auxiliar a sus
compañeros y de salvarlos o perecer con ellos\ldots{} Esta singular
hazaña de los húsares fue de las más audaces que en guerras humanas se
han visto; acto de sublime demencia, en que el valor personal, acumulado
en un punto por la temeridad de unos cuantos hombres, altera la
normalidad de los principios de la táctica y descompone toda la lógica
militar. Los intrépidos jinetes que volaron en auxilio de los primeros
que habían caído en la celada, infundieron a estos los alientos
necesarios para que, reunidos todos, se desliaran del inmenso remolino
de bárbaros que les envolvió por todas partes. Combate fue cuerpo a
cuerpo, con eléctrica rapidez, a usanza de griegos y romanos, dando al
heroísmo toda la tensión posible en menos que se piensa y que se dice, y
sosteniéndola sin dar espacio ni tiempo al enemigo para poner una pausa
en su estupor y recobrarse del pánico.

\hypertarget{vi-1}{%
\section{VI}\label{vi-1}}

Los que vieron partir a los escuadrones para aquel lance de inaudito
arrojo, creyeron que no volverían. Volvieron, sí, enteros, trayendo su
bandera y la que el cabo Mur arrebató al Imperio marroquí con increíble
tirón de una mano de gigante; volvieron con orden, sin dejarse allá
ningún prisionero, con los dos comandantes de los primeros escuadrones,
Aldama y Fuente Pelayo, gravemente heridos, y pérdida de dos oficiales y
unos veinte soldados\ldots{}

Poco después de la vuelta de los húsares, a quienes todos contaban ya en
la eternidad, el pensamiento de O'Donnell era este: «Mantener las
posiciones conquistadas fortificándolas convenientemente, y no avanzar
ni un paso más hasta que no sepamos qué fuerzas de moros, todavía
intactas, se esconden en la encañada del río de los Castillejos y de su
afluente, así como en los demás recodos de esas montañas.» Esta
disposición revelaba al General en Jefe, que, sin perder de vista sus
deberes ni su responsabilidad, no quería fatigar a sus tropas, ni
lanzarlas a combates duros sin que antes se alimentaran bien\ldots{} Un
ayudante de O'Donnell llevó estas órdenes al General Prim; un ayudante
de Prim llevó a O'Donnell este recadito: «Que si me manda un par de
batallones y dirige una brigada por la izquierda, me apoderaré hoy del
campamento enemigo.»

Es fama que don Leopoldo puso mal gesto al oír la petición del General
de su Vanguardia. «¿Qué contesto, mi General?» le preguntó Gaminde,
ayudante de Prim.

---Dígale usted que allá voy yo.

En un instante de ruidosa confusión, Leoncio perdió de vista a su
compañero. Habían seguido los pasos de los batallones del
\emph{Príncipe}; vieron de cerca los diferentes ataques a la bayoneta
que \emph{Vergara} y \emph{Luchana} dieron a los moros; corrieron luego
a ver si volvían o no los húsares que se metieron por la angostura, y en
esto, Santiuste desapareció. ¿Había escapado hacia lugar seguro,
temeroso de que la curiosidad le costara la vida? Buscándole y
llamándole a voces, bajó Leoncio hasta la \emph{Casa del Morabito}, y a
poco de estar allí vio a O'Donnell partir a la carrera con su Estado
Mayor hacia el punto en que Prim activaba el atrincheramiento de las
posiciones conquistadas. Fue cuando O'Donnell dijo: «allá voy yo.»

Echó a correr Leoncio hacia donde la curiosidad y el patriotismo le
llamaban; de lejos vio a O'Donnell inspeccionando con Prim los trabajos
de fortificación. Sin duda no se pasaría de allí, ni era prudente
meterse en mayores aventuras. Avanzaba el día, y las tropas estaban sin
comer, rendidas de cansancio. ¿Y quién aseguraba que los malditos
muslimes no tenían encajonadas detrás de los montes fuerzas mucho más
grandes que las presentadas durante la mañana? Porque ya era evidente
que su falta de ciencia militar la suplían con la astucia y el arte de
las sorpresas\ldots{} Esto pensaba Leoncio Ansúrez, minúsculo táctico y
estratégico de afición, cuando un rumor venido de la sierra le dejó
suspenso y aterrado. Era como el silbo de un huracán que de improviso se
desencadenara en las alturas. Por todas las que rodean el valle de los
Castillejos aparecían moros formando nube: sus voces desconcertadas, que
en nuestra lengua conservan el nombre de \emph{al-garabía}, eran de
lejos como el zumbido de infinitas abejas abandonando infinitos
colmenares\ldots{} Todo el Ejército vio con mudo estupor el tempestuoso
nublado.

Razón tenía O'Donnell al creer que el enemigo no había presentado en los
combates de la mañana más que una parte mínima de sus muchedumbres a pie
y a caballo. Contra aquel aluvión se prepararon a luchar los fatigados y
hambrientos hombres de \emph{Luchana}, \emph{Vergara} y el
\emph{Príncipe}, y los quebrantados \emph{Húsares de la Princesa}. De su
flaqueza sacaban alientos, y de su amor a la bandera el coraje preciso
para no permitir que el enemigo se la llevara. En momentos de tanto
ardor y peligro, muchos habían de morir, hasta que la suerte decidiera
quién salía vencedor. Era forzoso matar todo lo que se cogía por
delante, con gran riesgo de la propia pelleja; retroceder era condenarse
a muerte segura. Cargó Pieltaín con los del \emph{Príncipe}, cargaron
\emph{Vergara} y \emph{Cuenca}. Las posiciones más altas que ocupaban
los españoles hubieron de ser abandonadas. En la segunda posición hizo
Prim esfuerzos sobrehumanos para sostenerse, y lo consiguió gracias a
dos batallones de \emph{Córdoba} (del Segundo Cuerpo) que llegaron como
enviados por la Providencia de los españoles. Pero la Providencia de
Mahoma desgajó de los montes nuevas masas de tiradores árabes, con lo
que aumentaba su fuerza el enemigo, en proporción mayor de lo que creía
la de los nuestros. Las dos Providencias, la musulmana y la cristiana,
redoblaban su ira, y los combatientes se enzarzaban con la ferocidad de
las guerras primitivas.

No sabiendo Prim de dónde sacar más fuerzas con que contener la
creciente avalancha, echó mano de la artillería de a pie, mandándola
desplegar en orden abierto, táctica bien distinta de la de su arma. Los
artilleros fueron a donde se les mandaba, batiéndose como la infantería
ligera. Mas no haciendo nada de provecho, tuvieron que retroceder,
buscando maquinalmente el orden cerrado para el cual se les había
instruido. Su Coronel, Berroeta, viéndose obligado a perder terreno,
maldecía la hora en que nació\ldots{} En tanto, Prim poníase al frente
de un batallón de \emph{Córdoba}, Gaminde al frente del otro, y mandando
a los soldados que soltaran las mochilas para ir más ligeros, avanzaron
con terrible decisión en busca de la muerte o la victoria. Ronco estaba
Prim de las voces que les daba, inflamando su patriotismo con el nombre
mágico de la Reina cien veces pronunciado. Pero no había nombres de
Reinas ni invocaciones patrióticas que multiplicaran a los hombres, y
sólo multiplicándose y convirtiéndose cada uno en seis, podían romper
los apretados haces de moros ensoberbecidos, rugientes, feroces. Un
momento más sin que se efectuara el milagro de la multiplicación de
hombres, y todo se perdía sin remedio.

El suelo estaba lleno de cadáveres, el aire de un alarido en que las dos
lenguas, árabe y española, juntaban sus maldiciones y los acentos de la
fiereza humana, lenguaje animal anterior al de los hombres. Retrocedían
los de \emph{Córdoba}, empujados por los moros, y casi tocaban ya al
sitio en que habían soltado sus mochilas\ldots{} Ya no había más salida
de aquel laberinto, ni más remedio del desastre, que no prodigio del
Cielo, o de los hombres por divina inspiración. Prim, lívido, vibrando
de pies a cabeza, imagen de la desesperación altanera que no admite la
derrota y borra la idea de muerte del espacio mental en que se pintan
las ideas, arengó por milésima vez a su gente. Gaminde había
desenfundado la bandera de \emph{Córdoba}, para que, desplegada, fueran
sus vivos colores como latigazo en la retina de los soldados, casi
ciegos ya del humo, atontados por la fatiga, y a punto de sentir apurada
y nula su brutal fiereza. Prim empuñó el mástil de la bandera; al viento
dio la tela, y con la tela unas palabras roncas, ásperas, como si las
soltara con un desgarrón de su laringe\ldots{} Más por la expresión que
por el sonido las entendieron los que le rodeaban\ldots{} Coger la
bandera, echar la tremenda invocación, hincar espuelas al caballo y
saltar este sobre el tropel de moros, fue todo un instante\ldots{}

Del lado allá de este instante, que era como vértice en los órdenes del
tiempo, estaba el milagro. El milagro fue que los hombres se
multiplicaron. Ya no se vio más que el cruzarse de bayonetas y
yataganes, el brillar de los ojos como brasas, el hervor de un mar en
que sobresalían miles de brazos agitando las armas. La masa española se
incrustó en la mora. El fiero caballo del General, aunque herido,
descargaba sus patas delanteras sobre cuantos cráneos a su alcance
cogía. Las bayonetas segaban los haces enemigos. Morazos de tremenda
estatura caían hacia atrás, elevando al cielo los remos inferiores como
si fueran brazos; españoles caían también, de bruces, heridos de muerte,
agujereados vientre y pecho. Otros pasaban sobre ellos\ldots{} seguían
creciendo y multiplicándose, a cada momento más esforzados, con mayor
desprecio de la vida\ldots{} El General, siempre delante, echando rayos
de su boca, a todos deslumbraba con su locura increíble.

Sin duda, la figura de Prim, arrojándose a la muerte y ofreciéndose con
cierta voluptuosidad de sacrificio heroico a las cuchillas y a las balas
enemigas, debió de producir en el ánimo de los moros una fascinación
inaudita\ldots{} Sobrecogidos los que recibieron terribles golpes;
desalentados los que veían la inutilidad de su bravura, corrieron todos
en querencia de lugares seguros\ldots{} Les llamaba el interior plácido
de su país\ldots{} Iban a sus aduares, a sus casas, a sus mezquitas,
bien como los animales acosados que siempre buscan la orientación de sus
viviendas. En bandadas huyeron. Las posiciones quedaron rescatadas; el
suelo limpio de moros vivos, no de muertos, pues tantos eran que daba
horror ver el campo. No pocos españoles yacían entre los despojos de tan
horrible matanza. Las dos patrias, las dos religiones, semejantes, en
aquel empeño de honor, a las antiguas divinidades iracundas que no se
aplacaban sino con holocaustos de sangre, ya podían estar satisfechas. Y
los muertos, el sin fin de hombres sacrificados en el ara sacrosanta,
¿qué pensarían de aquel furor con que los degollaban como carneros para
que desarrugase el ceño la diosa implacable?\ldots{} ¿Será verdad que la
diosa, cuando bebe mucha sangre, se pone muy contenta, y en su seno
acoge con amor a las innumerables víctimas de la guerra? Así por lo
menos se dice en todas las odas que consagran los poetas a cantar
batallas\ldots{}

Y así pensaba el buen Santiuste cuando echó la vista al terreno de las
victoriosas cargas, iniciadas por Prim. Sintió escalofrío ante el
espectáculo de tantos muertos caídos en trágicas posturas, y aunque por
un momento le movió la curiosidad de ver si estaban en aquellos montones
sus amigos Leoncio, Vallabriga, o el galleguito Pepe Ferrer, no se
atrevió a meterse entre los cadáveres: el miedo de encontrar a sus
amigos le sobrecogía más que le interesaba el deseo de saber su suerte.
En lastimoso estado de cuerpo y espíritu, tomó la dirección de la
\emph{Casa del Morabito}, adonde iban todos los que no quedaban en el
cuidado y defensa de las trincheras. El molimiento de sus huesos era
tal, que andar no podía con el garbo propio del uniforme. Todo había
sido contratiempos y desdichas para el pobre trovador desde que la
casualidad le separó de su amigo Leoncio. Por dos veces fue atropellado
por los soldados del \emph{Príncipe} y \emph{Vergara} cuando les hizo
retroceder a sus posiciones el empuje de los moros. Cara le costó su
curiosidad al buen poeta de la Paz, porque en la segunda de aquellas
caídas, centenares, a su parecer millares de pies, pasaron por encima de
su asendereado cuerpo. ¡Cómo quedarían los huesos, y sobre los huesos la
piel, y sobre la piel el uniforme, con estos pisotones y carreras! El
poncho y ros quedaron manchados de fango revuelto con sangre. Cuando le
vieron levantarse del suelo, alguien creyó que era un cadáver que
resucitaba para espanto de los vivos.

A estos desperfectos exteriores se unieron, para mayor suplicio de
Santiuste, el hambre que demacraba su rostro y el frío que mantenía sus
manos en continuo temblor\ldots{} Concluía de anonadarle el no encontrar
entre tanta gente un rostro conocido, y su desairado vagar por el campo,
donde no se batía ni prestaba ningún servicio\ldots{}

Anochecía. Las sombras nocturnas, indiferentes a los actos heroicos de
aquel día, se dejaban caer amorosas sobre los despojos trágicos de las
batallas\ldots{} Camino del \emph{Morabito} iba Santiuste, cuando vio
una fila de soldados conductores de camillas. La procesión de heridos no
tenía fin, y avanzaba con esa prisa lúgubre de los entierros que llegan
tarde al camposanto. Quiso Juan ser útil, y se brindó a relevar a uno de
los hombres que llevaban camillas. Pero su oferta no fue
admitida\ldots{} Más adelante vio que un camillero, rendido de inanición
y cansancio, no podía con su cuerpo y menos con el del herido. Al punto
acudió Juan a sustituirle, y echando mano a las parihuelas, arreó camino
abajo gozoso del humanitario servicio que prestaba. No había andado
veinte pasos, cuando el herido que transportaba se incorporó en la
camilla, y con una varita que esgrimía en la mano derecha, tocó a Juan
en el hombro diciéndole: «Arrea, bruto; arrea pronto, que me estoy
desangrando.» Sin para volviose Santiuste a ver quién hablaba, y
reconoció a Leoncio Ansúrez.

\hypertarget{vii-1}{%
\section{VII}\label{vii-1}}

«¿Es grave tu herida, Leoncio?»

---¿Yo qué sé? Una bala me pasó el muslo, y un tajo de yatagán me lo
acabó de arreglar\ldots{} Ahora me sale mucha sangre. Si no me curan
pronto, no sé qué será de mí. Arrea, Juan.

Juan y el zaguero avivaron el paso, y Leoncio calló. Pasado un buen
rato, dejose oír de nuevo su extenuada voz: «Juan, ¿viste la hombrada de
Prim? ¡Qué tío más valiente! Creí que a él y a todos nos acababan esos
perros.»

---Vi la hombrada, Leoncio\ldots{} la vi y creí que era sueño\ldots{}
También te digo que si no llega en aquel momento por la derecha el
General Zabala con cuatro batallones, y sacude a los moros como les
sacudió, la hazaña de Prim quizás no habría sido más que un heroísmo
inútil, y con hablar de \emph{muerte gloriosa}, ya estaba el asunto
despachado\ldots{} Yo pongo en su lugar de honor a mi General, al
General del Segundo Cuerpo, don Juan Zabala, gran soldado, de valor
sereno, de vista penetrante para la oportunidad. Si no es por él,
Leoncio, todo se pierde\ldots{} ¡Y cuántos muertos, Dios mío! De
infieles y cristianos ha quedado el campo lleno. Quítale a la guerra el
poquito interés que le da el ser arte y el ser ciencia, y no queda más
que un pasatiempo de caníbales\ldots{} ¿Qué dices?\ldots{} ¿Por qué
callas?

---Con cada palabra que echo de la boca, se me va un gran pedazo de
vida\ldots{} Estoy admirado\ldots{} de la sangre que tenemos en el
cuerpo\ldots{} porque con salirme tanta, todavía queda sangre dentro.
Arrea, Juan.

Llegaron por fin a la tienda-hospital; mas era tanta la afluencia de
heridos, que los médicos no tenían manos para curarlos. Mientras los
propios soldados aplicaban a Leoncio un vendaje provisional para
contener la hemorragia, Santiuste consolaba a su amigo con frases
afectuosas y esperanzas de pronta curación, y viéndole más animado con
el vino y pan que le dieron, se permitió reprenderle en esta forma:
«Esto te pasa por meterte a farolear, Leoncio, pues tú no has venido
aquí a combatir, sino a componer las armas de los que combaten\ldots{}
Lo que hoy te ha pasado te servirá de escarmiento\ldots{} y no volverás
a pintar el diablo en la pared, que maldita gracia tendrá que dejes
viuda a \emph{Mita} y huérfano a tu hijo.»

El recuerdo de su cara familia ausente afligió a Leoncio: algunas
lágrimas mojaron su rostro antes de la cura. En esta desahogó su dolor
con gritos más que con llanto, y estuvo muy firme. Allí quedó con la
pierna sepultada en parches y vendas, condenado a inmovilidad absoluta
durante luengos días. «Mira, Juan, vas a hacerme un favor---dijo a su
amigo:---vete por ahí, y búscame a Señá Ignacia\ldots{} ¿No sabes?, es
la cantinera del Tercer Cuerpo; una mujer muy buena y muy socorrida para
todo. Le dices que estoy con una pata hecha cisco; que venga a verme, y
me traiga de aquel aguardiente de caña que alegra y cría sangre. Después
de la soba que me ha dado el físico, tengo una sed horrible, y necesito
del aguardiente para que el agua no me encharque\ldots{} Corre, hijo, y
tráemela prontito.»

Corrió Juan por las calles del campamento, y aunque no tardó en
encontrar a la hombruna y bondadosa Ignacia, esta, con muy buena
voluntad, pero sin poder zafarse del sinnúmero de parroquianos que la
asediaban, no acudió a Leoncio hasta mucho después de recibido el
encargo. Vagando en acecho de la Ignacia, Santiuste vio al Coronel del
\emph{Príncipe}, don Cándido Pieltaín, en la entrada de una tienda, con
el brazo derecho en cabestrillo, fumando, en conversación con dos o tres
oficiales. Más allá, otra tienda, en cuya puerta se agolpaban curiosos
atrajo su atención: el bloque de gente, en su mayor parte artilleros,
que cerraba la entrada, no le permitió ver más que las botas de un
hombre yacente, al parecer muerto\ldots{} Alargando más el hocico, vio
el cuerpo hasta la cintura\ldots{} le alumbraban más velas de las que
para el uso común se encendían en el campamento. Era el Coronel don
Francisco Barroeta, jefe de la Artillería que se batió aquella tarde en
orden abierto. Tal ira y turbación le causó el ver a sus valientes
artilleros retroceder una y otra vez ante el ataque de los moros, que la
serenidad no volvió a su ánimo, y al retirarse a la tienda se pegó un
tiro\ldots{} Exaltación insana del sentimiento del honor militar le
precipitó a la muerte. ¡Qué desdicha! Oyendo contar el lance, Santiuste
lloraba, maldecía con toda su alma las brutales guerras, y las vanas
historias que de ellas se escriben para inducir a los hombres a poner
sus preciosas vidas en un punto caballeresco\ldots{} Cuando al Hospital
de sangre volvía, ya capturada la cantinera, llegaban a su oído aquí y
allá los comentarios del gran suceso reciente, burbujas de la acción
heroica, que aún hervía en todos los corazones\ldots{} ¡Qué oportunidad
la de Zabala!\ldots{} De los veinte hombres que formaban la escolta de
infantería de Prim, no habían quedado más que seis\ldots{} ¡Ah, España,
cuánto sacrificio por ti!\ldots{}

Con la excelente cura que se le hizo, y el remedio de aguardiente de
caña sobre la gran cantidad de agua que había bebido, pasó regularmente
la noche el buen Leoncio. Por indicación apremiante del herido, Ignacia
le dejó media botella del bendito licor, y Juan, que no se había de
separar de él, quedó en darle las tomas con la periodicidad conveniente.
Horrible fue la noche en la lúgubre tienda: de los ocho heridos graves
que había en ella, murieron tres, y dos, según opinión de los médicos,
no pasarían de la mañana siguiente. El castrense que allí prestaba
servicio fue relevado por don Toribio Godino, a quien su amigo
Santiuste, por confortarle el estómago desmayado, obsequió con una
copita del bálsamo de caña. «No sabes, hijo mío---le dijo el
cura,---cuánto te agradezco este precioso sostén de las facultades. Con
el trabajo de esta noche\ldots{} y cuenta que ya he despachado para el
Purgatorio a más de cincuenta\ldots{} con tanto ajetreo de Sacramentos,
sin parar, sin parar, a este, al otro, al de más allá, hasta las
palabras rituales se me helaban en la boca y no querían salir\ldots{}
Dios te lo premie, hijo\ldots{} y te lo aumente.»

Ya la luz del alba clareaba en la entrada de la tienda, cuando Leoncio,
que había caído en hondo letargo, despertó con cierta inquietud llamando
a voces a su amigo. «Aquí estoy---dijo incorporándose Santiuste, que
también descabezaba un sueño.---¿Te sientes mal? ¿Te molesta la herida?»

---No es la herida: es una idea, una idea, Juan, que me atormenta y no
me deja descansar\ldots{}

---Dímela\ldots{} Será una idea de las que trae la fiebre\ldots{} y las
ideas de la fiebre son locas\ldots{} No hagas caso.

---Arrímate más a mí, Juan\ldots{} más. Que no oiga nadie lo que tengo
que decirte.

---Nadie lo oirá, Leoncio\ldots{} Los más próximos están muertos, y los
más lejanos duermen.

---Pues lo que me atormenta\ldots{} a ti, a ti solo te lo digo\ldots{}
lo que me atormenta es que hoy\ldots{} poco antes de que Prim cogiera la
bandera, cuando los moros venían hacia acá y nos arrollaban\ldots{} vi a
mi hermano Gonzalo\ldots{} No se me despintó\ldots{} era él\ldots{}

---Tu hermano moro\ldots{} el que se hizo moro\ldots{} ya sé.

---Le vi primero vivo entre los que mandaban\ldots{} A caballo venía muy
arrogante, con un albornoz de tela vaporosa\ldots{} Debajo llevaba un
traje de seda verde\ldots{} Turbante blanco\ldots{} Era él, te
digo\ldots{} No sé el tiempo que pasó hasta que volví a verle. Fue antes
de caer yo herido, en el momento más terrible de la carga de los de
\emph{Córdoba}\ldots{} Le vi muerto, la cabeza partida por un tremendo
sablazo; el caballo muerto también y todavía pataleando\ldots{} Mi
hermano tenía los ojos vidriados, fijos; la boca muy abierta y rasgada,
mostrando todos los dientes, blancos\ldots{} una boca de risa que daba
mucho miedo\ldots{} El albornoz se había desgarrado, y era todo hilachas
manchadas de sangre y barro. Se veía el pecho ensangrentado\ldots{}
ensangrentado el magnífico traje verde\ldots{}

---¿No sería azul, Leoncio?\ldots{} Recuerda bien. En esos momentos de
emoción trágica, es cosa muy fácil confundir los colores.

---No, Juan; era verde\ldots{}

---Pues yo sostengo que era azul, Leoncio---dijo Santiuste con pleno
convencimiento de lo que decía, poniendo toda su atención en aquel
asunto.

No puede omitir el historiador que después de media noche, sintiéndose
el buen poeta de la Paz muy desconsolado del estómago, y además falto de
calor en todo su cuerpo, probó el precioso licor de Ignacia. Tan bien le
supo la media copita, y tan eficaz reparo notó en sus entrañas después
de beber, que repitió la medicación dos o tres veces en el curso de la
madrugada, disputándola por droga de maravillosos resultados. «Pues te
digo que azul y no verde, y en ello insisto---prosiguió Santiuste
bajando más la voz,---porque yo también he visto a tu hermano\ldots{} Le
vi, como tú, vivo y muerto, y toda la descripción que me has hecho de su
figura y arreo concuerda con lo que yo vi, menos lo del traje verde.»

---Pues sería, como dices, azul; que nada de particular tiene que,
trastornadas mi vista y mi cabeza, trabucase yo los colores\ldots{} Pero
dime, Juan: ¿cómo conociste a Gonzalo si no le has visto nunca?

---¡Ah!\ldots{} yo me entiendo\ldots{} Respóndeme: ¿se parecen tu
hermano Gonzalo y tu hermana Lucila?

---Todo lo que pueden parecerse un hombre con barbas y una mujer sin
ellas. Cuando Gonzalo era mozo, parecía mi hermana vestida de hombre.

---Los ojos son los mismos, ¿verdad?

---Tan iguales, que creíamos que se los prestaban el uno al otro para
mirar\ldots{}

---Y la carita hermosa de tu sobrinillo Vicente ¿no es igual a la de su
tío Gonzalo?

---Tan es la misma, que, según mi padre, Vicentillo es Gonzalo que ha
vuelto a nacer.

---Pues figúrate ahora si me habrá sido fácil conocerle, y si habré
tenido un sentimiento grandísimo al verle cadáver\ldots{} No olvidaré
nunca aquel rostro noble, los ojos vidriados, la carcajada
esculpida\ldots{} Ha muerto por su nueva patria\ldots{}

Después de una pausa en que cada cual sondeaba sus propios sentimientos,
Leoncio suspiró y dijo a su amigo: «¿Crees tú, Juan, que mi hermano
estará en el paraíso de Mahoma, gozando de Alá?»

---No sé, no sé---respondió Juan, poniendo una cara enteramente
estúpida;---pero yo te aseguro que si no en ese paraíso, en algún otro
paraíso tiene que estar.

Pasado más tiempo que el de la anterior pausa, el herido cambió de un
salto la conversación diciendo: «Veo la botella caída. Es que se nos ha
concluido el \emph{Sanalotodo}\ldots{} En cuanto aclare bien el día, te
vas a buscar a Ignacia. Ten cuidado, Juan, y no compres a ese otro
cantinero que llaman \emph{Borrascas}\ldots{} Todo lo que ese vende es
veneno\ldots{} Créeme a mí: como mujer de conciencia y que sepa mirar
por el Ejército español, no hay otra Ignacia.»

El día se presentaba espléndido. Brillaba el sol alegrando los ánimos.
Fácilmente se olvidaban los horrores del trágico día de los Castillejos,
para no pensar más que en la indudable gloria de la jornada. Ocho mil
hombres escasos habían luchado contra más de treinta mil. Aprovechando
el buen tiempo, seguiría el Ejército su marcha hacia Tetuán\ldots{} Ya
sabían los moros cuán caro les costaba entorpecer el camino\ldots{}
Aunque la herida de Leoncio no era grave ni exigía la intervención
quirúrgica, se pensó en mandarle a Ceuta en el primer convoy de heridos
que saliese, lo que supo muy mal al armero, pues abandonar al Ejército
era su mayor pena. Santiuste trató de ver a Pedro Antonio el día 2; pero
al dirigirse al campamento de la \emph{Concepción}, encontró este
levantado. El Tercer Cuerpo marchaba de vanguardia por el camino de
Tetuán. Alarcón había partido para Ceuta. De otra novedad importante
tuvo noticia Juan aquella tarde, y era que el General Zabala, Jefe del
Segundo Cuerpo, estaba enfermo. Al regresar a su tienda en la noche del
memorable día de los Castillejos, su cansancio era tan grande, que se
arrojó en la cama de campaña sin quitarse la ropa mojada del rocío. A la
siguiente mañana despertó con todo el lado derecho paralizado.
Consecuencia de este percance fue que el Segundo Cuerpo quedó a las
órdenes de Prim. Todo esto lo supo Juan por su amigo don Toribio, que
acabó diciéndole: «Bueno es el General que ahora nos manda; pero yo me
siento huérfano, porque en todo el Ejército y fuera de él no hay para mí
otro don Juan Zabala\ldots»

Al regresar a los Castillejos encontró Santiuste a su amigo Ferrer,
Teniente del \emph{Príncipe}, en un corro de oficiales que rodeaba a la
sin par Ignacia. Esta, sin cesar en su ordinario despacho de bebidas,
vendía castañas recién llegadas de Ceuta, y cigarros puros de los
llamados \emph{de dos manos}, porque las dos eran necesarias para
fumarlos: una para tener el cigarro, y otra para el fósforo. Abrazó Juan
a su amigo con verdadera efusión, pues le creía muerto en los terribles
combates del día 1.º «Yo también me tuve por muerto---respondió el
galleguito,---y no se me quitó de la cabeza la idea de estar en el otro
mundo hasta que vi que vivaqueábamos en las posiciones\ldots{} y hasta
que vi venir \emph{el} pote\ldots{} calentito\ldots{} ¡Batallas\ldots{}
potes\ldots{} la muerte, la vida!\ldots{} Esta que llevamos no es para
llegar a viejos.» Don Toribio se entristeció después con el relato de
los innumerables responsos que había echado sobre tantos y tantos
muertos. La tierra estaba henchida, harta: se indigestaba de cadáveres
cristianos y moros. El Infierno y el Cielo recogerían las almas\ldots{}
«Eso\ldots{} allá Dios\ldots{} No sabemos, querido Juan, no
sabemos\ldots{} Me preguntas por el \emph{Dios de las batallas}. Ya te
he dicho que no sé dónde está ese señor\ldots{} no le conozco. ¿Y ese
\emph{Allah}, qué pito toca? Para mí, ninguno. Yo mando a todos mis
muertos a donde me ordena mi ritual\ldots{} Cada cual lleva su pase; van
bien encomendados a la Misericordia del que hizo los Cielos y la Tierra.
Para mí que la encuentran\ldots»

\hypertarget{viii}{%
\section{VIII}\label{viii}}

Ya el Tercer Cuerpo acampaba en el cerro de \emph{La Condesa}, como a
una legua del valle de los Castillejos; ya se había recorrido más de la
mitad del camino de Ceuta al valle de Tetuán; los africanos, no
repuestos aún del susto que les dieron Prim, Zabala y O'Donnell el 1.º
de Enero, atacaban tímidamente y en corto número, asomándose por los
montes y volviéndose a meter en ellos. Guardaban sin duda sus ardides
astutos para \emph{Monte Negrón}, fortaleza natural de pura roca, con
picachos y cavernas de inestimable valor en las emboscadas y sorpresas.
¡Adelante, adelante! España, que tan formidables obstáculos había
vencido, no se detendría ya por un monte más o un monte menos
interpuesto en su camino\ldots{} El avance del Ejército traería la
forzosa incomunicación terrestre con Ceuta. Una escuadrilla mercante y
algunas goletas de guerra llevarían las provisiones a puntos abordables
de la costa.

Confiaba Leoncio en que su pierna se portaría como una pierna decente,
no poniéndole en el duro trance de ser retirado de la campaña.
Santiuste, que desde el 3 empezó a sufrir calenturas, se avino a ser
transportado con su amigo en la impedimenta. En las horas que la fiebre
le acometía, su espíritu se aplanaba en una indiferencia perezosa y
lúgubre, y lo mismo le importaba separarse del Ejército que permanecer
en él. Considerábase como un fardo inútil, y ni aun se sentía con
alientos para escribir a sus amigos y cumplir el único deber que al
África le llevara. Perezoso era de la acción muscular, no de la mental,
ni tampoco de la palabra, pues llevado con su amigo en lomo de mulas por
ásperos caminos, discurría con extraordinaria fecundidad, y no daba paz
a la lengua para sacar al exterior sus alambicados pensamientos. Apóstol
convencido de la Paz, todo lo de la guerra le tenía ya sin
cuidado\ldots{} Oían por el camino tiros lejanos. ¿Qué pasaba? Que el
General Ros rechazaba gallardamente al Moro en las alturas de \emph{La
Condesa}; que el General García, Jefe de Estado Mayor, hacía un
reconocimiento en el imponente paso de Monte Negrón\ldots{} Nada de esto
le interesaba, y por decirlo con honrada ingenuidad, tenía con su amigo
Leoncio fuertes peloteras.

El radical cambiazo en los sentimientos y en las ideas de Santiuste,
llevándole del nacionalismo épico a las amplias miras humanas, no secó
en él la vena rica de la elocuencia. Y aunque esta y los tópicos de
patriotismo parecían de igual naturaleza y tan trabados entre sí que no
podían separarse, ello es que las ideas cambiaron sin que la expresión
de las nuevas fuera menos hermosa. Elocuente era Santiuste aun después
de arrancar de su cerebro lo que él llamó después talco y lentejuelas
históricas; elocuente al desechar ese tono colérico que informa las
manifestaciones del patriotismo agudo, y al adoptar los tonos tranquilos
del que excluye en absoluto de su doctrina la muerte airada de nuestros
semejantes.

Tanto como en él aumentaba la pereza de escribir, acrecía la facultad
oratoria. Escribiendo no esperaba convencer a nadie; hablando, a todo el
mundo convencería. ¡Ah, si él pudiera explicar verbalmente a Lucila su
metamorfosis, mostrarle su corazón inflamado en el amor de la paz,
desplegar ante ella los mismos razonamientos que él se había hecho para
llegar a su presente estado mental, cuán fácilmente la persuadiría!
Porque Lucila y él, sin haberse declarado su conformidad y semejanza,
eran dos almas parejas y armónicas, con un solo sentimiento para las
dos. Pensando en esto, el pobre poeta se lamentaba de su incapacidad
para convencer a su amiga por escrito\ldots{} Además, para escribirle de
estas cosas necesitaba una confianza que aún no tenía; ponerse en
concierto de amor, declarando él el suyo, y esto no debía intentarlo
mientras no estuviese más avanzada la viudez de la dama. Aún no era
tiempo de romper la delicada etiqueta con que se trataban. Por el
momento bastaba con graciosas insinuaciones que la llevaran gradualmente
a conocer la verdad. Esto lo hacía en todas sus cartas, meditando mucho
lo que decía para que el agudo Vicentito, picado de curiosidad, no
hiciese a su madre preguntas que habían de turbarla. Una sola vez había
Lucila contestado a las cartas del trovador, y se mostraba muy
afectuosa, interesada vivamente en la salud del fiel amigo. Y entre
otras expresiones de ternura disimulada, le decía: «Por Dios, Juan: no
se ponga en ningún sitio donde corra peligro, que su vida es más
preciosa de lo que usted cree. Usted no es militar, sino cantor de las
glorias militares; y si en la guerra no puede ver estas para cantarlas,
cántelas por lo que le cuenten; y en último caso, mande las glorias a
paseo, que antes que ellas es usted y el deber en que está de volver acá
sano y salvo.»

Esto le decía la hija de Ansúrez, ¡y con cien mil de a caballo (como
decía Alarcón), que era bastante expresivo! ¿Cómo dudar que en esta
frase se dejaba caer del lado de la paz, y que ponía las glorias en el
secundario lugar que les corresponde, siempre más bajas que la vida
humana? Cuando Santiuste se veía solo y abrumado de tristeza, no tenía
más consuelo que pensar en aquel ídolo distante, y anticipar con la
imaginación los hermosos conceptos con que, después de conquistarla para
su amor, la conquistara para sus ideas.

«¿No escribes, Juan?---le dijo Leoncio una tarde, cuando llegaron al
descanso de la tienda tras un molesto viaje.---Te recuerdo tus
obligaciones, porque veo que te descuidas en ellas. La goleta
\emph{Rosalía}, que pronto llegará con víveres, llevará tus cartas y la
mía, porque yo escribiré también. Cuidado, Juan: si en tu carta me
nombras, di lo mismo que yo: que estoy bueno, y que no he tenido ni un
rasguño. ¡Buen susto se llevaría mi pobre \emph{Mita} si dijeses otra
cosa!»

En esto franquearon las tropas, sin ningún tropiezo, el desfiladero
entre Monte Negrón y el mar, tránsito arriesgadísimo que facilitó el
General García con la batida impetuosa que dio a los moros aquella
mañana. Ya llegábamos al valle de Asmir o del Río \emph{Capitanes},
planicie baja, fangosa, encharcada en parte, en parte poblada de juncos,
lugar de desolación, donde la hispana Providencia se despidió de
nuestras tropas diciéndoles: «Caballeros, ahí os quedad ahora, y yo me
voy, que todo no ha de ser bienandanzas y chiripones\ldots{} Y para que
hagáis prueba de vuestro tesón y cristiana paciencia, voy a desencadenar
hoy mismo, con permiso de Dios, uno de los más terribles Levantes que
aquí tenemos para uso de los providenciales designios, y el viento y la
mar no permitirán que os llegue el auxilio de víveres que de España se
espera. Resignaos, y llevad como podáis el ocio de vuestras armas y de
vuestros dientes en esa inhospitalaria marisma.»

Violencia horrible trajo el temporal desde su primer soplo. Trataban los
soldados de armar las tiendas, y una mano airada, invisible, arrebataba
las lonas y palitroques de que aquellas frágiles casas se componen.
Ninguna fuerza humana podía contrastar el empuje del viento, que para
causar mayor estrago se traía torrentes de agua, torrentes de granizo,
con fragor espantable que sobrecogía los más firmes corazones\ldots{} Y
los hombres desdichados que sufrían estas iras de la Naturaleza,
igualándose todos en el padecer, pues las jerarquías se borraban ante
tamaña desventura, perdían la última esperanza viendo el mar tan
inclemente como el cielo. Desde su mojado campamento miraban las olas
furiosas; veían estrellarse contra las peñas, a media legua por el lado
Norte, la goleta de hélice \emph{Rosalía}, cargada de víveres para el
Ejército\ldots{} Lo más que pudo hacerse fue salvar la tripulación y
papeles. Todo lo demás se lo tragó el mar a la vista de los hambrientos
y ateridos soldados españoles.

Y como el aspecto del mar era cada hora y cada día más imponente, ¿de
dónde había de venir el socorro, si España no podía mandarlo? Las
raciones se acortaban; pronto se acabarían en absoluto. Hombres y
caballos se veían amenazados de inanición, de muerte\ldots{} La sangre
se empobrecía, la pólvora se mojaba, los corazones eran un puro
estropajo, los rayos de la guerra se convertían en pajuelas húmedas, y
las almas guerreras en espectros que se asustaban unos a otros\ldots{}
La desolación tomó al segundo día de huracán caracteres siniestros. Los
individuos más decidores apenas hablaban; cada cual consideraba en sí
mismo el pavoroso infortunio, sin pedir impresiones a los demás por
miedo a recibirlas peores que las propias\ldots{} Los sanos parecían
enfermos, y los enfermos y heridos, cadáveres que por milagro hablaban y
se movían.

Arrojados de su tienda, que el viento desgarró, Leoncio y Juan se
refugiaron en otras mal sostenidas con refuerzos de madera y cuerdas;
las destinadas a hospitales no podían ya con más inquilinos; mezclados
estuvieron los heridos con los coléricos, hasta que se ordenó
separarlos, sin que la separación, por entorpecimientos materiales,
pudiera ser un hecho. Prefería Santiuste salirse al campo envuelto en su
manta, y aguantar allí el azote de la lluvia y el viento, a permanecer
en un estrecho local donde sólo se oían quejidos de enfermos y
moribundos, y el continuo lamentar y maldecir de los que no recibían lo
preciso para satisfacer su hambre. Las raciones de galleta húmeda
amenguaban de la mañana a la tarde, y los cocineros anunciaban la
terminación de toda comida caliente por las dificultades de encender
lumbre y de encontrar combustible en aquellos pantanos. Algunos soldados
que querían vivir a todo trance, bajaban a la playa en busca de
mariscos, y escurriéndose entre las peñas, encontraban lapas, erizos y
caracoles con que engañar su rabioso apetito.

Hecho un ovillo, arrimado al socaire de una de las tiendas que parecían
más sólidas, Santiuste conllevaba cristianamente su honda tristeza, su
inanición y su calentura. La quietud en que se mantenía, ayudábale al
adormecimiento que le hacía olvidar la realidad o apartarse de ella.
Entregábase con deleite a la modorra febril, deseando que no tuviera fin
y que le llevase al descanso eterno. Los efectos combinados de la
calentura y el pensar producían en él un estado parecido al nirvana, o
el éxtasis que transporta al cielo las almas semíticas sacándolas
temporalmente de sus cuerpos extenuados.

Flotaba el desdichado poeta y orador en regiones aéreas, donde veía las
cosas humanas en distinta forma y sentido del que abajo tienen. La
gallardísima temeridad del General Prim, el día de los Castillejos, que
más de una vez se había reproducido en el cerebro de Juan, inflamado por
la fiebre, reapareció aquella tarde con mayor fijeza y colorido más
real. El soñador se reconocía moro, sin recuerdo ninguno de haber sido
español, y entre los moros combatía\ldots{} Ya tenían los muslimes
acorralados a los castellanos; ya les llevaban por delante, haciéndoles
retroceder más allá de sus primeras posiciones, cuando de improviso
vieron que se les iba encima, como descolgándose de los aires, la figura
de Prim a caballo, blandiendo en una mano la espada fulgurante, en otra
la bandera de Castilla\ldots{} Y no era la figura del tamaño común de
los hombres y de los corceles, sino veinte veces mayor: cada casco del
caballo, al caer sobre los moros, aplastaba un gran número de ellos. El
mismo efecto de magnitud olímpica hacía Prim entre los españoles, que,
viéndose conducidos por caudillo sobrenatural, se creyeron de la misma
talla, y de vencidos se convirtieron al instante en vencedores\ldots{}
En este punto, el soñador no era moro ni cristiano, sino un vulgar
espíritu crítico, que diputó el engrandecimiento de la figura del Conde
de Reus como un efecto subjetivo en la retina y en el alma de los
combatientes embriagados por la lucha, y esta idea le llevó prontamente
a ver claro que la aparición del Apóstol Santiago en Clavijo fue un caso
semejante. Sin duda, en el Ejército del Rey de León hubo un Prim, que en
un momento propicio a las alucinaciones, produjo en todos, moros y
cristianos, la ilusión perfecta de lo sobrenatural, terror para unos,
enardecimiento para los otros\ldots{} El furor del combate ciega y
enloquece a los hombres\ldots{} Los hombres que creen firmemente en los
milagros, los hacen\ldots{}

Una mano vigorosa, sacudiendo a Santiuste, cuyo flácido rostro en el lío
de la manta casi desaparecía, le hizo al fin despertar\ldots{} Al abrir
los ojos vio un rostro desconocido, y oyó una voz que le decía: «Juan,
¿qué es eso? ¿Estás muerto, o quieres estarlo?»

La cara del que así hablaba no fue tan desconocida para Juan al poco
rato de fijarse en ella: habíala visto alguna vez; pero no acertaba, no
daba con el nombre correspondiente al rostro que veía\ldots{} Como el
otro siguiera tratándole en tono familiar y cariñoso, el poeta frustrado
le dijo: «Tenga la bondad, caballero\ldots{} la bondad de decirme quién
es usted\ldots{} porque yo\ldots{} maldito si lo sé.»

---Soy Rinaldi, Aníbal Rinaldi\ldots{} intérprete del General en Jefe.

---¡Ah!, ya voy recordando\ldots{} Hablas muchas lenguas\ldots{} ¿Y qué
se ofrece con tantas lenguas?

---Se ofrece que te he buscado toda la mañana\ldots{} Ese chico armero,
Leoncio, me dijo que te había perdido de vista. Yo te busco para
favorecerte, para darte algún socorro\ldots{} El General Ros de Olano ha
dispuesto repartir entre los enfermos más necesitados los pocos víveres
selectos y algunos vinos superiores que le quedan de su repuesto
particular. Lo mismo ha hecho el General en Jefe\ldots{} O'Donnell y Ros
de Olano, como buenos padres del Ejército, quieren que en esta calamidad
tan espantosa no haya distinción entre pobres y ricos, que todos sean
iguales, y que los más desvalidos sean los primeros en disfrutar lo poco
que Dios y el temporal nos han dejado. Ven\ldots{} no tienes que andar
mucho\ldots{} levántate\ldots{} apóyate en mí\ldots{}

\hypertarget{ix}{%
\section{IX}\label{ix}}

Si consideramos al Ejército español empantanado en las marismas del río
Capitanes como un gran cuerpo de hombre, y en todas las partes de este
cuerpo, entrañas, miembros, sangre y piel, suponemos el cruel
padecimiento resultante de la horrible situación moral y física, debemos
afirmar que el dolor más intenso y vivo estaba en el cerebro; y el
cerebro era O'Donnell. Hombre bien templado para el infortunio, lo
soportaba con estoica entereza. Pudo decir a su Ejército, imitando a
Felipe II: «Os he traído a luchar con los hombres, no con las
tempestades.» Pero más justo y más filósofo que aquel Rey, pensaba que
si era suya toda la gloria de haber iniciado aquella guerra, no debía
culpar del desastre a la casualidad, sino a sí mismo. ¿Cómo no vio que
la marcha de Ceuta al valle de Tetuán por la costa representaba un
enorme desgaste de fuerza y de tiempo? ¿No previó que a la mitad de este
arduo camino tenía que adoptar una de estas resoluciones igualmente
desastrosas: o dejar a la espalda la mitad de su Ejército para sostener
la comunicación con Ceuta, o aprovisionarse por mar, corriendo el riesgo
de que las tormentas le interceptaran el pan y las municiones? ¡Y el
enemigo siempre en posiciones altas, desde las cuales, con fuerza
inferior a la de los españoles, podía precipitarles al mar!

En verdad que si O'Donnell tuviera pecados, bien purgada quedaría su
alma con aquel intenso martirio, suficiente a franquearle de par en par
las puertas de la gloria eterna. Pero en los pecados del General no
podía buscarse la razón suprema de lo que parecía horrendo castigo,
porque era hombre puro, de una sencillez y rectitud admirables en su
vida moral; y en cuanto a la vida política, los actos de los gobernantes
no constituían estados éticos bien definidos. En todo esto y en la
pavorosa situación de su Ejército, incomunicado por el mar furioso y por
la tierra, plagada de enemigos, pensaba el General. Si alguna luz de
consuelo podía brillar en su angustiada mente, era la que una y otra vez
expresaba con esta idea: «La única ventaja mía en el presente desastre
es que jamás General alguno, en guerras antiguas o modernas, mandó
soldados tan resistentes, tan sufridos, tan dispuestos al sacrificio
como estos que yo he sacado de España\ldots» Pero inmediatamente después
de reflexión tan consoladora, venía la contraria, la negra, la que
tomaba su fatídica fuerza de la claridad de la anterior: «Si este
temporal dura días, y no hay medio de traer víveres, y los moros nos
atacan, toda esta noble juventud, esta flor de España, perecerá\ldots»

Contra tal idea se rebelaba su fe cristiana, su fe española, virtud
grande de una raza aventurera que confía en salir de todos los
atascaderos que pone en su camino la fatalidad, y al fin sale; no se
sabe cómo, pero sale. Hay una Providencia especial para los
locos\ldots{} Como hombre sereno, de los que no cuentan con la
colaboración del Acaso, O'Donnell no podía confiar extremadamente en la
Providencia de los locos. Algo pensó en ella, pero sin darle agasajo en
su pensamiento, y este lo consagró por entero a buscar y resolver los
medios de salir de aquel pantano mortal. ¡Adelante o atrás\ldots! Dos
muertes probables pesaban menos que una muerte segura.

En su tienda permanecía el caudillo dando órdenes, recibiendo partes de
los Jefes de Cuerpo, partes de Sanidad, partes de Provisiones. Algunos
ratos, quedándose solo, porque sus ayudantes habían ido a convocar para
el Consejo de Generales que debía celebrarse aquel día, se paseaba con
las manos a la espalda en el sentido más largo de la tienda, el cual
sólo permitía tres o cuatro medidas de compás de sus largas piernas. Sin
mover los labios, creyérase que hablaba con el suelo; volviendo en torno
las miradas, dijérase que quería interpretar como lenguaje las sacudidas
convulsas de la lona, y la trepidación de los mástiles que sostenían la
tienda. Cansado de andar, a la puerta salía\ldots{} interrogaba al
viento, que respondía con silbos aterradores; a la mar, que no paraba en
su mugir hondo\ldots{}

El primero que llegó al Consejo convocado por O'Donnell fue Turón, el
General más soldado que en aquel Ejército había, y se dice que era el
más soldado, porque siempre se resistió a politiquear, y consagraba todo
su ser a la devoción de la milicia y al culto de la ordenanza. De
carácter adusto y seco, y de pocas palabras, solía tener en algunas
ocasiones chispazos de gracejo. «¡Dichoso tiempo, Turón---le dijo
O'Donnell,---y dichoso valle de \emph{Capitanes!»} Y él replicó:
«Llamémosle el valle de Josafat.» Inapreciable General de división, era
la misma exactitud en el cumplimiento de las órdenes que se le daban;
brazo inflexible, con cuya ciega obediencia podía contar siempre el
pensamiento que dirigía los actos de la campaña\ldots{} Tras él llegó el
General García, Jefe de Estado Mayor, en quien descollaba el arte de
organización y el conocimiento estratégico, carácter duro y
esencialmente militar como el de Turón. Su colaboración técnica fue para
O'Donnell de gran provecho en la tan heroica como desatinada marcha de
Ceuta al Río Martín, cortando divisorias y marismas. Como conductor de
tropas a la lucha, García ilustró su nombre con uno de los actos más
eficaces para el éxito de aquella escabrosa marcha, protegiendo con el
Segundo Cuerpo, en los riscos de Monte Negrón, el paso del resto del
Ejército por los desfiladeros de la costa\ldots{} Acompañados de los
Generales de división Orozco, Gasset, don Enrique O'Donnell, Quesada y
Rubín, llegaron Ros de Olano y Prim, ambos con el cuello del capote
subido hasta las orejas, la risueña cara del primero enrojecida por el
fresco húmedo; la del segundo sombría en su color pálido verdoso.

Ya están en Consejo\ldots{} La tarde, hosca y ceñuda como la cara de
Prim, redobló la furia de los elementos. Estos dirían: «¡Consejitos a
mí!\ldots» Mientras deliberan los señores, conviene advertir que la
Providencia de los cristianos no dejó a estos en completo abandono como
las apariencias indicaban. Aquella Providencia, o la que llaman \emph{de
los locos} (no sé cuál sería), hizo tan sólo un medio mutis, quedándose
al paño entre los montes, fija la atención en los desgraciados hijos de
España. Si es cierto que no les protegió de un modo ostensible sosegando
las olas, hízoles el precioso favor de obscurecer el entendimiento de la
morisma, para que a esta no se le ocurriera desembarazarse de
cristianos, cosa facilísima en la precaria situación de estos. La
Providencia musulmana debía de estar durmiendo en aquellos tres días,
pues no se explica de otro modo que los moros dejaran pasar tan hermosa
coyuntura para caer sobre los españoles y aniquilarlos, sin que quedara
uno para traer la noticia. Que Mahoma se volvió tonto, quizás por
bebedizos que le dieron las Providencias de acá, no podemos dudarlo. La
cabeza de Muley el Abbás, o de los que dirigían entonces el cotarro
moruno, no dio de sí en aquellos días más resolución que soltar algunas
gavillas de berberiscos a robar las mulas y caballos que pastaban en las
marismas (y a pacer se les echó, ¡animalitos!, por economía de la
cebada), mientras otros hostilizaban las avanzadas del Segundo Cuerpo.
Pero el General Prim los espantó con los cazadores de \emph{Alba de
Tormes} y \emph{Chiclana} y algunas fuerzas de \emph{Castilla} y
\emph{Toledo}. Salieron estos infelices pisando fango, empapados los
ponchos, a pelear por aquellos cerros, y gracias que la humedad no había
inutilizado los cartuchos. Como insistieran los moros, unas cuantas
granadas certeras les persuadieron a tomar el portante, dejando en poder
de nuestros soldados las caballerías que ya tenían por suyas\ldots{}
¿Quién pudo dudar que Mahoma se había dormido en las deliciosas
ociosidades de su Cielo\ldots?

En una tienda-cocina del Cuartel General, hallábanse, ya entrada la
noche, el Comandante Castillejo y Leoncio heridos leves, dos Oficiales y
Juan Santiuste enfermos de calentura, y Aníbal Rinaldi, el único sano de
la reunión; el único no, que también allí estaba en perfecta salud don
Toribio Godino. Sanos y enfermos habían puesto un reparo a su
extenuación con los bocadillos y tragos de lo añejo que generosos les
repartieran O'Donnell y Ros de Olano. Ya era público en el campamento
que el Consejo de Generales había determinado que, al amanecer el día
siguiente, salieran para Ceuta en busca de víveres todas las acémilas,
escoltadas por algunos batallones al mando de Prim.

Con excepción de Santiuste, que liado en su manta se dejaba caer
nuevamente en el nirvana, todos comentaron el suceso, viendo algunos los
peligros antes que las ventajas, y confiados otros en que el Conde de
Reus triunfaría de los astutos marroquíes y de los elementos
desencadenados. Castillejo, que era el más pesimista, veía dificultosa
la ida, y mucho más la vuelta, pues no era de creer que los moros
perdiesen el sentido, y con el sentido, las ocasiones de hacernos daño.
Rinaldi, que a sus pocos años debía la felicidad del optimismo, confiaba
en el éxito de la operación; según él, con poco que protegieran la
marcha del convoy Echagüe por el Norte y O'Donnell por el Sur, las
acémilas llegarían felizmente. En lo que todos estaban conformes era en
que el temporal no tenía trazas de ceder, y su duración sería de nueve
días, cómputo de los prácticos: faltaban todavía siete\ldots{} El único
que discrepaba de este vaticinio fue don Toribio, y no tardó en
manifestarlo: sus articulaciones, así como sus callos, le anunciaban
cambio de tiempo. El buen señor se sentía barómetro, y no necesitaba
para las predicciones meteorológicas más instrumento que su propio
cuerpo\ldots{} Este le decía que los fuelles del Levante desmayarían
pronto, y que ya había corrido Eolo las órdenes para que viniesen los
fuelles del Norte a orear la tierra y aplacar las aguas.

No todos se burlaron del empirismo del capellán: algunos de los
presentes sentían en su naturaleza la indicación higrométrica y
barométrica, y otros se atenían a la tesis popular y marinera de los
nueve días, como duración de los fuertes Levantes. En esta y otras
discusiones entreveradas de somnolencias, pasaron parte de la noche, y a
la madrugada sintieron el barullo de la salida de Prim con sus
batallones y la recua de mulas. Quiso Dios que acertase don Toribio en
sus predicciones, porque al rayar el día calmó notoriamente el viento, y
hallándose Prim con su convoy como a una legua del campamento de
\emph{Capitanes}, los soldados que iban de vanguardia dieron la voz de
\emph{¡barco, barco!}, y en efecto, a poco de este aviso vieron todos
claramente el humo de un vapor que doblaba la punta del Hacho. Desde el
Cuartel general se vio también la embarcación que desafiaba el oleaje,
todavía imponente, y creyéndose ya seguro el socorro, un ayudante de
O'Donnell salió escapado a decir a Prim que retrocediera.

El barco que allá lejos navegaba con tremendas cabezadas y balances, era
el \emph{Duero}, vapor destinado al transporte de víveres: tras él
vendrían otros. El viento seguía calmado; pero la mar, aún alborotada y
ceñuda, no quería deponer su braveza, y la aproximación de buques a la
costa parecía poco menos que imposible. Con todo, el aspecto del cielo,
que rápidamente se despejaba de nubes; los rayos del sol, que se
desenfundaban de celajes, traían a todos los corazones alegría y
esperanza. De hora en hora mejoraba el tiempo; la vista lejana del
barco, que valiente acometía las olas como el hermano fuerte que acude
al socorro del hermano moribundo, a todos daba la impresión de la
Providencia, sin que nadie se metiera a discernir si era la cristiana o
la de los locos.

A medida que avanzaba el día, la esperanza se iba metiendo más en los
corazones de aquella gente infeliz\ldots{} Ya no veían un barco solo,
sino muchos. El júbilo del Ejército elevaba su número al infinito. Todos
ellos cabeceaban gallardamente sobre las olas. Inmensa muchedumbre de
soldados y oficiales los contemplaba con risueña expectación, midiendo
los espacios que las atrevidas naves recorrían en cada instante, y
acortando las distancias más con el deseo que con la vista\ldots{} Por
fin, viéndolos frente a \emph{Capitanes}, desde tierra los aclamaban,
agitando pañuelos, toallas y hasta sábanas para significar el gozo de la
visita. Llegaron los buques a tan poca distancia de la costa, que desde
esta se leían fácilmente los letreros que en sus costados habían puesto
para anunciar lo que traían: \emph{Arroz, harina, cebada, heno, patatas,
tocino, tabaco}\ldots{}

¡Comer, vivir! Buena es la gloria; pero no queráis encender esta divina
luz en una lámpara sin aceite\ldots{} Y O'Donnell, ¿qué decía, qué
pensaba? Descollando por su lucida estatura en el grupo de Oficiales
Generales que contemplaban los vapores despenseros, no dejaba traslucir
en su rostro alegría, vibrante, como tampoco en las horas de
incertidumbre dejó entrever la desesperación. Si algo expresaba su
sonrisa sutil era el convencimiento de que el socorro no le causaba
sorpresa. Lo esperaba, lo tenía por seguro. Un caudillo de tropas
regulares no podía recibir sus elementos de guerra de manos de la
casualidad\ldots{} Y volviendo la corva espalda al mar y los azules ojos
a la tierra, dijo a Turón, que a su lado iba: «No hay que
descuidarse\ldots{} Ya tenemos víveres\ldots{} Pero el enemigo querrá
que los partamos con él.»

\hypertarget{x}{%
\section{X}\label{x}}

Sucedió lo previsto por el General en Jefe: vieron los moros desde sus
altas atalayas los barcos, y en seguida les dio en las narices olor de
galletas; olor y vista que les pusieron en ganas de meter la mano en el
plato de los españoles. Aún no había empezado el desembarco de
comestibles, que se hacía con enredosa dificultad en barricas flotantes,
cuando las primeras partidas berberiscas obligaron a nuestros soldados,
hambrientos y ateridos, a entrar en faena. Un batallón de \emph{Saboya}
y otro de \emph{Córdoba} salieron con Prim a decir a los africanos que
no podíamos darles parte en el festín, y algunas horas de la tarde
empleamos en persuadirles a que fueran a buscar en otra parte el bendito
alcuzcuz. Esto no se logró sin algunas bajas, y los hospitales acabaron
de llenarse de heridos y enfermos. Daba pena, y al propio tiempo causaba
grande admiración ver a los pobres soldados, hundidos los pies en el
fango, batiéndose con tanto tesón como cuando sus estómagos llenos se
aplomaban sobre terreno firme. El extenuado poeta Santiuste, que con
lágrimas en los ojos les vio de lejos en tan heroico compromiso, se
decía para su manta: «Odio la guerra, y admiro a los que sin esperar
ningún beneficio de ella, inocentes piezas del ajedrez militar y
político, se lanzan a empeños heroicos por un fin que sólo a los
jugadores interesa. Cada día veo con más dolor de mi alma estos horrores
inhumanos; pero también digo, despojándome hasta del último plumacho de
la fanfarronería que fue mi encanto antes de venir aquí; también digo
que no hay en el mundo soldados que hagan esto\ldots{} batirse mojados y
muertos de hambre por un ideal colectivo, la gloria, de que sólo les
corresponderá parte inapreciable. O son ellos la misma inocencia, o
llevan dentro un poder anímico de extraordinaria intensidad. Si el poder
anímico produce estos actos en la guerra, ¿qué actos produciría en la
paz? Falta saberlo; falta verlo. Pero no lo veremos, porque no hay
caudillos que arrastren a los soldados a las hazañas pacíficas\ldots{}
No sé en qué consiste que el patriotismo es casi siempre un sentimiento
guerrero; no concebimos la patria sino incrustada en la idea de
conquista; no pronunciamos su nombre sin que en el aire repercuta con
son de trompetas y tambores.»

El día 10 llegó de Ceuta Perico Alarcón en el vapor \emph{Barcelona}.
Siglos se le habían hecho los días de ausencia, y de buena gana habría
cambiado el descanso de allá por compartir con su querido Ejército las
fatigas y angustias del valle de \emph{Capitanes}. Trajo noticias del
General Zabala, que iba mejorando, pero aún tenía la pierna derecha sin
gobierno. De los demás enfermos y heridos que allá quedaron en los
hospitales dio también referencia, y de la mortandad que causaba el
cólera. Uno de sus primeros cuidados fue buscar a Santiuste; se aterró
de verle tan agobiado de la fiebre, y vio con alarma los estragos que
había hecho en su cerebro la debilidad. Las ideas del poeta de la Paz se
habían sutilizado desdichadamente, llegando a ser, según Alarcón, una
bandada de pájaros que se alimentaban de moscas en los espacios del
delirio. Le oía con calma divagar en sus tesis utópicas, y trataba de
traerle a la razón y al buen sentido.

De las conversaciones que ambos tuvieron, sacó al fin en limpio Pedro
Antonio que Juan no debía continuar en el Ejército. Su endeble
naturaleza se quebraba en los trajines de la guerra, como la caña que
quisiera hacer veces de espada; las frecuentes conmociones que el terror
trágico producía en su cerebro, acabarían por darle a todos los
demonios. Convenía, pues, que a España se volviese, para reparar su
salud y poner en remojo sus ideas recalentadas\ldots{} Oídas las razones
de su amigo, convino Santiuste en que debía retirarse, aunque le
desconcertaba volver a España desilusionado y en tristísimo desacuerdo
con las ideas dominantes en toda la Península\ldots{} Con gran sentido
dijo el de Guadix que desde el punto en que se encontraban no convenía
volver a Ceuta, sino esperar a que el Ejército llegase al valle de
Tetuán, de donde le separaban no más que algunas leguas y otras tantas
victorias. A Río Martín había de llegar pronto una nueva División, al
mando del General Ríos, y con ella un tren de batir y material de guerra
y boca, lo que significaba sinnúmero de barcos yendo y viniendo entre la
costa africana, Málaga y Algeciras. En uno de estos barcos, en el mejor
de ellos, sería devuelto Santiuste a la madre patria.

No sabía el melancólico paladín de la Paz si alegrarse o entristecerse
de su regreso a España\ldots{} ¿Cómo iba él a vivir allí, sin la interna
armazón épica que era su único sustento en tierra española? Sería como
un cuerpo desmayado y vacío, cuerpo sin alma, o con un alma exótica no
comprendida de sus coterráneos. Por otra parte, la idea de ver pronto a
la sin par Lucila y al amado Vicentito, le regocijaba. Cierto que a la
divina mujer y al niño divino les encontraría en plena embriaguez de
patriotismo militar, en esa devoción ardorosa y sedienta que pedía más y
más sangre de moros con que satisfacerse. Pero ya cuidaría él, con la
virtud de su palabra, de desmoronar aquel ideal, sustituyéndole por otro
esencialmente religioso y humano.

Como un alelado durmiente, o más bien como sonámbulo, vivió Santiuste en
los días que mediaron entre la salida del atascadero de \emph{Capitanes}
y la gloriosa conquista de la altura de Cabo Negro, que dio a España la
clave del valle de Tetuán. Se dejaba ir, se dejaba llevar en la
retaguardia del Ejército, indiferente a las operaciones, oyendo tiros de
fusilería y disparos de cañón, sin que se le ocurriera indagar los
incidentes de la lucha. Aunque a la salida del pantanoso \emph{Azmir}
remitió la fiebre de Juan, había este tomado tal gusto a la envoltura y
calorcillo de la manta, que no sabía ya desembozarse de ella, y su
aspecto era el de un mendigo, moro por añadidura, pues habiendo
renunciado a la dureza del ros, que le lastimaba la cabeza, se lió un
pañuelo cuyas vueltas abultaban como las de un flaco turbante. La
querencia de la comodidad, estimulada por la pereza, le llevó también a
desechar el poncho, sustituyéndolo por un chaquetón pardo que le dio
Leoncio, muy holgado y de abrigo\ldots{} Su amistad única en aquellos
días, del 10 al 14, fue don Toribio, pues a Leoncio apenas le veía, y de
Clavería y de Pepe Ferrer sólo tuvo noticias vagas. El venerable
capellán, cuyo nombre abreviaba graciosamente Leoncio Ansúrez llamándole
\emph{don Toro Godo}, cuidaba de Santiuste, le procuraba los mejores
alimentos, y hacía por levantarle los espíritus con su ingeniosa charla,
entreverando burlas y veras al referir los incidentes de aquella parte
de la campaña. El día 12 había hecho el gasto el Segundo Cuerpo,
saliendo de guerrillas \emph{Arapiles} y \emph{Simancas}, o si se
quiere, de capeo y banderillas\ldots{} La artillería puso a los moros
bastantes picas, y luego salió Prim con el segundo de \emph{Cuenca},
\emph{Llerena}, \emph{Figueras} y el \emph{Infante}, y los mató de una
estocada superior arrancando\ldots{} No se reía Juan con estas
irreverentes aplicaciones de la tauromaquia al arte noble de la
guerra\ldots{}

El 14 rompe la marcha la División Orozco hacia las alturas de Cabo
Negro; la sigue la segunda División, al mando de don Enrique O'Donnell.
Atraviesan bosques y malezas, desfilan por entre rocas que imponen
pavor\ldots{} Hasta las diez de la mañana todo iba bien. Después de esta
hora empezaron a llover moros, y no hubo más remedio que abrir los
paraguas\ldots{} Siguió \emph{don Toro Godo} relatando en serio la
acción del 14 para dominar la divisoria del valle de Tetuán\ldots{} Pero
la atención de Santiuste, solicitada por imágenes e ideas de un orden
fantástico, no se fijaba en la palabra del castrense. Si en las batallas
vistas puede el espectador encontrar variedad grande, y notar en cada
una desarrollo y colorido propios, las referidas son casi siempre
iguales, y así lo pensaba Juan. ¿Qué le importaba que estuvieran
\emph{Cuenca} y \emph{Saboya} en el ala derecha o en la izquierda? ¿Qué
más daba que las hazañas del centro fueran obra de \emph{Córdoba} o del
\emph{Provincial de Málaga}? Los actos heroicos resultaban los mismos en
todas las narraciones, y fatigaban al oyente, que ya conocía de antemano
la furibunda carga de caballería, o la oportuna intervención de los
cañones, vomitando muertes. Lo importante era que habíamos triunfado;
que el campo quedó sembrado de cadáveres de enemigos, cosa muy bonita,
que siempre relatan con hinchada satisfacción los narradores de
batallas, diciendo a menudo con injuriosa y sacrílega frase que
\emph{mordieron el polvo}.

Con todo su cariño y amenidad no lograba \emph{don Toro Godo} aliviar
las melancolías de Santiuste, ni curarle del terror que e infundían los
cadáveres, así de cristianos como de agarenos. Huía de todo espectáculo
desagradable, y siendo estos lo común y corriente en un Ejército que se
batía de continuo y luchaba con el mal tiempo y la epidemia, el pobre
hombre apenas tenía momentos de tranquilidad. Más de una vez se le vio
requiriendo el sueño durante el día, como quien no tiene otro anhelo que
ausentarse de la realidad. Durmiendo en el rincón de cualquier tienda,
mientras las tropas descansaban, o arrimado a la impedimenta cuando se
batían, era un hombre que dejaba su cuerpo inerte en medio del trajín de
la guerra, y se iba, todo alma y pensamiento, a las distantes regiones
de la Paz.

Cuando más abstraído estaba en sus divagaciones, se le aparecía Lucila
rodeada de luz, no en calidad y empaque de Belona, sino con los arreos
más vulgares, que en ella resultaban divinos. Ya se le representaba como
Dulcinea del Toboso ahechando trigo, ya dando de comer a los pollitos
recién salidos del cascarón\ldots{} La dama labriega imperaba en su casa
de la Villa del Prado, y nada se advertía en ella que revelase aficiones
militares ni gusto de matanzas guerreras. Como matanza, allí no había
más que la del cerdo, y aun el sacrificio de animales sería menos cruel
y brutal que en otras casas\ldots{} Gozaba el trovador viendo a Lucila,
aunque la dama no le hablara. Sin mirarle se le aparecía, ¡cosa más
extraña!, y aunque él la llamaba ceceando con cierta angustia,
\emph{«Luci, Luci, que estoy aquí,»} la dama no hacía caso, y continuaba
con más atención en sus menesteres domésticos que en el pobre desterrado
de África\ldots{} Despierto o a medio despertar, continuaba Juan
cultivando el sueño, y le ponía en cuidado que habiéndosele aparecido
tres veces la madre, no se viera en derredor suyo ni rastros de
Vicentito Halconero\ldots{} ¿Qué hacía el precioso niño mientras la
madre daba de comer a los pollos?\ldots{} En una de las transformaciones
de su pensamiento o de su delirio, pues todo era lo mismo, \emph{vio} y
\emph{pensó} que el chicuelo había muerto \emph{abrazado a la bandera de
la patria}, llevándose al otro mundo su pasión guerrera y las
precocidades de su genio militar. Esta idea era intolerable suplicio
para Santiuste, que al punto buscaba nuevas ideas, nuevas imágenes con
que olvidar aquella tan desastrosa y terrible.

\hypertarget{xi}{%
\section{XI}\label{xi}}

Paseando con \emph{don Toro Godo} una tarde por las lomas de Cabo Negro,
en dirección a la cuenca anchurosa de Río Martín, se arrancó Santiuste
con unas ideas tan peregrinas, que su venerable amigo le tuvo por hombre
sin seso, o a punto de perderlo. «Ya sabe usted, \emph{don Toro}---dijo
el poeta,---que tengo por gravísimo mal el celibato eclesiástico. La
Iglesia lo puede todo en el terreno dogmático; pero no alterará jamás
las leyes de Naturaleza, ni la fundamental hechura de nuestras almas.
Cegada la fuente del amor humano, ¿cómo hemos de apreciar y comprender
el divino? Si nos sacáis los ojos, ¿cómo hemos de distinguir los
colores? Cerradnos el oído, y no sabremos gozar de ninguna clase de
música.»

---Esa es una cuestión, Juanito mío---dijo el ladino capellán,---sobre
la cual un viejo de setenta años no puede opinar discretamente; que no
está bien pedir dictamen al polo frío sobre los calores tropicales.
Quien ha perdido hasta el compás no puede hablar de baile, ni su opinión
vale de cosa alguna. Yo estoy en el caso de decir, con referencia a
nuestro celibato, que así lo encontré y así lo tengo que dejar. Si me
hubieras consultado cuarenta años ha, quizás, y sin quizás, te habría
dado algún parecer ajustado a los hechos y a la realidad del
vivir\ldots{} Pasemos a otro asunto.

---Paso a decir que si estimo como un mal el celibato de los sacerdotes,
peor me parece el de los ejércitos en campaña. ¿Qué razón hay, mi
respetable \emph{don Toro}, para que no acompañen mujeres a los pobres
soldados traídos a esta vida de perros?

---La razón es que esa impedimenta impediría demasiado la acción
militar, apagando la bravura de los hombres, y llevándoles a una vida
muelle y viciosa, incompatible con la actividad y virtud necesarias en
estas empresas. ¡Bonita cosa sería un ejército con mujeres! ¿Quién las
aguantaría en campaña; quién podría someterlas a la disciplina, ley dura
para los hombres, para ellas imposible?

---Cierto es que el sexo femenino, siguiendo a los hombres a la guerra y
consolándoles de sus penalidades, traería disgustillos, piques, y quizás
alguno que otro rifirrafe escandaloso. Pero este mal tendría
compensación en el bien grande de la alegría del soldado, en su mayor
coraje para la lucha\ldots{} con el estímulo de ser visto y alabado por
ellas. Crea usted que con mujeres existiría en los campos de batalla el
complemento de la vida, y las guerras serían menos sanguinarias\ldots{}
los ejércitos llevarían consigo el \emph{elemento de compasión}, que
ahora falta en absoluto\ldots{}

---Hijo mío---replicó \emph{don Toro}, tomando un tonillo de
unción,---también en esto del celibato militar en campaña te respondo,
como al tratar del otro celibato, que no pidas su opinión a un viejo
como yo, dispensado por su edad de discurrir sobre nada referente a
mujeres. El frío de los años trae la indiferencia de esas cuestiones,
que no pueden debatirse sino con calor de la mente. Si me hubieras hecho
esa consulta treinta años ha, yo te habría respondido que el
\emph{elemento femenino} está en el pensamiento del soldado, ¿me
entiendes?\ldots{} y ya sabe el soldado que para ser dueño de él, tiene
que ir a buscarlo al campo y a las ciudades enemigas\ldots{} Siempre se
ha entendido así el negocio de amor en las guerras, y no puede ser de
otro modo. Tu teoría es disparatada, absurda. Apliquémosla a esta
campaña española en África: suponte que traemos hembras, a las cuales
hay que llamar soldadas, sargentas y oficialas; supón que contra el
orden natural sufrimos un revés\ldots{} nos arrollan los moros, y
después de matarnos y de quitarnos las armas, cargan con las
señoras\ldots{} ¡Bonita cosa, Juan!

---Cierto que sería triste; pero usted ha dicho que cada ejército busca
sus damas en el campo contrario\ldots{} Los hombres morirían
defendiéndolas. Pasarían ellas de una mano a otra, como hoy pasan las
plazas fuertes, los cañones. Se cumplía la ley de humanidad; la total
armonía no se alteraba por eso. Las naciones tendrían un motivo más para
no lanzarse a guerras desatinadas y de pura ambición; ya se sabía que
corrían el riesgo de perderse todos los elementos de vida de un pueblo,
los hombres, las ciudades, la riqueza, y las mujeres\ldots{} Entretanto,
yo digo y sostengo que no puede estar esta masa de hombres en tan larga
ausencia y privación del bello sexo. A la larga, sin él la vida de
campamento se vuelve árida, tristísima, y la Gloria es una imagen
hombruna que acaba por causar espanto. Esto digo, esto siento, y miles
de hombres hay aquí que seguramente sentirán lo mismo.

En tonos de humorismo siguió \emph{don Toro} la polémica, cuidando de
acentuar poco la inflexión burlona para no irritar a su contrincante. Lo
que verdaderamente sacaba de quicio al pobre poeta era la narración de
batallas o de cualquier lance de guerra. Si con sus protestas no hacía
callar al castrense, se tapaba los oídos, y se echaba en tierra boca
abajo gritando: «No quiero, no quiero; cállese, o perdemos las
amistades.» Y divagando por el campo de la última acción tan gloriosa
para Ros de Olano y Prim, a cada paso hallaban despojos de la caballería
y de los infantes moros, espuelas, riendas, fragmentos de gualdrapas y
frontiles, algún arma, algún cantarillo portátil de peregrina
forma\ldots{} Todo lo recogía y guardaba cuidadosamente \emph{don Toro},
con idea de venderlo en Madrid a los aficionados que coleccionan
baratijas exóticas.

El mayor encanto del largo paseo de aquella tarde fue la repentina
emergencia de un inmenso y luminoso panorama, que les saltó a los ojos
al revolver de una loma pedregosa, como a media legua del campamento.
Era el valle de Tetuán, ancho y risueño, término de la fatigosa marcha
costera, y principio de una etapa militar más brillante y gloriosa.
Lanzó Santiuste de su pecho exclamaciones de júbilo, y quedó absorto,
saciando bien los ojos antes que la admiración descendiese a la palabra.
No estaba menos sorprendido y alelado \emph{don Toro}, que al instante
hizo gala de los conocimientos geográficos adquiridos en el campamento.
«Estos montes que vemos a nuestra derecha---dijo al poeta,---son los
llamados Sierra Bermeja, estribación del Atlas que se corre por aquí
hasta asomarse al mar\ldots{} Hacia esta parte, entre riscos ásperos,
verás allá lejos una cinta de blancos muros almenados. Por San Toribio,
mi patrón, que aquella es la opulenta Tetuán, objetivo de nuestra
campaña\ldots{} Allí está el reposo, allí la recompensa de tantos
afanes\ldots{} Quiera Dios allanarnos estos verdes caminos, como nos
allanó los pedregosos de esa maldita costa, alternados de marismas
fétidas\ldots» Por un momento creyó Santiuste en la elocuencia del buen
capellán, y con sorna le dijo: «¿Qué es eso, \emph{pater}? Estáis
preparando un sermoncico para endilgarlo después de la primera misa de
campaña que se celebre.»

Y \emph{don Toro} prosiguió: «Echaré sermones, o guardaré silencio si
así me acomodara. La palabra del Señor suena en los corazones, y no es
menester que mi voz clueca la traduzca en sonidos usuales\ldots{}
Entérate bien de lo que estamos viendo, Juan, y alaba conmigo a Dios por
dejarnos ver tanta belleza. Este nuevo aspecto del África será regocijo
y orgullo de nuestro Ejército, porque ¿quién duda que conquistaremos a
Tetuán y todo lo que sigue tierra adentro? ¡Hosanna! ¡Lástima grande que
no puedan ver esto los pobrecitos españoles que se han quedado en el
camino! ¡Pobres cuerpos, pobres almas!\ldots{} Fíjate, hijo mío, en
aquella masa de verdor que se extiende como alfombra más acá de la
ciudad blanca. Pues hay allí naranjales tan hermosos, según dicen, como
los de Murcia y Valencia\ldots{} Las casitas blancas, salpicadas entre
lo verde, parecen tiendas. ¿No crees que en una de esas descansarían muy
bien los huesos de este cura? Pues vuelve los ojos a la otra parte, a
mano izquierda, y verás el mar, adonde lleva sus aguas el río grande que
serpenteando baja de la ciudad, y otro pequeño que corre más cerca de
nosotros, y también en la mar se vacía. Hay un tercero que si no me
engañan los ojos desagua en el grande\ldots{} Este es el Río Martín, o
\emph{Río Dulce}: se me ha ido de la memoria el nombre arábigo, que
pienso ha de ser uno de los célebres lemas de la historia de nuestros
días\ldots{} Sigue la dirección de mi dedo, Juan, y verás un caserón
blanqueado, que debe de ser (no quiera Dios que yo mienta) la Aduana de
esta región\ldots{} y más allá, pegadita al mar, verás una que no sé si
es torre o palomar grande, construcción estrambótica, cuyo cuerpo
inferior parece que lleva miriñaque. Es el fuerte con que la morisma
defiende la entrada de ese río: allí guardan (yo no lo he visto) cañones
del año Mil y quinientos, y otras máquinas de guerra anteriores al
tiempo en que Satanás inventó la pólvora\ldots{} Tú, que tienes mejor
vista, mira bien en la extensión del mar. ¿No distingues un barco,
quizás dos, tres?\ldots{} ¿No alcanzas a ver en el horizonte muchos
puntitos, que son la flota en que viene el General Ríos con ocho
batallones, un tren de batir, gran acopio de alimentos y bebidas, y
otras cosas de grande utilidad en la república, como quien dice, en los
Ejércitos?\ldots» Afinando su vista, Santiuste exploraba el mar azul,
sin distinguir escuadra próxima ni lejana; y como se habían alejado del
campamento más de lo regular, \emph{don Toro}, inquieto, propuso a su
acompañante una prudente retirada: «Volvámonos a casa, Juanito mío, y
desde mañana seguiremos en la retaguardia de nuestro ejército,
\emph{viendo venir} las cartas de este juego histórico.» Empezó a
lloviznar: el hermoso paisaje que atrás dejaban \emph{don Toro} y Juan
se empañaba, se desleía en una atmósfera lechosa y terne. Así el alma
desconsolada de Santiuste veía en sí misma el deslucimiento de las
glorias guerreras, como colores que se deslíen y rayos de sol que se
mojan.

\hypertarget{xii}{%
\section{XII}\label{xii}}

Al siguiente día, el sol se declaró francamente español desde las
primeras horas de la mañana (15 de Enero), rasgando las nieblas y
alegrando con su claridad y su lumbre así los montes y valles como los
corazones. Las naves que traían la nueva División echaron anclas en la
rada anchurosa. Las fragatas \emph{Blanca} y \emph{Princesa de Asturias}
inutilizaron con pocos tiros el fuerte Martín y sus anexos militares.
Los pobres moros que defendían con artillería vieja, del tiempo del
Diluvio, la entrada del Río, huyeron a la desbandada, imprimiendo en el
fango de las marismas la huella inequívoca de sus babuchas. Desembarcó
infantería de Marina para posesionarse del Fuerte; desembarcó en la
playa del Norte, entre Río Martín y Río Lil, la División del General
Ríos, compuesta de ocho batallones de Línea y Cazadores y un escuadrón
de Caballería; pisaron tierra sin dificultad las acémilas y todo el
matalotaje de impedimenta. Continuaban llegando barcos con el nuevo tren
de sitio, y copiosas remesas de provisiones para todo el Ejército. ¡Día
lisonjero para España, que olvidaba las horrendas fatigas de la marcha
por la costa! «¿Por qué no empezamos la guerra por aquí?» era la
pregunta que todos se hacían a sí mismos y a los demás. Consolábanse con
la idea de que el paso de Ceuta a Río Martín había sido un aprendizaje
necesario, un ejercicio de gloria y muerte, por el cual llegaban al pie
de los muros de Tetuán dotados de una fuerza invencible.

Al paso que se efectuaba el desembarco de hombres, víveres y municiones,
Ros de Olano avanzaba hacia el llano; Prim le cubría la retaguardia. De
lo alto de la Torre Geleli, donde el Imperio tenía su Cuartel general,
se destacó gran caterva de moros a pie y a caballo; mas no contaban con
las piezas rayadas que en batería mandó colocar O'Donnell en punto muy
bien escogido, cubriéndolas con fuerzas de Infantería y Caballería.
Avanzaron los árabes con la chillona algazara que les sirve de música, y
cuando se les tuvo a conveniente distancia, se abrieron las filas que
cubrían los cañones, y estos empezaron a escupir granadas. Los moros de
a caballo, que no bajaban de ocho mil, y los doce mil infantes, no
aguardaron a que los cañones echaran de sí toda su saliva, y
retrocedieron con horroroso pánico, refugiándose en las fragosidades de
Sierra Bermeja\ldots{} Los españoles no tuvieron aquel día ni una sola
baja: día y acción memorables.

Ya era don Leopoldo dueño del llano bajo de Tetuán. Al siguiente día,
molestados por un furioso aguacero, armaron los españoles sus tiendas en
los puntos conquistados. El Cuartel general acampó junto al Fuerte; a su
derecha, en el sitio más próximo al mar, Prim con el Segundo Cuerpo; río
arriba, junto al caseretón de la Aduana, también abandonado por los
moros, Ros de Olano con el Tercer Cuerpo, Ríos con su División y la de
Reserva. El grupo de tiendas de esta gran masa de tropa, con los
parques, acémilas, maestranza, \emph{etcétera}, formaba una ciudad
populosa y animada. Corta distancia la separaba de aquella en que
moraban O'Donnell y Prim. Alguien dio a los dos campamentos los nombres
de \emph{Carabanchel de Abajo} y \emph{Carabanchel de Arriba}.

Extremaba Leoncio la broma dando el nombre de \emph{Leganés} al fuerte
que se empezó a construir en un sitio llamado \emph{La Estrella}, a la
orilla izquierda del río Alcántara, afluente del Martín. Por cierto que
iba muy bien de su herida el simpático armero con los puntos de sutura
que le dio el Físico, y los emplastos y la quietud. Andar podía ya sin
dolor y con marcada cojera, y consagrar al trabajo algunas horas.
Recobró su alegría, y se le encendió más el entusiasmo por el buen giro
que a su parecer llevaba la campaña; escribía largas epístolas a su
mujer, y guardaba en el pecho como escapularios las que de Virginia
había recibido. «Oye tú, Juan---dijo a su amigo una mañana, sentados a
la puerta de la tienda:---en mi carta he participado a \emph{Mita} que
no puedes seguir aquí, que no te prueban los aires de África\ldots{} Ya
puedes ir liando tu petate\ldots{} Por lo que me ha dicho Alarcón,
entiendo que te despachan, con las pipas vacías, en el primer barco que
salga.» Nada respondió Santiuste; mas con un mohín de su rostro
demacrado, expresó un asentimiento fatalista. En esto se aproximó al
grupo Enrique Clavería, risueño, zumbón, y soltó, no diremos bomba, pero
sí esta carretilla de pólvora, ruidosa como una explosión de risa
picaresca: «¿No saben qué cargamento ha venido en los barcos, con los
sacos de harina y las cajas de galleta? ¿De veras no lo saben?»

---¿Qué nos han traído? ¿Mazapán de Toledo, carne de membrillo, jamón en
dulce?

---Es mejor carne y mejor pastelería que todo eso. Anoche llegó un vapor
abarrotadito de mojama, y de otro artículo superior\ldots{}

---¿De qué, hombre? Vomita pronto\ldots{}

---Lo sabéis, y os hacéis los tontos\ldots{} ¡Hipócritas! No finjáis
disgusto por lo que os alegra. Lo que trajo el barco es un bonito
cargamento de mujeres.

---Ya, ya\ldots{} eso decían; pero no cuela\ldots{} ¡Mujeres al
campamento!

---Cierto es---indicó un alférez, convaleciente del cólera.---Pero no
las han traído, sino que han venido ellas de su \emph{motu proprio} y
por querencia natural.

---Pero, señores---dijo el Comandante Castillejo, que se arrimaba
siempre a las tertulias de muchachos,---¿para qué nos traen mujerío, si
en Tetuán, allí\ldots{} tenemos los harenes?\ldots{} A los harenes
vamos, y podremos mandar a España cargamento de huríes\ldots{} En fin,
si han llegado las huríes de pega, sean bien venidas\ldots{} ¿Y dónde,
dónde han metido ese simpático ganado?

---Para mí, que las han encerrado en el polvorín\ldots{}

---¿Por qué tú, Clavería, y tú, Santiuste, no vais allí, y hacéis un
reconocimiento? Traednos noticia de si son muchas o pocas las cabezas de
ese ganado; si viene en buen estado de carnes, y si es el Cuartel
general quien lo suministra, o es cosa de arreglarse cada uno para el
consumo particular\ldots{} ¿Trae ese ganado pastoras?\ldots{} ¿Nos
repartirán boletas como las de alojamiento?\ldots{} En fin, que sepamos
a qué atenernos, porque esto no es cosa de juego\ldots{} ¡Cáscaras!,
todo no ha de ser batirse y exponer uno la pelleja a cada triquitraque.

Esto decía Castillejo, que siempre de buen humor convertía en espuma
picaresca las amarguras y penalidades de aquella vida. Llevaba un brazo
en cabestrillo, y habíanle sometido a un régimen riguroso por
complicaciones de enfermedades internas. También apareció por allí
\emph{don Toro} Godo, que reprendió a la partida por sus licenciosos
apetitos, diciendo con buena sombra: «¡Que no pudiera daros yo mis
setenta años para que con el frío de ellos se os apagaran esas
liviandades!\ldots{} ¡Puercos, disolutos, almas de cántaro! ¡No os
parece bastante penosa la vida de campaña, y queréis traer a ella el
Infierno, o dígase niñas!\ldots{} Cuando yo era joven, los soldados iban
a buscarlas en los serrallos libres del enemigo\ldots{} Pero vosotros,
gandules, queréis que os las traigan al Ejército, como parque del vicio
y ambulancias de corrupción\ldots{} ¿Y para qué? ¡Para llevar con
vosotros dos guerras en vez de una, y duplicar las muertes que han de
acabaros!\ldots{} Y ahora, libertinos, sacos de podredumbre,
decidme\ldots{} ¿dónde, dónde están esas desgraciadas?»

Las risas avivaron más el humorismo del castrense, que, como Castillejo,
gustaba de platicar con gente moza, y de encender en ella el regocijo y
amor de vida que él no podía disfrutar. Santiuste, sin decir palabra,
embozado siempre en la taciturnidad como en su manta, se fue a las
tiendas de \emph{Ciudad-Rodrigo} en busca de Alarcón, que por Clavería
le había llamado con urgencia. En \emph{Ciudad-Rodrigo} le encontró y
hablaron, manifestándole Pedro Antonio que estuviera dispuesto para
embarcar al día siguiente, en un vapor que de retorno llevaba heridos y
enfermos a Málaga o Algeciras. En el campamento no se quería gente
ociosa, consumidora de víveres, sin producir ninguna fuerza. Mejor
estaba él en España que en África. El mismo Beramendi, que tanto le
apreciaba, se haría cargo de la razón de su vuelta a España, le
sostendría en su destinillo del Ministerio de Fomento, y le abriría las
puertas de un periódico para que \emph{propter panem} escribiese de la
guerra, de la paz o de la \emph{inmortalidad del cangrejo}. Nada objetó
Santiuste a las palabras cariñosas de su amigo. Teníase por un ser
inútil, lanzado a las corrientes del Acaso, sin rumbo ni norte. Iría,
pues, a donde cualquier fuerza extraña le empujase, a menos que alguna
fuerza interior suya surgiera del seno mismo de su enervante debilidad.

Díjole también Alarcón, mostrándole unos líos de telas, que con él
enviaba a sus amigos de Madrid regalo de dos chilabas, parda la una,
azul la otra; dos yataganes cogidos en el campo de batalla, un tapiz y
varios pares de babuchas para señora y caballero. Le previno que haría
con todo ello un fardo bien acondicionado, envuelto en una tela cosida,
y a su tienda se lo enviaría con una carta para la persona a quien debía
entregarlo. Firme en su fatalismo, aceptó Juan la comisión sin decir
nada en contrario, lacónico, frío, insensible. Volviose a su tienda,
donde halló notificación escrita y orden verbal para que estuviese en la
Aduana a las primeras luces del día siguiente, dispuesto a embarcar en
el vapor \emph{Ter}\ldots{} A todo dijo \emph{amén}, y luego se echó a
dormir, poniendo por almohada el fardo que Pedro Antonio había confiado
a su buena amistad.

En su nebuloso sueño, se le apareció Lucila, que por lo visto no tenía
otra cosa que hacer en el mundo más que aparecerse aquí y allá\ldots{}
Hacia él llegaba sin mover los pies, con andar trémulo, semejante al de
las imágenes en las procesiones\ldots{} Vestía negra túnica de Dolorosa,
y su rostro expresaba compunción grave. ¿Lloraba la muerte de la épica
militar? ¿Lloraba la muerte de su hijo Vicentito? Esta idea fue para el
soñador una gran congoja. Viviera el niño y viviera con su pepita, esto
es, con su delirio por las glorias del soldado español. Creyó Santiuste
que la mujer aparecida clavaba en él una mirada rencorosa. ¿Por qué le
miraba con odio? ¿Qué había hecho él más que amar a la madre con
platónica y casta fe, y al hijo con pasión semejante a la de San José
por el Niño Dios? Si alguna desgracia había ocurrido, él, pobre poeta y
trovador desengañado, no tenía la culpa. Algo de esto debió de decir a
la figura o espectro de la celtíbera, porque ella tomó actitud de
escuchar, llevando al oído su mano ahuecada, y luego habló con palabra
iracunda. Lo que entonces dijo Lucila fue para Santiuste como si un rayo
cayera sobre su cabeza\ldots{} Del estremecimiento despertó, quedándose
un mediano rato entre la realidad y el sueño. Despierto y alucinado aún,
decía: «Yo no le he matado, Luci\ldots{} ¿Cómo había de matarle yo, que
tan de veras le quiero?\ldots{} Lo que hay, Luci, es que se ha venido
abajo el castillo de la epopeya, y si al caer todo ese matalotaje quedó
Vicentito enterrado entre los escombros, no es culpa mía, Luci\ldots{}
Luci, no es culpa mía\ldots{} ¡Vicente entre las ruinas!\ldots{} Pero
¿qué culpa tengo?\ldots{} Yo no derribé el castillo vetusto\ldots{} se
cayó él solo\ldots{} porque quiso caerse\ldots{} Yo no he sido,
Luci\ldots»

\hypertarget{xiii}{%
\section{XIII}\label{xiii}}

No se sabe lo que duró este delirio, y sí que a la madrugada, cuando aún
no mostraba el Oriente ni presagios de aurora, salió Juan de su tienda,
solo, sin más compañía que un palo, llevando a cuestas los dos petates,
el suyo y el que le había confiado Pedro Antonio. Atravesó casi todo el
campamento, recogido en medio de la plácida obscuridad; pasó por las
tiendas de \emph{Baza}, de \emph{Segorbe}, del \emph{Primero de
montaña}, de \emph{San Fernando}, de \emph{Bailén}, de \emph{Soria}, de
\emph{Iberia}, hasta llegar a la Aduana. A las guardias dijo: «Voy a la
Aduana para embarcarme,» y ningún obstáculo halló en su camino\ldots{}
Reconociendo el disforme edificio que le habían designado como depósito
de los que volvían a la patria, y en el cual vio como un vasto panteón
de muda blancura, erigido en las tinieblas, torció a mano derecha y
anduvo un corto trecho hasta dar en la margen del río Alcántara\ldots{}
Por la ribera pantanosa, chapoteando en el fango, llegó a un puentecillo
jorobado que había visto de día\ldots{} Detúvole el temor de tropezar
con centinelas o escuchas; pero cerciorado de que no había nadie, pasó a
la otra orilla, donde un lugar seco, entre juncales, brindábale a
cambiar tranquilamente de vestido. Quitose el chaquetón; endilgó sobre
la camisa la chilaba parda; de cintura abajo quedó desnudo de pie y
pierna, calzadas las babuchas amarillas, después de refregarlas en la
tierra húmeda para que tomaran aspecto de prendas muy usadas. Con todo
lo demás, lo que se quitó y lo que no se puso, hizo un envoltorio que
arrojó al río. Desliado y vuelto a liar con esmero el pañuelo retorcido
y nada limpio que llevaba en su cabeza al modo de turbante, creyó que su
facha moruna era de intachable propiedad\ldots{} Echando a andar
resueltamente río arriba, no se le ocultaban las dificultades de su
situación\ldots{} Podría engañar su figura, que con la corta barba que
se había dejado crecer podría pasar por rostro agareno; pero
desconociendo el árabe, ¿cómo engañar con la palabra? Ocurriole la
salvadora idea de fingirse mudo\ldots{}

Enfermo y sin palabra podría mendigar, hasta que el Acaso, en quien
confiaba ciegamente, le llevase a donde pudiera descubrirse y hacer vida
de paz\ldots{} Hallábase en aquellos instantes el infeliz poeta y orador
en un estado de absoluta confusión. Si alguien le preguntara cuál era su
objeto al disfrazarse, y a dónde iba, no habría podido dar respuesta.
Una inquietud mecánica le movía; su voluntad se encaminaba hacia un fin
abstracto, nebuloso, como las promesas de ultratumba. No obstante su
estado mental de éxtasis ambulatorio, cuando aclaró el día y pudo
distinguir los contornos del paisaje, a su derecha los cerros en que
suponía las avanzadas moras, a su frente la torre Geleli, Cuartel
general de Muley el Abbás, tuvo una visión vaga del peligro que
corría\ldots{} Pero sus piernas, como si funcionasen en franca
independencia, seguían llevándole adelante por la margen derecha del Río
Martín, de curso perezoso, con lentas ondas, de las cuales dijo
Santiuste que eran el paso de un río pensativo.

Constituidas en cabeza directora de todo el ser, las piernas de Juan
seguían impávidas su camino; la vista recelaba; el oído no estaba
tranquilo; el corazón dejábase caer en la indiferencia de la vida y la
muerte\ldots{} Ya era día claro; ya distinguía los verdes naranjales que
alfombran la vega de Tetuán; pasó junto a chozas que parecían
abandonadas, junto a huertos con cerca de cañas, y ningún ser viviente
encontraba en su camino\ldots{} Llegó a un lugar apacible, como glorieta
rústica formada por cipreses viejos y arbustos lozanos. Sentándose a
reposar, contempló la bella Naturaleza que le rodeaba, y en tal
contemplación sintió hambre, mas no vio con qué podría repararla\ldots{}
Tras un descanso que él no podría decir si fue largo o breve, las
piernas recobraron súbitamente su poder directivo, y se lanzaron a un
andar acelerado, sin pedir permiso al corazón ni a la mente. Los ojos
miraban a la otra parte del río, considerando que si hubiera en éste un
vado seguro, el hombre procuraría recabar de sus piernas que le pasaran
a la orilla derecha\ldots{} En esto oyó rumor de voces humanas\ldots{}
Eran voces de mujer, confundidas con ladridos de un perrillo juguetón.
Se sobrecogió; mas no quisieron parar las piernas, por más que el hombre
les ordenó que contuviesen su marcha rítmica\ldots{}

Vio Santiuste tres figuras extrañas que por la vereda marchaban hacia
él: se componía cada cual de un pesado envoltorio de tela blanca, que
por debajo dejaba ver dos piernas gordas y amoratadas, los pies con
babuchas; por encima una mofletuda cara medio cubierta con la misma tela
burda, a manera de embozo sostenido por un brazo gordinflón. Por un
momento dudó Juan si eran hombres o mujeres las estantiguas que veía;
luego, recordando noticias y cuentos del personal marroquí, cayó en que
eran moras viejas y fuera de uso. Tras ellas venían dos chicos ágiles,
morenos, las cabezas rapadas, conservando un mechón junto a la oreja:
jugaban con un perro. Llevado de sus piernas autónomas, Santiuste se vio
muy cerca de aquella gente, y con maquinal impulso, movido del hambre
que sentía, alargó una mano en demanda de algo de comer; pero, sin
olvidarse de que debía parecer mudo, sólo echó de su boca sonidos
inarticulados, que a su parecer imitaban perfectamente el ladrido de los
que perdieron o no adquirieron jamás el uso de la palabra. Rodeado por
aquella caterva, que no le mostraba compasión, oyó Juan un lenguaraje
que para él no tenía ningún sentido; mas por los ademanes y el rostro de
las feas y vetustas mujeres comprendió que le reñían, que le increpaban,
que le preguntaban su nombre, nacionalidad y condición\ldots{}

Tan acosado se vio el vagabundo, y tal temor le entró de aquellas, más
que mujeres, bestias en dos pies, que no se opuso a que los suyos
echaran a correr hasta ponerse a distancia de tan bárbaros gestos y de
las voces airadas, incomprensibles. Metiose Juan por un prado, entre
arbustos, sin saber a dónde saldría, y en su retirada recibió la
horrorosa pedrea con que le despidieron los dos moritos acompañantes de
las endiabladas hembras. En el momento de agachar la cabeza para
guardarla del nublado, recibió detrás de la oreja una peladilla que le
hizo ver el sol y la luna. La descalabradura no era cosa de juguete: de
ella salió un hilo de sangre que puso el cuello del pobre Juan como si
le hubieran degollado. La mano se llevó a la parte dolorida, retirándola
ensangrentada\ldots{} Y al punto las piernas, azuzadas por el desastre,
dieron todo el impulso posible a sus musculares resortes, lanzándose a
la carrera por un terreno desigual, aquí blando y cubierto de hierba,
allá pedregoso\ldots{} Ello es que fue a parar, jadeante, a otro sitio
despejado, donde igualmente oyó voces de mujeres\ldots{} Creyérase que
el bello sexo, objeto siempre de sus afectos más vivos, le perseguía,
tomando las formas menos gratas a la vista y la imaginación, como
emblemas de remordimiento o de castigo.

La carrera que llevaba el prófugo terminó frente a un extraño grupo,
formado por tres mujeres, un hombre y un asno\ldots{} Una de las hembras
estaba en pie, las otras a gatas, arrancando hierbecillas de entre la
espesa vegetación de un extenso prado que abrillantaban las gotas del
rocío. En la misma actitud, cuchillo en mano, había visto Santiuste, en
campos españoles, a las aldeanas cogiendo verdolagas y cardillos. La
mujer que estaba en pie, más vieja que las otras, parecía también de
superior categoría, aunque no se marcaba mucho la diferencia: las tres
eran ordinarias, nada limpias y de dudosa belleza. Vestían faldas
azules, calzaban babuchas rojas, y en la cabeza llevaban pañuelo de
colorines, liado con un arte nuevo a los ojos de Santiuste. La que
parecía principal era la única que llevaba medias, y en el busto un chal
amarillo, de crespón, muy usado\ldots{} El burro pacía con avidez de
atrasado apetito, y el hombre, tan pequeño que bien podría llamarse
enano, vestía un haraposo balandrán azul, y se cubría la coronilla con
un gorrete del mismo color. Calzaba viejísimas babuchas que parecían de
tierra; su rostro era lívido, con bigote lacio; su edad difícil de
precisar.

Al llegar Santiuste junto a tan extraña gente, el lenguaje que hablaban
a español le sonó\ldots{} La mujer principal le vio venir entre curiosa
y asustada\ldots{} Temeroso él de ser mal recibido, señaló con la
izquierda mano su herida, que manaba sangre, y se llevó al pecho la
otra, inclinándose como persona humilde que pide socorro a un prójimo
desconocido\ldots{} La del chal habló así: «¿Quién sodes tú, desdichado?
¿Qué es tu demanda?»

Y otra de las que gateaban, dijo: «Tírate atrás, que atemorizas. Por el
Dio de Israel dinos tus coitas\ldots{} que bien se cata que has trocado
tu ley para venir ende acá.»

Y la del chal siguió: «Ya sabemos quién te ha ferido. Oye de mí: so
mujer buena, y mi corazón sabe apiadar de ti mas que seas culposo\ldots»

Absorto quedó el pobre fugitivo ante lo que veía y oía. Aunque ya se
preparaba para soltar los mugidos que le harían pasar por mudo, contestó
en habla de cristiano a las expresiones afectuosas de la señora con
medias. Preguntado de nuevo por su nombre, patria y condición, no
repuesto aún del trastorno mental que el hambre y la fiebre le
producían, habló de este modo: «Yo soy \emph{Juan el
Pacificador}\ldots{} Si sois amantes de la guerra, matadme, porque yo
enseño a condenar los males de la guerra; si sois gente piadosa, curadme
esta herida y dadme algún alimento, que por Dios vivo os juro que no
puedo ya con mi alma.»

Las dos que cogían hierbas dejaron esta operación para ponerse a lavarle
la herida con agua de un cercano arroyuelo. Entre tanto, la del chal le
dijo: «Agora veráis que hais topado con familia bondadosa. Afloja tu
pena, y ven a mi casa, do toparás remedio y paz\ldots{} Monta en el
asno, y seguro venrás a la cibdad\ldots» Al enano, luego que Juan se
encaramó en la cabalgadura, le dijo: «No intraremos por
\emph{Bab-el-aokla}, que allí fincan hombres recios de mucha
guerra\ldots{} Daremos güelta por porta alta, donde no mancarán los
portaleros amistosos\ldots{} No tener cuidado, y vámonos aina\ldots{}
Arre, adelantre vos; nosotras adetrás con hierbas de curación\ldots{}
Arre\ldots{} arre, hijos, sin amedranto\ldots{} que naide haberá que
pesquise\ldots{} Porta alta, Esdras\ldots{} ca por allí salvamos sin
peligración.»

Ved aquí por qué extraño modo penetró \emph{Juan el Pacífico} en la
poética \emph{Tettauen}, dulce nombre de ciudad, que significa
\emph{Ojos de Manantiales}.

\hypertarget{tercera-parte}{%
\chapter{TERCERA PARTE}\label{tercera-parte}}

\begin{flushright}
\textbf{Tettauen, mes de Rayab de 1276.}
\end{flushright}

\hypertarget{i-2}{%
\section{I}\label{i-2}}

En el nombre del Dios Clemente y Misericordioso.

He aquí la historia que para recreo del \emph{Cherif Sidi El Hach
Mohammed Ben Jaher El Zebdy}, escribe su amigo y protegido \emph{Sidi El
Hach Mohammed Ben Sur El Nasiry}.

Es esta la guerra del Español desde que apareció en el valle de
Tettauen, y se refiere con verdad y estimación natural de todos los
hechos presenciados por el narrador, para que los venideros conozcan la
brava defensa que de su religión venerada hacen los hijos de \emph{El
Mogreb El Aksá}.

Nuestros aborrecidos hermanos, los de la otra banda, los hijos del
\emph{Mogreb El Andalus}, avanzaron desde \emph{Sebta} hasta \emph{El
Medik}, sosteniendo combates terribles con nuestros valientes montañeses
y tropas regulares. El número de cristianos que perecieron en aquellas
refriegas no se puede calcular; los moros perdimos escaso número, y en
casi todos los encuentros quedábamos vencedores. El avance de los
españoles, tras tantos descalabros, y su paso de un terreno a otro, no
se explica sino por combinaciones astronómicas, mágicas y cabalísticas,
cuyo secreto tienen aquellos Generales y que los nuestros no han podido
penetrar. El enemigo consulta de día la marcha del Sol; de noche las
posiciones de los astros que esmaltan de bellas luces el firmamento, y
combinando estos signos con las cifras y figuras que en unos deformes
libros traen, del estudio de todo ello sacan la pauta de sus
movimientos, que siempre resultan hacia adelante, nunca hacia atrás.

Pero estas artes mágicas no les valdrán: para desbaratarlas y confundir
a los infieles, nos basta con las dotes singulares de nuestro caudillo
Muley El-Abbás, asistido de las bendiciones de Allah, que le tiene por
ejecutor de sus altos designios. Si es fuerte con su espada, no lo es
menos con sus oraciones. En ellas dice: \emph{«¡Oh profeta, excita los
creyentes al combate! Veinte hombres tuyos aniquilarán a doscientos
infieles\ldots»} En el alto de Kal-lalin, que los enemigos llaman
\emph{Torre Geleli}, tiene su campamento el hermano del Sultán, y desde
allí, con el milagroso anteojo de aumento que le regaló el Inglés,
observa las posiciones y movimientos de los infieles. Nada se le escapa;
no se mueve una mosca en el campamento cristiano, sin que nuestro
General se entere, asistido además por referencias que le traen
numerosos espías, ora renegados, traicioneros a su patria, ora fieles
berbiriscos que, fingiéndose locos o enfermos, van a mendigar al campo
español.

¡Loor a Allah único! He visitado al Príncipe marroquí en su lujosa
tienda: la confianza brilla en su noble rostro; ha preparado tan bien
sus planes, que ya no tiene nada que hacer, y espera tranquilamente que
el enemigo se mueva, para disponer salirle al encuentro y atajar sus
pasos. Confiado en la protección del Cielo, no sólo practica la oración
mañana y tarde a las horas que marca la ley, sino que recomienda a sus
\emph{ascaris} y a los jefes de ellos que ante todo cuiden de practicar
la oración\ldots{} En el momento del combate, mientras unos pelean,
otros deben rezar\ldots{} alternando en la matanza y en el rezo. Por eso
les dice: \emph{Allah es vencedor}\ldots{}

Los infieles ocupan su tiempo en ridículos preparativos. Han levantado
un fuerte que llaman de \emph{la Estrella}, donde se les ve afanados en
trabajos semejantes al trajín de las hormigas\ldots{} Sabemos que al
campo de O'Donnell ha llegado un Príncipe francés, emparentado con la
familia Real de Españai\footnote{El Conde de Eu.}; es hijo de un hermano
del esposo de la hermana de la Reina, y parece que trae la misión de
instruir a los españoles en ciertos particulares de la guerra del
Francés en Argelia\ldots{} inútil ciencia, pues lo que venció a los
argelinos fue su falta de fe y no el valor de la Francia. No hay
semejanza entre la Argelia y El Mogreb, pues este antes que militar es
creyente, y perdura en las vías de Allah\ldots{} Allah es la fuerza;
Allah es la astucia militar y el amparo de las naciones\ldots{}
Aguardamos, pues, tranquilos el choque de armas que ha de poner fin a
esta guerra\ldots{} Los infieles perecerán en las lagunas de
\emph{Guad-el-Gelú} como en las aguas del mar Bermejo pereció Faraón,
cuando iba en perseguimiento de los hijos de Israel, conducidos por
Moisés o \emph{Mouçá}.

Alabanzas a Dios Misericordioso, que ayer ordenó el movimiento de
nuestros Ejércitos. Queriendo ver de cerca la gloria del Islam, me
agregué al séquito del victorioso Muley El Abbás\ldots{} El día era
hermoso, día dispuesto por Allah con todo esplendor de luces y limpieza
de ambiente para que el triunfo fuera más visible en la tierra y en el
cielo. Muy temprano vino del campo español ruido de salvas. Nadie sabía
la razón de aquel cañoneo; yo, que por mis aficiones al estudio entiendo
un poquito de la historia de nuestros enemigos, expliqué el suceso
brevemente. El día de ayer corresponde a un día en que los cristianos
aclaman y santifican a los reyes suyos que se llamaron Alfonsos, y al
Príncipe heredero de la Corona, que también lleva este nombre\ldots{}
Desde que oyeron las salvas querían nuestros valientes guerreros
lanzarse a destruir el fuerte que los hispanos construían; mas el
General tuvo especial empeño en contenerlos, a fin de madurar el plan de
ataque, y disponer las fuerzas del modo más conveniente para quitar a
los españoles el fuerte. No cesaba de mirar al campo y a las posiciones
de ellos, como si con sus ojos asistidos del catalejo quisiera medir las
distancias, y anticipar los pasos de unos y otros. Yo admiraba su celo
por la causa de la fe, y la paciencia que ponía en ordenar sabiamente
sus disposiciones. Por fin, al filo de mediodía soltó \emph{El-Abbás} la
gente de a pie que se abalanzó contra la izquierda de los españoles, y
mientras estos respondían al ataque avanzando hacia nosotros, nuestra
Caballería se lanzó como tempestad para embestir por su flanco derecho a
los infieles. ¡Qué hermosa carrera la de tantos hombres a caballo,
enardecidos y locos de ira contra la usurpación! Caballo y jinete
parecían en cada uno de una sola pieza, y en esta un corazón ardiente
irradiaba el fuego de la pasión guerrera. Nunca vi Caballería más fiera
y gallarda. ¡Loor\ldots! La paz sea con el que sigue el buen camino.

Descollaban en aquel volador enjambre los \emph{facies} o jóvenes
voluntarios venidos de Fez, de \emph{Zarhun} y de \emph{Ait Yamuz}, con
vistosos arreos y pulidas armas, y furibundas ganas de morir por la fe.
A esta noble y distinguida tropa pertenece el ya famoso guerrero
\emph{El Horain}, apodado \emph{Abu-Riala}, que en las acciones de Cabo
Negro realizó prodigios de valor y temeridad sólo comparables, según se
dice, a las hazañas de los compañeros del Profeta. Cuentan que en lo más
recio de las peleas se arroja este divino \emph{Abu-Riala (el del duro)}
en medio de las filas enemigas, tremolando un pendón amarillo, sin otra
fianza que su esforzado corazón y el ardimiento de su caballo. El grito
de guerra, para llevarse tras sí a los que quieren ser émulos de su
valor, es este: \emph{Adelante; yo soy vuestro escudo invulnerable}.
Sobrenatural prodigio es que vuelva siempre sin que le causen la menor
herida ni las balas ni el acero de los españoles\ldots{} Debemos
explicar este milagroso caso por la protección que dan los invisibles
ángeles guerreros al bueno, al creyente y heroico soldado de Allah.

Desde mi puesto en el séquito del General contemplé la fogosa
Caballería. Los de vista larga que me rodeaban gritaron roncos de
entusiasmo: «Allí va el santo combatiente, el gigante \emph{Abu-Riala},
corazón de Dios y brazo del Profeta. Ved su estandarte amarillo; ved su
mano poderosa señalando al Cielo; ved la cabeza de su caballo hendiendo
las filas españolas.» Esto me decían que viera y mirara; mas yo no veía
sino una confusión de patas de animales, y de cabezas y brazos de
hombres corriendo en espantoso torbellino. Yo miraba más bien hacia mi
derecha, donde ocurría lo más interesante de la acción. Por lo poco que
vi y lo poco que me decían, entendí que un gran número de españoles se
metió en un terreno que había sido encharcado previamente, sangrando el
Alcántara. La risa que soltó el General me indicó que allí les quería
ver, y que la entrada de los españoles en los pantanos era el error por
él previsto, y por su astucia preparado para ganar fácilmente la
batalla\ldots{}

Las exclamaciones gozosas de nuestra gente indicáronme que estaban
cogidos en la trampa los pobres españoles, y que ya no teníamos que
hacer más que una cosa bien fácil: rematarlos allí tranquilamente y sin
riesgo. Mas lo que yo creí cacería de patos, fue cosa distinta: los
malditos patos, o sea españoles, formaron con gran presteza el cuadro,
táctica que no se ha enseñado a los de acá, y fortalecidos de este modo,
no pudo hostilizarlos la Caballería por la blandura del suelo en que
tenía que maniobrar. Quedaba, sí, el recurso de atacar el cuadro a pie:
ya iban a ello nuestros valientes moros; ya se cruzaban armas con armas;
ya caían algunos de allá con las cabezas hendidas, y los de acá con las
barrigas ensartadas\ldots{} Teníamos gente de sobra; podíamos dar cuenta
de ellos\ldots{} pero ¡ay!, Satán maldito, que rara vez deja de
introducirse en estas decisivas luchas, tomando partido por los
infieles, puso en movimiento a la muchedumbre de tropas del llamado
Tercer Cuerpo, para venir en socorro de los que tenían jugada la vida en
el pantano\ldots{} ¡Allah disperse a los injustos!

Aterrado vi yo las tropas a pie y a caballo que venían como a distancia
de dos tiros de fusil. Pareciéronme millones de hombres, y a medida que
su paso veloz acortaba la distancia, se me representaban en mayor
número. Con risa de júbilo, Muley Abbás y los que le acompañaban
exclamaron: «No pueden, no pueden llegar a socorrerlos\ldots» «¿Por
qué?\ldots» «Porque entre esas tropas y el terreno fangoso donde está el
cuadro no hay más que pantanos, lagunas hondas, donde perecerán sin
remedio. ¡Allah los precipite!» Evidente, como los hechos fatales de la
Naturaleza ciega, parecía esto; mas no lo fue, porque Satán perverso,
enemigo de los creyentes, lo arregló de modo que los españoles que
venían al socorro no temieran meterse en el agua hasta la
cintura\ldots{} Yo les vi, nadie me lo contó\ldots{} yo les vi atravesar
las charcas, alzando los brazos para que no se les mojaran el fusil y
los cartuchos que en sus manos traían\ldots{} y en esta postura hicieron
un fuego tan horroroso contra los nuestros, que no parecía sino que el
Infierno desataba toda su furia.

Personas prácticas del campamento, que ya conocen a todos los caudillos
españoles como si los hubieran parido, me contaron por la noche que
vieron al General Ros de Olano, al Brigadier Galiano, y al propio
General O'Donnell, atravesar la laguna con el agua hasta la cincha del
caballo, dando a todos ejemplo de valor, y arengándoles con voces roncas
para que no temieran al agua, como no temían al fuego. ¡Ah, sin las
artes infernales empleadas en favor vuestro por maléficos espíritus, qué
sería de vosotros, pobres hijos de España!\ldots{} Esto pensaba yo,
caído en gran tristeza al ver que nuestros montañeses bravos y nuestros
atrevidos jinetes \emph{facíes} se retiraban hacia las posiciones
próximas a \emph{Torre Geleli}; y buscando, según mi costumbre, la causa
recóndita de los hechos, me decía: «¿Cómo es que esas lagunas que
teníamos por profundas, y que lo eran según el dicho de hombres
entendidos en cosas de la Naturaleza, han resultado con hondura no mayor
que la de medio cuerpo de un hombre? Misterios son estos que no
desentrañaremos mientras no nos sea dado penetrar los designios del Dios
Único, que gobierna el mundo así en las grandes como en las pequeñas
cosas. Huir del examen y conocimiento de tales honduras es el verdadero
principio de sabiduría que debe guiar al hombre discreto y virtuoso.»

Pregunté por \emph{Abu-Riala}, no bien llegábamos a nuestras tiendas, y
me dijeron que había consumado aquel día descomunales proezas, matando a
multitud de cristianos, sin que le tocara el más leve rasguño. El corcel
que montaba fue menos dichoso: quedó muerto. Para consolar al guerrero
de esta pérdida, mandó Muley El Abbás que se le diese uno de los mejores
caballos que tenía para su servicio, y luego ordenó que las músicas
fueran a tocar junto a la tienda del héroe; honor y merced con que se
hacía pública la virtud y merecimientos de un hombre tan excelso. Hasta
hora muy avanzada de la noche oímos los dulcísimos acordes de las
chirimías, pitos y tambores que daban serenata al soldado del Cielo.

No obstante ser considerables las pérdidas del Ejército de la fe en
aquel día, no advertí descontento en los valientes soldados de a pie y a
caballo. Por la noche, comentando la batalla, predominaba la opinión de
que había sido victoria manifiesta, y no derrota como creían los menos
en número, y los mal pensados y agoreros. Cierto que no habíamos tomado
el fuerte de la Estrella; mas los cristianos no habían avanzado una
pulgada en sus posiciones\ldots{} Cada paso valle arriba les había de
costar muy caro\ldots{} Debíamos dejarles subir, internarse, para
exterminarles más a gusto. Esto decían. ¡Dichoso pueblo, que con el
fuego de la creencia en Dios enciende el de la confianza en sí mismo!
Nada teme: los obstáculos le enardecen. Nunca espera lo malo: sus ojos,
iluminados por la fe, ven con tintas de rosa y azul los días venideros.
¡Pueblo noble y santo, digno de dominar toda la tierra!

¡Loor al Muy Alto! Invitado a cenar con el Príncipe, encontrele sombrío,
como si no estuviera satisfecho del giro que llevaban las cosas de la
guerra. Contaba, sí, con mayor contingente de tropas, que el Sultán le
mandaría bajo la bandera del Príncipe Muley Ahmed Ben Abderrahman;
contaba con el valor indomable de los montañeses, de los \emph{facíes} y
demás elementos de su Ejército; mas no tenía tranquilidad, viendo la
creciente arrogancia de los españoles, sus obras de atrincheramiento, su
poderosa artillería, y la perseverancia calmosa con que iban
conquistando el terreno. A esto le dije yo, para consolarle y levantar
su ánimo, que la acción de aquel día me revelaba poca decisión de los
cristianos para seguir adelante. Aparentaban más fuerza de la que
tienen, y tras de su afectado coraje, se advertía el cansancio, y las
ganas de volverse a su país. Movió la cabeza Muley El Abbás con
expresión de tristeza dubitativa, y yo proseguí con mayor fuego de
persuasión: «Creed que si alguna ventaja obtienen los enemigos de Allah,
es porque Allah les favorece en apariencia para estimular el ardimiento
de los fieles. Así el Profeta, en sus luchas contra los traidores, no se
acobardaba ante los avances de estos, sino que les dejaba llegar hasta
donde podía destruirles sin que quedara uno solo para contarlo. En el
Libro Santo encuentro ejemplos mil de esta consoladora táctica del Único
Dios. Ya sabéis que está escrito: «Satán había preparado sus batallas, y
les decía: \emph{soy vuestro auxiliar y os hago invencibles}. Mas
llegado el momento, les volvía la espalda diciéndoles: \emph{Pereced
ahora y sufrid los terribles castigos de Dios}\ldots» Seguid leyendo, y
veréis que está escrito: «Hiriéndoles en el rostro y en el pecho, los
ángeles quitan en un punto la vida a todos los infieles\ldots{} y les
gritan: \emph{Id a gustar las penas del Infierno.»}

\hypertarget{ii-2}{%
\section{II}\label{ii-2}}

Y he aquí que el noble y sabio Príncipe me dice: «Pues eres tú creyente
fervoroso, y a más de esto sabio en cosas mil de la tierra y del cielo,
y tienes el don de elocuencia y gran influjo sobre las gentes, puedes
prestar ahora un gran servicio a la causa del Mogreb. Te vas a
\emph{Ojos de Manantiales}, donde tienes tu casa y estancia de tu
comercio, y ves si es cierto que están los habitantes inquietos y
afligidos porque algunos riffeños revoltosos han cometido el delito de
pillaje o saqueo\ldots{} Entérate de si las familias huyen de la ciudad
temiendo ya la entrada de los españoles. Tengo por cierto que los judíos
tratan de ir al campo cristiano en son de embajada para pedir a
O'Donnell que no se detenga y se haga dueño de Tettauen, sin otro fin
que proteger las vidas y haciendas de ellos, de los que recibieron las
Escrituras, para venderlas después a precio vil.»

---Cierto es---repliqué yo---que Dios ordenó a los judíos que explicaran
el Pentateuco a todos los hombres y no lo ocultaran. Mas ellos
comerciaron indignamente con los santos libros\ldots{} Pero un doloroso
castigo les espera.

---No les hables ahora de castigos---dijo vivamente el Príncipe,---ni
pongas en tu lenguaje rencor ni amenaza, porque a decir verdad, están
las cosas para que pongamos en práctica la conocida regla de ciencia
vulgar: \emph{Sé como el caracol en el consejo y como el ave en la
acción.} Usarás con los hebreos un lenguaje benigno y amistoso,
induciéndoles a permanecer tranquilos, sin ningún temor, y enterándote
bien de sus pensamientos y de sus planes, que por muy escondidos que los
tengan en el arca de su hipocresía, tú hallarás modo, con tu lenguaje
astuto, de sacarlos afuera.

No fue preciso que me dijera más el augusto Príncipe, y decidí partir a
la madrugada\ldots{} En \emph{Ojos de Manantiales} reanudo mi trabajo
epistolar, tres días después de lo que anteriormente referí. \emph{¡Loor
al victorioso!} Oíd lo que digo: en cuanto llegué a este santo pueblo,
no me di paz para ponerme al habla con los tetuaníes pudientes y con los
judíos altos y bajos. La verdad, a todos les hallé muy cariacontecidos.
Respecto a saqueo y desmanes de los montañeses, supe que sólo en el
\emph{Mellah} (barrio de los hebreos) habían cometido algún desaguisado.
Recorrí toda la ciudad; vi en algunas calles cofres y líos de ropa,
señal de que algunas familias partían; no traté de disuadir a nadie,
pues me habrían echado en cara que yo he mandado a los míos a Fez para
rescatarlos de todo mal\ldots{}

En mi casa, sin más compañía que la de la esclava que quedó para mi
servicio, he sentido la opresión del silencio, como losa que pesa sobre
mi espíritu. La soledad de mi vivienda, días antes embellecida y
alegrada por seres queridísimos, dábame la impresión de estar emparedado
en anchurosa tumba\ldots{} No había más ruidos que los que yo llevaba en
mi memoria: la risa jovial, cristalina, de mi adorada \emph{Puerta de
Dios (Bab-el-lah)}, en quien cifro todos mis cariños; el habla dulce y
discreta de mis otras dos mujeres, \emph{Quentza} y \emph{Erhimo}, a
quienes tengo también grande afecto, y más que nada el pisar rápido, la
inquietud traviesa y los chillidos deliciosos, como piar de pájaros, de
mi hijo \emph{Ali Ben Sur} y de mi encantadora niña \emph{Luz-il-lah}, a
quien Dios hizo archivo de todas las gracias. La fatal guerra me ha
obligado a separar de mí estas prendas queridas. Confinadas en Fez hasta
que vuelva la paz, mi pensamiento vuela sin cesar a donde ellas moran, y
trato de endulzar el amargor de la ausencia con la miel del
recuerdo\ldots{} Mi casa vacía de aquellas voces, vacía también de tan
bellas imágenes, arroja sobre mí la pesadumbre fría de sus paredes, que
no me deja respirar\ldots{} Sea Dios benigno, y no me prive de mis
mujeres y mis hijos. Ellas son buenas, recatadas, hacendosas. Superior
inteligencia y bondad resplandecen en la sin par \emph{Puerta de Dios},
dotada por mí con largueza y estimada en doscientas onzas españolas.

Me sobrepongo a la emoción para tomar disposiciones urgentes. Reviso mis
papeles comerciales para encontrar confusión en ellos cuando la paz
vuelva a nuestro pueblo; escribo a Fez ordenando que permanezcan allí
los camellos hasta mi aviso; dispongo que salga un propio con este
mandato, y por él envío a mis hijos y a mis mujeres cajitas con amorosos
regalos. Entrada la noche, me entrego al descanso; sueño con los tiros
que oí en la batalla junto a los pantanos\ldots{} oigo los alaridos de
\emph{Abu-Riala}\ldots{} corro perseguido por cristianos que quieren
hacerme prisionero\ldots{} despierto en las angustias de mi huida
fatigosa\ldots{} cojo un rosario, y en ferviente oración recibo los
consuelos de Allah, que con mano suave alivia mi corazón del anhelante
susto\ldots{} Por la mañana, después de los rezos y abluciones, salgo a
recorrer la ciudad; visito una tras otra mis tres casas alquiladas, para
saber si las abandonan sus habitantes; si alguno de ellos, al huir, ha
dejado la puerta mal cerrada; si en los pasadizos de las calles hay
hacinamiento de paja y estiércol. Me tranquilizó el ver que mis buenos
inquilinos permanecen en la ciudad. A los tres endilgué un largo
discurso sobre el peligro de los incendios en tiempo de guerra, y otro
con diversidad de razonamientos para llevar a su ánimo la persuasión de
que jamás entrarán los españoles en nuestra ciudad. Por las caras que
ponían oyéndome, entiendo que les convencí. Son hombres de grande
inocencia, por lo que Dios tendrá piedad de ellos.

Despachados estos asuntos, me dirigí al \emph{Mellah}. Mi primera visita
fue para \emph{Yakub Mendes}, traficante en piedras preciosas, mi amigo
desde que me establecí en Tettauen. Encontrele muy afanado, con su mujer
y sus hijas, recogiendo todo el material valioso que posee, aljófar,
topacios, esmeraldas\ldots{} Hacían paquetitos chatos que pudieran
fácilmente ser cosidos en la ropa interior, para transportar consigo
toda su riqueza en caso de forzosa partida. A Yakub y su familia
prediqué la tranquilidad, la confianza en el Mogreb para desembarazarse
de los españoles; pero no conseguí calmar su inquietud. Fácilmente había
convencido a los pobres, que no tienen nada que perder; pero a los
ricos, ¡Allah me conforte!, no podía convencerles. Díjome Yakub que él
conocía bien la fuerza de los españoles, por haber recorrido la
Península sin fin de veces, y vivido en Córdoba, Sevilla y Madrid
luengos días, y que no podía tener confianza en las fuerzas
desorganizadas del Mogreb. Tan cierto era que O'Donnell entraría en
Tettauen como que el Sol sale hoy, mañana y siempre; y el día de la
entrada de los vencedores, lo que no habían saqueado los riffeños, lo
saquearían los soldados de O'Donnell, a quien aplicó con malicia un
refrán hebreo que dice: \emph{ni ajo dulce ni todesco bueno}. Díjele yo
que no es el General español de origen tudesco, sino irlandés, y él
afirmó que lo mismo da, pues no tiene sangre \emph{andalús}, sino de
raza \emph{goética} y \emph{normándica}, que es la que más aborrece a
Israel\ldots{} En esto llegó a la casa un vecino de Yakub, llamado Ahron
Fresco, usurero y comerciante en especias y gomas de sahumar. De lo que
hablaron uno y otro colegí que la noche anterior habían celebrado una
junta, en la cual se debatió si debían pedir a O'Donnell que les
amparase contra los riffeños. No prevaleció tan traidora proposición, y
por ello debemos dar gracias a Dios. ¿Pero quién se fía de esa gente?
Con razón dice el Libro Santo: \emph{La confusión reina en los juicios
hebreos, y sus acuerdos son como los remolinos del aire}.

Sobre mis dos amigos descargué yo un diluvio de elocuentes razones,
incitándoles a que por ningún caso solicitaran la protección del infiel
español. Cuando más enardecido estaba yo en mi retórica, llegaron
\emph{Tamo} y \emph{Noche}, dos hebreas de aquella vecindad, muy guapas,
que tiraron de mí familiarmente para llevarme a su casa. No pude
esquivar la premiosa invitación, y pasando del tugurio de Yakub al de
\emph{Ha Levy Seneor}, padre de las antedichas, este, su mujer
\emph{Hanna} y las hijas, hablando los cuatro a la vez con desacorde
griterío, me contaron que la noche anterior habían asaltado su casa tres
desalmados riffeños, quitándoles veinte duros en moneda \emph{macuquina}
española, catorce pesetas columnarias, diez napoleones, y que por
milagro (no quiso Dios que dieran con el escondrijo) no les aliviaron de
la moneda de oro que guardaban. Después se surtieron de ropa blanca;
lleváronse los dos chales mejores de \emph{Tamo}, los zarcillos de
\emph{Noche}, que eran de \emph{filigre} de Córdoba, y unas
\emph{belghas} (babuchas bordadas de oro). Traté de aplacar su enojo
diciéndoles que desde hoy se reforzará la guarnición con gente de
confianza, y que todas las puertas de la ciudad se adornarán con las
cabezas de los saqueadores\ldots{} Sin detenerme a escuchar sus
lamentaciones airadas, me fui en busca de mi amigo Simuel Riomesta,
hombre rico, influyente sobre la caterva de Israel, y pensaba yo que
persuadiendo a este, los demás quedarían desarmados de su coraje y
repuestos de su miedo.

Iba yo por la calle más angosta y puerca del \emph{Mellah}, para salir a
la casa de Riomesta, cuando me sentí llamado por fuerte voz de mujer.
Era \emph{Mazaltob (Afortunada)}, hebrea viuda de más que mediana edad,
que desde su puerta echó sus gritos en mi demanda. Trafica en bálsamos
por ella misma compuestos, y tiene fama de hechicera o mágica, por su
acierto en adivinanzas y su buena mano para curar enfermos con garatusas
y oraciones, ayudadas de zumos de hierbas y raspaduras de huesos. En su
juventud fue, según oí, más cautivante por sus decires agudos que por su
hermosura. Lo que me habló fue de esta manera: «Te he llamado para
decirte que la otra mañana, estando yo en prado de Almorain arrecogiendo
herbas, topé a un mancebo ferido, que me demandó agasajo\ldots{} Yo
lastimosa le truje a mi casa, aonde me dijo ser español. Su nombre es
\emph{Juan el Pacificante}, y tié semblan de profeta\ldots{} Anda en
perjudicación de la paz, y del campo cristiano echáronle por sus
perdicas, y agora viene acá para que aproclamemos la paz y no la
guerra\ldots{} Él es bueno, es sencillo, y el habla tiene bonica
española, que adulza el oído. Entra y verasle.»

Sospeché que el español de que me hablaba \emph{Mazaltob} era espía, o
algún perdulario hambrón que viene so color de renegar para que le demos
de comer. Insistió la hebrea en que su huésped no era nada de esto, y
para calmar mis recelos me dijo: «Tú, que de achaque de españolerías
sabes más que nadie, habla con él y asóndale\ldots{} Yo no te asiguro
que sea profeta; pero sí que por el su semblan y por su voz cantora lo
parece. ¿No hubieron los cristianos un profeta que se llamó Juan? Pues
cata que este es lo mesmo, o que viene en figuranza de quillotro\ldots»

---El profeta cristiano que dices es el que llamamos \emph{Yahia}, hijo
de Zacarías, varón de extremada virtud. Este será todo lo contrario: un
pillastre, un embustero\ldots{} Pero si, como dices, viene del campo de
O'Donnell, no será malo que yo le coja por mi cuenta y le interrogue.
Llévame pronto a la presencia de ese mancebo predicador de paces, que
con verdades o con imposturas algo ha de decirnos que pueda sernos útil.

Cogiéndome del albornoz me metió adentro por obscuro pasadizo hasta una
estancia humilde, y oliente a comida pasada, donde paredes y mueblaje
parecían trasudar materia grasienta. Adelantose ella por otro pasadizo,
y luego volvió con estas razones: «Se ha quedado adormilado. Hoy anduvo
luengas horas por la cibdad, calle adelantre, calle adetrás, y ha venido
con cansera\ldots{} Pero puedes entrar y verasle. Todo en él yace como
muerto, menos la respiración, que vela como guardián en las puertas del
rostro, boca y nariz, y ella es la que avisa cuando el ánima ida quiere
volver a su casa.» Entré con \emph{Afortunada} en una estancia que de un
patio sucio y ahumado recibía la luz, cernida por cortina roja, y sobre
una cama que alzaba poco del suelo vi una estirada figura de hombre,
derechamente tendida en todo su largo. Era el durmiente de poquísimas
carnes y de más que mediana estatura, bien formado de esqueleto y
miembros, por las partes que de él se veían. Pecho y brazos tenía
vestidos de una \emph{kmiya}, y sobre ella un \emph{caftán} amarillo
rayado, que se recogía en la cintura y muslos, dejando ver las piernas
al aire. Su cabeza me pareció perfecta; bello y afilado el rostro, con
una barba leve, que más parecía pintada que nacida. Barba y pelo eran
negros, y el color de la piel como el de madera de olivo, con ligero
bruñimiento y lustre de cosa embalsamada.

Yo me senté, pues muy a propósito hallé un taburete junto a la cama.
\emph{Mazaltob} me dijo: «Hablemos en voces altas para que se acuerde,»
y rompió en gritos\ldots{} No pasó mucho tiempo sin que el dormido
despertara, lo que sucedió abriendo él los ojos, y quedando rostro,
cabeza y cuerpo en completa inmovilidad. Primero vio y miró a su
patrona, después a mí, y su mirada estuvo posada en mí largo tiempo, sin
querer desclavarse de mi faz\ldots{} Hablele yo en árabe preguntándole a
qué había venido, y él no respondió con discurso, sino con una rápida
incorporación, clavándome otra vez los ojos, negros y con luz como los
carbones encendidos. De veras me hizo pensar en el profeta cristiano
\emph{Yahia}, hijo de Zacarías, en quien Dios puso el signo de su
predilección, y de él dice el \emph{Libro Santo}: \emph{Escogido fue
para enseñar a los hombres la paz}.

\hypertarget{iii-2}{%
\section{III}\label{iii-2}}

Como no daba señales de entender el árabe, le hablé en su lengua,
obedeciendo a \emph{Mazaltob}, que me decía: «Háblale en español bonico
y de son pacible.» Sentado en el lecho, \emph{Yahia}, sin pronunciar
palabra, me tocó en el brazo, en la rodilla, como si quisiera con el
tacto completar el examen que sus ojos hacían de mi persona. Por fin oí
el metal de su voz. A mi pregunta de si le gustaba nuestra tierra,
contestó que le agrada porque en ella todos los hombres se tratan de tú,
señal de la completa igualdad ante Dios, y porque el Islam y el Israel
practican su fe sin estorbarse el uno al otro. Esta paz entre las
religiones le sorprendía y le encantaba. Después me dijo: «Oigo tu
lenguaje como una música triunfal, y veo tu rostro como un rostro
amigo.»

A mi pregunta sobre los motivos de su peregrinación, respondió que había
huido del campo español porque le agobiaba el alma el espectáculo de la
guerra, y la ferocidad con que unos y otros hombres acuden a matarse. La
guerra va contra la Humanidad, como el amor en favor de ella. Las armas
destruyen las generaciones, que son reedificadas en el seno de las
mujeres. Puede la Humanidad vivir sin armas; sin mujeres no
vivirá\ldots{} En verdad declaro que esto me pareció dictado por la más
alta sabiduría. No pensé lo mismo después, cuando dijo cosas tan sin
sentido como estas: «Por tu cara y gesto, por la forma de tu nariz y de
tus labios, así como por la voz y el mirar luminoso, mi pensamiento te
liga con tu noble familia.» Sin duda la mente de \emph{Yahia} era una
extraña mixtura de pensamientos celestiales y de bajos yerros humanos,
porque tras una hermosa invocación a la paz como ley superior de los
hijos de Adán, soltaba este desatino: «Tú no quieres la guerra, ni
bajarás con arma homicida al campo de O'Donnell, porque en el campo de
O'Donnell está tu hermano.» Sin duda quería decir que entre todos los
nacidos existe el lazo de hermandad, y verdaderamente concuerda esto con
lo que dice la Escritura: «No hacemos diferencia entre los enviados de
Dios. Todos los que adoramos un Dios Único y le tememos, vamos a ti,
Señor, y entraremos en los jardines de inefables delicias.»

Por fin, requerido a darme noticia de los planes de los españoles y de
los medios que traen para combatirnos, dijo que él, después de haber
sido voceador de la guerra, había pasado por la gran revolución de su
espíritu, viniendo a detestar lo que antes adoraba. En el Ejército tenía
muchos amigos, y en Madrid dejó personas muy amadas, que también eran
afectas a la tradición guerrera y a las glorias de su patria. Él no
estimaba esas glorias como legítimas, y buscaba otras en armonía con la
Naturaleza humana, deseando ver extinguida la ferocidad, los instintos
de destrucción\ldots{} Suspira por la paz, por el amor entre todos los
humanos y la universal concordia\ldots{} No estaban estas ideas en
desacuerdo con las mías, pues yo pienso lo propio, si bien entiendo que
todavía no ha llegado el tiempo en que nos convenzamos los hijos de Adán
del desvarío de las guerras. \emph{Yahia} tan pronto iluminaba con
resplandores divinos nuestra conversación, como la obscurecía con
disparates manifiestos. Preguntome si había estado yo en la acción de
los Castillejos; respondile que no, y él dijo: «Razón tuve en creer que
no eras tú el que vimos, vivo primero, muerto después. Nos alucinó el
terror de aquellos espectáculos de matanza, y en sueño nos visitaron
imágenes ensangrentadas de los seres queridos.»

---Aunque tu misión en el mundo---le dije,---más bien es ver fantasmas
que predicar la paz, dame una idea de los planes de O'Donnell, que algo
has de saber, si en el campamento cristiano tenías amigos. ¿Crees tú que
los españoles romperán y desbaratarán la grande hueste marroquí que les
cierra el paso a esta ciudad?

---La romperá y desbaratará como el cuchillo deshace esas paredes de
cañas con que cercáis vuestros huertos. El moro es valiente, pero no
sabe nada de artes de guerra. Sus armas son primitivas, o de sistemas
diferentes si algunas tienen modernas. Los hombres no saben formar
cuerpos tácticos, y el valor, en vez de concentrarse y unificarse,
tiende a esparcirse y desmenuzarse en infinidad de actos aislados. No
hay Jefes, no hay Generales, no hay organización, no hay cabeza\ldots{}
Imposible la victoria del Mogreb.

No pude contenerme. Levanteme, y con voz colérica le mandé
callar\ldots{} le amenacé si no callaba. Él con humildad, inclinando la
cabeza, respondió: «Me has pedido mi opinión y te la he dado. En mi
opinión he puesto la verdad: nunca pensé que la verdad te ofendiera.»

---¿Te atreverás a sostener delante de mí que O'Donnell se abrirá paso
hasta la ciudad y entrará en ella?

---Sin ofensa para ti ni para el Mogreb, yo digo que O'Donnell entrará
en Tetuán antes de ocho días. Sus planes, como de General que todo lo
calcula, y que pesa y mide toda contingencia, son infalibles.

¡Loor al Dios Único! Comprenderás, noble señor, cuánto me indignó el
vaticinio del desquiciado \emph{Yahia}. Le increpé con altas voces, y si
no estuviéramos en ajena casa, habría castigado su atrevimiento\ldots{}
Todo lo que le dije fue en lengua árabe, porque el español que sé no me
sirve para incomodarme. Él se quedó en ayunas de mis imprecaciones, y yo
salí de la estancia ofendiéndole con el gesto desdeñoso tanto como con
las palabras. En el pasadizo estrecho, camino por donde divagan los
malos olores, me detuvo \emph{Mazaltob}, y poniéndome en el pecho sus
manos crasas, me dijo: «No hagas ofensión a \emph{Yahia}, ni le amotejes
con griterío, porque él es bueno y hate dicho verdad\ldots{} Tan cierto
como ahora es día, Donell entrará en Tettauen\ldots{} Ven y veraslo
agora en sinos que nunca marraron.» Desmayada no sé cómo mi voluntad,
dejeme conducir a un aposento, en el cual tenía la oficina de sus
inmundos hechizos. Vi fuego en un anafre, agua en varias redomas; vi
lagartos vivos, papeles con endiabladas escrituras, y un círculo de
metal con signos astrológicos, que giraba entre agujas negras y verdes.
«No quiero, no quiero ver tus artimañas sacrílegas,» grité desprendiendo
mi albornoz de sus uñas. Y ella a mí: «Cuando te profeticé, años ha, que
serías rico, que de onde vien el Sol vernían para ti ochenta camellos
menos uno, e ainda te dije que en tal luna te serían dados doscientos
ducados de oro, bien lo creíste, y bien se enjubiló tu ánima viendo que
era verdad mi adivinancio, con merced del Alto Criador.»

---Déjame; no creo nada---repetí, anhelando zafarme de ella; pero no me
valió mi deseo, porque la maldita me puso delante una tableta con sin
fin de rayas y garabatos, los cuales, vistos al revés, eran la propia
figura del número 18, y debajo estaba escrita en arábigos caracteres la
palabra \emph{Tzementhash} (diez y ocho). Me mostró luego una redoma con
agua teñida de amarillo, en la cual flotaban varias hojuelas de
plantas\ldots{} Agitó la redoma; corrían las hojuelas dentro del agua
como traviesos pececillos, y una salió a la superficie tiñéndose de
color de rosa\ldots{} Pues bien: la cifra y este juego de las hojuelas
en la redoma querían decir que \emph{el día 18 de Schebah} (mes
corriente en el calendario judiego) entrarán los españoles en Tetuán. De
sus profanas manipulaciones, invocando a Satán, sacó \emph{Mazaltob} la
siniestra profecía, y se obstinaba en que yo había de creerla. Ella,
como profesora en brujerías y artes satánicas, lo creía o afectaba
creerlo, diciendo: «Que muerta me caiga yo ahora mesmo si no es la vera
palabra de Dios que el día 18 de \emph{Schebah} serán ellos en Tettauen,
El Donell y El Prim\ldots{} Créeslo tú; mas no lo dices por no adolorar
a los tuyos.»

«¡Guárdeme Allah Misericordioso de las asechanzas de Satán el Pérfido,
el Corruptor de Adán y de toda su prole!» Con esta exclamación arrojé de
mi lado a la impostora, dándole un empujón que la hizo vacilar sobre sus
pies como la estatua sacudida por terremoto, y salí de su casa. En la
puerta, mujeres hebreas y chiquillos de la misma casta gritaban: «¡Paz,
paz!» azuzándome con burla. Seguí mi camino sin echar una mirada sobre
tan ruin caterva, y doblando la esquina me dirigí a la casa de Riomesta,
una de las pocas que en el \emph{Mellah} reciben al visitante con olor
de sahumerios, y así previenen nuestra respiración en favor de los
dueños. En el patio estrecho me recibió la hija de mi amigo,
\emph{Yohar} (Perla), hermosa joven que cautiva por su ideal blancura.
Díjome que su padre estaba en la Sinagoga, donde tenían reunión los
Principales para tratar de su defensión\ldots{} Añadió la buena moza que
había venido una orden de Muley El Abbás, prohibiendo a las familias
tetuaníes ausentarse de la ciudad. Nada de esto sabía yo; mas lo tuve
por cierto, y la medida me pareció acertada, pues la fuga de los ricos
era mayor pánico de los que quedaban, y fomentaba el ladronicio y
pillaje\ldots{}

¡Loor al Grande, al Dueño de todo el Universo!\ldots{} Estas novedades
desviaron mis propósitos del camino que llevaban, y prometiendo a
\emph{Yohar} que volvería para platicar con su padre, salí del
\emph{Mellah}, y me fui en busca de los moros de más cuenta y poderío,
cuya opinión necesitaba conocer. Visité a \emph{Brisha}, después a
\emph{Erzini} y a \emph{Ibn El Mefty}, que son los más acomodados. Los
tres me dijeron que la orden de Muley Abbás les parecía bien; pero que
ellos no la obedecían, mirando sobre todo a la seguridad de sus
familias. Se marcharían, pues, desafiando las iras del \emph{Kaid}, pues
maldito lo que confiaban en que la plaza, con cañones viejos, artilleros
inhábiles y una guarnición insubordinada, pudiera defenderse y amparar
los intereses de sus moradores. Que estas manifestaciones llenaron mi
alma de tristeza, no es menester decirlo. Religión ¿dónde estás?\ldots{}
¿Qué víbora se anida en el pecho de los que debieran ser tus defensores?
¿El egoísmo y el ansia de guardar las riquezas tienen su asiento donde
antes lo tuvieron las virtudes? ¿Qué haces, Allah potente, Allah
Soberano \emph{el día de la retribución}?\ldots{} Andando de calle en
calle, la suerte me hizo topar con uno de mis más respetables
convecinos, \emph{El Hach Ahmed Abeir}, natural de Tánger, establecido
en Tettauen, el cual me saludó cariñosamente en español, pues esta
lengua es muy de su agrado, y sabiendo que la poseo, en ella me habla
para ejercitarse y no darla al olvido. Díjome que aunque todos los
pudientes salgan, él se quedará, suceda lo que sucediere, conforme a los
designios de Allah Fuerte y Misericordioso. Más temía de los soldados
riffeños que guarnecen la plaza, que de los españoles que amenazan
meterse en ella.

Por no enojarle, creí de mi deber aparentar cierta conformidad con Ahmed
Abeir, a quien debo acatamiento, pues son grandes el respeto y cariño
que todos, pobres y ricos, le tienen en la ciudad. La conversación
recayó luego en los judíos, de quienes podía temerse que hicieran algo
destemplado y fuera de la decencia. Díjome que él hablaría con el
\emph{Rabbí}, y que no descuidara yo el apaciguar a Riomesta y a otros
pudientes del \emph{Mellah}\ldots{} He aquí por qué torné a la Judería,
donde tuve la desgracia de volver a encontrarme con la embaucadora
\emph{Mazaltob}, acompañada del borriquero que la sirve, un hebreo
revejido, sarnoso y casi enano que se llama Esdras Molina. La
nigromántica, que a Satán tiene por maestro, entregaba al dueño del asno
líos de ropa para que los transportase a un huerto próximo al Santuario
de Sidi Sideis\ldots{} Al verme, soltó con áspero chillido la brutal
sentencia extraída de sus diabólicas alquimias: \emph{18 de
Schebah}\ldots{} y se metió como escurridiza culebra en la casa de Ahron
Fresco. Solo ya frente a Esdras, le detuve, conteniendo por el cabezal a
su tranquilo burro, que me agradeció la parada. Sabía yo que aquel
desdichado escuerzo de Israel había vivido en Ceuta algunos años; que
desde Cabo Negro andaba rastreando la retaguardia del Ejército de
O'Donnell, ya para merodear lo que cayese, ya para traficar con los
proveedores, llevándoles limones y naranjas, tal vez alguna pieza de
caza\ldots{} Los cantineros y él se entendían, y recíprocamente se
ocultaban sus latrocinios y contrabandos\ldots{} Aunque no confiaba en
que de los envilecidos labios de Esdras saliese la verdad, le
interrogué\ldots{} Si su borrico hablara, me daría quizás informes más
verídicos que los de su amo; pero como el animal callaba su hondo
pensamiento, con el otro tuve que entenderme, recordando aquel sabio
versículo del Libro Santo que dice: \emph{La boca del mentiroso deja
escapar la verdad}.

Pidiéndome que le anticipara el precio de las declaraciones que me
haría, y aflojadas por mí dos pesetas columnarias, Esdras me contó que
los españoles habían desembarcado un tren de batir, cañones relucientes
al sol, y unos montajes tan bonitos que daba gloria verlos. Pero él,
Esdras, lo había examinado bien. ¡Todo farsa y aparato de mentira! Los
cañones eran de un metal que parecía latón, y el día en que con ellos se
hiciera fuego, los artilleros saldrían volando por los aires\ldots{}
«Ainda, no tien polvra---prosiguió el borriquero.---La polvra de cañón
que vino de España en el barco que trujo los mantenimientos, no arde en
el Marroco, porque el aire y el fogo del Marroco son otros fogos y otros
aires\ldots{} Yo lo sé, yo lo entiendo\ldots{} Ainda, la Reina española
Isabela dice que no quié guerra más; que la guerra aumenta sus pecados,
y los clergos de España perdican que no más guerra\ldots» Acabó su
informe diciendo que los españoles no harían ante los muros de Tettauen
más que una simulación de batalla, y se tornarían para su tierra\ldots{}
Esto dijo aquel indino, cuya palabra oí con repugnancia\ldots{} Pero
algo hay de verdad en lo de que la pólvora española no arde en África
tan bien y con tanto fogonazo como allá, por ser nuestro aire diferente
de aquel; opinión que oí manifestar a un sabio de aquí, muy docto en
cosas físicas y matemáticas\ldots{}

Te cuento, señor mío, estas particularidades, porque me encomendaste que
al par de los hechos de la guerra pusiese en mis cartas copia fiel de la
opinión de la gente. Opinión larga hallarás en mis renglones, sabio y
prudente señor, para que juzgues por ti mismo lo que aquí sucede. La
resultancia de todos estos hechos y opiniones no la sabemos. Es locura
querer penetrar los santos designios. Concluyo por hoy repitiendo estas
sublimes palabras del Profeta: «Si Dios no contuviera a las naciones
unas con otras, la tierra sería corrompida. Los beneficios de Dios no se
manifiestan en las naciones, sino en el Universo\ldots» Y yo digo: «Si
Dios da la victoria a los infieles, es porque así conviene al Universo.
La justicia nos será conocida el día de la resurrección\ldots{}
Esperemos tranquilos ese día.»

\hypertarget{iv-2}{%
\section{IV}\label{iv-2}}

¡Loor al Dios Único!

La paz sea contigo, y la Misericordia de Allah con bendiciones.

Volví, como decía, a la morada de Simuel Riomesta, que es uno de los
hebreos más ricos de esta ciudad, amigo de los que bien pagan, prestador
de dinero con grande seguridad, acechante de los engañadores y
perseguidor inexorable de tramposos. Conmigo tuvo siempre miramiento
grande, acudiendo solícito a facilitarme plata y oro cuando mis negocios
me ponían en algún compromiso transitorio y urgente. Su opinión de mí y
su confianza en mi crédito corresponden a mi puntualidad: nunca hemos
tenido la menor cuestión. Añado que si es Simuel el hombre de más
formalidad y rigor en los negocios de préstamos, no hay otro más rezador
y cumplidor de los preceptos de su ley. Según me han dicho, es el
primero que entra en la Sinagoga los viernes por la tarde y sábados por
la mañana, y el último que sale: tiene permiso para pronunciar lección
en fiestas señaladas. En los días de Kypur sale descalzo, conforme marca
la ley, y practica el ayuno con verdadero fervor, que parece un deleite.
En \emph{Ros-Ashanah}, en las Vigilias de \emph{Purim}, \emph{Taanit},
\emph{Schabuot}, la observancia del culto y la práctica de todos los
ritos le aleja de sus negocios más de lo preciso, y en el \emph{Sucot},
o fiesta de \emph{Las Cabañas}, arma en su azotea las frágiles chozas
para dormir en ellas, y salir tempranito a mirar al Oriente, esperando
la aparición del Mesías.

Al entrar en el patio de su casa, me sorprendió el rumor de ásperos
rezos que de la estancia salía, y dije a la blanca \emph{Yohar}, que me
recibió muy risueña: «¿Pero a tu padre, después de pasarse medio día en
la Sinagoga, aún le quedan ganas de rezar?»

---No se harta de oración el padre---me respondió la del color de las
azucenas (que Allah le conserve),---para que el Dio de Dioses nos
desaparte guerras y calamidades.

No pude contenerme, y llegándome a la puerta por donde salía la
salmodia, vi a Simuel, con otros dos usureros, uno por cada lado,
berreando devotamente. Libro en mano, llevaba mi amigo la voz principal
de una recitación judaica, al modo de letanía, y a cada frase que él
pronunciaba, respondían los otros con bronca voz: \emph{Bedil
vayahabor}. Sonaba en mis oídos este estribillo como si me dieran con un
hierro en la cabeza\ldots{} Interrumpí sin reparo el rezo, gritando a
Simuel desde la puerta: «¡Eh! Riomesta, que estoy aquí. No es cortés
recibir a los amigos con esos graznidos lúgubres\ldots{} Parecéis aves
de agüero malo. \emph{¡Con doscientos y el portero}, vuestra cancamurria
da dolor de cabeza! Suspende la matraca y ven acá un momento.» Con la
mano hízome señal de que esperase, y siguió echando los fragmentos del
salmo, a que contestaban los otros con el machacante \emph{Bedil
vayahabor}.

Salió al cabo de un rato mi amigo, y mirándome por encima de sus
antiparras, que resbalaban por el caballete de su nariz, me dijo: «¿Qué
quieres, mi señor?» Y yo: «No te necesito para un solo fin, Simuel; pero
empiezo por el primero: has de darme doscientos duros en oro.»

---¿Cuándo?\ldots{} y la paz sea contigo.

---Ahora mismo, y tu paz te sea dada.

---Siempre vienes premoroso. Para servirte, heme quitado otros días el
pan de la boca, y agora me quito el rezo santo.

---Bastante has graznado ya, y bien segura tienes el alma contra el
fuego eterno. Sabrás que no me voy de aquí sin los doscientos de
oro\ldots{}

---Oye de mí, \emph{Yohar}: toma la llave, sube y cuéntale a El Nasiry
doscientos de oro, en el entre que acabamos el cántico. Y tú, cuando
bajes, me harás el recibo.

Subí con \emph{Yohar} a un aposento en que está el arca del dinero,
entre las estancias donde duermen el padre y la hija. ¡Loor a Allah, el
Indulgente y Bondadoso! Me agradaba lo indecible verme solo junto a la
mujer cuya blancura me enamoraba; blancor de rostro y manos, albor
visible en el cabo de pierna y en los pies medio escondidos en las rojas
babuchas bordadas de oro. El tilín del dinero que \emph{Yohar} contaba,
y la blancura de esta, que a la de los jazmines eclipsaría, me llenaban
de gozo. Recordé las santas palabras: «Allah es quien hace germinar los
seres en el seno de las madres. Él ha colgado las estrellas en el Cielo,
para que os guíen en la obscuridad. Él ha creado las flores, las
palmeras y mil frutas delicadas. Es el \emph{Sutil} y el
\emph{Instruido.»} El arrobamiento a que me llevaron el tilín del oro y
la belleza nítida de \emph{Yohar}, era turbado por el rezongar de los
ancianos, que desde la planta baja subía. En mis orejas seguía zumbando
el insufrible \emph{Bedil vayahabor}.

«Gentil \emph{Yohar}---dije a la moza,---¿cuándo te casas? Oí que has
desechado a muchos pretendientes\ldots{} Acabarás por fugarte con un
pelagatos, con un cristiano español\ldots{} o conmigo.»

---Contigo no, El Nasiry---respondió con voz blanda.---Eres casado.
Cuatro mujeres y cuatro esclavas son tuyas por merced de tu Dios\ldots{}
Toma el dinero, y no me apellizques el brazo con melindre, que esta
carne no es para tu sabor.

---Ya sé que será para el sabor de los ángeles\ldots{} ¡Loor al
Glorioso!\ldots{} De veras siento que seas judía. Toda tu blancura se
desleirá en la mugre de Israel.

---No blasfemes. Si mi padre te oye, no te hablará en son de amigo.

---Más que por sus riquezas debe tu padre mirar por ti, si la guerra
sigue. Corre tanto peligro como el oro tu blancura. La codician los
españoles que vienen hambrientos de mujeres.

---Ni mi padre ni yo tememos a los del \emph{Andalús}, que son
caballeros valientes, y barraganes muy cumplidos.

---Los del \emph{Andalús} quemaron en España a tus abuelos, y aquí te
derretirán a ti, como alba cera, en el fuego que traen. Vente conmigo a
Fez y te salvarás de la quema.

---Vete a Fez tú y tu generación, y déjame a mí, que \emph{bien está en
el peral la pera; cada cosa en su puesto, y la masa en el Pesah}\ldots{}

Bajábamos, y nada más pudimos hablar, porque salió a nuestro encuentro
Simuel, presuroso de que le extendiese y firmase el pagaré, como lo hice
en la estancia donde él y los otros rezaban. En cuanto examinó el papel,
quitose las antiparras sacándolas por la nariz adelante: tan sólo usa
los vidrios para poner aumento y claridad en la letra de los libros de
devoción o de los documentos de crédito. Luego, respondiendo a mis
exhortaciones para mantener la fidelidad al Mogreb y la confianza en su
fuerza, me dijo que los judíos, o no tienen ninguna patria, o tienen
dos, la que ahora les alberga y la tradicional: esta es España. De allá
provienen él y los suyos: su antecesor Abraham Riomesta había sido
\emph{Recabdador} de las Alcabalas y Tercias reales en la Aljama de
Talavera. Verdad que de allí se les echó, y algunos de su propia familia
fueron quemados públicamente, otros quedaron en Castilla con el nombre
de \emph{conversos} o \emph{marranos}\ldots{} Pero de entonces acá, ya
no había en España inquisidores ni tostamiento de personas. \emph{Onde
que por ello} ya no tenían los hebreos rabia contra españoles, ni
miraban \emph{como enemiga dañante la potestanía de España}. Añadió que
en Ceuta, habiendo pasado meses largos con su hijo Rubén, avecindado en
aquella plaza, tuvo ocasión de tratar con gran cuenta de españoles, y en
todos ellos encontró amistades, cortesía y fina voluntad. Militares y
civiles conoció, muy cumplidos y \emph{barraganes}. A muchos prestó
dinero, y ellos, que de España venían necesitados, por ser aquella la
tierra de la necesidad, no se asustaban por cuantía de réditos, y en el
pago eran liberales, dando ganancias sin que hubiera precisión de andar
en \emph{perjudizios}\ldots{} \emph{Ainda}, su hijo Rubén le ha escrito
cartas diciéndole que Echagüe y O'Donnell ordenaban a sus tropas el
respeto de las religiones islamita y mosaica, amenazando castigar a los
que hicieran daño en mezquitas y sinagogas, y ambos Generales, lo mismo
que Prim y Zabala, prometido habían amparar vidas y haciendas de moros y
hebreos.

A estas razones contesté yo con otras, infundiéndoles el recelo y
desconfianza de los cristianos; mas no se daban a partido: lo que afirmó
Riomesta fue apoyado por uno de los vejetes que le acompañaban en sus
rezos, \emph{Ahron Fresco}, el cual se dejó decir que había recibido
recaditos de españoles solicitando préstamos, que se harían efectivos al
ocupar la plaza. Comprendí que nada podía con aquella gente sin fuego de
patria en el corazón. Les dejé con desprecio y repugnancia. Al salir,
despedido en la puerta por la blanca \emph{Yohar}, oí de nuevo los rezos
lúgubres, y recordé las palabras del Profeta: «Escrito está que sus
corazones se petrifican en el egoísmo\ldots{} Está escrito que cuando se
hayan quemado en el Infierno, se les pondrá nueva carne y nueva piel
para volver a quemarlos.»

Al salir vi que a la casa de Riomesta se llegaba la embaidora
\emph{Mazaltob} con un ramo de hierbas aromáticas y medicinales que no
dudo serían para \emph{Yohar}. ¿Estaría en aquellas plantas el secreto
de la extremada blancura de la joven hebrea?\ldots{} Pensé yo que la
ciencia llamada Botánica por los infieles ofrece medios de encender el
amor en las naturalezas frígidas y aplacadas. ¡Oh, \emph{Yohar},
guárdate de la hechicera y de sus diabólicas artes! Estos pensamientos
me llevaron lejos del \emph{Mellah}\ldots{} dirigime hacia la Alcazaba,
y en el camino tuve el disgusto de ver que una de mis casas había sido
abandonada por el moro inquilino, y que este se había llevado la puerta,
arrancándola de sus goznes. Era ya mi casa albergue de mendigos y vagos,
que me la llenaban de su inmundicia. Indignado, traté de arrojarlos de
allí; mas ningún caso me hicieron. En la Alcazaba vi al \emph{Kaid}, que
en buenas palabras me expresó sus graves apuros para contener a la gente
pobre, que se había hecho dueña de la ciudad. El principal cuidado de él
era sostener el orden y atender al aprovisionamiento de las tropas de
Muley El Abbás.

Allí me encontré al venerable \emph{Hach Ahmed Abeir}, también con
achaque de reclamaciones, que por un oído le entraban al \emph{Kaid} y
por otro le salían. Entristecidos bajamos mi amigo y yo al \emph{Zoco},
donde vimos turbamulta de montañeses que se quejaban de no tener con qué
alimentarse; algunos tetuaníes pedían armas, y con ira ponderaban la
voracidad de los cristianos, que todo se lo comían y no dejaban nada
para los pobres moros. Había visto recaderos judíos que cargaban de
víveres sus burros y los llevaban al \emph{Sbañul}\ldots{} No pudimos
permanecer allí, porque el vocerío de aquella infeliz gente nos
agobiaba. Quiso \emph{Ahmed} llevarme a visitar las baterías de la plaza
y sus cañones y artilleros; pero a ello me resistí, previendo mayor
desengaño del que ya ennegrecía mi alma. Despedime del respetable señor,
encomendándole a la misericordia de Allah, y me salí solo por \emph{Bab
Echijaf}, para irme a Samsa, donde contaba pasar la noche y aun
descansar algunos días en casa de un amigo. Muy necesitado me sentía de
respirar aire campestre, y de espaciar mi vista por las hermosuras que
prodigó Allah en este rincón del África, sin duda destinado a que en él
tuvieran su Paraíso Terrenal los predilectos.

El alma, sobrecogida por los siniestros augurios que en la ciudad oí, y
por mi temor de la derrota del Islam, se me ensanchó al contemplar las
risueñas colinas próximas, el lejano y majestuoso \emph{Djibel Musa},
coronado de nieve, y al recibir en mis pulmones el aromoso ambiente que
de los montes venía. Ya los almendros empezaban a vestirse de sonrosada
blancura; ya el suelo se cubría de menudas florecillas; ya diversas
plantas daban señales de la temprana germinación, por la cual África es
maestra y precursora de Europa en la labor de la Naturaleza\ldots{}
Nunca me pareció tan bello este suelo de bendición; nunca oí con deleite
tan vivo el murmullo de los arroyos que del monte descienden; nunca
admiré con tanto fervor la obra de Allah, que creó toda la tierra y los
cielos sin el menor cansancio\ldots{} Todo el camino hasta Samsa lo
recorrí en muda contemplación. La obra de Dios no ponía ninguna parte de
sí en la guerra que nos asolaba: bosques y peñas, montes y colinas eran
indiferentes a los combates entre hombres, y si algo decían, era
\emph{paz} y siempre \emph{paz}. Mirando las sierras elevadas, que como
ningún otro signo expresan la grandeza del Criador, pensé en el Día del
Juicio\ldots{} «En aquel día---dice la Escritura,---Allah dispersará los
montes como polvo, para que toda la tierra sea inmensa planicie, por la
cual irán avanzando los hombres resucitados. Condúcelos ante el trono
del Juez el ángel \emph{Israfil}\ldots{} Avanzarán los hombres en
falanges, y no se oirá más que el ruido de sus pasos.» Ante la majestad
del Juicio supremo, ¿qué significa esta guerra, ni cien guerras, ni las
riñas y trapisondas en el rebaño de Adán?

En Samsa me hospedó mi grande amigo \emph{Mohammed Requena}, anciano de
luenga barba blanquísima, encorvado ya por el peso de los años, pero con
el entendimiento y la mirada fulgurantes de animación, viveza y gracia.
Pertenece a la nobleza tetuaní, y en su casa conserva las llaves de la
que en Granada ocuparon sus antecesores, hasta que Isabel y Fernando (¡a
quienes Allah dé su merecido!) les arrojaron con Boabdil a las playas
africanas. Es padre de generaciones: sus hijos y sus nietos y biznietos
masculinos no se pueden contar\ldots{} Es hombre instruido: ha estado
dos veces en la Meca; ha viajado por Oriente, y algo también por España
y por Italia; habla regularmente el español, y es, como sabéis, buen
creyente, de los que interpretan el \emph{Korán} a gusto de todos. Con
él he pasado las mejores horas de mi descanso, y no hay que decir que
nuestra conversación ha sido un continuo girar en torno al tema de la
guerra.

Debo deciros que \emph{Requena} no disimula su desconfianza de que el
Mogreb se \emph{sacuda fácilmente las moscas españolas}. Empleó esta
frase, que copio fielmente. Y la sinceridad del sutil viejo no se ha
recatado para manifestarme cierta simpatía por los españoles. En mucho
tiene sus cualidades de valor y de natural despejo para todo. Entre mil
cosas, me ha contado que años atrás, hallándose en Ceuta, hizo
conocimiento con el General Ros de Olano, Comandante entonces de aquella
plaza fuerte, y quedó prendado de su cortesía. Es, según dice, hombre
sabio en guerra y en paz; su instrucción abraza hasta el círculo de la
religión, de la poesía, y de la historia de los pueblos antiguos,
mayormente del que se llamó Roma, que luego vino a perderse como todos
los imperios de grandeza desmedida. Entretenía \emph{Mohammed Requena}
dulcemente las horas con el \emph{Chej} español, y desde aquellos días
no ha pasado uno sin que le recuerde. Siente en el alma que la guerra
del Mogreb con España le impida hoy bajar al llano para saludar a su
amigo con la Paz y la Misericordia de Allah\ldots{}

En nuestra última conversación me dijo \emph{Mohammed Requena} estas
palabras que jamás podré olvidar: «En toda guerra sale finalmente
vencedor el combatiente que sabe más, no sólo de guerra, sino de todas
las cosas de humano conocimiento, porque la guerra es un arte que pide
la reunión del saber militar y de todos los demás saberes y entenderes.
Los españoles, aunque algo alocados, saben o tienen de los diferentes
saberes luces incompletas; lucecitas que todas juntas hacen un gran
resplandor en las almas, por el cual se guían hacia donde está la
victoria\ldots{} Y no te digo más, hijo. Anda y ve\ldots{} y tráeme
pronto noticias del triunfo de nuestros hermanos\ldots{} que sobre todo
lo que te he dicho está la voluntad del Excelso.»

\hypertarget{v-2}{%
\section{V}\label{v-2}}

¡Loor al Grande, al Justo! Sean contigo la misericordia y las gracias.

Transcurridos cuatro días gratos en compañía del bendito \emph{Requena,
mina de excelencias}, salí en averiguación de lo que pasaba, pues desde
las inmediaciones de Samsa oíamos cañonazos y el granear de la
fusilería. Bajé a campo traviesa, y pasando junto al cementerio mosaico,
me encontré a mi criado \emph{Ibrahim}, que volvía del campamento, y me
contó las peleas de moros y cristianos en los días de mi ausencia. Sin
precisar fechas, pues era mi hombre bastante torpe en el conocimiento
del Almanaque, me informó de que los españoles habían rechazado a los
creyentes siempre que estos quisieron estorbar sus obras de
fortificación; que el día \emph{tantos} llegaron al campo nuestro las
tropas que manda el Príncipe \emph{Sidi Ahmet}, ocho mil hombres bien
armados: se les saludó con salvas y juego de pólvora. El día tal, que
debía de ser día \emph{cual} en el Calendario de ellos, visitó el
campamento cristiano el Gobernador de Gibraltar, que no iba más que a
curiosear. En todo metió las narices aquel señor, para informar a su
Gobierno del armamento del Español y de cómo llevaban la guerra. En
\emph{Torre Geleli} se comentó esta visita como favorable: creíamos que
el Inglés había de aconsejar a O'Donnell que se retirara, y no se dejase
coger en la trampa que preparada le tenemos. Pero el Español, despedido
el Inglés con zalemas, no tiene trazas de retirarse, y bien lo probó al
día siguiente y al otro, provocándonos a batallas en que Allah no quiso
favorecernos. De nada nos valió echar los \emph{facíes} por la parte
próxima al río, porque la Infantería del Prim no los dejó maniobrar, y
entre tanto los batallones ligeros y la Caballería española se nos
colaron por la parte alta, al pie de \emph{El Dersa}. Por fin, otro día,
que Ibrahim designó más claramente diciendo \emph{el bárah} (ayer), los
españoles celebraban fiesta de una santa que llaman \emph{La Virgen}, y
no combatieron, sino que se dedicaron al rezo, poniéndose todos a mirar
para la azotea de la Aduana, donde estaba el santón vestido de blanco y
oro, delante de un altar\ldots{} Y atentos a los gestos del \emph{imam},
se arrodillaban o se ponían en pie, y luego tocaron todas las músicas en
celebración del sacrificio. Oyó contar \emph{Ibrahim} que en cuanto
concluían los cristianos la ceremonia que llaman \emph{Misa}, degollaban
en aquel altar cien carneros y veinticinco bueyes, que es la ofrenda con
que obsequian a su Dios, el cual es un ídolo que gusta de ver correr la
sangre en su ara.

Nada contesté a los errores y disparates de \emph{Ibrahim} acerca de la
religión hispana, por parecerme que constituyen un estado moral
favorable a nuestra causa, y ordenándole que se fuese a Tetuán para
estar al cuidado de mi casa, seguí hasta \emph{Torre Geleli}, ansioso de
ver al Príncipe y de comunicarnos recíprocamente nuestras ideas y
observaciones. Encontrele revistando los trabajos de fortificación de su
campamento, en el cual unos dos mil hombres trabajaban abriendo fosos,
acumulando tierras, hacinando obstáculos en las escarpas, con piedras,
matojos, enredijo de pencas de pita, raíces y cuanto hallaban a mano.
Trabajaban con fe, riéndose algunos anticipadamente de la cara
chasqueada que pondrían los españoles cuando se vieran enredados de pie
y pierna en tales laberintos\ldots{} A Muley El Abbás le observé sereno
y grave: oyó mis noticias del estado de la opinión en Tettauen, sin
mostrar alarma ni abatimiento, asegurándome que había reforzado la
guarnición de la plaza con gente guerrera de la mejor que tenía. Díjome
luego que sabía por su espionaje la llegada de un refuerzo de tropas
cristianas, llamadas \emph{Voluntarios catalanes}, y quiso saber por mí
qué gente es esta, de dónde viene, y a qué \emph{kabila} o tribu de
españoles pertenece.

Acudí a ilustrar al Príncipe diciéndole que esta tropa viene de un
territorio hispano que se llama \emph{La Catalonia}, país de hombres
valientes, industriosos y comerciantes; país que está todo poblado de
talleres donde labran variedad de cosas útiles, papel, telas,
herramientas, vidrio y loza. Como expresara extrañeza de que los
\emph{catalonios} dejaran sus telares, alfarerías y fraguas para venir a
una guerra en que morirían como moscas, le respondí que allí sobra gente
para todo, y que los trabajadores pacíficos no temen interrumpir su
faena para ayudar a los fogosos militares, pues los pueblos de Europa
saben por experiencia que después de la guerra es más fecunda la paz, y
mayor el bienestar de las naciones\ldots{} Dije esto dejándome llevar de
una sandia pedantería, que aprendí no sé dónde ni cómo, y el Príncipe,
risueño y burlón, me cortó la palabra con los movimientos dubitativos de
su hermosa cabeza casi negra.

Siguiendo por el campamento atrincherado, vi los cañones en su sitio y
todo dispuesto para el combate. No pude ocultar mi satisfacción: las
robustas piezas me parecieron de terrible hermosura, y los artilleros
que habían de servirlas eran a mis ojos los primeros del mundo. Oyó el
Príncipe mis ponderativos aspavientos, y con modestia melancólica me
dijo: «Ellos traen cañones gruesos de sitio, y otros ligeros que llevan
fácilmente de un lado para otro. Pero sobre el bronce está la voluntad
de Allah\ldots{} A los débiles hace fuertes, y a los fuertes débiles. Ya
habrán visto los españoles que los moros van aprendiendo de sus
enemigos, con rápida instrucción, el arte de pelear en campo abierto.
¡Ah!, ¡qué sería de los cristianos si no tuvieran de General a ese
O'Donnell, hombre sereno que en los puntos y momentos de la confusión da
sus órdenes con la calma del que sabe el cómo y el por qué de mover una
pieza! Todo lo tiene previsto; nada se le escapa\ldots{} Las faltas que
cometen los muy arrebatados avanzando más de lo preciso, las enmienda
con los pasos medidos de los más prudentes\ldots{} Así es que siempre le
sale la idea suya\ldots{} Te digo con toda el alma que para el Mogreb
quisiera yo un hombre así, tan sabio y tan entendido en el mover de
tropas\ldots{} Pero ahora y siempre, sobre todo la voluntad de Allah.»
Terminó manifestando que las pérdidas en el día 7 de \emph{Rayab} (31 de
Enero), habían sido muchas por una y otra parte. En efecto: yo había
visto sin fin de heridos arrastrándose o llevados a hombros por las
veredas de Samsa, y en todo el campo gran número de muertos que aún no
habían sido enterrados\ldots{} ¡Lleva sus almas, oh Perfecto, a los
jardines de perdurables delicias!

El gozo me inundó contemplando la actividad de la muchedumbre guerrera
en el campo. En los ojos de aquellos hombres, resplandecía el fuego de
la fe\ldots{} Confiaban en Allah y en sí mismos. Recorrí de grupo en
grupo todo el terreno ocupado por los defensores del Mogreb; vi miles de
miles de musulmanes de distintas castas y familias, y en ningún rostro
noté señales de desaliento. Hablaban con animación, reían, y entre las
faenas obligatorias y los pasatiempos gimnásticos, ello es que tenían en
continuo ejercicio sus músculos de acero. Cuando la batalla no les
enardecía, jugaban a vencer o morir.

Allí estaba el Mogreb: todo lo vivo y sano de esta tierra de bendición
que Allah tiene por suya. Contar los hombres que pisaban el suelo desde
las alturas medias de \emph{El Darsa} a la vaga corriente de \emph{Guad
El Gelú}, habría sido tan difícil como sacar cuenta exacta de las
estrellas del Cielo. En el enjambre bullicioso distinguí las rudas
facciones del \emph{bereber}, de ojos encendidos y ágiles movimientos;
vi los negros del \emph{Sus}, de expresión triste y dulce mirar; los
\emph{muladís}, o mestizos de sudanés y \emph{bereber}, veloces en la
carrera y astutos en la intención; vi el árabe de Oriente, cuyo rostro,
de belleza descarnada, trae a la memoria la imagen del Profeta, y el
árabe español o granadino, de fina tez, fácilmente reconocido por su
compostura aristocrática. ¡Y qué variedad de trajes y atavíos! ¡Cuánto
más pintoresca nuestra tropa que la de España, en que los soldados van
igualmente vestidos, como frailes o alumnos de una escuela eclesiástica!
No son personas, sino muñecos fabricados conforme a un vulgar patrón de
la industria de sastres. Aquí veo la rica variedad de colores que me
dice los gustos de cada tribu y de cada país. Los montañeses del Riff
traen sus pardas chilabas terrosas, para que el color les ayude a
confundirse con los tonos del suelo; los más pudientes las adornan con
caireles y flecos de risueños colores. Ved allí los \emph{talebes}, de
blanca vestidura, y los \emph{bereberes} de Semmur, gustosos de que los
vivos matices de sus trajes ofrezcan blanco seguro al enemigo. De esta
otra parte aparecen los ricos árabes \emph{tettuaníes} y \emph{facíes},
con el blanco albornoz que ennoblece la figura; los negros
\emph{bukaras} ostentan el rojo de sus gorros puntiagudos; los del
\emph{Sus} visten \emph{caftanes} listados de blanco y rojo, y los
\emph{beni-argas} y \emph{tsuliés} combinan el negro y blanco\ldots{}
¡Qué armonía en esta variedad, y qué hermoso espectáculo el de tanta
gente que trae a la guerra la unidad de su fe, manteniéndose cada cual
en la forma y colorines que la tradición de su tribu le impone!

Cayó la noche sobre esta muchedumbre de creyentes guerreros. La oración
suspiró en muchas bocas, y en la mente de todos hubo un pensamiento que
salió y subió en busca del Dios Misericordioso. El bullicio se fue
apagando, y la movilidad resolviéndose en quietud apacible. Unos en las
tiendas, otros al raso, requerían el descanso. Yo me uní a un grupo de
amigos que, arrimados a las formidables trincheras de la \emph{Casa de
Assach}, se prepararon a pasar la noche. En aquel grupo había soldados
de indomable ferocidad y creyentes de gran virtud: uno de estos,
\emph{Bu Haman}, camellero que largo tiempo estuvo a mi servicio, me
guardaba fidelidad y adhesión cariñosa. La noche pasamos hablando más
que durmiendo, exponiendo cada cual sus pensamientos con libre
franqueza. Entre las mil peregrinas cosas que oí, recuerdo una
observación interesante del camellero: dijo que la noche anterior, de
centinela junto al río, frente al llano de \emph{Benimadan}, había visto
que todos los perros de Tettauen pasaban por una y otra orilla en
dirección del campo de los españoles. Sólo dos o tres se detuvieron en
el campo moro. Hizo constar uno que los canes olfatean el buen comer y
nunca se equivocan. Otro puso en duda la decantada fidelidad de aquellos
animales, y yo, sin decir nada, pensé que el desfile de perros hacia el
campamento cristiano era un hecho de malísimo augurio\ldots{} Mi mente
se llena de dudas. Para desvanecerlas, mi memoria revuelve el
\emph{Korán}\ldots{} que habla de todo lo divino y lo humano\ldots, pero
no dice nada del talento de los perros.

La noche fue desapacible, por el vientecillo helado que venía del Norte.
A la madrugada cayó alguna nieve, obligándonos a buscar el abrigo de una
tienda. Al amanecer, el viento cambió a Levante, y la nieve en llovizna
fastidiosa. Se presentaba un día de temporal, desfavorable para la
guerra. Por fortuna o por desgracia, a poco de amanecer, corrió el
viento a la otra banda, y el Poniente trajo sequedad y despejo del
cielo\ldots{} El \emph{¿qué pasará hoy?} a todos nos tenía en gran
inquietud, y el temor y la esperanza, unidos del brazo, eran huéspedes
de todos los corazones marroquíes. Apenas fue de día, nuestro campo
recobró la actividad de la víspera: los que tenían algo que comer, se
prevenían contra el ayuno forzoso de las horas de pelea. Otros, comidos
o sin comer, tanteaban sus armas y se surtían de balas y pólvora\ldots{}
Recorrí todo el espacio entre la \emph{Casa de Assach} y \emph{Torre
Geleli}, rodeando trincheras, sorteando obstáculos y metiéndome por
entre las manadas de hombres afanados, inquietos. Vi a Muley El Abbás
hablando sucesivamente con este y el otro \emph{Chej}, con el \emph{Kaid
et tabyia}, jefe de los artilleros, con los diferentes \emph{kaides} y
\emph{bajaes} de la caballería regular \emph{(Jaiali)}, de los
\emph{Bukaris} (Guardia negra), y de las irregulares masas de tropa
\emph{(harca)} que componían aquella inmensa grey. El Príncipe
\emph{Ahmet} salió a caballo con lucida escolta de jinetes árabes, y fue
a inspeccionar la gente que acampaba al pie de la montaña\ldots{} Luego
volvió a \emph{Casa de Assach}. El sol se desembarazó de nubes; sus
rayos hacían brillar las armas, y con suave picor, hiriendo la piel de
los hombres, los llevaba de la ansiedad a la confianza.

Un \emph{Kaid} de los \emph{facíes} me ofreció caballo y armas; pero no
acepté, pues no me sentía con las necesarias aptitudes de agilidad y
resistencia para seguir a la Caballería en sus atrevidas carreras. No
pudiendo permanecer ocioso, mi puesto no debía ser otro que las
trincheras de \emph{Torre Geleli} o la \emph{Casa de Assach}. Acompañé
al Kaid hasta las alturas que hay pasado el arroyo de \emph{Virgech}:
desde allí vimos que los españoles habían levantado su campamento, y
marchaban ordenadamente hacia nuestras posiciones, en dos grandes masas
que debían de ser los Cuerpos \emph{Segundo} y \emph{Tercero}. La
verdad, era un espectáculo imponente ver marchar tan gran número de
hombres formando líneas, que de lejos parecían trazadas sobre el papel.
Avanzaban con paso tranquilo en dos enormes conjuntos de diez mil
hombres cada uno. Detrás, junto al fuerte de la \emph{Estrella}, quedaba
otro golpe de gente, que debía de ser la \emph{Reserva}. Todo lo que vi
suspendió mi ánimo: era como la perplejidad calmosa con que la
Naturaleza anuncia las tempestades. ¿Hasta dónde llegarían aquellos
hombres, que yo veía como nube parda arrastrándose por la tierra, y que
llevaba dentro de sí el rayo y la destrucción?\ldots{} Pasaron los
españoles el Alcántara, sin duda por puentes que les habían construido
sus ingenieros, y seguían adelante con grave marcha de gigantes,
esquivando los terrenos pantanosos, pero sin perder su orden ni sus
alineaciones admirables.

Desde las lomas donde dejé a los \emph{facíes}, bajé rápidamente, y
pasando el arroyo \emph{Virgech} me volví a las trincheras que en
extensa línea, con entrantes y salientes, conforme a las ondulaciones
del terreno, serpenteaban de Norte a Sur, cortando el camino de
Tettauen\ldots{} Seguían los españoles su marcha pavorosa, y los dos
Cuerpos de Ejército se separaban más conforme iban ganando terreno.
Entre ellos distinguí otro bloque rastrero y movible, más bien azul que
pardo, que me pareció la Artillería montada. Detrás, a larga distancia
de los dos Cuerpos, venía la Caballería en abierta y descomunal falange,
dos inmensas filas que parecían trazadas con regla\ldots{} En nuestro
campo, a medida que a las trincheras me aproximaba, advertí, más que
silencio, un susurro, bajo el cual vibraba un escalofrío. Pude creer que
el oído aplicaban todos queriendo escuchar el estremecimiento del suelo
por las pisadas de los españoles con mesurada cadencia. Duró este
susurro, a mi parecer, cerca de una hora. Los cañones de una y otra
parte callaban lúgubremente\ldots{} El primer tiro lo disparó, según oí,
una cañonera que subía por el Río Martín para impedir que las partidas
de moros derramadas por la orilla izquierda hostilizaran a los
españoles\ldots{} El avance de estos era constante, como el tormento de
una idea fija\ldots{} Al segundo disparo de la cañonera, nuestras
baterías rompieron el fuego contra los dos Cuerpos españoles que venían
de frente. La Artillería de ellos seguía callada; la nuestra, demasiado
impaciente quizás, empezó a mandar balas; pero iban tan mal dirigidas
que casi todas caían en los claros de los batallones, los cuales
continuaban su marcha lenta, de aterradora pesadilla, sin hacer caso de
nuestra temprana furia.

Mas llegó un momento en que los españoles se detuvieron. Hallábanse en
el punto preciso que su sabio General les había marcado. Amenazaban el
extremo derecho de nuestra línea de trincheras. Ya les veíamos a
distancia como de un cuarto de legua, o menos. De su Artillería
avanzaron diez y seis cañones, que rompieron el fuego sobre nuestros
parapetos. ¡Allah Grande y Justo, asiste a los tuyos! El horrible
estruendo de tantos cañones de una y otra parte no puede ser expresado
por ninguna voz humana\ldots{} Tan formidable sonido no parecía cosa de
la tierra, sino del Cielo. En medio del fragoroso sacudimiento del suelo
y vibración de los aires, vino a mi mente lo que está escrito en el
\emph{Libro Santo}: «El Trueno canta las alabanzas del Excelso. Los
Ángeles, poseídos de terror, le glorifican. Allah lanza el rayo; ruedan
las Nubes; las Tempestades repiten que Allah es inmenso en su furor.»

\hypertarget{vi-2}{%
\section{VI}\label{vi-2}}

Y en esto, como si de la sierra se desgajase uno de los montes más altos
rodando en pedazos mil hacia el llano, vimos que se arrancaba nuestra
Caballería en número de cinco mil jinetes, con infinidad de colorines y
relumbrón de arreos y armas, corriendo a envolver a los españoles por su
flanco derecho. ¿Cómo podrían contener los de O'Donnell este formidable
pedrisco? Me han dicho que el suelo retemblaba, y que por el aire
surcaban como llamaradas las exclamaciones de los jinetes, enardecidos
por la fe y envalentonados por la seguridad del triunfo. Este hubiera
sido grande y decisivo, si Satán, que entre las filas españolas andaba
con todos sus diablos para dañar al Islam, no sugiriese a nuestros
enemigos un infernal ingenio de guerra, el más indigno y bárbaro que
puede imaginarse. El General de la Reserva, que me parece se llama Ríos,
destacose del fuerte de la \emph{Estrella}, que era el puesto que
O'Donnell le había marcado, y disparó sobre nuestros cinco mil caballos,
no balas o granadas, sino unos traidores cohetes que, corriendo y
reventando por bajo, al modo de buscapiés, espantaban a los nobles
animales y hacían imposible todo concierto en el ataque. ¡Maldito sea de
Allah, y precipitado en la \emph{Géhenna} (los Infiernos), el que
inventó tales aparatos de confusión y burla canallesca! Contra esto nada
vale el arrojo de los guerreros más audaces, nada las órdenes, planes y
reglas de batalla. Desesperados, los jefes de la Caballería gritaban que
no se tuviese miedo de los estampidos de los cohetes; pero los pobres
caballos, como irracionales y privados de entender la palabra humana, no
podían repararse de su terror, sintiendo que por entre sus patas se
enredaban todos los demonios con carcajada de pólvora restallante y
corrimiento de ruidos espantosos. No obstante, trabajo le costó al
\emph{Cheje} Ríos, con sus cohetes y sus batallones, atajar el empuje de
nuestra Caballería, aunque esta se enroscaba en sí propia, y se dio el
caso de que algún jinete, medio loco, hiriese a sus propios hermanos.

Satán o \emph{Eblis} y todos los genios malos, creados del fuego, se
concordaron para ayudar a los españoles. A los diez cañones que
vomitaban balas contra nosotros, otros tantos se unieron pronto lanzando
granadas encendidas. Felizmente, nuestros parapetos no estaban mal
armados, y el daño que nos hacían no era grande. Yo vi que a cada
disparo saltaban al cielo surtidores de tierra; a veces, entre ella, un
pedazo de árbol, una cabeza, una pierna de hombre\ldots{} ¡Espectáculo
terrible! Otros cañones cristianos fueron en ayuda del General Ríos, que
se desenredaba de los caballos moros como su Dios o Satán le dio a
entender. ¡Allah le ataje pronto sus días!

Y las dos masas de Infantería cristiana se aproximaban más a cada
momento, esperando que se les diera orden de atacarnos. La una ya estaba
como a seiscientas varas de nosotros; la otra como a cuatrocientas. Por
el lado del río también había fuego vivísimo. Un \emph{cheje} español se
batía con los moros de a pie y de a caballo que desde la margen del
\emph{Guad El Gelú} nos ayudaban, y contra estos también echaron cánones
los cristianos; que en este día de ira y de fuego todo era labor de
artilleros, y se creería que de la tierra brotaban las condenadas piezas
de montaña. ¡Sea quemado y vuelto a quemar infinidad de veces en el
Infierno el que inventó estos execrables tubos de bronce, que traerán,
si Allah no lo remedia, el acabamiento de los hijos de Adán!

Por lo visto, los españoles querían inutilizar nuestras baterías antes
de atacarnos cuerpo a cuerpo. Mas no era fácil, no era nada fácil, ¡ira
de Allah!, porque los parapetos de tierra, dirigidos en su ejecución por
sargentos ingleses, presentaban admirable defensa para los cañones y los
sirvientes de estos. El fuego continuo de los enemigos nos mataba mucha
gente; pero no lograba inutilizar nuestras piezas\ldots{} Estas callaban
algún rato, por falta de sirvientes; pero luego volvían a soltar su
tremenda voz en los aires inflamados. Señal indudable de intervención
del pérfido \emph{Eblis} en contra nuestra fue que una granada
cristiana, en vez de caer en la contra-escarpa, se metió muy adentro,
guiada del infernal espíritu, y vino a reventar en el propio depósito de
nuestra pólvora. Quemose esta de una vez, escupiendo al cielo un
pavoroso y horrísono volcán. ¿Qué mayor prueba de que los genios del mal
tenían hecho trato con O'Donnell y servían a España como traicioneros y
burlones diablos?

El maldito, el infiel O'Donnell no se apartaba un punto del pérfido plan
que había compuesto para perder al Mogreb. Su titánica Infantería, poca
cosa como quien dice, la friolera de \emph{treinta y dos batallones},
continuaba impávida detrás de las baterías, aguardando a que estas
hicieran el mayor estrago posible. La tenía el \emph{Gran Español} como
trincada y sujeta con inmensa rienda, y aunque ella quería embestir, no
la dejaba el muy perro. Los cañones, que a cada instante crecían en
número, como si salieran de la tierra, continuaban abrasándonos en toda
la línea\ldots{} Las trincheras de \emph{Casa de Assach}, donde estaba
el príncipe \emph{Ahmet}, eran las que más quebrantadas parecían por el
cañoneo incesante\ldots{} Llegó, por fin, el momento que el sagaz
O'Donnell esperaba, el momento de la madurez, o sea cuando nos
halláramos en punto de cochura, como quien dice, para ser comidos
calentitos. Las vibrantes cornetas de ellos, y las músicas para que nada
faltara, dieron a una la señal de ataque\ldots{} Ello fue cuando la
Infantería se hallaba a la distancia precisa para poder llegar de un
aliento a nuestras posiciones\ldots{} Quien pudiera ver desde los aires
la veloz carrera de los treinta y dos batallones desplegados como por
encanto en una línea de extensión poco menor de media legua, vería un
espectáculo tan horrible como grandioso. ¡Inmenso choque de la vida y la
muerte! Por la parte que yo vi, puedo imaginar el conjunto de esta feroz
acometida de hombres contra hombres. Y para que no dijesen los soldados
que sus jefes les mandaban a morir, quedándose ellos en el seguro,
delante de las masas de infantería venían los Generales gritando:
\emph{«Avante, hijos\ldots{} Carguen\ldots{} A ellos\ldots»}

En el lugar donde yo estaba, junto a \emph{Casa de Assach}, me tocó ver
a O'Donnell, a quien nunca había visto\ldots{} Le vi trayéndose detrás
una ola de furiosos hijos de Adán discípulos de Cristo, hombres mil
vestidos del pardo poncho, con los casquetes o roses echados atrás, y la
fiera bayoneta relumbrante al sol, apuntando a los pechos y a las
barrigas de los pobres hijos de Adán que éramos discípulos de
Mahoma\ldots{} Y pude observar en aquella visión de relámpago, que era
el llamado \emph{Gran Español} un diablo largo y rubio, de tez
enardecida por el fuego de su sangre hirviente\ldots{} Y visto un
instante, ya no le vi más, porque tuve que poner mis ojos en el pedazo
de tierra por donde yo debía escabullirme para librar mi cuerpo del
horrible filo de las bayonetas\ldots{} Recuerdo bien que hice fuego
sobre los enemigos que se colaban en nuestro campo, salvando las
trincheras; y no disparé una sola vez, sino dos o tres; y no mentiría si
asegurase que maté, o herí por lo menos gravemente, a uno, quizás a
dos\ldots{} Pero considerándome yo también hijo de Adán, y acordándome
de \emph{Puerta de Dios} (Bab-el-lah) y de mis adorados hijos, creí que
era un deber conservar la existencia, o que mi muerte no habría de traer
ya ninguna ventaja al apabullado Islam.

Y así como yo vi al máximo diablo O'Donnell echarse con su caballo sobre
nuestras trincheras, trayéndose detrás el huracán de sus tropas, otros
me han contado que vieron al \emph{Eblis} Prim en tal punto de la línea,
y al \emph{Eblis} Ros de Olano en tal otro\ldots{} Diablos eran todos, y
cada soldado echaba fuego por los ojos, fuego por la bruñida bayoneta, y
fuego escupían de su boca en bárbaras y blasfemantes expresiones\ldots{}
En medio de la confusión de nuestro campo, viéndome obligado a no estar
ocioso y a no escapar cobardemente, imité a los \emph{chejes} que vi
cerca de mí, y como ellos, dediqueme a dar palos sobre los infelices que
retrocedían\ldots{} ¡Atroz revoltijo de pelea, y espantosa algarabía de
voces y tiros, de cañonazos próximos y lejanos! Llegué a perder toda
orientación y a no saber dónde me encontraba. Yo no sabía hacia qué
parte caía Tettauen, pues creí verla por el lado del Río Martín, hacia
la mar salada; me figuré que las olas ocuparían el sitio del enhiesto
\emph{Djibel Musa}, y que este se había ido de paseo por la banda de
Oriente\ldots{} En fin, ni Norte ni Sur había ya para mí, y tierra y
cielo cambiaban de sitio.

Las feroces luchas cuerpo a cuerpo eran aquí y allá favorables a los
españoles. Muchos de estos avanzaban como locos campo adentro\ldots{} Vi
muertos a los que un momento antes había visto vivos, gritando y
matando. Caídos vi moros o cristianos, que volvían a levantarse, teñidos
de sangre, para caer de nuevo\ldots{} No sé por qué parte\ldots{} debía
de ser por la parte de \emph{El Dersa}\ldots{} moros a caballo y a pie
se alejaban de la refriega\ldots{} Mirándoles, sentí vehementes ansias
de tomar aquella dirección; pero no me determinaba. Seguía yo sacudiendo
a los flojos, y recordándoles con ardiente palabra las dulcísimas
venturas que encontrarían en los jardines paradisiacos si se dejaban
morir por el Mogreb\ldots{} Pero, la verdad, no se convencían
fácilmente, y, sin quererlo yo, me transmitieron su desánimo. Confieso,
señor, sin avergonzarme que la seguridad de la inmortal dicha cautivaba
mi espíritu menos que las imágenes de la felicidad temporal y
transitoria, accesible en este mundo. Todas mis ansias eran para mis
hijos y para \emph{Puerta de Dios} (Bab-el-lah).

En esto, como desmayase yo en apalear a los que volvían al enemigo la
espalda, en la mía descargó furiosamente su garrote un \emph{kaid}
desconocido y bárbaro. No fue preciso más que para que siguiese yo el
ejemplo de muchos moros principales, o no principales, que quisieron
acortar la distancia entre el campo de muerte y la montaña de salvación.
A huir me impulsaba, más que el horror de la matanza, el furibundo miedo
que tomé a los rostros de los españoles. Ni los cadáveres que pisábamos,
ni el espectáculo de los hombres que yacían expirantes, con la cabeza
hendida, el vientre rasgado, algún miembro separado del tronco, entre
charcos de sangre, me causaban horror tan intenso como los rostros de
los españoles vivos que iban entrando en nuestro campo y posesionándose
de él. Y si alguno me miraba, mi pánico me hacía buscar un agujero donde
esconderme, o ancha tierra por donde correr\ldots{} No puedo darte,
señor, explicación de esto, pues yo mismo no lo entendía ni lo entiendo.
Ello debió de ser obra de los genios malvados que, invisibles entre
nosotros, nos llevaron a la catástrofe, aflojando nuestra valentía; y no
satisfechos aún, querían volvernos locos para que los cristianos nos
destruyeran en la confusión de nuestra retirada.

Ya iba yo más allá de \emph{Torre Geleli}, faldeando con paso vivo la
montaña, cuando otros infelices que a mi lado pasaron a todo el correr
de sus ágiles piernas, profirieron blasfemias horribles, natural
desahogo de la vergüenza y humillación que todos sufríamos. Lo peor,
Señor, fue que yo también blasfemé: mi lengua, como máquina obediente a
las soeces exclamaciones que me entraban por los oídos, pronunció
también voces y frases altamente ofensivas para el Poderoso Allah, Dios
Grande y Único\ldots{} Entiendo, Señor, que en aquel trance de tanta
turbación y amargura, mi lengua emancipada y sola, sin estímulo del
pensamiento, echó de sí las atrocidades que confieso ahora para que veas
mi pecado y me ayudes a obtener el perdón. Oyendo las perrerías que los
otros decían de Allah por haber consentido a los ángeles maléficos la
derrota del Islam, yo le llamé \emph{cochino}, nombre que dan los
cristianos al inmundo animal cuya carne nos está vedada por enfermiza y
corruptora de nuestra sangre\ldots{} Y para acabar de arreglarlo, voces
españolas de mal gusto se me escaparon de la boca, como
\emph{calzonazos} aplicado al Sumo Creador, y \emph{cabrón} o
\emph{macho cabrío}, con que desvergonzadamente motejé al
Profeta\ldots{} Pero estábamos ebrios de despecho y vergüenza, y no
sabíamos lo que decíamos; casi no éramos responsables de tan nefando
sacrilegio, y Allah, que nos oía, porque todo lo oye y lo ve, debió de
menear la majestuosa cabeza, y esclarecer todo el Universo con una
indulgente sonrisa\ldots{} ¿Verdad, Señor, que si Allah nos condujo al
desastre fue porque así nos conviene? ¿Verdad que ha querido castigarnos
por nuestra poca fe y el descuido de las prácticas religiosas? Así lo
pensé yo por la noche, y me privé del descanso y sueño para implorar el
perdón de mi culpa, y reconocer humildemente la Sabiduría del Creador y
Ordenador de todas las cosas.

Y dicho esto en descargo mío, sigo contando. Íbamos en gran desorden,
temerosos de que el cañón cristiano nos diera la despedida. Faldeando el
áspero monte frente a la Alcazaba, saludábamos tristemente a la blanca
\emph{paloma} que pronto había de ser esclava del soberbio
\emph{Sbañul}. No vi al Príncipe Ahmet, que era de los que habían tomado
la delantera para llegar pronto al descanso; al otro Príncipe, a mi
amigo Muley El Abbás, sí pude verle, y aun cambiar con él afligidas
palabras. El noble señor se cubría el atezado rostro con un pañuelo,
para que no viéramos las lágrimas que de sus ojos echaba. Hombre de
tesón militar y de ardiente patriotismo, no hallaba consuelo a su dolor
y vergüenza, como no fuera en la santa religión. «Dios lo ha
querido---me decía.---Nada podemos contra Dios\ldots{} El Mogreb es
vencido por la tibieza de nuestra fe\ldots{} No acuden como debieran los
voluntarios musulmanes a la guerra santa\ldots{} Mahoma está perplejo,
Allah muy enojado\ldots»

Andando sin parar, oí de labios de mis compañeros de fuga las opiniones
más estupendas. \emph{Bu Haman}, el que fue mi camellero, nos explicó el
desastre con un criterio teológico muy peregrino. Aficionado el hombre a
leer las Escrituras, blasonaba de muy sagaz en la interpretación de las
causas divinas que producen los efectos humanos. No nos había derrotado
Allah deliberadamente para castigarnos por nuestra falta de fe: la fe
crece como planta lozana en el Mogreb. Nos habían derrotado los genios
rebeldes burlando al Poderoso. El Dios Único, al crear a estos malditos
seres incorpóreos formándolos del fuego, les dio la facultad de
introducirse sin ser vistos en el Paraíso, y de poder escuchar lo que el
Dios Único habla con los bienaventurados. Así se enteran de los secretos
divinos, y luego bajan a la tierra y arman sus enredos. «Si Allah no
hubiera dado a los genios malos la facultad de oír lo que se dice en el
Cielo, no pasarían estas cosas\ldots{} Los tales escucharon lo que Dios
decía del plan de guerra de los españoles y de lo que pensado tenía para
desbaratarlo\ldots{} ¿Qué hicieron entonces? Pues descolgarse a la
tierra y sugerir a O'Donnell que cambiara de plan\ldots» Sin duda el
buen \emph{Bu Haman} se había vuelto loco de la irritación y furia del
combate, porque sólo a un demente se le puede ocurrir el sacrílego
disparate con que terminó su explicación. «Creedme: lo que debe hacer
Allah Grande y Único, en casos de una batalla que compromete la suerte
de su pueblo, es callarse\ldots{} callarse, digo, y no revelar su
pensamiento a los \emph{rostros blancos} (bienaventurados) que van a
preguntarle: \emph{¿qué hay, Señor?, ¿qué has resuelto?\ldots»} Si sabe
Allah que los genios rebeldes tienen facultad de esconderse y oír, ¿para
qué habla?\ldots{} Adorémosle con un nuevo nombre: \emph{El Silencioso.}

\hypertarget{vii-2}{%
\section{VII}\label{vii-2}}

Al caer de la tarde, entre cinco y seis, cuando ya el sol trasponía,
dorando las cumbres de \emph{El Dersa}, nos tiramos al suelo en un
recuesto seis o siete hombres que caminábamos juntos. El herido que dos
de nosotros transportábamos por turno se nos quedó muerto, y
desembarazados de la carga (dejándole junto a un árbol, acompañado de
otros que los delanteros soltaban conforme morían) nos dimos un rato de
reposo. \emph{Boabit Musa}, comerciante de Rabat, amigo mío, sacó del
zurrón con su mano ensangrentada unas naranjas que repartió, y chupando
su ácida frescura departimos sobre lo pasado y lo futuro.
\emph{Bu-Haman} se lamentó de que en poder de los cristianos quedase el
sin fin de tiendas de nuestros cuatro campamentos, y las provisiones
ricas que en ellas teníamos. Era un dolor perder tanta riqueza y
hermosura. \emph{El Yemení}, negro del Sus, no podía echar de sí la
visión horrible del furioso ataque de los españoles. Lo que vio en
aquellos momentos de sublime espanto, quedó impreso en sus ojos, y del
espanto no se aliviaba sino refiriendo lo que aún veía. Y con tal viveza
lo narraba, que los demás creíamos haberlo visto. En la tronera o
boquete del parapeto estaba \emph{El Yemení} cuando Prim, con gallardo
atrevimiento, se metió a caballo en nuestro campo. La sorpresa misma de
tal audacia impidió matarle en el instante de su aparición. Luego se fue
a él, yatagán en mano; pero a punto entraron detrás de Prim seis, ocho,
diez de aquellos voluntarios que llaman \emph{catalonios}, hombres
fornidos, con un gorro morado y luengo a manera de bolsa, que les cae
para delante o para detrás según mueven la cabeza\ldots{} Ha contado
\emph{El Yemení} que él solo mató a cuatro de aquellos malditos,
hundiéndoles su cuchillo en el vientre o en el costado\ldots{} A uno de
estos lo mató en el mismo momento en que él mataba a un riffeño. Fueron
dos muertes entrelazadas, como las rayas de un arabesco\ldots{} Antes de
esto vio a los \emph{catalonios} de las primeras filas caer en un charco
de agua honda, y sobre los cuerpos caídos pasar los demás como por un
puente\ldots{} En esta disposición los fusilaban desde el parapeto,
cuando se metió Prim como un terrible diablo contra el cual nada podían.
Llevaba consigo un espíritu malo, pues le tiraban golpes y tiros, y no
podían herirle.

Y \emph{Boabit Musa} refirió que de los gigantes \emph{catalonios}
habían muerto la tercera parte, o más, pues caían como moscas. En una
trinchera de \emph{Casa de Assach} había visto a O'Donnell echando
llamas por los ojos y por la boca. Podía jurarlo\ldots{} Una compañía de
cazadores había entrado tras él. Mataron moros muchos; pero estos no se
dormían, porque allí quedó el capitán de la compañía, todos los
sargentos, y más de treinta soldados. \emph{Boabit} mató cuantos quiso,
y de ello estaban sus manos teñidas de sangre. Otro que venía con
\emph{Boabit}, y que yo no conocía, refirió que en \emph{Torre Geleli}
entró un General, que según dijeron es hermano de O'Donnell, llevando
consigo un batallón, del cual murió la mitad para que la otra mitad
pudiera llegar hasta la misma Torre. Al que esto contaba le diputé por
renegado, fijándome en las exclamaciones españolas que entre frase y
frase ponía. Interrogado acerca de su condición, nos reveló su origen
cristiano, y yo caí en la cuenta de que él fue quien, al iniciarse la
retirada, blasfemó al lado mío, haciéndome blasfemar a mí. Aquel maldito
español fue el causante de que mi boca se disparara en insultos
desvergonzados contra el \emph{Excelso}\ldots{} A pesar de esto,
quedamos amigos, y como \emph{El Gazel}, que así se llama, dijese que en
cuanto fuera de noche entraría en Tettauen, donde tenía que mirar por
algunos efectos de comercio guardados en su almacén, entre ellos tres
sacos de almendra, me animé yo a ir con él, pues me convenía dar un
vistazo a mi casa y a mis sagrados intereses.

En esto llegaron otros amigos, de los últimos en la fuga, y con ellos
venía \emph{Sid Afailal}, hijo de un famoso \emph{sheriff} y más
aficionado a la Poesía que a la Guerra. Venía como loco, dando gritos y
extendiendo los brazos, ya para increpar a los que entregaban al
cristiano la bella ciudad, ya para dirigir a esta, que entre sombras se
veía melancólica, dulces requiebros amorosos. Callamos oyéndole, pues
aquel hombre que clamaba con poéticas voces en medio de los caminos,
poseía seductora elocuencia; los heridos se reanimaban oyéndole, y hasta
se creería que los muertos ponían atención al vago discurso difundido en
la noche. Leed aquí, señor, lo que el mágico poeta cantaba con
entonación solemne que a todos nos hizo derramar llanto de ternura:
«Dime, Allah, ¿por qué has desbaratado el Ejército de la Fe?, ¿por qué
lo has expuesto a tantas calamidades?, ¿por qué has rebajado una tan
gran dignidad entregándola a un enemigo que no vale ni sus
desperdicios?» Así declamaba con mística exaltación, mirando al cielo,
elevadas con rigidez ceremoniosa las palmas de sus manos. Luego se
volvía hacia \emph{Ojos de Manantiales}, y con plañidera y delgada voz
le decía: «Tú, que has sido siempre pura como paloma blanca, o como el
turbante del \emph{Imam en el Mumbar} (el sacerdote en el púlpito); tú,
que eras un jardín espléndido y hermoso, cuyas flores sonreían de
felicidad como un lunar en la mejilla de una desposada; tú, cuya belleza
es superior a la de Fez, Egipto y Damasco, ¿qué es ahora de ti?» Oyendo
estos bellos canticios, lagrimones como puños brotaban de nuestros
afligidos ojos, y el pecho senos oprimía. Volvíase luego el poeta hacia
nosotros, y nos declaraba que \emph{Tettauen} era víctima del \emph{mal
de ojo}, y que padecía la misma suerte que la fabulosa heroína
\emph{Zarka El Jamama}. Los españoles no eran más que unos infames
hechiceros que habían hecho \emph{mal de ojo} al Islam\ldots{} La
emoción no nos permitió añadir comentario alguno a las sublimes
inspiraciones del tierno poeta, que luego se volvió otra vez hacia la
ciudad arrancándose con esto: «¡Oh país de la felicidad y del placer! Si
la estrella de tu buena suerte se ha eclipsado ante los resplandores de
otra estrella de fatalidad, pronto nacerá una luna que con su esplendor
borre las tinieblas presentes.» Esto dijo el exaltado poeta. Le besamos
la orla de la chilaba, y él siguió, hasta encontrar más moros fugitivos
a quienes obsequiar con las mismas cantinelas.

Cuando le vio lejos, \emph{Bu-Haman} me dijo: «Yo soy el único que no se
ha conmovido con los gritos de este farsante. Ya sabes que el
\emph{Korán} habla pestes de los poetas. Los demonios malos inspiran a
los hombres mentirosos, estos a los poetas que andan declamando por los
caminos, y a los musulmanes extraviados que les aplauden y los siguen.»

A esto replicó \emph{El Yemení} que los poetas deben ser oídos con
deleite y respeto, porque a ellos desciende el espíritu de Allah. El que
acabamos de oír, \emph{Sid Afailal}, es hijo de un veneradísimo
\emph{Sheriff el-baraca}, llamado así porque Allah le ha concedido la
facultad de hacer milagros. Puede hacer todos los milagros que quiera;
pero él es tan modesto que nunca los hace, o los hace en familia, para
que no sean milagros públicos\ldots{} Algo dijo el camellero
\emph{Bu-Haman} sobre la milagrería corriente en el Mogreb; pero no
pudimos enredarnos en discusiones sobre tan grave punto, porque los
compañeros querían seguir para reunirse a los Príncipes y acampar con
ellos. \emph{El Gazel} y yo les deseamos la paz en el paso del arroyo de
Samsa, y retrocedimos, entrando en Tettauen por la Puerta de Fez.

¡Allah soberano, Allah justiciero! Descienda tu infinita misericordia
sobre la muchedumbre de nuestras iniquidades, y lávanos de ellas\ldots{}
No tenemos palabras con que implorar tu clemencia al ver los infortunios
que ha derramado tu justicia sobre la inocente Tettauen. ¿Por qué,
Señor, desatas sobre tu hija predilecta las furias del Infierno?
¿Quiénes son estos enemigos que la hieren, la deshonran y la ultrajan?
No son, ¡ay!, los feroces secuaces del \emph{Hijo de María}, no los
infieles, no los idólatras, sino nuestros propios hermanos, o quizás
genios diabólicos disfrazados con figura y rostro del Islam.

No habíamos dado veinte pasos en el interior de la ciudad, cuando vimos
los efectos del plebeyo desorden que en ella reinaba, y mi compañero, el
renegado \emph{El Gazel}, cuyo verdadero nombre es Torres, sin poder
reprimir el grito de la raza que del alma le salía, exclamó en español:
«¡María Santísima\ldots{} tenemos aquí la canalla!\ldots{} Me cisco en
Allah y en la pendanga de su madre. ¿Pero no ves, no ves? Por aquí ha
pasado el demonio.»

Exhortele yo a ser más comedido y limpio en su lenguaje, y seguimos por
las calles tenebrosas, tropezando en objetos mil abandonados, en figuras
yacentes que exhalaban quejidos, en muertos que no decían nada, en
escombros y maderas a medio quemar. Ante tanta desolación, no tuve otro
pensamiento que dirigirme a mi casa, próxima al palacio Imperial.
\emph{El Gazel} corrió a la suya, cerca de la gran Mezquita. Nos
separamos\ldots{} Al pasar yo por la Alcaicería, halleme entre un
miserable gentío que con grande algazara se arremolinaba en torno a una
puerta, de la cual salía humo. Mujeres, viejos y chiquillos clamaban
desconsolados. Los bárbaros montañeses habían huido por \emph{Bab
Eucalar} después de pegar fuego a varias casas, llevándose lo que de
algún valor encontraron en ellas. Ansioso de llegar a la mía, tuve la
suerte de encontrar a \emph{Ibrahim}, que me anticipó la tranquilidad
que yo buscaba\ldots{} Ningún atropello había sufrido mi vivienda, según
me contaron mis sirvientes y la esclava, por lo cual me apresuré a dar
gracias a Dios pidiéndole además que en lo restante de la noche me
librara de toda maldad.

Díjome \emph{Ibrahim} que Muley El Abbás acamparía probablemente a
orillas del \emph{Busceha}, y que sus tropas no guardaban ninguna
disciplina. Multitud de montañeses se habían quedado en las afueras de
Tettauen, por Occidente, y cuando les parecía bien entraban en busca de
comida, muertos de hambre y locos de rabia. Al tiempo que esto escuché,
oí el cañón de la Alcazaba, que con jactancia estúpida seguía mandando
balas al campo español, horas antes campo moro, seguramente sin hacer
daño alguno, pues las balas habían de caer frías y desmayadas como las
maldiciones del vencido moribundo. Al ser conocida la derrota de los
musulmanes, había en la ciudad partidarios de la resistencia; pero
después de los escandalosos desmanes ocurridos al anochecer, ya no hubo
ningún tetuaní de mediano pelo y posición que no deseara la entrada de
los cristianos.

Informáronme también mis servidores de que multitud de menesterosos
moros y hebreos habían ido a mi casa durante el día, creyéndome allí, en
demanda de socorro. ¡Infelices! Conocían el fervor musulmán con que
practico la limosna, y acudían a mí. Sólo restos guardaba mi despensa;
pero de ellos participaron los que padecían hambre. Mis criados hicieron
lo que habría hecho yo si presente estuviera. Entre los pedigüeños
estuvo la hechicera \emph{Mazaltob}, que reiteró sus ansias de verme y
hablarme. Creyendo que la engañaban al decirle que estaba yo en el campo
de batalla, se metió por todos los aposentos y rincones en busca mía. Lo
que buscaba no encontró; pero sí un gran trozo de \emph{mharsha} (pan de
cebada) como de media libra, y unos pastelitos dulces y ya revenidos
\emph{(el macrod)}. Todo se lo apropió gozosa antes que se lo dieran, y
partió veloz, dejando en mis criados la mala impresión o sospecha de
que, al recorrer sola las estancias, patios y corredores, pudo dejar en
alguna parte de mi vivienda la huella maligna de su espíritu dado a los
demonios. Sobre este punto tranquilicé a mis buenos sirvientes,
asegurándoles que mi fe musulmana es escudo mío y de mi familia contra
las asechanzas de los hijos del fuego.

Largo rato estuve en mi casa, meditando en las calamidades horrendas que
Allah nos enviaba como llamas de purificación, y buena parte de aquel
rato dediqué a implorar la clemencia del Augusto Criador por el pecado
de ultrajar su nombre con dicterios inmundos, al lanzarme a la fuga
después de la batalla. Cumplidos este deber y el de mis abluciones, tomé
algún alimento para repararme de tanta debilidad, me vestí de limpio, y
salí acompañado de Ibrahim, el cual me indicó que en la morada de
\emph{Ahmed Abeir} se congregaban los principales de la ciudad para ver
qué determinaciones se tomarían ante el peligro de los desmandados
riffeños por una parte y de los cristianos por otra. Palpando la
obscuridad avanzamos por las angostas calles; a cada paso nos detenían
informes bultos yacentes, otros movibles. Uno de estos, que nos infundió
pavor supersticioso, resultó ser un pobre burro abandonado. El
hambriento animal fue largo trecho detrás de nosotros, como pidiéndonos
que le diéramos de comer. No me sorprendió la escasez de perros en las
calles: los suponía, según el dicho de \emph{Bu-Haman}, apegados a las
abundancias del campamento español. A lo mejor, de los montones de
escombros o de muebles hacinados salían lamentos débiles, la voz ahilada
de algún mendigo anciano, o de pobres ciegos que imploraban socorro.
Limosna de pan querían, no de dinero, y aquella no podía yo dársela,
porque el comercio estaba paralizado y en las tiendas no había provisión
de ningún comestible.

Para ir a la casa de \emph{Ahmed Abeir}, que vive cerca de
\emph{Bab-el-aokla}, habíamos de pasar por el \emph{Zoco}. Allí nos
salieron al encuentro moros haraposos y judíos de ambos sexos gritando
con voces desesperadas: «Paz, Señor. Abrir puerta españoles.» Esta
súplica vino a mis oídos en las dos lenguas, árabe y judiego-española, y
en las dos contesté yo: «Confiad en la autoridad, que resolverá lo que
convenga.» Mi respuesta les exasperó más, y allí fue el maldecir a Muley
El Abbás, al \emph{Bajá}, y a los hombres tercos que, guarecidos en la
Alcazaba, sostenían una sombra de poder irrisorio\ldots{} No era mi
ánimo detenerme a escuchar lamentaciones agoniosas, ni relatos de
desdichas que no podía evitar. Pero me vi rodeado de pobres viejos
moros, del comercio menudo, amigos y clientes míos, que lloraban por sus
miserables tiendas del \emph{Zoco}, saqueadas y destruidas aquella
tarde. Habían llegado al punto anímico en que el sentimiento patriótico
se contrae, se aniquila, desaparece, quedando en su lugar y dueño de
toda el alma el sentimiento de la subsistencia y de la propiedad. Los
que dos días antes llamaban \emph{perro} al Español, ahora claman por
él, pues aun siendo \emph{perro} había de traer comida, y otra cosa que
ellos no aciertan a definir, y es algo semejante a lo que los europeos
llaman \emph{Orden público}. «Que vengan---gritaban,---que vengan con
justicia, y al ladrón, palo mucho.»

Una mujer me tiró del jaique. «¿Eres tú \emph{Noche}? ¿Y tu hermana
\emph{Tamo}? ¿Y tu padre \emph{Ha-Levy?»} Con voz turbada, tartajosa,
que expresaba el hambre en cada sílaba, la infeliz \emph{Noche} me contó
que ellas y su padre habían intentado la fuga, \emph{denque supieron
perdida la batalla}; pero en \emph{Bab Eucalar} toparon una turbamulta
que las metió para adentro. No eran montañeses todos los que entraban
atropellando con griterío. También venían entre ellos mancebos tetuaníes
de los que andaban en la guerra\ldots{} Furiosos, insultaron a las dos
hermanas tirándoles de la \emph{justata} para desnudarles la
\emph{pechera}, y al padre le agarraron de las barbas canas sin respetar
su \emph{vejetud}\ldots{} La pobrecica \emph{Tamo}, al volver a casa, se
había caído en un montón de maderos, \emph{desgobernándose} un pie, y
estaba cojosa; a su padre, cuando pasaban por el \emph{Zoco}, un tropel
de \emph{moríos} jóvenes quiso tirarle a tierra, y uno de ellos le
\emph{aderezó} un palo en la cabeza, de lo que ha quedado el pobre
\emph{adolorado, sin judicio}\ldots{} En la casa no habían dejado los
robadores ni una hilacha. Todo, menos el oro que estaba soterrado, se lo
llevaron. \emph{Tamo} y \emph{Noche} con su padre se habían refugiado en
casa de \emph{Ahron Fresco}, \emph{aonde} juntadas familias muchas,
podían defenderse si otra vez tornaban los malos. Lo que a todos más
agobiaba era no tener nada de comida, pues a ningún precio se
encontraba.

«¿Pero nada tenéis que pueda serviros de alimento---le dije:---higos,
mojama, el gato?\ldots»

---Nada hay en nuestra casa ni en la de \emph{Fresco} más que las drogas
que vendemos: azufre, aloes, incienso, agalla, matalahúva y
zarzaparrilla\ldots{} Con algún enjuagatorio de esto, refrescación de
tripas, vamos engañando el hambre\ldots{} Ven y verás nuestra miseria.

Respondile que no podía en aquel momento ir a su casa, por tener que
personarme en la de \emph{Ahmed Abeir}, donde los Principales estaban
reunidos. Allí acordaríamos algo que aliviase la miseria y previniera
nuevos desmanes. Seguí mi camino, apartando a un lado y otro los grupos
de hambrientos y llorones. En casa de \emph{Abeir} hallé unos catorce
individuos, de posición los unos, otros dedicados al transporte
comercial, como el renegado \emph{El Gazel} (Torres). En pocas palabras
me informó el dueño de la casa de que se había llegado al acuerdo de
enviar al campo español, al día siguiente, una comisión de cinco vecinos
con el fin de ofrecer a O'Donnell la entrega de la ciudad, siempre que
el General español prometiese respetar vidas, haciendas y religiones.
Más de tres y más de cuatro dijeron que en la embajada debía ir yo, a lo
que me negué, alegando que he tenido cuestiones desagradables con
españoles del comercio de Ceuta y de Algeciras, y que sonaría mal en los
oídos cristianos el nombre de \emph{El Nasiry}. Razones di con
fundamento lógico y hasta con elocuencia, y por término de mi perorata
propuse que fuese Torres en la embajada. Así se acordó. ¡Loores mil al
Poderoso Allah!

Habíamos determinado lo que te escribo, ilustre Señor, sin contar para
nada con los locos que aún seguían presumiendo y fanfarroneando en la
Alcazaba. Mas era preciso que nos armáramos de valor, y nos atreviéramos
a decirles que se retiraran dejándonos dueños de la plaza. Con otros dos
fui comisionado para poner en conocimiento del \emph{Bajá} y su tropa la
destitución que acordó la Junta del Pueblo, cosa desusada en nuestras
historias, y una novedad más que aprendíamos de los españoles. ¡Sobre
todo los designios de Allah!

\emph{¡Con doscientos y el portero!}, no me acobardé ante las
dificultades de mi comisión, ni tampoco los que en ella habían de ser
mis compañeros. Pero sucedió lo más inesperado y peregrino, pues sin
duda Satán, que nos había hecho tan malas partidas en el curso de la
batalla, también en aquella tristísima noche de la ciudad, ni vencedora
ni conquistada, tramó los mayores enredos que pueden imaginarse. He aquí
que apenas salimos a la calle los tres comisionados para colgar el
cascabel en el pescuezo de los de la Alcazaba, oímos estruendo
terrorífico de voces y vimos por encima de las azoteas resplandor rojizo
de incendio\ldots{} Corrimos hacia el \emph{Zoco}, de donde al parecer
venían la bullanga y el resplandor, y al pasar por un pasadizo cubierto
de los que en la ciudad tanto abundan, distinguimos un bulto negro y
pavoroso que hacia nosotros venía en la actitud más amenazante. Íbamos
armados: requerí una pistola, di la voz de \emph{¡quién vive!}\ldots{}
Como no nos respondiera el terrible sombrajo negro, ya los tres en
concertado movimiento nos lanzábamos hacia él, cuando del bulto mismo
salió un formidable rebuzno que al primer sonido nos hizo estremecer de
susto, después de admiración\ldots{} Caso fue sobrenatural, según dijo
uno de los tres, que creía en el poder de los genios maléficos para
transformarse en pollinos. Era el infeliz asno que yo había encontrado
no lejos de mi casa, y que recorría la ciudad buscando algo que comer.
Más afortunado que los habitantes de la raza de Adán, aquel descendiente
de la burra que habló, según nos dice el Pentateuco, había encontrado
entre las basuras y escombros un montón de paja, en el cual metía con
delicia sus desocupados dientes. Rebuznaba de júbilo triunfal.

\hypertarget{viii-1}{%
\section{VIII}\label{viii-1}}

¡Bendito Allah, confunde a los injustos, que no creen en tus signos! ¡El
ángel Malek, encargado de tus castigos, les dé a beber el agua
hirviente!\ldots{} ¡Horrible espectáculo se presentó a nuestros ojos en
el \emph{Zoco} y puerta del \emph{Mellah!} La canalla que en las
angustias de la ciudad hallaba ocasión para sus tropelías entró a media
noche, cebándose en los pobres hebreos. Buscaba el dinero escondido, y
no hallándolo, apaleaba a los hijos de Israel, sin respetar mujeres ni
ancianos. Cuando yo llegué, algunos de aquellos desalmados habían huido
ya, llevándose ropas y cuanto encontraban de fácil transporte; otros
trataban de pegar fuego a las casas hacinando paja y la madera vieja y
las astillas de los tenduchos destrozados. En el barullo perdí de vista
a mis compañeros; pero la suerte me deparó a \emph{Ibrahim}: él y yo
acudimos con palos a dispersar a la chusma, que las armas no eran del
caso contra malhechores cobardes que huían a cualquier intimación de
hombres decididos\ldots{} Quiso Allah que de súbito se nos unieran tres
fornidos moros de buen porte que llegaban de la Alcazaba, y entre todos
pudimos dar su merecido a los que avivaban la hoguera y metían haces
encendidos dentro de las casuchas pobres\ldots{} De pronto, de lo más
recóndito del \emph{Mellah} nos llamaron voces de angustia\ldots{}
Corrimos allá. Una cuadrilla de montañeses audaces y bárbaros, indómita
plebe del Riff, sacaba de una de las casas más escondidas del barrio (a
la derecha conforme entramos) a una pobre mujer, que si no salía ya
muerta, poco le faltaba. A rastras la traían, vociferando. La pobre
víctima, magullada en rostro y brazos, y teñida de sangre, no podía ya
ni soltar el aliento para pedir socorro. Otras mujeres hebreas clamaban
tras ella, y ningún hombre de su raza sabía salir gallardamente a su
socorro\ldots{}

Te confieso, Señor, que me quedé espantado al reconocer en la tan
cruelmente arrastrada mujer a la hechicera \emph{Mazaltob}. El espíritu
de caridad surgió en mí con irresistible fuerza, y sin acordarme de que
la impostora me había ofendido, ni reparar en su raza usurera ni en su
religión condenada, me fui contra los verdugos, y a uno le di un tajo en
la cabeza, a otro tiré al suelo, y me harté de patearle mientras mis
compañeros arremetían contra los demás y les ponían en rápida
dispersión. Con mano generosa levanté del suelo a la embaidora
diciéndole: «No por tu maldad ha de negarte el buen musulmán auxilio
piadoso, que mi Profeta me ordena perdonar las ofensas y dar socorro al
enemigo acosado de ladrones.» Lleváronla adentro, y en las pestíferas
estancias la metieron mujeres compasivas, a las que recomendé que le
aplicaran a los cardenales y magulladuras paños con vinagre\ldots{} Y si
vinagre no tenían, que fueran a buscarlo a mi casa, donde en abundancia
lo hay. ¿Verdad, señor y amigo mío, que obré como buen musulmán y fiel
seguidor de las máximas divinas? No fue mi conducta inspirada de la
jactancia ni de la ostentación, que esto habría sido como echar simiente
en pelada roca, sino de la compasiva piedad, que es como sembrar en
terreno blando y fértil\ldots{} «Los que no tengan piedad del débil, se
nos ha dicho, aunque este débil sea idólatra o desconozca los signos de
Dios, no entrarán en los jardines refrescados por corrientes de agua y
embalsamados por un aire que lleva en sus átomos todas las delicias.»

Los tres moros venidos de la Alcazaba, \emph{Ibrahim} y yo, formábamos
ya un núcleo de fuerza y autoridad que podría dominar la situación, si
otros moros se nos agregaban. Les propuse que en unión de los dos
compañeros que habían salido conmigo de la casa de \emph{Abeir} nos
constituyéramos en fuerza pública para mantener el orden al uso europeo,
en nombre de nuestro Señor el Sultán. Antes de escribir aquí su
respuesta, debo decirte que dos eran negros del \emph{Sus}, el otro
\emph{kaid-et-Tabyia} (jefe de artilleros), y a mi parecer (perdóneme
Allah) entendía tanto de manejar cañones como yo de afeitar
ranas\ldots{} Pues a mi propuesta de subir a la Alcazaba respondieron
que ya el \emph{Bajá} y los demás hombres que en la fortaleza servían se
habían retirado, saliendo por Puerta de Fez, o permaneciendo en la
ciudad en espera de los acontecimientos.

«Según eso---dije yo,---podremos subir a la Alcazaba y tomar posesión de
ella.»

---No es cosa fácil---respondió uno de los negrazos del Sus, tan grande
como algunas casas del \emph{Mellah},---porque en cuanto desocupamos
nosotros la Alcazaba, cual bandada de ratones se metieron en ella los
montañeses libres, de estos que no reconocen ley, de estos que aquí
roban y hacen maldades muchas. Metidos en la Alcazaba, ¿quién sino ellos
dominará la ciudad?

---Y qué quieren: ¿rendición?

---No rendición quieren, porque los españoles cortarían sus cabezas.

---¡Y vosotros y yo y otros amigos que encontraremos, no somos capaces
de cortar las de ellos!---exclamé indignado ante la flema de aquellos
hombres sin sentido de la patria, ni del orden ni de nada.---¿Qué
hacemos entonces? ¿Dejar que esa canalla robe y asesine?\ldots{} ¿Estáis
vosotros decididos a permanecer aquí conmigo, con \emph{Abeir} y otros
hasta que entren los españoles?

---No: nosotros nos retiraremos esta noche, porque no queremos
rendición. Ni rendir nosotros, ni ver a Tettauen entregada al
cristiano\ldots{} Dejamos el caso en manos de Allah. La voluntad del
Excelso decidirá.

---Pero Allah, ya ves que está dormido. No hace nada por su pueblo; dice
a su pueblo: «Gobiérnate solo, y endereza tus destinos como puedas.»
Allah se duerme.

Al oír esto, aquel negro de mirada candorosa, de estatura colosal que a
la mía, no pequeña por cierto, sobrepujaba en el tamaño de una cabeza o
de cabeza y media, me puso la mano en el pecho, y con grave tono me
dijo: \emph{«El Nasiry}, tú no eres creyente. Decir que Allah dormita es
la mayor blasfemia, porque Allah es \emph{el Vivo}, \emph{el Vigilante},
es \emph{El que no duerme nunca}, y con estos nombres debemos adorarle
ahora.» Dejome aterrado y mudo con estas solemnes expresiones, cuya
verdad reconocí al instante. Sí: Allah no duerme; los ojos de Allah
velan con mirada profunda sobre todo el Universo. Dejemos que los hechos
corran y que la solución venga de lo alto. No imitemos la insana
inquietud de los cristianos y europeos, que se arrogan las facultades de
Dios, interviniendo en los sucesos humanos y enmendando la obra del
tiempo, como los chicos sin juicio que con el dedo adelantan o atrasan
los relojes sometiendo las horas a su pueril deseo.

Ya salíamos del \emph{Mellah} cuando me encontré a Riomesta, de tal modo
alterada su faz por el miedo y la consternación, que a primera vista no
le conocí. Para desfigurarse más, traía pañuelo azul por la cabeza,
atado debajo de la barba a estilo de mujer, ordinario empaque de los
judíos pobres. Llegose a mí antes que yo a él, y posando en mi mano las
dos suyas, me dijo con dolorido acento: «¡Oh, \emph{El Nasiry}, ventura
mía es toparte agora! Tú fuerte, tú señor, yo miserable\ldots{} soy
asemejado a pájaro solitario sobre techo\ldots{} Ceniza de pan comí, y
se acabaron cual humo mis días.» Comprendí que algún grave accidente
lloraba: su voz era como la del profeta hebreo llorante cabe las ruinas.
¿Le habían incendiado su casa, le habían robado el dinero? A mis
preguntas sobre la causa de su tribulación, respondió con mayor duelo:
«Hanme robado con ultrajaciones; mas no es esa la causa de mi lloro,
\emph{El Nasiry}. ¿No sabes que mi hija \emph{Yohar} huyó de mí, como
hembra liviana, culposa y aviciada de perversión? ¿No sabes que contra
su padre pecó, ladrona y escapadiza, llevándose llaves de mis arcas
soterradas, y joyas pulidas de esmeralda y aljófar?\ldots» Ninguna
noticia tenía yo de que la blanca \emph{Yohar} hubiese abandonado el
hogar paterno. ¿Cómo fue? ¿Quién la indujo a tan horrendo delito?

«Sabrás---dijo Riomesta mezclando el furor con las lágrimas---que
\emph{Yohar} se envoluntó con ese profeta cristiano que responde por
\emph{Yahia}, y que vino so color de predicar paces entre los hombres;
pero a lo que vino fue a meter víboras venenosas en el corazón de mi
\emph{Perla}, y dañar su mente con vicio\ldots{} ¡Oh, \emph{El Nasiry}!,
a mi soledad no hay consolación. Abandonado soy de Adonai. Polvo soy en
mis vidas, cuanto más en mi muerte\ldots{} En instante maldito salió
viva \emph{Yohar} del vientre de su madre. Engendrada fue con luenga
hondura de pecados\ldots{} La que antes me alegró, ogaño me ha trocado
en vasija de vergüenza y deshonra.» Lastimado del infortunio de mi
amigo, y sintiéndome además lastimadísimo en mi amor propio, como si
tuviese por mía la belleza y blancura de \emph{Yohar}, monté en cólera y
dije a Riomesta que si en alguna parte de la ciudad me topaba con el
mentiroso profeta \emph{Yahia}, le cortaría la cabeza.

«Acabo de saber---dijo sin aliento el afligido padre---que has salvado
la vida a \emph{Mazaltob}. ¡Oh, qué mala piedad la tuya, \emph{El
Nasiry}! Esa perversa es culpable de la huida de mi \emph{Yohar}; ella
envoluntó al \emph{Yahia}, enguapeciéndole como a barragán español; ella
le encendió con hechizos; ella trastornó los pensirios de mi
\emph{Yohar}; por ella moraron \emph{Yahia} y mi hija luengas horas en
su casa y en la de \emph{Simi}, la destiladora de perfumes. Entre las
dos han percudido el alma de \emph{Perla}, llenando la mía de pena y
cordojo. ¿Para qué has librado a la bestia \emph{Mazaltob} del fuego
eterno? Ya la tenía Belceboth clavada en su tenedor de tres puntas para
meterla en la paila de aceite hirviendo, cuando has venido a quitarla de
los hombres que hacían justedades\ldots{} Eres torpe, \emph{El
Nasiry}\ldots{} Mas si quieres estar entre los buenos, búscame a
\emph{Yahia}, el de la pacificación, y tráeme su cabeza en un plato,
ansí como trujo Salomé la del otro \emph{Yahia}, falso y engañador
profeta al igual de este\ldots»

No pude detenerme más, porque los compañeros que iban conmigo, fatigados
ya del lamentar angustioso del hebreo, me daban prisa para salir del
\emph{Mellah}. Dejé al pobre Riomesta en gran desesperación, tirándose
de las barbas y rasgando el pañuelo azul que con airado gesto se quitó
de la cabeza. Al separarme de él, fueron tras mí en corto trecho sus
últimas exclamaciones, que eran plegarias de su rito: «Dio piadoso,
luengo de furores, cata a mí, y apiádame\ldots{} ¿Por qué me alzaste y
me echaste? ¿Por qué maldeciste mi simiente?\ldots{} Mis días son sombra
declinada\ldots{} Se pegó mi hueso a mi carne\ldots{} Soy asemejado a
cernícalo del desierto\ldots{} En día de mi angustia te llamo que me
respondas\ldots»

Los dos cumplidos hombrachones del \emph{Sus} y el jefe de artilleros no
veían la hora de escapar, más que por miedo, por zafarse del desdoro que
pudiese caberles en la rendición de Tettauen. No podían defenderla ni
entregarla. Dejaban el suceso a la voluntad de Allah, manera muy cómoda
de salir del paso. Les acompañé un rato, y despedidos con toda
cortesanía, me volví a casa de \emph{Abeir}. La Junta o Asamblea de
Ancianos y Principales continuaba reunida: ya sabían el cambio de gente
por gentuza en la Alcazaba. Como no teníamos fuerza para impedir los
atropellos, se acordó fiarnos también en la divina voluntad, y esperar
el día, hasta que nuestra embajada fuese a O'Donnell y volviese con la
respuesta del \emph{Gran Español}.

Díjeles yo: «Mañana es domingo, día santo para los secuaces del
\emph{Hijo de María}. Los cañones de sitio estarán callados, y el
Ejército de O'Donnell no hará más que rezar y oír misas. Pero el lunes,
de fijo, veremos caer sobre nosotros espantosa lluvia de bombas y
granadas.» Soñolientos ya, entregados al fatalismo inherente a la raza,
no se mostraron inquietos por mis presunciones y anuncios alarmantes, ni
por los hechos positivos de que al poco rato tuvimos conocimiento. Había
yo dado a \emph{Ibrahim} órdenes de recorrer toda la ciudad y buscarme a
los dos compañeros que se nos habían perdido en el bullicio del
\emph{Zoco}, poco después del susto del asno hambriento. Llegó mi criado
a decirme que \emph{Ben Zuleim} y \emph{Abdalá Núñez} habían encontrado
al \emph{Bajá} que descendía de la fortaleza, dejándola en poder de los
malos: el \emph{Bajá} les habló y con él abandonaron la ciudad, como
buenos musulmanes que ponen en manos de Dios los conflictos que no saben
resolver. Abandonados de aquellos amigos, a cada instante éramos menos,
y a medida que se achicaba nuestro poder, las dificultades crecían de un
modo aterrador. Apuré yo mi fácil labia para señalar con los peligros
los deberes a que nos obligaban las circunstancias. Debíamos penetrarnos
de que constituíamos un pequeño \emph{Majzen}, o institución de
Gobierno, por poderes tácitos del Sultán. Éramos la autoridad, el
Estado, en una palabra, y en nuestras manos estaba la suerte de una de
las más bellas ciudades del Mogreb\ldots{} ¡Allah me asista! Fuera de
\emph{Ahmed Abeir}, que ponía vaga atención en mi discursillo, la Junta
de Principales no me comprendía, ni se hacía cargo de que éramos un
\emph{Majzen} más o menos chico. Hartos de tomar tazas de té, los
junteros se obsequiaban recíprocamente con estruendosos eructos, o
descabezaban un sueño sobre las blandas alfombras y mullidos cojines. A
una orden de \emph{Abeir}, los esclavos nos trajeron raciones amplias de
\emph{elquefthá} (carne asada en pinchitos), hojaldre, huevos cocidos y
pastelillos dulces. Yo no tomé más que un huevo y un pastel; alguno de
los Principales no fue parco en el devorar, y casi todos se tumbaron
luego en las colchonetas, y con sus ronquidos ásperos me recordaban los
estruendos de la batalla de aquel día. ¡Allah les conserve frescas sus
asaduras!

Quise dormir: pensaba en la blanca \emph{Yohar} y en el moreno
\emph{Yahia}, que debía de ser pájaro de cuenta, como aquel falso
profeta de la familia de los \emph{koreichitas}, de quien dijo el Santo:
«Con sus pérfidas ficciones de inspiración celeste, difundió la
idolatría y arrastró a las gentes al vicio.» Ya le sentaría yo las
costuras al tal \emph{Yahia}, si le encontraba\ldots{} Comprenderás,
Señor, que con tales pensamientos y la inquietud en que me tuvieron las
frecuentes noticias de nuevos desmanes, era imposible mi reposo\ldots{}
Hasta que aclaró el día no pude dormir; pero fue tan profundo el hoyo de
sueño en que cayó mi cansancio, que no sentí salir a los cinco
compañeros que iban de embajada al Cuartel General de O'Donnell.

Pasó más de una hora desde que me desperté, y estábamos \emph{Abeir} y
yo engolfados en nuestros devotos rezos, cuando volvieron los de la
embajada. La curiosidad, unida al patriotismo, nos movió a dejar para
otra hora las devociones, y oímos de boca de \emph{El Gazel} la relación
de la solemnísima entrevista con O'Donnell. Al llegar al campo español,
supieron que el Generalísimo había salido a caballo a reconocer las
inmediaciones de la ciudad por aquella parte. En tanto, la oficialidad y
tropa recibió a los comisionados moros con simpatía y afecto\ldots{}
Aguardaron mirando las tremendas baterías que armaban a toda prisa para
hacernos polvo, y en esto, y en hablar alguna cortés palabrita con los
oficiales, se dio tiempo a que volviera de su paseo el \emph{Gran
Español}. Este les recibió con exquisita urbanidad; entró en su tienda,
suplicándoles que le siguieran. Tomaron todos asiento, y\ldots{} Para
abreviar: antes que nuestra embajada llegase, ya había dispuesto el
\emph{Irlandés} otra que a Tettauen subiría con el siguiente recado
escrito en un papel. \emph{El Gazel} leyó la comunicación, de la que
copio aquí los párrafos de más substancia: «Entregad la plaza, para lo
que obtendréis condiciones razonables, entre las que estarán el respeto
de las personas, de vuestras mujeres, de vuestras propiedades y leyes, y
de vuestras costumbres\ldots{} Os doy veinticuatro horas de tiempo para
resolver: después de ellas, no esperéis otras condiciones que las que
imponen la fuerza y la victoria.» Con esto tuvo bastante la embajada, y
no necesitaba prolongar la conferencia. Al despedirlos sonriente,
O'Donnell les dijo: «Mañana a las diez se disparará el primer cañonazo,
si no recibo contestación satisfactoria.»

\hypertarget{ix-1}{%
\section{IX}\label{ix-1}}

La voluntad del Excelso estaba bien clara. España sería dueña de
Tettauen, aunque otra cosa dijese un \emph{Kaid} de las tropas acampadas
al Oeste, el cual nos mandó un emisario con la notificación de que ellos
defenderían la ciudad hasta morir, y que no se hablara de rendición ni
cosa tal\ldots{} Ni aun le dejamos concluir, y despachado fue sin
ceremonia. Luego se nos dijo que algunos de estos valientes de última
hora, entrando en la ciudad, ocuparon las baterías que protegen las
principales puertas del recinto\ldots{} Supimos también que no éramos
nosotros la única Junta de vecinos inclinados a la rendición, pues otras
dos se habían formado en la Alcaicería y barrio de Curtidores, y nuestro
primer cuidado en el resto del día fue ponernos en comunicación con
ellos. ¡Oh, qué desconsolado y afanoso aquel día que los cristianos
llamaban \emph{Domingo, 5 de Febrero!} En algunos puntos de la ciudad,
tumulto y hervidero de riñas; en otros soledad de cementerio; en todos
escombros, restos del pillaje, sangre, lodo y basura. Si bien éramos
pocos los partidarios de la rendición, lo corto del número se compensaba
con la calidad de las personas, con su valor y poderío. Esto se vio
claramente aquella tarde, cuando se acordó desalojar de revoltosos
riffeños y anyerinos la batería de \emph{Bab-el-aokla}. Siete estacas en
manos de siete señores realizaron felizmente la breve operación militar.

De estas cosillas y otras no pude enterarme por mí mismo, y de ello tuvo
la culpa \emph{El Gazel}, que, como español, es un pozo de astutas
maldades\ldots{} Antes de seguir, Señor mío, confesarte quiero un
horrendo pecado que cometí aquella tarde, y que me puso a dos dedos del
infernal abismo. Y fue que en vez de evitar yo la compañía del execrable
Gazel, dejé a mi alma en la libertad de gustar de ella\ldots{} Señor, no
supe resistir a la tentación del renegado cuando quiso llevarme a su
casa prometiéndome el descanso y la dulzura que nuestros amargados
humores necesitaban. Vive el pérfido español junto a la gran Mezquita,
en casa de regular apaño para una existencia cómoda. Sus mujeres había
mandado a Tánger o Arsila, no estoy bien seguro, dejando aquí de
servidumbre a un negrito vivaracho. Apenas entramos Torres y yo en la
casa y nos tumbamos sobre los blandos almohadones, trajo el negrito una
garrafa de aguardiente y vasos para beberlo\ldots{} Yo me resistí; hice
muchos ascos; pero tales fueron las instancias de \emph{El Gazel} y tan
extremados y persuasivos sus elogios de la virtud de aquel licor, que me
determiné a probarlo\ldots{} ¡Ay, Señor!, nunca lo hubiera hecho, pues
catarlo fue lo mismo que sentir el ardiente deseo de nuevas pruebas y
cataduras, y a medida que cataba, mi cabeza se iba inflamando en insanas
alegrías\ldots{}

Para castigar mi olvido de la sacra ley que nos prohíbe beber zumo
fermentado de uvas, el Señor permitió que yo me encendiera en un bárbaro
apetito de beber más y más, hasta llegar a un estado de infernal
demencia\ldots{} Ya no necesitaba yo que \emph{El Gazel} me ofreciera
nuevas tomas de aquel veneno, porque yo mismo, espoleado por un gusto
superior a toda razón, cogí la botella, llenaba el vaso mío y el del
otro\ldots{} En fin, Señor, que se me fueron a los aires la cabeza, los
nervios, el sentido, y perdí mi conciencia musulmana, y se hizo polvo la
torre de mi fe. No puedo decirte la cantidad de vasitos que llevé a mi
boca; sí te digo que mi borrachera fue de las más soberanas que se han
conocido en la historia del vicio, y mi pecado de los que no pueden ser
redimidos sino con una vida entera de abstinencia. ¡Ay, ay, ay!,
lágrimas amargas corren de mis ojos al referirlo, Señor. Ten piedad de
mí, y encomiéndame a la misericordia del Benigno.

Sin poder precisar ahora las necedades que hice y dije en mi vergonzosa
embriaguez, sé que mis carcajadas debieron de oírse en los picos de
\emph{El Dersa}, y que, sensible al mal ejemplo de mi perverso amigo,
pronuncié frases vejatorias contra el Dios Único, injurias contra el
santo Profeta y sus mujeres \emph{Khadidja}, \emph{Aicha} y \emph{María
la Copta}, y contra su afamada camella \emph{Koswa}, poniéndolas a
todas, camella y mujeres, como hoja de perejil\ldots{} ¡Ya ves, Señor,
qué monstruosos pecados! Verdad que yo no supe lo que decía; pero mi
ignorancia no me disculpa, porque con plena conciencia hice la primera
catadura del maldecido brebaje\ldots{} Por fin caí en profundo sopor,
que tal es el término y resolución de estas crisis infernales. Los
españoles, dueños de un lenguaje riquísimo en voces picarescas y
desvergonzadas, llaman a estos sueños vinosos \emph{dormir la mona}. No
sé cuánto tiempo estuve tendido en las alfombras de \emph{El
Gazel}\ldots{} no sé cómo salí a la calle después de esta primera
\emph{mona}\ldots{} Me contó luego un amigo que salí vociferando,
suponiéndome montado en \emph{Koswa}, la camella del Profeta, y que
proferí no sé qué atrocidades indecentes contra el Sultán, contra el
\emph{Majzen} y contra la respetable Junta de Principales\ldots{} A esta
\emph{mona} primera, otra siguió, la cual dormí ¡oh vilipendio!, en el
último escalón del pórtico de la sagrada Mezquita, y en este sopor fui
más estrafalario y licencioso que en el primero. Soñé que estaba yo en
brazos de la blanca y tersa \emph{Yohar}, y que delante tenía, en una
bandeja de plata, la cabeza del profeta \emph{Yahia}, aderezada con buen
golpe de sal para que tuvieran tiempo de adorarla sus discípulos los
\emph{pacificantes}\ldots{}

No puedo precisar la hora de mi despertar de la segunda \emph{mona}. Me
sentía con todos los huesos doloridos, el entendimiento envuelto en
pesadísima niebla, la memoria como desleída en una papilla opaca\ldots{}
Quise levantarme, y no pude: mi voluntad era otra papilla espesa, en la
cual no podía vibrar ninguna resolución\ldots{} Chiquillos hebreos y
moros vinieron a hacerme compañía; perros vi escarbando en las basuras,
y unos y otros, con distinto lenguaje, me dijeron que yo estaba dejado
de la mano de Allah y que nunca obtendría perdón. Pero no debió de
abandonarme enteramente Dios Misericordioso, porque mi fiel
\emph{Ibrahim}, que toda la noche me había buscado por la ciudad, halló
a su amo en la situación lamentable que para mi vergüenza describo.
\emph{«Sidi}---me dijo sentándose a mi lado,---bendiga Dios el instante
en que te encuentro. Grandes calamidades sufrimos, y es bueno que juntos
señor y criado hablen del remedio de tantas desdichas. Sabrás que los
salteadores han vuelto, y no hallando en el Mellah nada que robar, han
saqueado viviendas de moros\ldots{} Sidi, no extrañes que no te cuente
con pormenores lo que ha pasado esta noche, porque estoy sin aliento; mi
cuerpo se desmaya, se aniquila; la vida se me quiere escapar, sin que
con toda mi voluntad pueda detenerla.»

---¿Estás herido, \emph{Ibrahim}? ¿Cuál es tu mal? Por Allah que si no
es hambre, no entiendo qué mal pueda ser.

---No se me va la vida por la puerta de ninguna herida, sino por otra
puerta, no hecha con arma blanca ni arma de fuego\ldots{}

Diciendo esto se retiró presuroso, dejándome sobrecogido, y a poco tornó
a mi presencia con los alientos más desmayados. Su voz salía del pecho
como de un fuelle roto las ráfagas débiles del aire. «Por Allah
Reparador, lo que tú padeces, \emph{Ibrahim}, es el cólera. Vete pronto
a casa, aunque vayas arrastrándote. Acuéstate, y que \emph{Maimuna} te
haga té bien caliente.»

---A tu casa no voy, \emph{Sidi}, si no me das escolta de los ángeles
\emph{Djebreil} e \emph{Israfil}, ni tú irás tampoco, porque tu casa
está llena de maleficio. ¿No te dije que la maga \emph{Mazaltob}, al ir
con el falso motivo de pedirnos limosna, cuando tú estabas en la
batalla, fue a poner en tu morada el más nefando sortilegio que
inventaron los demonios? Yo sospeché, \emph{Sidi Mohammed El Nasiry}; te
conté mis barruntos, y tú soltaste la risa. Pues lo que yo sospeché y
temí ha salido cierto, y ahora no puedes ir a tu albergue, porque está
lleno de infernales espíritus que después de quitarte la vida, te
cogerán por los cabellos y te arrastrarán a la \emph{Gehenna}.

Perplejo y acongojado, pregunté a \emph{Ibrahim} qué sortilegio había
llevado a mi casa la discípula de Satán, y él, después de alejarse otro
momento para ir a un menester apremiante de su maligna enfermedad,
volvió y me dijo: «Bien puedes imaginarlo, El Nasiry: es el
embrujamiento más terrible; el que contra el mismo Profeta emplearon los
\emph{mosaístas}, y consiste en lo que se llama \emph{soplar sobre los
nudos}. \emph{Mazaltob}, profesora en el embrujar, posee el secreto, y
ahora tú eres la víctima. ¿No lo entiendes? Esa perra, es loba de
Israel, hizo once nudos en una cuerda, y después de soplar en cada uno
de ellos, diciendo unas oraciones endemoniadas, colgó la cuerda dentro
del pozo de casa. Con esto basta para que tú, tu familia y criados
sufran algún golpe de adversidad muy dura, que acabará en muerte, y el
primer ejemplo tienes en mí, que me veo con el terrible corrimiento del
cólera.»

---¿Pero has visto tú la cuerda con los once nudos, \emph{Ibrahim}?

---Pues si la hubiera visto, segura era mi muerte instantánea. Para que
te convenzas, \emph{Sidi}, y no dudes de que la \emph{Mazaltob} te ha
soplado los nudos, te bastará saber que al anochecer, hallándonos
\emph{Maimuna} y yo en la casa disponiendo nuestra cena, sentimos que
puertas, ventanas y ventanillos daban horribles traqueteos, como si un
furioso viento se paseara por todos los aposentos de la casa. Cuando
tratamos \emph{Maimuna} y yo de ver lo que aquello era, caímos al suelo
y se nos encandilaron los ojos con un gran resplandor de relámpago
verde\ldots{} Vimos luego diablos que recorrían la casa, azotando con
sus rabos los muebles, echando a rodar toda la loza y cristales, y
entonando unos canticios desvergonzados que nos helaron la sangre en las
venas\ldots{} Te contaré ahora lo más grave, \emph{Sidi}. He aquí que
hallándonos aturdidos y deslumbrados, vino a nosotros una diabla, por
más señas muy parecida a \emph{Mazaltob}, y nos machacó los huesos con
un palo, echando de su boca conjuros indecentes; después le quitó a
\emph{Maimuna} las llaves de la casa, que en la cintura llevaba; a los
dos nos empujó hasta echarnos a la calle\ldots{} La sentimos cerrar por
dentro\ldots{} Apenas pusimos el pie en la calle, a los dos nos atacó
este mal\ldots{} A un tiempo fuimos acometidos del primer desmayo frío
de nuestro vientre. Ella echó por un lado, yo por otro. Después de mucho
andar, desmayándome del cuerpo bajo\ldots{} infinidad de veces, he
tenido la suerte de encontrarte para decirte: \emph{«Sidi}, no vayas a
tu casa.»

---No iré\ldots{} Me has puesto en cuidado. Pero pienso que en la Fe y
en las Escrituras encontraremos algún arbitrio para chasquear al perro
Satán\ldots{} Dime, \emph{Ibrahim}: ¿me engañan mis ojos, o es verdad
que amanece?

---Ya viene el día, \emph{Sidi}\ldots{} Bendita sea la luz del Sol. ¿Te
acuerdas del capítulo \emph{Ciento y tres} del \emph{Korán}?

---Sí que me acuerdo. Ese capítulo recito yo todos los días en cuanto
veo la luz solar. Es breve y hermoso de toda hermosura y unción.
Repitámoslo juntos: «Busco un refugio contra ti, Señor del Alba, Señor
del Día\ldots{} Refugio contra la iniquidad de los seres malos que has
creado\ldots{} Refugio contra el mal de la noche sombría\ldots»

---Refugio contra la perversidad de \emph{los que soplan sobre los
nudos}\ldots{} Refugio contra los envidiosos.

Tres o cuatro veces repetimos con intensa devoción las sublimes palabras
del Profeta. Después me dijo \emph{Ibrahim}: «En otro lugar del Libro
Santo encontrarás el remedio que empleó el Profeta contra el
embrujamiento judaico de los once nudos. Has de leer con grandísima
devoción y recogimiento once capítulos del \emph{Korán}; a cada lectura
de un capítulo, siempre que sea lectura con piedad, se deshará uno de
los nudos, y en cuanto los once sean deshechos, desaparecerá el
maleficio.»

\hypertarget{x-1}{%
\section{X}\label{x-1}}

La claridad del día reanimó mi espíritu abatido, infundiéndome la
esperanza de salir airoso de tantas calamidades. Propuse a
\emph{Ibrahim} que fuéramos a la casa de la Junta, donde yo encontraría
un Korán que leer, y él mejor acomodo para su enfermedad. No me
respondió, porque otra vez había ido a su negocio\ldots{} Le esperé, y
enlazándonos del brazo para darnos apoyo recíproco, nos dirigimos a casa
de \emph{Abeir}, la cual por fortuna no estaba lejos\ldots{} Diversa
gente encontramos por el camino, en su mayoría judíos pobres y moros
pordioseros, y más de cuatro nos preguntaron: «¿Entran ya los
españoles?\ldots{} ¿Traerán comida?» Respondíamos afirmativamente, y
observábamos que nuestra respuesta ponía el júbilo en todos los
semblantes. Al verme entrar en su patio, el buen \emph{Abeir} me dijo
con la más honrada convicción: «Allah te lo premie. Ya sé que has pasado
la noche apaciguando a los exaltados y consolando a los menesterosos. En
tu casa has dado albergue a los que perdieron el suyo. Dios Benigno
aumentará tus bienes, \emph{El Nasiry}.» Con una reverencia grave
asentí, no atreviéndome a responder de otro modo, por no mentir con
palabras, que es el verdadero mentir. Dije que a su casa iba en busca de
sosiego para el rezo y las abluciones, así como para prestar auxilio a
mi servidor en su enfadosa dolencia. Risueño y afable me franqueó
\emph{Abeir} su vivienda grata. Antes de media hora, ya los diligentes
esclavos cuidaban de \emph{Ibrahim}, y yo me entregaba al piadoso rezo
en el Libro Santo, comenzando la serie de lecturas que habían de
producir el desate de los fatídicos nudos del sortilegio.

Pero he aquí que cuando me hallaba yo en el tercer nudo, o sea en la
lectura y meditación correspondientes, un gran ruido de la calle me
apartó de mi espiritual ejercicio. Fui llamado con apremiantes voces.
Corrí\ldots{} \emph{Abeir} se había lanzado afuera con otros compañeros.
Los demás y \emph{El Gazel}, a quien Allah confunda, tiraron de mí. ¿Qué
ocurría? ¿Qué terremoto estremecía la ciudad en sus cimientos? ¿Qué
tempestad disparaba en los aires exclamaciones de ira y de muerte? Pues
nada: sucedía que por una parte los españoles, levantado su campo,
marchaban hacia la ciudad, mientras los descontentos musulmanes del
Ejército vencido se aproximaban por la otra, amenazando con pasar a
cuchillo al vecindario si abría las puertas al \emph{perro} cristiano.
De modo que \emph{la blanca paloma}, cogida entre dos fuegos y entre dos
iras, no tendría ya salvación. El peligro me infundió valor. Quiso Allah
que el corruptor de mi virtud, Torres \emph{El Gazel}, se hallase al
lado mío en aquellas difíciles circunstancias. ¿Qué había de hacer yo
más que seguirle y obrar con él mancomunadamente, pues se trataba de
asuntos políticos y no de cosa pertinente a las buenas
costumbres?\ldots{}

Corría la medrosa multitud hacia las puertas por donde presumía que los
españoles harían su entrada. Grupos de riffeños procedentes de la
Alcazaba intentaban ocupar los baluartes artillados próximos a dichas
puertas. \emph{El Gazel}, más sereno que yo, me dijo que no debíamos
acudir a \emph{Bab-el-aokla}, sino a \emph{Bab-el-echijaf}, pues él
sabía que O'Donnell intentaba entrar por esta parte. En medio del
tumulto, supimos que \emph{Ahmed Abeir} y otros compañeros Principales
se habían ido a \emph{Puerta de Fez}, por donde querían entrar los
insensatos partidarios de la resistencia. ¿Lograrían atajarles? Más
fácilmente les atajaría el General Prim, que con los \emph{catalonios},
según allí dijeron, se encaramaba por los muros exteriores de la
Alcazaba, con la diabólica idea de ocupar aquella posición eminente y no
dejar allí títere con cabeza. Tomada la fortaleza, ¿qué podían hacer los
levantiscos montañeses más que ponerse en salvo, como los ratones a la
vista y olor del gato que ha de comérselos?

De fuera de la ciudad venía un rumor de cornetas que hacía temblar de
emoción a los que, hambrientos y sin hogar, habían perdido toda noción
de patriotismo. «Ya están ahí,» me dijo \emph{El Gazel} con una
expresión de júbilo picaresco que nunca podré olvidar, y corrió hacia
\emph{Bab-el-alcabar}. No fui tras él, porque en aquel instante se
reprodujo en mí el extraño sentimiento que paralizó mi acción en la
batalla, el terror del rostro de los españoles, a que no podía
sobreponerme. Como niño asustado, llegué a creer que tapándome mi cara,
no podían las suyas inspirarme tan singular confusión y
azoramiento\ldots{} Mas he aquí que en esto veo venir una banda de
riffeños procaces, que clamaban en roncas voces contra España, y de paso
arrojaban al suelo a desdichados ancianos judíos y a infelices mujeres.
Me cegué; tiré de yatagán y les acometí con fiereza, desembarazándome al
instante del que más próximo tenía. Dos moros de buen pelo se pusieron a
mi lado, y con garrotazos bien dirigidos me ayudaron a la dispersión de
la chusma\ldots{} Envalentonados por mi pronta defensa, los judíos
corrieron hacia \emph{Bab el-alcabar} dando vivas a España y a su
Reina\ldots{} Pero estaba de Allah que yo no saliera en bien de aquellas
aventuras, porque al volverme hacia los dos moros de buena traza que me
habían auxiliado, no vi más que a uno, y el que vi\ldots{} pareciome
sueño\ldots{} era el maldecido y execrado profeta español, ladrón de la
blanca \emph{Yohar}.

Dudé un momento que fuera \emph{Yahia} quien frente a mí tenía, porque
su elegante porte y fina vestidura desdecían del empaque pobrísimo con
que le vi en casa de \emph{Mazaltob}. Pero él mismo disipó aquella
sombra de duda, diciéndome: «Yo soy, yo soy Juan, no \emph{Yahia}, como
tú me llamas, y harás bien en declararte mi amigo, pues yo te tengo ley,
no sólo por lo que eres y lo que vales, sino por memoria de tu familia.»
Fue mi primer impulso echarle mano al pescuezo; pero la dulzura de sus
expresiones afables me alivió del coraje que sentí. «No hallarás en mí
benevolencia---le dije,---sino un terrible castigo, como no me expliques
al instante qué has hecho de \emph{Yohar}, cuya piel obscurece la
blancura de las azucenas.»

---Pues la dulce \emph{Yohar}, cuyo corazón de miel labraron las abejas
del cielo, está buena y sana, en lugar seguro. En su nombre, sabiendo yo
lo que te estima, te deseo la paz\ldots{} Pero si quieres más informes,
apartémonos al abrigo de aquel caserón derruido, que allí veo unos
gandules que a mi parecer están en actitud de apedrearnos. Vente acá,
\emph{El Nasiry}, y con explicaciones te demostraré que debes ser mi
amigo.

Dejeme llevar a donde él quiso, moviéndome a ello, no sólo la
curiosidad, sino el deseo de hallar en sus explicaciones motivo, más que
de afianzar amistades, de desatar furores. Nos hallábamos muy cerca de
\emph{Bab-el-echijaf}, cuyos aproches y baluartes invadía la multitud.
Al amparo de unas ruinas, prosiguió \emph{Yahia} de este modo: «Me
alegro de verte en esta ocasión, que es de grande alegría para todos. Yo
celebro la entrada de los españoles en Tetuán, porque esto significa la
paz próxima, beneficio para nosotros, y más aún para el Mogreb. La paz
es mi sola idea, \emph{El Nasiry}; la paz es mi aliento. Odio la guerra,
y deseo que todos los pueblos vivan en perpetua concordia, con amplia
libertad de sus costumbres y de sus religiones. Echar a pelear a Dios
contra Allah, o a este contra Jehovah, es algo semejante a las riñas de
gallos, con sus viles apuestas entre los jugadores. Pero la paz no sería
buena y fecunda sin el amor, que es el aumento de las generaciones, y la
continuación de la obra divina. Dios no dijo \emph{Menguad y dividíos},
sino \emph{Creced y multiplicaos}. Luego Dios bendijo el amor, y condenó
las estúpidas guerras. A mí, trayéndome a este pueblo por extraños
caminos y con evidente cariño tutelar, me ha dado aquí el amor, pues si
yo quedé prendado de la hija de Riomesta en cuanto la vi, ella me mostró
desde el primer instante una inclinación ciega. ¡Paz y amor! ¿Qué más
pude soñar?»

---Farsante, impostor, hilandero de frases galanas con palabras
floridas, no pienses que me engañas o que me adormeces con tu hablada
música traidora\ldots{} Dime, dime pronto dónde escondes a \emph{Yohar},
que quiero rescatarla y devolverla a su padre dolorido. Si no me
contestas pronto, te trataré como mereces, y no verás la entrada de los
tuyos.

---Veré la entrada de los míos---replicó el maldito \emph{Yahia} con
frío tesón,---porque en mí no hay maldad. ¿Cuándo fue maldad el amor?
\emph{Yohar} es mía, y tú, tú mismo, \emph{El Nasiry}, vas a decirle al
buen Riomesta que me deje a su \emph{Perla} y no interrumpa nuestra
felicidad.

---¿Por ventura estás decidido a comprar la blancura de \emph{Yohar} con
tu abjuración de la fe del \emph{Hijo de María}?

---Nunca tal pensé, y cristiano he de morir. Aspiro a que ella confiese
la religión de Cristo nuestro Redentor\ldots{} España está ya en Tetuán,
y a la sombra de la bandera de O'Donnell, \emph{Yohar} será cristiana;
cristiana como yo\ldots{} como tú.

Esto de llamarme a mí cristiano, la más grande y mentirosa injuria que
en mi vida escuché, debió causarme irritación; pero por la enormidad del
disparate sólo sentí desprecio y ganas de echarme a reír. No pudiendo
soportar las insolencias de aquel miserable, le agarré por un brazo, y
no sé lo que habría hecho con él, si en el instante mismo no resonara un
clamor que nos notificó la entrada de Prim en la Alcazaba, escalados los
muros de esta por los aguerridos \emph{catalonios}.

«De tus violencias conmigo---me dijo \emph{Yahia},---te arrepentirás
pronto, y me concederás tu amistad\ldots{} No temo revelarte lo que aún
ignoras. ¿Me preguntas que dónde está la \emph{Perla}? Pues en el lugar
más seguro de Tetuán; en tu casa, \emph{El Nasiry}, en tu propia
casa\ldots{} Allí buscamos amparo, acosados y hambrientos. Confiando en
tu benevolencia, fuimos a pedirte hospitalidad; no quisieron dárnosla, y
la tomamos. Tú habías dicho: «Si no tenéis vinagre para curar sus
heridas a \emph{Mazaltob}, id a buscarlo a mi casa\ldots» Fuiste
obedecido, ilustre señor. Tu casa es el refugio de los
menesterosos\ldots{} ¿Por qué te asombras de lo que te cuento? ¿Qué
sentimientos expresa tu rostro? ¿Es la ira, es la compasión? A fe que no
te entiendo.»

Ni yo, en verdad, tampoco me entendía. Ved aquí el motivo, Señor. Sobre
el grave murmullo de la multitud apelmazada y ansiosa, se destacaba el
son vibrante de cornetas. Los españoles se aproximaban; les precedía la
voz metálica de sus músicas guerreras, que rasgaban el aire, o lo
cortaban con estridencia, como el diamante corta la plancha de vidrio.
El ruido de cornetas renovó en mi espíritu con indecible fuerza el
terror que los rostros de españoles me causaron el día de la batalla.
Pero en aquel \emph{Lunes 6 de Febrero} fue tan intensa mi pavura, que
ni aun me dejaba fuerzas para huir. Huir era mi anhelo más hondo; pero
este hondísimo anhelo me decía: «No te muevas.» ¿Verdad que es raro,
incomprensible?\ldots{} Deseaba yo que los españoles entrasen; pero no
quería verlos\ldots{} verlos no.

Cayó mi ser en intensa perplejidad; me sentí pececillo a quien meten
dentro de una redoma con su agua correspondiente. En aquel estado, oía
las cornetas fatídicas; oía el relato de \emph{Yahia}, sin poder
contestarlo. Y la voz del español, penetrando en mi cerebro con claridad
y vibración semejantes a las de los clarines guerreros, me decía: «En tu
morada hallamos consuelo los perseguidos. \emph{Mazaltob} es mujer buena
y sin hiel, aunque tú creas lo contrario. Si le salvaste la vida, ¿por
qué te asombras de que viera en ti el hombre pío y generoso, y buscara
el abrigo de tu casa? Allá fuimos todos, yo con \emph{Yohar} la blanca,
\emph{Mazaltob} con sus cardenales, y \emph{Simi} la destiladora de
perfumes\ldots{} Bajo tu techo encontramos seguridad\ldots{} ¿Qué fue de
tus servidores? ¡Huyeron, dejándonos las llaves, hermoso acto de agudeza
y discreción, que creímos ordenado por ti mismo!\ldots{} De estancia en
estancia, lo recorrimos todo. El infalible olfato de \emph{Mazaltob}
descubría los manjares guardados en las alacenas. Comida encontramos, y
especias, miel y té\ldots{} En tanto, \emph{Simi} revolvía la cocina,
donde halló carbón y leña, pedernal y yesca para encender lumbre.
Nuestras bocas bendecían al sabio, al caritativo \emph{Ben Sur El
Nasiry}. Para que nada faltase, \emph{Yohar} descubrió los blandos
lechos que nos ofrecían dulce descanso\ldots{} Y no paró aquí el talento
de mi \emph{Perla}, pues revolviendo arcones y armarios, dio con estas
elegantes ropas, y mostrándomelas me dijo: «Amado mío, honrarás la casa
del señor adornando con sus galas tu mancebía\ldots» Me vestí\ldots{}
reproduje tu persona gallarda.

\emph{¡Con doscientos y el portero}, y por Allah Gracioso, que no sé, al
escribir esto, si debieron moverme a indignación o a risa las
manifestaciones de \emph{Yahia}, original y desvergonzado profeta! Pero
en aquel momento, yo era tan incapaz de regocijo como de cólera, por el
tristísimo estado de atonía y de inmovilidad en que me puso mi pavor de
los rostros hispanos\ldots{} El estupor me convirtió, no diré que en
estatua, sino en muñeco relleno de paja o serrín\ldots{} Ya estaban los
españoles al pie de los muros; ya la multitud se arremolinaba en la
trágica disputa de abrir o no abrir las puertas\ldots{} Yo, mudo y
alelado, miré en el cuerpo de \emph{Yahia} mi elegante caftán listado de
rosa y amarillo, en su cabeza mi turbante tan blanco como el rostro de
\emph{Yohar}, y\ldots{} lo mismo pude acogotarle que abrirle mis
brazos\ldots{} lo mismo arrancarle el traje que felicitarle por su
agudeza. Como el estridor metálico de las cornetas ya próximas,
retumbaron en mi cerebro estos dichos de \emph{Yahia}: «Odio la guerra,
y en ella soy todo ineptitud. Pero si no sirvo para combatir, en los
pueblos asolados por la guerra sé encontrar pan para los hambrientos y
ropa para los desnudos. Créeme, \emph{El Nasiry}: la guerra deja en
cueros a los hombres, y la guerra los viste.»

No supe contestarle. Mi turbación ¡ay!, iba en aumento; yo no podía
tenerme en pie. Ya estaban allí los españoles; ya se les franqueaba la
puerta\ldots{} Aparté de \emph{Yahia} mis aterrados ojos, y humillándome
en tierra, oculté con las manos mi cara, para que ningún nacido la
viera\ldots{} El grito de ¡Viva España! ¡Viva la Reina de España!,
proferido por los hebreos, me dio tal escalofrío, que hoy mismo me
estremezco al recordarlo. Oía la voz de \emph{Yahia}: «Ya estamos en
Tetuán; ya Tetuán es nuestra. Alégrate, \emph{El Nasiry}, y celebremos
juntos la victoria de España y la paz\ldots» Seguía yo tapándome
cuidadosamente el rostro para que el desvergonzado profeta no viera las
lágrimas que de mis ojos a raudales salían\ldots{} ¡Allah sea conmigo y
me libre de los perversos que \emph{soplan sobre los nudos}!

Punto final pongo a mis cartas, ¡oh sabio y poderoso \emph{Cheriff Sidi
El Hach Mohammed Ben Jaher El Zébdy}!\ldots{} He cumplido tu encargo.
Vencido el Islam, y dueños ya de Tetuán los españoles, hoy \emph{Lunes
13 de Rayab de 1276}, te pide tu bendición y la venia para no escribirte
más de estas cosas tu ferviente amigo y deudo, \emph{Sidi El Hach
Mohammed Ben Sur El Nasiry}.

\hypertarget{cuarta-parte}{%
\chapter{CUARTA PARTE}\label{cuarta-parte}}

\begin{flushright}
\textbf{Tetuán, Enero-Febrero de 1860.}
\end{flushright}

\hypertarget{i-3}{%
\section{I}\label{i-3}}

No siendo cosa segura que el descarado profeta \emph{Yahia} escriba el
relato de sus aventuras pacificantes, conviene utilizar aquí datos y
noticias de la propia \emph{Mazaltob}, para llenar el vacío biográfico
de Santiuste desde que abandonó a los españoles hasta que los encontró
victoriosos dentro de los muros blancos de \emph{Ojos de Manantiales}.

Transportado, como se ha dicho, en el asno de Esdras, entró el profeta
con sus bienhechoras por \emph{Bab-et-tsuts} sin ningún tropiezo, y con
la misma felicidad llegaron todos a la casa de la hechicera en el
\emph{Mellah}. Compadecidas del herido y admiradas de su mansedumbre,
\emph{Mazaltob} y \emph{Simi} (que era una de las que cogían hierbas en
el verde prado), se aplicaron a curarle la contusión que tenía detrás de
la oreja, lo que no fue difícil. Con la quietud y el alimento, este no
muy del gusto del enfermo, pero eficaz para repararle, la contusión
quedó remediada; pero el estado total de Juanito no era satisfactorio,
pues a más del decaimiento y de la fiebrecilla que no quería remitir, se
hallaba privado en absoluto del uso de la palabra. La idea de fingirse
mudo había obrado en su organismo con demasiada intensidad\ldots{} Diole
\emph{Mazaltob} caldos de ranas, que aseguró eran eficacísimos para
estimular las facultades oratorias, y no obteniendo el resultado que se
esperaba, discurrió \emph{Simi} aplicarle un remedio cabalístico llamado
el \emph{Abracadabra}, palabra mágica de origen caldeo, que, según el
médico famosísimo Sereno Sammónico, tiene la virtud de despertar en la
humana laringe el apetito de la conversación. Sabía \emph{Simi} la forma
y manera de la aplicación del \emph{Abracadabra}, que consistía en
escribir el mágico vocablo en un papel, desarrollando sus letras en
triángulo; este papel se doblaba de modo que no se vieran las letras, y
se ajustaba a la garganta del individuo atacado de mudez. Hecho esto, se
encomendaba el caso con oraciones, haciendo constar en ellas que
\emph{Abracadabra} fue la primera palabra que oyó Adán de boca del Padre
Eterno, cuando este creyó conveniente hablar con su criatura\ldots{}
Tuviese o no virtud efectiva este divino talismán, ello es que, al día y
medio de tenerlo aplicado a su nuez, salió Santiuste echando cada
discurso que daba gloria oírlo.

En tono familiar exento de pedantería el poeta y trovador hablaba de la
paz, y era elocuente por lo mismo que no se curaba del efecto oratorio.
Su gracia persuasiva se manifestaba desde que abría la boca, y el puro
lenguaje castellano, adornado de bellas imágenes, la pronunciación
castiza y musical, eran el encanto de su auditorio, hecho al desabrido
acento judiego-español. Además, su éxito era mayor por hablar a
convencidos. Los hebreos, raza mercantil esencialmente pacífica, sin
hogar propio, privada en absoluto de arrogancias militares, ni amaba ni
entendía la guerra. La espada de Josué desde luengos siglos había sido
vendida como hierro viejo. Por su carácter dulce y su fácil y sugestiva
palabra, Satiuste fue bien quisto en la \emph{Judería} y su arrabal de
\emph{Meca}, así como en el que llaman \emph{El Prado}. Vistió
\emph{Mazaltob} a su huésped con un balandrán viejo, que no venía mal al
cuerpo del español; le puso la faja encarnada y el bonete negro, y le
mandó a que viera la ciudad y la corriese por todo el misterioso
enredijo de sus calles. En el \emph{Mellah} y fuera de él, los que no le
oían hablar teníanle por un \emph{sephardim} que había venido de
Salónica o de Jerusalén a negocios comerciales.

Rodando por Tetuán, pudo apreciar el aventurero que si moros y judíos se
peleaban por cuestiones de ochavos, nunca lo hacían por motivos
religiosos: sinagogas y mezquitas funcionaban con absoluta independencia
y recíproco respeto de sus venerados ritos. Observó también que los
sacerdotes hebreos, así como los musulmanes que sin carácter
eclesiástico prestan servicio en los templos del Islam, eran casados, o
disfrutaban la posesión de mujeres con más o menos amplitud. De esto
quizás provenía la tolerancia, porque, a juicio de Santiuste, el
celibato forzoso es como amputación que trae el desarrollo de los
instintos contrarios al amor: el egoísmo y la crueldad. Observó asimismo
que la falta de libertades políticas y el desconocimiento absoluto de
las constituciones producían en el Mogreb una sencillez legislativa y
jurídica que facilitaba la existencia. Érale grato el país en que había
caído; la dignidad y el flemático determinismo de los musulmanes le
encantaban. Si alguno de estos, con conocimiento del castellano, le caía
por delante, Juan le hablaba de la guerra, naturalmente para condenarla.
Decía entonces el moro que ellos no habían declarado la guerra, sino que
era el Español quien traía la muerte al santo territorio del Mogreb. A
los cristianos, que no a los moros, debía el sujeto predicador de paz
endilgar sus amenos discursos.

No tomaba Juan en serio la misión de profeta que \emph{Mazaltob} y
\emph{Simi} querían ver en él. El espíritu del exaltado mozo se había
serenado desde que le llevaron aquellas buenas mujeres a la sosegada,
aunque no muy limpia, existencia del \emph{Mellah}. Profeta de paz no
podía ser con los hebreos, que ya desde siglos remotos abominaban de la
guerra, ni con los moros, que sólo peleaban a la defensiva, ni con los
españoles, que jamás se quitarían de la cabeza el delirio deslumbrador
de las empresas militares. Pero no creyéndose llamado a catequizar
directamente a las tres razas afines, sentía dentro de sí un vago
prurito de manifestar sus ideas, no por los discursos, sino por la
acción\ldots{} más claro: creíase llamado a ser apóstol de la paz, no
sermoneándola, sino haciéndola. Ni él mismo se daba explicación del
punto de partida de este anhelo en su alma exaltada, ni del fin a que se
dirigía con fuerza más instintiva que voluntaria\ldots{} Pero él, cuando
en los camastros de \emph{Mazaltob} se reponía de sus caminatas
callejeras, pensaba: «¿No será vano el artista que predique los
principios de la escultura y no sepa labrar una estatua? ¡Ah!, no seré
yo ese artista estéril y baldío. A un lado las retóricas que enseñan
reglas infecundas, jamás comprendidas del oyente, y hagamos, aunque sea
en barro tosco, la estatua de la Paz.»

Estas ideas le rondaban la mente cuando fue visitado por \emph{El
Nasiry}, en quien, por la pureza del lenguaje, se le reveló un español
musulmanizado, y por las líneas y la expresión del rostro, el fugitivo
hermano de Lucila, que supo cambiar de religión, de patria y de
costumbres con flexibilidad inaudita. No podía Juan asegurar que el
arrogante moro que le visitó fuera Gonzalo Ansúrez; pero sus sospechas
vehementes casi tocaban en la certidumbre. Hablando de esto con
\emph{Mazaltob}, la maga le dijo que \emph{El Nasiry} era de la casta
árabe granadina, y que se distinguía por su nobleza y generosidad.
Hablaba español por haber vivido largas temporadas en Málaga y
Algeciras; no pensaba ella que fuese renegado, aunque algunos había en
Marruecos circuncisos en toda regla, y tan perfectos en su
transformación de lengua y costumbres, que el mismo ángel justiciante,
el día del Juicio Final, no sabría si ponerlos entre los \emph{moríos} o
entre los del \emph{Andalús}. Despertó esto más la curiosidad de Juan y
sus ganas de tratar a \emph{El Nasiry}, para echarle la sonda y ver si
en él se repetía el extraordinario ejemplo de \emph{Alí Bey El Abassi}.
Pero pasaban días, y el moro, disgustado por las diabluras proféticas de
\emph{Mazaltob}, no volvió a parecer por el \emph{Mellah}\ldots{} Siguió
en tanto el joven español haciendo conocimientos, y entre estos fue muy
interesante el del rabino \emph{Baruc Nehama}, varón provecto, de
relativa ilustración y de cierta templanza en su fanatismo, el cual,
creyéndole hombre desamparado y errante, y apreciando además su
peregrino talento, quiso atraerle al rebaño judaico. Mas a las primeras
insinuaciones vio el \emph{levita} que se las había con un cristiano
inexpugnable, y que su sermón catequista era como echar jarros de agua
en los arenales del desierto.

Fuerte en su doctrina y dotado de brillante palabra para exponerla,
Santiuste rebatía las opiniones del viejo \emph{Baruc} apenas salían de
su boca por entre las aborrascadas barbas, que le daban aspecto de
profeta bíblico. Y ante el reposo y serenidad del cristiano para
combatir la rancia doctrina, el hebreo se incomodaba, perdía el grave
continente, y sacaba, no digamos el Cristo, sino las tablas de la Ley,
como vicario del amigo Moisés en la tierra\ldots{} Pero estas
exaltaciones del sacerdote de Jehovah pasaban como nubecilla, y el
razonar manso de Santiuste llevaba la controversia al terreno
escolástico y de esgrima intelectual, descartada toda idea de
catequismo. Respetuoso con antagonista de tanto poder, \emph{Baruc} oía
el elocuente panegírico de la Fe Cristiana y de su prodigiosa difusión
en todo el mundo. Con algo que recordaba de su maestro Emilio Castelar,
y lo que él de su propia cosecha ponía, trazaba el poeta de la Paz
cuadros admirables ante los cuales el moderno Aarón permanecía
cejijunto, enredando sus amarillos dedos en la luenga barba. Por fin, no
sabía el Rabino cómo y por dónde meter una opinión entre el follaje
espléndido de la oratoria del joven \emph{Yahia}; se reconocía inferior,
aunque por dignidad de sus funciones sacerdotales y talmúdicas se
guardaba muy bien de dar a torcer su brazo. En él resplandecía el
orgullo de los que afectan poseer la única verdad, y antes mueren que
soltar el signo autoritario con que guían, custodian y apalean a su
dócil rebaño.

Hizo Santiuste la apología del Cristianismo en variedad de tonos,
descendiendo del sublime al patético; ensalzó la intensa ternura de la
predicación de Cristo, por la cual este penetró en las entrañas de la
Humanidad, conquistándola y haciéndola suya para siempre; marcó luego la
obra inmensa de los apóstoles, para afianzar la doctrina del Redentor
sobre las ruinas del Imperio, y la siguiente labor de los Padres para
fijar en dogmas inmutables todo el organismo de la Hermandad Cristiana;
describió la tenaz gestación de la Iglesia para formarse, para edificar
su imperio militante y docente, y sostenerlo con robusta trabazón
arquitectónica en el curso de los siglos. ¿Cuándo había visto la
Humanidad obra tan grande y sintética, ni organización tan poderosa? La
doctrina de Cristo había venido a ser la única normalidad espiritual de
los pueblos civilizados. Todo lo demás era fetichismo, o bien residuos
deshechos de una teogonía bárbara y sin calor. Declaró Santiuste con
emoción y solemnidad que de las confesiones cristianas, prefería la
católica, porque en ella había nacido y porque era la más bella, la más
latina, en el sentido etnográfico, y la que a su parecer responde mejor
a los fines humanos. Todo lo que la Iglesia Católica enseña con riguroso
método escolar a los pueblos sometidos a su espiritual magisterio, él lo
encontraba de perlas: en un solo punto disentía, y era la durísima
abstención que llamamos \emph{celibato eclesiástico}. He aquí el nudo
negro. Todo lo encontraba muy bien, menos el negro y apretado nudo.
Doctores tiene la Santa Madre Iglesia que deben poner mano en este
negocio, si no quieren que se les venga encima un cisma que será de los
más agitados y calientes que amenicen la Historia de las disensiones
religiosas. Y en este punto, declaraba tenazmente el poeta su intención
cismática, porque él sentía en sí un vigoroso temperamento sacerdotal:
amaba los interesantes ritos, la dulce comunión del alma con Dios, la
penitencia confesional, la propaganda evangélica; en fin, todo le placía
y encantaba. Pero al propio tiempo sentía irresistible atracción hacia
la bella mitad del género humano que Dios formó de una costilla de Adán;
hacia la que, acabadita de crear, embelleció con sus gracias el Paraíso
y todo el Universo.

Dijo esto el poeta con delicadeza exquisita; y como el Rabino le
indicase que el amor de mujer no está vedado a los sacerdotes en ninguna
de las religiones, fuera de la papista o católica, declaró Santiuste que
esta, siendo la mejor y casi la perfecta, aún tenía que dar el paso que
le faltaba para ser la misma perfección, celebrando eternas paces entre
la Fe y la Naturaleza. A esto contestó \emph{Baruc Nehama} sacando a
colación con cierto orgullo un texto litúrgico de su Ley, que dice: «Dio
gracioso y piadoso, luengo de iras y grande de mercedes, hartarme he de
ver tus faces\ldots{} Bendice simiente de hombres tuyos adorantes, y al
templo tráenos chiquitos de tu semejanza. Veamos crecer generancio tras
generancio\ldots» Quería decir esto que Dios bendice toda unión de mujer
y hombre conforme a su Ley, sin exceptuar los enlaces o casamientos de
sacerdotes. Agregó el venerable levita esta sagaz observación: «Si el
tener mujer los oficiantes del templo es bueno y saludable por los
bienes que produce, lo es más, pero mucho más, amigo Juan, por los males
que evita.»

Quiso Dios que estos paliques sabrosos sobre la compatibilidad de amor y
cleriguicio sirvieran de prefacio al encuentro de \emph{Juan el
Pacificador} y la bella \emph{Yohar}, hija de Riomesta. Acaeció este
notable suceso en la puerta misma de la casa rabínica, a la sazón que
entraban las dos hijas de \emph{Baruc} llamadas \emph{Rebeca} y
\emph{Alegría}, y con ellas la de Riomesta, cuya hermosura eclipsaba la
de las otras niñas, como apaga el sol el brillo de las estrellas. Quedó
Juan suspenso, y apenas la vio desaparecer tras de la puerta, no sin que
la moza echase a la calle una miradita, sintió en su interior un
tremendo vaivén, como el de un barco sobre las olas bravas, de lo que le
resultó un estado semejante al mareo, terror, ansiedad\ldots{} Tiró el
hombre hacia su domicilio, y encontrándose de manos a boca con la maga,
le dijo: «¿Quién es esa divinidad que ahora entraba en casa de señor
Rabino? Te aseguro que me ha deslumbrado, como estrella que bajada del
cielo anduviese por la tierra vestida de mujer. Bien se ve que es de tu
raza, por la blancura y fineza del rostro, y su aire de familia con
Esther, Betsabee y otras tales que ilustran vuestras historias.» Y
\emph{Mazaltob} le respondió: «Es \emph{Yohar}, hija de Riomesta, tan
rico él, que veinte camellos no podrían cargar todas sus \emph{patacas}.
Tanto como el padre es rico, es ella hermosa, y \emph{ainda} buena de su
natural, amorosa y cargada de virtudes blandas, y con habla de sonido
dulce que se te apega en el alma. Aplícate a ella, Yahia, que no podrían
encontrar mejor apaño tus partes buenas. Si ella es polida, tú barragán,
y \emph{ainda} sabidor mucho. Háblale como tú sabes, con todo el
melindre de tu suavidad, y verás cómo te responde con sonriso\ldots{} No
temas, y la tendrás enternerada, y \emph{aina} serás camello que cargue
a un tiempo la mayor riqueza y la mayor hermosura del \emph{Mellah}.»

Aunque lo de ser camello no fue muy del agrado de Santiuste, abrió sus
oídos a las palabras de \emph{Mazaltob} para que las ideas le entrasen
holgadamente en la cabeza. Sintiose cautivado de las gracias de
\emph{Yohar}, sin que la riqueza fuese en él estímulo de su inclinación,
pues era hombre absolutamente desinteresado y sin ningún apego a los
bienes materiales. Tratando con su patrona del cómo y cuándo de
aproximarse a la \emph{Perla}, se le propuso que podían celebrar sus
vistas en casa de \emph{Simi}, la destiladora, pues esta tenía
parentesco con los Riomesta por parte de madre. A menudo la visitaba
\emph{Yohar} por el atractivo de los perfumes, a que era muy aficionada.
Su padre, confiado y bondadoso, seguro de la virtud de la bella moza, no
la celaba con impertinencia, ni le ponía estorbos para que fuese sola a
las viviendas próximas de parientes o amigos.

Pues, Señor, he aquí que al día siguiente de ser Juan deslumbrado por la
blancura de la hija de Riomesta, la vio de cerca, la tuvo al alcance de
su voz, y mismamente de sus manos, en el taller o laboratorio donde
\emph{Simi} extraía las delicadas esencias de rosas y jazmines. Y Juan
habló con palabra turbada: «Yo bien sé, amable \emph{Perla}, que no soy
digno de llegar a tu hermosura y bondad, prendas excelsas en que se
esmeró el Criador de cuanto existe. Pero los hombres ambiciosos miran a
lo que no pueden alcanzar, y solicitan lo que no merecen. Yo soy de
esos, \emph{Yohar}; ambicioso que no se sacia con nada pequeño, ni con
bienes de la tierra; busco y pido los del cielo, que en ti están
cifrados. Niégame el amor que te pido, porque así ha de ser, siendo tú
tan perfecta y yo tan miserable\ldots{} Niégamelo y despídeme, que con
ser despreciado por ti me contento, si el desprecio trae en sí un poco
de misericordia.»

Y ella: «Tírate atrás, \emph{Yahia} o Juan, y no me encariñes el oído.
Ya sé que eres decidor fino, y que con tus decires graciosos y mielosos
envoluntas a una piedra. Pero conmigo no te vale tu virtud, que so de
nieve como ves\ldots{} Ya ves cómo me río\ldots{} cómo me río de ti,
\emph{Yahia}.» La risa de la linda moza cayó en los oídos del poeta como
lluvia de perlas sobre cristal\ldots{} Esto pensaba; pero al punto
rehízo la imagen, diciéndose que el mismo ruidillo gracioso sobre el
cristal podía ser producido por garbanzos o granos de maíz.

\hypertarget{ii-3}{%
\section{II}\label{ii-3}}

Y él: «Bendiga Dios el instante en que te vieron mis ojos. Deslumbrado
fui; obscuridad triste llenó toda la tierra cuando desapareciste\ldots{}
Lloré yo mi miseria y escondí mi rostro, creyendo que para mí había
concluido el reino de la luz. Ahora te veo, y mi alma se llena de
gratitud, pues con mirarme sólo has tenido toda la piedad que como
criatura de Dios merezco\ldots{} ¿Qué más puedo desear después de verte?
Sólo verte otra vez es mi deseo, y si no te enojaras, te pediría que me
dejases gozar de tu presencia y de tu voz, aunque ninguna esperanza
dieras a mi admiración de ti. Eres como divinidad a quien se debe todo
acatamiento, y un culto que no puede ser callado, pues la voz se dispara
sola en tu alabanza.»

Y dijo \emph{Yohar} risueña: «Cállate ya, embustero gracioso\ldots{} que
por querer ser fino demasiado en el requerimiento, echas flores de
trapo, sin olor. Exprime tu corazón con verdad y sin tanto requilorio, y
ansí te entenderé\ldots{} Para decirme que so mujer bella y que penas
por mí, no hay precisión de tanta cuenta de palabras vacías\ldots{} Y no
me hables de tu miseria, que es mentirosa, pues sé que vienes aquí con
fingimiento de omildad, y que con ropas puercas tapas tu señorío de
príncipe cristiano. Tu cara dice que de padres altos naciste, y tu
lenguaraje suena con lustración, que yo no entiendo, porque so
inorante\ldots{} ¡Ay, \emph{Yahia}, qué bestia bonica verías en mí si me
trataras despacio!»

---Si eres joya sin pulimento, más me agradas así. ¿Quieres que este
pobre maestro te instruya, y adorne con luces de saber humano el divino
entendimiento que posees?

---Sí que deseo polirme, y ser menos bruta de lo que so, que aquí en
nuestras partes de Marroco no ha escuelas ande deprender cosas muchas y
finas de lustración de Espania, Viena o la Rumanía.

---¿Quieres que proponga a tu padre tomarme de maestro tuyo? ¿Crees que
pondrá en mí su confianza?

---No: antes ha de poner mi padre un garrote en tus costillas, y
quitarme a mí de que te hable y oiga tus loores graciosos.

---Pues véate yo sin conocimiento de tu padre, y te instruiré, que en
ello no ha de haber malicia, \emph{Yohar}.

---Ni malicia ni perjudicio, sino ganicas mías de ver, de catar
sabiduría. Creime, Juan, que es dolor de una mujer verse inorante y
abrutada de tantas cosas.

Diciendo esto, y sin esperar la réplica de Juan, dio media vuelta con
graciosa rapidez, arremangándose la túnica holgadísima de paño azul que
vestía. Los despojos de hierbas, y el polvo y ceniza que invadían el
suelo del laboratorio, exigieron el remango airoso de la guapa hembra,
la cual sin querer descubrió por un instante hasta media pantorrilla.
Fue \emph{Yohar} hacia la mesa o mostrador en que Simi filtraba y
trasegaba líquidos, y cogiendo un frasco chiquito que casi no se veía
entre sus blancos dedos, volvió junto al profeta, y le acercó el frasco
a la nariz, diciendo: «Confiésame tú que nunca has golido desencia tan
primorosa como esta. Es de una hierba silvestrina que aquí llamamos
\emph{enchíchoru}, la más prefumosa de los montes, y la que más halaga
el sentido. Güele más, y hártate de este olor que es el mío. En tu
camisa échate gotas, y golerás lo mesmo que yo.»

Dejose el poeta embriagar de aquella fragancia, que se sobrepuso a los
demás olores difundidos en el aire espeso del laboratorio. Tanto aroma
fuerte le desvanecía, y su cerebro se adormeció en vagas sensaciones.
Bellas cosas quiso decir después de perfumarse, como su ídolo le
mandaba; pero ella no le dio tiempo a soltar las alambicadas retóricas.
«Adiós, mi señor---le dijo mirándole los ojos.---Ya no más plática hoy.
Quédate con la paz, Juan.» Y él: «¿No veré mañana la luz de mi vida?»

---La verás, para que estés diluminado, que en el obscuro podrías
trompicar y caerte\ldots{}

---Si me engañas, \emph{Yohar}; si no te veo mañana, al otro día
encontrarás muerto al que quiere ser tu preceptor.

---No hagas malas mientes de mí---replicó la hebrea arremangándose por
detrás para salir, pero sin mostrar más que los blancos tobillos, y los
pies en babuchas rojas.---Antes mancarás tú que yo\ldots{} La primera
lición que me des será de los modos de hablar bonicos\ldots{} So la
bestia de Dios\ldots{} Como me criaron, ansí me ves, sin ningún
perfilorio\ldots{} Adiós, Juan\ldots{} No me acompañes, ni me sigas con
alocamientos. Puede que haiga genterío en la calle. Quitemos razón a los
malos pensares.

Trastornado quedó el profeta de la Paz con la gallardía estatuaria, la
gracia inocente y bíblica de la hija de Riomesta. Nunca vio mujer que
pudiera igualársele. ¿Qué comparación tenían con \emph{Yohar} ni Teresa,
ni Lucila, ni tantas otras bellezas de allá, embutidas en feísimos
trajes negros o pardos, y hablando un lenguaje de hipócrita corrección?
\emph{Yohar} era la mujer oriental o asiática, la Reina de Sabá,
Semíramis, Herodías, María de Magdala, y ¿por qué no la mismísima Eva
con la menor cantidad de ropa? Después de amar a \emph{Yohar}, podía un
hombre morirse tranquilo, llevándose a la eternidad los dejos de
inefable ventura\ldots{} Se enamoró y \emph{envoluntó} con el fuego de
todas las hornillas de amor encendidas por la juventud y sopladas por
los poetas.

La imagen de \emph{Yohar}, tal como en la oficina de perfumes la vio
Juan, por instantes se le reproducía en el pensamiento con ilusión
perfecta de realidad; por instantes se le borraba, no quedando de ella
ni siquiera una vana sombra, y esta privación de la imagen le
exasperaba: sin necesidad de conjuro, de improviso volvía la imagen
hechicera\ldots{} Declaraba el poeta que no existía debajo del Sol
rostro como el de \emph{Yohar}, tan bello de frente como de perfil,
blanco, amoroso, con resplandor de ternura sentimental, y de gracias
veladas aún por la timidez. Los ojos rasgados, dormilones cuando la moza
permanecía en silencio, echaban y recogían raudales de luz cuando
hablaba. La boca, sin soltar una sílaba, expresaba tanto como los ojos.
Los ojos, mirando, no hablaban menos que la boca\ldots{} ¿Y qué decir de
la negrura del pelo, que en dos ondas asomaba tan sólo por la frente;
qué de aquel pañizuelo de colorines liado en la cabeza con arte
exquisito, formando por delante como el pico de una montera, y atrás un
bulto que envolvía la madeja liada del abundante cabello? Sobre sus
orejas, no pendientes de ellas, sino suspensos del pañuelo por un gancho
casi invisible, colgaban dos aros de oro como de cuatro pulgadas de
diámetro. Nunca vio Santiuste adorno tan bonito, ni tan oriental, ni tan
acomodado a la belleza de Judith o de Dalila. ¡Y qué manos finas,
vigorosas! Aquellas manos pudieron cortarle los cabellos a Sansón o
separar del tronco la negra cabezota de Holofernes.

El cuerpo, descrito vagamente por los pliegues del túnico, y por lo que
de él contaban las extremidades, o las muestras que de estas se veían,
no exaltó menos que la cabeza el entusiasmo y la admiración de Juan. ¿A
dónde iban a parar los cuerpos de europeas con la falaz anatomía que dan
los corsés, y el andar corto y medido, sin el meneo de faldas de la
mujer de Oriente?\ldots{} En fin, señalando y ponderando bellezas, el
profeta no acababa\ldots{} \emph{Mazaltob}, que siempre le oía con gusto
por la riqueza y buen son del habla, se burló de él aquella noche
mientras le servía la cena, y riéndose le dijo: «Bien garrida es
\emph{Yohar}, por merced del alto Criador\ldots{} pero más, más\ldots{}
oye de mí\ldots{} más que su blancura valen las arcas pretas del padre
de ella, hombre apañador\ldots{} ¡Goy, no desmayes, ni te acortes en el
pedir cuando tengas a la moza bien sobajada de amor y endulzada de tu
querer, clamando por boda!\ldots{} Ansí te vea yo padre de cien
chiquitos como he de verte rico y holgado de dinerales, si haces lo que
te digo\ldots» No tenía traza de parar en esta cantinela; pero Santiuste
le cortó la palabra, pues su corazón noble y recto no sentía jamás
inquietud por cosa tocante al oro y la plata, ni dejaría de prendarse
locamente de la incomparable \emph{Perla} si fuese huérfana y pobre.

La segunda entrevista fue más breve que la primera. Mas la tercera
superó en interés y extensión a las dos anteriores. Llevó aquel día la
israelita medias de seda, como tributo a la civilización de Europa, y
otra túnica azul con una franja delantera y vertical bordada de oro. Por
el descote y mangas asomaban encajes. Era un vestido caprichoso,
bastardeando un poco la usanza, con lo que quería significar su gusto de
la iniciativa y de la variación, como sintiendo los desconocidos
encantos de la moda. Y dijo \emph{Yohar}: «He soñado contigo,
Juanito\ldots{} Érades tú un hermoso caballo español negro\ldots{} yo
una mulita blanquita. Venías a mí con relincho gracioso trotando, y yo
te tiraba coces\ldots{} No te rías, que ansí lo soñé. Dirás que so
bruta, muy bruta, y que ni en sueños puedo quitar de mí la condición de
animala sin sabidoría\ldots»

---Eres encantadora, y tu inocencia vale más que todas las ciencias del
mundo. En mi corazón has pegado tus coces divinas, que me destrozan el
alma.

---Dime otra vez que si no te quiero te morirás de muerte amorosa, que
es lo que más adentro del alma me allega para quererte\ldots{} No sé si
me has entendido, porque no tengo el habla tuya, como diamante tallado
que echa luces.

---Sí que me moriré, porque mi vida no sabe ya vivir sola, y es llama
que necesita arder en ti\ldots{} Si no, se apaga. Tú eres el haz seco
que ansía mi llama\ldots{}

Y con esto Juan le echó los brazos, como para sellar juramento de
próxima unión ante los altares, sin cuidarse de qué altares serían, o
creyendo tal vez que para el caso todos los altares eran lo mismo. Sin
hacer gran violencia para desprenderse, \emph{Yohar} cumplió con lo que
el pudor y la decencia le dictaban; lo demás lo hizo la delicadeza de
Santiuste. Y ella dijo con seriedad: «No nos aloquemos, y seyamos
conocientes del mandato de Dio\ldots{} Quietas manos, y los ojos con
virtú; hagamos promisión de ser juntos siempre, y luego pensaremos en
las procuras para casarnos con ley.»

Y él: «Valor de compromiso solemne doy a todo lo que digo, \emph{Yohar}.
Serás mía, y yo tuyo en este mundo visible y en el otro.»

Y ella, con emoción mística: «Oíd, Cielos y Tierra, porque Adonai
habló\ldots{} Conoció buey su comprador, y asno pesebre de su dueño.»
Con estas palabras rituales que pronunció al modo de juramento, y que en
los oídos de \emph{Yahia} sonaron como la más inspirada fórmula poética
que pudiera imaginarse, expresó la israelita su propósito de pertenecer
al español en cuerpo y alma. Y dejándose besar las manos, y algo de lo
que asomaba de sus torneados brazos, completó así la idea: «¡Comprador
mío, dueño mío!\ldots{} Pesebre nuestro tengamos pronto para siempre.»

Toda hipocresía y remilgos, acudió \emph{Simi}, que presente estaba, a
interrumpir un coloquio amenizado con aproximaciones, en las cuales
creía ver grave riesgo de la honestidad. Dijo el profeta: «No hemos
hecho más que jurar, \emph{Simi}.» Y \emph{Yohar}: «Tírate allá,
pringosa entremetida, que no hemos rompido ningún vaso, ni vaso nuestro,
ni del decorío de tu casa. Virtú tenemos, delantre cielo y tierra.»

No hay que decir que volvieron a verse al siguiente día, y a ratificar
su juramento con expresiones ardorosas, y con todos los gestos y mímica
que tan dulce intimidad requería, sin que la presencia de \emph{Simi}
viniese a turbarles. ¡Oh, \emph{Yahia}, profeta gracioso y venturoso!
Tus empresas de paz dejarán memoria entre los humanos, por lo atrevidas
y eficaces: tú domas el fanatismo, aproximas las razas enemistadas, y
pides para todos los pueblos la bendición del Sumo Dios Único\ldots{}
Fue dichoso Santiuste, y su felicidad le tuvo día y noche como en
éxtasis, viendo en su pesebre a la que reunía todas las gracias de Eva
nuestra madre. Por bien empleadas dio sus fatigas desde que se lanzó al
trajín de la guerra. En su viaje al África vio la inspiración del Cielo,
o \emph{el dedo de Dios}, como dicen los historiadores y los políticos
cuando quieren dar calidad de cosa divina a sus majaderías pomposas.
Obediente también al dedo de Dios, que le sañalaba la puerta de su casa,
abandonó \emph{Yohar} el hogar paterno (llevándose alhajas, algún
dinerito suyo, y no llaves, como Riomesta decía en sus imprecaciones
lastimeras), para seguir a Juan hasta el fin del mundo: en tal ceguera
de amor la puso el poeta con su labia fogosa y el buen gancho que tenía
para enamorar. Fue la primera idea de los amantes huir de Tetuán; mas
olfateando el peligro, se acogieron al parador llamado \emph{el fondak}.
De allí escaparon más de prisa, por estar lleno el local de montañeses
desalmados y de parásitos feroces; vagaron por calles y pasadizos hasta
que el borriquero Esdras, a quien \emph{Yohar} mantuvo a su servicio
recompensándole con largueza, les deparó albergue en el tenducho
miserable de un zapatero remendón, que había escapado de la ciudad. La
pobreza y el desaseo de aquellas viviendas no abatió el espíritu de los
amantes, ni enfrió la juvenil pasión que a entrambos inflamaba. Eran
felices, y sus almas serenas flotaban sobre tanta inmundicia sin
contaminarse de ella, como la luz que pasa por los aires infectos sin
obscurecerse ni ensuciarse.

Llegó el 4 de Febrero. En la siniestra noche que siguió al desastre,
pasaron los amantes horrible susto, viéndose en peligro de ser
cruelmente asesinados. Dios, Allah y Adonai juntos defendieron las
preciosas vidas de los que por ley de amor eran predilectos de la
divinidad. Esdras les puso en comunicación con \emph{Simi}; esta, en la
mañana del domingo, les contó los horrores acaecidos en el
\emph{Mellah}, atropellos, incendios, muertes, y por fin el terrible
caso de \emph{Mazaltob}, que por milagro de Dios y mediación de \emph{El
Nasiry} no pereció a manos de los bandidos\ldots{} Salidos los amantes
de su escondite por indicación de \emph{Simi}, se fueron a un almacén
ruinoso de la calle \emph{Caid Hamed}, donde ya estaba escondida la
hechicera, y allí esta sagaz mujer, asistida de los poderes infernales,
concibió el magno proyecto de buscar refugio en la próxima casa de
\emph{El Nasiry}\ldots{} De la idea pasaron a la ejecución, conforme
entró la noche del 5 al 6, y tan admirables disposiciones estratégicas y
tácticas dio la maga para el atrevidísimo acto, que un éxito brillante
coronó la sutileza de ella y la prontitud de todos.

Cuentan los que lo vieron que en la mañanita del 6 salió Juan de su
nuevo alojamiento con el airoso traje que encontró en los roperos de
\emph{El Nasiry}, y recorrió el centro de la ciudad, informándose de lo
que había pasado durante la noche. El aspecto de las calles y el cariz
de la gente que en ellas veía le afianzó en su idea de la fácil entrada
del ejército vencedor. En \emph{Garsa Es-seguira}, vio muchos hombres
que disputaban en alta voz, señal de que no había unidad en los
pareceres, y sin unidad la resistencia era imposible. Unos corrían
después hacia la puerta de Fez, otros hacia las del lado Este; no vio
tipos de militar fiereza, sino figuras demacradas, famélicas, con la
insana movilidad de quien no sabe lo que quiere ni a dónde va. Pasó
luego por la calle \emph{Emtamar} donde habitaba un gaditano con quien
había hecho conocimiento. Deseaba por su mediación ponerse al habla con
Riomesta, pues de este y del Rabino era grande amigo el tal andaluz, que
fue a Tetuán de barbero y luego puso comercio de ferretería y loza
ordinaria. Halló Santiuste la casa y tienda cerradas a piedra y barro, y
allí se detuvo un momento dudando qué dirección tomar. En esto sintió
voces de tumulto, y vio correr la gente en dirección de la gran
Mezquita. La curiosidad le llevó hacia allá\ldots{} Siguió luego por
calles que conducían a una de las puertas de la ciudad\ldots{} ignoraba
cuál de las puertas era. Oyó que por allí entrarían o querrían entrar
los españoles, y esto le empujó más por aquel camino. Al desembocar en
una encrucijada irregular, llena de basuras y escombros, formada por
casuchas de una parte, de otra por ruinas, vio que unos montañeses
atropellaban a dos pobres hebreos ancianos y a las mujeres de la misma
raza que salieron a su defensa. Un moro de buen porte y calidad, a
juzgar por su vestimenta, corrió al socorro de los débiles. Pronto se le
unió en la caballeresca acción otro señor bien vestido. Santiuste, que
con su prestado traje se tenía por tan principal como el primero, acudió
a reforzar a los caballeros. En un santiamén quedaron estos vencedores,
y dispersos los desalmados\ldots{} Dio algunos pasos Juan, atraído de un
rumor de cornetas que del campo venía\ldots{} Llegó a la vista de los
baluartes que franquean la puerta de la ciudad; vio que al lado suyo,
tocándole casi, iba uno de los bravos personajes moros que medio minuto
antes habían cerrado contra la canalla. Paráronse ambos, se miraron, y
el profeta \emph{Yahia} se encontró frente a la gallarda figura de El
\emph{Nasiry}.

\hypertarget{iii-3}{%
\section{III}\label{iii-3}}

No hizo Santiuste por evitar la mirada del moro, ni menos trató de
escabullirse y poner pies en polvorosa; antes bien afrontó gustoso la
presencia de aquel sujeto y se fue a él con donaire y confianza. «Yo soy
Juan---le dijo,---no \emph{Yahia}, como tú me llamas;» y de esta sola
frase surgió una larga conversación. Ráfagas de cólera, ráfagas de
benevolencia notó el poeta en la cara del moro y en su lenguaje de
perfecta entonación castellana. Lo que hablaron se perdió en el bullicio
del pueblo que les rodeaba y en el rumor de cornetas que del campo
venía. No se maravilló poco Santiuste de ver que el arrogante moro
palidecía, que sus miradas inquietas se volvían de la tierra al cielo y
del cielo a la tierra, y que de su pecho arrojaba suspiros, en los
cuales iba envuelto el sonido de alguna palabra ininteligible. Sin duda
sufría grave trastorno moral y físico, enfermedad del cuerpo, o profunda
turbación del ánimo. El griterío de dentro de la plaza y el ruido
militar de fuera crecían. Entre ambos rumores la puerta permanecía
cerrada. ¿Se abría o no se abría la puerta?

En el sitio donde estaban Juan y \emph{El Nasiry} no se veía la puerta,
y sí el torcido callejón que a ella conduce. Junto a ellos, entre las
ruinas y un paredón interior de fortaleza, vieron la escalera de
gastados peldaños, por donde subían y bajaban \emph{moríos} de mal
pelaje que pretendían ocupar el reducto defensor de la puerta, artillada
con dos cañones de figurón\ldots{} Sin verlo, bien se comprendía que los
españoles habían llegado a la puerta, y encontrándola cerrada amenazaban
con abrirla de par en par a cañonazos. El altercado entre los cristianos
de fuera y los muslimes que por las troneras del reducto asomaban sus
famélicos rostros, se oía desde dentro. No teniendo entereza para
resistir ni para franquear gallardamente la entrada, los de arriba
dijeron: «No podemos abrir\ldots{} El Kaid se llevó las llaves.» Siguió
a esto un estruendo de vigorosos golpes dados en la puerta.

España colérica gritaba: «Abrid, miserables, o pegaré fuego a la
ciudad.» Con enormes piedras y con las culatas de los fusiles, los
españoles cascaban las herradas maderas\ldots{} Vieron entonces Juan y
su acompañante que del reducto bajaban despavoridos los bergantes que
allí hacían un vil simulacro de defensa. Al verlos huir, \emph{El
Nasiry}, sin abandonar su actitud de abatimiento les dijo: «La voluntad
de Allah sea cumplida\ldots» En el mismo instante, la caterva de judíos
y de moros pobres se lanzó por el callejón que conduce al interior de la
puerta, y ayudó con piedras a romper lo que los españoles querían romper
desde fuera. La \emph{Blanca Paloma}, la virginal doncella \emph{Ojos de
Manantiales} quedó pronto a merced de su conquistador\ldots{} Tras un
silencio de estupefacción, estalló bajo la bóveda de la puerta, como un
trueno subterráneo, la marcha real española. Todo aquel viejo armatoste
arquitectónico se estremeció, dando piedra con piedra\ldots{} Los que
tocaban la marcha permanecieron un instante quietos; luego se vieron las
bayonetas, los fusiles, los hombres que entraban con paso grave\ldots{}
\emph{El Nasiry}, en el paroxismo de su terror, cogió del brazo a Juan y
lo llevó por un callejón que desde la puerta se empinaba entre casuchas
gibosas. «No puedo ver esto---le dijo.---Vámonos\ldots{} escondámonos.»
Y \emph{Yahia}: «Déjame, señor, que les vea. Son mis amigos\ldots{} Ya
entran\ldots{} avanzan ya con paso ligero. Mira cómo les aclama la
multitud. Entran con respeto, como hombres de buena educación que
delicadamente se acercan a la desposada y le quitan los velos\ldots{} Al
frente viene el General Ríos\ldots{} también Mackenna\ldots» Estirando
toda su estatura para echar una mirada por encima de las cabezas de la
multitud, dijo \emph{El Nasiry}: «Viene con ellos \emph{El Gazel}, para
enseñarles los caminos y guiarles por las calles\ldots{} Vámonos,
\emph{Yahia}; yo no debo ver esto.»

Avanzaron algo más callejón arriba. En una rinconada donde asomaban, por
entre construcciones humildes, algunas peñas del cerro en cuya cúspide
está la Alcazaba, \emph{El Nasiry} no pudo ya mantener en tensión las
fuerzas del alma que sostenían su disimulo. Dejando correr un raudal de
lágrimas, sin cubrirse el rostro ni alterar su voz plañidera, habló de
este modo: «La turbación que siento es de las que pueden matarle a uno
si se descuida\ldots{} Asístame Dios\ldots{} Pues adivinaste tú quién
soy, poco será lo que yo tenga que decirte\ldots{} Esas músicas, esa
gente que entra en Tetuán con alegría de victoria, no me dicen cosas
olvidadas. Lo que veo y lo que oigo es mío, tan mío como mi propio
aliento\ldots{} No digas a nadie lo que has visto en mí, ni repitas mis
palabras. Yo debo alejarme de esta pompa y fingir que me entristece lo
que me regocija\ldots{} Tengo aquí un nombre, tengo una posición, tengo
un estado, que gané a fuerza de trabajo y de astucia inteligente. No
puedo renegar de mi estado, \emph{Yahia}; no puedo arrojarlo a la calle
por un melindre de patriotismo\ldots{} Guárdame el secreto, y
adelante\ldots{} Sigamos, observemos y disimulemos. El traje que vistes
te obliga, como a mí, a ser cauto y prudente.»

Desde el sitio en que se hallaban, vieron que entraba el raudal de
tropas; los haces de bayonetas brillaban al revolver de la marcha en las
angostas calles; el color pardo de los ponchos se iba extendiendo y
llenando calles y plazuelas, como sangre inyectada en las venas vacías
de la ciudad. La virginal \emph{Ojos de Manantiales} estaba ya hinchada
de españoles, y pletórica de aquel rico elemento vital que se difundía
por todo su cuerpo\ldots{} Las azoteas, coronadas de gente, coronaban
también de vagas aclamaciones el estruendo de las músicas que invadían
las calles\ldots{} «Acerquémonos ahora---dijo El Nasiry,---y veamos si
entra también O'Donnell.» No por donde habían subido, sino por otro
callejón que iba a desembocar a la plazuela llamada \emph{Garsa El
Kibira}, fueron ambos a satisfacer la curiosidad y la emoción, el
insaciable sentimiento que nunca se hartaba. A distancia, por un largo y
recto pasadizo cubierto, que era como anteojo, vieron pasar soldados,
recorriendo una vía de relativa anchura. Así estuvieron mediano rato:
«Mira, mira---gritó de improviso Santiuste:---ese que ahora pasa es
O'Donnell\ldots{} Ya pasó, ya no lo ves\ldots» «Le vi---replicó \emph{El
Nasiry},---y le conocí por su grandeza, que a mi parecer superaba a la
de las casas.» Detrás del General en Jefe siguieron entrando secciones
de todos los Cuerpos con sus músicas correspondientes, las cuales
tocaban la marcha de la ópera \emph{Macbeth}, muy del gusto de O'Donnell
por su marcial aliento.

«En el corazón---dijo \emph{El Nasiry} retrocediendo con su amigo,---se
me queda pegada esa música, y creo que la estaré oyendo mientas
viva\ldots» Empujada la puerta más próxima, penetró en una casa de
apariencia humilde. Era una de las tres de su propiedad que alquiladas
tenía. El pobre viejo que moraba en ella, almuédano a sus horas, a ratos
escribiente de un \emph{Kadí}, había salido a ver las tropas. En el
patio, una mora vieja y demacrada recibió al casero: este y su
acompañante, descansando en un poyo revestido de azulejos, continuaron
su interesante coloquio. Reiteró \emph{El Nasiry} a Santiuste la
recomendación de guardar secreto sobre cuanto le dijese, movido del
irresistible impulso de abrir su pecho, en tan grave ocasión, a un
individuo de su raza y de su tierra. A las innumerables preguntas que
hizo acerca de España y de la familia de Ansúrez, pidiendo detalladas
noticias de su padre y hermanos, contestó Juan con interés minucioso,
apurando su memoria para que nada se le quedase por decir. Con esto
acabó el buen \emph{Yahia} de ganar la confianza del que tenía por
poderoso señor musulmán, o renegado de alta escuela, al estilo de
\emph{Alí Bey}\ldots{} De veras admiró Juan el prodigio de una
metamorfosis bastante perfecta para cautivar en confiada ilusión a todo
un pueblo.

Ponderó \emph{El Nasiry} las ventajas de vivir en Marruecos en calidad
de moro, disfrazándose para ello de lenguaje, de costumbres y de
religión, y ensalzó el beneficio grande que resulta de existir allí muy
pocas leyes, simplificación legislativa que compensaba el bárbaro
despotismo del Sultán. Este no era tan intolerable para el hombre
flexible y astuto que supiera adaptarse al suelo, y hacer sus pulmones
al ambiente de un país sin gobierno excesivo, tiranía ciega y
caprichosa. Era cuestión de marrullería, de estudio de los hombres y de
conocimiento de la fundamental ciencia del Mogreb, que es la Gramática
Parda. Él había estudiado más que cien bachilleres de Salamanca para
llegar a la cabal asimilación del Islamismo por el lado religioso, por
el civil y moral, y podía decir, aparte toda modestia, que pocos picaron
tan alto en la sutileza de la conquista. «La llamo
así---prosiguió,---porque conquista personal es lo que yo he realizado,
y no hay otra manera de penetrar en esta salvaje familia. Los españoles
no imitarán en conjunto mi obra, y por no imitarme, no serán nunca
dueños de Marruecos, a pesar de estas guerras y de estas batallitas
vistosas\ldots{} sí, muy vistosas y con música, hijo mío, pero nada
más\ldots{} Y por fin, si tu intención es quedarte aquí, tómame por
maestro, y no des un paso ni respires sin consultarme previamente.
Prepárate a una labor dura, y trae a tu entendimiento todas las luces
que andan por esos mundos, y alguna más que tú inventes, pues la
sabiduría y picardía labradas por los demás no son bastantes, y hacen
falta picardía y saber nuevos que cada cual debe sacar de donde pueda.»

Tocole después a Santiuste explicar el rapto de \emph{Yohar}, y en
verdad que lo hizo con perfecta honradez histórica, refiriendo los
antecedentes del caso y el caso mismo sin jactancia ni floreos
sentimentales. Frunció el ceño \emph{El Nasiry} a la conclusión de la
historia, y dijo: «Bien, \emph{Yahia}: empuje grande de ilusión hubo,
según veo, por una parte y otra, y no mediaron más que los engaños
propios de amor. Ordena la Naturaleza que se le rinda homenaje, y no hay
forma de desobedecerla\ldots{} Es una tirana que manda en la
juventud\ldots{} ¡Como que ella es siempre joven, y está engendrando sin
cesar!\ldots{} Bien, hijo: lo que no me parece acertado es tu pretensión
de que \emph{Yohar} abrace el Cristianismo. Si logras catequizarla,
despídete de las riquezas de su padre, que son cuantiosas, hijo. Conozco
a Riomesta; sé que no sólo es el más rico, sino el primer rezador del
\emph{Mellah}, apegado fanáticamente a su Ley rancia y a los ritos
hebraicos. No, no cederá\ldots{} Tienes que largarte a España con la
moza, si es que quiere seguirte\ldots{} Hoy, como está enamorada, te
dirá que sí, que será cristiana, que quiere el agua del bautismo\ldots{}
Pero no te fíes, hijo, no te fíes, ni creas que esas lindas coces de
\emph{Yohar} que me has contado han de ser siempre blandas y
amorosas\ldots{} Ya coceará de otro modo\ldots{} Deja que se enfríe un
poco el amor, pues no hay cosa caliente que el tiempo no enfríe, y verás
cómo la borrica tira al pesebre paterno\ldots{} Dime otra cosa: ¿tienes
tú con qué mantenerla?, ¿piensas que se resignará a la pobreza?
\emph{Yohar} gusta de los ricos vestidos, de las joyas\ldots{} Sin duda
esa víbora de \emph{Mazaltob} le ha hecho creer que eres tú algún
magnate disfrazado de pobre\ldots{} Sigue mi consejo: haz paces con
Riomesta; pídele su borriquita blanca; dile, o hazle creer, que por
poseerla en forma de ley entrarás por el aro judiego y te hincarás
delante de Adonai.»

Como Santiuste declarara enérgicamente que no haría jamás abjuración
verdadera ni fingida de su fe cristiana, \emph{El Nasiry}, luengo de
marrullería, astuto y nada corto de explicaderas, le dio palmadas en el
hombro diciéndole: «Hijo, vete pronto a España, vete a cualquier país
civilizado, que en África no tienes más carrera que la de mendigo si no
estudias todas las artes del fingimiento. El cristiano que acá venga y
no sepa fingir, o muere o tiene que salir pitando. Se hace aquí fortuna
más o menos grande según el grado de simulación que cada uno se traiga
para poder vivir entre esta plebe\ldots{} En mí tienes ejemplo vivo del
arte de figurar lo que no es\ldots{} Después de tanto tiempo y de
aprendizaje tan largo, ya vencedor en la lucha, todavía me veo precisado
a representar más papeles, según las ocasiones que se van
presentando\ldots{} Y para que lo comprendas mejor, te pondré un ejemplo
mío, un ejemplo reciente, de estos días, de hoy\ldots{} Verás,
\emph{Yahia}\ldots{} atiende un poco.»

Limpió su gaznate \emph{El Nasiry} con ligeras toses, y bien preparado
de ideas y razones, prosiguió así: «Tengo yo un amigo llamado \emph{El
Zebdy}, residente en Fez, buen hombre, intachable musulmán, rezador y
creyente a macha-martillo, rico y de no escasa influencia cerca del
Sultán. Su bondad y humanidad no tienen más límite que la línea del
fanatismo; cuando traspasa esta línea, es \emph{El Zebdy} tan bárbaro y
cruel como cualquier otro de su raza, y aún más que tantos y tantos que
se ven por ahí. Pues bien: este amigo me suplicó que le contara por
escrito todas las ocurrencias de la guerra, desde la llegada de los
españoles al valle del Río Martín, hasta que quedaran deshechos ante los
muros de Tetuán\ldots{} No era de mi gusto escribir historias; pero no
podía negarme a la pretensión de \emph{El Zebdy}, porque este señor me
ha protegido con largueza; me salvó una vez la vida; por él tengo aún
esta mi cabeza sobre los hombros; me ha dado dinero y crédito para mis
negocios; consiguió que el Sultán me cediera gratis el terreno donde he
construido tres casas; y más, más favores le debo. ¿Qué podía yo hacer,
Juan? Ponte en mi lugar. Pues Señor\ldots{} agarro mi pluma y ¡zas!:
todas las acciones se las he contado, y sólo me falta la de Tetuán y las
trapisondas en la ciudad, tarea que tengo dispuesta para esta tarde, si
Dios me da tranquilidad y tiempo\ldots»

---Linda historia será---dijo Santiuste,---escrita sobre el terreno,
interpretando la realidad honradamente.

---Quítate allá. ¿Crees tú que es historia lo que escribo para \emph{El
Zebdy}? No, hijo, no es nada de eso, porque he tenido que escribirlo al
gusto musulmán, retorciendo los hechos para que siempre resulten
favorables a los \emph{moríos}. Y cuando no me ha sido posible
desfigurar el rostro de la verdad, hele puesto mil mentirosos adornos y
afeites para que no lo conozca ni la madre que lo parió. En cada párrafo
he metido exclamaciones del \emph{Korán} y gran porción de esas
pamplinas con que aquí se alimenta el fanatismo. Allah y la variedad
infinita de sus nombres no se me caían de la pluma. Así queda el amigo
muy contento y al leer dice: «¡Qué buen creyente es \emph{El Nasiry}!
¡El Benigno le alargue sus años!» Cierto que si el fárrago de mis cartas
cayera en manos de un español listo y versado en letras, vería que por
los huecos de aquella balumba de citas \emph{koránicas} y de adulaciones
al Mogreb y a sus bárbaras tropas, asoman las ideas cristianas, todo el
saber que se trae uno al mundo desde que le ponen en la frente la sal
del bautismo. Claro que el bestia de \emph{El Zebdy} no verá más que la
superficie de lo escrito; en el fondo no penetrará, porque su entender
romo es incapaz de penetración, como el de todo muslim que no ha salido
de estas ciudades apestosas; se holgará mucho de mis falsas historias, y
las mostrará a sus amigos. No quiera Dios que ojos cristianos las lean,
pues entonces saltará de los renglones el engaño que en ellos se oculta,
y adiós fingimiento mío\ldots{} Allah me guarde siempre\ldots{} o Dios,
si tú lo quieres\ldots{} y en confundirlos no hay pecado, que de
estrellas arriba el que manda es quien es, y no se cura de que aquí le
demos este nombre o el otro. Entiéndelo, hijo.»

Calló \emph{El Nasiry}, quedando un ratito en meditación. Juan, metido
también en sí, no echaba en saco roto la lección de fingimiento. La
pausa terminó con un suspiro del caballero moro, y con decir este a su
amigo: «Creo, Juan, que es hora de que vuelvas a casa. \emph{Yohar} la
blanquísima estará inquieta porque tardas\ldots{} Yo me quedo aquí: mi
inquilino, que como amanuense del \emph{Kadí} es hombre de letras, me
tendrá preparados los trastos de escribir. Aquí enjareto mi carta al
gaznápiro de \emph{El Zebdy}, y hago tiempo hasta que llegue la noche,
pues de día no verán mi rostro las calles de Tetuán. Cuando obscurezca
iré a mi casa, que ahora es tuya, y te visitaré a ti y a toda la caterva
que allí se me ha metido. Procuraré recoger a \emph{Ibrahim} y a
\emph{Maimuna}, que amedrentados huyeron de vosotros, teniéndoos por
diablos\ldots{} Entre todos me cuidaréis la casa, que ha venido a ser
refugio maternal de moros, cristianos y judíos\ldots{} Anda, hijo, no te
detengas\ldots{} Allah y la Virgen te acompañen\ldots{} Dios y la Virgen
digo. Todo es lo mismo\ldots{} Dios hizo al hombre, y el hombre ha hecho
los nombres de Dios\ldots{} Abur.»

\hypertarget{iv-3}{%
\section{IV}\label{iv-3}}

Camino de su prestada vivienda, Juan pasó por España\ldots{} España
invadía las calles, pasadizos y rinconadas de Tetuán, gozosa,
entusiasta, decidora, con todo su vigor de espíritu y toda la sal de su
lenguaje. ¿Quién se acordaba ya de las fatigas, de las hambres, de la
muerte de compañeros mil, de las penalidades de todos? Gustaban los
soldados la victoria como un manjar celestial que asemejándoles a los
dioses les revestía de la más pura dignidad, y les inspiraba mayor
indulgencia con los vencidos, y más vivo amor a la patria ausente.
¡Fenómeno singular! Traídos a la victoria por O'Donnell, todos se
parecían a él; en todos se reflejaba la serenidad majestuosa del héroe
triunfante. No se maravilló poco Santiuste cuando vio y supo que ni el
más leve atropello habían cometido los soldados vencedores: a moros y
judíos trataban con afable generosidad, repartiendo entre ellos el pan
que llevaban para sí. El triunfo ganado con las dos grandes virtudes
militares, el valor y la obediencia, la suma acción, la suma pasividad,
a todos infundía ideas y talante de caballeros.

Al pasar por el \emph{Zoco}, advirtió Juan que en el \emph{Mellah} gran
número de soldados confundían su júbilo bullicioso con la bullanga de
las hebreas. No quiso entrar en el barrio judío, donde pudiera
aparecérsele la irritada figura de Riomesta, y abriéndose paso entre la
muchedumbre de militares, tomó la dirección de su casa. Buscaba rostros
amigos, y el primero que vio por dicha suya fue el del beatífico clérigo
castrense \emph{don Toro Godo}, que al pronto no le conoció: de tal modo
le desfiguraba la morisca vestimenta. Se abrazaron; mucho tenían que
hablar y que contarse; pero Juan iba deprisa, y ya charlarían en mejor
ocasión\ldots{} Con interés vivo y palabra rápida preguntó por los
amigos: «¿Y Alarcón, y Pepe Ferrer, y Clavería, y el dibujante Vallejo,
y Rinaldi, y este y el otro y el de más allá?» De casi todos le dio
\emph{don Toro} noticias lisonjeras\ldots{} «Abur, hasta luego\ldots»
«Nos veremos mañana\ldots» Diez pasos más, y el poeta de la Paz se
encontró frente a frente del poeta de la Guerra, Pedro Antonio de
Alarcón, que venía de la casa de Erzini con su amigo Carlos Iriarte,
escritor y dibujante francés. Grande fue el estupor del de Guadix al ver
a su amigo sano, limpio, alegre de rostro y mirada, y con aquel airoso
empaque musulmán que cuadraba tan bien a su tipo y figura.

«¿Qué tienes que decir, Pedro, de la metamorfosis de tu amigo? ¿Me
creías muerto? Muerto fui, resucitado soy. Abrázame una y cien
veces\ldots{} ¡Viva el África hospitalaria!\ldots{} ¿Para qué hemos
conquistado a la blanca Tetuán sino para establecernos en ella?»

---¡Viva Tetuán, y España por los siglos de los siglos viva!---gritó el
granadino con toda la fuerza de su voz, los brazos en cruz.---¡Cuánto me
alegro de verte! ¡Qué guapo estás! ¿Quién te ha dado esta ropa?
Pillastre, ¿has conquistado alguna morita?

---Ya te contaré\ldots{} Tengo prisa\ldots{} vuelvo. ¿Dónde me esperas?
Tenemos mucho que hablar.

---¿Estabas aquí cuando la batalla del 4 de Febrero?\ldots{} ¡Acción
clásica de guerra! Yo veo en ella el triunfo de la Artillería, y la obra
maestra de O'Donnell. Ensalcemos esta grande ocasión de los tiempos
presentes. ¡Con cien mil de a caballo, cuándo nos veremos en
otra!\ldots{} ¿Pero tú qué has hecho, qué haces ahora?

---Si viene la paz, haré la historia de ella\ldots{} Lo que falta para
llegar a la paz, yo lo contaré al mundo. No me mires con burla. Ya te
demostraré que alguna hojita de los laureles que habéis conquistado me
corresponde a mí\ldots{} Tetuán, la \emph{Blanca Paloma}, nuestra
es\ldots{} Si vosotros con el acero y la pólvora habéis hecho una gran
conquista de guerra, yo, con pólvora distinta, he hecho una conquista de
paz. ¿Cuál será más duradera, Perico?\ldots{}

\flushright{Madrid, Octubre-Noviembre-Diciembre de 1904-Enero de 1905.}

~

\bigskip
\bigskip
\begin{center}
\textsc{fin de aita tettauen}
\end{center}

\end{document}
