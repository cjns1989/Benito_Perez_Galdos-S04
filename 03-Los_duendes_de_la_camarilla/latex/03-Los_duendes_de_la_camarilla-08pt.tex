\PassOptionsToPackage{unicode=true}{hyperref} % options for packages loaded elsewhere
\PassOptionsToPackage{hyphens}{url}
%
\documentclass[oneside,8pt,spanish,]{extbook} % cjns1989 - 27112019 - added the oneside option: so that the text jumps left & right when reading on a tablet/ereader
\usepackage{lmodern}
\usepackage{amssymb,amsmath}
\usepackage{ifxetex,ifluatex}
\usepackage{fixltx2e} % provides \textsubscript
\ifnum 0\ifxetex 1\fi\ifluatex 1\fi=0 % if pdftex
  \usepackage[T1]{fontenc}
  \usepackage[utf8]{inputenc}
  \usepackage{textcomp} % provides euro and other symbols
\else % if luatex or xelatex
  \usepackage{unicode-math}
  \defaultfontfeatures{Ligatures=TeX,Scale=MatchLowercase}
%   \setmainfont[]{EBGaramond-Regular}
    \setmainfont[Numbers={OldStyle,Proportional}]{EBGaramond-Regular}      % cjns1989 - 20191129 - old style numbers 
\fi
% use upquote if available, for straight quotes in verbatim environments
\IfFileExists{upquote.sty}{\usepackage{upquote}}{}
% use microtype if available
\IfFileExists{microtype.sty}{%
\usepackage[]{microtype}
\UseMicrotypeSet[protrusion]{basicmath} % disable protrusion for tt fonts
}{}
\usepackage{hyperref}
\hypersetup{
            pdftitle={LOS DUENDES DE LA CAMARILLA},
            pdfauthor={Benito Pérez Galdós},
            pdfborder={0 0 0},
            breaklinks=true}
\urlstyle{same}  % don't use monospace font for urls
\usepackage[papersize={4.80 in, 6.40  in},left=.5 in,right=.5 in]{geometry}
\setlength{\emergencystretch}{3em}  % prevent overfull lines
\providecommand{\tightlist}{%
  \setlength{\itemsep}{0pt}\setlength{\parskip}{0pt}}
\setcounter{secnumdepth}{0}

% set default figure placement to htbp
\makeatletter
\def\fps@figure{htbp}
\makeatother

\usepackage{ragged2e}
\usepackage{epigraph}
\renewcommand{\textflush}{flushepinormal}

\usepackage{indentfirst}

\usepackage{fancyhdr}
\pagestyle{fancy}
\fancyhf{}
\fancyhead[R]{\thepage}
\renewcommand{\headrulewidth}{0pt}
\usepackage{quoting}
\usepackage{ragged2e}

\newlength\mylen
\settowidth\mylen{...................}

\usepackage{stackengine}
\usepackage{graphicx}
\def\asterism{\par\vspace{1em}{\centering\scalebox{.9}{%
  \stackon[-0.6pt]{\bfseries*~*}{\bfseries*}}\par}\vspace{.8em}\par}

 \usepackage{titlesec}
 \titleformat{\chapter}[display]
  {\normalfont\bfseries\filcenter}{}{0pt}{\Large}
 \titleformat{\section}[display]
  {\normalfont\bfseries\filcenter}{}{0pt}{\Large}
 \titleformat{\subsection}[display]
  {\normalfont\bfseries\filcenter}{}{0pt}{\Large}

\setcounter{secnumdepth}{1}
\ifnum 0\ifxetex 1\fi\ifluatex 1\fi=0 % if pdftex
  \usepackage[shorthands=off,main=spanish]{babel}
\else
  % load polyglossia as late as possible as it *could* call bidi if RTL lang (e.g. Hebrew or Arabic)
%   \usepackage{polyglossia}
%   \setmainlanguage[]{spanish}
%   \usepackage[french]{babel} % cjns1989 - 1.43 version of polyglossia on this system does not allow disabling the autospacing feature
\fi

\title{LOS DUENDES DE LA CAMARILLA}
\author{Benito Pérez Galdós}
\date{}

\begin{document}
\maketitle

\hypertarget{i}{%
\chapter{I}\label{i}}

Medio siglo era por filo\ldots{} poco menos. Corría Noviembre de 1850.
Lugar de referencia: Madrid, en una de sus más pobres y feas calles, la
llamada de Rodas, que sube y baja entre Embajadores y el Rastro.

La mañana había sido glacial, destemplada y brumosa la tarde; entró la
noche con tinieblas y lluvia, un gotear lento, menudo, sin tregua, como
el llanto de las aflicciones que no tienen ni esperanza remota de
consuelo. A las diez, la embocadura de la calle de Rodas por la de
Embajadores era temerosa, siniestro el espacio que la obscuridad
permitía ver entre las dos filas de casas negras, gibosas, mal
encaradas. El farol de la esquina dormía en descuidada lobreguez; el
inmediato pestañeaba con resplandor agónico; sólo brillaba, despierto y
acechante, como bandido plantado en la encrucijada, el que al promedio
de la calle alumbra el paso a una mísera vía descendente: la Peña de
Francia. Ánimas del Purgatorio andarían de fijo por allí; las vivientes
y visibles eran: un ciego que entró en la calle apaleando el suelo; el
sereno, cuya presencia en la bajada al Rastro se advirtió por la
temblorosa linterna que hacía eses de una en otra puerta, hasta
eclipsarse en el despacho de vinos; una mendiga seguida de un perro, al
cual se agregó otro can, y siguieron los tres calle abajo\ldots{} En el
momento de mayor soledad, una mujer dobló con decidido paso la esquina
de Embajadores, y puso cara y pecho a la siniestra calle, sin vacilación
ni recelo, metiéndose por la obscuridad, afrontando animosa las
molestias y peligros del suelo, que no eran pocos, pues donde no había
charco, había resbaladizas piedras, y aquí y allá objetos abandonados,
como cestos rotos o montones de virutas, dispersos bultos que figuraban
en la obscuridad perros dormidos o gatos en acecho.

Que la mujer era joven se revelaba en la viveza de su marcha, y en la
gracia exquisita de aquel paso de baile con que sus pies ligeros sabían
evitar las mojaduras y asentar en los puntos más sólidos. Tan pronto se
arrimaba a las casas de la derecha como a las de la izquierda, con
pericia de práctico navegante. Las gotas de lluvia bailaban en los
charcos, produciendo un punteado luminoso: era la única claridad que
permitía discernir los contornos de aquel archipiélago, y precisar sus
sirtes engañosas o el seguro de sus islotes. La moza, que tal era sin
duda, pues no hay disfraz que disimule la juventud, iba totalmente
vestida de negro, falda y pañuelo de manta del color de la noche, lo
mismo que el pañuelo de la cabeza. Sólo llevaba color chillón en los
pies, calzados con zapatos o borceguíes rojos, de un tono vivo de
púrpura, como la sotana de los monagos. Esto era en verdad
singularísimo, y cuando se levantaba la faldamenta, no tanto para evitar
el lodo, como para tener mayor desembarazo en sus ondulaciones
coreográficas, el paso de la consabida mujer hacía pensar en artes y
travesuras de brujería\ldots{} En la pendiente de la calle estaba ya,
donde los baches y pedruscos entorpecían más el perverso camino, cuando
vio sombrajos de personas que subían del Rastro. El recelo y la
curiosidad la detuvieron; se metió detrás de un esquinazo para observar.
Su actitud habría podido trasladarse al lenguaje común sin más
literatura que esta sencilla interrogación: «¿Serán\ldots?» Parecía que
se tranquilizaba oyendo y reconociendo sus voces; y cuando les vio
escurrirse por la Peña de Francia, descender aprisa dando tumbos, por lo
que más parecía torrente que calle, y sumirse por un agujero, como
alimañas que habitan en los cimientos de los edificios, la moza recobró
completamente su tranquilidad. Los chapines rojos, que eran lo único de
ella que en aquel silencioso navegar hablaba, dijeron claramente,
brincando de guijarro en guijarro: «No hay cuidado; son\ldots» A poco de
esto, empujaba una puerta, en la acera derecha, y se metía en un
antro\ldots{}

El cual no era otra cosa que un vasto depósito de puertas, ventanas,
balcones, rejas y persianas, despojo de casas derribadas, todo ello, por
obra de la obscuridad de aquella noche tristísima, convertido en
aglomeración de formas durmientes. Dormían las filas de puertas
ordenadas por tamaños, como inmensos tomos de interminables
enciclopedias; dormían los que fueron balcones y ya parecían jaulas;
dormían las rejas, que ya eran como descomunales parrillas para el asado
de bueyes enteros. Peor estaban aquellos pisos que los de la calle,
porque junto a la entrada se había formado una laguna de riberas
lejanas, desconocidas. Pero la viajera de los rojos escarpines, que ya
dominaba la orografía de aquellos lugares, se escabulló lindamente con
viradas o quiebros oportunos, hasta que arribó al puerto\ldots{} Vio
luz, entró bajo techo, y una mujer o señora, que esto no podía definirse
aún, le tocó la ropa y con lástima cariñosa le dijo: «Vienes caladita.
Vete a la cocina y sécate, y come alguna cosa, mujer.» La de los zapatos
colorados respondió con una fórmula de gratitud, añadiendo que no podía
entretenerse\ldots{} Fácilmente se comprendía que una mayor querencia
que el secarse y comer solicitaba con imperio su voluntad. «Vete, vete
pronto---le dijo la que sin duda era dueña de la casa.---Estás deshecha
por llegar pronto, y hartarte de mimo\ldots{} ¡Ay, hija! la juventud es
un gusanillo que pide ilusión y tienes que dársela: si no, te come toda
la vida. Más vale suspirar de joven por enamorada que de vieja por
desconsolada. Aprovecha el tiempo, que vuela, hija, llevándose las
ocasiones, y el muy perro se las guarda para que no puedan volver\ldots»
Más dijo, más quiso decir, revelándose en tan corto instante como
habladora sin tasa; pero la otra, que ya conocía y padecer solía el
torbellino de sus vanos discursos, no la dejó aquella noche asegurar la
hebra, y extremando sus prisas impacientes dijo: «Señá Casta, con
permiso\ldots{} déjeme subir, que vengo retrasada y estará con cuidado.»

Sin dar espacio a más razones metiose por un pasillo anguloso, saludó a
una criada, acarició a dos niños que de los aposentos alumbrados y
calentitos salían a verla, y por una puerta próxima a la cocina humeante
pasó a otro patio más pequeño que el primero, y como aquel, húmedo,
tenebroso, atestado de material de derribos, predominando los fragmentos
de altares, de púlpitos y demás carpintería eclesiástica. Por la
estrecha calle que las pilas de aquellos nobles vestigios dejaban al
tránsito, se escabulló con ligereza hasta dar con una escalera por la
cual subió, como si dijéramos, de memoria, palpando y reconociendo con
manos y pies. De ladrillo y nada corto era el primer tramo. Torciendo a
la derecha encontró la moza el segundo, de madera, interminable serie de
peldaños temblorosos y gemebundos, sin ningún descanso, sin vuelta, todo
seguido, seguido, en fatigante línea recta trazada en los senos de la
pesadilla. La última parte de aquella lucha opresora con las alturas iba
por descubiertos espacios. Mirando hacia abajo se veía el patio grande,
parte de la calle de Rodas, y a la izquierda patios de casas
domingueras, en cuyas celdas se veían claridades, y a lo largo de los
corredores o en las entornadas puertas sombras movibles. Rumor de
humanidad subía también, y un cuchicheo de la vida afanosa requiriendo
el descanso nocturno\ldots{}

Vencido el último escalón encontrose la mujer en un secadero de pieles,
que antes de ser visto se denunciaba por el olor nauseabundo. Pasó la
viajera, conteniendo el aliento, por los bordes del tenderete, y llegó a
una como azotea, secadero abandonado y en ruinas, conservando los pies
derechos que habían sostenido su techumbre. Allí se detuvo un instante
para tomar resuello y meter aire limpio en sus pulmones. Vio el patio de
otra casa de corredor, correspondiente a la calle de la Pasión, y por
costumbre de mirar al cielo en tales alturas echó atrás la cabeza con
movimiento de astrónomo. Pero el cielo, que otras noches desplegaba su
soberana hermosura sobre este montón de miseria y porquería en que
vivimos, aquella noche parecía espejo en que se retrataba lo de abajo,
un fangal sucio, tenebroso. Arreciaba la lluvia en aquel instante, y el
agua, escurriéndose aquí, goteando allá, buscando presurosa todos los
caminos y conductos que la condujeran a la tierra, hacía los ruidos más
extraños. En los apanzados techos mohosos corría un bullicio de
arroyuelo campesino, y en las canales rotas entregábase a ejercicios de
fontanería burlesca. Los absorbederos en buen uso la paladeaban antes de
tragarla.

En todo esto fijó brevemente su atención la de los rojos chapines,
buscando en la observación de tales ruidos un alivio al miedo que otros
le causaban, como el galopar frenético de ratones en retirada, y el
bufido de gatos feroces que les buscaban las vueltas en las entradas y
salidas del colgadero de pellejos\ldots{} Aún tenía que franquear la
moza un paso difícil para llegar al término de su viaje. Pisando tablas
rotas, metiose por estrecho espacio entre una medianería y un grupo de
chimeneas; llegó al alcance de un ventanón de vidrios emplomados, en
parte rotos y sustituidos con papeles, y al reconocerlo por la claridad
que los sucios cristales transparentaban, golpeó con los nudillos como
anunciando su llegada\ldots{} De allí pasó a un segundo hueco, que lo
mismo podía ser ventana que puerta, con un batiente de cuarterones y
otro de cristalera alambrada: empujó\ldots{} Entró como paloma que
vuelve al nido.

Era un recinto abohardillado, como de seis varas de largo por tres de
anchura; por un extremo, de elevación bastante para que personas de
buena estatura pudieran estar en pie, por el otro suficiente no más para
un perro de mediana talla\ldots{} La entrada de la mujer fue ruidosa: en
ella, como un júbilo triunfal; en el que la esperaba, como término de
ansiedad expectante. El farolillo que alumbraba la mísera estancia daba
la claridad precisa para determinar vagamente los objetos, y no tomar
por personas las prendas de ropa colgadas de una cuerda. La moza se
adelantó hacia un camastro, que más bien debiera llamarse rimero de
pieles, mantas y enjalmas; de aquel diván humilde surgió el busto de un
hombre, que abiertos en cruz los brazos, exclamó: ¡Cuánto has tardado,
mi alma! ¡En qué ansiedad me has tenido, corazón! No me consolaba más
que la idea de morirme esta noche.

---¿Morirte tú, mi \emph{Tolomín}, sin que tu Cigüela te dé
licencia?\ldots{} No faltaba más\ldots---dijo ella sin abrazarle más que
con la intención.---Chiquillo, no me abraces tú\ldots{} Toca, y verás
que estoy hecha una esponja. Déjame que me sacuda\ldots{}

Diciéndolo, de un tirón desenlazó el pañuelo de la cabeza, quitose el de
manta, y ambas piezas colgó en la cuerda de que pendían otras. Luego,
risueña, con gracioso brinco, llegose al camastro, y alzando una pierna
mostró el chapín rojo puntiagudo: \emph{Mia}, \emph{mia} qué pinreles
traigo, Tolomín.

---¡Ay, qué bonitos! ¿De dónde has sacado eso?---dijo el hombre tirando
del borceguí, que chorreaba.

---Ya te contaré---replicó la moza alargando el otro pie para que lo
descalzara.---Pero antes de hablar eso, tengo que contarte otras
cosas\ldots{} muchas cosas, Tolomín.

Desnudos quedaron los pies de Cigüela, y mojaditos como si hubiera
venido descalza. El hombre acostado le tiró de la falda, la obligó a
sentarse junto a él, y le secó un pie diciéndole cariñoso: ¡Pobre
\emph{Güela}, los trabajos que ella pasa por su Tolomín!\ldots{} Dame
ahora el otro: están heladitos.

---Ya se calentarán\ldots{} ¡Con sentarme sobre ellos\ldots! Pero antes
tengo que arreglarte un poco tu sala, tu gabinete, tu camarín y toítas
estas dependencias \emph{maníficas}, como decimos las manolas, y
\emph{maggg}\ldots{} \emph{níficas}, como decimos las señoritas del pan
pringado\ldots{} Verás, Tolomín, qué pronto despacho.

---Mientras me ordenas el mechinal, cuéntame lo que pasa en Madrid, que
ello habrá sido gordo\ldots{}

---No pasa nada, hijo\ldots{}

---¿Cómo que no? Yo he oído tiros.

---Estás soñando.

---Tiros de fusilería, y alguno, alguno de cañón---afirmó el hombre con
sincero convencimiento.---Oyéndolos, me dije: «Ya se armó.» Y como
tardabas, pensé que por estar cortadas las calles no podías pasar hacia
acá, y también me asaltó la idea de que te cogiera una bala
perdida\ldots{}

---¡Pobre Tolomín!\ldots{} Dormido has oído los tiros; que quien
despierto sueña con revoluciones y trifulcas, más ha de soñar cuando
cierra el ojo.

---No, no: bien despierto estaba cuando oí los disparos de
fusilería\ldots{} y ello sonaba por esta parte: primero lejos, como en
la Puerta del Sol; después más cerca, como en Puerta Cerrada.

---¡Ay, qué engañoso y qué visionero!\ldots{} Te aseguro que esos tiros
no han sonado más que en tu pobre magín enfermo, y que Madrid está más
tranquilo que un convento de monjas\ldots{} no, no es buena
comparación\ldots{} más tranquilo que un cementerio\ldots{}

---¿De veras no hay barricadas?\ldots{} ¡Cigüela!

---Tolomín, no hay barricadas. Las habrá; consuélate con la esperanza.
Las habrá\ldots{} y tan altas que lleguen a los pisos terceros, si
quieres\ldots{} Pero lo que es hoy\ldots{} ¡Bueno ha estado el día, y
bonita la noche para esas bromas! Con las calles mojadas y la pólvora
revenida ¿quieres tú jarana?\ldots{} Las revoluciones quieren sol, como
los toros, y el patriotismo no ha de ser pasado por agua\ldots{}

---Por decírmelo tú lo creo, que cuanto tú dices es para mí artículo de
fe; pero yo estoy bien seguro de lo que oí\ldots{} segurísimo\ldots{}
¡Pim, pam\ldots! ¡Fuego\ldots! ¡pim, pam\ldots! duro y a la
cabeza\ldots{} ¡pim, pam!

---Ea, no te encalabrines\ldots{} Te volverá la calentura.

---¡Libertad o muerte! ¡Fuego!

---Juicio, mi Capitán\ldots{} No estamos tan lejos del mundo,
que\ldots{}

---¡Viva Isabel II!

---¡Chitón!

---¡Viva España, viva la Libertad! Todo esto va contigo, boba: la Reina
eres tú; tú eres España, tú la Libertad\ldots{}

\hypertarget{ii}{%
\chapter{II}\label{ii}}

Cigüela reía. Lo primero que hizo, al acometer sus menesteres
domésticos, fue sacar del bolsón pendiente de su cintura bajo la falda,
dos paquetes con envoltura de papel fino cruzada de cinta roja, y
ponerlos sobre una caja que servía de mesa. Descalza, diligente, iba de
un punto a otro con suma presteza; y sosteniendo la conversación con el
aburrido Tolomín, al deber de mirar por su existencia y su salud
atendía. En el lado donde era más alto el techo, tenía un anafre, y en
sitio cercano provisión de carbón, teas y una caja de fósforos. Encendió
lumbre y puso a calentar agua. «¿Qué me has traído esta
noche?»---preguntó Tolomín, que no quitaba ojo de los paquetes cerrados
con desusada elegancia y finura.

---¡Cosa rica!\ldots{} Ya lo sabrás\ldots{} Antes tengo que
contarte\ldots{}

---¡Vaya! Pues no gastas poca solemnidad para tus cuentos\ldots{} ¡Antes
con antes!\ldots{} ¿Pero dónde está el principio de tus historias?

---No se debe contar lo segundo sin contar lo primero---dijo Lucila
risueña y un tanto maliciosa.

---Pues échame lo primero de una vez\ldots{} ¿Dónde estuviste esta
noche? ¿Por qué has tardado? ¿Es esto el principio, o dónde demonios
está el principio de lo que tienes que contarme?

---¡El principio!\ldots{} Cualquiera sabe dónde está el principio de las
cosas.

---No te diviertes poco con mi curiosidad. Vamos, ¿a que te acierto de
dónde son esos paquetes? Son de la repostería de Palacio.

---¡Huy\ldots{} qué desatino!\ldots{} ¡Vaya un zahorí que tengo en casa!

---¿Con que no son de Palacio? Pues de las monjas no son, porque esas
señoras no envuelven sus regalitos con papeles a estilo de París, sino
con papel viejo del que venden las covachuelas, y que parece pergamino,
y a lo mejor te trae un pleito de principios del siglo pasado\ldots{}
Pues a ver si acierto\ldots{} Dame los paquetes, a ver si por el
olor\ldots{}

---Luego, Tomín---dijo Lucila cogiendo una jofaina del depósito de loza
que en un rincón tenía, piezas diferentes en mediano uso, alguna
desportillada, todas muy limpias.---Ahora, caballero, a lavar las
heridas.

---¡Ay, ay, qué fastidio!---exclamó Tomín incorporándose.---Pero tienes
razón. Si me duele, que me duela. Lávame, cúrame: tus manos de madre me
sanarán.

---Y para que mi pobre niño no se devane los sesos con
adivinanzas---añadió Lucihuela avanzando con la jofaina, una esponja y
trapos,---le diré dónde estuve esta noche\ldots{} ¿No me encargaste esta
mañana que me viera con mi padre?

---¡Ah, sí!

---¿Y no sabes, tontaina, que a mi padre lo han empleado en el teatro
nuevo de la Plaza de Oriente?

---¡Ay\ldots{} qué tonto yo! ¡no caer\ldots! Verde y con asa\ldots{}
Esta noche es la inauguración\ldots{}

---Y hoy los días de nuestra Soberana.

---«¿No te dije que había oído cañonazos? Pues la verdad, siento mucho
que los tiros fueran por Santa Isabel y no por un bonito
pronunciamiento. Créelo: más falta nos hace la Libertad que todos los
santos del Almanaque, y más cuenta nos tiene una revolución bien traída
que el mejor coliseo para ópera y baile.»

Penosa era la cura y el poner los nuevos apósitos, después de bien
despegados los del día anterior; pero los dedos de Lucila, que en aquel
caso clínico se habían adestrado, instruidos por el amor más que por la
ciencia, llegaron a adquirir singular delicadeza. El bueno de Tolomín,
valiente hasta la temeridad y sufrido cual ninguno en los lances de su
militar oficio, era en las dolencias de una flojedad infantil y
quejumbrosa. Por cualquier dolor ponía el grito en el cielo, y la
sujeción a planes médicos le desesperaba. Conociendo su flaqueza,
reservaba Lucila para el momento de la cura todas las referencias
humorísticas que tuviera que hacerle, y al contarlas forzaba la
inflexión cómica para entretenerle o provocarle a risa. Dígase, para ir
construyendo todo el aparato informativo de este personaje, que las
heridas eran dos: una de cuidado en la región femoral derecha, de arma
blanca; la otra de bala en el antebrazo izquierdo, herida nada grave,
aunque lo parecía por la proximidad de unas erosiones molestísimas en el
hombro, que interesando los músculos del cuello impusieron al paciente,
en los primeros días, cierta rigidez de busto escultórico.

---Ea, ¿ya empezamos con chillidos?---decía Cigüela.---La culpa tengo yo
por darte tanto mimo. ¡Si no te duele!\ldots{} Ya ves con qué suavidad
voy levantando el trapo, después de mojarlo bien con agua
templada\ldots{} Otro tironcito\ldots{} Ya falta poco\ldots{} Pues te
contaré: Loco de contento está mi padre con su destino en el Teatro que
ahora se llama Real\ldots{} Me ha dicho que de balde desempeñaría la
plaza sólo por rozarse tarde y noche con el cuerpo de baile, y por ser
demandadero de las \emph{cantarinas}, como él dice.

---No me hagas reír\ldots{} ¡Ay, ay, que me arrancas la carne!\ldots{}
¡Ay!\ldots{} No sé cómo me río. Sigue.

---Yo no sé lo que es mi padre allí. ¿Es portero, celador? ¿Corre con la
tramoya, con el gas, con la vestimenta? Vete a saber. Me ha contado que
nunca creyó que hubiera en el mundo cosa tan bonita como las comedias
cantadas. De todas las mentiras del mundo, dice, esta de la comedia con
música y en italiano es lo que más se parece a la Gloria\ldots{} Y de
ello saca que mientras más grande es la mentira y más separada de la
verdad, mejor nos da idea del Cielo.

---Que no me hagas reír, Lucila\ldots{} ¡Ay, ay!

---En los ensayos se queda como embobado, y cuando oye la orquesta con
tantos violines, todos tocando al mismo son, le entran ganas de llorar,
de ponerse de rodillas, y de arrepentirse de todas las picardías que ha
hecho\ldots{}

---¡Ay, ay, qué gracioso!

---Dice que oyendo el habla dulce de las italianas, le entran ganas de
ilustrarse para entender bien lo que dicen, y ser como ellas pulido y de
mucha cortesía\ldots{} Y que cuando las tales y otras cuales españolas
le mandan con recados para costureras, o para los maestros de música, le
entran ganas\ldots{}

---¡Ay, qué risa!\ldots{} Por todo le entran ganas\ldots{}

---Ganas de servirlas con diligencia, y de adivinarles los mandamientos
para cumplirlos, siempre que sean honrados.

---Estarán contentísimos de él\ldots{}

---«Y él más contento que nadie, porque, según cuenta, en aquel puesto
está, noche y día, mano a mano con todo el señorío\ldots{} La ópera es
el puro señorío, y el aquel más fino de las aristocracias nobles, como
quien dice, porque todo allí es de familias reales, y por eso el teatro
se llama Real, siendo reyes los tenores y reinas las cantarinas, o
\emph{verbigracia} tiples\ldots»

En esto, terminados felizmente el lavado y cura, Tomín suspiró. Cesaron
los fugaces dolores, y el hombre, consolado, expresaba en su mirar
contento y gratitud. Lucila procedió a lavarse y jabonarse manos y
brazos, después de devolver a su sitio todos los adminículos de la cura,
sin interrumpir su graciosa charla, de que tanto gusto recibía el
desdichado enfermo. Pues esta noche, cuando fui a ver a mi padre, me le
encontré muy sofocado, por tener que acudir con un solo cuerpo a tantos
puntos y menesteres. Entré por la plazuela de Isabel II, y tuve la
suerte de encontrarle en la escalera que sube al escenario. Reñía con
unos tagarotes que subían trastos, y que a mi parecer estaban peneques.
Subí con él y entramos en un cuartito donde había no sé cuántos hombres
vestidos de frailes, fumando\ldots{} Yo había corrido por Madrid desde
media tarde, lloviendo a mares, las calles como lagunas, y mis zapatos,
que ya venían rotos, daban por delante y por detrás las boqueadas. Los
pies me dolían, me pesaban, y donde quiera que yo los ponía dejaban un
charco. Uno de aquellos frailuchos, que tenía en la mano derecha el
cigarro y en la izquierda la barba postiza, me miró los pies y dijo a mi
padre: «¿Cómo consiente el gran Ansúrez que ande esta preciosa niña por
Madrid como los patos?» Yo alargué ambos pies para que mi padre se
compadeciera. «Ya te entiendo---me dijo.---Vienes a que te dé para
calzado\ldots{} Qué más quisiera este padre que tener a toda la plebe de
sus hijos bien apañadica. Pero el dinero no abunda, lo que no quiere
decir que me falten medios para remediarte. Por poco nos apuramos, hija
del alma. Ven conmigo, y pisa ligero para no mojar tanto\ldots» Llevome
por unas escaleras que no tienen fin y que marean de tantas vueltas como
hay que dar por ellas, y de todo, atravesamos una sala donde vi sin fin
de hombres vestidos de colorines\ldots{} Adelante siempre: en los
pasillos encontramos mujeres pintadas, feas las más, guapas muy pocas;
algunas arrastrando cola; todas con alhajas de vidrio y diademas de
cartón dorado. Eran las coristas. Con llave que sacó de su bolsillo
abrió mi padre la puerta de un cuartón, lleno de ropas de máscara.
Parecía una tienda de alquilador de disfraces. En el suelo vi un montón
de zapatos y borceguíes de todos colores. Mi padre me dijo: «De esta
zapatería de comedias cantadas o por cantar, escoge lo que más te guste.
Nos trajo ayer todo este material un marchante que tuvo el suministro de
equipos para teatros donde salen séquitos y acompañamiento de reyes, o
donde figuran diablos, ninfas y personas mágicas. Pretende que el
intendente de acá lo compre, y mientras se determina, me ordenaron que
aquí lo metiera y guardara\ldots{} Paréceme que esos chapines encarnados
que acaban en punta son de la medida de tus pies. Cógelos y no repares,
hija de mi corazón.» Pues vistos y examinados los chapines, me
parecieron bien. Me quité la miseria de mis zapatos, y con las medias
empapadas lo tiré todo en aquel montón, poniéndome los borceguíes, que
me servirán mientras no tenga cosa mejor. Díjome el padre que este
calzado es para unas brujas de no sé qué tragedia con solfa, en la cual
sale un caballero al que las viejas malditas, amigas del demonio, le
anuncian que será Rey, y él se lo cree, resultando que, por la comezón
del reinar, mata a su soberano, y luego\ldots{} no me acuerdo de más.
«Cuando me ponía mis escarpines, me contó mi padre que él había
encontrado entre aquellos trapos un coleto magnífico, como para un
príncipe cazador que matara las perdices cantando, y que con la tal
prenda y unos pedazos de otra se había hecho un buen chaleco de abrigo.»

Muy entretenida con este relato, el pobre Tolomín no quería que tuviese
término; pero Cigüela hizo un paréntesis. Llegándose a él con otra
jofaina, agua nueva y esponja distinta, le dijo con gracia: «Ahora,
pobretín mío, me dejarás que te lave un poco la carátula\ldots{} Luego
comeremos. ¡Verás qué cosas ricas!» El Capitán hizo un mohín de
protesta, plañidero. Pero ante la insistencia de la moza, cariñosamente
manifestada, cedió y se dejó lavar. Con tiernas frases iba Lucila
marcando la operación: «Primero los ojitos, para que no estén
pintañosos\ldots{} ¡Si vieras qué bonitos te los he dejado!\ldots{} Pues
ahora voy con la nariz, con las mejillas\ldots{} ¿Y este bigotito que
está lleno de pegotes?\ldots{} Si supiera yo afeitarte la barba, te
dejaría más guapín que un sol\ldots{} Voy ahora con las orejas: un
poquitín de paciencia\ldots{} Pronto acabo. Más agua, más. Eres como un
santo viejo, que tu sacristana ha encontrado en un desván. Lo cojo, lo
lavo\ldots{} Pues entre el polvo y las moscas lo habían puesto bueno.
¿Ves? Ahora, ya eres santo nuevo, acabadito de poner en el
retablo\ldots{} Si tuviéramos espejo, verías qué lindo estás\ldots{} ¡Y
qué bien se ha portado mi nene dejándose lavar tan calladito\ldots! Se
merece un beso\ldots{} digo, dos\ldots{} digo, tres.»

---Ciento, mi alma---replicó Tomín besándola con intensa emoción.

---Ahora te paso un peine, y quedarás tan precioso como cuando te
conocí\ldots{}

---Comamos, alma---dijo el herido.---Tengo hambre.

---Ahora mismo. Medio minuto se tarda en poner la mesa y servir el
primer plato---dijo Lucila, que retirando el servicio de lavar trajo al
instante el de comer, y comenzó a deshacer los paquetes.---Pues sigo
contándote. Mi padre, cuando bajábamos, luciendo yo mis zapatos de
bruja, me habló así: «Hija del alma, si yo te hubiera criado desde
chiquita para cantatriz, poniéndote a la solfa con buenos maestros
cantores y salmistas, otro gallo a todos nos cantara\ldots{} Lo que
hicieras con el juego de garganta lo realzarías con el juego de ojos y
toda la sal de tu rostro, que en este beaterio entra por mucho el buen
palmito y el salero del cuerpo\ldots{} Pero la voz es lo
principal\ldots{} y lo que más se paga. ¿Sabes lo que gana la señora
Alboni? Pues mil y pico de duros cada mes\ldots{} Echa duros, hija.
¡Mira que si yo te viera a ti ganando esos dinerales\ldots! Pues otra:
sabrás que a la señora Alboni le he caído en gracia. Dos veces me ha
llamado a su presencia. ¿Para qué creerás? Pues sólo para verme, para
echarme unas miradas tiernas, y decirme que soy la imagen del
\emph{fiero castillano}\ldots{} que si me compongo, fácilmente me
tomarán por un señor duque vetusto.»

---Cigüela---dijo Tolomín soltando la risa,---eso lo inventas tú para
divertirme.

---Tontín, no invento nada. A contarme iba los \emph{rendibús} que hizo
a la cantante; pero había empezado el acto segundo, y tuvimos que
callarnos, arrimaditos a unos bastidores. A mis orejas llegaba un bum
bum; la música no la distinguía yo del ruido; los aplausos y la orquesta
me parecían la misma cosa. Este que ahora canta por lo más fino---me
dijo mi padre,---es el Rey, quien parece ha tenido que ver con mi señora
Alboni, quiere decirse, con la persona figurada que el papel reza y
canta\ldots{} Vi a la Alboni, cuando entró para adentro: es una
gordinflona, una caja de música dentro de otra caja de carne\ldots{}
Refirió mi padre que siempre está comiendo; en su cuarto tiene dos
mesas, una con las cosas de tocador, y otra con el recado de golosinas,
platos de sustento, como jamón con huevo hilado y bartolillos de
tantísimas clases.

---Pero no me has dicho cómo y por qué han venido a nuestra pobre mesa
el solomillo lardeado y la lengua escarlata de la \emph{prima donna}.

---Pues muy sencillo: esta señora, como toda cantante, tiene ida la
cabeza. El seso se le escapa con los gorgoritos. Como es pura música, no
se acuerda de nada. Al instante de mandar una cosa la olvida. Primero
fue por los comistrajes un criado italiano; después la doncella\ldots{}
luego mi padre\ldots{} los tres para un solo encargo\ldots{} y cuando la
señora entró en su cuarto creyó que entraba en la tienda de en casa
Lhardy\ldots{} Enfadándose consigo misma por su poca memoria, empezó a
echar trinos y gorjeos para y para abajo, que es una receta que tiene
para enflaquecer, y luego, todo el sobrante de comida lo repartió entre
los de la servidumbre, tocándole la partija mayor a mi padre, que me la
dio a mí\ldots{} Pues una vez que cogí este regalo, que con los chapines
era bastante para dar por bien empleada mi noche, no quería yo más que
echar a correr. Mi padre no me soltaba. «No, no te vas sin que yo te
enseñe el \emph{golpe de vista}\ldots{} No verás cosa semejante hasta
que ganes el Cielo.» Esperamos al entreacto, y mientras corrían por el
escenario dando patadas los que quitan y ponen los lienzos, mi padre me
llevó al telón que sube y baja, y que en aquel momento parecía una
pared. Díjome que pusiera el ojo en una mirilla con cruzado de alambres,
y por allí vi todo el señorío público, que es cosa para quedarse una
encandilada y trastornada por tres días. ¡Qué lujo, Tomín; qué tienda de
piedras preciosas, de rasos y terciopelos, de pechos mal tapados, de
encajes, de caras bonitas y caras feas, de cruces, bandas y entorchados!
Era como una feria, y yo decía: Parece que todas y todos compran o
venden algo\ldots{} Enfrente vi a la Reina vestida de color de aromo con
adorno de plata, guapísima: diadema, collar de perlas, sin fin de
diamantes; la Reina Madre hecha un brazo de mar y despidiendo luces a
cada movimiento. Mucha gente de Palacio, muchas Ministras, Generalas y
Mariscalas de Campo, y ellos\ldots{} coqueteando más que ellas\ldots{}
Visto el \emph{golpe de vista}, como decía mi padre, ya no me quedaba
nada que ver. Me fui a la calle, rompí con trabajo las filas de coches,
y chapoteando me vine acá.

---Al mirar por el agujero del telón, ¿no viste alguna cara conocida?

---Vi muchas, Tomín\ldots{} Madrid, que parece grande, es chico, y el
que una vez ha visto su gente, la ve luego copiada en todas
partes\ldots{} Tienes sueño\ldots{}

---Sí: me duermo\ldots---dijo el herido abatiendo con dulce pereza los
párpados.---Cigüela\ldots{} si ves que duermo demasiado, me despiertas,
¿eh?\ldots{} no me vaya a quedar muerto\ldots{}

\hypertarget{iii}{%
\chapter{III}\label{iii}}

Con una recomendación semejante se dormía todas las noches el desdichado
Tomín. Si en los primeros días de su doloroso cautiverio le atormentó el
insomnio, una vez descansado y convaleciente, la naturaleza en vías de
reparación abandonábase a un sopor parecido a la embriaguez, sólo
turbado a ratos por la idea de que dejándose caer sin interrupción por
la resbaladiza pendiente del sueño, iría sin pensarlo a parar en la
muerte\ldots{} Viéndole aquella noche al borde de la caída, Lucila o
Cigüela le empujó en vez de contenerle; le pasó la mano por los ojos, le
besó la frente, le acunó con suaves arrullos de nodriza, no sin decirle
que durmiera descuidado: ella le despertaría cuando fuera tiempo. Al
sentirle dormido, se acomodó a su vera, en lo más bajo del camastro,
sentándose a la turca y reclinando su cabeza blandamente sobre el hombro
sano del Capitán. Antes apagó la luz de la linterna, que a su lado
tenía.

En esta postura y disposición, que apenas alteraba por no turbar el
sueño del herido, se pasaba Lucila la noche, descansando algunos ratos,
los más despierta, ante la presencia de sus vigilantes pensamientos que
no querían dormir, ni apagarse en su caldeada mente. La obscuridad del
mechinal no era completa, ni aun en noches turbias como aquella del 19
de Noviembre, pues se veía el rectángulo luminoso del ventanón
cuadriculado por los vidrios. En noches claras, Lucila veía y gozaba la
luz difusa del cielo y alguna estrella resplandeciente. Ruidos no
faltaban. La noche de referencia, los dedos de la lluvia toqueteaban sin
cesar por un lado y otro de aquella frágil construcción; pero ni esto,
ni el mayar de gatos trovadores, ni los golpes que daba un palo roto y
colgante en el secadero, molestaban a Lucila. Sus inquietudes surgían de
su propia imaginación, a veces cuando sus sentidos se apagaban en el
sueño\ldots{} Despertaba como de un salto, creyendo que las
desvencijadas escaleras por donde a su tugurio se trepaba, crujían bajo
el peso de dos, tres o más personas. Las voces se aproximaban\ldots{}
Eran primero un susurro, después un coro como los de las comedias
cantadas.

Más de una vez se levantó, aterrada, y con menos ruido que el que
pudiera hacer un gato se iba derecha a la puerta, y aplicaba el
oído\ldots{} Tardaba un rato la infeliz mujer en convencerse de que los
rumores inquietantes eran querellas en algún patio vecino, o vocerío de
borrachos en la tasca de la calle de Rodas\ldots{} Cuando todo callaba,
el pensamiento se iba del seguro, poniéndose a decir unas cosas, y a
razonarlas con lógica tan bien urdida, que no había más remedio que
creerlo. ¡Dios sacramentado, lo que decía! Pues nada, que el
Sr.~Melchor, alias el \emph{Ramos}, y su esposa señá Casta, poseedores
de aquellos endiablados tenderetes, se cansaban de ser caritativos
encubridores del tapujo y lo denunciaban a la fiera policía, o permitían
que algún taimado servidor lo revelara\ldots{} Hasta que la luz de la
mañana no despejaba su cabeza, limpia de nieblas su tormentosa mente, no
recobraba Lucila la confianza en sus honrados y leales protectores.

Por estos o los otros pensamientos iba siempre a parar al examen de la
tristísima situación a que había llegado, sin ver por ninguna parte
remedio ni salida; todo por el amor a un hombre, razón esencial del
infortunio mujeril. En proporción de su desgracia estaba el origen de
ella: amor tempestuoso, irregular, semejante a un soberano desorden de
los elementos; si amó a Tolomín con ternura cuando le vio y conoció
fugitivo y condenado a muerte, locamente le amó después, teniéndole a su
lado en lastimosa invalidez y acechado por cazadores de hombres. El
Tolomín herido, enfermo, en extrema pobreza, y oculto en un albergue
mísero, merecía un amor que resumiera todos los amores humanos: era,
pues, para Lucila, el prójimo, el amante, el hermano, el niño desvalido,
a quien la cariñosa vigilancia materna defiende de la muerte en todos
los instantes. El inmenso padecer de aquella situación no había
entibiado el ardiente amor de Lucila: por el contrario, la abnegación,
fundiéndose con él, llegaba a constituir un sentimiento formidable, y
del fondo de tanto infortunio brotaban espirituales goces. Por todos los
bienes de la tierra, ofrecidos y dados en montón, no cambiara Lucila su
vida de sacrificio y de protección en aquellos días, y antes muriera
cien veces que abandonar al desgraciado Capitán, aun sabiendo que le
dejaba en manos salvadoras. Y era mayor el mérito de su paciencia
enamorada cuando se daba a pensar soluciones y no encontraba ninguna.
Especiales accidentes de su vida, que aún no conoce bien el historiador,
dieron a la hija de Ansúrez, dos años antes, ocasiones de valimiento en
dos lugares donde residía todo el poder humano; pero ni en uno ni en
otro sitio podía ya solicitar socorro. En el Convento de Franciscanas de
la Concepción no querían ni verla siquiera, como no fuese allá con
propósito de reingresar en la vida religiosa y de abominar de sus culpas
pasadas y presentes; en Palacio, las amistades que creó y mantuvo con su
leal servicio habían perdido ya toda su eficacia.

No podían faltar a Lucila, cuando conciliaba el sueño en las tristes
noches del palomar, pesadillas angustiosas. Consistían siempre en la
súbita presencia de la policía. Soñando que estaba despierta, veía la
moza entrar en la estancia hombres con linternas, y uno de ellos se
adelantaba con mal gesto y decía: «No moverse, no hacer resistencia, no
negar lo que no puede negarse, que ya nos conocemos, señor Capitán D.
Bartolomé Gracián.» Por acostumbrada que estuviera la mujer a tan
terrorífico ensueño, siempre despertaba de él sin aliento, el corazón
disparado\ldots{} ¡Bartolomé Gracián! Habría querido Lucila anular este
nombre, suprimirlo, arrojarlo a los senos de la Nada, donde, a su
parecer, están las cosas que no han existido nunca. De este modo,
eliminado aquel nombre de todas las partes del Universo, quedaría en
salvo la persona que lo llevaba. Jamás lo pronunciaba con el rigor de
sus letras, y el familiar mote de \emph{Tolomé} que en días felices
usaba, lo fue cambiando sucesivamente en \emph{Tolomín}, luego en
\emph{Tomín}, con tendencias a extremar la síncopa pronunciando tan sólo
\emph{Min}. El apellido, aquel \emph{Gracián} tan sonoro y expresivo, lo
declaraba caducado y sin valor acústico, como perteneciente a los
dominios del silencio.

Amaneció el 20 de Noviembre con intermitencias de llovizna y despejo del
cielo. Antes de que el herido despertara, Lucila se levantó diligente, y
puso mano en la limpieza y arreglo de la vivienda mísera: a bien poco se
reducía su trabajo; pero se daba el gusto de variar el sitio de algunas
cosas y de sacudir el polvo de las prendas de vestir. Viendo a su amigo
desperezarse, le dijo: «Min, voy a hacerte tu chocolatito.» Las primeras
palabras de Tolomé fueron estas: «Dime, Cigüela, ¿ha caído Narváez?»

---Hijo, no sé\ldots{} no he oído nada.

---Entonces lo he soñado yo. Sí, sí, sueño ha sido; pero tan claro como
la misma realidad. Las Reinas Hija y Madre despedían a Narváez, como en
aquellos días del \emph{Relámpago}; pero ahora con peor sombra para Don
Ramón, porque no volvían a llamarle, y formaban un Ministerio
eclesiástico\ldots{} No te rías: a esto hemos de llegar, si no lo
remedia quien puede remediarlo, que es el Santo Ejército. España vive
siempre entre dos amos: el Ejército y la Clerecía: cuando el uno la
deja, el otro la toma. ¿Duermen las espadas?, pues se despabila el
fanatismo. Tan despierto anda, que me parece que estamos en
puerta\ldots{} ¿no lo crees así?

---Yo no entiendo de eso, hijo mío---replicó Lucila engolfada en su
trajín.

---Y el propio D. Ramón, o Figueras, o Lersundi, serán los primeros que
saquen los batallones a la calle. Dime que sí, Lucila: dame esa
esperanza.

Afirmó Cigüela todo lo que él quiso, y le regaló el oído con la
confirmación de las ideas que manifestaba. No hay España sin Libertad, y
no hay Libertad sin Ejército---prosiguió Tomín, enardeciéndose más a
cada frase.---Al Ejército debe España sus progresos, y el tener cierto
aire de familia con los pueblos de Europa\ldots{} No hablen mal de las
revoluciones los que son personas y llevan camisa por haberse
pronunciado. ¿La sedición, qué es? El instinto de la raza española, que
por no caer en la barbarie, da un grito, pega un brinco, y en su
entusiasmo viene a caer un poquito más acá de la Ordenanza. Dime que
piensas como yo.

---Sí, hijo, todo está muy bien pensado---y llegándose a él calzada con
los borceguíes rojos y puntiagudos de las brujas de Macbeth,
añadió:---\emph{Min}, tú serás General.»

Aquel día, iniciada ya la reparación de su organismo, Bartolomé estuvo
muy animado, y algunos ratos locuaz. Se desayunó con apetito, y cuando
llegó la hora de la cura y abluciones de la mañana, sometiose sin
remusgar a los requerimientos de su cariñosa enfermera. Quiso esta que
hiciese nuevo ensayo de andar un poquito, probando el renaciente vigor
de la pierna herida, y él aceptó gozoso la idea. Poco tardó Lucila en
vestirle, a medias, echándole una manta por los hombros, pues no había
de salir del cuarto, y puesto en pie con algún dolorcillo en los remos
inferiores, comenzó el paseo. Daremos diez o doce vueltas en la Plaza de
Oriente---le decía Cigüela llevándole bien agarradito a lo largo del
tabuco,---y luego pasearemos a lo ancho, o sea desde el Teatro Real a la
Puerta del Príncipe. No dirás que no estás fuerte, \emph{Min}. De
anteayer a hoy ¡qué mejoría tan grande!

---Di: ¡qué progreso! Esto es progresar, Lucila\ldots{} En los primeros
pasos me ha dolido un poco la pierna. Ya no siento nada. En todo
progreso pasa lo mismo. Duelen los primeros pasos\ldots{} Oye una cosa:
no te olvides hoy de traerme \emph{El Clamor}\ldots{} Me traerás también
\emph{La Nación} y \emph{La Víbora}.

\emph{---La Víbora} me parece que no sale ya.

---Habrá disgustado a la Camarilla\ldots{} Pues me traerás otro papel
cualquiera: \emph{El Mosaico}, \emph{El Duende Homeopático}. La cuestión
es leer\ldots{}

A la vuelta de su paseo, que le probó muy bien, recobró su actitud
perezosa en el camastro bien mullido. Cigüela se puso a coser,
preparándose para salir en busca de recursos con que prolongar un día
más la existencia de ambos,

problema inmenso, cuyas angustiosas dificultades ella sola conocía.
Taciturna estaba la moza, el Capitán, despejado y comunicativo. Su
locuacidad le llevó pronto al optimismo y al mental derroche de
proyectos, contando con un risueño porvenir. Véase la muestra: «Tú me
has dicho que seré General; me lo has dicho por consolarme. Tu profecía
puede ser un halago, y puede ser una gran verdad\ldots{} Porque\ldots{}
fíjate bien, Cigüela\ldots{} lo que no ha pasado todavía, pasará mañana,
o la semana que viene. Narváez cae lanzado de un puntapié: triunfan las
monjitas y sus valientes capellanes. ¡Viva la Inquisición!\ldots{} Pero
no cuentan con la vuelta; que estas partidas siempre la tienen; y el
perro que han echado de casa es de mala boca, mordelón rabioso cuando lo
azuzan. Corren los días, dos semanas no más, y el de Loja, con tres o
cuatro Generales, saca las tropas de sus cuarteles y tira con ellas por
la calle de en medio. La revolución viene a poner las cosas en su lugar.
El Ejército gobierna, y la Clerecía escupe\ldots{} Vuelve todo a ser
como Dios manda, o como manda la Libertad\ldots{} Primer efecto: indulto
general a los que por la Libertad y la Constitución del 12 o del 37
faltaron a la Ordenanza\ldots{} Pues aquí me tienes pasando de condenado
a recompensado. En estos casos, la costumbre es celebrar el triunfo
concediendo a toda la oficialidad un ascenso, o dos ascensos\ldots{}
casos hubo de tres. Me verías pronto restituido a lo que fuí, saltando
de Capitán a Teniente coronel\ldots{} De ahí para \ldots{} figúrate.
Cualquier servicio en persecución de los \emph{rebeldes}, que rebeldes
habrá con este o el otro nombre, me dará los tres galones. Luego\ldots{}
tú fijate en lo que tardó Riego en subir de comandante a General\ldots»

Hizo Lucila un gracioso mohín, como indicando que no sabía la Historia
suficiente para dar su opinión de aquellos asuntos, y él continuó
impávido: «Pasa tu vista por todos los Generales que tenemos, y veme
señalando los que en tal o cual punto de su carrera no fueron condenados
a muerte, o no merecían serlo por sediciosos, por faltar a esa preciosa
Disciplina. Imagina tú el cumplimiento estricto de la Ordenanza en lo
que va de siglo, y dime lo que con ese cumplimiento estricto sería la
Historia de España. Tendrías que decirme una cosa que ya sé, y es que
con la Ordenanza virginal no habría Historia de España, o sería tan sólo
una página muy aburrida y muy negra de la Historia Eclesiástica.»

Recomendole Cigüela que no se ocupara de política ni pensara en
revoluciones. Si estas venían, muy santo y muy bueno; pero si no querían
venir, ¿a qué repudrirse la sangre por traerlas fuera de tiempo?\ldots{}
No podía extenderse a más largo palique sobre estas materias, porque ya
era hora de lanzarse a la calle en busca de medios de vida. Mucho sentía
dejarle solo; creía que no llevaba consigo más que la mitad del alma,
alentada por los afanes, dejándose allí la otra mitad con los
pensamientos de vigilancia y temor. ¿Pasaría algo en su ausencia? Al
volver, ¿le encontraría como le dejaba?\ldots{} Una y otra vez le
recomendó que no se moviera de su lecho, que no cayese en la mala
tentación de levantarse y salir al ventanal, que no hiciese ruido y
permaneciera quietecito, leyendo las entregas descabaladas, que ella
había traído, de \emph{La Italia Roja}, \emph{Historia de las
Revoluciones}, por el Vizconde de Arlincourt, obra que, aun leída en
sueltos retazos, debía de ser de mucho entretenimiento\ldots{} Mutuas
ternezas: «Adiós, adiós\ldots» «Que vengas prontito\ldots» «Volaré.»

\hypertarget{iv}{%
\chapter{IV}\label{iv}}

Una sola persona (sin contar el viejo Ansúrez y los dueños de la casa,
calle de Rodas) poseía, por confianza de Lucila, el delicado secreto de
aquel escondite en altos desvanes: era una monja exclaustrada con quien
la linda moza tenía amistad, contraída superficialmente en el Monasterio
de Jesús, reanudada con honda cordialidad fuera de la vida religiosa. En
esta se llamó Sor María de los Remedios; su nombre de pila era
Domiciana, y había vuelto al mundo de una manera un tanto irregular, por
enferma de locura, que se estimaba incurable. El delirio que padeció
consistía en la idea fija de ahorcarse, en otras manías inocentes, pero
incompatibles con la vida de contemplación, en el furor de gritar y de
ofender cruelmente a personas eclesiásticas muy respetables, todo lo
cual determinó el designio de devolverla sin violencia ni escándalo a su
padre y hermanos para que la cuidasen, y corrigieran sus desvaríos por
el método doméstico, con paciencia, cariño y honestas distracciones.

Volvió, pues, Domiciana a su casa y al amparo de su familia, que era de
origen extremeño, establecida en Madrid, calle de Toledo, desde tiempo
inmemorial, con el negocio de cerería; y no bien tomó tierra en el hogar
paterno, acomodose lindamente al vivir secular, echando, como si
dijéramos, un nuevo carácter. Ansiaba morar con los suyos, ver gente,
ocuparse en menesteres gratos, lucidos, y de eficacia inmediata para la
vida. Pasado algún tiempo, no se mordía la lengua para decir que su
temprana inclinación religiosa no había sido más que una testarudez
infantil, nacida del odio a su madrastra, y fomentada por un sacerdote
de cortas luces, amigo de la casa. Cayó la venda de sus ojos algo tarde,
cuando ya su irreflexiva determinación no tenía remedio, y del despecho,
más aún de las ganas recónditas de libertad, le sobrevino aquel
destemple nervioso con ráfagas cerebrales, que se manifestaba en la
necesidad irresistible de correr por los claustros, en imitar con
destemplada voz los pregones callejeros, y a veces en liarse al pescuezo
una cuerda con lazo corredizo. Esto ponía la consternación y el espanto
en sus tímidas compañeras, pues aunque nunca tiraba del lazo lo bastante
para estrangularse, hacíalo hasta ponerse roja como un pimiento y echar
fuera un buen pedazo de lengua.

Lograda al fin la libertad en la forma que se ha dicho, en todo tuvo
suerte Domiciana, pues como por ensalmo se le curaron aquellas
neuróticas desazones, y entró en su casa en circunstancias felicísimas.
La madrastra que motivó su reclusión religiosa se había muerto, y casado
en cuartas nupcias el honrado cerero D. Gabino Paredes, había enviudado
por cuarta vez. No había, pues, mujer en la casa, y Domiciana podía
campar con todo el imperio que apetecía, así en la familia como en el
establecimiento. Antes de seguir, conviene dar noticia del
patriarcalismo matrimonial de aquel D. Gabino, varón inapreciable para
rehacer una comarca despoblada por la emigración. De su primer
matrimonio, que sólo duró tres años, tuvo dos hijas, que el 50 vivían:
la una era monja en Guadalajara, la otra casó con un cerero de la misma
ciudad. De la segunda mujer nacieron siete hijos, de los cuales vivían
sólo Domiciana y dos hermanos que se habían ido a América. El tercer
matrimonio dio de sí ocho vástagos, en seis partos, y el cuarto cinco.
De estas trece criaturas sólo vivían en 1850 tres varones, dos de los
cuales habían seguido la carrera eclesiástica y desempeñaba cada cual un
curato en pueblo de la Mancha: el Benjamín, llamado Ezequiel, trabajaba
en la cerería al lado de su padre, y era un bendito, todo mansedumbre y
docilidad. Había llevado al censo el buen Don Gabino cuatro mujeres y
veintidós hijos legítimos\ldots{} El censo de los naturales lo formaban
las malas lenguas del barrio.

Si afortunada fue Domiciana al encontrarse, en su regreso al mundo, sin
madrastra y con la menor cantidad posible de hermanos, no fue menos
dichoso el cerero al recobrar a una hija que pronto reveló su
extraordinaria utilidad. Pasados los primeros días, Domiciana se
reconoció continuadora de su historia personal anterior a la vida del
convento. Había sido esta como un paréntesis, como un sueño, del cual
despertaba con cierto quebranto del alma, pero sintiéndose poseedora de
cualidades que no eran menos positivas por haber dormido tanto tiempo.
No tardó en revelar su carácter mandón y autoritario: lo estrenó
desbaratando un nuevo plan casamentero de su padre, que aún se sentía,
con senil ilusión, llamado a enriquecer el censo. Andando días desplegó
en el gobierno de aquella industria dotes de administradora, y puso
puntales a la ruina. Con tantas nupcias, partos y viudeces, con
tantísimos bautizos y crianza de criaturas, y principalmente con el
desbarajuste de Don Gabino en los últimos años, la cerería no se hallaba
en estado muy floreciente. La concurrencia de establecimientos
similares, la falta de tacto y agudeza para retener a la feligresía
tradicional, y el desmayo creciente de la fe religiosa, obra del tiempo
y de la política, habían traído desorden, atrasos, dispersión de
parroquianos, deudas. A todo esto quiso Domiciana poner remedio con
firme voluntad, practicando el axioma de «principio quieren las cosas.»

En esta empresa de reparación, la ex-monja no habría encontrado el éxito
si no empleara como instrumento de autoridad un genio áspero, y fórmulas
verbales de maestro de escuela. Su padre, que al principio protestaba y
gruñía, se fue sometiendo con un espíritu de transacción parecido al
miedo; Ezequiel y el dependiente Tomás obedecían silenciosos, y al fin,
entrando grandes y chicos por el aro, todos comprendían lo saludable de
aquel método de gobierno. Subía de punto el mérito de Domiciana haciendo
estas cosas con apariencias de no hacer nada. Diez o doce meses habían
transcurrido desde su evasión, y vivía confinada en el entresuelo, sin
bajar a la tienda y taller. Los parroquianos y los amigos de casa,
clérigos en su mayor parte, que solían armar su tertulia las más de las
tardes a la vera del mostrador o en la trastienda, rara vez la veían, y
ella no se cuidaba de que formaran idea ventajosa de su regeneración
mental; antes bien le convenía que la opinión dijera y repitiera por
todo el barrio: «Sigue tocada la pobre\ldots{} aunque tranquila y sin
molestar a nadie.» Obra lenta del tiempo fue la corrección de este
juicio; al año y medio ya era público y notorio que Domiciana gozaba de
excelente salud.

Observándola en la intimidad, fácilmente se descubría en la hija del
cerero la mujer de iniciativa, de personalidad propia en su organismo
intelectual y ético. Lejos de poner toda su atención en la industria
cerera, se lanzaba con ardor a nueva granjería, partiendo de aficiones y
conocimientos experimentales adquiridos en el claustro. Procedía en esto
por imperiosa moción de su voluntad, y además por cálculo egoísta. Más
de una vez había pensado que a la muerte de D. Gabino (la cual, por ley
de Naturaleza no podía estar lejana), la parte de cerería que a cada uno
de los hijos tocase no habría de sacarles de pobres. Y como ella
anhelaba libertad y no quería vivir a expensas de sus hermanos,
procuraba labrarse con afanes de hormiga un peculio propio, que le
asegurase vejez holgada, independiente. Ved aquí por qué, sin desatender
el negocio de su padre, cultivaba en reservado laboratorio sus artes y
preparaciones propias. Trasladó la sala al despacho de D. Gabino, este a
un rincón de la tienda, tras una mampara de cristales, y en la sala
instaló lo que podríamos llamar herboristería o droguería, con unos
trozos de anaquel que compró en el Rastro, dos hornillas, mesa alta para
el filtro y pesos, y otra pequeña, por el estilo de las de los
zapateros, destinada a las manipulaciones que exigían largas horas de
atención y paciencia. Enorme cantidad de hierbas tintóreas, cosméticas u
oficinales difundían variados aromas en la estancia, ya colgadas del
techo en ramos, ya guardadas en cajoncillos. No digamos que Domiciana
cultivaba la Botánica y la Química, sino que era una profesora empírica
de arte herbolario y de alquimia doméstica.

Pocas personas veían a la monja en su retiro de alquimista, y la única
que en él a todas horas tenía entrada era Cigüela. Amistad y confianza
recíproca las unían, a pesar de la diferencia de edades. Se conocieron
en \emph{Jesús} durante tres penosos días, que fueron los últimos de
Domiciana y los primeros de Lucila en el convento, y cuando salió esta,
buscó amparo junto a la exclaustrada, que a su servicio la tuvo dos
meses largos. En la triste situación a que había venido la hija de
Ansúrez, la que fue su ama y era siempre su amiga le daba consuelos y
socorro; pero no lo hacía sin echar por delante expresiones agrias,
creyendo que la guapa moza necesitaba corrección moral tanto como
auxilios de boca, y que los buenos consejos y las lecciones dolientes
para uso de la conducta no serían menos eficaces que el chocolate o el
pan. Entró Lucila en el laboratorio, y fatigada se sentó después de un
breve y cordial saludo.

---¿Ya estás aquí otra vez?---le dijo Domiciana, que aunque se alegrara
de verla, no dejaba de emplear esta fórmula displicente.---Pues hija, ya
podías comprender que no puedo socorrerte tan a menudo\ldots{} Lo que
entra por cera no da más que para el gasto de casa. Muy deslucidas han
sido las Ánimas este año, y nadie diría que estamos en Noviembre\ldots{}
Pues el Adviento también se nos presenta muy mediano. ¿Qué tenemos
ahora? La novena de San Nicolás de Bari, que da poco de sí. La de la
Purísima será otra cosa. Ten paciencia, espérate y\ldots{}

Incapaz de formular un exordio apropiado a la pretensión que llevaba,
Lucila no hacía más que suspirar hondo, metiéndose en la boca las puntas
del pañuelo. Y Domiciana, que jugar solía con la ansiedad de las
personas que más amaba, enseñándoles el bien que pedían y guardándolo
después, dio estos puntazos, con dedo muy duro, en el dolorido corazón
de su amiga: «No se te puede favorecer todos los días. Vaya, vaya:
tenemos aquí una historia que no se acaba nunca\ldots{} ¿Pero cuándo se
muere ese hombre, o cuándo lo prenden y se lo llevan a Filipinas, para
que descanses tú y descansemos todos?»

Estas expresiones, dichas con fría crueldad, desbordaron la pena de
Lucila, que se deshizo en llanto, arrimando su cabeza a la estantería
cercana. Y la otra, cambiando el juego mortificante por el juego
compasivo, le dijo, sin abandonar su tarea: «Para, para, hija, que con
tanta llorera le metes a una el corazón en un puño. Ya sabes que no te
dejaré marchar con las manos vacías. Domiciana tiene siempre para ti las
dos, las tres onzas de chocolate, media hogaza y un par de reales de
añadidura. No lloréis más, ojuelos; sosiégate, corazón\ldots»

\hypertarget{v}{%
\chapter{V}\label{v}}

---Aunque usted se enfade, aunque usted me pegue---contestó Lucila
sacando las palabras del seno de su intensa amargura,---le digo\ldots{}
Domiciana, le digo que no he venido por la limosna que suele darme, para
un día, o para tres\ldots{} Ya sé que eso, su buen corazón no me lo
niega\ldots{} Domiciana, no vengo a eso\ldots{} Pégueme, Domiciana,
pero\ldots{} yo le digo que estoy atribuladísima\ldots{} Un miedo
horrible, un presentimiento\ldots{} Imposible guardar mucho tiempo más
el escondite de Tolomín\ldots{} Siento los pasos de la maldita
policía\ldots{} los siento aquí, en mi corazón\ldots{} ¡pum,
pum!\ldots{} ya vienen\ldots{} y si cogen al pobre Tolomín, yo,
Domiciana\ldots{} yo\ldots{} Nada; pasará una de estas tres cosas: o me
muero, o me mato\ldots{} o mato a alguien. Créalo usted: soy una leona;
pero una leona\ldots{} Figúrese una madre a la que le quitan su hijo, un
niño chiquitín\ldots{} Pues Tolomé perseguido, condenado a muerte,
herido y enfermo, es para mí como una criatura\ldots{} Hasta me parece
que le he dado la vida\ldots{} Y se la doy, sí: yo me hago cuenta de que
se muere todos los días, y que lo resucito con mis cuidados, con mis
ternuras, y con este afán grandísimo de que viva y se salve\ldots{}
Domiciana, se lo digo a usted aunque me pegue. Se me ha ocurrido sacar a
Tolomé de Madrid, ponerle en salvo, huyendo con él a Portugal o a
Francia. Vea usted lo que he pensado\ldots{} es una gran idea\ldots{}
Sí, dígame que sí, Domiciana, y dígame también que me ayudará a
salvarle, a salir de este infierno. Vivir como vivimos es peor que la
muerte\ldots{} Usted me ayudará, usted me dará lo que necesito para
hacer por ese hombre desgraciado lo que haría una madre y una hija, una
hermana y una esposa, porque todo eso junto soy y quiero ser yo para él.

---Válgate Dios por lo enamorada---dijo la ex-monja mirándola con
seriedad, en la cual no era difícil sorprender algo de
admiración.---Bueno: pues dime ahora cuál es tu plan. ¿Conoces las
dificultades de una fuga semejante? Tendréis que salir disfrazados. Y el
dinero para esa viajata, que habrá de ser en coche, ¿dónde está? ¿Has
creído que yo podré dártelo?

---Sí que podrá\ldots{} Los gastos no subirán mucho, Domiciana. Le diré
mi plan para que se vaya enterando. Lo primero ha de ser comprar un
burro\ldots{} ¿Se ríe? Todo lo tengo muy estudiadito\ldots{} Un burro
necesito, porque nos disfrazaremos de gitanos. La ropa no la tengo; pero
sé dónde está y lo que ha de costarme, que es bien poco.

---Realmente, tú no harás mal tipo de gitana; pero él\ldots{} ¿Es muy
guapo?

---Mil veces he dicho a usted que es guapísimo, Domiciana, y nunca se
entera.

---¿Pelinegro?

---Sí\ldots{} Pero los ojos son azules. Tiene tal hechizo en el
mirar---dijo Cigüela con ingenua sinceridad descriptiva,---que no puedo
explicar a usted lo que una siente cuando Tomín habla de cosas que
llegan al corazón\ldots{}

---Ya, ya---murmuró Domiciana perdida la mirada en el espacio, en
persecución de una imagen ideal, fugitiva.---Ojos azules, color
trigueño\ldots{} como nuestro Señor Jesucristo\ldots{} Bueno: pues te
digo que no haréis Tomín y tú pareja de gitanos, y no resultando el
disfraz, corréis peligro de que os sorprendan en el camino y os
maten\ldots{} Conozco la manera de dar a la tez el color
agitanado\ldots{} Para esto se emplea el sándalo rojo, mezclado con
vinagre fuerte dos veces destilado, y añadiendo alumbre de roca,
molido\ldots{} Para lo que no hay secreto de alquimia es para trocar en
negros los ojos azules\ldots{} y como saques a tu hombre con ojos azules
y vestido de gitano, cátate descubierta y él preso y pasado por las
armas.»

Desconcertada, Lucila miró a su amiga, como pidiéndole que al rebatir y
desechar una solución propusiese otra.

---Más seguro será, tontuela, que le disfraces de amolador---prosiguió
la exclaustrada.---¿No me has dicho que habla francés?

---Sí: lo hablaba de niño, y aún le queda el acento. Su madre era
francesa; se apellidaba Chenier. Él dice que por el nombre materno tiene
la revolución en la sangre.

---Pues el habla francesa se apareja muy bien con los ojos azules,
siempre que el pelo sea rubio. Aquí tengo yo la lejía para teñir de
rubio los cabellos---dijo Domiciana mostrándole un frasco que contenía
sustancia opaca.---Sé hacerla, y surto a dos señoras morenas que quieren
ser rubias. Tomo dos libras de ceniza de sarmientos, media onza de raíz
de brionia y otro tanto de azafrán de Indias; le añado una dracma de
raíz de lirio, otra de flor de gordolobo, otra de estaquey amarillo; lo
cuezo, lo decanto, y ya está. Lavando el pelo de Tomín seis o siete
veces, se lo pondrás rubio como el oro; le afeitas para no tener que
pintar la barba y bigote, y con esto y un poco de francés chapurrado, ya
le tienes de perfecto amolador. Por poco precio, puedes proporcionarte
la piedra de asperón y todo el aparato. Toma tu hombre unas lecciones de
ese oficio, y salís por esos pueblos, él amolando y tú tocando el chiflo
para pregonar la industria\ldots{}

---Tomín no puede afilar por causa de la herida en la pierna---dijo
Cigüela reflexiva, argumentando en contra, pero sin rechazar en absoluto
la tesis amolatoria.---Gracias que se tenga en el burro, y que podamos
caminar en jornadas cortas. Yo he de ir a pie, arreando\ldots{} Además,
los afiladores son mal mirados en los pueblos, y si diera la gente en,
creer que llevamos algunos cuartos, nos haría alguna mala
partida\ldots{} Si él estuviera bueno, y pudiera, de pueblo en pueblo,
amolar de verdad, cobrando poco, escaparíamos bien\ldots{} Desde luego
es mejor idea que la de agitanarnos. Pero de seguro habrá un tapadizo
más seguro. Búsquelo, invéntelo, usted que discurre tan bien y tiene la
cabeza fresca. La mía es un horno, y no saco de ella más que
disparates.»

Cambió el rostro de Domiciana, recobrando la orgullosa expresión de
confianza en sí misma y de sábelo-todo. «Pues solución verdadera y
segura no hay más que una, Lucila---le dijo levantándose,---y vas a
saberla\ldots{} Pero como la cosa es larga y tenemos que hablar mucho,
bueno será que te quedes aquí toda la tarde\ldots{} Ya no tienes que
\emph{correr} tras la pitanza, porque asegurada la tienes por mí. En
pago de ella y del consejo que voy a darte para tu salvación y la de ese
caballero, me ayudarás en mis tareas. Quítate el pañuelo de manta; ponte
este delantal, siéntate delante de mí, coge el almirez, y entretente en
moler estas dos onzas de almendras amargas, que ya están peladas, y una
dosis de alcanfor, que voy a darte bien medida\ldots{} Has de moler
hasta que estén unidas las dos materias y formando una pasta\ldots{} Yo
prepararé un frasco de \emph{Leche de rosa}, que me han encargado para
hoy mismo\ldots{} Trabajemos aquí las dos, y hablemos. Cuenta te tiene
oírme, y más cuenta reflexionar en lo que me oigas.»

Hizo Lucila cuanto Domiciana le ordenaba, y calló esperando la solución
y consejo, no sin temor y ansiedad grandes, pues siempre que su amiga
hablaba en aquella forma, era para proponer actos difíciles, si por un
lado saludables, por otro dolorosos. Un rato estuvo la ex-monja
trasteando junto a una credencia de la cual sacó botellas y tazas con
diferentes líquidos. Después, sin hablar palabra, por tratarse de una
mixtura que reclamaba toda su atención, midió diferentes porciones, ya
con cucharillas, ya con cazos; coló el aceite de oliva, le añadió gotas
de aceite de tártago, y cuando su labor parecía vencida en su parte más
delicada, dijo a su amiga: «Esta es la \emph{Leche de rosa}, que hago
con todo escrúpulo y sin omitir gasto, para una señora Marquesa que la
emplea como lo mejor que se conoce para la conservación de la tez. Con
eso que tú mueles hago el jabón de tocador que llamamos \emph{de lady
Derby}, cosa rica, y por tanto un poquito cara. Te daré lección, si
quieres; podrás hacer la \emph{Pasta de almendras} para blanquear las
manos, y el \emph{Agua de carne de ternero para calmar los picores de la
piel}\ldots{} Con todo esto bien preparado y bien servido a los que
saben y pueden pagarlo, se gana dinero, y se combate la ociosidad, que
es la madre de todos los vicios\ldots»

Hizo los últimos trasiegos, se lavó las manos, y parándose con los
brazos en jarras junto a Lucila, la contempló risueña, y aprobó con
monosílabos expresivos su trabajo. La infeliz moza majaba en el almirez
con fe y aplicación, acompañando el movimiento de la mano con hociquitos
muy monos, sin apartar del fondo del mortero su atención sostenida.
«¡Qué bien va eso, Lucila! Cuando lo acabes, te pondré a majar, en
distinto mortero, jibiones, ladrillo rojo y palo de Rodas con otros
ingredientes, para tamizarlo y hacer \emph{Polvo de coral}\ldots»

Era Domiciana de mediana estatura, bien dotada de carnes, airosa de
cuerpo, desapacible de rostro, descolorida, ojerosa, negros los ojos, la
ceja fuerte y casi corrida. Si de media nariz para podría su cara
pretender la nota de hermosura, del mismo punto hacia abajo ganaría
fácilmente el premio de fealdad por la nariz un tanto aplastada y la
conformación morruda de la boca, de labio gordo tirando a belfo. No era
fácil designar su edad por lo que de ella se veía: declaraba treinta y
ocho años. De la vida claustral le habían quedado los ademanes y
compostura señoril, en visita o ante personas extrañas, y el habla fina,
correcta, en muchas ocasiones atildada. Quedábale también la costumbre
de expresar su pensamiento graduando la sinceridad por dracmas y hasta
por escrúpulos, según le convenía. Entre lo adquirido al reaparecer en
el mundo, se notaba la asimilación de algunas voces nuevas de reciente
uso social y callejero, y el cuidado de la dentadura, buena por sí y
mejorada con la \emph{Lejía jabonosa} y los \emph{Polvos de coral}. Era
un excelente muestrario de su industria. Continuaba vistiendo
modestamente de negro. En visita, nunca se desmintió la monja encogida
que por graves motivos de salud había tenido que volver a la casa
paterna, y su conversación copiaba el prontuario de todas las muletillas
de respeto para cosas y personas, así humanas como de tejas . Su voz no
era gangosa, sino bien timbrada y de variadas inflexiones. Lo más bello
de su cuerpo eran brazos y manos.

Pues como se ha dicho, Lucila machacaba en silencio, aguardando la
ansiada solución, que la maestra no quería soltar sin preámbulos.
Sentose Domiciana junto a la mesa que parecía de zapatero, frente al
sitio que ocupaba Lucila, y se puso a dividir en pequeñas dosis, medidas
con una conchita, ciertas cantidades de polvo de rosa, de iris en polvo,
de goma molida, y a guardarlas en papelillos doblados a lo boticario.
Luego formó dosis más grandes de nitro, de estoraque, de clavillo y
canela, midiendo con cáscaras de nuez, y cuando estaba en lo más
empeñado de su trajín rompió el silencio con estas palabras, que
resultaron solemnes: «Si quieres salir pronto y bien de esa terrible
situación, y salvar a tu hombre y salvarte tú, en tu mano está. El
camino es corto, Lucila. No hace falta más que un poco de resolución
y\ldots{} Fuera miedo, fuera escrúpulos. Te vas al convento, pides ver a
la \emph{Madre}; la \emph{Madre} te recibirá gozosa; te armas de valor,
le cuentas tus penas; la \emph{Madre} te oye como ella sabe oír; tú
lloras un poquito, naturalmente: la \emph{Madre} te consuela, te anima;
le dices toda la verdad, todita, Lucila: quién es ese hombre, lo que ha
hecho, la crueldad con que es perseguido\ldots{} y para que no se te
quede nada por decir, le cuentas cómo le conociste; haces la pintura
de\ldots{} de\ldots{} lo guapo que es, del amor que le tienes, y\ldots{}
Hija, como hagas esto, según yo te lo digo, ten a tu Tomín por
salvado\ldots»

Lucila estupefacta, suspensa, miraba a su amiga como si dudara de lo que
oía. Los morros de Domiciana, al soltar la palabra, le hacían el efecto
de una trompeta de son estridente, desgarrador.

\hypertarget{vi}{%
\chapter{VI}\label{vi}}

---¡Pero usted se burla, Domiciana!---le dijo al fin Lucila cuando el
estupor dio paso a la expresión clara del pensamiento.---¿En serio me
aconseja que le cuente esto a la \emph{Madre} y le pida su protección?

---Seriamente te lo digo\ldots{} y tan cierto tendrás su divina
protección como este es día. Yo la conozco bien. Por grande que sea la
culpa de Tomín, si le pides a la \emph{Madre} el indulto, lo
tendrás\ldots{} Tus planes de escapatoria son desatinados. Si no vas por
el camino que te marco, tú y tu capitán estáis perdidos\ldots{} Fuera de
este camino, no veas más que la muerte\ldots{} ¡y qué muerte,
pobrecilla!

---¡Ay, Domiciana: de una amiga como usted, que me quiere de veras, no
esperaba yo ese consejo!---exclamó Cigüela triste, dolorida.

---¿Dudas que la \emph{Madre} pueda sacarte de ese Purgatorio? El poder
de la \emph{Madre} es tal, que con escribir su voluntad en un papelito y
mandarlo a donde guisan, hace y deshace los acontecimientos, así en lo
grande como en lo chico. Y diciendo ella «esto quiero» no valen para
impedirlo todos los Narváez del mundo con sus bufidos de mal genio, ni
la caterva de monigotes viles que llaman Ministros, los cuales no son
más que refrendadores de lo que manda\ldots{} quien manda. Ya tú me
entiendes. Como la \emph{Madre} diga: «Sobreséase la causa del Sr.
Tomín, y désele encima jamón en dulce,» ya puede estar tranquilo tu
amigo\ldots{} Los que hoy le persiguen, le ayudarán a ponerse las botas
para que se vaya a su casa, y luego, cuando le vean paseándose libre por
la calle, le harán mil carantoñas.

---Creo en el poder de la \emph{Madre}---dijo Lucila,---creo también que
sirve, pero no de balde. Si concede un favor a tal o cual persona, es a
cambio de otro favor, o de que la adoren como a los santos. Nadie me lo
cuenta, Domiciana; lo he probado por mí misma. Cuando empezó este
martirio mío, no sabiendo a quien volverme, fui al convento a pedir
protección. La \emph{Madre} no quiso recibirme. Sor Catalina, que
siempre fue conmigo muy cariñosa, me dijo que si quería protección para
mí, o para persona que me interesara, debía pedirla de rodillas con
todas las señales del arrepentimiento, renegando de mi libertad,
dejándome encerrar y corregir con remuchísimo aquel de severidad\ldots{}
Buena cosa querían: cogerme, arrancarme el corazón que tengo, y ponerme
otro de papel para que con él sintiera lo que ellas sienten:
nada\ldots{} la muerte\ldots{} ¡Y por casa un sepulcro, y por ocupación
el aburrimiento!\ldots{} Esto no me conviene, esto no es para mí.

---Pero, Lucila---dijo la otra apoderándose de un argumento que creía de
grande eficacia,---¿tú crees que en este mundo se logran nuestros deseos
sin algo de sacrificio? ¿Querías tú que la \emph{Madre} te salvara al
hombre por tu linda cara, dejándote en libertad para seguir ofendiendo a
Dios?\ldots{} Ponte en lo razonable, y no esperes que te saquen de este
pantano sin que digas: «A cambio de la vida y de la libertad de ese
hombre, ahí va la libertad mía, ahí va mi amor; doy también mi vida: a
Dios me ofrezco toda entera para que Dios, por mediación de sus
ministros\ldots{} o ministras, devuelva la paz a un desgraciado.» Esto
es lo meritorio, esto es lo cristiano.

---Eso\ldots---dijo Lucila desdeñosa, disimulando su enojo con una
violenta presión de la mano de mortero sobre la pasta,---eso se lo
cuenta usted a quien quiera. Lo cristiano es favorecer al prójimo sin
pedirle nada.

---Veo que no tienes pizca de trastienda, Lucila; por eso eres tan
desgraciada, y lo serás siempre. Si llevas al convento tus cuitas y las
cuitas del caballero de los ojos azules, ¿qué ha de pedirte la
\emph{Madre} a trueque de la salvación del sujeto? Pues nada entre dos
platos. Te darán cama y comida; te mandarán que confieses, no una vez,
sino muchas. Ningún trabajo te cuesta confesar, ni el confesar a menudo
con las penitencias consiguientes es para matar a nadie. Te sometes, te
santificas, sufres un poquito, trabajas, rezas. De tu aburrimiento y
soledad te consuelas pensando que el caballero está en salvo, que la
policía no se mete con él, que le dan el ascenso, y vive bueno y sano,
engordando y poniéndose cada día más guapetón.

---Domiciana---dijo Lucila traspasando a su amiga con la mirada,---o es
usted una hipócrita y me recomienda la hipocresía, o es la mujer sin
corazón, la mujer muerta, que así llamo a las que se han dejado secar y
amojamar en los conventos, convirtiéndose en animales disecados como los
que están en la Historia Natural. Cuando la conocí a usted en
\emph{Jesús}, la tuve yo por mujer viva; pero ahora me habla como las
muertas. No sabe lo que es amor, no tiene idea de él; tiene el corazón
hecho cecina, y con la uña me ha desgarrado el mío, que vive y
sangra\ldots{} Domiciana, no sea usted cruel, no me martirice\ldots{}

---Tontuela, yo seré todo lo marchita que tú quieras; pero sé discurrir
y veo las cosas con claridad---replicó Domiciana ansiosa de
mortificarla.---Para que te salven al caballero ese, tienes que
renunciar a él, ser mujer muerta. ¿Pues qué quieres, niña? ¿Que la
religión te saque de este mal paso y encima te dé cabello de ángel y
tocino del cielo? No puede ser. Si quieres que él viva, es preciso que
tú te amojames\ldots{} Ya sé yo lo que temes\ldots{} Aunque desconozco
el amor, ¡maldito amor!, he calado lo que piensas. Tú dices: «¡Pues
estaría bueno que mientras yo me estoy aquí, reza que te reza y
secándome y acecinándome, mi Tomín, salvado por mí, ande por esos mundos
divirtiéndose con otra!» ¿Acierto?

---Eso he pensado, sí. No quiero, no, venderme a las monjas por la
salvación de Tomín.

---Pues mira tú: hay un medio de conciliarlo todo. Te vas a
\emph{Jesús}\ldots{} haces tu trato con la \emph{Madre}; te encierras,
te dejas disciplinar y penitenciar todo lo que quieran\ldots{} siempre
con la reserva mental de volver a escaparte cuando estés bien segura de
que Tomín está en salvo\ldots{}

---¡Hipócrita, más que hipócrita!\ldots{} ¿Y cuánto duraría esa comedia?

---Poco tiempo\ldots{} quince días, un mes\ldots{} ¿No tienes confianza
en tu Tomín? ¿Dudas que te guarde fidelidad en plazo tan corto?\ldots{}
Si lo dudas, ponle bajo mi custodia en este tiempo. Yo, como mujer
muerta y corazón convertido en bacalao, no debo infundirte celos. Yo
seré para él como una madre, como una hermana mayor, y le trataré a la
baqueta, no le dejaré respirar, leyéndole a todas horas la cartilla:
«Eh, caballerito, ándese con tiento, que si antes estuvo condenado a
muerte, ahora está condenado a fidelidad y gratitud, bajo mi vigilancia.
Para salvarle a usted se puso en esclavitud, digamos en rehenes, con
Dios, una mujer de tierno corazón. Si usted cumple como caballero,
guardándole consecuencia, ella cumplirá como señora, escabulléndose
lindamente de su prisión, y así volverán una y otro a juntarse.» Esto le
diré, y con mis exhortaciones y el cuidado que he de poner en vigilarle
y seguirle los pasos, te le tendré bien sujeto\ldots{} ¿Qué?\ldots{} ¿te
ríes? ¿No te parece sutil esta combinación?

---Demasiado sutil\ldots---contestó Lucila con graciosa desconfianza.

---¿No me tienes por buena guardiana?

---No me fío\ldots»

La monja ladina alargaba los morros afectando toda la seriedad del
mundo. Mirábala Lucila entre burlona y asustada. En sus labios oscilaba
ese mohín del niño, que no sabe si reír porque le entretienen o llorar
porque le asustan. Y repitió la frase: «No me fío\ldots» Tras una pausa
en la cual Domiciana frunció su tenebroso entrecejo y dio a los morros
toda la longitud posible, Cigüela, casi casi compungida, volvió a decir:
«No me fío, Domiciana.»

---Pues si soy mujer muerta y corazón disecado, ¿qué temes?

---Por si acaso, Domiciana, por si acaso no fuera usted como yo
creo\ldots{}

---¿Esta combinación no te peta? Peor para ti\ldots{} porque no hay
otra, Lucila.

---Si para que la \emph{Madre} me favorezca necesito engañarla, y birlar
a la Comunidad, me quedo donde estoy. ¡Pobre Tomín!\ldots{} Moriremos
juntos.

---Sí, sí: a eso vais.

---Ya me dio un vuelco el corazón cuando usted nombró a la \emph{Madre}.
Desde el día en que allí estuve y me despidió Sor Catalina con las
despachaderas que usted sabe, no he vuelto a parar mientes en aquella
casa. Por la \emph{Madre} siento respeto; pero nada más que
respeto\ldots{} Cierto que no es una mujer como las naturales\ldots{}
Algo hay en ella que es\ldots{} de ella nada más; pero nunca he podido
quererla\ldots{}

---Yo sí---dijo Domiciana con firme acento; y la vaguedad de su mirada,
perdida y parada como la de los ciegos, indicaba que su mente perseguía
las imágenes distantes.

---¿De veras la quiere? Será porque ha sido buena para usted. ¿Y cree
usted en las llagas?

---¿Cómo he de creer en las llagas, si sé cómo se hacen? Alguna vez ha
recurrido a mí para que se las reprodujera cuando se le estaban
cicatrizando. Tengo el secreto: la misma monja que reveló a Patrocinio
este artificio me lo enseñó a mí, una vieja que murió cuando aún
estábamos en el Caballero de Gracia: Sor Aquilina de la Transfiguración,
aragonesa ella. Pues sí: sé hacer llagas. Ello es bien fácil. Tengo la
\emph{clemátide vitalba}, que el vulgo llama \emph{yerba pordiosera}.
¿Quieres probarlo? Verás qué pronto\ldots{}

---No, gracias. No me llama Dios por ese camino.

---Ni a mí. Por eso jamás me pasó por la cabeza llagarme a mí
misma\ldots{} Las razones que ha tenido Patrocinio para ponerse los
estigmas son de un orden superior, y no debemos meternos a decir si hace
bien o hace mal\ldots{} Lo que en ti o en mí, que somos tan poco y no
valemos para nada, sería bárbaro, pecaminoso, y hasta sacrílego, en
otras personas, llamadas a empresas altas por méritos de su caletre y de
su voluntad, puede ser bueno, necesario y hasta indispensable. ¿Qué
dices? ¿Que no entiendes esto, bobilla?

---Yo, Domiciana, pienso siempre por derecho: creo que lo que es malo en
mí, malo ha de ser en las reinas y emperatrices.

---No estamos conformes. Eres una simplona y no conoces el mundo. Corto
tiempo has estado en el Convento, y eso en días en que allí había poco
que aprender. Veinte años, los mejores de mi vida, pasé yo en la
Comunidad, y en tiempos tales, que entonces fue la casa como un pequeño
mundo, dentro del cual el mundo grande de nuestra España estaba como
reproducido y encerrado. ¿Me entiendes? Pues yo, por lo que allí he
visto, puedo dar fe de las grandes dotes y facultades que el Señor
concedió a Patrocinio. No hay mujer como ella. Yo la admiro, por muchas
razones; por otras la temo\ldots{}

---Y por otras la quiere\ldots{} ha dicho usted que la quiere.

---Y no me vuelvo atrás. Para que te hagas cargo de las razones de este
querer mío, así como del admirar y del temer, será preciso que yo te
cuente muchas cosas\ldots{} ¿No te parece que ya hemos trabajado
bastante?

---Yo, la verdad, no estoy cansada. Deme otra cosa que majar.

---Antes descansemos y merendemos. Hagamos un alto en nuestros afanes
para cobrar fuerzas\ldots{} No podrás negarme que estás
desfallecida\ldots{} Se te abre la boca y se te caen los párpados.
Recógeme todo eso\ldots{} No: yo lo recogeré mientras tú bajas a la
calle, y te traes dos pares de bartolillos de la pastelería de Cosme.
Toma los cuartos. Mejor será que traigas media docena: los remojaremos
con un rosolí exquisito que me mandaron los de la botillería de la
Lechuga, para reparo del estómago en las mañanas y en las tardes
frías\ldots»

Salió la moza diligente, y en el rato que estuvo fuera, recogió la
ex-monja los ingredientes que en la mesa de trabajo había, ordenándolo
todo en otro sitio. Después sacó de un estante la botella de rosolí, y
dos copas. Al salir Lucila por los bartolillos, había reparado Domiciana
en los rojos zapatos puntiagudos que calzaba su amiga, y cuando la vio
entrar fijó más en ellos su atención, diciendo: «Has de contarme de
dónde sacaste esos chapines tan majos, y luego trataremos de que me los
des a cambio de otro calzado, porque te aseguro que me gustan muchísimo,
y quiero ponérmelos y usarlos dentro de casa.» Contestó Lucila que
dispusiese de aquella prenda y de cuanto ella poseía, y acto continuo se
sentaron y cada cual la emprendió con un bartolillo, Domiciana como
golosa y Lucila como hambrienta.

\hypertarget{vii}{%
\chapter{VII}\label{vii}}

Sirviendo a su amiga el dulce rosolí, e invitándola a no ser demasiado
melindrosa en el beber, la exclaustrada dio principio con desordenado
plan y gracioso estilo a sus cuentos monjiles: «Yo entré en el Convento
cuando aquel mal hombre y peor Rey Fernando casó con Cristina\ldots{}
no: cuando ya estaban casados, y Cristina encinta de Isabel. Me movió a
ser monja una tema de chiquilla tonta y cabezuda, y el odio a mi
madrastra, Faustina Baranda, de esa familia de peleteros establecida en
la calle Mayor, y cinco años estuve en aquella vida boba sin percatarme
del gran desatino que había hecho. Fue mi madrina en la profesión Doña
Victorina Sarmiento de Silva, dama de la Infanta Carlota\ldots{} Pues
como te digo, caí de mi burro a poco de tomar el hábito y cuando ya mi
locura no tenía remedio. De novicia, vi los primeros milagros de
Patrocinio, que en el siglo se llamó Dolores Quiroga y Cacopardo, y las
entradas del Demonio en nuestra santa casa\ldots{} Terribles dudas tuve
al principio; pero como ya entonces era yo muy reparona y todo lo
observaba, llegando hasta no creer en ningún fantasma que no viese con
mis ojos y tocara con mis manos, pronto me convencí de que el diablo
intruso y visitante era un fraile de Sigüenza, que entraba por las
habitaciones del Vicario y a los tejados se subía, y a los claustros y
celdas bajaba. Otra novicia y yo, las dos valientes y decididas, le
acechamos una noche, y corriendo tras él y agarrándole por donde
pudimos, yo me quedé con un pedazo de rabo en la mano, el cual era como
una cuerda forrada en bayeta roja, y mi amiga le arrancó un cuerno, que
resultó ser al modo de un gordo chorizo de sarga verde, relleno de
pelote\ldots{} Como se confunden en mi cabeza los recuerdos y no puedo
fijar bien el orden de los sucedidos, te diré que antes o después de
aquellas visitas infernales recibía nuestra Comunidad en el locutorio
las de D. Carlos María Isidro y su mujer Doña Francisca, y con ellas las
de innumerables señorones del bando absolutista, que era el de nuestra
devoción. En clausura entraban cuando querían un capuchino llamado el
Padre Alcaraz, el Padre la Hoz, que a muchas de nosotras confesaba, Fray
Cirilo de Alameda y otros del mismo fuste. El Padre Arriaza, que luego
nos pusieron de Vicario, no creía en la santidad de Patrocinio, y tuvo
con ella y con la Priora no pocos altercados. Nosotras, acechando fuera
de la puerta de la celda prioral, oíamos el run run de las voces, y
luego veíamos salir a la Priora sofocada, a Patrocinio fresca y
sonriente, desafiando al mundo entero con aquella serenidad que nos
llenaba de admiración.

»Que todas allí éramos carlinas furiosas, no tengo por qué decírtelo.
Adorábamos a D. Carlos, y aunque en Patrocinio veíamos actos de la mayor
extravagancia, creíamos en ella, por aquel don magnético que tenía y
tiene para imponer sus ideas, sus propósitos y hasta sus milagros.
Podían ser falsas las llagas, pero las reverenciábamos; podía ser
impostora la llagada, pero embargaba los ánimos con la blancura de su
rostro y con su voz meliflua, con aquel modito suave de decir las cosas
y de hacerlas, con aquel amor verdadero o falso que a todas mostraba, y
al cual correspondían nuestros corazones, tan necesitados de un querer
entrañable en vida de tanto hastío y soledad\ldots{} La queríamos,
Lucila, porque cuando una es monja, no se satisface con el amor de los
santos o santas de palo, y quiere santos vivos, sean como fueren.
Patrocinio, mujer extraordinaria, tuvo el arte y el valor de hacerse
santa viva: de este modo conquistó el afecto de sus hermanas, y de
muchas personas de fuera que la visitaban con admiración, con fervor,
con todo el sentimiento místico que el alma guarda y acaricia para
emplearlo en lo primero que salga\ldots{} ¡Pues no te quiero decir lo
que nos maravilló el caso de desaparecerse Patrocinio sin que en la casa
quedara rastro de ella, y aparecerse luego a horcajadas en el tejado,
con el rostro tan bien encendido en un divino resplandor que parecía una
celestial visión!\ldots{} Bajada de aquel lugar eminente, y después de
ponerse a orar nos contaba que, arrebatada por el Demonio en una nube
densa, fue conducida al camino de Aranjuez, y del camino al Palacio del
Real Sitio, donde había visto con sus propios ojos a la Reina María
Cristina en tal descompostura de ademanes, que con ella bastaba para
tenerla por malísima mujer\ldots{} que luego la transportaba el mismo
diablete a la Sierra de Guadarrama y al Real Sitio de San Ildefonso, y
allí veía y comprobaba que Isabel no podía ser Reina de España; por fin,
después de otras milagrosas visiones y avisos, en demostración de que D.
Carlos ceñiría la corona, el Demonio nos traía de nuevo a nuestra
compañera montadita en la nube, y nos la ponía en el tejado, no sin
algún quebranto de huesos de la monja volandera\ldots{} ¡Habías de ver
su cara y sus modos cuando nos contaba tales prodigios! Yo, sin
creerlos, me dejaba vencer de no sé qué respeto al arte superior y nunca
visto de tal mujer, y hacía coro a las alabanzas, a los regocijos, a las
esperanzas de mis compañeras, que veían en todo ello días gloriosos para
la Orden.

»Patrocinio, cuando no estaba en oración, se pasaba las horas en su
celda escribiendo cartas. Llevaba larga correspondencia con personas
desconocidas de fuera, que la tenían al tanto de todas las intrigas y
diabluras masónicas\ldots{} Pero un día vino el Demonio, por cierto todo
vestido como un oso, y arrebatándole los papeles, salió, dejando tal
peste de azufre que no podíamos respirar. ¿Era este diablo el mismo que
se la llevó en una nube a los Reales Sitios? Yo entonces nada sabía.
Después entendí que el segundo Lucifer era el Padre Alcaraz, que había
reñido con el Demonio de marras; supe también que el viaje no había sido
por los aires, sino por tierra, y no a los Sitios Reales, sino al
convento de Cuéllar, donde desterrado estaba el frailón de Sigüenza,
confesor que fue de Patrocinio. Bien podíamos decir: riñen los diablos y
se descubren los hurtos.

»Pues ahora daré un brinco en el relato: tengo que decir lo primero que
me salta a la memoria. Si no es por la traición de Maroto, no habría
quien le quitara la corona a D. Carlos\ldots{} Patrocinio, mujer de gran
pesquis, en cuanto tuvo noticia del convenio de Vergara, empezó a
entenderse con los diablos cristinos, y con los \emph{angélicos} o
\emph{isabelistas}\ldots{} Mucho antes de estos días\ldots{} y ahora doy
otro brinco para atrás\ldots{} empecé yo a sentir en mí el hastío y la
repugnancia de la vida monástica; y de tal modo se me iba sentando en el
alma el desconsuelo, que no tenía un rato de paz; perdí la salud, y me
entraron las murrias más horrorosas que puedes figurarte. Y es que como
había visto tantos diablos que entraban y salían, y a más de los
diablos, diabluras tantas dentro y fuera de la casa, me sentí también un
poco diabla, y harta de convento, no vi mejor remedio que las diabluras
para salirme de él\ldots{}

»Déjame que pegue ahora otro brinco, no sé si hacia delante o hacia
atrás, porque el encadenado de las cosas en el tiempo se me borra de la
cabeza\ldots{} Por aquellos días empecé a sufrir los achaquillos de no
dormir, de querer pegar a dos monjas que solían hacerme burla, y el
irresistible deseo de clavarle un alfiler gordo en la nalga a la Hermana
que tenía más próxima. Cuando me entraba el mal, o daba satisfacción al
antojo, o me entraban unos vapores que me ponían a morir\ldots{} Y
cátate aquí a Patrocinio procesada. Después de tanto absolutismo,
vinieron al poder progresistas masónicos, y la emprendieron con nuestra
santa. Del disgusto que a todas nos causaron aquellas trapisondas (y el
proceso fue por los papeles que le robó el maldito diablo), yo me puse
peor; me entró una tristeza tal, que en ella me hubiera consumido si no
quisiera Dios enviarme una distracción, un consuelo, con que me fui
recobrando, y al fin se me fortaleció el seso y me volvieron las ganas
de vivir. Desde los primeros años de la vida claustral solía
entretenerme cogiendo hierbas en la huerta, aprendiendo a distinguirlas
y a conocer sus cualidades y virtudes. Esta, en cocimiento, es buena
para las muelas; aquella, en infusión, inspira pensamientos alegres; tal
otra, purga a los pájaros; cuál otra, blanquea y afina las manos.

»Y ahora otro saltito. Cuando el tribunal masónico dispuso que, para
observar a Patrocinio y ver si eran verdaderas o fingidas sus llagas, la
trasladasen del Convento a una vivienda particular; cuando fue llevada
nuestra santa a la casa de D. Wenceslao Gaviña, en la calle de la
Almudena, y de allí a las Recogidas, se ordenó también desocupar el
Convento del Caballero de Gracia. Al de la Latina nos mandaron, donde
por ser la huerta muy chica y pobre de vegetación, no encontré el solaz
que me daba la vida, y tan mala me puse, que medio muerta me despacharon
para Torrelaguna. ¡Oh! allí fueron mis delicias, porque a más de
encontrar abundancia de toda la maravilla vegetal que derramó Dios por
el mundo, también me deparó su Divina Majestad a Sor Facunda de los
Desamparados, valenciana, que es la primera sabedora del mundo en
achaque de hierbas y sus virtudes, y sobre la ciencia y experiencia,
poseía una divina claridad para dar razón de todo. Allí mis goces de
hortelana, de herbolaria y de farmacéutica fueron tan vivos, que hasta
las obligaciones religiosas se me olvidaban, y más de una vez me
reprendió y castigó la Priora\ldots{} Pero yo lo llevaba con paciencia;
no se me ocurría clavar alfileres gordos en las caderas de nadie, y me
sentía fuerte, rebosando salud\ldots{}

»Y con tu permiso, pego aquí otro salto, en el espacio más que en el
tiempo. Viéndome repuesta me llevaron a Madrid. ¡Adiós mi Sor Facunda
del alma, adiós alegría de mi huerta y de mis queridísimos hierbatos!
¡Oh, qué tristeza me causó Madrid! En el tiempo de mi feliz residencia
en Torrelaguna, habían ocurrido muchas cosas: cambio de personal y aun
de casa, porque ya la Comunidad no estaba en la Latina, sino en
Jesús\ldots{} Por cambiar, hasta la política era otra, pues los carlinos
figuraban poco, y eran amos de España los isabelinos con su Reina
imperante. Sor Pilar Barcones, ancianita, seguía de Priora; pero la que
nos gobernaba realmente era Patrocinio, maestra y madre de todas
nosotras. Con satisfacción y orgullo veíamos el sin fin de personajes
que iban a platicar con ella. El señor Infante Don Francisco presentó a
su hijo, ya Rey o marido de la Reina; este llevó a su esposa, y tras
estos egregios visitantes, iban Duques, Condes y Marqueses con sus
mujeres y otras que no lo eran\ldots{} Jubileo más lucido no se vio
nunca. Patrocinio, a mi regreso de Torrelaguna, me pareció una figura
enteramente celestial. ¡Qué blancura de tez, qué caída de ojos, qué
majestad en las posturas, y qué modito de hablar echando las palabras
como si fueran ecos de otras que sobre ella en invisibles aposentos se
pronunciaran! Comprendí entonces su poder, y que Reina y Rey se
postraran ante ella\ldots{} Tan mística era su hermosura, tan soberanos
sus modos de andar, de sonreír, de llamar a una de nosotras para que se
acercase, y tan dulce el timbre de su voz, que causaba en los que la
veían y oían por primera vez efecto semejante al de la presencia de un
ser sobrenatural. Te contaré un caso para que te maravilles. Cuando la
llevaron a las Magdalenas, una monja de fe muy viva, que había oído
contar sus milagros y creía en ellos como yo creo en la luz del sol, en
cuanto la vio quedose como pasmada; se le doblaron las rodillas; el
rostro de Patrocinio fue para ella como un conjunto de la claridad de
todos los rayos y centellas del cielo\ldots{} La pobre monja dijo: «¡Ay,
Jesús!», y se quedó ciega.»

---Pues esa era la ocasión---dijo Lucila prontamente,---de probar la
\emph{Madre} su santidad, porque debió llegarse a la pobre monja, y
ponerle la santa mano en los ojos y decir con arrebato: «Ojos engañados,
en nombre de Dios os mando que veáis.»

---Algo de eso hizo Patros; pero no consta que la otra recobrara la
vista, y sólo al cabo de unos días empezó a ver algo por el ojo derecho,
quedándose con el izquierdo a obscuras\ldots{} En fin, yo te cuento el
prodigio como me lo contaron, y lo que haya de verdad ya lo dirán las
escrituras\ldots{} Pues sigo: si me fue muy grato ver que la
\emph{Madre} me tomaba cariño, por otra parte me causó un dolor muy
acerbo cierto día, diciéndome que moderara mi afición a la botánica y a
la composición de menjurges caseros\ldots{} así lo llamaba, con
desprecio de cosa tan útil como aquel arte mío mal aprendido. Ya ves: yo
que no me había puesto tasa en la admiración de ella, ya la temía tanto
como la admiraba\ldots{} Disimulé un poco mis aficiones, que cada día se
apoderaban más de mi pobre alma sepultada en aquella región del
fastidio. Hablando yo conmigo misma o con Dios en la soledad de mi
celda, me comparaba con Patrocinio; llegaba a creerme que tenía delante
de mí su rostro blanquísimo, sus ojos que ven los pensamientos, sus
manos de cera con los estigmas de las llagas, sombrajo entre rosado y
verdoso\ldots{} Y viéndola de presencia, como hechura de mi imaginación,
le decía: Tú haces milagros, y yo combinaciones naturales, que son los
milagros de la tierra; tú trabajas con las cosas que están por encima de
las nubes, con lo invisible y espiritual; yo trabajo con plantas
humildes que tú pisas creyéndolas cosa despreciable. De estas plantas
extraigo zumos, de otras aprovecho las flores, las raíces, las cortezas,
y preparo bebidas medicinales, ingredientes que sirvan para realzar la
hermosura, o para mil usos y aplicaciones útiles de la vida que, por ser
tantas, no se pueden contar. Tú haces tus arrumacos y tu arte de los
cielos para dominar a las criaturas y someterlas a tu mando, para ayudar
o estorbar a Reyes y Ministros en el mangoneo de la dominación, o en
guiar a ese ganado hombruno que, como el ovejuno y el vacuno, se deja
llevar por el miedo o por el engaño. Yo no aspiro a gobernar a nadie,
sino a ser útil a unos cuantos, y a emplear mis días en un trabajo
modesto que a mí me sostenga y me dé mejor y más cómoda vida. Tú
manipulas con lo divino, yo con la Naturaleza, y en mis milagros no
entran para nada el Dogma, ni la Pragmática Sanción, ni la Legitimidad;
no entran más que las hierbas de Dios, el agüita de Dios, y el fueguito
de Dios\ldots{}

»Esto le decía yo en mis pláticas solitarias, y aun creo (no puedo
asegurarlo) que se lo dije de palabra viva, frente a frente, en alguna
de las agarradas que tuvimos cuando me llamaba a su celda para
reprenderme.»

\hypertarget{viii}{%
\chapter{VIII}\label{viii}}

Por segunda o tercera vez escanció rosolí en las dos copas, y pasando
por el gaznate un buche de agua para aclarar la voz, prosiguió de este
modo: «De entonces, y digo entonces por no poder marcarte la fecha,
datan mis mayores trastornos. Las paredes y el techo de \emph{Jesús} se
me caían encima. Las locuras de otros días se repitieron con mayor
gravedad; yo no me contentaba con dar gritos, sino que se me salían de
la boca, sin pensarlo, palabras feísimas, las más feas que hay, y que yo
no había dicho nunca. Pasados días me divertía mucho asustando a las
monjas; mejor será decir que me vengaba. Algunas no me podían ver. El
susto de más efecto era figurar que me ahorcaba, y apretándome el cordel
y sacando la lengua, yo les metía un miedo horroroso\ldots{} A tanto
llegué con aquel desatino, que ya no me dejaban sola en mi celda, y
dormía siempre con dos guardianas. Andando los meses me sosegué, no
influyendo poco en ello la divina \emph{Madre}, que muy cariñosa me
amonestó y consoló, permitiéndome coger plantas y hacer con ellas
apartadijos como los de los herbolarios\ldots{} Pero un día,
¡ay!\ldots{} Voy a contarte lo más atroz que hice, y el más
estrafalario, el más ridículo y cruel de mis disparates. No sé qué día
fue, ni la fiesta solemne que celebrábamos, porque en esto de fechas y
festividades siempre he sido muy corta de memoria. Lo que sí recuerdo
como si lo estuviera viendo es que aquel día tuvimos procesión por el
claustro, a la que asistió el Rey bajo palio, con cirio, acompañado del
Infante D. Francisco, su señor padre y del Padre Fulgencio, su
confesor\ldots{} Después de esto hubo refresco; se sentaron todos en el
jardinillo que hay en el centro del claustro. Recordando el calor que
hacía, calculo que ello era al apuntar del verano, quizás en la fiesta
de la Pentecostés o de la Santísima Trinidad\ldots{} Yo me acuerdo de
que llevé sillas para que se sentaran los convidados. Frente al Rey
estaba Patrocinio; a su derecha el Infante D. Francisco, y a su
izquierda un fraile que no sé si era el Padre Carrascosa, confesor de la
\emph{Madre}, o Fray Toribio Martínez Cuadrado. Es muy raro esto de que
se me confundan en la memoria dos frailazos de época muy distante el uno
del otro. La confusión será porque se parecían; ambos eran grandullones,
fornidos, de anchos hombros y pecho, caderas muy señaladas; unos
hombrachos como castillos, con gordura de mujeres apopléticas. Yo
llevaba bandejas con refrescos, y me las traía con los vasos
vacíos\ldots{} En una de estas idas y venidas me entró de repente la
mala idea, una idea rencorosa y asesina, que con ninguna reflexión pude
dominar. Ello eran unas ganas muy vivas, muy ardientes, de ofender al
buen fraile, que a mí no me había hecho daño alguno ¡pobre señor!, pero
que en aquel momento me inspiró un odio mortal y una repugnancia
inaudita, por el bulto que hacían sus carnazas amazacotadas. Ciega de
aquel furor que me acometió como una instigación del demonio, dejé en el
suelo la bandeja vacía, metí la mano bajo el escapulario, saqué un
alfiler muy gordo y largo, de cabeza negra, que llevar conmigo solía, y
cogiéndolo con disimulo, y llegándome bonitamente al fraile, se lo clavé
en la nalga con presteza y saña, metiéndoselo hasta la cabeza\ldots{}
Hija, el grito que soltó Su Paternidad, y el respingo que dio, saltando
del banco y echándose mano a la parte dolorida fueron tales, que al
primer momento todas las monjas soltaron la risa\ldots{} Bufaba el
fraile; yo salí huyendo avergonzada, y aquello fue un escándalo, una
tragedia\ldots{} Luego me contaron que el Rey se había reído, y
consolaba al Padre diciéndole que el alfilerazo no había sido más que
una broma, y que sin duda mi intención no fue irme tan a fondo\ldots{}

»Ya comprenderás que esta barrabasada mía, hecha tan sin pensar, agravó
mi situación\ldots{} En el convento se hablaba de mandarme al
\emph{Nuncio} de Toledo, donde hay un departamento para monjas que están
mal de la jícara. Las que me querían mal me lo dijeron, y al saberlo yo,
tuve el arrebato de ahorcarme de verdad, que sólo me duró un
ratito\ldots{} En esto me llamó Patrocinio a su celda y hablamos lo que
voy a contarte: «Yo me someto a todo lo que Su Caridad determine---le
dije,---menos a que me lleven a una casa de Orates, pues aunque parezca
loca no lo soy. El clavarle el alfiler al Padre confesor fue una
travesura\ldots{} Él nos había dicho que el dolor es muy bueno y que
debe regocijarnos. Tuve la mala idea de causarle dolor para que se
regocijara\ldots{} Pero no volveré a jugar con alfileres; yo se lo
prometo a Su Caridad.» Y ella a mí: «Hermana, está usted enfermita del
caletre, y es menester curarla. Su mal proviene, según entiendo, de una
fuerte inclinación a las cosas temporales, que perdura después de tantos
años de vida religiosa. ¿Qué quiere decir eso de rebuscar y exprimir las
plantas para comerciar con su jugo? Pues es codicia, es preferir lo
humano a lo divino, y lo menudo a lo grande\ldots» Yo repliqué: «Así es.
Su Caridad está en lo cierto. Me llama lo menudo y andar a cuatro pies
por la tierra. ¡Dichosas las almas que se apacientan en los campos del
cielo comiendo estrellas! Yo no tengo esa perfección. Al lado de Su
Caridad soy como una burra que pasta en los prados y no ve más que lo
que come\ldots{} Tengo la pasión de las cosas necesarias, o si se
quiere, menudas, y de entretener mis manos en labores vulgares que den
de comer a alguien, a mí la primera. Me gusta trabajar, hacer cosas; me
gusta vender, me gusta cobrar\ldots{} Si eso es pecado, soy gran
pecadora; pero no demente.» Y ella: «No diré que sea pecado en el siglo;
aquí podría serlo. Hermana mía, yo no le deseo ningún mal; quiero para
usted todos los bienes, y puesto que se ha llamado burra, le diré que
este pesebre no le cuadra\ldots» A la semana siguiente volvió a llamarme
y me notificó que yo no podía seguir en el convento, y que por no dar la
campanada de mandarme al \emph{Nuncio} había escrito a mi madrina, Doña
Victoria Sarmiento, para que supiera lo que ocurría, y a mi padre para
que fuese por mí y se encargara de mi curación. Estas palabras de la
divina \emph{Madre} me causaron tanto gozo, que sólo con oírlas se me
quitaron como por milagro todos mis males de corazón y de nervios. Ve
aquí por qué quiero a la \emph{Madre}. Llorando de gratitud le di las
gracias, y al despedirme me dijo con gracia: «Celebraré mucho que el
trajín de \emph{hacer cosas} y de venderlas la cure de esos arrechuchos,
hermana querida. Ayude usted en la cerería al buen D. Gabino, que ya
debe de estar gastadito y achacoso; trabaje con él, cobre salud, y no
nos olvide. Haga vida de recogimiento para que su salida no cause
escándalo, y viva en concepto y opinión de enferma que busca su
reparación en la casa paterna\ldots{} Antes de abandonarnos, déjenos,
con sus recetas, todo lo que tenga hecho de la pasta para blanquear y
afinar el cutis de las manos.»

»Los días que transcurrieron desde esta conversación hasta que mi padre,
la \emph{Madre}, Doña Victorina y el Vicario se pusieron de acuerdo para
mi salida, los pasé en gran ansiedad. Me atormentaba la idea de que el
fraile, cuyas carnes orondas traspasé con el alfiler, influyera para
que, en vez de mandarme a mi casa, me encerraran en el Nuncio. O me
perdonó mi víctima, o no quiso ocuparse de mí\ldots{} En aquellos días
entraste tú y te pusieron a mi servicio. Simpatizamos; me inspirabas
lástima; pensé que te catequizaban para sepultarte allí toda la vida. Mi
padre llegó al fin, y solté los hábitos para venirme a casa. De la
fuerza del alegrón yo estaba como idiota cuando salí, y en los primeros
días que aquí pasé, el ruido de la calle me ensordecía, y mi padre, mi
hermano y Tomás eran como fantasmas que alrededor de mí se
paseaban\ldots{} Poco a poco fui entrando en la nueva vida y
regocijándome más con ella. La tarde en que te me presentaste,
diciéndome que te habías escapado y que en mi compañía querías estar
hasta saber el paradero de tu padre, me alegré de veras: tu libertad me
afirmaba en el contento de la mía\ldots{} Me referiste lo del
\emph{Relámpago}, y nos reímos del gran mico que se llevaron las monjas
y el Padre Fulgencio\ldots{} He concluido. Nada más tengo que contarte.»

Emancipada la atención de Lucila del interés del cuento, volvió a caer
en el asunto que embargaba su espíritu: el amor de Tomín, su salud, su
libertad. Observábala Domiciana alargando los morros. Por fin, la moza,
sacando un suspiro de lo más profundo, se levantó y dijo: «Es tarde ya.
Tengo que irme.»

---Pero no te hagas la desentendida. Quedaste en darme los zapatos. Ya
supondrás que no los quiero de balde. Te doy por ellos unos míos casi
nuevos, y unas medias que no he estrenado todavía. Mis pies y los tuyos
son tan hermanos que parecen los mismos\ldots» Al decir esto se
descalzaba. «Mira: no eres tú sola la que puedes ufanarte de un bonito
pie. Ven a mi cuarto y haremos el cambio.»

Llevola al gabinete próximo, y allí trocaron su calzado. Lucila iba
ganando; pero la otra parecía más satisfecha, y reía mirando en sus pies
las rojas chinelas puntiagudas. Luego recogió Cigüela de manos de su
bienhechora lo que esta le había ofrecido: chocolate, pan, alguna
golosina, y de añadidura media peseta columnaria. «Ya ves---dijo la
exclaustrada contrayendo los morros:---te doy dos reales y medio.»

---No sé cómo agradecerle favores tantos, Domiciana. Si no se enfadara,
si no dijera usted que me ha hecho la boca un fraile, me
atrevería\ldots{} ¿De veras no se enfada? Pues quisiera llevarle un
poquito de ese licor\ldots{} ¡Le sentará tan bien!

---¿Un poquito has dicho? Pues te llevas la otra botella que tengo. Ya
me dará más Alonso. ¡Pobre Tomín, qué bien le probará! No le des más que
un poquito a cada comida: esto ayuda a la reparación de fuerzas\ldots{}
Dime otra cosa: ¿fuma el Capitán?

---Sí que fuma cuando tiene qué. Yo recojo todas las colillas que
encuentro; se las pico muy bien picaditas\ldots{}

---Toma, toma otro real\ldots{} Le compras un paquete de picadura, o un
macito de a veinticinco\ldots{} Para el fumador, no hay privación más
penosa que la de este vicio. Hemos de estar en todo\ldots{} Vaya, no te
detengas. Adiós.»

Salió Lucila muy consolada y muy agradecida, pero también un tanto
recelosa. En su alma tomaba fuerza el deseo de ser sola en cuidar y
proteger al infeliz Capitán. No quería compartir con nadie su
abnegación, porque partiéndola o admitiendo la abnegación extraña,
creería ceder o enajenar parte de sus derechos al amor de Tomín. Temía
que la gratitud del hombre tuviera que dividirse, y ella no admitía tal
división, mayormente si la partija de aquel sentimiento recaía en una
mujer, quien quiera que esta fuese. Cierto que la generosidad de
Domiciana era desinteresada; nunca había visto al Capitán; pero podía
llegar a conocerle, extremar sus beneficios, y reclamar siquiera algunos
rayos de la mirada de los ojos azules\ldots{} Cuando llegó a la Cabecera
del Rastro, disipáronse como bocanada de humo estas vagas cavilaciones,
dejando todo el espacio de su alma a la previsión y ansiosas dudas de lo
que a su regreso encontraría. ¿Habría pasado algo?\ldots{} Acordose
entonces de los periódicos que le había encargado Tomín, y volvió atrás,
muy disgustada de su mala memoria y de la tardanza que el largo rodeo en
busca de los papeles le ocasionaría\ldots{} Vaciló, detúvose en la calle
de San Dámaso, pensando si sería más conveniente entrar tarde con los
periódicos o temprano sin ellos, y al fin decidiose por lo segundo,
amparándose de esta especiosa razón: «El tabaco y el rosolí bien valen
los papeles\ldots{} Otro día será.»

Como siempre, subió temblando por la luenga escalera que bajo sus pies
gemía. Famélicos gatos la saludaron con mayidos melancólicos. La noche
era plácida, estrellada, y del suelo subían el vapor y el ruido de la
vida urbana, mezclados con el desagradable olor de las fábricas de velas
de sebo. En las primeras noches que la pobre Lucila vivió en tan
desamparadas alturas, el vaho del sebo derretido se le metía en la
cabeza, y de tal modo a su mente se adhería cuanto contemplaban sus
ojos, que llegó a creer que olían mal las estrellas. Pero a todo se fue
acostumbrando, y la delicadeza de su olfato se embotaba de día en
día\ldots{} Sin aliento llegó a su desván. No había ocurrido nada: Tomín
la esperaba risueño y tranquilo. Se abrazaron.

Entre los abrazos, dio Cigüela explicación de no haber llevado los
periódicos, y mostrando el botín de aquel día, más pingüe de lo que
Tomín pudiera imaginar, le permitió catar el rosolí, como medicamento
tónico. Antes de la cura y cena, la enfermera le dio un paseíto por la
estancia, durante el cual el preso estuvo ágil de remos, despabilado de
cabeza, decidor de palabra. Y antes de recogerle a su descanso, le
arrimó al ventanal para que contemplara el cielo. Lucila le enseñaba las
estrellas más brillantes, las más hermosas\ldots{} que olían a rosolí.

\hypertarget{ix}{%
\chapter{IX}\label{ix}}

Pasaron días, entre los cuales se deslizaron los de Navidad,
confundiendo su barullo con el trajín de los ordinarios; acabose el año
50, y entró su sucesor con fríos crueles, que obligaron al vecindario de
Madrid a recogerse al amor de las camillas para sacar los estrechos. ¡Y
qué graciosísimos disparates resultaron de aquel juego en algunas casas!
Al sacar las papeletas, todo el concurso reventaba de risa. ¡Martínez de
la Rosa con la Petra Cámara; el Nuncio con la dentista Doña Polonia
Sanz, y la Reina Madre con D. Wenceslao Ayguals de Izco! Entre Navidad y
Reyes, hizo Lucila no pocas visitas a Domiciana, encontrando a esta tan
magnánima y dadivosa, que parecía constituirse en Providencia nata del
pobre Tomín y de su atribulada compañera. Una tarde le dio, envueltos en
un papel de seda, dos cigarros puros, que ella misma había comprado. A
tan hermoso obsequio, siguieron: ya el cuarto de gallina, ya la perdiz
escabechada, bien las lucidas porciones de garbanzos, patatas y otros
comestibles. Huevos hubo un día, otro jamón, y nunca faltaban chocolate
y pan. Los cuartos y las medias pesetas o pesetas, a veces columnarias,
menudeaban que era un gusto, y cuando apretaron las heladas, se descolgó
con una buena manta, nuevecita. Lo de menos era la limosna material, que
más que esta valían el buen modo y las recomendaciones cariñosas. «¡Ay,
hija, evitemos a todo trance que pase frío!\ldots{} Ten cuidado, por la
noche, de que no se ponga a dar manotazos, destapándose\ldots{} Arrópale
bien\ldots{} Dale la comida con método, sin dejarle que se atraque de lo
que más le guste; y el vino con medida\ldots{} Para cuando pueda salir
de casa, le estoy preparando un chaleco de mucho abrigo\ldots{} Mira:
estos cigarros que te doy son para que fume hoy uno y otro mañana. No
permitas que se fume los dos en un día.»

Y Cigüela, con estas crecientes efusiones caritativas, agradeciendo
mucho y recelando más. ¿Pero qué remedio tenía sino tomar lo que le
daban, librándose así de la fatigosa y triste correría en busca de
socorro? Atenta siempre a los actos y dichos de Domiciana, observó en
aquellos días alguna variación en sus hábitos: la que no salía de casa
más que para ir a misa a San Justo muy temprano, ausente estaba largas
horas en pleno día. Dos veces dijeron a Cigüela en la cerería que la
señora había salido, y tuvo que esperarla. Al entrar fatigada, decía la
monja que la necesidad de colocar sus drogas la sacaba de su quietud y
recogimiento. En todo ello resplandecía la verosimilitud; pero la guapa
moza, llevada por su desamparo y la tenacidad de sus desdichas a un
horrendo escepticismo, en los hechos más inocentes veía sombrajos o
barruntos de nuevas tribulaciones.

Ya iban los Reyes de vuelta para su tierra de Oriente, y llevaban tres
días o cuatro de camino, cuando Lucila, al entrar en la cerería, se
sorprendió de ver en ella más gente de la que allí solía pasar el rato
charlando. Las primeras palabras que oyó hiciéronle comprender que había
caído Narváez. Ya era Jefe del Gobierno D. Juan Bravo Murillo. Se alegró
de la noticia, pues a Narváez, visto del lado de su particular
desventura, le juzgaba como el peor de los gobernantes. «Luciíta---le
dijo D. Gabino con la melosa inflexión de voz que para ella
reservaba,---pasa al taller, que hoy es día de cera, y allá está mi hija
regentando.» Corrió la moza a la trastienda, y de allí, por estrecho
patinillo en que había un pozo cubierto, ganó la puerta de un aposento
ahumado. Salió Domiciana a recibirla con mandilón de arpillera y el cazo
en la mano, y a gritos le dijo: «Ven, mujer\ldots{} Ya te esperaba. Hoy
estamos de enhorabuena.» No era la primera vez que su amiga la recibía
en las funciones del arte de cerero, aplicando a ellas el elemento más
varonil de su compleja voluntad. Aquel día la vio Lucila más radiante de
absolutismo, más fachendosa y con los morros más prominentes.

---¿Enhorabuena ha dicho usted?

---¿Pero no sabes que ha caído ese perro? Tendido le tienes ya en medio
de la calle, y no volverá a levantarse, pues\ldots{} quien yo me sé le
pondrá el pie sobre la jeta para que no remusgue. Alégrate, mujer; ya
nos ha quitado Dios de en medio al causante de la desgracia de tu
pobrecito Tolomín.»

No podía la cerera extenderse en mayores comentarios, porque la cera,
derritiéndose en la olla puesta al fuego, decía con su hervor que ya
estaba en el punto de licuación, y que anhelaba correr sobre los
pábilos. A una señal de Domiciana, Ezequiel y Tomás cogieron la olla por
sus dos asas y la llevaron al centro de la estancia, junto al arillo,
rueda colgada horizontalmente. De la circunferencia de este artefacto
pendían los pabilos de algodón e hilaza cortados cuidadosamente por D.
Gabino. A plomo bajo el arillo fue puesta la paila, que debía recibir el
gotear de la cera. Ezequiel ocupó su sitio, arrimando a su pecho el
bañador. Iniciado el girar lento del arillo, a medida que iban llegando
frente al operario los pabilos colgantes, aquel derramaba en la cabeza
de estos la cera líquida que con un cazo sacaba de la olla. Los pabilos,
pasando uno tras otro y repasando en circular procesión de tío-vivo,
iban recibiendo la lluvia o baño vertical de cera, que pronto blanqueaba
y vestía de carne los esqueletos de algodón. Domiciana no apartaba de su
hermano los ojos, vigilando la obra y recomendando que los chorretazos
del líquido fueran administrados con esmero, para que todos los hilos se
revistiesen por igual, y engrosaran sus cuerpos sin jorobas ni
buches\ldots{} El arillo se aceleraba conducido por Tomás demasiado a
prisa, y Ezequiel, que era en sus movimientos muy parsimonioso, dejaba
pasar algunos pábilos sin echarles el riego. Pero Domiciana, templando,
midiendo y coordinando las dos fuerzas, logró al fin la perfecta
armonía, y el trabajo siguió su curso, remedando la eficaz lentitud de
las funciones de la Naturaleza.

Lucila seguía con su mirada el paso de los pábilos, como si algo le
dijera o expresara la ceremoniosa marcha, y el irse vistiendo unos tras
otros, siendo cada cual punto en que concluía y principiaba la
operación, imagen de las cosas eternas y del giro del tiempo. Como había
entrado de la calle muerta de frío, el calor del taller la confortaba, y
hastiado su olfato del tufo de sebo que respiraba en las calles del Sur,
el noble olor eclesiástico de la cera le resultó sensación grata, como
la de besar el anillo de un señor Obispo a la salida de función solemne.
El abrigo del taller y la conversación de Domiciana atrajeron a más de
un tertulio de los que tiritaban en la tienda: un señor de mediana edad,
vestido con buena ropa de largo uso, con todas las trazas de cesante de
cierta categoría, entró de los primeros, y arrimando sus manos al
rescoldo de la hornilla donde estuvo la olla, manifestó con gruñidos el
regocijo del animal que satisface un apremiante apetito. «Caliéntese
aquí, D. Mariano---le dijo la cerera,---y quiera Dios que el sol que
ahora sale le caliente más todavía.

---En ello pienso, señora\ldots{} ¿Sabe usted que en el nuevo Ministerio
tenemos a Bertrán de Lis, amigo mío desde que éramos muchachos? Pienso
que ahora se ha de reparar la injusticia que hicieron conmigo los
hombres del 44.»

Sentose junto a Lucila, que le saludó con inclinación de cabeza: le
conocía de verle en la tienda. Era D. Mariano Centurión, palaciego
cesante, que bebía los vientos por recobrar su plaza.

---Y no se diga de mí---prosiguió,---que soy de los hombre del 40, pues
también Bertrán de Lis es del 40, y si me apuran tendré que ponerle
entre los del 34, el año de la matazón de frailes\ldots{} El cambiar de
los tiempos me ha traído a mí a un cambio completo de dogmas. Narváez me
quitó mi destino sin más fundamento que mi amistad con Olózaga, y hace
poco me negó la reposición porque soy amigo de Donoso Cortés. ¿En qué
quedamos? ¿A qué santo debe uno encomendarse?

En esto entró un clérigo, que se refregó las manos junto a las brasas
diciendo: «Créame el amigo Centurión: son los mismos perritos del 37,
con los collares que se pusieron para hacer la del 43\ldots{} Pero a mí
no me la dan. No me trago yo el bolo de que Don Juan Bravo Murillo viene
a desembarazarnos de la Constitución y a devolvernos la sencillez
clásica del Absolutismo\ldots{} Para esto necesitaría traer otra gente.
A estos hombres no les entra en la cabeza el Gobierno de Cristo. Mírelos
usted bien, y verá que por debajo de los faldones de las casacas
bordadas se les ve el rabo masónico\ldots{} ji, ji\ldots{} No me fío,
Sr. D. Mariano; no veo la Moralidad, no veo la Fe\ldots{}

---¡Ah! perdone el amigo Codoñera---dijo Centurión con ironía
grave.---Lo que darán de sí estos caballeros en política no lo
sé\ldots{} pero en Moralidad han de hacer primores. Como que no vienen a
otra cosa. ¡Moralidad y Economías! Y no me negará usted que todos traen
divisa blanca, como procedentes de la ganadería de la Honradez.

---Eso sí: y el pueblo, que otra cosa no sabrá, pero a poner motes
graciosos y oportunos no hay quien le gane, llama al nuevo Ministerio
\emph{El honrado concejo de la Mesta.»}

Los pábilos ya no se veían bajo la vestidura de cera; las velas
engordaban a cada revolución del arillo, presentándose a recibir el
riego, y siguiendo su paso de baile ceremonioso por todo el circuito.
Sin desentenderse de la vigilancia del trabajo, Domiciana llamó junto a
sí a Lucila para decirle: «Hoy no podremos charlar: ya ves. Si no estoy
encima de esta gente, me harán cualquier chapucería. A mi padre dejé el
encargo de darte ocho reales: compra lo que necesites para hoy; no
olvides de llevarle a Tomín papeles públicos para que se entere bien de
que entran a mandar los Honrados. En confianza te diré que creo en el
indulto como si ya lo viéramos en la \emph{Gaceta}\ldots{} Oye otra
cosa: mientras viene el indulto, convendrá que tengáis un alojamiento
más seguro y decoroso, con más comodidades, donde Tomín pueda reponerse
y cobrar fuerzas\ldots{} De eso me encargo yo\ldots{} Puedes marcharte
ya si quieres. Si mañana vienes temprano y no me encuentras aquí, estaré
en San Justo.»

Echó Lucila la última mirada a las velas, que seguían bañándose en cera
y engrosando a cada chorro, y se fue hacia la tienda. Allí le salió al
encuentro D. Gabino, y empujándola hacia la rinconada donde tenía el
pupitre y el cajón del dinero, le puso en la mano las dos pesetas
designadas por su hija, y otra, columnaria, que de tapadillo el buen
señor por su cuenta le daba. Le cerró y apretó la mano en que ella las
había recibido, y alegrado su rostro con una confianza un tanto
picaresca, le dijo: «Luciíta, eres tan guapa, que no está bien andes
suelta por el mundo, donde te solicitarán pisaverdes sin juicio y
mozuelos \emph{de poca pringue}. Oye mi consejo: debes tomar estado.
Piensa bien lo que haces. Te conviene un marido maduro, un marido
sentado\ldots{} Los hay, yo te lo aseguro; los hay muy respetables, algo
añosos; pero que saben cumplir, y bien probado lo tienen\ldots{} ¿Con
que lo pensarás, Luciíta? ¿Me prometes pensarlo?

---Sí, D. Gabino, lo pensaré---replicó Lucila con verdaderas ansias de
perder de vista al patriarca fecundo.---Déjeme que lo piense\ldots{} y
muchas gracias.»

Otros dos estantiguas, que de mostrador adentro, arrimados a un brasero
mustio, rezongaban críticas del \emph{honrado} Ministerio, la
despidieron con amables adioses y sonrisas de bocas desdentadas. Salió
Cigüela con el corazón oprimido, no sabiendo si bendecir a Dios por la
creciente abundancia de los socorros y dádivas, o maldecir su propia
suerte, que la incapacitaba para la debida gratitud\ldots{} Era media
tarde, y vagó largo rato por Madrid haciendo sus compras, y buscando
periódicos para que Tomín leyese y juzgase por sí mismo las cosas
políticas. Movida de la curiosidad, y andando ya para su casa, parábase
a leer algo en los papeles que había comprado, por si alguno hablaba ya
de indulto a los militares condenados en Consejo de guerra. Pero nada de
esto encontró, sino una palabrería ininteligible sobre la Deuda pública
y sus arreglos, y noticias sobre la próxima inauguración del ferrocarril
de Madrid al Real Sitio de Aranjuez. Nada le importaba a Lucila la
llamada Deuda pública, que no era otra cosa que las trampas del
Gobierno, y en cuanto al Camino de Hierro, admitió su utilidad pensando
que siempre es bueno llegar pronto a donde se quiere ir.

Con esta idea avivó el paso, sin desviarse del recto camino. Al subir a
su camaranchón aéreo, encontró a Tomín levantado, impaciente\ldots{} Ya
podía pasear solo; ya se desvanecían y alejaban, con los dolores de su
cuerpo, las sombras de su espíritu\ldots{} El día era la vida, la noche
la esperanza.

\hypertarget{x}{%
\chapter{X}\label{x}}

Fue a San Justo Lucila en busca de Domiciana, como esta le había
mandado; pero no la encontró. En la cerería tampoco estaba. Prescindió
de ella por aquel día, y al siguiente le dio D. Gabino el socorro por
encargo de su hija, que andaba en ocupaciones callejeras. Otro día
volvió, en ocasión de estar ausente el cerero. Ezequiel entregó a la
moza, de parte de su hermana, un paquete de comestibles y dos moneditas
de a real y cuartillo, agregando frases de afecto dulce, y una vanidosa
ostentación de las velas que estaba rizando detrás del mostrador. «Mira,
Lucila, ¿qué te parece esta obra?» Digno era en verdad aquel rizado de
los sacros altares a que lo destinaba la piedad, y por la gracia y
pulcritud del trabajo, competía con lo mejor que labraran manos de
angelicales monjas. Verdad que las de aquel mancebo manos de monja
parecían, en consonancia con su rostro lampiño y terso, con su expresión
de honestidad y la inocente languidez de su mirada. La tez era del color
de la cera más blanca, el cabello negro, los ojos garzos y tristes.
«Mira, Lucila, mira esta que rizo y adorno para ti---dijo atenuando su
orgullo con la media voz de la modestia, al mostrar una vela chica que
sacó de un hueco del mostrador.---Ayer la empecé, y no quiero hacerla de
prisa, para que me salga a mi gusto\ldots{}

---¡Oh, qué precioso, qué maravilla!---exclamó Lucila cogiendo la vela,
girándola suavemente para verla en redondo.

Admiró la moza el fino adorno que ya transformaba más de la mitad de la
vela. Los pellizcos hechos con tenacilla eran tan delicados, que
parecían escamas, erizadas con una simetría que sólo se ve en las obras
de la Naturaleza. Más abajo, el \emph{rizado} al aire, que se hace
levantando tenues virutas de la pasta con una gubia, y dándoles curva
graciosa, imitaba los estambres de flores gigantescas, o las delgadas
trompas con que las mariposas liban la miel de los dulces cálices.
Lucila no había visto mayor fineza ni arte más soberano para embellecer
la cera. Con toda su alma estimaba y agradecía la pobre mujer aquel
obsequio, y sentía que no recayera en persona de posición y medios para
ostentarlo dignamente. Algo de esto insinuó a Ezequiel, procurando
alejar de su palabra todo lo que pareciese intención de desaire, y el
hábil mancebo le dijo sonriendo: «¿Pero qué, Lucila, en tu casa no
tienes altar? ¿No tienes ninguna imagen? ¿Dices que no? ¿Quieres que te
regale una virgencita que fue de mi mamá?\ldots»

Respondió Lucila que lo sentía mucho; pero que no tenía ni altar, ni
casa, ni muebles dignos de aquella joya, que merecía ser guardada y
manifiesta dentro de un fanal.

---Pues también te daré un fanal---dijo Ezequiel poniendo cuidadosamente
la vela en una gruesa tabla agujerada, donde se ajustaba el cabo como en
un candelero.---¡Ay, qué primorosa quedará esta obra cuando la concluya!
Lo que ves no es nada. Después que acabe de hacer el rizado al aire y el
de tenacilla, pondré las flores. Ya tengo hechos los moldes de patata
para campánulas, jazmines y narcisos, que luego pintaré de colores
variados. Los aritos de talco serán de lo más fino. Y dime ahora, pues a
tiempo estamos: ¿cuál es la combinación de color y metal que más te
gusta? ¿Te parece que ponga azul y plata?

---Pon lo que creas más lucido, Ezequiel. ¿Quién lo entiende como
tú?\ldots{} Pero si te empeñas en consultarme, pon el rojo, que es el
color más de mi gusto. Doble combinación de encarnado con plata y azul
con oro será muy linda.

---Preciosa, como ideada por ti.»

No pudo prolongarse más el interesante coloquio, porque entró D. Mariano
Centurión, por cierto de muy mal talante, y al poco rato D. Gabino. Este
pareció sentir mucho la presencia de testigos, que le impedía disertar
con Lucila acerca de sus proyectos referentes al aumento de población.

Los misteriosos quehaceres de Domiciana fuera de casa, que tan
singularmente rompían el método de sus monjiles costumbres, tuvieron en
aquellos días algunas horas de tregua y descanso para recibir a su
protegida y platicar extensamente con ella. En su oficina de hierbas y
drogas la encontró Lucila una tarde, calzada con los chapines rojos,
vestida con bata nueva, de cúbica, a la moda, que en ella era radical
mudanza de los antiguos hábitos. En modales y habla notó asimismo Lucila
un marcado intento de transformación. «Siéntate, mujer, que estarás
cansada---le dijo acercando una silla a la mesa baja.---Yo llevo unos
días de ajetreo que me han ocasionado agujetas\ldots{} Pero ya me voy
acostumbrando.

Si aquel concepto sorprendió a Lucila, mayor extrañeza y confusión le
produjo estotro, expresado al poco rato: «No te asombres tanto de verme
un poquito tocada de reforma en lo de fuera. Es que cuando una llega
tarde a la vida, forzada se ve a marchar de prisa, haciendo en semanas
lo que es obra de años largos\ldots{} Aturdida y sin saber por dónde
andaba me has tenido días enteros\ldots{} A la inteligencia que se ha
embotado con el desuso, no le salen los filos cortantes sino después de
pasarla y repasarla por la piedra\ldots{} y no te digo más por
hoy\ldots»

Llegada al punto divisorio entre la confianza y la discreción, calló
Domiciana. Pidiole Cigüela mayor claridad; pero la cerera se limitó a
decirle: «Algún día, quizás muy pronto, no tendré secretos para ti.
Espera y no seas preguntona.»

Tratando de lo que más a Lucila interesaba, dijo Domiciana: «Ten ya por
asegurado tu diario sustento y el del caballero. Yo sola proveeré; yo
corro con todo. ¿Me preguntas si habrá indulto?\ldots{} Como indulto
general, nada sé. No me consta que haya tratado de esto el señor Conde
de Mirasol. Pero el indulto personal de Tolomín, ya es otra cosa. No te
digo que esté ya concedido, ni tramitado, ni que se haya pensado en
él\ldots{} Como que ni siquiera sé el nombre y apellido del que llamamos
Tomín. De hoy no pasa que me lo digas, para apuntarlo en un
papel\ldots{} Vendrá el indulto. ¿Cuándo? No puedo decírtelo. Pero
vendrá, no lo dudes. Bien pudiera ser que en favor de Tomín se
interesara el director de Infantería D. Leopoldo O'Donnell. Pero
interésese o no, el rayo de gracia partirá de muy alta voluntad, a la
cual nadie puede resistirse\ldots{} Y mientras esto llega, se tratará de
librarle a él y a ti de la ansiedad en que vivís; se vendarán los ojos
de la policía para que no vea lo que no debe ver. ¿Me has entendido?»

Poniendo sobre todas las cosas el amor de Tomín y la salvación y salud
de este, no podía menos Lucila de celebrar tan lisonjeras esperanzas,
aunque en ello viera un desmerecimiento grave de su personalidad en la
protección del Capitán perseguido. Alguien que no era ella cuidaba de
darle sustento y comodidades; alguien que no era ella mejoraba su
alojamiento; alguien que no era ella le aseguraba al fin la libertad y
la vida misma. Dejaba de ser Lucila la Providencia única, insustituible,
y otra tutelar bienhechora resurgía de improviso, compartiendo con ella
el trono de la abnegación, o quizás arrojándola de él. Atormentada de
esta idea, sintió caer sobre su corazón una gota fría cuando precisada
se vio a dictar el nombre y apellido de Tolomé para que Domiciana lo
escribiese. Luego sacó esta del armario en que guardaba sus selectos
productos industriales un diminuto frasco, y mostrándolo al trasluz,
dijo: «Conviene que vaya pensando el Sr.~de Gracián en arreglarse y
acicalarse un poco, que un buen caballero no debe olvidar sus hábitos de
toda la vida. Aquí tienes aceite aromatizado para el cabello. De su
bondad no dudarás cuando sepas que ha sido compuesto para persona de
sangre real. Llévatelo, y úsalo para él y para ti. Lo primero es que le
desenredes y le desengrases el pelo, que de seguro lo tendrá hecho una
plasta. Te daré la receta para desengrasar con yema de huevo. Después de
bien desengrasado, se lo cortas, que tendrá melenas de poeta muy
impropias de un rostro militar\ldots{} Mañana llevarás peines y un
cepillo. Me dirás cómo anda el hombre de calzado, y si está, como
supongo, en necesidad, tráeme la medida del pie. Hoy puedes llevarte el
chaleco de abrigo, y mañana una corbata, y para ti algunas cosillas que
te estoy preparando.»

Dio las gracias Lucila, y reiterando su deseo de que la protectora le
encomendara algún trabajo, no sólo por corresponder a sus beneficios,
sino también por no estar ociosa, Domiciana le dijo sonriendo: «Trabajo
te daré hasta que te canses. Pero no es cosa de mortero ni de filtro.
Ven a mi gabinete y verás.» Quería la cerera que Lucila la enseñara a
peinarse, pues perdida en la vida claustral la costumbre de componer con
donaire su cabeza, encontrábase a la sazón muy desmañada para este
artificio en que son maestras casi todas las mujeres. Y como no había
transcurrido tiempo bastante desde su libertad para el total crecimiento
del cabello, tenía la señora que aplicar añadidos y combinaciones que
aumentaban la torpeza de sus manos. Al instante procedió a complacerla
la diligente amiga, que a más de poseer en grado superior el arte de
acicalarse, había tomado lecciones de peinado. El cabello de Domiciana
era negro, fino y abundante, con dos ramalillos de canas en la parte
anterior, que bien puestos no carecerían de gracia. Lucila, silenciosa,
pasaba el carmenador pausadamente desde la raíz hasta los cabos, y en
este ir y venir del peine, surgieron en su pensamiento repentinas
aclaraciones de aquel enigma de la transfiguración de la exclaustrada.
No pudo atajar la vehemencia con que sus ideas pasaron de la mente a los
labios, y se dejó decir:

---La persona que a usted la trae tan dislocada y callejera, la que le
da tanto conocimiento del mundo y con el conocimiento influencia, es
Doña Victorina Sarmiento de Silva, dama de la Reina, que fue madrina de
usted cuando profesó, y si de monja la quería, ahora también.

---¡Qué tino has tenido!---dijo Domiciana risueña, mirándola en el
espejo de pivotes que delante tenía.---Acertaste: no hay para qué
negarlo.

---Las personas que manejan los palillos detrás de Doña Victorina, no
puedo adivinarlas\ldots{}

---Ni tienes por qué calentarte los sesos en esos cálculos, que yo a su
tiempo te lo diré, bobilla. Ten discreción y juicio.»

Calló Lucila, y con paciencia peine y tragacanto, utilizando el cabello
natural de su protectora y aplicando hábilmente los añadidos, rellenos y
ahuecadores, armó el peinado conforme a la moda en señoras graves,
sencillo, majestuoso; perfiló bien las rayas delantera y transversal, a
punta de peine; construyó el rodete, abultándolo con escondidos
artificios; extendió los bandós bien planchados y ondeados hasta rebasar
las orejas, dejando fuera tan sólo la ternilla agujerada para el
pendiente; recogió los cabos en el rodete, y todo lo remató con la
peineta graciosamente ladeada. Con ayuda de un espejo de mano miró
Domiciana su cabeza por detrás y en redondo, y satisfecha de la obra, no
fueron elogios los que hizo: «Estoy desconocida. Parezco otra, ¿verdad?
¡Lo que puede el arte!»

Al despedir a Lucila, recompensó su servicio con estas dulces promesas,
que valían más que el oro y la plata: «Poco te queda ya que sufrir,
pobrecilla. Vamos en camino de asegurar la vida y la libertad al pobre
Tomín. Yo estoy tranquila: tranquilízate tú, y no temas nada de la
policía. Lo que hay que hacer es cuidarle mucho para que acabe de
reponerse\ldots{} Que vaya cobrando fuerzas, animación y alegría\ldots{}
No dejes de venir pasado mañana para que estés aquí cuando me prueben
los dos vestidos que me encargué el lunes: el uno de merino obscuro, un
color así como de ratón con pintitas; el otro de seda negra. Hija, no he
tenido más remedio que hacerlo; pero es para calle, o para cuando tenga
que asistir a un acto religioso, procesión, Viático, consagración de
Obispo\ldots{} No vayas a creer que andaré yo en ceremonias
palaciegas\ldots{} Todo lo que ahora deseche será para ti. Verás también
mi mantilla nueva. Cuenta con una o dos de las que ahora uso\ldots{}
Ropa interior, medias, refajos, peinadores, también he tenido que
comprar\ldots{} Todo es preciso\ldots{} Tontuela, no me mires con esos
ojazos, que no te olvido nunca, y menos cuando voy de tiendas\ldots{} Ya
participarás\ldots{} De todo un poquito para la pobre Luci\ldots»

\hypertarget{xi}{%
\chapter{XI}\label{xi}}

Aunque al volver a la vida del siglo hacía Domiciana visitas frecuentes
a Doña Victorina Sarmiento de Silva, con la encomienda de ofrecerle
aguas aromáticas, polvos dentífricos y leche de rosa, la comunicativa
amistad que entre ellas se estableció, obra lenta del tiempo, no llegó a
consolidarse hasta muy avanzado el año 50. Achaques añejos turbaban la
existencia de Doña Victorina, y tenían su salud en constante amenaza.
Desengañada de médicos y boticas, había tomado afición al tratamiento
doméstico puramente vegetal, y como Domiciana le facilitara
combinaciones ingeniosas para el alivio de sus complejos males, puso en
ella confianza, y de la confianza y del frecuente trato nació una
cordialidad que cada día iba en aumento. Escogidas y preparadas por sus
propias manos, Domiciana le administraba la brionia para los nervios, la
cinoglosis para la tos, el sauce para los efectos diuréticos, la
genciana para combatir la fiebre, y como últimamente se le manifestaran
a la señora unos sarpullidos molestísimos, acudió contra ellos la
herbolaria, preparando, con esmero exquisito, el \emph{Agua de carne de
ternero para calmar el ardor de la piel}.

Si estos servicios no produjeron por sí la intimidad, fueron sin duda
sus conductores más eficaces, porque en el curso de ellos tuvo la dama
ocasión de apreciar la grande agudeza de Domiciana y los varios talentos
de que Dios habíala dotado largamente. Gustaba de su compañía, y no
había para ella conversación más grata que la de la cerera. La historia
de su reclusión monástica que Domiciana refirió a Lucila, según consta
en esta relación, fue contada mucho antes a Doña Victorina con lujo de
sinceridad y pormenores muy instructivos. Oyéndola y saboreándola, la
señora formulaba este juicio sintético: «Tu locura, en la cual no hubo
poco de fingimiento, fue tan sólo lo que llamaremos contra-vocación, o
irresistible necesidad de volver al siglo, de apagar el fuego místico,
por encender el no menos sagrado fuego de los afanes de la vida libre y
del trabajo.»

Era Doña Victorina de madura edad, ya pasada de los sesenta, y
desempeñaba el puesto de camarista en Palacio desde la caída de
Espartero. Tenía parentesco con la Priora de la Concepción Francisca,
Sor María del Pilar Barcones, y era grande amiga de la seráfica
Patrocinio. Con esta habló de Domiciana, encareciendo su don de
simpatía, su gran saber de cosas prácticas; y la de las llagas declaró
con sinceridad que nunca la tuvo por loca \emph{de hecho}, y que le
había facilitado la salida creyendo que mejor podría servir a Dios
dentro que fuera 1. De estas conversaciones entre las dos ilustres
señoras, provino que Domiciana fuese a visitar a sus antiguas
compañeras, y que después, por encargo de algunas, les suministrara
líquidos o polvos de su industriosa producción; y como en tales días,
que eran los del verano del 50, sufrieran algunas Madres rabiosas
picazones de cuerpo y manos, por efecto sin duda del calor (aunque
autores de crédito sostienen que ello entró de golpe, como contagiosa
epidemia fulminante), no paraba Domiciana de preparar para ellas sus tan
acreditadas \emph{Aguas de carne de ternero}, y además, con otros fines,
les llevaba la \emph{Purificación de hiel de buey para quitar manchas de
las ropas de tisú}.

Ya tenemos a Domiciana en comunicación diaria con las que fueron sus
Hermanas; entraba en clausura siempre que quería, y a solas platicaba
con Patrocinio en su celda, refiriendo con tanta prolijidad como gracia
todo lo que en el mundo veía, y las conversaciones que pasaban por sus
despiertísimas orejas. Fue creciendo y estrechándose esta comunicación,
no sin que Doña Victorina, condenada por sus achaques a cierta
inmovilidad, la utilizase para transmitir al convento, antes que los
acuerdos políticos de la \emph{casa grande}, los simples rumores y las
más insignificantes palpitaciones del vivir palatino. No necesitaba la
prestigiosa \emph{Madre} que le contaran lo que pensaban los Reyes, pues
esto ellos mismos se lo decían; pero gustaba de reunir y archivar una
viva documentación humana, de accidentes, menudencias o gacetillas, que
eran de grande auxilio para juzgar con acierto y enderezar bien las
determinaciones\ldots{} La exactitud, la sinceridad concienzuda con que
Domiciana transmitía de un extremo a otro las opiniones o noticias que
se le confiaban, sin quitar ni poner ni una mota gramatical o de estilo,
eran el encanto de Doña Victorina, que la diputó como el mejor telégrafo
del mundo, muy superior al sistema de torres de señales que en España se
establecía, y aun al llamado eléctrico, que ya funcionaba entre muchas
capitales europeas.

Sospechas muy cercanas al conocimiento tenía de aquel trajín telegráfico
de su amada hija el buen D. Gabino, y de ello se alegraba, esperando
algún medro para la familia y para los amigos. Y D. Mariano Centurión,
desde que su olfato perruno le reveló el porqué de tantas idas y
venidas, no dejaba en paz a la exclaustrada y en acecho vivía para
cazarla en la casa o en la calle. De la impaciente ansiedad del
desgraciado cesante dará idea este diálogo que con Domiciana sostuvo en
la tienda, un día de Febrero, que por más señas era el de San Blas. El
despacho había sido de consideración en la pasada festividad de las
Candelas, y D. Gabino estaba poniendo las cuentas de lo que tenía que
cobrar en las Carboneras y San Justo, en San Pedro y el Sacramento;
Domiciana y Lucila, que acababan de llegar, hacían pábilos; Ezequiel y
Tomás prepraban en el taller una tarea de velas\ldots{} Apoyado más que
sentado en un saco de cera en grumo, Centurión simbolizaba con su
postura la inestabilidad de su existencia, y con su palabra el
desasosiego en que vivía. «Tenga paciencia---le dijo Domiciana,---y
agárrese bien a los faldones del Sr.~Donoso, que es quien ha de sacarle
adelante. Yo, tonta de mí, ¿qué puedo?

---A los faldones me agarro---replicó Centurión;---pero como no soy
solo, como tantas manos acuden allí, pesamos mucho, y el hombre tiene
que sacudirse\ldots{} También digo y sostengo que no es mi amigo Donoso
el más prepotente, porque no fue quien nos trajo las gallinas,
\emph{vulgo} Ministerio, aunque lo parezca por el discurso famoso que
disparó contra Narváez en Diciembre. Sin discurso habría caído D. Ramón,
de quien estaban hartos en Palacio, y más hartas las \emph{Madrecitas}
de Jesús\ldots{} No se haga usted la asombradiza, Domiciana. Mejor que
yo sabe usted que a este Gobierno lo traen para que ponga la Religión
sobre la Libertad, y el Orden sobre el Parlamentarismo. ¿Cumplirá?

---Este Gobierno \emph{honrado}---afirmó D. Gabino,---bien claro lo han
dicho sus órganos, viene a moralizar la Administración y a santificar al
pueblo, apartándolo de los vicios. Ya se anuncian dos grandes medidas:
el arreglo de la Deuda, y la supresión del Entierro de la Sardina, que
es un gran escándalo popular en día como el Miércoles de Ceniza,
destinado a meditar que somos polvo\ldots{} Aunque no lo ha dicho,
porque esto es cosa delicada, también viene este Gobierno a quitar el
Militarismo, que es una de las mayores calamidades del Reino, y a poner
Economías, limpiando de vagos las oficinas, y rebajando sueldos a tanto
gorrón\ldots{} y por último, a librarnos de la plaga de la langosta,
pues en el Ministerio se ha presentado un proyecto muy útil, que el
Gobierno hace suyo, y consiste en acabar con el maldito insecto cebando
pavos\ldots»

Rompió a reír Centurión, y le quitó a su amigo la palabra diciendo:
«Antes digerirán los pavos toda la langosta de Andalucía y la Mancha,
que nosotros las bolas que nos hacen tragar los papeles públicos. Los
\emph{honrados} no han venido para quitar el Militarismo, ni para el
arreglo de la Deuda, ni para la moralidad, ni para las economías. Todas
esas son pantallas del disimulado pensamiento de la \emph{Honradez}, que
es comerse la Constitución, cerrar las Cortes, o dejarlas siquiera con
la puerta entornada, y abolir la imprenta libre\ldots{} A esto han
venido, y creer otra cosa es ver visiones. ¿Cumplirá el Gobierno?,
vuelvo a preguntar. Me temo que no, como sea reacio en llevar a su lado
a los hombres que abundamos en esa idea\ldots{} Si D. Juan, y Bertrán de
Lis, y González Romero nos postergan, no faltará quien mire por
nosotros. Manos blancas les condujeron a ellos a las poltronas. A esas
mismas manos nos agarraremos, y ¡ay de ellas si después de levantar a
los grandes no dan apoyo a los chicos!

---Este D. Mariano---observó la ex-monja encubriendo su sinceridad con
graciosa máscara,---es de los que creen la paparrucha de que las pobres
\emph{Madres} dan y quitan empleos. Dígame: ¿se ceba usted con esas
mentiras, como los pavos con la plaga de la langosta?

---Si me cebo o no me cebo con mentiras va usted a verlo,
Domiciana---replicó Centurión avanzando hacia ella, y asustando a todos
con su gesto iracundo y el temblor de su boca famélica.---A fines del
año pasado la Madre Patrocinio dijo: «quiero que sea Gentilhombre de
Palacio D. Ángel Juan Álvarez,» y al instante se mandó extender el
nombramiento. En Enero, Isidrito Losa, protegido de la misma
\emph{Madre}, quiso una plaza de Gentilhombre, con ocho mil reales.
Abrió la \emph{Madre} la boca y al instante se la midieron. Desconsolado
quedó el hermano de Isidro, Faustino Losa; pero la \emph{Madre} le
adjudicó una capellanía de honor con veinte mil reales. El sobrino de la
Priora, Vicente Sanz, fraile Francisco, hipaba por un destinito de
descanso. «Espérate un poco, hijo.» Abre otra vez su boca la Seráfica, y
hágote también Capellán de honor con veinte mil.

---Pero esos son empleos de Palacio, no del Gobierno.

---Pues de empleos de Palacio se habla. Y también digo a usted que lo
mismo decreta Su Caridad en destinos palaciegos que en destinos de la
Administración, y lo probaré cuando se quiera\ldots{} Ahora tienen las
monjas toda su atención en mejorar de vivienda, y para arreglarles el
palacio viejo de Osuna en la calle de Leganitos se está gastando la Casa
Real obra de dos millones de reales\ldots{} Pero como no es esta
bastante protección, la Reina dota con veinte mil reales a toda novicia
que allí tome el hábito, con lo que tenemos un jubileo de señoritas que
pasan del mundo al convento para descanso de sus padres. Ocho van ya del
verano acá, que le han costado al Real Patrimonio\ldots{} pues ocho mil
duretes. Luego decimos que aquí no hay dinero para nada, y que España es
un país de tiña y piojos\ldots{} La tiña la tenemos los infelices que no
sabemos o no queremos arrimarnos a los cuerpos bien incensados, y contra
piojos no hay remedio como el agua bendita\ldots{} Deme usted una vela,
Don Gabino; regáleme usted una vela de desecho, Domiciana, que no quiero
ser menos que mi amigo Bertrán de Lis, el cual armó un gran escándalo el
año 45, cuando Pidal quiso abolir la libertad de la imprenta, y después,
viéndose olvidado y desatendido, fue y ¿qué hizo?\ldots{} pues ponerse
escapularios y pedir ingreso en el \emph{Alumbrado y Vela}. A eso voy
yo, como una fiera, y no me contentaré con asistir a procesiones, sino
que a todas horas saldré por la calle con mi cirio, rezando el rosario,
para que me oigan, para que se enteren, para ser alguien en la comparsa
social; para que no me llamen \emph{Don Nadie}, y poder comer, poder
vivir\ldots{} Díganme dónde están la última monja y el último capuchino
para ir a besarles el borde las estameñas pardas y la suela de la
sandalia sucia. Locura es querer alimentarse con las soflamas de mi
amigo Donoso, forraje místico que no produce más que flato y acedías. Si
a él le pagan sus discursos con embajadas y títulos de Marqués, a los
que le aplaudimos y vamos por ahí dando resoplidos de admiración para
hinchar los vientos de su fama, nada nos dan, como no sea desaires y
malas palabras. Mucha religión, mucha teología política, mucha alianza
de Altar y Trono; ¿pero las magras dónde están? Yo las quiero, yo las
necesito: las reclama mi estómago y el estómago de toda mi familia que
es tan católica como otra cualquiera. Denme la vela, D. Gabino y
Domiciana\ldots{} Aquí está un hombre que se declara huérfano, y sale en
busca de una \emph{Madre} que le consuele\ldots{} Dígame, Domiciana,
cómo llegaré a la \emph{Madre}, y qué debo llevar, a más del escapulario
y vela, y qué arrumacos he de hacer para la adoración de sus llagas, que
yo pondría también en mis manos, y en toda parte de mi cuerpo donde
pudieran darme el olorcito de santidad que deseo.»

Sofocado y como delirante, sin saber ya lo que decía, terminó su arenga
el desesperado D. Mariano, y girando sobre sus talones fue a desplomarse
sobre el saco. La familia del cerero le oyó al principio con regocijo,
después con lástima, al fin con pena\ldots{} Todos suspiraban.

\hypertarget{xii}{%
\chapter{XII}\label{xii}}

Ganando fuerzas y cobrando ánimos en su lenta reparación, gracias a los
cuidados y al cariño de Lucila, que así le proveía de alimentos como de
esperanzas, el Capitán parecía otro en los últimos días de Febrero. El
renacimiento moral iba delante del físico; a medida que entraban en la
zahúrda las comodidades y el buen vivir, se iba marchando la tristeza, y
con las seguridades que llevaba la moza de que ya no debían temer acecho
de polizontes, recobraba Gracián toda la gallardía de su persona. Una
serena y tibia noche, después de cenar, sentáronse los dos en el
ventanón, y abrieron los cristales para contemplar el cielo y los
términos lejanos que a la claridad de la luna desde aquellas alturas se
distinguían. Ya Lucila, sin desmentir su modestia, se vestía y arreglaba
esmeradamente con lo que le daba la cerera, y Tomín, por no ser menos,
gustaba de componerse, para que ella viéndole se alegrara: se reían y
recíprocamente se alababan. «Ya estás como antes de la trifulca,
Tominillo; y si no fuera por la barba crecida, parecería que no habían
pasado días por ti. La cabeza es la misma: tu pelito cortado, como lo
tenías antes, y bien perfumadito, y tan suavecito. No dirás que no soy
buena peluquera.

---¡Tú sí que estás guapa!---contestaba él cogiendo a su vez el
incensario.---No sé si decir que estás ahora mejor que cuando te conocí.
Tus ojos son no sólo el alma tuya, sino el alma de todo el Universo.

---No, no, Tomín: los ojos tuyos son los que más cosas traen en su
mirar\ldots{} Miras, y se queda una pensando, asustada de lo grande que
es el mundo\ldots{} el mundo del querer, Tomín\ldots{}

---Grande es. Tus ojos lo miden, y aún les sobra medida. Yo veo en ellos
todas las cosas creadas\ldots{} y las que están por crear.

---El querer es gloria y martirio: por eso es un mundo que no tiene fin.

---El martirio tuyo por mí, Lucila, es mi gloria. Y mi padecer, ¿qué ha
sido más que la gloria tuya? Tú me has resucitado\ldots{} No me digas
que no eres santa, porque eso será lo único que no te creeré.»

De este tiroteo de ternezas, en elevada región de sus almas exaltadas,
descendían a las ideas prácticas, y trataban del problema que ya pedía
inmediata solución: cambiar de vivienda, estableciéndose en sitio más
holgado y decoroso. Después de divagar un rato sobre esto, iban a parar
al asunto que más embargaba la curiosidad y los pensamientos de Tomín.
Había cuidado Lucila de referirle todo lo que Domiciana hacía por él, o
por los dos, que en un solo sentimiento confundía el interés por
entrambos; contole también las relaciones de la ex-monja con una dama de
la Reina. Ni a Domiciana ni a Doña Victorina las conocía Gracián. La
cerera no le había visto nunca; ignoró su nombre hasta que Lucila se lo
dijo para que lo apuntara, el día mismo en que la enseñó a peinarse a la
moda. ¿No podría creerse que detrás de Domiciana y de la Sarmiento
existía, bien tapujadita entre sombras discretas, alguna persona que era
la verdaderamente interesada en la libertad y la vida de Bartolomé? Y
aquí encajaba la pregunta ansiosa de Lucila: «Dime, Tominillo, dímelo
como si hablaras con Dios; repasa bien tus recuerdos; di si en ellos
encuentras alguna mujer, dama de Palacio, o dama de una casa cualquiera,
que en otro tiempo fue tu amiga, y ahora te protege, nos protege por
mano de Domiciana.»

Revolviendo los más hondos asientos de su memoria, Tomín dijo: «Por más
que cavilo, no encuentro lo que buscas, ni puedo afirmar nada\ldots{}
Aparece, sí, en mis recuerdos alguna mujer\ldots{} ¿Dices que tiene que
ser dama?

---Sí; y de influencia, de mucho poder.

---Pues entonces no\ldots{} No hay nada de eso.

---Busca bien, Tomín\ldots{} Y a falta de dama influyente, ¿no podrías
encontrar alguna monja?

---¿Monja?\ldots{} Eso ya es mas grave. No te diré que no me salga
alguna monja. Pero ello es en tiempos remotos y muy lejos de Madrid,
nada menos que en Mequinenza.

---La distancia no importa.

---Además\ldots{} ahora recuerdo que la monja que entonces conocí,
vamos, que la sacamos del convento entre un amigo y yo, se ha muerto.

---Tú me has contado que de los veintitrés a los veintiocho años fuiste
muy calavera, un galanteador tremendo\ldots{} ¿Entre tantas fechorías de
amor, Tomín mío, no habrá el caso de haber querido a una mujer, de
haberla dejado, como se deja una prenda de ropa que ya no sirve? ¿No
pudo suceder que esa mujer, viéndose despreciada, volviera todos sus
amores a Dios, y escondiera su tristeza en un convento, y allí tomara el
hábito?

---Por Dios, Lucila, haces preguntas y presentas casos que le confunden
a uno\ldots{} No, no: eso es cuento, una novela de Carolinita Coronado o
de Gertrudis Gómez\ldots{} Y si me apuras, no podré negar en conciencia
que exista ese caso\ldots{} ¡Cualquiera sabe si\ldots! Me vuelves
loco\ldots{} Deja, deja que corran los acontecimientos y se cumpla el
Destino\ldots{} ¿Esa dama de Palacio, o esa monja que me protege, han de
ser personas de gran poder?

---Así parece, Tomín\ldots{} No pensaba hablarte de esto; pero ya que ha
salido conversación, sabrás que hoy me ha dicho Domiciana: «Téngase el
buen Gracián por indultado\ldots{} La policía no se meterá con él.» Y
después dijo, dice: «Pero conviene que no salga a la calle todavía. Ya
se le advertirá cuando pueda salir.»

---Pues ¡viva la Libertad! ¡Respiremos, vivamos!---exclamó el Capitán
levantándose como de un salto, y midiendo con mirada de hombre libre la
opresora pequeñez del cuartucho.

Mientras Lucila se abismaba en tenebrosas inquietudes, el Capitán veía
risueños espacios, azules como sus ojos. Hasta muy tarde estuvo
desvelado, sin hablar más que de política, haciendo un formidable pisto
en su cabeza con las ideas propias y las que de su lectura de periódicos
había sacado en aquellos días. «¿No crees tú, Lucila, que este
\emph{Honrado concejo de la Mesta}, como dicen los guasones, viene a
trasquilar al Militarismo, para que le crezca la lana a los cogullas?
Esto es bien claro: se quiere arrumbar a la Tropa para que suba y medre
el cleriguicio\ldots{} Combatir el Militarismo significa quitarle la
espada a la Nación para que no pueda defenderse. ¡El Militarismo! Así
llaman a nuestro imperio, a la fuerza legítima que hemos adquirido
construyendo la España civilizada sobre las ruinas de la retrógrada.
Desde aquí veo yo la gran conspiración militar que se está fraguando en
Madrid y en provincias para volver las cosas a su estado natural: las
armas , los bonetes abajo. Y cuanto más pienso en esto, más me inclino a
relacionarlo con el misterio de las personas desconocidas que miran por
mí. Tu idea de que me protegen monjas o damas de Palacio es un desvarío
de mujer, que no penetra en el fondo de las cosas. Alma mía, aquí no hay
mujerío ni monjío; el socorro y las esperanzas de libertad nos vienen de
mis compañeros de armas agazapados en las logias. En la casa de Tepa
estuvo y está siempre, aunque otra cosa piense y diga la policía, el
centro de la eterna revindicación 2; aquel fuego nunca se apaga; de allí
ha salido la voz que me dice: «Gracián, no desmayes; tus martirios tocan
a su fin. Por ti velamos los leales; no está lejos la hora del
triunfo\ldots» Y no me contradigas, Cigüela del alma, trayéndome otra
vez a colación tu resobada leyenda de la monja y la dama. ¿Sabes tú,
pobrecilla, las ramificaciones que por una y otra parte de la sociedad
tiene nuestra comunidad masónica? ¿Quién te ha dicho que no enlazamos
nuestros hilos con hilos muy finos de conventos y palacios? ¿De dónde
sacas que el señorío y el monjío no se dejan también camelar por los
caballeros \emph{Hijos de la Viuda}? ¡Tonta, más que tonta! ¿Y cuándo ha
sido un disparate, como crees tú, que la misma policía nos pertenezca?
¿Qué han de hacer esos pobres esbirros, sabiendo que ya rondan la casa
de Tepa todos los Generales residentes en Madrid, O'Donnell, Lersundi,
el mismo Figueras, y que D. Ramón Narváez dirige los trabajos desde
París, donde Luis Napoleón le trata a cuerpo de Rey?\ldots{} ¿Dices que
esto es ilusión, locura? ¿Crees que aún tengo la cabeza débil?

---¡Pobrecito mío---exclamó Lucila,---tanto tiempo encerrado en este
nido de murciélagos! Cuando salgas y veas gente, y respires el aire que
todos respiran, pensarás de otro modo.»

Calló el Capitán, no sin que le pusieran en cuidado las últimas palabras
de su amiga. Sentada frente a él, Lucila también callaba, viendo pasar
por su mente, con marcha circular de tío-vivo, una repetida procesión de
monjas y damas. Del propio modo, andando y repitiéndose, iban las velas
colgadas del arillo en el taller del cerero. Sobre las almas del Capitán
y Lucila se posó una nube de tristeza; pero ninguno decía nada. Tomín
rompió el silencio, preguntándole: «¿En qué piensas?

---Bien podrías adivinarlo, \emph{Min}---replicó Lucila.---Pienso que a
los dos no nos protegen, sino a ti solo; a mí, si acaso mientras pueda
sacarte adelante; a mí no más que por el tiempo en que necesites
enfermera\ldots{} Me debes la vida\ldots{} no lo digo por
alabarme\ldots{} pero ¿verdad que me la debes? Una vez asegurada tu
vida, llegará el día en que conozcas a quien hoy mira por ti. ¿Será
monja, será dama? Sea lo que fuere, cuando estés salvo, toda tu gratitud
será para esa persona, todo tu amor para ella\ldots{} ¡\emph{Min}, ay mi
\emph{Min}! y ya no te acordarás de la pobre Cigüela\ldots{} Sí, mi
\emph{Min}, no digas que no.

---Lucila, me matas\ldots{} no sabes el daño que me haces---dijo Gracián
apartándole las manos, que se había llevado al rostro, anegado en
llanto.---¡Olvidarte yo\ldots{} ser yo ingrato contigo! ¡Nunca!\ldots{}
Tú y yo unidos siempre, siempre, unidos en la felicidad como lo hemos
estado en la desgracia.

---No, no\ldots{} Ahora lo crees así, ahora me dices lo que sientes;
pero después\ldots{}

---No hay después que valga. Si eso pudiera ser, téngame Dios toda la
vida en esta miseria\ldots{} Que me cojan, que me fusilen. Muera yo mil
veces antes que separarme de ti, corazón. ¿Qué soy yo sin ti?

---Lo que fuiste antes de conocerme.

---Me acuerdo de lo que fuí, y no quiero ser aquel hombre, no quiero ser
el hombre que no te conocía, que ignoraba la existencia de Lucila. Por
Dios, no tengas esa idea, que es para mí peor que una idea de muerte.
Todas las protectoras del mundo, si es que las hay, no valen lo que mi
ángel. Lucila, no ofendas a tu \emph{Min}, no mates a tu
\emph{Min}\ldots»

Las ternuras que le prodigó, sincero, rendido, con alma, sosegaron a la
enamorada moza, que se secaba las lágrimas diciendo: «Bueno, mi
\emph{Min}, te creo; sí, te creo\ldots{} No te hablo más de eso\ldots{}
ni lo pienso tampoco, mi \emph{Min}, no lo pienso\ldots{} Duérmete,
descansa\ldots»

\hypertarget{xiii}{%
\chapter{XIII}\label{xiii}}

Con las buenas prendas de ropa, nuevas las unas, las otras apenas
usadas, que le iba dando Domiciana, llegó a ponerse Lucila tan bien
apañadita, que daba gloria verla. Si sus extraordinarias gracias
naturales adquirían realce con la pulcritud y el decente atavío, la ropa
puesta sobre tal belleza resultaba mucho más linda y elegante de lo que
era realmente. Por la calle veíase seguida y acosada de mozalbetes, y
por todos requerida de amores. Tenía que cuadrarse a menudo, tomando los
aires de arisca manola, para sacudirse de los señores de \emph{levosa}
(así solían llamar a las levitas) y de los militares de \emph{chistera}
(mote aplicado a los tricornios). Por su parte Domiciana no se
descuidaba, y cada día se iba poniendo más guapetona. Peinábase
lindamente; sus trajes eran elegantes dentro de la sencillez y modestia;
presumiendo de pie pequeño y bonito, calzaba con fineza, y era por fin
extremada en el aseo de su persona. Lucila se maravilló de ver los
variados objetos que en su alcoba y gabinete tenía para la diaria faena
de sus lavatorios.

La confianza entra las dos mujeres crecía y se afianzaba de día en día,
llegando hasta la fraternidad. Domiciana propuso a Lucila que se
tutearan, y así quedó practicado y establecido, hablándose como
compañeras o amigas de la misma edad. En tanto, la exclaustrada
consagraba ya menos tiempo a la preparación de ingredientes de tocador o
de medicina casera, sin llegar al abandono de su industria. La cerería
teníala confiada a Ezequiel y Tomás: iba y venía, contenta y orgullosa,
como el que ve sus facultades aplicadas a un fin práctico con resultado
eficaz. Pero no le faltaban quiebras y desazones, y una de estas era el
continuo asedio del pobre cesante, amigo y azote de la casa, que habíala
tomado por buzón en que diariamente depositaba el eterno memorial de sus
cuitas. D. Mariano era su sombra: le cogía las vueltas en la calle, la
estrechaba en la tienda y trastienda, seguíala con frescura descortés al
sagrado de sus habitaciones particulares, se colaba en el gabinete, y
hasta la sorprendió una vez en papillotes, preparándose para su limpieza
corporal.

---Por Dios, D. Mariano, respete\ldots{}

---Señora, los cesantes no respetamos nada. Somos una plaga española;
somos una enfermedad de la Nación, una especie de sarna, señora mía, y
lo menos que podemos pedir es que se nos oiga, o que se nos rasque.
Ningún español se puede librar de nuestro picor. Óigame usted y
perdone.»

Un día de Marzo, hallándose Lucila y Domiciana en la sala-droguería
ribeteando, con prisas, una capa que habían comprado en corte para
Tomín, se les presentó de improviso Centurión con aquellos modos
serviles y aquel gracejo un tanto cínico que gastaba, y no hubo manera
de quitársele de encima. «Soy yo, la sarna---dijo al entrar, enseñando
en una rasgada sonrisa toda su dentadura.---Vengo a picar, señoras.
Rásquense ustedes; óiganme.

---Vamos D. Marianito---le dijo Domiciana,---que no estaba usted poco
devoto esta mañana en San Justo\ldots{} ya, ya.

---Hay que dar ejemplo, quiero decir, tomarlo. \emph{Sigue las pisadas
de los que van por el recto camino}, cantó el Salmista.

---Y usted no se descuida\ldots{} a un tiempo pica y reza.

---Siento que no me viera usted en la iglesita del nuevo convento de las
señoras Franciscas, calle de Leganitos. Allí me pasé ayer toda la tarde,
y hoy en la Encarnación, donde es Abadesa una prima segunda de mi
esposa, y sobrina del Sr. Tarancón, Obispo de Córdoba, que ahora está en
Madrid\ldots{} Ya me inscribí en dos Cofradías. Pico todo lo que
puedo\ldots{} El maldito Gobierno es el que no se rasca. Y eso que en
todas las sacristías me hago lenguas de la piedad de estos señores
Bravo-Murillistas, que para entenderse con Roma y hacernos un
Concordato, han nombrado Embajador al imponderable D. José del Castillo
y Ayenza. La impiedad habría mandado a un regalista; la ortodoxia manda
al más rabioso de los ultramontanos. Los que tenemos memoria recordamos
que en 1846, cuando se discutió en el Congreso la persona de Castillo y
Ayenza, el Sr.~González Romero le llamó \emph{incapaz} y dijo de él
horrores. Pues este González Romero que era entonces cismontano, como lo
éramos todos los de la cuerda liberal, y hoy se ve encumbrado a la
poltrona de Gracia y Justicia, ha elegido al mismo \emph{incapaz} sujeto
para que vaya a Roma por todo, es decir, por un Concordato. Yo me
felicito: todos hemos venido a ser \emph{ultras}.

---¡Mala lengua!---le dijo Domiciana más compasiva que iracunda,---con
la hiel que usted derrama habría para poner una gran botica de venenos.

---¡Oh! señora, no derramo yo hieles ni venenos, sino cerato simple y
bálsamo tranquilo---replicó Centurión.---Desde que me metí a
\emph{ultra}, me tiene usted consagrado a desmentir todos los rumores
que corren contra el Gobierno, y contra Palacio y el monjío. Voy algunos
ratos a los corrillos de la Puerta del Sol, donde están las peores
lenguas de la cristiandad, y allí pongo de vuelta y media a los
maldicientes. Sabe usted que cada semana tenemos un notición nuevo,
pedazo de carne podrida que se arroja a los pobres cuervos para que
vayan viviendo. La comidilla putrefacta de hoy es esta: Su Majestad el
Rey, que no puede vivir sin visitar cuatro veces al día a las señoras
Franciscas de la calle de Leganitos, se incomoda de que el público le
vea pasar en coche tan a menudo, y de que la guardia de Artillería del
cuartel de San Gil señale su paso con toque de corneta\ldots{} ¿Y qué ha
discurrido para guardar el incógnito? Pues vestirse de clérigo. Así ha
podido hacer de noche sus visitas, atravesando a pie las calles\ldots{}

---Eso es mentira---afirmó indignada la cerera,---y el que tal crea y
diga merece que le emplumen\ldots{} por tonto, más que por malo.

---Ya sé que es falsedad. ¡A quién se lo cuenta! Yo estuve en acecho dos
noches, y no vi nada. Pero hay quien por haberlo soñado, asegura que lo
ha visto. En las tertulias de la Puerta del Sol tenemos mentirosos de
buena fe que le dan a uno espanto\ldots{} Yo me seco la lengua
rebatiendo sus disparates. Hoy, por poco le pego a uno que me sostenía
con toda formalidad que el Rey se entretiene en jugar a la gallina ciega
con las novicias\ldots{}

---¡Vaya un disparate! Hace usted bien en destripar esos cuentos
ridículos. Pique usted, Hermano Centurión, pique por ese lado y se le
hará justicia.

---Hermana Domiciana, yo pico; pero la justicia no llega. Ya dije a
usted en qué paso mis tardes: a prima noche me tiene usted en los
ejercicios de Italianos o Bóveda de San Ginés, alternando, y de allí me
voy a mi casa. Nadie me verá en teatros, cafés, ni alrededor de las
mesas de billar. En mi casa trabajo a moco de candil. Consagro los ratos
de la noche a la Poesía, con quien algún trato tuve en mi mocedad. No me
faltaba lo que llamamos estro\ldots{} Dirá usted que quién me mete a
poeta, y yo contesto que si somos plaga, seámoslo en todos los terrenos.
¿Ha observado usted que los poetas del día no se tienen por tales si no
enjaretan una o más composiciones religiosas? Los viejos, los de mediana
edad, y aun los jovencitos que rompen el cascarón retórico, nos regalan
cada día, bien la \emph{Oda al Santísimo Sacramento}, bien la
\emph{Canción a los Reyes Magos}, este \emph{Octavas reales a San José
bendito}, el otro \emph{Quintillas a la Creación del Mundo,} cuando no
un \emph{Romance a los Misterios gozosos de Nuestra Señora}\ldots{}
¡Pues no han sido poco celebradas las composiciones de mi amigo Baralt
\emph{A la Encarnación,} y de Cañete a la \emph{Transfiguración del
Señor}! Pero a todos supera el numen del insigne poeta D. Joaquín José
Cervino, que en su \emph{Oda a las Bodas de Caná}, refiere el milagro
con estos rotundos versos:

\small
\newlength\mlena
\settowidth\mlena{Mira en licor de Engadi convertida.}
\begin{center}
\parbox{\mlena}{\textit{Ya linfa en hidria pura contenida,           \\
                        Mira en licor de Engadi convertida.}}        \\
\end{center}
\normalsize

»Pues los chicuelos que empiezan ahora, como Pepe Selgas y Antonio
Arnao, también enjaretan su metrificación correspondiente sobre un tema
religioso. Hasta los padres graves de la pasada era romántica, los
Hartzenbusch, los García Gutiérrez, los próceres como Saavedra y Roca de
Togores, y el grandullón D. Juan Nicasio, se descuelgan con su
\emph{Silva al Sacrificio de Isaac}, o con un lindo \emph{Panegírico de
la Pentecostés} en alejandrinos. ¿Qué es esto más que una señal de los
tiempos? No vivirían los poetas si no se arrimaran a los pesebres del
Estado, y como el Estado es hoy manos y pies invisibles del cuerpo de la
Iglesia, que tiene su visible cabeza en Roma, todos los jóvenes y viejos
que andan por el mundo con una lira a cuestas, o la tocan para Dios y
los Santos o no comen\ldots{} Vea usted por dónde yo he resucitado mis
antiguas debilidades poéticas; desempolvo mi lira y poniéndole cuerdas
nuevas, me lanzo a tocarla con plectro y todo, y endilgo mi \emph{Canto
Épico} al \emph{Centurión Cornelio}\ldots{} En la invocación a la
\emph{Musa Cristiana} para que me sople, doy a entender que de aquel
romano Centurión procede mi familia, y que por ello estoy obligado a
cantarle con tanta gratitud como entusiasmo y fe. En \emph{La Patria}
podrá usted leer mi \emph{Canto Épico}. He preferido este periódico
porque es el que viene pegando fuerte a Narváez, Sartorius y a toda la
fracción caída, que ha tenido en tal desamparo a la religión y sus
ministros. ¿Ha leído usted lo que dice del donativo que hizo la Reina a
Narváez?

---No leo periódicos, D. Mariano, ni me importa lo que digan.

---Pues el regalito fue de ocho millones, para que pudiera vivir con
decorosa independencia el hombre que\ldots{} Hoy hablaban de esto en la
Puerta del Sol, y allí hubo quien, echando fuego por los ojos y ácido
prúsico por la boca, hizo la cuenta del número de cocidos que con esos
ocho millones se podrían poner en un año, para los tantos y cuantos
españoles que se pasan la vida ladrando de hambre\ldots»

Cansadas del insufrible lamentar de aquel mendigo de levita, Lucila y
Domiciana le miraban esperando un punto, o punto y coma, en que pudieran
meter cuña para despedirle. «Hermano Centurión---dijo al fin la
cerera,---acabe ya y déjenos, que tenemos que hablar las dos de nuestras
cosas, y con su salmodia nos ha levantado jaqueca.

---Como benemérita plaga española que soy, Hermana Domiciana, no tiene
usted más remedio que sufrirme\ldots{} No puedo retirarme mientras no
ponga en conocimiento de usted algo que debe saber, y para que lo sepa
he venido hoy aquí.

---¡Pues hubiera empezado por eso, Santa Bárbara!

---¡San Caralampio! Yo empiezo por el fin y acabo por el principio, a
causa de tener mi pobre cerebro del revés, como es uso entre
cesantes\ldots{} Vamos al caso: usted sabe que la Madre Patrocinio bebía
los vientos por destituir al señor Patriarca de las Indias, D. Antonio
Posadas Rubín de Celis\ldots{} Nunca le perdonó a este señor que se
burlara de las llagas, y las calificara, como las calificó, de
\emph{farsa indigna de una nación católica}\ldots{} El odio de Su
Caridad levantó gran polvareda contra el Prelado, por si era o no era de
la cáscara amarga. Se decía que en 1823, gobernando la diócesis de
Cartagena, renunció la mitra y se fue a la emigración por no bajar la
cabeza ante el absolutismo\ldots{} Esto le imputaban, y de tal modo
atronaron los oídos del Rey y de la Reina, que al fin\ldots{} usted lo
sabe\ldots{} le largaron el cese al Patriarca, y en su lugar fue
nombrado D. Nicolás Luis de Lezo, confesor de la Reina Madre, el cual,
se endilgó la sotana morada, antes que de Roma vinieran las
Bulas\ldots{} Usted sabe que lo que vino de Roma fue un soberano
rapapolvo desaprobando todo lo hecho, y confirmando en su puesto al
Sr.~Posada y Rubín de Celis\ldots{} Usted sabe que\ldots{}

---Ya me está usted estomagando con si sé o no sé---dijo Domiciana.---Lo
que yo sepa o ignore no es cuenta de nadie.

---Todo el mundo sabe que el Sr.~Lezo, a pesar del rapapolvo, siguió con
su capisayo morado, tan guapín, olvidando que ni es Obispo ni nada.
Nuestra graciosa Reina, que de niña era muy salada, puedo dar fe de
ello, y de mujer es la sal misma, le puso a D. Nicolás Luis un mote
graciosísimo\ldots{} usted lo sabe: \emph{el Obispo de Farsalia}\ldots{}

---Bueno, ¿y qué?

---La señora \emph{Madre} aguantó el cachete, por venir de Roma, y
esperó; el señor Patriarca, ya muy viejecito, no podía ser
eterno\ldots{} Al fin se lo ha llevado Dios: ya está vacante el puesto.
Y ahora, Hermana Domiciana, yo le pregunto a usted por si quiere
decírmelo: ¿Sabe quién será el nuevo Patriarca?

---No lo sé, ni aunque lo supiera se lo diría.

---Porque la Reina Cristina hace hincapié por su candidato, el de
Farsalia; el Infante D. Francisco se interesa por el Padre
Cirilo\ldots{} y el Gobierno\ldots{} Esta es la noticia que le traigo a
usted, noticia que aparejada lleva una preguntita. El Gobierno propone
al señor D. Joaquín Tarancón, Obispo de Córdoba, que,

como he dicho antes, es tío de la señora Abadesa de la Encarnación. Me
consta que una gran parte de lo que podríamos llamar \emph{elemento
eclesiástico} vería con gusto al Sr.~Tarancón en el Patriarcado de
Indias, y yo\ldots{} no le digo a usted nada: casi, casi es mi
pariente\ldots{} Pues ahora viene la pregunta: ¿Sabe usted quién es el
candidato de la \emph{Madre}? Porque yo me pongo a discurrir y digo: «O
hay lógica o no hay lógica. Si un Gobierno que tiene por dogma la
perfecta obediencia a las soberanas voluntades que nos rigen, se lanza a
proponer a Tarancón, ¿no quiere esto decir que Tarancón es grato a la
\emph{Madre}, y por ende a la \emph{Hija}, o en otros términos, que
Tarancón es el candidato del \emph{Altar y el Trono}, como decimos los
ultras?\ldots» Con que, Hermana Domiciana, dígame si esto que pienso es
la verdad, o si me falla la dialéctica; dígamelo, y hará un gran favor a
un padre de familia con mujer y siete criaturas. ¿Es D. Joaquín Tarancón
candidato de la \emph{Madre}?\ldots{} Porque si lo es, \emph{Patriarcam
habemus.»}

Nerviosa y un tanto descompuesta le contestó Domiciana que no tenía nada
que contestarle ni que decirle, como no fuera que tomara la puerta y se
largase con sus historias a la del Sol, o a cualquier mentidero.
Lastimado en lo vivo D. Mariano, levantose afectando dignidad y dio
algunos pasos hacia la salida. Mas no quiso irse sin venganza de aquel
desprecio: calose el sombrero, requirió las solapas del levitón, y en
actitud un tanto estatuaria, con temblor de la mandíbula y ronquera de
la voz, se dejó decir: «Por no ser amable y franca, usted pierde más que
yo, porque no le doy una noticia tremenda\ldots{} noticia de un suceso
reservado, que lo será por algún tiempo todavía\ldots{} suceso\ldots{}
noticia de un valor que usted no puede figurarse\ldots{} y que ignora
todo Madrid, menos unos cuantos\ldots{} y yo. En castigo, no se lo digo,
no. Fastídiese, rabie.

---¿También de monjas y Patriarcas?

---No\ldots{} Es cosa militar\ldots---dijo Centurión escurriéndose a lo
largo del pasillo.---Cosa militar\ldots{} gravísima\ldots{} y no lo
sabe\ldots{} Fastidiarse\ldots»

Domiciana corrió tras él murmurando: «Hermano, aguarde, oiga\ldots»

Pero, él, impávido, desapareció en la obscuridad de la angosta escalera
repitiendo: «No digo nada, nada\ldots{} Fastídiese, rásquese\ldots{}
Cosa militar\ldots»

\hypertarget{xiv}{%
\chapter{XIV}\label{xiv}}

---¡Pero este D. Mariano\ldots! ¿No te parece que está loco?\ldots{} Y
esa noticia de militares, ¿qué será?\ldots{} ¿Pues sabes que me ha
dejado perpleja y con ganas furiosas de saber\ldots? Es un perro\ldots{}
Cuando le da por callar, molesta más que cuando nos aturde con sus
ladridos\ldots{} No me sorprende que sepa cosas muy reservadas\ldots{}
Estos cesantes rabiosos se meten en todos los rincones para olfatear lo
que se guisa, y lo mismo entran en las sacristías que en las
logias\ldots{} Cosa de militares dijo. ¿Será alguna intentona?\ldots{}
¿Tendremos en puerta un pronunciamiento\ldots?»

Esto decía Domiciana, en frases desgranadas, que revelaban su inquietud.
Con paso inseguro fue de la puerta del pasillo a la ventana; miró a la
calle, y al ver que en efecto salía Centurión, volvió junto a su amiga
rezongando: «Disparado sale\ldots{} Va echando demonios\ldots{} Esa cosa
militar ¿qué será? ¿Tú qué crees, Lucila?

---¿Yo qué he de creer?\ldots{} Ya te sacará de dudas D. Mariano cuando
vuelva.

Se puso en pie presentando la capa colgada de sus manos para que
Domiciana viese el efecto del embozo. Resultaba muy bien, una prenda
seria y poco llamativa, como para persona que durante algún tiempo no
debía salir al público sin cierta reserva. Ya no faltaba más que acabar
de coser la trencilla: la infatigable costurera se sentó de nuevo para
proseguir la obra por una banda, mientras Domiciana trabajaba por otra.

---Sí que volverá, creo yo---dijo la cerera,---y si no vuelve le mandaré
venir con cualquier pretexto. Hablemos de nuestras cosas.

Antes de la entrada de Centurión había Lucila dado cuenta a su amiga de
las buenas condiciones de la vivienda en que se había instalado con
Tomín. Se creían transportados de un infierno aéreo a un cielo
terrestre. Con frase menos sintética y más vulgar expresó Lucila su
pensamiento\ldots{} Satisfecho estaba Tomín; sus ojos, hechos a la
miseria desoladora, veían los vulgares muebles revestidos de una dorada
magnificencia. Ya recobraba el apetito y el buen color; pero la
inmovilidad a que todavía se le condenaba le tenía en gran desasosiego.
¿Cuándo podría salir?

Un rato tardó Domiciana en contestar a esta pregunta. Sin apartar los
ojos del mete y saca de la diligente aguja, alargando los morros, dijo:
«¿Qué prisa tiene? Ya saldrá. Conviene esperar un poco, hasta ver en qué
para eso del Patriarca\ldots»

La aguja de Lucila se paró, señalando el zenit. Turbación, estupor y
silencio grave en el ambiente que mediaba entre las dos mujeres. ¿Qué
tenía que ver la libertad de Gracián con el nombramiento del Patriarca
de las Indias?

---Todas las cosas de este mundo---dijo Domiciana sin mirar a la
otra,---vienen y van más enzarzadas de lo que parece. ¡Con este lío del
Patriarca hay cada disgusto\ldots! A donde quiera que una va, encuentra
caras malhumoradas\ldots{} y se oyen cosas que\ldots{} Parece que están
todos locos\ldots{}

Conoció Lucila que no gustaba su amiga de aquella conversación, y puso
punto en boca. Nombrando de nuevo a Tomín, Domiciana fue a parar a la
promesa de solicitar para él el alta en el escalafón, y si las cartas
venían bien dadas, un ascenso y pase al servicio activo. «En este
caso---añadió,---¿crees que Gracián se casará contigo?» «¡Ay, no lo sé!»
fue la única respuesta de Lucila. La cerera prosiguió así: «Nada tendría
de particular que, volviendo las cosas a su nivel, Tomín se casara con
una persona de su clase\ldots{} Tú entonces, poniéndote en lo razonable,
podrías casarte con un hombre de tu clase, y serías feliz\ldots{} No te
faltarían protección y ayuda en tu matrimonio.» Dijo a esto Cigüela que
tenía por absurdas las ideas de su amiga, y repitió su antigua cantinela
de que Domiciana no entendía palotada de amor, y que continuaba tan
muerta y amojamada como cuando salió del convento.

---Estás en lo cierto---replicó la exclaustrada,---y siempre que hablo
de amores, o de amoríos, salen de mi boca desatinos muy garrafales; tan
ignorante soy. Conozco al hombre en diferentes condiciones de vida: en
la condición de avaro, de hipócrita, de cortesano, de astuto, de
religioso; en la condición de enamorado no le he conocido nunca, ni sé
lo que es. ¿Quieres más explicaciones? Pues allá van. Se ha dicho, y tú
lo habrás oído tal vez, que yo me metí en el convento por desesperada de
amor\ldots{} la historia de siempre\ldots{} un novio ingrato, una niña
tonta que se vuelve mística\ldots{} Pues no creas nada de eso, si de mí
lo has oído. En mi caso no hubo despecho amoroso. Yo me arrojé al
convento huyendo de mi madrastra, como me habría tirado a un pozo;
llegué a la vida religiosa enteramente ayuna de amores, sin haber tenido
novio, pues los que algo me dijeron no me interesaron nada, ni nunca les
hice caso. Entró, pues, en el claustro mi corazón nuevecito, intacto de
pasiones, y sin saber lo que era eso. Allí dentro, ya puedes suponer que
no amé más que las hierbas. Naturalmente, como tú has dicho, con aquel
vivir me marchité\ldots{} El amar vago, que lo mismo puede fijarse en
Dios que en personas de la Naturaleza, se fue secando; la voluntad se me
iba tras de las cosas, no tras del hombre\ldots{} Cuando salí, ya no era
tiempo de pensar en melindres, ni me lo permitían los votos que hice y
de que no estaré nunca desligada\ldots{} Otra cosa te diré para que lo
comprendas mejor. Yo, que soy bastante despierta, sé desenvolverme muy
bien en una reunión de hombres, si tengo que departir con ellos de
cualquier asunto; pero delante de un hombre solo, me entra tal cortedad,
que no acierto ni a decir Jesús. Me ha sucedido alguna vez encontrarme
sola frente a un hombre, el cual con la mayor inocencia y sin asomo de
malicia venía para tratar conmigo de un negocio de cerería, o de
hierbas, o de qué sé yo\ldots{} ¿Pues sabes lo que me ha pasado? Que al
verme sola con él me ha entrado no sé qué desazón\ldots{} algo de
repugnancia primero, de espanto después\ldots{} y al fin no he tenido
más remedio que echar a correr, dejándole con la palabra en la
boca\ldots{} Luego he tenido que mandarle recado, diciéndole que me
dispensara, que me había puesto mala\ldots{} Así soy yo. Y de veras te
digo que muertecita está una mejor. ¿No crees que es una bendición ser
así, y estar asegurada contra lo que, en mis circunstancias, no había de
ser bueno?

Contestó afirmativamente Lucila, añadiendo que no se diera por asegurada
tan pronto, pues no era vieja y conservaba lozanía, frescura\ldots{}
Recaída la conversación en Tomín y en las probabilidades de que
reanudara pronto su brillante carrera, conquistando los puestos más
altos, Domiciana preguntó a su amiga si era valiente el Capitán, de
temple duro para la guerra.

---¿Que si es valiente?---dijo Lucila dando a su palabra tonos de
entusiasmo.---Por mucho que yo te diga sobre eso no podrás tener idea de
la bravura de Tomín y de su fiereza ante cualquier peligro. Cuando se
emborracha de valor, no repara en si es uno o son veinte los enemigos
que tiene delante.

---¡Vaya, vaya! Ahora que me acuerdo: me has dicho que es extremeño, de
la tierra de Hernán Cortés y Pizarro. Todos los hijos de Extremadura son
arrojados y valientes, menos mi padre que no es héroe más que para
casarse.

---En Medellín, como ese Cortés, nació mi Bartolomé Gracián. Su padre,
caballero noble, tiene allí bastante hacienda, ganados\ldots{} Su madre,
que todavía no es vieja, conserva una hermosura superior. De ella sacó
Tomín los ojos azules; de su padre la tez trigueña. El hermano mayor no
se separa de los padres, y es el que hoy lleva las labores. Dos
hermanitas tiernas siguen a Tomín. A este, desde muy niño, le llamaba la
milicia; no jugaba más que a soldados, y él era el que a los otros
chicos mandaba, llevándolos a correrías del diablo, asaltos de peñas,
porfías de unos bandos con otros\ldots{} Entró en el ejército el año 45,
si no estoy equivocada, y al poco tiempo estuvo en la guerra de Cataluña
contra Tristany y Cabrera. Las acciones en que peleó, las heridas que
pusieron su cuerpo como una criba, y las hazañas\ldots{} porque hazañas
fueron\ldots{} que hizo él solo, escritas están en alguna parte, no sé
donde\ldots{} y pueden dar fe de ellas el General Concha, el General
Pavía, y un sin fin de brigadieres, coroneles y capitanes. Ya desde la
guerra de Cataluña venía Tolomé, según él mismo me contó, dado a la
política, con la cabeza encendida en esas cosas del Patriotismo y de la
Libertad, y no se mordía la lengua para despotricar contra los
Absolutos, Carlinos, Moderados, y demás gente que no quiere
Constitución, sino Cadenas; en Madrid se hizo amigo de otros que
aborrecían las Cadenas, y todos juntos iban a unas reuniones escondidas
en no sé qué calle, y hacían allí sus santiguaciones, que paraban
siempre en armar algún enredo para salir con los soldados gritando
\emph{Libertad, Viva esto, Abajo lo otro}. Hubo en Madrid hace dos años
o tres\ldots{} no me acuerdo de la fecha\ldots{} una trifulca muy grande
en la Plaza Mayor, tropa y pueblo contra tropa. Tolomín estaba en un
regimiento que también debió salir sublevado, y no salió porque se
echaron encima generales y jefes, y ello quedó así\ldots{} Pero a mi
hombre le traían ya entre ojos; su fama de liberal y algún mal querer de
envidiosos o soplones le perdieron. Formada sumaria, le mandaron con dos
tenientes a las Peñas de San Pedro, que es allá en la Mancha, pueblo con
fortaleza, y fortaleza sin pueblo, como dice Tolomé. Mal lo pasaban allí
los pobres oficiales deportados; pero Tomín, que es poco sufrido, y uno
de los tenientes llamado Castillejo, de la piel del diablo, tramaron una
conjura, en la que fueron entrando algunos más, para revolverse contra
la guarnición y hacerse dueños de la plaza. Me ha contado él que todo lo
hicieron con más arrojo que picardía, y que fue como si dijeran: «morir
matando antes que morirnos de tristeza.» La noche que intentaron
desarmar a la guarnición fue noche de tronada y rayos en los aires, en
la tierra de furor y rabia de los hombres. Muertos y heridos
muchos\ldots{} sangre corriendo\ldots{} compasión ninguna. Pudo más la
guarnición, y Tomín y Castillejo, viendo perdida la batalla, escaparon
favorecidos del diluvio que empezó a caer cuando unos y otros, como
demonios, se estaban matando\ldots{} Huyeron armados; con algún dinero
que llevaban se procuraron disfraz, caballerías, y por atajos, como Dios
quiso, se vinieron a Madrid. Aquí se ocultaron, y escondidos supieron
que estaban condenados a muerte por Consejo de Guerra\ldots{} Pues
Señor: vino a parar Tomín a casa de un tratante en leñas, José
Perdiguero, amigo de su familia, calle Segovia, y para más disimulo de
su escondite, le pusieron su vivienda en el mismo almacén de las leñas,
que da a un patio grande, y en aquel patio vivía yo con mi padre y mi
hermano chico\ldots{}

---Y allí os conocisteis\ldots{} Vamos, ya llegó ese momento de la
historia, que yo esperaba. Sigue, que ahora entra lo más bonito.

---Bien feo era aquel patio, y más el almacén; pero a mí me pareció lo
más bonito del mundo\ldots{} te lo digo como lo siento: hablo con el
alma, Domiciana. Bonito fue todo, cuando vi a Tomín tendido sobre un
montón de leña, y más cuando él me preguntó cómo me llamo\ldots{} El
amo, que tenía que salir al campo, me mandó que hiciera cena para aquel
hombre\ldots{} «para este caballero,» fue como me dijo. Hice la cena, y
el caballero se negó a comerla si no cenaba yo con él. Yo dije que
bueno. Me sentí, puedes creérmelo, arrebatada de una compasión que me
encendía toda el alma, y apenas empezamos a cenar, aquel señor me dijo:
«Soy muy desgraciado.» Obsequioso y fino estuvo conmigo, sin decirme
cosa ninguna que pudiera ofenderme\ldots{} Yo le miraba, y cuando él me
miraba a mí, tenía yo que bajar los ojos\ldots{}

---De veras va siendo muy interesante la historia. Sigue, y no suprimas
nada.

---Dos días pasaron así. Mi padre y mi hermanillo salían de sol a sol.
El caballero y yo hablábamos, él a la otra parte del rimero de leña, yo
a la parte de acá. Charla que charla, las tardes se me hacían horas, las
horas minutos\ldots{} Yo estaba como en la gloria.

---¡Y que no te diría cosas poco tiernas y\ldots!

---Me decía\ldots{} qué sé yo\ldots{} vamos, lo contaré todo. Me decía
que soy muy guapa\ldots{} Esto lo había oído mil veces; pero nunca hice
caso. Me decía que soy bella, bellísima\ldots{} y más, más me decía.

---No lo calles, mujer.

---Que soy hechicera\ldots{} y otra cosa más bonita\ldots{}

---Dímela.

---Que tengo un alma más bella que mis ojos\ldots{} Por mis ojos me veía
él el alma\ldots{} Yo también le veía la suya\ldots{} Por fin me dijo
que le quisiera, que le haría un gran bien queriéndole\ldots{} que
estaba dejado de la mano de Dios\ldots{} Contome toda su historia: yo
temblaba oyéndole. Luego me dijo que era para él grave caso de
conciencia pedirme mi amor, y que antes de responder yo a su petición de
amor, debía saber una cosa terrible: estaba condenado a muerte. ¿Sería
yo capaz de amar a un condenado a muerte?

---Al pronto, y poniendo por delante un poco de puntillo, responderías
que no.

---¡Ay, Domiciana! Si me hubiera dicho «soy rico, feliz, poderoso,» le
hubiera contestado con puntillo diciendo que no, que veríamos y qué sé
yo\ldots{} Pero diciéndome «soy condenado a muerte» le contesté que sí,
con el alma, y me fui hacia él\ldots{} ¡Ay, Domiciana, qué paso!\ldots{}
Llorando me abrazó Tomín, y yo le dije: «Que nos fusilen juntos.»

\hypertarget{xv}{%
\chapter{XV}\label{xv}}

Dejando correr, en una pausa breve, lágrimas dulces, lágrimas amargas,
continuó Lucila su triste historia, que en algunos puntos más le causaba
gozo que pena; siguió por terreno a veces llano, a veces escabroso, sin
esquivar los pasos al borde del precipicio, incitada por la cerera, que
le pedía sinceridad, franqueza gallarda. Contó las querellas que con su
padre tuvo por el amor de Tolomín; cómo estas desavenencias la separaron
al fin de Ansúrez y del hermano pequeño (el cual en aquellos días entró
de aprendiz en el taller de unos boteros de la calle de Segovia, amigos
de su padre); cómo unió su suerte a la del Capitán, locamente enamorada
y obedeciendo a una fuerza imperiosa, irresistible; cómo fueron
obsequiados por \emph{el Ramos}, manolo viejo de ideas revolucionarias,
retirado de la patriotería activa y enriquecido en su comercio de
maderas viejas; cómo hallándose un día refrescando con \emph{el Ramos}
en cierta botillería de la calle de los Abades, se les apareció el
Teniente Castillejo emparejado con la viuda de un capitán, y cómo, en
fin, los cuatro se fueron a vivir a un piso quinto, en la calle del
Azotado, con anchura de local y estrechez grande de recursos. A poco de
instalarse les sorprendió de madrugada la policía, cuando estaban en el
primer sueño, pues nunca se acostaban hasta después de media noche. A
tiros y sablazos les atacaron tres hombres. Defendiéronse Tomín y el
Teniente con gran coraje; mataron a uno; los otros dos tuvieron que huir
en busca de refuerzo. Antes que volvieran, Castillejo y la Capitana se
bajaron al segundo piso. Tomín fue más previsor: a pesar de hallarse
herido de arma blanca en una pierna, de arma de fuego en un brazo,
escapó por la bohardilla al tejado vecino, pudiendo descolgarse de un
modo casi milagroso al patio de una posada de la Cava Baja. Lucila en
tanto cogió calle más pronto que la vista, corrió a la posada y ayudó a
su Tomín a escabullirse por la calle de Segovia abajo; tomaron resuello
en un corralón de la Cuesta de Caños Viejos, y allí le vendó como pudo
las heridas para contener la sangre. La situación era en extremo
apurada. Gracián no podía valerse. Con rápida iniciativa ante el
peligro, corrió Lucila en busca de \emph{el Ramos}, única persona de
quien podía esperar socorro, y el patriotero jubilado no desmintió en
aquel caso su magnánimo corazón, ni su abolengo de sectario
constitucional que había vestido el glorioso uniforme de la Milicia
Urbana. Al amanecer, en un carro de cueros fue transportado el Capitán a
la calle de Rodas. Sin que nadie le viese, fue subido al \emph{nido de
murciélagos}, lugar al parecer distante del acecho policiaco, y allí
quedó entre los gatos y el cielo, asistido de su fiel amiga, que con su
cuidado y ternura le sostuvo el alma para que no cayese en la
desesperación, atajó la muerte, aseguró la vida, y restituyó a la
sociedad el hombre que esta había cruelmente repudiado.

---Del valor de Gracián---dijo Domiciana, oída con tanto respeto como
admiración la dramática historia,---nadie podrá dudar. Pero si él es
bravo, más brava eres tú. Te has portado como mujer heroica, y aunque
has pecado, creo yo que Dios te perdonará.

Lo más que hablaron aquella tarde careció de interés. Partió Lucila con
la capa sin terminar, proponiéndose rematarla por la noche en su casa.
Fue Domiciana con Ezequiel a San Justo, a la novena de San José, y allí
vio a Centurión, que no se acercó, como de costumbre, a cotorrear con
ella; tampoco la cerera hizo por él, ni quiso mostrar ganas de
conversación. Ezequiel pasó a la sacristía, donde tenía más de un amigo,
y solía ayudar al culto, bien endilgándose la sotana como turiferario,
bien subiéndose al coro para cantar un poco con voz angélica,
desafinadita. Habló un rato la cerera con un clérigo que en San Justo
decía misa y confesaba, D. Martín Merino, hombre impasible, de una
frialdad estatuaria. A Domiciana le agradaba el tal sacerdote por la
sequedad cortante con que expresaba sus pareceres, ya en cosas de
religión, ya cuando por incidencia hablaba de política. Le tenía por
hombre entero, de arraigadas convicciones, de notoria austeridad en sus
costumbres. «¿Viene usted a la novena, D. Martín?---le preguntó. Y él:
«No, señora: yo salgo; he venido a ajustar una cuenta. \emph{Aquí no
toco pito} esta noche; me voy a mi casa, donde tengo mucho que hacer.» Y
tomó la puerta. Chocó a Domiciana la escueta familiaridad de la frase
\emph{no toco pito}; y como el hombre solía ser tan áspero en cuanto
decía, resultaba de un gracejo fúnebre en sus labios secos la expresiva
locución\ldots{} Terminada la novena, volvió la cerera con Ezequiel a su
casa; cenaron, y de sobremesa, solos, porque D. Gabino con el último
bocado solía coger el sueño y se quedaba cuajadito en un sillón,
hablaron del cumplimiento de ciertas comisiones encargadas aquella tarde
al bendito mancebo. «Llevé el lío de ropa y los cuatro libros, y todo lo
entregué al señor, en su mano---dijo Ezequiel.---Lucila no estaba en
casa.

---¿Y el señor qué tal te recibió? ¿Es amable, de buena presencia?

---Tan buena, que se me pareció a Nuestro Señor Jesucristo.

---Eso no puede ser. A Nuestro Señor no puede parecerse ningún mortal,
por hermoso que sea.

---Dices bien, y ahora caigo en que más que a Dios se parece al Buen
Ladrón. ¿Has visto el Buen Ladrón del Calvario de San Millán\ldots{}
clavado en la Cruz, y guapo él?

---¿El caballero de Lucila tiene barba?

---Sí: una barba corta y bonita\ldots{} como la del San Martín que parte
su capa con el pobre.

---¿Y reparaste en el color de los ojos?

---No reparé el color; pero sí que tiene un mirar que no se parece a
ningún mirar de persona.

---¿Qué dices, Ezequiel?

---Digo que ningún mirar de hombre es como el de ese señor.

---¿Serán sus ojos como de oro\ldots{} como de plata?

---Como de plata y oro en derredor de una esmeralda.

---Luego, son verdes.

---No te puedo decir que sean verdes; pero algo tienen, sí, de piedras
preciosas.

---¿Serán\ldots{} así por el estilo de la piedra que llevaba en su
anillo el señor Obispo que ofició en San Justo el día de la Candelaria?

---No, mujer\ldots{} No hay ojos de persona que sean de ese color que
dices\ldots{}

---Pues entonces, Ezequiel, serán azules\ldots{} ¿Has visto tú esa
piedra que llaman zafiro?

---No\ldots{} En el talco es donde yo aprendo los colores. El talco
azul, si lo pones en cera que no sea muy blanca, se te vuelve verde.

---Y dime otra cosa: cuando le diste a ese señor los libros, ¿qué hizo?
¿se alegró?

---Leyó el forro y no dijo nada. Se levantó y fue a ponerlos en la
cómoda.

---¿Notaste si al andar cojeaba? ¿Es airoso, es gallardo?

---Me parece que sí. El juego de piernas, andando, es de militar,
¿sabes?\ldots{} Cogió de la cómoda cigarros, como cojo yo mi
cachucha\ldots{} sin reparar\ldots{} y vino a mí ofreciéndome uno. Yo le
dije que no fumo. Él fumó echando el humo muy para , muy para \ldots{}
Luego me preguntó si seguía yo la carrera eclesiástica\ldots{} y yo
respondí que eso quiere mi padre\ldots{} pero que mi hermana, tú,
quieres que estudie para abogado\ldots{} Pues él dijo que es preciso ser
militar o abogado\ldots{} y que todo lo demás es vagancia pura\ldots{}
Habías de oírle, Domiciana: que todo está muy malo, y que tenemos aquí
mucha tiranía, mucho obscurantismo y muchísima inquisición\ldots{} De
repente, dejó caer la mano con que accionaba, dándose tan fuerte
palmetazo en la rodilla, que yo\ldots{} salté en mi taburete. Me asusté
del golpe y de los ojos que el caballero puso.

---¿Es hombre de mal genio?

---De genio muy fuerte\ldots{} ¡Pobre del enemigo que coja por delante,
en una guerra, o en una revolución!

---¿Crees tú que pega?

---¡Vaya! Creo que pega a todo el mundo menos a Lucila.

---¿Y quién te asegura que no pega también a Lucila?

---No, eso no\ldots{} ¡La quiere tanto!---dijo Ezequiel echando a
torrentes de sus ojos la infantil ingenuidad.

---Por eso, porque la quiere\ldots{} Los hombres pegan y las mujeres
lloran\ldots{} Eso es el amor, según dicen\ldots{}

---Así será en los matrimonios disolutos.

---Y en todos, Ezequiel\ldots{} y el llorar y el pegar no quitan para
que traigan al mundo la familia\ldots»

Aquí paró la conversación. Ezequiel tiraba de sus párpados, que el sueño
quería cerrarle. Domiciana le mandó que se acostara, pues había que
madrugar. Al siguiente día comenzaban las grandes tareas cereras para
Semana Santa\ldots{} Cerrada la tienda y apagadas las luces, la casa no
tardó en quedar en silencio, turbado sólo por el áspero roncar de D.
Gabino. Domiciana, recogida en su aposento, empezó a desnudarse. En
aquella hora inicial del descanso nocturno, en que el silencio y la
calma derraman tanta claridad sobre las cosas próximamente transcurridas
y sobre las futuras que no están lejanas, la cerera reunía sus ideas
dispersas, sintetizaba, expurgaba, desechando lo inútil, y como un hábil
general distribuía sus mentales fuerzas para las batallas del siguiente
día. Resumiendo sus impresiones de los hechos recientes y adivinando las
que muy pronto habría de recibir, echó a rodar estos pensamientos sobre
el fino lienzo de sus almohadas: «No habrá mañana poco tumulto en la
\emph{casa grande} cuando llegue yo y suelte la bomba\ldots{} la bomba
escrita y la bomba parlada por mi boca, diciendo: «No hay más Patriarca
de las Indias que el Sr.~D. Tomás Iglesias y Barcones\ldots» ¡Y luego me
hablan a mí de la cuestión de Oriente! ¿Qué tienen que ver la cuestión
del Oriente ni la del Occidente con la cuestión Patriarcal?\ldots{} A
Bravo Murillo se le ha metido en la cabeza que Tarancón es grato a la
\emph{Madre}, porque así se lo dijeron el Marqués de Miraflores y el
mismo Sr.~González Romero\ldots{} Pero estos son de los que no se
enteran de nada, y cuando desean una cosa se forjan la ilusión de que
los demás también la quieren\ldots{} ¡Valiente ganado el de los
caballeros políticos!\ldots{} Andad, andad, hijos, por donde os llevan
vuestros pastores, y no salgáis del caminito que se os marca\ldots{}
Duro ha de ser para la Reina decirle a D. Juan: «Mira, Juan, ese
nombramiento que traes a favor de Tarancón, te lo guardas y haces de él
lo que quieras\ldots{} No has de ser más que mi madre, y a mi madre
tengo que decirle también que se guarde su candidato, el pomposo
Sr.~Lezo, a quien yo, por mí y ante mí, nombré \emph{Obispo de
Farsalia}\ldots{} Ni has de querer compararte con mi tío D. Francisco de
Paula, que me traía puesto en salmuera para Patriarca al Padre Cirilo, y
también tiene que guardárselo para mejor ocasión. Patriarca de las
Indias será D. Tomás Iglesias y Barcones, y no se hable más del asunto.»
Esto le dirán, y D. Juan se irá a comer calladito sus chorizos, y a
discurrir, para cuando se desocupe del arreglo de la Deuda, la reforma
de la Constitución, dejándola en los puros huesos\ldots»

Y ya cogiendo el sueño, apagadas las ideas, dispersas las imágenes, las
recogió de la blanca almohada para dormir con ellas: «Y acabada una, se
arma otra\ldots{} la cuestión de la Comisaría General de Cruzada\ldots{}
Esa sí que será gorda\ldots{} Los Ministros, que siempre están en babia,
quieren meter en la Comisaría a ese Nicasio Gallego, que según dicen es
poeta\ldots{} Ya podéis limpiaros, que estáis de huevo\ldots{} Y parece
que los poetas ya le dan la enhorabuena al D. Nicasio\ldots{} como si lo
tuviera en la mano. ¡Pobres majagranzas!\ldots{} Con estas peripecias no
puede una pensar en sus cosas\ldots{} Mañana tarea de cera. La Semana
Santa, con la nueva feligresía, será muy lucida, muy lucida, y\ldots{}
¡dinero, dinero!\ldots{} Lindas botas con caña de tafilete verde te voy
a comprar\ldots{} Tomín\ldots{} ¡Ay! que no me ponga a soñar
ahora\ldots{} Rezo un poquito: «Dios te salve\ldots»

\hypertarget{xvi}{%
\chapter{XVI}\label{xvi}}

La nueva morada de Lucila y Tomín era un segundo piso, calle de San
Bernabé, lugar ventilado y alegre, con vistas al Manzanares y lejanos
horizontes que comprendían la Casa de Campo, pradera de San Isidro y
término de los Carabancheles. Para escoger aquella vivienda no se fijó
Lucila principalmente en su amena situación ni en los aires salutíferos
que la bañaban: aunque todo esto era muy de su agrado, no se determinó a
mudarse mientras el tratante en leñas, José Rodríguez, primer amparador
de Gracián, y \emph{el Ramos} de la calle de Rodas, no le dieran, con su
palabra honrada, garantía de la seguridad que allí tendría el perseguido
Capitán. Bajo tal fianza, accedieron ambos a compartir la casa modesta
de un acomodado matrimonio. Era él propietario de tierras en la Villa
del Prado, su patria, pero a la descansada vida de labrador prefería la
inquieta de tratante en uvas por Agosto y Septiembre, y en ganado los
demás meses del año. Antolín de Pablo salía cada quincena para
Villaviciosa, Sevilla la Nueva, Villa del Prado y Cadalso de los
Vidrios, y volvía con carneros y terneras para el matadero de Madrid. Su
mujer, Eulogia Ciudad, había sido criada de una Marquesa, que al morir
le dejó un legadito: era persona de agrado y habla fina. Privada de
sucesión, Eulogia se consolaba en la cría y cuidado de animales. Sus
gatos llamaban la atención por la brillantez del pelo así como por la
mansedumbre; sus perros sabían llevar y traer un cesto con recado. La
casa se comunicaba por la planta baja con un corralón donde Eulogia
tenía gallinas ponedoras, dos cabras, un cordero, un gamo, dos
galápagos, un erizo, una jabalina de corta edad, domesticada, dos
maricas también en vías de civilización, y un borriquillo. Representaban
el reino vegetal dos almendros, un saúco y un albaricoquero, que un año
sí y otro no cargaba enormemente de fruto.

Simpáticos fueron a Lucila y Tomín sus patronos, y para el Capitán fue
una expansión gratísima el permiso que se le dio para bajar al corral,
siempre que quisiese engañar allí las horas aburridas de su prisión.
Cuando a sus quehaceres salía Cigüela, el prisionero cogía un libro,
bajaba con ella, y la despedía en el portal diciéndole: «Yo me voy al
Paraíso Terrenal, y allí me encontrarás cuando vuelvas.» Comúnmente le
encontraba gozoso, distraído, con un perro a cada lado, que se habían
constituido en amigos y guardianes,

y allí se pasaba las horas muertas, sin leer nada, tratando de
entenderse en primitivo lenguaje con las maricas.

Por la noche, en la habitación que ocupaban, la cual era muy espaciosa y
alegre, Lucila le daba cuenta de lo que sabía referente al indulto, y él
no ocultaba su tristeza por la prolongación de un estado que no era de
cautiverio ni de libertad. Aquel auxilio que de persona para él
desconocida recibía le llenaba de inquietud. «Yo no quiero agradecer mi
libertad más que a ti, Cigüela---le decía,---y los recursos que no
vienen de ti me enfadan y me lastiman. Si yo escribiera a mis padres,
bien pronto me vendría de Medellín todo lo necesario para vivir. ¿Sabes
por qué no les escribo? Por que si escribiera, mi padre vendría sin
demora por mí, y su primera providencia sería llevarme consigo y
separarme de mi Lucila, de mi ángel tutelar\ldots{} Eso no será. Contigo
siempre\ldots{} O nos salvamos juntos, o juntos pereceremos\ldots{} Pero
también te digo que ya me cansa esta vida boba. El Paraíso Terrenal ya
da poco de sí, y ahora me entretengo en dar vida real a las Fábulas de
Esopo. Ya he conseguido que se entiendan el galápago y el burro, y que
las maricas dejen de soliviantar a las cabras para impedir a la jabalina
que vaya a pastar con ellas\ldots{} El gallo es de una pedantería
irresistible, y uno de los perros, el llamado \emph{Moro}, se entiende
con el carnero y el erizo para quitarle al gallo la gallina que más ama,
que es una pintadita, con aires de manola\ldots»

Opinaba Cigüela que una vez logrado el indulto, debía tratar Tomín de
volver a la gracia de su familia; no veía tan difícil que los de
Medellín transigiesen con la que había sido compañera y sostén del
Capitán en aquel terrible infortunio. Confiaba ella en conquistar a los
padres con su buena conducta, y terminaba diciendo: «Si tú me quieres,
como dices, y tienes mi compañía por tan necesaria en la felicidad como
en la desgracia, no necesitamos ir en busca de tus padres: ellos vendrán
a nosotros.»

Esto decía la moza, y a veces lo pensaba; mas ni su pensamiento ni sus
propias palabras optimistas la desviaban de su negra suspicacia. Una
tarde de fines de Marzo, o principios de Abril (que la fecha no está
bien determinada en las Historias), hallándose con Domiciana en San
Justo, hubo de apremiarla con energía para que obtuviese resolución
clara y pronta del dichoso indulto. Dio respuesta la protectora, como
siempre, reiterando las seguridades de gracia, y encareciendo la
prudencia mientras aquella no fuese un hecho. Abstuviérase, pues, el
Capitán de presentarse en público, lo que no era en verdad gran
sacrificio, toda vez que tenía buena casa, y disfrutaba del desahogo de
un corral poblado de animalitos. A esto replicó Lucila que no podía ya
sujetar a Tomín, cuyas ansias de libertad le movían a temerarias
imprudencias. Por una puerta que rara vez se abría, comunicaba el
corralón con los despeñaderos que desde aquellos lugares descienden
hasta la Ronda de Segovia. Contraviniendo las exhortaciones de Eulogia y
Lucila, el Capitán desatrancaba alguna tarde la puerta, y se daba el
verde de un paseíto por los andurriales de la Cuesta de la Mona o por
Gilimón. «Ayer mismo---dijo Lucila para terminar su referencia,---me dio
un horroroso susto. Cree que si Tomín fuese niño no me habría cansado de
pegarle. Pues llego a casa, entro en el corral, y me dice Eulogia que el
señor Capitán se había ido por la puerta de abajo\ldots{} Salí como un
cohete\ldots{} ¡Qué angustia! No puedes figurártelo\ldots{} Por fin,
¿dónde creerás que le encontré? En un secadero de ropa que hay por
aquella parte, no sé cómo se llama, orilla de la calle de la Ventosa. Me
dijo que se aburre, que siente una querencia loca de ver gente y de
hablar con todo el mundo\ldots{} Le cogí por un brazo y me le llevé a
casa. Yo lloraba\ldots{} Prometió no volver a escaparse; pero yo no me
fío\ldots{} Es el valor, Domiciana, el maldito arrojo, el desprecio del
peligro. Lo tiene en la masa de la sangre, y no puede con él.

---Pues para sujetarle y poner trabas a ese valor, que no viene a
cuento, hay un recurso, Lucila, y es meterle mucho miedo.

---¡Miedo\ldots{} a él!

---No se trata de ponerle un espantajo como a los gorriones, sino de
amenazarle con peligros muy verdaderos. Dile que en estos días anda la
policía muy atareada, cazando con bala o con liga, como puede,
pajarracos masónicos y militares sin seso. Sepa el buen Gracián que ya
han caído algunos, como él escapados de las Peñas de San Pedro. Ya están
en el Depósito de Leganés algunas docenas de estos desgraciados, y
cuando caigan los que quedan se formará una linda cuerda para Filipinas,
que deje tamañitas a las que mandó en su tiempo el muy \emph{crúo} de
Narváez\ldots{} A su casa no han de ir a buscarle; pero en la calle
¿quién responde\ldots?

Aterrada, no pudo Lucila ni aun pedir aclaración de tan graves noticias.

---Parece que lo dudas\ldots---añadió la otra.---Para que te
convenzas\ldots{} lo he sabido por el propio cosechero, D. Francisco
Chico\ldots{} ¿No me viste ayer en la tienda hablando con un señor de
lucida estatura, patillas de chuleta, viejo él, pero muy tieso, ojos
vivos, nariz chafada?\ldots{} Pues aquel es el jefe de nuestro ejército
policiaco y el más listo pachón que ha echado Dios al mundo. Mi padre y
él son amigos\ldots{} A mí me considera\ldots{} Rara vez llega por la
tienda. Ayer vino; subió a casa y vio aquel bargueño antiquísimo que
tenemos\ldots{} porque Chico es un águila para dos cosas: la cacería de
criminales y el compravende de cuadros y muebles de mérito.

Lucila suspiró. En rigor, alegrarse debía de aquellas amistades de los
cereros con el temido y famoso Chico, y ellas daban fuerza y lógica a
las seguridades de que Tomín no sería cogido en su casa. ¿Pero cómo
explicarse que Domiciana no le hubiera en anteriores ocasiones hablado
de aquel conocimiento? Las dudas y el recelo, como bandada de siniestras
aves, revolotearon en torno suyo, y una sombra nueva se añadió a las que
ya entenebrecían su alma.

Salió de la iglesia con intento de ir a su casa; pero acordándose al
paso por Puerta Cerrada de que no había visto a su hermano pequeño,
Rodriguín, en tantísimos días, tiró por la calle de Segovia en dirección
del taller de botería donde el muchacho aprendía el oficio. Mala hierba
había pisado aquel día la guapa moza, porque, no bien entró en el
taller, le salió al encuentro una nueva desdicha en la figura de su
señor padre, Jerónimo Ansúrez, el cual le saludó con el tremendo
jicarazo, \emph{verbigracia} noticia, de que le habían dejado cesante.

---Hija de mis entrañas---dijo el afligido y gallardo castellano,
desentendiéndose de los consuelos que los maestros boteros le
daban,---ya ves la mala partida de ese indecente Gobierno de los
\emph{honrados}, por mal nombre\ldots{} Aquí tienes a tu padre,
despedido de aquella gloria, donde estaba tan a gusto, que ya no habrá
para él lugar que no le parezca infierno; aquí le tienes otra vez en
mitad de la calle, con el día y la noche por hacienda y el vagabundear
por oficio. Díganme todos si no es esto una marranada, dispensando, y si
no nos sobra razón a los españoles para tronar, como tronamos, contra
este Gobierno, y el otro y todos, y contra la pastelera alianza del
\emph{Trono y el Altar}, contra tanta cancamurria de Libertad y
Constitución, y contra la birria asquerosa de Moralidad y Economía, que
es pura materia, perdonando\ldots{} ¿Qué hice yo para que me
despidieran? ¿a quién falté, con trescientos y el portero? ¿quién dio
queja de mí, si todas las cantatrices y bailadoras, así de plana mayor
como de filas, me querían como a las niñas de sus ojos?\ldots{} Pues
ello ha sido por colocar al marido de la pasiega que le está criando el
nene al sobrino de un Ministrejo, y busca buscando plaza, han visto la
mía, y ¡zas!\ldots{} Nación maldita, ¿por qué no te arrasaron los moros,
por qué no te taló el francés y te descuajó el inglés, y entre todos no
te raparon el suelo hasta que no quedara en él simiente de persona
viva?»

Esta y otras imprecaciones, desahogo de su furia, fueron oídas con
lástima por todos los presentes, con espanto por Lucila, que rondada se
sentía de negros presagios. La desdicha del pobre Ansúrez retumbaba en
el corazón de su hija como los pasos de un terrible viajero afanado por
llegar pronto. Era su infortunio, el dolor de ella, más intenso que el
de su padre, dolor inminente, cercano ya\ldots{}

\hypertarget{xvii}{%
\chapter{XVII}\label{xvii}}

Con pena de abandonar su casa y el cuidado de Tomín, consagró Lucila la
mañana siguiente a los deberes filiales. El buen Ansúrez necesitaba
consuelos, tiernas palabras que le infundieran ánimo y confianza, ideas
y razonamientos juiciosos para pescar otro empleo. Hija y padre
disertaron, esparciendo ansiosas miradas por todas las políticas aguas
que conocían. ¿A qué pescadores podrían arrimarse? Con el Sr.~Taja, que
había dado a Jerónimo su primer destino, en la portería del \emph{Fiel
Constraste y Almotacén}, no había que contar ya. El Sr. Zaragoza, que le
había empleado en el Teatro Real, no era ya jefe político ni estaba a la
sazón en Madrid, y para llegar al nuevo Gobernador, D. Melchor Ordóñez,
no veían ningún camino. ¿A quién volverse, a quien marear y aburrir
hasta obtener la credencial, concedida por la fuerza del tedio más que
por la piedad? Indicadas y discutidas diferentes personas, el astuto
Ansúrez, sabedor de las amistades de Lucila con Domiciana y de las
excelentes agarraderas de esta en Palacio, o sabe Dios dónde, la diputó
por la mejor santa en quien debía poner toda su fe. Conforme Lucila con
esta opinión, quedaron en que al siguiente día se verían hija y padre
con la cerera para empezar la ruda campaña. En estas y otras
conversaciones se le fue a Lucila toda la mañana y parte de la tarde,
porque cuando impaciente quería despedirse, su padre la cogía de los
brazos y la retenía, gozoso de verla y escucharla. Rodriguín también
tiraba de ella, y los maestros boteros no se cansaban de admirar su
hermosura. En la botería se aposentaba Ansúrez, y allí aguardaba la
visita diaria de su hija. Prometió esta no faltar ningún día, y
abrazando a su padre le dejó entre sus amigos, rodeado de aquellos
imponentes pellejos hinchados de viento, que tanta semejanza tenían con
los hombres públicos de aquel tiempo\ldots{} y de otros.

Desalada tomó Lucila el camino de su casa. Por evitar un largo rodeo y
ganar tiempo, puso a prueba sus pulmones apechugando con la Cuesta de
los Ciegos, que subió de un tirón hasta Yeseros y la Redondilla, y de
allí en cuatro brincos se plantó en la calle de San Bernabé. Llegó a su
casa pensando que Tomín estaría inquietísimo, poniendo en fábulas
tristes a todos sus animales\ldots{} Como exhalación pasó de la puerta
al corral, donde le salió al encuentro Eulogia con cara de susto, que a
Lucila le pareció una máscara, pues nunca había visto tan alteradas las
facciones de su casera. Antes de que se le preguntara por el Capitán
soltó la buena mujer esta bomba: «No está\ldots{} no ha vuelto desde las
diez de la mañana.» El primer impulso de Lucila fue rebelarse animosa
contra el Destino; y sacando de su alma las primeras fuerzas con que a
la lucha se disponía, respondió: «Ya vendrá\ldots{} le
encontraremos\ldots{} ¡Qué loco es, Dios mío! No vale que una le
diga\ldots{} no vale que se le recomiende\ldots{} Andará por ahí hecho
un tonto, viendo tender ropa\ldots» Reiterando la noticia en forma
desconsoladora, Eulogia dijo que ya habían pasado más de seis horas
desde que se perdió de vista; que sobre las doce, alarmada de la
tardanza, había mandado a Colás (un chico de la vecina) en su busca, y
que Colás volvió a la una diciendo que, recorridos todos los lavaderos,
todos los secaderos, las vueltas, recodos y precipicios de la Mona y
Descargas, registrada después la Ronda de Segovia de punta a punta, sin
omitir taberna, figón, juego de bolos ni herradero, no había encontrado
rastro del señor Capitán. Oído esto por Lucila, quedose la buena mujer
paralizada del pensamiento y la voluntad, sin que su mente pudiera hacer
otra cosa que medir la longitud de los espacios recorridos inútilmente
por Colás. Pronto se rehízo, y apartando con una mano a uno de los
perros, con otra a la jabalina, que le estorbaban el paso, más con la
actitud que con la palabra dijo que ella le buscaría\ldots{} Todo era
posible menos la desaparición, la pérdida del Capitán, como podría
perderse una de las maricas, o el gamo de pies ligeros.

Salió, pues, en loca marcha, corriendo de un lado a otro, y esparciendo
su mirada por aquellos polvorientos espacios\ldots{} Si en un instante
creía ver a Tomín, el instante siguiente traía el frío desengaño.
Decidiose a preguntar a diferentes personas que encontraba. Algunas
mujeres, sentadas al sol en la cuesta de la Mona, dijeron que le habían
visto subir, a mano derecha\ldots{} otras que bajar, a mano izquierda.
En la Ronda de Segovia, repitió Lucila su angustiosa pregunta precedida
de señas inequívocas: «un caballero joven, de buena presencia, con
zamarreta de paño azul obscuro, botas de caña verde, gorra sin
visera\ldots» Una mujer que llevaba cesta de ropa declaró por fin haber
visto al caballero: viéronle pasar ella y su marido; este, que le
conocía de anteriores encuentros, habíale saludado\ldots{} Dos horas
después, al caer de mediodía, su Fabián, que era medidor en un almacén
de granos, le había visto con dos sujetos, uno de los cuales le pareció
\emph{guindiya}\ldots{} No pudo esclarecer su informe la buena mujer,
que sólo repetía cláusulas sueltas de su marido, y apreciaciones en que
ella no se fijó porque maldito lo que le interesaban. Cuando su Fabián
volviese de Carabanchel Alto, adonde había ido por cebada, podría dar
mayores explicaciones y noticias\ldots{}

Rendida y sin aliento volvió a la casa Cigüela, y de tal modo a su
espíritu se adhería la esperanza, que al subir pensaba encontrar a
Tolomé. «Habrá dado la vuelta grande---se dijo,---subiendo la Cuesta de
los Ciegos y entrando por la calle del Rosario, o de San Bernabé.» Nuevo
desengaño al ver la cara triste de Eulogia: hasta los perros decían con
su grave quietud que el Capitán no había dado vuelta grande ni
chica\ldots{} Ya no pensó Lucila más que en correr en busca de la cerera
para comunicarle su mortal ansiedad. Sin darse cuenta de la distancia ni
del tiempo empleado en recorrerla, fue a la cerería, donde se le dijo
que Domiciana no había regresado aún, ni regresaría hasta después de
prima noche. No quiso esperarla: angustiada voló otra vez hacia Gilimón,
desoyendo la voz de Ezequiel, que con lastimero acento pueril se brindó
a ser su acompañante. En el corral, mientras la casera recogía diligente
a los animales menores, a otros daba el pienso y a todos prodigaba su
maternal solicitud, viose Lucila lanzada a senos profundísimos de
tristeza, la cual acreció al extender la noche su lenta obscuridad.
Pasado algún tiempo, Eulogia y ella subieron. Cuando entró la moza en el
cuarto que habitaba, toda su entereza cayó de golpe al ver la ropa de
Tomín, su cama, la mesa en que tenía libros, tabaco, un latiguillo, una
caja de mixtos, papel y obleas, una herradura que había recogido en la
Ronda, como signo de buena suerte, pues no le faltaban sus puntos de
supersticioso\ldots{} Ante estos objetos, se desató el dolor de Lucila,
sin que la buena Eulogia con ninguna expresión de consuelo pudiese
calmarla, y cogiendo la ropa entre sus brazos como habría cogido el
cuerpo mismo del perdido Tolomé, echose de bruces sobre la cama, y en
las dulces prendas vertió todo el torrente de sus lágrimas con
silencioso duelo.

Inútiles fueron las instancias de las vecinas para que Cigüela cenara:
no cenaría mientras Tolomín no volviese. Eulogia le daba esperanzas que
no tenía, y ella las tomaba sin hallar en su pensamiento lugar donde
meterlas\ldots{} Las diez serían cuando llegaron casi juntos Ezequiel y
el medidor de granos Fabián, cuya mujer había dado a Lucila informes
vagos del caballero desaparecido. Era un hombre de madura edad, grave,
bondadoso, y su traza y modos inspiraban confianza. Eulogia le conocía,
y Antolín de Pablo le apreciaba. Tan importante fue su declaración desde
las primeras palabras, que en ella puso Eulogia todo su oído y Lucila
toda su alma. Había visto tres veces al Capitán, la primera solo, en la
bajada de la Mona, la segunda al pie del jardín del Infantado con dos
hombres, que no eran amigos, porque le hablaban con malos modos\ldots{}
Después le vio con los mismos, o más bien llevado por ellos, en la
vereda que hay entre la huerta de Barrafón y la de las Monjas del
Sacramento. «Para mí que le llevaban por atajos, o como se dice, por
sitios de poca gente, hacia las Cambroneras, para de allí pasar el
puente de Toledo y conducirle al Depósito de Leganés\ldots» La angustia
no permitió a Lucila formular pregunta relacionada con el temido nombre
de Leganés. «¿Y crees tú, Fabián---murmuró Eulogia con escalofrío,---que
el Capitán está\ldots{} allá?

---Como si lo estuviera viendo---replicó el informante.---¿A dónde sino
allí podían llevarle aquellos Caifases? No pierdan el tiempo buscándole
por acá, y acudan pronto\ldots{} que pasado mañana sale cuerda. En
Carabanchel me lo han dicho los guardias que harán la \emph{conduta.»}

Silencio de muerte siguió a estas palabras.

---Pasado mañana sale cuerda---repitió Fabián con el acento que suele
darse a las recomendaciones leales de previsión. Dudas crueles movieron
el alma de Lucila, alterando en ella las fases del pesimismo. «¿Y si no
está en Leganés?\ldots{} ¿Si le han llevado a otro punto?\ldots» En esto
le tocó a Ezequiel expresar su mensaje, el cual era que hallándose Doña
Victorina Sarmiento en peligro de muerte, Domiciana no podía separarse
de su lado en toda la noche. A las ocho y media se recibió en la cerería
el recado de Palacio diciendo que no la esperaran\ldots{} Diferentes
pensamientos, que no habría podido manifestar aunque quisiera, armaron
gran alboroto en el cerebro de Lucila, que con las manos en la cabeza
expresaba su enloquecedora confusión. Eulogia y Ezequiel la instaron
para que comiese alguna cosa, no dejándose vencer de la debilidad en tan
angustiosas circunstancias, y al fin la desolada moza probó algo de un
guisote que la casera le trajo, y casi a la fuerza pasó para dentro
medio vaso de vino. Despidiose Fabián llamado por sus quehaceres.
Silenciosa y espantada hallábase Lucila como el que discute consigo
mismo dos diferentes especies de muerte, entre las cuales forzosamente y
sin dilación tiene que elegir una\ldots{} Su dolorosa perplejidad vino a
parar, al fin, a una determinación súbita y rectilínea. Se levantó, fue
a coger su pañuelo de manta que pendía de una percha, y echándoselo por
los hombros, dijo: «Me voy a Leganés\ldots{} Algún medio habrá de saber
la verdad\ldots{} Acompáñame tú, Ezequiel. Si necesitas licencia de tu
padre, vete por ella y vuelve pronto.»

Respondió el bondadoso chico que la licencia la tenía ya, pues su padre
le había encomendado, al salir de casa, que si Luciíta se veía precisada
\emph{a dar pasos} a cualquier hora de la noche, o toda la noche entera,
la asistiese y custodiase como lo haría el propio D. Gabino, si en tan
honrosa obligación se viera. No le pareció bien a Eulogia que en noche
obscura y con tan menguada compañía emprendiese una mujer caminata larga
y peligrosa; pero no pudo desviar a Lucila de aquel propósito, semejante
a la veloz derechura de la flecha lanzada. Salieron por el corral.

\hypertarget{xviii}{%
\chapter{XVIII}\label{xviii}}

Ya embocaban a la cabecera del puente de Toledo cuando un desgarrón de
las nubes, que cubrían casi totalmente el cielo, dejó ver un cuarto de
luna, con desmayada luz entre cendales, corriendo hacia los bordes
grises que habrían de ocultarla de nuevo\ldots{} «Lucila, mira, mira la
luna---dijo Ezequiel creyendo que podría distraer de su pena a la pobre
joven, comunicándole su admiración candorosa. Pero ni lunas ni soles
podían iluminar la noche obscura que en su alma llevaba la hija de
Ansúrez, y siguió en silencio. Marcha sostenida y regular llevaban: con
el aire que al paso de los dos imprimió Cigüela en la bajada de Gilimón,
se aproximaron a la entrada de Carabanchel Bajo. Pero aquí el potente
impulso de ella empezó a flaquear; se detuvo un momento mirando las
primeras casas, y preguntó a su acompañante si estaban ya en Leganés.

---¡Ay! no\ldots{} Esto es Carabanchel Bajo\ldots{} Si quieres,
descansaremos un poquito.

---No\ldots{} Entre casas y donde haya gente, no nos detengamos---dijo
Lucila.---Sigamos, y a la salida nos sentaremos.»

Atravesaron el pueblo, esquivando el encuentro con los escasos grupos de
personas que al paso veían, y al salir de nuevo al campo, Lucila hubo de
aquietar un poco su marcha. «Nos cansamos sin necesidad---observó
Ezequiel,---pues ¿qué adelantas con llegar a Leganés a media noche?
Andemos despacio, y si a mi brazo quieres agarrarte hazlo con confianza,
que yo no me canso. Por este camino venimos Tomás y yo de paseo algún
domingo, y todo este campo me lo sé de memoria.» Con lento andar
llegaron a Carabanchel Alto; acelerando un poco pasaron el pueblo, y al
rebasar de las últimas casas, Lucila, sin aliento, echando en un suspiro
toda esta frase: «no puedo más, \emph{Zequiel}\ldots{} aquí me siento,»
cayó al pie de un árbol. El cerero acudió a levantarla, cariñoso,
diciéndole que un poco más encontrarían mejor y más cómodo asiento, y
puesta ella en pie, bien asida la mano del mancebo, siguieron despacio,
él sosteniéndola, ella dejándose llevar, hasta que les brindaron
descanso unos troncos de negrillo apilados en el suelo y protegidos de
una maciza pared en ruinas.

---Estoy muerta de cansancio---dijo la moza después de recobrado el
aliento.

---Pues tómate el tiempo que quieras para recobrar fuerzas, porque aún
hay algunas horitas de aquí al amanecer\ldots{} Y si te entra sueño y
quieres dormir, no tengas miedo a nada; yo velo y estoy al cuidado.

---Mira, \emph{Zequiel}, mira aquella lucecita que allá lejos se
ve\ldots{} por esta parte\ldots{} por donde te señala mi dedo\ldots{}
¿Será aquello Leganés?

---Por esa parte cae el pueblo; pero el cuartel está más . Entre el
cuartel y el pueblo hay unas casas muy grandes del Duque de Medinaceli
donde van a poner Hospital de locos.

---Casa de locos\ldots---dijo Lucila.---Pues que sea grandecita, pues
bien de gente hay que la ocupe\ldots»

Dicho esto, permanecieron silenciosos, Ezequiel a la izquierda de su
amiga, mirando a las lejanías obscuras donde se divisaban, no ya una
sola luz, sino tres o cuatro formando como una constelación. Requirió
Lucila los bordes de su pañuelo de manta para abrigarse, y como
expresara su desconsuelo de ver al muchacho sin capa ni ningún abrigo,
dijo él: «Yo nunca tengo frío ni calor. No te ocupes de mí y abrígate
bien, que tú eres más delicada.» Así lo hizo Lucila, y a la media hora
de estar allí, el abrigo, el descanso, la soledad, rindieron su fatigada
naturaleza, llevándola sin sentirlo a una sedación intensísima\ldots{}
Su pena se recogió en el fondo del alma, ahuyentada momentáneamente por
la reparación física; la inercia impuso un paréntesis de la vida para
seguir viviendo\ldots{} Dio dos o tres cabezadas. «Lucila---le dijo el
cerero, inmóvil,---si quieres descansar tu cabeza sobre mi hombro, aquí
lo tienes\ldots{} A mí no me incomodas\ldots{} descarga tu cabeza y
duerme un poquitín\ldots» La moza no respondió\ldots{} Por instintivo
abandono, vencida de un sopor más fuerte que su propósito de estar
desvelada, dejó caer la cabeza sobre el hombro del mancebo y quedose
dormida. Desde que sintió el dulce peso, Ezequiel fue un poste, más bien
almohadón en figura de persona: respiraba con pausa y ritmo, para que ni
el menor movimiento turbase el reposo breve de su infeliz amiga. La
inocencia del muchacho despierto no era menos bella que la de la mujer
dormida.

El sueño de Lucila, que en realidad fue como una embriaguez de
cansancio, duró apenas un cuarto de hora. Despertó sobresaltada,
creyéndolo de larga duración. «¡Si apenas has dormido el espacio de tres
credos!---le dijo Ezequiel.---Duerme más y descansa, que yo velo: yo
velo por los dos\ldots{} y estoy al cuidado\ldots{} Como si quieres
echarte bien envueltita en tu pañuelo, y apoyando la cabeza en mis
rodillas\ldots{}

---No, no, \emph{Zequiel}\ldots{} Yo no tengo sueño. Fue un momento no
más, como si de la fuerza de mis pesares perdiera el sentido. Se moriría
una si alguna vez, por un ratito, no se borrara de nuestro pensamiento
el mal que sufrimos, y no se escondiera el dolor\ldots{} \emph{Zequiel},
duerme tú ahora si quieres, que yo velaré.

---No: rezo y velo yo, que debo estar al cuidado.»

Hablando a ratitos, o entregándose cada uno por su cuenta a la
contemplación del cielo y de la noche, escapados hacia el infinito
exterior para recaer luego en el interno infinito que cada cual en sí
mismo llevaba, pasaron horas no contadas ni medidas, porque ni ellos
tenían reloj, ni campanadas lejanas venían a marcarles los pasos del
tiempo. Tampoco sabían leer la hora en los astros, y estos\ldots{}
malditas ganas tenían aquella noche de ser leídos.

Engañada por su deseo de acelerar el tiempo, creyó ver Lucila un viso de
aurora en el horizonte, y dispuso continuar la marcha. «Ya viene el día,
\emph{Zequiel}\ldots{} Sigamos. No nos será difícil averiguar si está
Tomín en el Depósito. Y si está, tenemos que volver corriendo a Madrid
para dar los pasos y ver de sacarle\ldots{}

---Con alma y vida mirará Domiciana por él---dijo el cerero gozoso,
ingenuo.---¡Pues no le quiere poco en gracia de Dios!\ldots{} Y eso que
nunca le ha tratado\ldots{} Verdad que le conoce como si le hubiera
visto mil veces, y sabe cómo tiene los ojos, y lo arrogante que
es\ldots{} Tanto le has hablado tú de Tomín, que sin verle le ha visto.
Domiciana es muy buena: a ti te quiere muchísimo, y todo su empeño es
proporcionarte un buen matrimonio. Al Capitán le quiere porque le
quieres tú. Yo le dije un día que fuese conmigo a ver a Tomín, y ella me
dijo, dice: «no voy, porque Lucila es muy celosa y podría metérsele en
la cabeza cualquier disparate.» Yo le contesté que tú no pensabas nada
malo de ella, pues harto sabes que es monja, y que no tiene licencia del
Padre Eterno para enamorarse de un hombre\ldots»

Lucila, que aún permanecía sentada, pensó que llevaba de compañero a un
ángel del Cielo.

---Si quieres---dijo el muchacho,---sigamos nuestro camino. Despacito,
podremos llegar, creo yo, cuando esté amaneciendo\ldots{} Pues Domiciana
me dijo eso: «No quiero que Lucila padezca celos por mí\ldots{} Podría
suceder que el Capitán, al verme, fuera conmigo rendido y galante, como
corresponde a un caballero. No dejaría de apreciar mi señorío y buena
educación, no dejaría de ver que si no soy hermosa, tampoco espanto por
fea\ldots{} Los hombres de gusto aprecian mucho, en nosotras, los
modales y el hablar finos\ldots{} Por esto quiero estar apartada de
Bartolomé\ldots{} para que esa pobrecilla Luci no se arrebate.» Esto me
dijo, y en ello verás lo mucho que te estima.

---Sí que lo veo, y lo agradezco de veras---indicó Lucila poniéndose en
marcha.---Tu hermana, desde que anda en tratos con gente de Palacio, se
compone y acicala. Con su buen ver, y con la gracia de su conversación,
haría conquistas si quisiera.

---Pero no le hables a ella de conquistas de hombres---dijo Ezequiel
ajustando su paso al de Lucila,---que eso no le cuadra, ni mi hermana es
mujer que falte a sus votos por nada de este mundo. En ella no verás el
coquetismo de otras que se emperifollan al cuento de gustar a los
caballeros. Lo que hace mi hermana es adecentarse, porque tiene que
andar entre personas de la aristocracia fina\ldots{} Ella para sí tiene
el gusto del aseo, que ya es como una tema; tanto, que algunos días no
se pueden contar las cubas que el aguador sube a casa para sus
lavoteos\ldots»

Algo más habló el ángel en el caminar lento por la carretera polvorosa,
y momentos hubo en que molestó grandemente a Lucila el batir de las
blancas alas de su compañero: en un tris estuvo que de un manotazo le
arrancase las plumas\ldots{} Callaba la moza para que él moderase sus
expansivas manifestaciones, y andando, andando, vieron casas, mulos,
personas. Como Ezequiel anunció, llegaban al término de su viaje a punto
de amanecer. Guió el mancebo hacia un edificio grande y aislado que a
derecha mano se parecía, y cerca de él vieron grupos de mujeres que
volvían hacia el pueblo. Hallándose a corta distancia del grande
edificio, con trazas de convento, oyeron toque de cornetas y tambores. A
Lucila le saltó el corazón. Hablaba el Ejército, que para ella era como
si Tomín hablase; y estando en esto, parados los dos en espera de algo
que determinara sus resoluciones, creyó Cigüela oír su nombre. Volviose,
y entre los bultos de personas que pasaban vio que se destacaba una
mujer, toda envuelta en cosa negra como una fantasma. Por segunda vez
sonó la voz, agregando otras palabras al nombre: «Lucila, Lucila, ¿no me
conoce? Soy Rosenda.»

Ya\ldots{} Era \emph{la Capitana}, amiga del Teniente Castillejo,
compinche de Bartolomé Gracián en políticas trapisondas. Al reconocerla
y contestar al saludo, advirtió Lucila que tenía el rostro bañado en
lágrimas, y que revelaba en sus facciones y en su fúnebre actitud una
gran tribulación.

---Vengo, ya usted supondrá---murmuró Lucila, que al punto se contagió
del lagrimeo,---vengo porque\ldots{} Pasado mañana\ldots{} digo, mañana,
sale la cuerda.

---Hija, no---replicó la Capitana ahogándose:---la cuerda salió ya.

---¿Cuándo?

---Hoy\ldots{} hará un cuarto de hora. ¡Mala centella para el
Gobierno!---exclamó Rosenda, que era en su lenguaje un poquito
amanolada.---En los hombres no hay ya vergüenza\ldots{} Las mujeres
tendremos que hacer alguna muy sonada\ldots{} pasear por las calles en
un palo mondongos de Ministros\ldots{} ¿De veras no cree usted que haya
salido la cuerda? Por allí va\ldots{} ¿Ve usted aquella nube de polvo,
como las que se levantan cuando pasa un ganado? Pues allí van\ldots»

Miró Lucila hacia el punto lejano que Rosenda le señalaba, y vio en
efecto, la columna de polvo, como una cabellera desgreñada en sus
extremos. Iluminada por el resplandor de la aurora, que a cada instante
era más vivo, la nube blanquecina andaba lentamente. No se veían los
hombres conducidos al destierro: se veía sólo una cresta de polvo que en
su camino les acompañaba. Lanzó Cigüela un rugido, y antes de que en
otra forma expresara su inmenso dolor, Rosenda le dijo: «¿Por qué ha
venido usted, si Bartolomé no va en la cuerda?

---¡Que no va! ¿Está usted bien segura?\ldots{}

---Les he visto a todos uno por uno, anoche y esta madrugada, en el
mismísimo Depósito\ldots{} Infierno lo llamo. Las cosas que he tenido
que hacer para que el Comandante me dejara entrar no puedo decirlas
ahora\ldots{} Pues verá usted: militares van seis\ldots{} Mi pobre
Castillejo, Zamorano, Socías\ldots{} ¿se acuerda usted de Socías?
Angulo, el de Provinciales de Cuenca, y dos que trajeron ayer de
Guadalajara. Los demás son gente de pluma: van en la cuerda porque
llamaron ladrones a los Ministros, o porque repartieron papelitos en los
cuarteles. Van también dos extranjeros que parecen \emph{gringos}, y un
\emph{franchute}. ¡Ay, qué infame tropelía! ¡Llevar a hombres cristianos
en traílla, como a perros con rabia para echarlos al agua! ¡Lástima que
todas las mujeres de corazón no nos volviéramos perras rabiosas!\ldots{}
¡No eran mordidas, Señor, no eran mordidas las que habíamos de
pegar!\ldots{} ¡Ay, mi Castillejo! ¡Pobrecito de mi alma!» Decía esto
mirando la cabellera de polvo, que alejándose se achicaba ya, y removida
del vientecillo de la mañana desparramaba en el aire sus guedejas.

\hypertarget{xix}{%
\chapter{XIX}\label{xix}}

---Con lo que dice esta señora---indicó Ezequiel a su amiga,
satisfecho,---ya puedes estar tranquila. Demos gracias a Dios. Tomín no
va en la cuerda.

Sintiendo su alma casi libre del horrendo peso que había traído consigo
desde Madrid, Cigüela no podía llegar a un estado de completa
tranquilidad y menos de alegría. Porque aun descartado el hecho
tristísimo de la deportación de Gracián, el problema seguía ofreciendo a
la pobre mujer aspectos pavorosos. ¿Dónde estaba el hombre? El cúmulo de
probabilidades, todas muy negras, que esta interrogación ponía frente a
Lucila, incitándola a escoger la más lógica, era motivo suficiente para
que la paz no reinara en su alma. De que Tolomín no había ido en la
cuerda se convenció escuchando de nuevo el informe de la Capitana,
autorizado por un Teniente de servicio en el Depósito, hombre compasivo
y amable que las acompañó cuando se retiraban al pueblo\ldots{} Vio,
pues, Lucila claramente que su afán continuaba en Madrid, y allí habría
de padecerlo hasta que Dios la curara o la matara.

Cuando se desvaneció en el horizonte la nube de polvo, señal de que los
presos iban ya cerca de Getafe, las dos mujeres, desconsoladas por la
desaparición de sus hombres, echaron suspiros, la una en dirección de la
cuerda, la otra hacia los mismos Madriles, y al punto se percataron de
que nada tenían que hacer en aquel sitio. «Vámonos al pueblo---dijo la
Capitana, bostezando de sueño y hambre;---yo estoy con lo poco que comí
ayer al mediodía.» Demostraciones de desfallecimiento hizo también
Lucila, secundada por Ezequiel; y el Teniente, que en aquel caso estaba
obligado a ser galante, las invitó a matar el gusanillo en una venta
próxima. Aceptaron las mujeres, y poco después sus pobres cuerpos se
reparaban del grande ajetreo de la noche, ya que del vivo dolor no
podían sus almas repararse. Durante el desayuno, que el Teniente proveyó
con liberalidad, se desató la Capitana en denuestos contra \emph{el
ladronazo de Bravo Murillo}, que quería ser más \emph{crúo} que
Narváez\ldots{} Esto no podía permitirse a un \emph{facha}, a un
\emph{Don Levosa}, personaje \emph{de poco acá}; y los de Tropa debían
volverse todos contra él, negando el derecho del paisanaje a mandar a
los españoles. Cigüela, interrogada después por su amiga, tuvo que
relatar el cómo y cuándo de la extraña desaparición de Gracián. El
Teniente le conocía desde la campaña de Cataluña, en que sirvieron
juntos, y a un tiempo encomiaba su bravura en la guerra y su temeridad
en las intentonas políticas.

Repuestas de su quebranto físico, las mujeres hablaron de volverse a
Madrid. Rosenda propuso que, si no se encontraba calesa, se buscara un
carro en que podrían ir tumbadas, como sacos de patatas o seretas de
carbón. Mientras iba Ezequiel a esta diligencia, la curiosa Capitana
pidió a Lucila noticias de aquel joven tan modosito y guapín que la
acompañaba, y satisfecha su curiosidad,

dijo: «¿Con que cerero? Ya pensé yo para entre mí que ese
\emph{descosío} tenía que ser de iglesia. Bien pensado está eso de
arrimarse a lo eclesiástico, que en estos tiempos no hay otro
camino\ldots{} ¡Ay, bien se lo dije a mi Castillejo! Él no me hacía
caso\ldots{} Ocasiones tuvo de ampararse de la clerecía. Yo le abrí
camino, por un señor cura, mi amigo, que está en el Vicariato General
Castrense; pero Castillejo no quería\ldots{} Por poco reñimos\ldots{} Y
ya ve las resultas de ser tan arrimado a la libertad de religión, de los
cultos \emph{ateístas}, o como se llame\ldots{} ¡A Filipinas! ¿Y hasta
cuándo, Señor?\ldots{} ¿Sabe usted lo que digo? Que maldita sea esta
Nación.»

Encontrado el carro, y despedidas del Oficial las tristes mujeres,
emprendieron su regreso a Madrid. «Acuéstense en estas sacas---les dijo
Ezequiel,---y duerman tranquilas; que yo velo y estaré al cuidado.»
Tumbáronse a su comodidad; pero sólo en esto se cumplieron las
indicaciones del mancebo, pues él fue quien, rendido de la mala noche,
se durmió como un cesto, y ellas, velando, hablaban de sus cosas.
Referidos por Cigüela ciertos antecedentes de la desaparición de Tomín,
dijo con agudeza la Capitana: «Este es un caso, amiga mía, en que yo
tengo que preguntar: ¿\emph{quién es ella}? Me da en la nariz olor de
mujerío\ldots{} Gracián es un real mozo\ldots{} Sé por Castillejo que a
muchas enloqueció sólo con mirarlas. En Madrid, hija, pasan cosas que si
se cuentan nadie las cree\ldots{} Va usted a oír un sucedido que pasó en
Lorca, mi tierra. Érase un oficial muy simpático que estaba preso por
mor de un desafío. Entre dos mujeres, que al parecer no le conocían, la
una muy rica, le sacaron de la cárcel, sobornando a los guardias, y se
le llevaron a un campo lejos, lejos\ldots{} La rica, que era viuda y
fea, apareció al año en Murcia con un niñito de pecho; poco después
llegó el oficial con el canuto de la absoluta, y se casaron\ldots{} Ate
usted este cabito y aprenda\ldots{} No dude usted que si hay robos de
mujeres por hombres, y testigo soy yo, pues mi marido siendo alférez me
robó a mí lindamente de la casa de mis padres, como quien coge del árbol
una pera o melocotón; si hay, digo, casos mil de muchachas robadas por
varones, casos se han visto, aunque son menos, de caballeros arrebatados
por señoras\ldots{} Indague usted, Lucila, y haga por descubrir la
verdad\ldots{} ¡Ay, si eso a mí me pasara, y supiera yo dónde está la
ladrona!\ldots{} ¡No eran bofetadas, no eran azotes en semejante parte,
no eran estrujones hasta quedarme con el moño en la mano, y no era
zapateado sobre sus costillas hasta dejarla como una pasa!»

---¡Robado por una mujer!\ldots{} ¡Imposible!---exclamó Lucila, que
aunque bregaba en su magín con un pensamiento semejante, no lo tuvo por
absurdo hasta que lo oyó expresado por extraña boca. Le sonaron las
historias y comentarios de Rosenda a cosa trágica, compuesta para causar
lástima y terror a las gentes, como lances de teatro inventados por los
poetas\ldots{} Y le pareció aún más extraño que tales cosas le pasaran a
ella, criatura insignificante y pacífica, pues las tragedias eran
siempre entre reyes o personas de elevada alcurnia\ldots{} Recordó
entonces lo que su padre le refería de los dramas cantados, y de las
bellezas grandilocuentes de la ópera\ldots{} Su inmensa desdicha, con
las nuevas formas que tomaba, se le iba volviendo cosa de canto, o por
lo menos de verso, que viene a ser la música parlada.

Nada más, digno de ser contado, ocurrió en el viaje, que tuvo su fin
después de mediodía. Dejolas el vehículo junto a la Puerta de Toledo, y
a pie hicieron su entrada en la Corte, despidiéndose la Capitana en la
esquina de la calle de la Ventosa, para seguir hasta la Cava de San
Miguel, donde moraba una tía suya\ldots{} Al entrar la buena moza en su
casa, grande ansiedad, negra con tornasoles de esperanza, embargaba su
espíritu. ¡Estaría bueno que hubiera parecido Tomín, que le encontrara
sano y salvo, creyendo que ella era la extraviada y no él!\ldots{} Pero
esta ilusión tardía, triste como flor de cementerio, se desvaneció al
entrar en el corral y ver la cara de Eulogia, que no dijo nada
lisonjero. Rápidas preguntas cambiaron una y otra. «¿Ha ocurrido algo;
ha venido alguien?» «Nadie ha venido: no sé nada. ¿Dices que no ha ido
en la cuerda?» «No va en la cuerda. ¿Ha venido alguien?\ldots» «Nadie,
mujer\ldots»

Toda la tarde estuvo Cigüela muy abatida y lacrimosa\ldots{} Por la
noche se salió de la casa sin que Eulogia la viese, y dejándose llevar
de una atracción irresistible bajó a la Ronda: su memoria, eficaz
auxilio de su locura, le reprodujo la relación que la noche anterior
hizo Fabián de los lugares donde había visto a Tolomín, conducido por
dos hombres, y se lanzó por solares y callejuelas entre tapias,
recorriendo o pensando recorrer los mismos sitios por donde aquel fue,
perdiéndose al fin de toda vista humana. Era una conmemoración, un
viacrucis por estaciones que ignoraba si conducían a la casa de Pilatos,
al Gólgota, o a otro nefando lugar, peor que todos los Calvarios\ldots{}
Llegó a verse entre tapias, que eran guardianas de árboles raquíticos y
de unos caseretones destartalados, siniestros: en alguno de estos vio
luces\ldots{} Pasó junto a un lavadero; vio un altozano que más bien
parecía montón de escorias, las cuales bajo los pies sonaban como
huesos, y al subirse a él distinguió más caseretones de formas absurdas,
más árboles escuetos, y vapores lejanos, como humos de caleras o
resuello de hornos. En lo más alto de aquel montículo, sintió imperioso
anhelo de llamar al perdido Capitán, con la crédula ilusión de que este
le respondería, y soltando toda la voz, se puso a gritar
¡Tolomín!\ldots{} Entre grito y grito dejaba un espacio\ldots{} Aguzaba
el oído, creyendo que de la inmensidad distante vendría un
\emph{¿qué?}\ldots{} Pero no venía nada\ldots{} Los pulmones fatigados y
la garganta enronquecida, ya no podían más. Bajó Lucila del montículo, y
arrimada a una tapia, la voz, no ya vigorosa y tonante, sino plañidera,
con angustioso timbre, dijo: \emph{«¡Min!\ldots»} Recorrió como unas
treinta varas clamando \emph{Min}, en son parecido al balar del
cabritillo\ldots{} cada vez más tenue hasta que se extinguió en un
\emph{Min} casi imperceptible, como si a sí misma se lo dijera\ldots{}
Cuando volvió a su casa, cerca de media noche, Eulogia creyó que su
pobre huéspeda se había dejado en el paseo la razón.

Si no volvía loca, enferma sí que estaba: en la cama hubieron de meterla
contra su voluntad, acudiendo a calmar con mantas y botellas de agua
caliente el intenso frío precursor de horrible calentura. Por no ser
fácil encontrar médico en la vecindad a tal hora, llamose a un
veterinario, habitante en la misma casa, el cual, viendo muy arrebatado
el rostro de Lucila y que de su cabeza echaba fuego, ordenó una sangría.
No creyó prudente Eulogia administrársela. A la mañana siguiente fue un
físico de tropa, muy entendido, y aprobado lo que había hecho la casera,
diagnosticó el caso como grave, de lenta resolución\ldots{} En efecto:
bien malita y casi a dos dedos de la muerte estuvo Cigüela, delirando
furiosamente por las noches, y de día como alelada, diciendo mil
desatinos y sin conocer a nadie: en los ratos de alivio, su
entendimiento no daba de sí más que estas preguntas: «¿Quién ha
venido?\ldots{} ¿Qué se sabe?\ldots{} ¿Domiciana\ldots?» Eulogia le
contestaba: «Sí, sí: ha venido la señora cerera\ldots{} La primera vez
no quiso pasar: no venía más que a enterarse. La segunda vez le dije que
estabas sin conocimiento\ldots{} llegó a esa puerta y miró\ldots{} No
quiso entrar\ldots{} parecía medrosa, muy medrosa\ldots{} Te miraba
desde la puerta, y dijo\ldots{} «Cuidarla mucho. Si muere,
avísenme\ldots» También ha venido tu padre\ldots{} muy triste de verte
enferma, alegre porque ya le han colocado\ldots{} Está muy agradecido a
Doña Domiciana\ldots{} No bien abrió la boca, la señora se puso la
mantilla y salió a la calle en busca del remedio. Al día siguiente
¡pumba! el destino. Esto es servir con prontitud y equidad.»

---Domiciana tiene influencia; digo, se la prestan. Es una culebra que
lleva de aquí para allá los recados de las águilas\ldots{} Otra cosa:
¿en qué oficina está mi padre ahora?

---No le han metido en ninguna cosa del Gobierno, oficina ni
\emph{viceversa} teatro, sino en una casa particular, y por ello está tu
padre más contento. Ha entrado a servir a ese que regenta toda la
policía, el D. Francisco Chico, que prende criminales y espanta
masones\ldots{}

---Entonces, mi padre estará al cuidado de las horcas.

---No, que el destino que tiene no es más que limpiar el polvo a los
cuadros, cornucopias, urnas y tapices que el D. Francisco tiene en una
gran casa de la Plaza de los Mostenses\ldots{} Y por quitar el polvo y
cuidar de aquellos almacenes, le dan a tu padre ocho reales y casa. Dice
que no hay en Madrid destino de más descanso. Si satisfecho estaba el
hombre en el Teatro Real, gozando de tanta música y baile, ahora salta
de gozo porque come y vive con poco trabajo, entre tantas cosas lindas y
nobles\ldots{} ¿Qué te parece, mujer, de la colocación de tu padre?»

Lucila no respondió más que con un áspero rechinar de dientes.

\hypertarget{xx}{%
\chapter{XX}\label{xx}}

Hallándose mejorada, recibió Lucila las visitas de su hermano y de su
padre, el cual reiteró su contento por el buen acomodo que tenía en la
casa del jefe de los guindillas; pero no habló nada de Domiciana. Esta
preterición de la protectora le pareció a Cigüela un delicado tributo de
Ansúrez al dolor de su amada hija. Sin duda el \emph{fiero castillano}
comprendía o sabía que las que fueron amigas hallábanse ya a un lado y
otro de un espantoso abismo. No quería él meterse a medir la sombría
cavidad, y callaba. Con interés real o fingido escuchó después Lucila
las descripciones que hizo su padre de los primores cuya limpieza le
estaba encomendada, y tomando pie de esto se procuró personales informes
del Sr.~Chico: si en su casa tenía el mal genio que desplegaba en la
persecución de gente mala; si recibía con buenas palabras o con bufidos
a las personas que iban a verle. Las opiniones de Ansúrez sobre estos
particulares eran vagas. Desconocía completamente a su amo en las
funciones policiacas. Sólo de pensar que ante él se veía como
delincuente, como sospechoso, siquiera como testigo, le entraban
temblores y se le descomponía todo el cuerpo. Terminó recomendando a su
querida hija que no pensara en tal sujeto, \emph{al cuento} de averiguar
por él cosas que valía más dejar en el estado que tenían, cuidándose
menos de descubrirlas que de olvidarlas. Esto fue, en substancia, lo que
el innato filósofo celtíbero dijo a su amada Lucila.

La Capitana Rosenda, que también a la guapa moza visitaba muy a menudo,
no le habló nunca con tan filosófico tino como el viejo castellano.
Divagaba locamente en su charlar, a las veces gracioso. Deportado
Castillejo, se había ido a vivir con una tía suya, en la Cava de San
Miguel, señora de circunstancias, que tenían dos loros, una cotorra y
cuatro jilgueros\ldots{} En la misma casa, piso principal bajando del
cielo, vivía el desesperado cesante D. Mariano Centurión, cuya familia
se comunicaba con la de la tía de Rosenda por ser esta y la
\emph{Centuriona} del mismo pueblo. Los niños bajaban; la señora
pajarera subía, y D. Mariano, cuando no tenía con quién desfogar, le
contaba sus desventuras a la Capitana. Por él supo que la cerera se
empingorotaba cada día más. En coche salía por Madrid, y en coche
llegaban \emph{personajas} a platicar con ella. Vestía muy elegante, los
morros le habían crecido, y con ellos y con su entrecejo, cuando iba por
la calle, parecía decir: «quítense, quítense, que paso yo.» Rosenda la
había visto salir una mañana de la Vicaría. Llevaba una falda con
volantes, y tan ahuecada, que no cabía por la calle de la Pasa. Una
manola que tuvo que meterse en un portal para darle paso, le dijo con
desgarro insolente: «Madama, cuando paran los faldones guárdenos usté la
cría\ldots» Y otra vez: «Está tan echada a perder la cerera, que el
mejor día la vemos de Ministra. ¿Pero no sabe usted lo que dicen? Pues
que ha pedido a Roma dispensa de votos para casarse\ldots{} Con
influencias todo se consigue en la Curia Romana, y ella cuenta con el
Embajador Castillo y Ayensa, con el Nuncio de acá, con las
\emph{Madres}, los \emph{Padres} y el Rey Marido. Y se saldrá con la
suya, que esta gente tiene la Santísima Trinidad en el bolsillo\ldots{}
¿Qué\ldots{} usted no lo cree?» Y el mismo día: «Si le dicen a usted,
Lucila, que el desaparecerse Bartolomé es cosa de sus padres, y que
estos, por medio de la policía, le cogieron para llevársele a Medellín y
esconderle allá, no haga caso. El padre de Bartolomé, D. Manuel Gracián,
no se ha movido de Medellín, y tiene a su hijo por cosa perdida. Lo sé
por un sobrino de D. Manuel, tratante en ganado de cerda, con perdón. A
Madrid llegó la semana pasada; le conocí cuando estuvimos de guarnición
en Don Benito\ldots» Y al día siguiente: «No esté usted tan alicaída, ni
tome estas cosas con demasiada calentura\ldots{} Ya parecerá el buen
mozo cuando menos se piense. Calma, y ojo a la cerera, pues por los
pasos de la gallina se ha de llegar a la nidada\ldots{} Como esta es luz
del sol, el Capitán está en la misma situación que estaba: sólo que
ahora el encierro es más riguroso, y no faltarán guardianes y
centinelas\ldots»

---Rosenda, por los clavos y las espinas de Nuestro Señor
Jesucristo---dijo Lucila ronca de ira,---no me diga usted eso; no me
encienda la sangre más de lo que la tengo\ldots{} Mire que del corazón a
la cabeza me suben llamas, y que le pido a Dios que me mate de
enfermedad, no de ira. Rosenda, lo que usted dice no tiene
sentido\ldots»

Esto dijo y esto pensaba, aunque en el caos de su mente y en el delirio
a que la conducía la tremenda desgarradura de su corazón, pensaba
también otras cosas, de peregrina originalidad, algunas muy semejantes a
lo que había expresado la Capitana. Todo su afán era examinar una tras
otra las probables versiones del suceso, y escoger la más lógica después
de bien pasadas por el tamiz dialéctico. Dígase en mengua del entender
suyo, que a veces designaba por más lógica la más absurda.

Y tres días después, volvía con nuevos datos la tremenda cronista: «¿No
le dice a usted nada el que la cerera no parece por aquí, y cumple
mandando al avefría de su hermano con un recado y unas pesetillas
envueltas en un papel? ¡Tan amigas antes, y ahora no viene a verla! Es
el miedo\ldots{} es la conciencia. Tan valentona para todo, y ahora se
asusta de un cordero\ldots{} Pues conmigo no le valía el
esconderse\ldots{} Bendito sea Dios, que soy de caballería, y si el que
me la hace huye de mí, ya sé yo ir a buscarlo y ajustarle la cuenta.
Mujeres como su amiga son poco para mí, y de esas necesito yo cuatro lo
menos para enjuagarme la boca. No es mal trote el que yo le daría por
encima de todos sus huesos\ldots{} Le quitaría yo todo el pelo
artificial, y si las muelas son naturales, pronto tendría que llevarlas
postizas\ldots{} ¡Ay! me figuro al pobrecito Bartolomé en la esclavitud
de esa tarasca\ldots{} Ya estará el hombre asqueado de aquellos morros
como los de una vaca, y hará cualquier brutalidad por libertarse\ldots{}
Cogidito le tiene, y bien sujeto, con la amenaza constante de la espada
que llaman de \emph{Demonocles}, que es la sentencia del Consejo de
Guerra, colgada sobre su cabeza. Porque el indulto será con su cuenta y
razón, y ella lo da o lo quita según cumpla o no cumpla el bendito
Bartolo\ldots{} Mucho se adelantaría si supiéramos dónde ha metido la
gavilana el gallito que se llevó entre sus uñas puercas.»

---Pronto lo sabré yo---dijo Lucila con el aplomo que le daban sus
inquebrantables resoluciones.---Ya estoy buena; Dios me ha hecho la gran
merced de dejarme con vida después de este horrible padecer\ldots{} y
con la vida me va dando salud y fuerza, señal de que no quiere que yo me
deje pisotear\ldots{} Estos días saldré a la calle, iré a buscar
trabajo, pues de algún modo he de vivir\ldots{}

---¿Trabajo ha dicho, para una mujer pobre y sola? Diga que va en busca
de miseria\ldots{} ¡Afanes, vida de perros! ¿para qué? ¿para un mal
comer y para que se rían de una? Siga usted el consejo de una
desengañada, que ha visto lo que dan de sí trabajitos y honradeces de
poca lacha. Lo que tiene usted que hacer es vestirse decentita y bien
apañadita, y darse aire por ahí, para que su mérito sea como quien dice,
público. En los tiempos que corren no le aconsejaré que se vaya por los
paseos y sitios mundanos, sino que frecuente dos o tres iglesias y haga
en ella sus devociones, a la mira de los señores buenos, de asiento y
juicio, que no por pertenecer a cofradías y ser buenos rezadores se
olvidan del culto de Santa Debilidad\ldots{} pues el hombre siempre es
hombre, aunque peque de beato\ldots{} Si no tiene usted ropa decente,
más claro, si no quiere ponerse la que le dio la cerera, yo le
facilitaré cuanto necesite, y aunque soy de más carnes y corpulencia,
usted, que es buena costurera arreglará mis vestidos a su talle\ldots{}
Aquí me tiene usted a mí, que escarmentada de andar con loquinarios,
barricadistas y patrioteros, que cuando no están presos los andan
buscando, me voy por las mañanas muy bien arregladita, como viuda
consolable, a San Justo o la Almudena, y por las tardes a las Cuarenta
Horas de San Sebastián o San Ginés, parroquias de feligresía muy buena,
superior. De seguro que allí me ven y estiman caballeros viudos
respetables, de cincuenta y pico, o de los sesenta largos, que desean
hablar con mujer ya sentada\ldots{} No le digo a usted más\ldots{}
Piénselo, y escoja sus caminitos. Como la quiero a usted, por cincuenta
coros de arcángeles le pido, amiga mía, que no se meta en trabajillos de
aguja, quemándose las pestañas por dos reales y medio al día, porque en
ese trajín se morirá de hambre, y se perderá con un albañil o un
zapatero, que es la peor perdición que puede salirle.»

No expresó Lucila su conformidad con estas exhortaciones; pero tampoco
las rechazó. Aceptado y agradecido el ofrecimiento de ropa, el mismo día
le llevó la Capitana no pocas prendas, en cuyo arreglo se puso a
trabajar para poder usarlas cuanto antes\ldots{} Por fin se echó a la
calle, y recorrió las que a su parecer frecuentaba Domiciana en su
diario trotar de Palacio a la cerería o al Convento. No la encontró
nunca. Acechando en la calle de Toledo, vio que la exclaustrada llegaba
por la noche a casa en coche de dos caballos. El mismo coche iba en su
busca al siguiente día y a variadas horas\ldots{} Divagando topó Lucila
una tarde con Centurión, que puso en su conocimiento pormenores de
indudable interés. La señora Sarmiento de Silva estuvo en efecto
malísima; algunas noches Domiciana dormía en Palacio; y tanto se había
remontado en su orgullo la misteriosa hija de D. Gabino, que era preciso
echarle memoriales para poder hablar con ella dos palabras. Últimamente,
apiadada o aburrida, le había prometido colocarle en la Comisaría de
Cruzada, ya que en Palacio no podía ser hasta mejor ocasión\ldots{} Al
despedirse del cesante, tomó Lucila el camino del Rastro, ávida de
comprar algunas cosillas que le hacían mucha falta.

Una mañana fresca, luminosa y risueña, en que un sol artista iluminaba
los alegres colorines de la calle de Toledo, y sobre la variedad
infinita de gamas chillonas derramaba el oro y la plata, acechó Cigüela
la cerería, desde la acera de enfrente, ocultándose entre la muchedumbre
que sin cesar pasaba. Por una naranjera cuyo espionaje había comprado en
días anteriores, supo que Domiciana estaba en casa. Llegó tempranito en
carruaje de dos caballos. Sin duda pasó la última noche en la vela y
guarda de Doña Victorina. Sabido esto, continuó la moza su vigilancia
hasta que vio salir a D. Gabino y perderse calle . Segura de que
Ezequiel quedaba al cuidado de la tienda; contando con que Tomás estaría
en el taller, entró decidida\ldots{} «Dichosos los ojos---le dijo
Ezequiel, encantado de verla.---Lucila, ¡qué soledad sin ti!» Fue la
moza, en derechura, hacia la puerta que con la escalera comunicaba. El
chico la contuvo expresando temor. «Aguarda. Ha dicho Domiciana que no
suba nadie.» Viéndole en actitud de interceptarle el paso, la mano
puesta en la llave, Cigüela le desarmó con una frase cariñosa que al
mismo recelo habría inspirado confianza. «Tontín, conmigo no va eso. Mi
amiga es Domiciana, hoy como siempre. Vengo a pedirle un favor\ldots{}
¿No sabes que estoy desamparada?» Vacilaba el mancebo. Para ganarle por
entero, Lucila empleó una sonrisa pérfida; le pasó la mano por la cara,
diciendo estas palabras de pura miel: «Déjame, rico.»

Cedió \emph{Zequiel}, y en aquel momento alguien que había entrado en la
tienda daba golpes en el mostrador. «Vete a despachar,
rico\ldots---murmuró Lucila, y bonitamente quitó la llave, la puso por
dentro, cerró con cuidado para no hacer ruido\ldots{} Guardó la
llave\ldots{} con paso de gato se deslizó escalones arriba, diciendo:
«No sale; no me ha sentido cerrar la puerta. Está dormida.»

\hypertarget{xxi}{%
\chapter{XXI}\label{xxi}}

La cerera, que nunca se acostaba de día aunque hubiera hecho noche
toledana, habíase despojado de sus ropas mayores, quedándose en las
menores, que reforzó con un \emph{desabillé} holgadísimo en forma de
brial, de lana azul guarnecido de seda negra. Quitado el corsé para que
los pechos descansaran en libertad, estirándose a su gusto, y sustituido
el calzado duro por las blandas chilenas rojas, se acomodó en un sillón
de su alcoba. Al poco rato, medio pensando en lo pasado, medio
imaginando lo futuro, empezó a descabezar un sueñecillo\ldots{} En él
estaba cuando hirió sus oídos el ligero son rasgado de la cerradura de
abajo\ldots{} se estremeció; abrió los ojos, los volvió a entornar,
diciéndose: «Es Ezequiel que cierra\ldots{} Le mandé que cerrara.»

Al oído de la señora adormilada no llegó ruido de pisadas gatunas en la
escalera y pasillo. Más que por efectos de sonido, por efectos de luz se
le sacudió aquel sopor. La menguada claridad solar, como de entresuelo,
que alumbraba el gabinete, a la alcoba llegaba tan reducida, que si la
interceptaba en la puerta un cuerpo de persona, era casi nula. La
obscuridad que proyectó el bulto de Lucila fue para la cerera un brusco
despertador que le dijo: «Despabílate, que hay moros en la costa.»

Dudó por un instante la exclaustrada si era realidad o sueño lo que
veía. Conoció a Cigüela, como a un espectro ya familiar; mas como era
espectro nada le dijo; no hacía más que mirarlo, aterrada, esperando que
se desvaneciera\ldots{} que al fin los espectros, después de asustar un
poco, acaban por desvanecerse. «¿Duermes, Domiciana?---dijo Lucila
avanzando, y la voz de la guapa moza sonó con tan extraña alteración de
su timbre ordinario, que la cerera la desconoció. La voz de esta sonaba
también muy a hueco, al decir tras una breve pausa: «Lucila, ¿eres tú?»

---Yo soy. ¿Ya no me conoces?---murmuró Lucila con la misma voz de
secreteo lúgubre.---¿Creías que me había muerto?»

Ya no hubo duda para Domiciana. Lo que veía no era espectro, sino
persona. La realidad de esta poníala en el duro caso de afrontar la
situación para ver de sortearla. No había escape. Era Lucila, en su
propio ser, y a juzgar por el tono y por la forma insidiosa de su
entrada en la alcoba, seguramente venía de malas. Domiciana tuvo
miedo\ldots{} El miedo mismo le sugirió el empleo de frases de
concordia, fingiendo naturalidad: «Mujer, qué cara te vendes\ldots{}
Siéntate\ldots{} Pensaba ir a verte\ldots{} Yo muy ocupada, hija.»

---Para que no te molestaras he venido yo---dijo aproximándose
Lucila.---Necesitaba preguntarte una cosa\ldots{} una cosa que se te ha
olvidado decirme, ya supondrás\ldots{} Acortemos conversación. Vengo a
que me digas dónde está Tomín.»

Había previsto Domiciana la tremenda reclamación de su amiga. Quiso
hacer frente al conflicto por medio de fórmulas evasivas, de expresiones
conciliadoras, de paliativos mezclados con promesas\ldots{} El gran
talento de la cerera se equivocó por aquella vez. «Ven aquí\ldots{}
hablaremos\ldots{} ¡Pobrecilla\ldots! Te contaré---le dijo levantándose,
en actitud de llevarla al gabinete.» «No, de aquí no sales\ldots{} aquí
hablaremos todo lo que sea preciso---contestó Lucila deteniéndola con
mano vigorosa. En aquel momento, viendo más cerca el inmenso peligro, la
cerera evocó su sangre fría para sortearlo, ya que no pudiese acometerlo
de frente. ¿Por qué no hemos de salir a la sala? Allí estaremos
mejor\ldots{} Bueno, pues si quieres\ldots{} aquí\ldots{} Verás\ldots{}
Me alegro de que hayas venido, porque así\ldots»

Lucila, mirándola frente a frente, y poniéndole la mano en el pecho, le
soltó con voz iracunda toda la hiel de su alma: «Mala mujer, dime al
momento dónde está Tomín\ldots{} Quiero saberlo\ldots{} Vengo a
saberlo\ldots{} No me voy sin saberlo\ldots{} Y como te niegues a
decírmelo, Domiciana\ldots{} te mato.»

Creyó Domiciana que el \emph{te mato} era un decir, pues arma no
veía\ldots{} «Mujer, no escandalices---le dijo.---No hay para qué tomar
las cosas de esa manera. Yo te explicaré\ldots{} Pero sosiégate\ldots{}
no escandalices.»

Con sólo un ligero impulso de la mano que Lucila le había puesto en el
pecho, Domiciana dio un paso atrás y cayó en el sillón. «Si no
escandalizo\ldots{} y aunque escandalizara, aunque tú chillaras, no te
valdría. He cerrado con llave la puerta, y no vendrán a
defenderte\ldots{} Porque yo te mato, Domiciana; he venido a
matarte\ldots{} siempre y cuando no me contestes a lo que te pregunto:
¿Dónde está Tomín? Porque tu amiga, la que conociste cordera, es ahora
leona. Días hace que toda la sangre se me ha subido a la cabeza\ldots{}
Yo era buena; tú me has hecho mala como los demonios\ldots{} Al infierno
voy; pero tú por delante\ldots»

---¡Lucila, por Dios\ldots!

---¡Traidora! Tú me has enseñado la maldad, y como traidora entro
también en tu casa\ldots{} Por mala que yo sea, no seré nunca tan mala
como has sido tú conmigo, tú, que me has engañado con limosnas y con
palabras de cariño para entontecerme y robarme lo que es mío\ldots{} lo
que nunca será tuyo\ldots{} vieja ladrona.

---¡Lucila, Lucila\ldots!---exclamó la cerera cruzando las manos,
abrumada.

---Me has robado lo que no podías tener más que por el
ladronicio\ldots{} porque soy joven, soy hermosa, y vale más un cabello
mío que toda la fisonomía de tu rostro sin gracia, y más sal echo yo de
una mirada que tú de todo tu cuerpo y persona de animal en celo\ldots{}
Monja salida, hembra sin corazón, boticaria, intriganta, encomiéndate a
Dios, sí no me contestas al instante.»

Diciendo esto, de entre los pliegues de un manto de talle que llevaba
cruzado sobre el pecho, sacó un largo cuchillo de afilada y espantable
punta. Vio Domiciana la hoja que brillaba como un rayo, vio la vigorosa
mano que empuñaba el mango, y se tuvo por perdida. Encomendó a Dios su
alma\ldots{} Mas en aquel instante, el poderoso talento de la cerera y
el grande esfuerzo de voluntad que hizo concurrieron a darle una fuerza
resistente ante la agresiva fuerza de su rival, ciega, disparada, fácil
de desarmar con una palabra y un gesto que la hirieran en lo vivo.

Con un inspirado grito en que puso toda su alma, detuvo Domiciana el
impulso trágico, y fue así: «Lucila, amiga y hermana, no mates a una
inocente. Cálmate, y sabrás\ldots{} lo que quieres saber del hombre que
te adora.» La vacilación de Lucila en el momento de oír esto fue la
primera ventaja de la cerera, débil ventaja, pero que habría de ser más
considerable si aprovecharla sabía. Para ello necesitaba Domiciana
condensar en un punto toda su voluntad, dirigiéndola con el soberano
talento que le había dado Dios. Por lo que hasta aquí se conoce de la
vida de esta mujer singular, se habrá comprendido que eran
extraordinarias su penetración y astucia. Poseía en alto grado el
sentido de las circunstancias, el repentino idear y el rápido resolver
ante un conflicto. Si estas cualidades bastaran para gobernar a los
pueblos, habría sido Domiciana una gran mujer de Estado\ldots{} Pues en
aquel inminente peligro, la hoja desnuda en la mano de Cigüela, el alma
de ésta embravecida, vio que entre la vida y la muerte había menos
espacio que el grueso de un cabello, y menos tiempo que la duración de
un relámpago. Relámpago fue este razonamiento: «Muerta soy si me
achico\ldots{} Sálveme mi entereza\ldots{} Sálveme medio minuto de
talento mío y de vacilación de ella.» Prosiguió en alta voz:

---Déjame que hable, y mátame después si quieres. Yo no temo la
muerte\ldots{} Sé morir por la verdad\ldots{} ¿Qué es eso de matar sin
oír? Mis explicaciones han de ser largas.

---Pues abrévialas todo lo posible. ¿Dónde está Tomín?»

Repitió la pregunta con menos fiereza que la primera vez. Otra ventaja
pequeñísima de la cerera; pero ventaja\ldots{} Rápidamente la aprovechó,
como perfecto estratégico. «¡Pobre Cigüela! veo que tu amor por Tomín no
desmerece del que él te tiene a ti\ldots» Lucila la miró perpleja sin
mover la mano en que el arma tenía. Con genial inspiración, Domiciana
hizo un quiebro repentino, caudillo que ordena un movimiento de
sorpresa. «Oye una cosa, y espérate un poquito, si de veras es tu
intención matar a tu amiga, que tanto te ama: ¿Verdad que todo tu furor
es porque han pasado mucho días sin que yo te viera, sin que yo te
llamara\ldots? Dímelo, confiésalo\ldots{} ¿Verdad que es por esto?»

---Huías de mí porque yo era tu conciencia, porque me tenías miedo,
porque el mirarte había de ser para ti como si Dios te mirara, porque
tienes el alma negra, y los malos como tú no quieren que les vean los
buenos, los engañados, los burlados. Habla pronto, respóndeme a lo que
te pregunto\ldots{} Mira que estoy frenética, mira que no te dejo hasta
que me digas lo que sabes, o me entregues tu sangre, toda tu sangre.»

Desventaja de Domiciana, y no floja. Vio el punto culminante del
peligro, la muerte, y acudió con un recurso heroico y de extrema
agudeza. Necesitaba para emplearlo de un valor casi sobrehumano y de un
fingimiento de serenidad que era el supremo histrionismo. Pero no había
más remedio. Se trataba de no perecer. «Bestia---dijo abriendo los
brazos y mostrando indefenso su pecho,---si quieres matarme, aquí estoy.
Ni sé ni quiero defenderme\ldots{} ¿Para qué sirve esta miserable vida
humana? Para ver tanta infamia, tanta ingratitud\ldots{} para que las
personas que miramos como hermanos quieran asesinarnos\ldots»

---Hermana te fingiste, pero no lo eras---dijo Cigüela con pérdida de
energía.

---Y ahora resulta que soy mala---prosiguió Domiciana con avidez de
aumentar la pulgada de terreno que la otra le diera.---¡Mala yo, que a
ti y a Gracián favorecí; mala yo, que a él le he salvado la vida, no
tanto por él como por ti, sabiendo que te ama; mala yo, que no miro más
que a conseguir que se case contigo\ldots!»

Excediose un tanto en la maniobra lisonjera, y de este exceso tomó
ventaja Lucila, que aunque muy crédula en situación normal, en aquella
tiraba instintivamente a la desconfianza. «Domiciana---dijo apretando el
mango del cuchillo,---si crees que ahora jugarás también conmigo, te
equivocas\ldots{} No vengo por dedadas de miel, sino por verdades. Las
verdades te las sacaré de la boca, o te dejaré seca\ldots{} Soy mala
ya\ldots{} y no perdono.

---Lucila---replicó la otra con rápido pensamiento,---¿cómo he de
decirte verdades si no quieres oírme? Para decirte las verdades necesito
hablar, referirte muchas cosas. Te juro por lo más sagrado que nunca
dejé de quererte, ni de interesarme por ti\ldots{} ¿No lo crees? Peor
para ti y para tu alma. Yo tengo mi conciencia tranquila; no temo la
muerte; pero por mucha que sea mi serenidad, ¿cómo quieres que hable y
me explique, en cosas tan delicadas, viendo delante de mí un puñal, y
oyendo decir \emph{te mato}, \emph{te mato}? Una cosa es no temer la
muerte, y otra es el asco de ver una derramada su propia sangre, y la
dentera que dan esos cuchillos, y el ver a una persona tan querida
poniéndose al nivel bajo de los matachines y rufianes, de la última
gentuza del Avapiés\ldots{} Mujer, si eres realmente mala, no lo
parezcas mientras estés delante de mí.

---Si quieres que yo te crea, explícate pronto---dijo Lucila perdiendo a
escape terreno.---Te da miedo el cuchillo. ¿Pues no me dijiste `mátame'?

---Sí: yo acepto la muerte\ldots{} Pero mi resignación al martirio no me
quita la repugnancia de verte como una \emph{chulapona}, como una maja
torera de las más indecentes\ldots»

Comprendiendo con segura perspicacia el efecto que hacía, apretó de
firme en esta forma: «No me espanta el odio, no temo el extravío ni la
locura de un enemigo; rechazo, sí, las malas formas, la grosería, la
chabacanería, la estupidez bajuna. No puedo acostumbrarme a verte a ti,
tan linda, tan señorita de tu natural, convertida en gitana asquerosa,
en charrana mondonguera, tan diferente a ti misma\ldots{} No puedes
hacerte cargo, hija mía, de lo ridícula que estás, y de lo repulsiva y
fea\ldots»

---No te cuides tanto de como estoy, y contéstame, Domiciana---dijo la
guapa moza apoyando en la cama la mano en que tenía el cuchillo.---A mí
no me importa estar fea o bonita, pues sólo quiero ser justiciera.

---¡Justiciera, y empiezas por amenazar antes de oír!

---Amenazo; pero eso no quiere decir que no escuche. Si para explicarte
con claridad es estorbo el cuchillo, aquí lo dejo\ldots{} ya ves\ldots{}

---Está bien---dijo Domiciana, que sin mirar la mano vio el arma muy
distante de esta.---¡Si para matarme tienes tiempo! Pero no lo harás,
pobrecilla, porque con lo que voy a decirte quedarás convencida y te
avergonzarás de haberme ofendido bárbaramente.

---Domiciana---dijo Lucila sin darse cuenta del progresivo enfriamiento
de su furor homicida,---loca entré en tu casa, y tú vas a volverme más
loca de lo que vine\ldots{} Dices bien: tengo tiempo de matarte. Como yo
vea que me burlas, de mí no escapas. Te lo juro, por Dios te lo juro,
que si hay justicia en el cielo, también debe haberla en la tierra. Dejo
el cuchillo y te escucho.

---No basta que lo dejes; es menester que arrojes lejos de ti lo que
deshonra y mancha tu mano honrada---dijo Domiciana cogiendo el arma con
rápido movimiento, y arrojándola por detrás de la cama, próxima a la
pared. Sólo de esta la separaba el preciso espacio para que el cuchillo,
lanzado con ojo certero, cayese al suelo en lugar donde Lucila no podía
recobrarlo fácilmente, porque bajo el lecho hacían barricada
infranqueable un cofre chato y dos cajas de ingredientes químicos.

\hypertarget{xxii}{%
\chapter{XXII}\label{xxii}}

Desarmada Lucila, Domiciana se vio salvada, y celebró mentalmente su
triunfo sin dar a conocer su alegría. Menos cauta la otra y de escaso
talento histriónico, dejó ver su desconsuelo por la distancia entre su
mano y el arma. «Me ha cortado la acción: ya no me tiene miedo---dijo
para sí clavando sus miradas en la cerera.---Pero no le vale\ldots{} La
mataré otro día si me engaña, para que no engañe a nadie más.»

Recobró Domiciana el timbre neto de su voz, de la cual solía decir
Centurión: «Es dulce y dura como el azúcar piedra.» Con dureza dulce,
dijo la exclaustrada: «Amiga querida, debiera yo ser un poco severa
contigo, pues lo que has hecho, en verdad que no te recomienda; pero te
quiero tanto, que sin sentirlo me voy al perdón\ldots{} Ahora sabrás,
ahora te contaré\ldots{} verás quién es y cómo se porta esta tu amiga,
esta mala mujer, a quien querías matar\ldots» Dejó el sillón con ademán
de vencer la pereza, y cogiendo del brazo a Lucila le dijo: «¿No te
aburres de esta obscuridad?\ldots» La guapa moza, sacudiéndose el brazo,
siguió detrás de Domiciana, que al pasar al gabinete ampliaba la frase:
«La obscuridad me entristece, y tú más\ldots{} con tus tonterías. Ven
acá. Sentémonos aquí, y despéjense nuestras cabezas\ldots»

Los pocos pasos que había entre alcoba y gabinete llevaron a Domiciana
desde el mundo del miedo al de la seguridad. La luz benéfica, el ruido
de la calle, la confortaron, como conforta la realidad después de
oprimente pesadilla. La idea del tremendo peligro pasado aún estremecía
sus carnes; el recuerdo de cómo lo conjuró con un prodigioso rasgo de
inteligencia la colmaba de vanagloria. «¡Qué lista soy!---se dijo.---He
sabido engañar a la misma muerte, que ya me tenía cogida. Con la argolla
al cuello, he convencido al verdugo\ldots{} para que se estuviera quieto
y no apretara\ldots{} Si esto no es talento, que venga Dios y lo vea.»

Al pasar de la penumbra del dormitorio a la luz del gabinete, tuvo
Lucila clara conciencia de que Domiciana, con heroica maña más potente
que la fuerza heroica, se había hecho dueña del campo de combate. Mas no
por esto se acobardó la moza, que firme en su plan justiciero esperaba
llevarlo adelante de una manera o de otra. ¿Y por qué había de ser la
muerte el mejor instrumento de justicia? ¿No había instrumentos más
eficaces que realizaran el fin de justicia sin manchar la mano del juez?
Pensando en esto y antes que la exclaustrada rompiera el silencio, le
dijo: «Si has tenido arte para desarmarme, no creas que te libras de mí.
Por lo ocurrido en tu alcoba se ve bien claro que no soy mala, que me
doy a razones, y que si entré a matarte fue por arrebato y furia de
venganza\ldots{} cosa natural\ldots{} Una es mujer, es una joven\ldots{}
tiene corazón, sangre\ldots{} Bueno: pues te digo con toda franqueza que
si motivos tengo muchos para odiarte, también te debo gratitud, no por
los socorros de aquellos días, que eran traicioneros como el beso de
Judas, sino por lo de hoy\ldots{} Tú, por tu defensa, me has quitado de
la cabeza el matarte, que habría sido grande atrocidad, un bien para ti
porque te ibas al descanso, al Purgatorio quizás, puede que al Cielo, y
mal para mí, que ya estaba perdida, y la cárcel, quizás el palo, no
había quien me lo quitara\ldots»

Con lástima la miraba ya la cerera. «¡Cuitadilla!---dijo para sí.---Ya
no tiene más arma que estas teologías que ni pinchan ni cortan. Se deja
coger como una pobre pulga, y si quiero la estrujo entre mis dedos.»

Lucila prosiguió así: «Domiciana, más baja te veo despreciada que
muerta.»

---Y yo te digo que lo mismo te quiero alucinada que con sentido---dijo
la otra trasteándola con suprema habilidad.

---Pues si me devuelves el sentido, si con razones y explicaciones que
vas a darme me convences de que eres buena y de que yo no he sabido
comprenderte, la que quiso matarte te pedirá perdón\ldots{} será
capaz\ldots{} si fuese menester\ldots{} de dar la vida por ti\ldots»

Y Domiciana, mirándola y moviendo la cabeza con acento de maternal
tolerancia, se regaló a sí misma este mudo juicio acerca de su rival:
«De esta simple haré yo lo que quiera. Alma de Dios, corazón inocente,
toro que obedece al trapo\ldots{} tú sola te amansas, tú sola te
entregas\ldots{} Consérveme Dios la inteligencia para con ella
merendarme a estos corazones arrebatados\ldots» Y luego, en alta voz:
«Lucila, hermana mía, yo no te ofendí; yo no soy responsable de que se
desapareciera Tomín. Sobre el poder que yo tenía y tengo, se levantó
cuando menos lo pensábamos, un poder superior\ldots{} Siéntate, ten
calma; no te impacientes. Yo, de algunos días acá, estoy mal del
pecho\ldots{} no sé qué me pasa\ldots{} Tengo que tomar aliento a cada
cuatro sílabas\ldots{} y si hablo mucho rato sin parar, me quedo como
ahogada\ldots»

Estas últimas indicaciones no tenían más objeto que ganar tiempo.
Después del gran esfuerzo intelectual para esquivar el inmenso riesgo de
morir asesinada, la cerera necesitaba de un colosal derroche de
inteligencia para levantar el artificio de figurados hechos ante el cual
se desplomaran los agravios de Lucila; érale preciso construir una
historia y presentarla luego con tal riqueza de lógicos razonamientos y
tal encanto narrativo, que a la misma verdad imitase y a la misma
incredulidad convenciese. Esto, ni aun para tan hábil maestra del
pensamiento y de la palabra era cosa fácil: necesitaba serenidad, algo
de reflexión de filósofo, algo de inspiración de artista, y para estos
algos hacían falta los del tiempo\ldots{} Favorecida por el Cielo aquel
día, cuando acabó de decir que la fatigaba el mucho hablar llamaron a la
puerta de abajo. Esto fue muy de su gusto; contaba ya con que alguien de
la familia echase de ver que la puerta estaba cerrada por dentro, y
llamara con alarma impaciente. Así fue: arreciaron los golpes. Domiciana
dijo: «Mira en qué ocasión vienen a interrumpirnos. Ahora caigo en que
cerraste la puerta. Más vale que abras, pues si no, se asustarán, y con
razón. Creerán lo que no es, y\ldots{} hasta puede suceder que echen
abajo la puerta.» Vaciló Cigüela. ¿Pero qué hacer podía la infeliz más
que abrir? A merced estaba de su enemiga.

Entraron y subieron D. Gabino y Ezequiel, inquietos, y anticipándose a
sus manifestaciones, Domiciana les dijo: «Mandé a esta que cerrara
porque teníamos que hablar, y me sabía muy mal que nos interrumpieran.
¿Quién ha venido?»

---Ha estado el amigo Centurión---dijo el cerero recobrando su
tranquilidad,---pero se ha cansado de esperar\ldots{}

---Y ahí tienes el coche; viene a buscarte---anunció el mancebo, que
dirigía las locuciones a su hermana y las miradas a la hija de Ansúrez.

---Tengo que vestirme. Lucila, ¿has visto que vida llevo? Apenas
descanso un ratito, ¡hala otra vez!

---Si comes tú en Palacio---dijo D. Gabino acaramelando la
mirada,---Luci comerá con nosotros.

---Quería yo llevarla conmigo. Pero si ella prefiere quedarse\ldots{}
¿Verdad que está Cigüela más guapa?

---En la guapeza de esta joven no cabe más ni menos. Es como la bondad
de Dios---declaró D. Gabino, reblandeciendo la expresión de sus ojos,
que eran manantiales de ternura, y alargando la boca, húmeda como el
hocico de un becerro.---Si Cigüela come con nosotros, traeremos dos
platos de casa de Botín, y de la pastelería huevos moles o huevo hilado,
lo que a ella más le guste.»

Encandilado, moviendo los brazos en forma de un batir de alas de ángel,
Ezequiel aprobaba con mudo entusiasmo.

---Mucho se lo agradezco, Sr.~D. Gabino---dijo Lucila;---pero\ldots{}
Otro día comeré con ustedes. Hoy no puede ser. ¿Verdad, Domiciana?

---Hija mía---dijo la cerera con admirable afectación de cariño,---tú
dispones lo que gustes. Has reconocido hace poco que soy para ti como
una hermana, como una madre\ldots{} Después que hablemos otro ratito,
quédate a comer. Estás en tu casa.»

Oyendo esto, no sabía Cigüela si admirarla por su ingenio, o tronar
indignada contra tan cruel ironía. Pensó que sería justicia y además un
desahogo muy placentero, arrancarle el moño y chafarle los morros de una
o más bofetadas. En un tris estuvo que lo intentara. Midió la acción y
vio que cabía perfectamente dentro de sus facultades, pues le bastaban
las manos para despachar a la cerera, reservando las extremidades
inferiores para D. Gabino, a quien tiraría al suelo de una patada. A
Ezequiel le derribaría sólo con el aire que hiciera en toda esta
función. Mas para esto siempre había tiempo. Convenía esperar\ldots{}

En aquel punto entró la asistenta que a la familia servía, mujer de gran
talla, bigotuda, con todo el aire de un cabo de gastadores, y después de
un breve saludo al ama, llevando consigo el cesto de la compra ya
repleto, se fue a la cocina. Creyérase que Domiciana, viéndose asegurada
por aquella guardia formidable, recobraba en absoluto su tranquilidad.
Despidió a su padre y hermano, encargándoles que a nadie dejaran subir,
y sintiéndose bien custodiada y defendida, pues el son del almirez le
sonaba como los tambores de un ejército próximo, dedicose a su
vestimenta con todo sosiego. Quedó la otra en el gabinete, mientras la
cerera trasteaba en la alcoba, donde lo primero que hizo fue sacar el
puñal del abismo en que había caído y esconderlo en lugar seguro. Lucila
la vio salir risueña apretándose el corsé, y sin decir nada la ayudó en
aquella operación. En este tiempo, pudo la exclaustrada levantar en su
fecundo caletre el andamiaje de la soberbia historia que tenía que
construir, y apenas encaró con su enemiga, echó en esta forma los que a
su parecer eran sólidos cimientos:

---Tomín fue apresado por la policía y encerrado en Santo Tomás. Yo lo
supe un día después\ldots{} ya puedes figurarte mi disgusto\ldots{}
Naturalmente, acudí al instante. No me permitieron verle.

---¡Domiciana, por la salvación de tu alma---exclamó Lucila con solemne
acento,---por las promesas de Nuestro Señor Jesucristo, en quien tú y yo
creemos y esperamos, aunque seamos pecadoras, dime la verdad! ¿De veras
no has visto a Tomín? Júramelo, júrame que no le has visto\ldots{}

---Aguárdate, tonta, y no precipites mi relación. He dicho que no le vi
en aquel momento; luego sí\ldots{} Ten paciencia. Decía yo que acudí a
salvarle. No conté contigo porque estabas enferma. ¿A qué aumentar tu
desazón, tu desconsuelo?\ldots{} Habría sido matarte\ldots{} Pasaron dos
días en mortal ansiedad. Supimos que se trataba de aplicar al pobre
Capitán la pena terrible\ldots{} ¿sabes? la sentencia del Consejo de
Guerra. Tres señoras, tres, éramos a pedir misericordia por él. Doña
Victorina y yo\ldots{} y la de Socobio, que se nos agregó el segundo
día\ldots{} Eufrasia, hoy marquesa de Villares de Tajo: no la conocerás
por este nombre.

---La Socobio---dijo prontamente Lucila,---conspiró hace dos años por
los del \emph{Relámpago}.

---Pues ahora conspira por Narváez; es el más firme apoyo del
\emph{Espadón} en la Camarilla de la Reina\ldots{} Sigo contándote. Al
tercer día, después de haber hablado con O'Donnell, que nos dio
seguridades de que no sería fusilado el Capitán, fui a ver a
este\ldots{} Doña Victorina no podía ir; fui yo sola.

---¡Y le viste\ldots!

---Le vi\ldots{} y entre paréntesis, como me habías ponderado tanto su
hermosura, y creía yo encontrarme con un Adonis, o con el dios Apolo, la
verdad, no vi en él nada de particular\ldots{} un hombre como otro
cualquiera. Entré\ldots{} Con él estaba la Socobio, que sin darme tiempo
a exponer lo que me había dicho O'Donnell, saltó y dijo: «Ya no tiene
usted que ocuparse de nada. Yo lo arreglo todo\ldots{} Es cosa
mía\ldots»

---¿Y Tomín?

---En el corto rato que allí estuve, no habló más que de ti\ldots{} En
pocas palabras me dio las gracias por los favores que os hice, y luego:
¿qué es de Lucila, qué hace Lucila\ldots{} está buena Lucila?\ldots{} y
vuelta con Lucila. Bien echaba por los ojos el amor que te tiene.

---¿Y después\ldots?

---Volví al siguiente día\ldots{} Dijéronme que el Capitán estaba
libre\ldots{} Había ido por él la Socobio, y se le había llevado en su
coche\ldots{} ¿A dónde? Esta es la hora que no he podido saberlo.»

\hypertarget{xxiii}{%
\chapter{XXIII}\label{xxiii}}

La historia contada por Domiciana con acento tan firme que parecía el de
la propia Clío, produjo en el cerebro de Lucila efectos muy extraños,
pues si tales hechos encontraban en él como una nube de incredulidad
sistemática que los empañaba y obscurecía, de los mismos hechos brotaban
rayos de verosimilitud que esclarecían lentamente los espacios de
aquella nube. ¿Era mentira que parecía verdad, o una de esas verdades
que se adornan con las galas del arte de la mentira verosímil?

---¿Y por qué---preguntó Lucila con viveza ruda,---por qué al saber que
Tomín estaba libre, no fuiste a decírmelo?

---Porque me aterraba el tener que darte una mala noticia---dijo
Domiciana parando el golpe con gran destreza.---Lo era la de aquella
libertad, que tuve por una nueva esclavitud. Decirte que Tomín estaba en
poder de la Socobio era como decirte: «despídete de él por mucho
tiempo.»

---Por algún tiempo, quieres decir.

---Claro: terminado el secuestro, Tomín volverá a ser tuyo.

---¿Has dicho que esa Eufrasia conspira por Narváez?

---Por Narváez y Sartorius. El Gobierno la teme; mas no puede nada con
ella, porque se ha hecho uña y carne de la Reina, y es su confidente y
amiga. Se trata de combatir y anular esta influencia, expulsando para
siempre de la Cámara Real a la Socobio; en ello trabaja la persona que
más influye en el ánimo de Isabel\ldots{} ya puedes figurarte de quién
hablo\ldots{}

---¿Y qué importa que la Socobio sea o deje de ser amiga de la Reina?

---Esas amistades torcerán más el arbolito, que bastante torcido está
ya.

---No será la Eufrasia peor que otras, peor que tú. Dijo la sartén al
cazo\ldots{} Palaciegas de este bando y del otro, damas santurronas,
damas casquivanas, monjas aseñoradas, y señoras afrailadas, todas son
unas, y todas tuercen el árbol, porque torciéndolo, se suben a él para
coger fruta\ldots{} ¡Valiente ganado estáis!\ldots{} Pero en fin,
dejando eso, que no me importa, ¿sostienes lo que has dicho?\ldots{}
¿que la Socobio hizo escamoteo y se llevó a Tomín\ldots? ¿No temes que
yo hable con esa señora, y que ella me diga que la escamoteadora has
sido tú?

---Si hablas con ella, no te dirá una palabra, y te mandará a paseo. Es
gran diplomática. ¿Crees que una persona tan lista se franquea con el
primero que llega? ¿Quieres probarlo? Nada más fácil: en Aranjuez la
encontrarás. Ya sabes que allá se ha ido la Corte hace tres días. Ahora
tienes ferrocarril. Por catorce reales puedes ir en segunda\ldots{} Dos
horas menos minutos.

---¿Y cómo es que estando la Corte de jornada, aquí se queda Doña
Victorina, y tú con ella?

---Porque Doña Victorina sigue mal de salud, y no le convienen las
humedades del Real Sitio\ldots{} Y hay otra razón: mi amiga y yo somos
un cuerpo de ejército destinado a ocupar esta plaza y a vigilar en ella
los movimientos del enemigo. Tememos\ldots{} para que veas si te confío
cosas delicadas\ldots{} tememos que los narvaístas nos ganen el corazón
de la \emph{Madre}\ldots{} Lucila, ya sabes que estos secretos quedan
entre nosotras.

---Si el poder de la \emph{Madre} es tan grande, porque con su
misticismo y sus llaguitas hace creer que es enviada del Cielo, ¿qué
teméis de una disoluta como la Socobio, que ni tiene llagas, ni habla
con el Espíritu Santo?

---Se la teme porque es otra especie de santa, o por lo menos
sacerdotisa de un santo que no está en el Almanaque, de un santo que
siempre tuvo, tiene y tendrá tantos devotos como personas hay en el
mundo\ldots{}

---El Amor. ¡A quién se lo cuentas!

---Y dentro de ese culto infame, gentil, la Socobio es al modo de gran
teóloga o \emph{Santo Padre}, al modo de profetisa, definidora y
taumaturga\ldots{} y también tiene sus llagas o cosa parecida para
imponer veneración\ldots{} Se entiende con el dios de esta baja
idolatría, y trae recados de él para las criaturas\ldots{}

---Domiciana---dijo Lucila gozosa de ver a su amiga en aquel
terreno,---confiésame la verdad y todo te lo perdono. Confiésame que tú
también eres un poco, o un mucho, sacerdotisa de ese dios de los
gentiles, que tú también a la calladita adoras al ídolo\ldots{} porque
eres mujer\ldots{}

---Yo no. Ya sabes que no siento en mí esa devoción---dijo la
exclaustrada metiéndose en su concha.---Yo abomino de tales dioses
gentílicos\ldots{} He hablado de ello por explicarte la influencia de la
Socobio sobre una mujer joven, linda, y por poderosa caprichosa, y por
buena fácil a la maldad\ldots{} O hemos de poder poco, o apartaremos a
Eufrasia del Trono\ldots{}

---Del Trono y el Altar: dilo como lo decís en los papeles
públicos\ldots{} Déjate de hipocresías, y ya que hablas de eso, habla
con claridad. Tú y tu bando no miráis a que nuestra Reina sea buena,
sino a que seáis vosotras las únicas que le suministren sus diversiones.
Así la tenéis más cogida. Entre visiones celestiales por un lado y
terrenales por otro, no se os puede escapar.

---Hija, no hables así de nosotras, que tiramos siempre a la virtud y la
honradez\ldots{} Pero equivocándote, lo que has dicho revela talento.

---Esto que llamas talento, no lo es, Domiciana. Lo que yo sé, el
corazón me lo enseña\ldots{} Pues te digo que me alegraré mucho de que
con toda vuestra virtud seáis derrotadas por la Socobio, por esa gentil,
por esa idólatra\ldots{}

---¡Ah! no creas que estamos tranquilas---dijo Domiciana, tirando
siempre a ganarse la voluntad de Lucila y a desarmarla con las
confidencias verdaderas o falsas.---Esa maldita manchega es de la piel
del diablo. Hace meses, y cuando más descuidados estábamos, nos dio una
paliza tremenda\ldots{} Llamamos paliza a la derrota que sufrimos en un
asunto que creímos de los de clavo pasado; tan fácil nos parecía
resolverlo a gusto de la \emph{Madre}. Pues verás: Vacó la Comisaría
General de Cruzada, que es plaza muy lucida, enorme golosina de
clérigos; el Gobierno quería meter al poeta D. Juan Nicasio; la
\emph{Madre} hipaba por el Padre Batanero, que a sus muchos títulos unía
el de haber sido carlistón. Los moderados presentaron a D. Manuel López
Santaella, arcediano de Cuenca. De nada nos valió el tocar con tiempo
todas las teclas, porque esa perra se nos anticipó a mover los títeres
de Roma, donde su marido tiene relaciones y gran amaño por el negocio de
\emph{Preces}; y nada\ldots{} que nos ganó la partida, y quedaron
satisfechos Narváez y Sartorius, y nosotras burladas\ldots{} Para que la
\emph{Madre} no chillara, le dieron dedada de miel presentando al
Capuchino Fray Fermín de Alcaraz, el diablo de marras, para la mitra de
Cuenca\ldots{} Ahí tienes un triunfo del sacerdocio gentil sobre este
otro sacerdocio de ley. Eufrasia se quedó riendo, y Santaella pescó la
Comisaría. ¿Tienes noticia del famoso pasquín? Por cierto que cavilando
en quién podría ser autor de aquella chuscada, di en sospechar de
Centurión, y tanto hice y tanto le estreché que al fin me confesó que él
puso al pie de la estatua de Isabel, en la plaza del mismo nombre, el
letrerito de que tanto se habló en Madrid: \emph{Ni Santo él, ni Santa
ella}.

A este punto, ya Domiciana estaba vestida. Pero no quería partir sin ver
a su cara enemiga en completo desarme físico y moral. Sus confidencias
eran el plateado que a las píldoras ponía para que no amargasen, y en
las píldoras se mezclaban substancia de verdad y la mentirosa substancia
fina que usan los diplomáticos en las relaciones internacionales.
Verídico era mucho de lo que dijo referente a Eufrasia, y sobre el
sólido fundamento de estos hechos, asentó con gran maestría el artificio
del rapto del Capitán por la Socobio. Quedose Lucila meditabunda,
arrastrando sus miradas por el suelo y por las rayas de la estera frente
a la silla baja en que se sentaba. Interrogada por la cerera sobre la
causa de tan hondo meditar, dijo la guapa moza: «Me estoy devanando los
sesos para recordar qué persona conozco yo, o debo conocer, que es muy
íntima de esa señora Doña Eufrasia. Fue mi padre, cuando andábamos locos
en busca del empleo, quien me nombró a tal persona, y dijo: no hay
aldaba como esa, si se acordara de nosotros y quisiera servirnos\ldots»

---¿Persona de la intimidad de\ldots? No puede ser otra que el Marqués
de Beramendi.

---Ese\ldots{} ese mismo señor. Yo le conocí en Atienza, cuando todavía
no era Marqués\ldots{} A mi padre encontró un día en la calle, en
Madrid, no sé cuándo, meses ha, y le preguntó por mí. Yo\ldots{} si le
veo, no le conozco, no me acuerdo\ldots{}

---Pues si has pensado que ese señor podría servirte para entrar en
amistad con Eufrasia, no sabes lo que te pescas. No es hoy íntimo de
ella: lo fue\ldots{} Hace tiempo le atacaron unas melancolías que
parecían principio de locura. Su mujer tomó la resolución de sacarle de
Madrid, y a Italia se fueron él y ella con el niño que tienen. Sé todo
esto por los suegros de Beramendi, los señores de Emparán, que a menudo
visitan a Doña Victorina\ldots{} Pues en Italia se estuvieron todo el
año pasado y largos meses de este. No hace mucho que han vuelto, y no sé
que el Marquesito haya pegado otra vez la hebra con la Socobio\ldots{}
Dices que tu padre le encontró y habló con él\ldots{} Fue sin duda antes
del viaje a Italia, si no fue el mes pasado.

---No, no: debió de ser antes del viaje\ldots{} Por lo que mi padre me
dijo, el nombre de ese caballero se relaciona en mi cabeza con el de
Doña Eufrasia, que hoy es Marquesa.

---De Villares de Tajo\ldots{} Si dudas de mí, vete a ver a esa señora.
Puede que se confiese contigo; yo lo dudo mucho\ldots{} pero quién sabe.
Esa lagarta no entrega sus secretos al primero que llega.

---Naturalmente---dijo Lucila, que en aquel instante recobró todo su
candor,---si sabe que Tomín me quiere, y tiene que saberlo, porque él
mismo se lo habrá dicho, me recibirá con una piedra en cada mano.»

Aprovechando aquel estado de inocencia, soltó Domiciana la mentira
final, la que había de ser cúspide y remate del gallardo artificio que
había levantado. No creyó prudente emplear la última pieza de su grande
obra hasta que llegase el oportuno momento. Este llegó. Dijo la señora:
«Para concluir, Lucila, para que te convenzas de que debes dar por
concluso ese negocio, sabrás que la Socobio no ha hecho lo que ha hecho
por adorar al Capitán en sus propios altares, sino que lo ha llevado
como en holocausto, fíjate bien, a otro altar de más altura, donde
oficia el Supremo Sacerdocio de esos dioses gentílicos\ldots{} ¿No lo
entiendes? ¿Quieres que te lo diga más claro?»

---Sí lo entiendo. Mas para que yo crea eso, que parece cuento de
brujas, dime dónde está Tomín, dónde le tienen guardado para esos
holocaustos malditos\ldots{}

---¡Vete a saber\ldots!---rezongó la cerera un tanto
desconcertada.---Guardado lo tendrán como lo tuviste tú.

---Según eso, sigue condenado a muerte.

---Claro. Boba, el indulto vendrá después, cuando ya la devoción gentil
se acabe por cansancio, o por cualquier motivo, y entonces le verás
restituido a su jerarquía, Comandante, pronto Coronel\ldots{} y caminito
de General. Hay casos, Lucila\ldots{} ¿Pero aún dudas?

---Sí, siempre dudo\ldots{} pero no te negaré que lo tengo por posible.
Mi padre, hombre de pueblo, sin instrucción, que piensa muy al derecho y
tiene un talento natural que ya lo quisieran más de cuatro, me ha dicho
muchas veces: «No hay cosa, por desatinada que sea, que no pueda ser
verdad en este país, mayormente si es cosa contra la justicia y contra
la paz de los hombres\ldots{} Aquí puede pasar todo, y la palabra
\emph{increíble} debe ser borrada del libro ese muy grande donde están
todas las palabras, porque en España nada hay que sea mismamente
increíble, nada que sea mismamente\ldots» ¿cómo se dice?

---Absurdo. Tu padre tiene razón. Los españoles, hija\ldots{} de varones
hablo\ldots{} son la peor gente del mundo, y no hay cristiano que los
entienda ni los baraje. Se les da lo bueno, y lo tiran; les hablas con
juicio, y dicen que estás loca. Progreso aquí significa andar para atrás
como los cangrejos, Libertad correr tras de un trapo colorado, Orden
pegar sin ton ni son, y decir Gobierno es como decir: «no hay quien me
tosa.» Mucho ganaría esta Nación si se dejara gobernar por mujeres
listas, que las hay\ldots{} A esos hombrachos que no sirven para nada y
reniegan de que una monja se meta en cosas de Gobierno, les diría yo:
callaos, imbéciles, y no echéis roncas contra la \emph{Madrecita}, pues
no merecéis otra cosa.»

Sumergida Cigüela en profunda abstracción, nada decía. Sentada, el codo
en la rodilla, la frente sostenida en tres dedos de la mano derecha, los
ojos fijos en el halda de su vestido, dejaba caer su pensamiento al
sondaje de profundos abismos. Domiciana, que vio en su enemiga señales
de confusión, de batalla tortuosa entre afectos, todo ello contrario a
la derechura de las resoluciones violentas, acabó de recobrar su aplomo.
Había vencido; con soberano talento, con pases y quiebros de
extraordinaria sutileza, había logrado encadenar a la fiera\ldots{} Ya
podía pasarle sin ningún riesgo la mano por el lomo. «Amiga querida---le
dijo levantándose,---yo no puedo detenerme más. Si quieres venir
conmigo, ven; si quieres quedarte, comerás con mi padre y con Ezequiel.
Te repito que estás en tu casa.»

Lucila, sin mirarla, sin cambiar de su postura más que la mano, que de
la frente bajó a sostener la quijada, le dijo: «Gracias, Domiciana. Yo
me voy también.»

---¿Y dudas aún que soy tu mejor amiga?

---Ya no dudo ni creo---dijo la guapa moza en pie, suspirando:---ya el
dudar y el creer, como el temer y el desear, son para mí la misma
cosa\ldots{} En nadie ni en nada tengo fe\ldots{} Estoy pensando que la
vida y la muerte\ldots{} todo es lo mismo\ldots{} y que en este mundo y
en el otro, hay la misma maldad, porque malo es todo lo que antes era
nada y ahora es\ldots{} lo que es\ldots{} No me entiendo\ldots{} Adiós,
Domiciana\ldots»

Suelta la mantilla, salió; tomando carrera al llegar al pasillo,
precipitose por las escaleras abajo. La cerera vio en aquella salida
fugaz, como ciertos mutis de la escena, una reproducción del arrebato
con que Lucila se había presentado en la alcoba; pero como iba en
retirada, no fue grande su inquietud. Con todo, rodando después de coche
por calles y plazuelas, camino de sus obligaciones, apartar no podía de
su pensamiento los horrendos pesares de la que fue su amiga, ni la
tenacidad con que a ellos se aferraba, rebelde al consuelo. «Me
equivoqué---se decía,---pensando que entre las heridas del alma y su
reparación no ponía el tiempo tanto de sí\ldots{} Cada día aprendemos
algo\ldots{} Me da lástima esta pobre, y me da miedo. Menester será
curarla o amarrarla.»

\hypertarget{xxiv}{%
\chapter{XXIV}\label{xxiv}}

Como animal derrotado y herido, a la fuga se lanzó la hija de Ansúrez,
sin reparar en las frases melosas que a su paso veloz por la tienda se
le dijeron, y en la calle corrió, tropezando con transeúntes y
vendedores, ignorando hacia dónde caminaba, pobre bestia huida.
Creyérase que alejarse quería de sí propia, o que en la rapidez de la
marcha veía como una forma o procedimiento de olvidar\ldots{} Sin darse
cuenta de su itinerario, pasó por Puerta Cerrada, calle del Nuncio, hizo
un breve descanso en el Pretil de Santisteban, bajó a la calle de
Segovia; metiose luego por la calle del Toro a la Plazuela del Alamillo;
tiró hacia la Morería vieja, y en las Vistillas tomó resuello\ldots{}
Apoyada en la jamba de una de las enormes puertas del caserón del
Infantado, echó mano con furia a su propio pescuezo, diciéndose: «Me
ahogaría; lo merezco por tonta, por estúpida y cobarde. Debí matarla,
fue gran burrada compadecerla, y darle tiempo a que con sus despotriques
me enfriara la voluntad de hacer justicia\ldots{} ¡Y se ha reído de
mí\ldots{} se ha quedado riendo, y yo sin cuchillo\ldots! no sé ya cómo
me quitó el cuchillo\ldots{} Pero si fui con la idea de matarla, con
toda la justicia de Dios dentro de mí, ¿por qué no la maté?\ldots{}
¡Perra traidora!\ldots{} ¡Y aún está viva, y gozando de su robo!\ldots{}
¡Sabe Dios a dónde habrá ido en el coche!\ldots{} Merezco su desprecio,
merezco todo lo que me pasa. Me caigo de boba\ldots{} Entré águila y he
salido abubilla\ldots{} Me ha engañado con mil embustes dichos como ella
sabe\ldots{} Abogada como ella y ministrila como ella, no han nacido,
no. Engañará al demonio, a Dios mismo engañará\ldots{} Lucila, eres
digna de que esa ladrona, después de robarte las guindas y de
comérselas, te arroje los huesos al rostro. Aguanta y límpiate, triste
pava\ldots{} escóndete donde nadie te vea.»

En su delirio, tuvo la feliz idea de esconderse en su casa. Aquella
noche, Antolín de Pablo, recién llegado de su excursión por los pueblos,
le confortó el ánimo con hidalgas ofertas de hospitalidad. Si no quería
recibir ya los socorros de la cerera, y gustaba de mantenerse honrada,
allí tenía su casa, allí no le faltaría lecho en que dormir y un
panecillo con que matar el hambre. Donde comen dos comen tres, y alabado
el Señor que a él y a su buena Eulogia les daba medios de mirar por el
prójimo. Este generoso proceder fue gran consuelo para Cigüela, tan
infeliz como hermosa. Por la noche, dormida con pesado sopor, soñó que
se ocupaba en la sabrosa faena de matar a Domiciana. Sobre el cuerpo
yacente de la cerera descargaba golpes y más golpes con el fiero
cuchillo, clavándoselo hasta el mango; pero no conseguía dar fin de
ella, ni aquella vida se dejaba rematar. La víctima recibía sonriente
las puñaladas, cual si su cuerpo fuera un saco relleno de paja o serrín,
y de él no salía sangre\ldots{} ¿Dónde demonios estaba la sangre de
aquella mujer? ¿Habíasela sacado para hacer con ella un elixir de amor,
un bebedizo con que emborrachar a Gracián y filtrar en su ser el olvido
y la degradación?\ldots{} Lucila se cansó de acuchillar a su enemiga, y
el cuerpo de esta coleaba siempre, siempre\ldots{}

Al día siguiente, recobrada del furor homicida, se apoderaron de su
espíritu las historias contadas por Domiciana. Cierto que el odio a esta
no se extinguía; pero las historias tomaban en la mente de la guapa moza
cuerpo y aires de cosa real. Nada de aquello era inverosímil. Bien podía
resultar que fuese verdadero. El efecto buscado por la exclaustrada no
se había hecho esperar, y su ingenioso artificio formaba un estado
anímico ya indestructible. Dentro de sí llevaba Cigüela razones y
aparatos lógicos, hechos bien tramados, que unas veces lucían como
verdades, otras se apagaban en dudas dejando siempre algún destello.
Momentos había en que reconstruidas las famosas historias con elementos
de realidad, las vio Lucila como novela verosímil; horas hubo, en los
días siguientes, en que fueron para ella como el Evangelio.

Si con delicada piedad Eulogia la socorría, no gustaba de que estuviera
ociosa. Mañanas o tardes la tenía lavando ropa en la artesa, y luego
tendiéndola en las cuerdas del corral. En costura y plancha invertían
las dos no poco tiempo, y por la noche, cuando estaba en Madrid Antolín
de Pablo, solían jugar a la brisca o al burro. Entre San Antonio y San
Juan, tuvieron que ir Antolín y su mujer a la boda de una sobrina carnal
de él, en la Villa del Prado, y llevaron consigo a la huéspeda, que se
reparó de sus quebrantos en veinte días de vida campestre. Allí le
salieron amadores sin cuento; de los pueblos vecinos acudían a verla los
mozos y a celebrar su hermosura. Con buen fin le hablaron muchos, y
otros con fines equívocos, que se habrían trocado en buenos, si ella
pusiera de su parte algo de amoroso melindre; pero ninguno de aquellos
requerimientos venció la frialdad y desvíos de la guapa moza, que era
como linda estatua en quien faltaba el fuego de los deseos y el estímulo
de la ambición. Para que ninguno de los inflamados pretendientes se
quejase, Lucila rechazó también, no sin gratitud, los obsequios y finas
proposiciones de un labrador muy rico de aquellas tierras, viudo y
entrado en años, que de ella se prendó con amor incendiario, y en una
misma frase expuso su petición de afecto y su oferta de inmediato
matrimonio. Contestó Lucila negativamente, con razones que al pobre
señor dejaron tan confuso como lastimado.

A poco de este memorable suceso, regresó la joven a Madrid con sus
patronos, y a medio camino, en el alto que hizo la tartana a la entrada
de Navalcarnero, oyó Lucila de boca de Antolín este substancioso sermón:
«Pues, hija, si apuestas a boba no hay quien te gane. ¡Hacerle \emph{fu}
al amigo Halconero, riquísimo por su casa, y más bueno que rico!\ldots{}
No sabes tú lo que te pierdes. ¿Qué pero le pones, alma de cántaro? ¿Que
peina canas y va para Villavieja? Pues no podías soñar proporción más al
auto de tus circunstancias. Cásate, simple, con Vicente Halconero, que
es hombre sano, y ya verás como no tardas en tener familia, con lo que
has de distraerte y apagar todo el rescoldo que te queda de tus
pesadumbres. Y aún tendrás tiempo ¡cuerpo de San Casiano! de ser una
viuda joven, que tu marido, en ley natural no debe vivir mucho. Ea,
tontaina, yo le diré al amigo que aunque le has dicho que no, por el
punto, que se dice, luego soltarás el sí\ldots» Reforzó Eulogia esta
homilía con argumentos aún más especiosos, y ya en Madrid, volviendo a
la carga, se admiraban de que Lucila estuviese tan rebelde, no teniendo
más que el día y la noche. Tanto le dijeron, y memoriales tan llorones
envió desde el pueblo el bendito señor, que al fin la moza, sin abrir
camino a las esperanzas, propuso y suplicó que le dieran para pensarlo
todos los días que restaban hasta fin del año corriente. «Pero,
chica---le dijo Antolín,---considera que el hombre no es niño, y que la
esperanza es un pájaro que no gusta de anidar en las cabezas canas.» No
hubo manera de apear a Lucila de la transacción propuesta; en ello
quedaron, y notificado al buen señor el emplazamiento, se puso tan
alegre, según decían, que le faltó poco para echarse a llorar del gusto.

Al volver a la Villa y Corte, encontró Lucila en ella los ardores del
verano, y mayor soledad y tristeza. Las aliviadas penas se recrudecieron
en el paso del sosiego campestre al bullicio urbano. Agitada fue de
nuevo por furores de venganza, y por el prurito loco de revolver el
mundo en busca de la verdad. Con la verdad se contentaría, ya que el
hombre no pareciese. Por la Capitana, que algún día la visitaba, supo
que la cerera se había ido con Doña Victorina a San Ildefonso, donde
estaba la Corte. La ausencia de su enemiga fue un motivo de sosiego para
Lucila. ¡Qué descanso no verla más ni saber nada de ella! Así cayendo
irían sobre su memoria esas capas de polvo que traen el lento olvidar,
la renovación pausada de las ideas. De este modo se llega, por gradación
suave, a ver y apreciar el reverso de las cosas.

En el curso de aquel verano, el estado de melancolía en que se fueron
resolviendo las amarguras de Cigüela, llevaba su espíritu a las
expansiones religiosas. No había consuelo más eficaz, ni mejor arrullo
para dulcificar y adormecer los dolores del alma. Oía misa en la Orden
Tercera o en San Andrés, y algunas mañanas corríase hasta San Justo,
donde entraba con la confianza de no ver a la cerera. Confesó y comulgó
más de una vez en San Pedro y en San Isidro. Su padre, el veterano
Ansúrez, acompañarla solía en estas devociones elementales, de dulce
encanto para las almas doloridas. Más de una vez se tropezó Lucila con
Rosenda, que diferentes iglesias frecuentaba, y de su mal humor coligió
que no había sido muy dichosa en sus cacerías, sin duda por el
sacrilegio de intentarlas en lugar sagrado. En San Justo, ya muy
avanzado Agosto, se encontró una tarde a Ezequiel, vestido de monago:
palideció el muchacho al verla, y después, en el blanco cera de su
rostro aparecieron rosas\ldots{} «¡Qué guapa estás, Luci!---le
dijo.---Nos contaron que te casabas con un señor muy rico, de ese pueblo
de donde vienen las buenas uvas. ¿Es cierto?» Negó Lucila, y el
cererillo le dio noticias que no la interesaban: que D. Gabino había
tenido un ataque a la vista, quedándose medio ciego; que Domiciana
seguía en La Granja, y que D. Mariano estaba colocado en la Comisaría de
Cruzada, con ocho mil reales. Luego se acercó a ella D. Martín Merino, y
la saludó secamente, recordando haberla visto con la cerera. «¿Es esta
señora la amiga de Doña Domiciana Paredes?\ldots{} Por muchos
años\ldots{} yo bueno\ldots{} ¿y en casa?\ldots{} ¡Qué calor!\ldots»
Esto dijo, retirándose con la fórmula vulgar: «Vaya: conservarse.»
Díjole después Ezequiel que D. Martín era un buen sacerdote que cumplía
muy bien su obligación. Domiciana le prefería con mucho a los demás
confesores que en San Justo había: últimamente, con D. Martín se
confesaba, y él también, por recomendación expresa de su hermana.
Trabajillo le costó acostumbrarse, porque el Sr.~Merino era muy rígido,
no ayudaba, no hacía preguntas, y el penitente tenía que ir
desembuchando pecado tras pecado por orden de mandamientos, pasando
muchas vergüenzas, hasta que no quedara nada en el buche, pues de otro
modo no había absolución. Y ya es uno un poco hombre, Lucila---decía con
inocente orgullo,---y cuesta, cuesta el rebañar bien la conciencia,
sacando a pulso todo, todo, hasta los malos pensamientos, hasta las
tentaciones que son y no son\ldots{} Bueno. Pues hablando de otra cosa,
te diré que mi padre, que ya no ve el pobre, pregunta por ti, y cuando
le decimos que no sabemos nada, se le cae una lágrima\ldots{} Vete a
verle, mujer, que aunque él padezca un poquito por no poder verte el
rostro, se consolará con oírte la voz\ldots»

¡Fecunda creadora es la madre Fatalidad! La idea de que Domiciana tuvo
por confesor a D. Martín arrastró hacia el austero sacerdote toda la
atención de Lucila. Pensaba mucho en él; fue a San Justo movida del afán
de observar su fisonomía; y viendo, no sin cierto terror, al depositario
de aquella negra conciencia, al que había sido como espejo en que el
alma de la traidora se mirara, dio en cavilar si no habría medio de
hacer salir de nuevo a la superficie del cristal las imágenes que en él
se habían reproducido. Pero esto era imposible. No hay confesor que
revele los pecados que se le confían. «Este lo sabe todo---se decía la
moza, oyéndole la misa.---Este conoce la historia infame, y cuando se
vuelve para decirnos \emph{Dominus vobiscum}, paréceme que veo a
Domiciana en sus ojos negros de pájaro de rapiña, penetrantes.» Un día
que D. Martín, bajando del presbiterio, la miró de lejos con fijeza casi
desvergonzada, Lucila, estremeciéndose, dijo esto dentro de su
pensamiento: «Sí, D. Martín: yo soy, yo soy la víctima de aquel crimen,
soy la pobre mujer engañada, robada. Esa ladrona, esa farisea, esa
Judas, me quitó lo que yo amaba más que mi propia vida, mi único bien,
mi único amor, y quitándomelo me ha dejado tan sola como si toda la
humanidad se hubiera concluido\ldots{} ¿Verdad que fue gran felonía, y
una maldad de esas que no tienen perdón? ¿Verdad que era justicia
matarla?\ldots{} ¿Verdad que no debí flaquear cuando llegué a ella con
el cuchillo, y que fuí muy necia en salir dejándola viva?» Y en su
delirio, creyó Cigüela que el clérigo, al retirar de ella su mirada, le
decía: «Sí, mujer: Domiciana merecía la muerte. ¿Y tú, zanguanga, por
qué no la aseguraste bien?»

\hypertarget{xxv}{%
\chapter{XXV}\label{xxv}}

Desde su campestre residencia en la Villa del Prado, escribía D. Vicente
Halconero cartas dulzonas a la que llamaba su prometida, y esta
puntualmente les daba respuesta, poniendo en ella lo menos posible de
ortografía, lo más de sinceridad y una fría expresión de gratitud y
afecto. No quería engañarle con fingidos entusiasmos, y en todas sus
cartas le abría, como si dijéramos, la puerta de aquel compromiso para
que se retirase cuando fuera de su gusto. Pero el buen labriego no
pensaba en abandonar un campo tan florido, y en cada epístola que
enjaretaba se ponía más tierno y dulzacho, como si mojara la pluma en el
arrope de aquella tierra.

Avanzaba Septiembre cuando el viejo Ansúrez manifestó a su hija que ya
le aburría y descorazonaba el empleo en casa del señor Chico, no porque
allí el trabajo le rindiera, ni por el adusto genio del amo, sino porque
se veía mal mirado del pueblo y de toda la vecindad. El aborrecimiento
de la gente de Madrid al cazador de ladrones y perdidos, recaía en los
servidores que de ello no tenían ninguna culpa; a tanto llegaba la
inquina \emph{ciudadana}, que de él, Jerónimo Ansúrez, huían más de
cuatro, y le miraban con miedo y repugnancia como si fuera criado del
verdugo. Quería, pues, \emph{presentar la dimisión de su cargo}, y
habiendo conocido ya la vanidad y \emph{poca pringue} de todos estos
empleíllos, era su anhelo buscarse la vida con independiente trabajo, en
un comercio de cosa que él entendiera. Proporción de establecerse se le
ofrecía, que ni cogida por los cabellos. Se traspasaba la tienda de
granos para simiente y de huevos, calle de las Maldonadas, y él podría
quedarse con aquel tráfico sin más que aprontar cuatro mil reales que
pedían por el traspaso. Cierto que ni él tenía tal suma ni su hija
tampoco; pero bien podían pedirla prestada, y no había de faltar quien
abriera la mano, por la seguridad de un buen interés o la participación
en el negocio. Aunque su padre no lo dijo claramente, Lucila le caló la
intención, la cual no era otra que tratar del préstamo con Antolín de
Pablo. Resueltamente se desentendió la moza de semejante embajada, y por
aquel día no se habló más del asunto. Conviene advertir que Lucila había
cuidado de no poner en autos a su padre de las intenciones y fines del
rico D. Vicente Halconero: temía que el \emph{celtíbero}, de la fuerza
del alegrón, se lanzase a explotar tempranamente la generosidad del
opulento villano.

Continuaba Cigüela parroquiana de San Justo, prefiriendo a las demás
esta iglesia por la singular atracción del clérigo, a quien suponía
viviente archivo de aquella historia lamentable. «Aquí está quien sabe
la verdad---se decía.---Me agrada el sentirme cerca de esta verdad, aun
sabiendo que no ha de querer descubrirse. Siempre que me mira este
maldito cura, feo y antipático, creo que le gustaría quitarse el velo.
Es ilusión, locura mía.» Una mañana la saludó al paso D. Martín: «Yo
bien, gracias\ldots{} Mucho calor\ldots{} ¿Qué se sabe de Doña
Domiciana? ¡Cuánto tiempo que no parece por aquí!\ldots{} ¿Qué dice
usted\ldots{} que ya no son amigas? ¡Vaya por Dios! Las mujeres por cosa
grande riñen, y por cualquier nadería hacen las paces\ldots{} Hoy
furiosas enemigas, mañana comiendo en un mismo plato\ldots{} Ea,
conservarse.»

Otro día que se encontraba en San Justo, allí fue Ansúrez en su
persecución, y viéndola saludada por D. Martín, le dijo: «Hija del alma,
lo que menos sospechas tú es que estamos tan cerca de nuestro remedio.
¿Ves ese sacerdote tan áspero, y de tan mal cariz que a mí se me parece
al verdugo que había en Zaragoza el año 43? ¿Lo ves? Pues es hombre de
posibles, y coloca su dinero a interés, que no digo sea mismamente
módico. Lo sé por quien le debe y no puede pagarle, de lo que resulta
que está el buen cura furioso, y por eso tendrá esa cara de
vinagre\ldots{} Pues óyeme: Al ver que te saludaba con aire de
estimación, pensé y dije que si vas y le pides para tu señor padre, que
quiere poner un comercio, cuatro, o aunque sean seis mil reales, con la
formalidad de pagaré en regla, y réditos consecuentes y puntuales,
cierto es que veo el dinero en tus manos, que es como decir en las
mías\ldots{} Con que atrévete, y verás a tu padre en su tienda de las
Maldonadas\ldots{} ¿Qué? ¿Sientes cortedad?\ldots{} Entendí que te
confiesas con él.»

---No me confieso porque me da miedo\ldots{} No es de los que la llaman
a una por ese aquél de la bondad cristiana\ldots{} Vamos, que no me
gusta para confesor\ldots{} Sabe historias que me tocan muy de cerca;
las sabe por confesión de otras personas; me parece que si con él me
confesara, se me trastornaría el sentido y le diría: `no vengo a
entregar mis pecados, sino a que usted me entregue los de
otros\ldots{}'. Esto es un disparate. Pero yo me conozco\ldots{} y por
eso no me acerco a su confesonario.

---Hija de mis entrañas, no seas simple. Arrímate a la reja, y haz una
confesión neta y clara, que a él le maraville por tu tribulación, por
tus ansias de enmienda y de no volver a pecar. Entre col y col, le dices
que tienes un padre amantísimo que se ve en grandes aflicciones, sin
explicar porque sí ni porque no\ldots{} Te absuelve\ldots{} Quedáis
amigos; eres su hija de confesión\ldots{} te considera, te tiene lástima
por lo que le dijiste de lo atropellado que anda tu buen padre. Dejas
pasar dos días, y luego le pedimos una entrevista en su casa, que es ahí
en el pasadizo de la Plaza Mayor; nos vamos los dos allá, y verás como
no salimos con las manos vacías.»

Protestando de que es gran sacrilegio confesar con la idea de pedir
dinero al confesor, Lucila opuso resistencia a los planes de su padre.
Después dijo que se tomaría tiempo para pensarlo, y que, si se
determinaba, había de ser sin previa confesión\ldots{} Por nada del
mundo mezclaría las cosas sagradas con las mundanas, ni la conciencia
con los intereses.

---Bueno, hija muy adorada, perla de mi familia: te dije lo del
confesonario, porque en todas las cosas nunca está de más abrir
cualesquiera caminos para los fines que buscamos, y eso al alma no daña;
ni el confesar que te propuse era con el fin único de los intereses,
sino para que con la limpieza de tu conciencia prepararas al sacerdote a
estimarte más. Total, que de un tiro matabas dos pájaros; con una sola
acción sacabas dos provechos: tu alma purificada y mi bolsillo
guarnecido. Ya ves\ldots»

Parte de aquella noche pasó Lucila en cavilaciones sobre lo propuesto
por su padre, y de cuanto pensó resultaba el propósito de avistarse con
Merino. ¿Qué perdía en ello? Podría suceder que hablando los dos se
espontaneara el hombre, en una distracción de la conciencia, o que aun
callando, con pausa brusca o con instintivo gesto diese a conocer la
verdad. Quería, pues, aproximarse a la esfinge, y contemplar sus labios
de bronce, por si de ellos alguna revelación al descuido caía\ldots{}
Hablaría con el clérigo, pero sola: la presencia de su padre la
estorbaba. Ante todo, érale preciso prevenir a D. Martín, pidiéndole
hora para la audiencia, y este trámite quedó cumplido a los tres días de
la expresada conversación con Ansúrez. Tal era la impasibilidad del
viejo cura, que no manifestó sorpresa ni disgusto de la visita que se le
anunciaba: sin duda penetró el objeto aparente de ella, que era
solicitud de préstamo. Contestó a Lucila que fuese cualquier mañana, o
cualquier tarde antes de las siete, hora en que infaliblemente cenaba y
se recogía.

Llegaron día y hora: una tarde, cuando se aproximaba el ocaso, fue
Lucila a la Plaza Mayor con su padre; este se quedó dando vueltas
alrededor del caballote de Felipe III, y la moza penetró en el siniestro
pasadizo, que oficialmente se llamaba \emph{Arco de Triunfo}, y por mote
popular \emph{Callejón del Infierno}. Entrando por la única puerta
numerada que allí se veía, subió hasta el segundo piso poco menos que a
tientas, pues ni había luz en la escalera, ni a esta llegaba la claridad
del día declinante. Tiró de un cordón mugriento\ldots{} abrió la puerta
una doméstica joven, fea y sucia\ldots{} y apenas nombró la visitante al
Sr.~D. Martín, vio que este surgía de las sombras de la casa, y le oyó
decir: «Pase, joven, pase. Dominga, traerás luz.»

Tras D. Martín entró Lucila en una estancia chica, con ventana que daba
al callejón. Había en ella un derrengado sofá de paja, una mesa camilla
con cubierta de hule negro y raído, y faldón de bayeta verde; en el
rincón próximo una papelera con libros apilados en la parte superior;
entre la mesa y la pared una silla, enfrente otra. El esterado era de
empleita con rozaduras; en las paredes no había ninguna estampa ni
cuadro; sobre la mesa, al lado izquierdo de D. Martín, papeles
manuscritos sujetos con un pedazo de mármol que debió de ser peana de
una figura, tintero de loza con dos plumas clavadas en los agujeros
laterales, polvorera de cobre y un pedazo de paño negro; el breviario,
arrimado al lado derecho, encima de otro libro de cubierta roja; un
almanaque con las hojas muy sobadas, un bote de hojalata con tabaco,
librillo de papel de fumar. Todo allí revelaba pobreza y avaricia.

A una indicación de Merino se sentó Lucila en la silla del lado exterior
de la mesa, y sentado él entre la mesa y la pared, quedaron frente a
frente. La sotana verdinegra que el clérigo usaba dentro de casa era
prenda antediluviana que le envejecía más. Lucila le vio más feo que en
la iglesia, más sucio, abandonado y desapacible. Abrió el cura la
conversación con estas palabras: «Hoy me ha dicho el chico de la cerería
que su hermana está para llegar.» Lucila no dijo nada: se alegraba de
que D. Martín relacionara siempre la persona de la criminal con la de la
víctima, pues ni una sola vez, al hablar a esta dejaba de nombrar a la
cerera maldita. ¿No podría esperarse que de la tangencia de personas en
el cerebro del cura resultara un abandono del secreto?\ldots{} «Y a
propósito de Doña Domiciana---prosiguió Merino,---voy a enseñarle a
usted los tres regalitos que me hizo antes de irse a La Granja.» De un
cajón de la papelera próxima fue sacando y mencionando los objetos que
mostró a Lucila. «Vea usted: una caja con bolitas de jabón, alumbre y
trementina, para quitar manchas de la ropa negra, y remediar el lustre
que llamamos de ala de mosca\ldots{} Vea usted: un rollo de cerillo fino
para alumbrarse en la escalera cuando uno entra de noche\ldots{} Y por
último, este cuchillo\ldots» Lo desenvainó para mostrarlo a Lucila, que
en todo su cuerpo sintió repentina frialdad al reconocerlo. «Es
precioso---dijo D. Martín, satisfecho de poseer aquella joya.---Vea
usted qué punta más afilada\ldots{} Es fino de Albacete, con grabados
árabes en las costeras; el mango muy bonito\ldots{} Era una lástima que
esta magnífica hoja no tuviese su vaina correspondiente. En busca de
ella me fui al Rastro algunas tardes, y al fin, mirando en este puesto y
en el otro, me encontré esta que le viene tan bien como si con ella
hubiera nacido\ldots{} Y no me costó más que dos reales\ldots{} Vea
usted\ldots{} Lo he limpiado\ldots{} Siempre es bueno tener uno alguna
defensa, por lo que pudiera ocurrir.»

La idea que a Lucila embargaba le sugirió con celeridad eléctrica esta
pregunta: «D. Martín, ¿no le dijo Domiciana de dónde sacó este puñal, o
cómo fue a sus manos?»

---Es un arma muy buena, la hoja de temple fino, el mango muy bien
labrado---dijo el clérigo guardando el cuchillo y sin parar mientes en
la pregunta de la joven. Esta la repitió con más énfasis.

---Si me dijo algo, ya no me acuerdo---contestó Merino con indiferencia
real o fingida.---Se lo encontró probablemente\ldots{}

---Pero usted sabe que Domiciana es muy mala\ldots{} Ese cuchillo, lo
mismo pudo ser suyo para matar, que de alguien que quiso matarla.»

En aquel momento entró la doméstica con un candil que apestaba.
Iluminando de frente el rostro amarillo y huesudo del presbítero, sus
ojuelos brillaron, fijos en la moza, y con su más bronco acento le dijo:
«Señora, ¿en qué puedo servirla?» Desconcertada por esta invitación a
seguir la derecha vía, el pensamiento y la palabra de la guapa moza se
lanzaron por un despeñadero. «Pues verá usted, D. Martín: como Domiciana
es tan mala, yo\ldots{} digo, mi padre\ldots{} Es que quiere
establecerse, tomar una tienda de granos y huevos\ldots{} y\ldots{} el
cuento es que no tiene posibles\ldots{} Si no fuera Domiciana lo que es,
una mujer infame y traidora, yo estaría en buenas relaciones con
ella\ldots{} y siendo amigas ella y yo, no había por qué molestarle a
usted\ldots»

---Acaba, hija, acaba---dijo Merino impaciente, tuteándola, con lo cual
expresaba lo que la linda joven había desmerecido a sus ojos en el
momento de declararse necesitada de dinero.---¿Y cómo se te ha ocurrido
venir a mí para esa necesidad? ¡Anda! creen que tengo yo el oro y el
moro\ldots{} No, hija: si en algún día dispuse de fondos, entre
tramposos y estafadores me han limpiado, cree que me han dejado como una
patena. ¿Qué dinero ha de tener nadie en un país donde no hay justicia,
donde no se castiga a los bribones, donde los más altos dan el ejemplo
de la inmoralidad y el ladronicio?

---¡Oh! sí, D. Martín, los más altos son los peores---dijo Lucila con
arranque.---¡Quién había de creer que Domiciana\ldots{} que todavía no
ha dejado de ser esposa de Cristo\ldots{} porque esos votos no los rompe
nadie, ¿verdad?\ldots{} quién había de creerla capaz de una tan villana
acción!\ldots{} No se queje usted de que le hayan robado algún dinero,
porque eso, el vil metal, ¿qué supone?\ldots{}

---¡Que no supone!---exclamó el clérigo con extraordinario brillo en su
mirada.---Los ahorros de toda mi vida, los cinco mil duros que a la
Lotería gané, lo que me daba la Capellanía de San Sebastián, todo me lo
han ido quitando con engaños y malos procederes.

---Quiero decir que esas pérdidas, aunque sean muy grandes, no se pueden
comparar con otras\ldots{} con que le quiten a una el corazón\ldots{} el
corazón y el alma, Sr.~D. Martín\ldots{} Por eso dije que el dinero no
supone nada\ldots{} El dinero no es más que una basura. Todo el que hay
en el mundo, si fuera mío, lo daría yo por que me devolvieran lo que me
ha quitado Domiciana\ldots{} ¿Y a quién reclamo yo? ¿Quién me hará
justicia?

---La justicia está en manos de los fuertes, y los fuertes no la usan
más que en provecho propio, y en vituperio y perjuicio del humilde, del
pobre, del limpio de corazón. Pero los fuertes caerán algún día\ldots{}
vaya si caerán\ldots{} No hay ídolo de barro que resista a un buen
empujón\ldots{} Muchos que nos espantan por poderosos, nos harían reír
si de un golpe los tiráramos al suelo y viéramos que son armadura de
caña forrada de papeles; y más nos reiríamos si al hacerlos rodar de una
patada, viéramos que ya por dentro, por dentro\ldots{} se los van
comiendo los ratones\ldots{} ¿Usted me entiende?»

Decía esto el maldito viejo iluminando con la luz siniestra de sus ojos
el rostro impasible, amarillo, de una rigidez estatuaria de talla vieja
despintada y cuarteada. Lucila le miró, observando el marcado resalte de
los pómulos que a la luz brillaban, redondos, con un deslucido barniz de
santo viejo; observó también las dos grandes arrugas que descendían de
la nariz chata hasta unirse con las comisuras de los delgados labios, y
la extensa curva que estos formaban cayendo por sus extremidades\ldots{}
No entendía bien Lucila el lenguaje gráfico de aquel rostro, en el cual
algo había de momia con vida, y lo que más claramente pudo descifrar en
él, a fuerza de deletrearlo, era un inmenso desdén de todo el Universo.

\hypertarget{xxvi}{%
\chapter{XXVI}\label{xxvi}}

Y no fue poca sorpresa de Lucila el oírle pasar, casi sin transición, de
las lúgubres consideraciones antedichas a esta vulgar pregunta: «¿Y ese
hombre, ese padre de usted, qué cantidad necesita?» Respondió la moza
que de cuatro a seis mil reales\ldots{} y añadió que el negocio de la
tienda de huevos y semillas era de seguro rendimiento\ldots{} «Es mucho,
mucho dinero---murmuró Merino sacando el labio inferior y arqueando más
la boca.---¡Seis mil realazos!\ldots{} digo, digo\ldots{} eche usted
reales\ldots{} y en estos tiempos en que el dinero anda escondido para
que no lo cojan las uñas moderadas\ldots{} El poquito que ha escapado de
esas uñas, tiénelo la soldadesca. Entre abogados y militronches, están
dejando en los huesos a esta Nación\ldots{} Pues no puedo, no puedo
servirles\ldots»

---Mi padre cumplirá bien; por eso no lo haga---dijo Lucila creyendo que
no aflojaba la mosca sin hacerse de rogar.---Y dispénseme que no
empezara por hablarle de mi padre; pero desde que Domiciana me hizo
aquella trastada, he perdido el tino, y hablo todo al revés. A usted le
consta lo mala que es la cerera, ¿verdad, D. Martín? ¿Quién lo sabe como
usted?

---Me hablabas de tu padre\ldots{}

---Decía que mi padre es hombre formal. Mi padre cumple.

---¿Y por qué no ha venido contigo, o no viene él solo? Si yo le hago el
préstamo, él será quien me garantice, no tú; a él podré confiarle mi
dinero, no a ti, que lo gastarás alegremente con tenientes o
capitanes\ldots{} ¡Ah! vosotras las enamoradas trastornáis a los hombres
y les apartáis de su obligación; por vosotras, por vuestros perifollos,
y el lujo\ldots{} \emph{asiático} que gastáis, están ahora los hombres
públicos tan corrompidos; vosotras tenéis la culpa de todo este
ladronicio\ldots{} y luego os quejáis cuando os quitan algo.

---Yo no he trastornado a nadie; yo no he gastado lujo; yo no quiero más
que paz, y el amor de un hombre\ldots{}

---No sueñes con amor de hombre, ni con paz, ni con ningún bien,
mientras no haya justicia y se dé a cada cual lo suyo\ldots{} Espérate a
que el mundo se arregle como es debido, y a que caigan todas las farsas
y rueden los ídolos\ldots{} Mientras eso no llegue, ¿qué hablas ahí de
amor de hombre, si ahora, según estamos, nada es de nadie, y no se sabe
a quién pertenece el hombre, ni la mujer tampoco? Donde no hay justicia,
donde todo es iniquidad, ¿qué sacas de lamentarte? Escribes tus
chillidos en el viento para que jueguen con ellos los pájaros\ldots{}
Todo es aquí tiranía, todo es dominio de los malos sobre los buenos,
opresión del pobre por el rico, y del débil por el fuerte\ldots{} ¿Dónde
está el tuyo y el mío y el de cada cual? Los mandones le quitan a uno la
camisa, y encima hay que darles las gracias porque no nos han quitado
los calzones. Deja tú que todo se estremezca, y el día del
derrumbamiento recobrarás lo tuyo, yo lo que me pertenece\ldots{} eso
es.» Sin transición, saltó con esto: «Vaya, joven, ¿no te parece que
hemos hablado bastante? Dile a tu padre que venga cualquier día\ldots{}
esta es buena hora: hablaremos, y\ldots{} ya se verá\ldots»

Lucila, que ya sentía un si es no es de temor, viendo el acento
rencoroso que ponía en sus divagaciones, se despidió con las fórmulas
corrientes, sin meterse en más dimes y diretes con la esfinge. Salió
Dominga con el candilón, pues la escalera era como boca de lobo, y al
llegar al pasadizo del Infierno, Cigüela se dijo, resumiendo la visita:
«Bien se ve que conoce toda la historia y los enredos de
Domiciana\ldots{} Puede que también sepa dónde y cómo le tienen
escondido\ldots{} Pero no lo dirá\ldots{} Estas cosas de amores y de
hombres robados le interesan poco, nada\ldots{} y las mira como cosa de
juego\ldots{} No piensa más que en el aquel de lo malo que está todo, y
en el latrocinio del Gobierno, y en que moderados y militares no son más
que sanguijuelas que le chupan a España toda la sangre\ldots»

Reunida con Ansúrez en la Plaza, le refirió la visita y las impresiones
que en ella recibiera, que no eran malas en lo tocante al préstamo.
Pensaba Cigüela que el clerizonte soltaría los cuartos; mas era preciso
regatearle, que el hombre, en su tacañería y desconfianza, se fingía
escaso de recursos para obtener mayor ventaja en el negocio. El viejo
\emph{celtíbero} acompañó a Lucila hasta su casa, y al retirarse se las
prometía muy felices. Pero en los días siguientes, resultó que de tan
buenas esperanzas había que quitar la mitad de la mitad. Para efectuar
el préstamo había que esperar a que pagara un cliente moroso, que ya
tenía fuera de tino al Sr.~D. Martín, pues ni devolvía el
\emph{principal}, ni aflojaba los réditos de un año vencido. Todo se
volvía prometer y dar largas, sin que le valieran al clérigo amenazas de
demanda judicial. El deudor se reía. Por fin, hubo esperanzas fundadas
de arreglo, pagando por el tal una señora, tía suya, y rebajando
intereses. Si en efecto se cobraba, se realizaría el nuevo préstamo,
obligándose Ansúrez a responder con todos sus bienes, y a más con la
tienda que habían de traspasarle.

Entretanto que estas cosas del orden económico iban pasando, observaba
Lucila que el grande afán suyo inextinguible por la pérdida de Tomín,
ocupaba y desocupaba las regiones más grandes de su alma con cierto
flujo y reflujo, como el del Océano que llena y vacía con el lento ritmo
de las mareas. Tan pronto la pena honda se aliviaba, y la dolorida mujer
entreveía reparación probable; tan pronto la pena tomaba mayor fuerza,
colmando el alma hasta rebosar; y cuando subía de este modo la hinchada
marea de su aflicción, Lucila deseaba la muerte, y aun acariciaba la
idea de procurársela por su propia mano. Sólo la muerte era verdadero y
eficaz descanso. Sólo el sueño eterno le daría paz, ya que no le diera
el ver a Tomín en la región de allá, donde dormidos vivimos de
nuevo\ldots{} Lo más extraño era que este recrudecimiento del dolor
recaía sobre la hija de Ansúrez cuando la Providencia enviaba sobre ella
sus bendiciones.

Los obsequios cada día más valiosos de D. Vicente Halconero no llevaban
ciertamente a Lucila por el camino del alivio. Eulogia no se cansaba de
amonestarla con severidad o con burlas. «No sé qué más podrías desear.
Viene Dios a verte y le pones cara de alcuza. ¡Vaya un orgullo! Te
llueven tortas y torreznos, y en vez de ponerte a bailar,
lloriqueas\ldots{} Yo pienso en la vida que te esperaba con ese maldito
Capitán si no te lo quitan de en medio\ldots{} Y aun con indulto y todo,
valiente pelo habrías echado de militara\ldots» Sin negar que Eulogia
tuviera razón, Cigüela también la tenía, que razones hay siempre para
todo\ldots{} Claro que no desconocía la inmensa gratitud que al
Sr.~Halconero debía, y se declaraba indigna de tanta bondad. ¡Vaya con
el rico albillo que mandó D. Vicente en aquel Agosto y en aquel
Septiembre, escogidos por él los racimos más hermosos, los más dorados,
de uvas transparentes, finas, dulces! Al albillo acompañaba el arrope
superior, hecho en casa del rico hacendado con todo esmero, y pollos y
capones que en aquellos amplios corrales se criaban. Para el próximo
Noviembre anunciábase ya el esquilmo suculento de la matanza, y para
Diciembre irían los corderos lechales, amén de la muchedumbre de caza, y
castañas y nueces.

Pero estas ricas ofrendas valían menos que otras del magnánimo señor.
Había ordenado a Eulogia y Antolín que por cuenta de él fuera provista
la guapa moza de todo lo correspondiente a una señorita de clase
acomodada; que se encargasen a una buena costurera vestidos honestos y
al gusto de Lucila, agregando cuanto de ropa interior decente necesitase
para completar su atavío, todo esto sin lujo, mirando sólo a la buena
calidad y finura de las telas, y al esmero de las hechuras. Un día se
encontró la joven en su cuarto un tocador de caoba modestito y elegante,
con todos los accesorios de porcelana y adminículos para su arreglo y
limpieza, y a la semana siguiente apareció como por magia un corpulento
armario para ropa con un gran espejo en la puerta, mueble precioso que
fue el pasmo de toda la vecindad. Dígase que todo esto agradó mucho a
Lucila, y elevó hasta lo increíble su gratitud.

Pero aún faltaba lo más hermoso de la generosidad del D. Vicente, la
cual ya tocaba en los linderos de lo sublime, y fue que dispuso en carta
muy extensa lo que se copia para mejor conocimiento: «Quiero que sin
dilación se le ponga un maestro pendolista, que le enseñe el trazo de
buena letra, y todo lo tocante a la ortografía y al uso de puntos y
comas como es debido. No sea ese maestro un mequetrefe, sino hombre que
sepa el oficio, maduro, y de bien probada honestidad, y la letra que le
enseñe sea por Torío, no por Iturzaeta, y nada de esto que llaman
bastardilla y rasgos a la inglesa. Póngasele también preceptor que le
enseñe la Geografía, y la Aritmética hasta la regla de tres no más; y de
Gramática nada, que eso es estudio baldío. Désele de añadidura algún
conocimiento de Historia Sagrada y profana, pero no mucho, nada más lo
preciso; y el Catecismo, por de contado, con las obligaciones del buen
cristiano. Escójanse pasantes graves y circunspectos, sin reparar el
coste\ldots{} y que no sean del estado eclesiástico. De otra clase de
enseñanza, tal como baile y música, nada; que todo este recreo de
mozuelas se deja fuera de la puerta del santo matrimonio.»

De cuanto regalaba y disponía el buen Halconero, estas órdenes,
reveladoras de un interés profundo y de un cariño intenso, fueron las
que más hondamente penetraron en el alma de Lucila. Eulogia le dijo:
«Dios viene a ti; Dios ha hecho de la más desamparada la más amparada; y
a la más pobre la rodea de bienes, y a la más triste le pone corona de
felicidades. Dale las gracias, y dile: Señor, hágase tu voluntad\ldots»

Voluntad de Dios era sin duda, y manifiesta con tales signos, no había
medio de rebelarse contra ella. En la red de estos beneficios tan
hermosos como delicados, se veía cogida, sin evasión posible. Ya su
compromiso no podía ser condicional, ni estar sujeto a definitiva
resolución en un marcado plazo, ya debía darse por prometida y aún más
por otorgada\ldots{} En todo Enero del año siguiente, según se decretó
en la Villa del Prado, sería Lucila la señora de Halconero.

El cual anunció su viaje a Madrid para Noviembre. Dos veces había estado
en Madrid durante el verano, y Lucila le miraba como uno de tantos
pretendientes, del cual más distante estaba cuando más cerca le tenía.
Pero cuando vino Halconero en Noviembre, ya era otra cosa. Aplicando al
caso toda su buena voluntad, vio en el que ya era su presunto marido
menos fealdad y desagrado que en otras ocasiones viera; vio en
extraordinaria magnitud su bondad, reflejada no sólo en sus nobles actos
y dichos oportunos, sino hasta en su figura\ldots{} Esta no le pareció a
Lucila tan rechoncha y maciza como cuando en el pueblo se ofreció por
primera vez a su atención, y los cuajados ojos de D. Vicente, redondos,
claros y casi siempre húmedos, revelando parentesco con ojos de peces
sacados de las aguas, ya tenían cierto brillo y aun vislumbres de
gracia, efecto sin duda del amor, que en el alma escondida tras ellos
había hecho su nido. En suma, que aunque el noble espíritu de D. Vicente
se hallaba prisionero dentro de una gordura que iba en camino de la
obesidad, Lucila no le encontraba absolutamente despojado de gallardía.
Cierto que era una gallardía muy relativa, y casi casi puramente
convencional. Halconero tenía la cabeza blanca, el rostro encendido,
redondo, afeitado, la dentadura sana, los labios sensuales, la nariz
aguileña, la frente despejada, y el ánimo, en fin, pacífico, amoroso,
propenso a los arrebatos de ternura, así como el entendimiento claro,
aunque tirando a lo imaginativo. Lucila vio en él un marido del tipo
paternal, y creyó firmemente que reinaría en su corazón por la bondad y
el tutelar cariño.

\hypertarget{xxvii}{%
\chapter{XXVII}\label{xxvii}}

Halconero había venido a la Corte de paso para tierras de Guadalajara,
en donde pensaba arrendar pastos para la trashumación de sus merinas.
Detúvose en Madrid sólo cuatro días, con ánimo de permanecer más tiempo
a la vuelta, y por estar más cerca de su presunta felicidad se aposentó
en la posada de \emph{San Pedro}, en la Cava Baja. Bien aprovechadas
fueron las cuatro noches: en ninguna de ellas dejó de llevar al teatro a
Eulogia y Lucila, armonizando el gusto de ellas con el suyo, pues los
lances de la escena le divertían e impresionaban grandemente. Vieron y
gozaron en \emph{El Drama} (Basilios) \emph{La Escuela de los maridos};
en Variedades, \emph{García del Castañar}, y en \emph{El Circo}, la
preciosísima zarzuela \emph{Jugar con fuego}. Aunque por no
contrariarle, Cigüela no decía nada, le causaba cierta inquietud el
frecuentar sitios públicos, temerosa de encontrar en ellos personas que
con sus dichos o sólo con su presencia la trastornasen. Ya por aquellos
días estaba la joven muy metida en el tráfago de sus estudios, los
cuales, por el múltiple beneficio que le causaban, eran entretenimiento
saludable y bálsamo instructivo.

Partió D. Vicente para sus diligencias de ganadero y labrador, y quedó
Lucila compartiendo su tiempo entre las lecciones y el corte y hechuras
de su nueva provisión de ropa. Con Eulogia iba alguna vez de tiendas;
acompañábala también Ansúrez, que, harto ya de verse mal señalado por
servir al impopular Chico, se había despedido, y no tenía más ocupación
que vagar por calles, visitando amigos, o arrimándose a los corrillos de
este y el otro mentidero. Atendido por Lucila en su primera necesidad,
que era el comer, no se apuraba gran cosa por la cesantía. Sabedor ya de
que le tendría por suegro el rico labrador de la Villa del Prado, casi
bailaba de contento por la feliz y casi milagrosa colocación de su
querida hija; pero a él no le petaba el vivir a lo parásito, yedra
pegada al tronco de un yerno; gustaba de la independencia, y no había de
parar hasta establecerse. A ello le animaba el buen cariz de sus
negociaciones con Merino, para el consabido préstamo. Si en la quincena
que siguió a la visita de Cigüela, el adusto clérigo le había mareado y
aburrido con largas y promesas, que hoy, que mañana, ya parecía que iban
las cosas por mejor camino. No se descuidaba el buen \emph{celtíbero} en
tener siempre bajo la mano al sacerdote prestamista; y si no le divertía
visitarle en su triste y lóbrega casa, gustaba de acompañarle algunas
tardes en su paseo, que era infaliblemente por la Cuesta de la Vega,
saliendo alguna vez por el Portillo, y metiéndose en el polvoroso
plantío que llaman \emph{La Tela}. Hablaban del mal Gobierno y de lo
perdido que está el país. «Es Don Martín tan filosófico---decía
Jerónimo,---que se queda uno con la boca abierta oyéndole. Gran meollo
tiene todo lo que dice\ldots{} sólo que cuando uno está en el punto de
cogerle la idea, el hombre se arranca por latines, y\ldots{} a obscuras
me quedo.»

En un comercio de telas de la Concepción Jerónima se encontraron una
mañana Lucila y Rosenda, esta trajeada tan a la moda, que sólo con ello
declaraba el reciente hallazgo de su remedio. A las preguntas de Lucila
contestó que en efecto tenía el mejor arrimo que ambicionar pudiera, en
circunstancias y condiciones inmejorables. «¿Quién\ldots?»---preguntó
Lucila. «No puedo decirlo---replicó la Capitana.---Hice juramento de no
revelarlo a nadie, ni a las personas más íntimas. Y antes reventará que
faltar a lo jurado, porque en ello me va \emph{el ajuste}, que es
superior. Valiente necia sería yo, si por boquear más de la cuenta
perdiera esta ganga.» Celebrando Lucila lo que su amiga le contaba,
limitó su indiscreción a preguntar si la pesquería había sido en la
iglesia, conforme a los planes de marras\ldots{} «En la novena del
Rosario---contestó Rosenda,---eché mis primeros anzuelos\ldots{} Picó en
la novena de Santa Teresa, y saqué el pez en las mismísimas
\emph{Ánimas}\ldots{} y no me pregunte usted más.» Hablaron
inmediatamente de trapos para la estación, y de las nuevas evoluciones
de la moda. «Esa tela \emph{marrón} con rayas le irá muy bien para traje
de señora rica de pueblo. Hágaselo usted con faldetas, el cuerpo muy
abierto por delante, con camisolín bordado, alto, honestito. Aquí
encontrará usted un organdí precioso, o si no, \emph{barege}. La
manteleta es de rigor.» Enterada ya Rosenda del proyectado casamiento de
su amiga con un ricacho viejo, siempre que la veía se extremaba en
felicitarla. Dios había trocado todas sus desgracias en beneficios, su
pobreza en abundancia, y su esclavitud en la más preciosa de las
libertades.»

Dos días después de esta entrevista, volvieron a verse en el mismo
comercio, no ciertamente de un modo casual, sino porque Rosenda,
advertida de los tenderos que esperaban a Lucila para cambiar un retal
por otro, allí la cogió descuidada, sorprendiéndola con este jicarazo:
«Despache usted a su padre con cualquier pretexto, para que podamos
irnos solitas a dar una vuelta por la calle. Tengo que decirle cosas de
remuchísima enjundia.» Tembló Cigüela como el pájaro herido; y atontada
despidió al viejo y aceleró sus quehaceres en la tienda. En la calle las
dos, Rosenda le dijo: «No se encampane usted con lo que voy a
notificarle, ni pierda su serenidad. Prométame por cien mil coros de
serafines que ha de ser juiciosa. ¿Lo promete?\ldots{} Pues allá va. Una
persona, que no necesito nombrar, ha visto a Bartolomé Gracián.»

La impresión de Lucila fue de intenso frío. Dando diente con diente,
pudo balbucir estas cortadas expresiones: «No me engañe\ldots{} ¿Está
segura? ¿Y esa persona le conoce bien?\ldots{} ¿Sería él de
verdad?\ldots{} ¡Oh! siento una pena horrible\ldots{} una alegría
loca\ldots{} ¿Con que vive? ¿No le han matado?\ldots{} Pero no es
alegría lo que siento; es pena, y pienso que ha de matarme.»

---No dudes que es él\ldots{} La persona que le ha visto le conoce como
nos conocemos tú y yo---dijo la Capitana, que, para inspirar mayor
confianza y explicarse con desahogo, inició el tratamiento de tú,
necesario ya entre dos amigas.---¿Pero qué\ldots{} te pones mala? No,
borrica: tómalo con calma, y que este notición no te saque de tus
casillas\ldots{}

---Rosenda, no me mandes que tenga calma---dijo Lucila aceptando el
tratamiento familiar sin darse cuenta de ello.---Me has removido toda el
alma, sacando lo que ya estaba debajo de todo, y parecía que se iba
ahogando\ldots{} ¿Le ha visto ese señor?\ldots{} ¿dónde\ldots{} dónde?

---Serénate. Si te pones muy nerviosa y empiezas a soltar chispas, me
callo.

---No, no: háblame\ldots{} di\ldots{} Ya me veo corriendo por un
precipicio, y aunque quiera volver atrás no puedo. Puede más la
pendiente que yo. ¿Dónde?\ldots{}

---Por hoy punto en boca\ldots{} Tu padre no puede tardar con los
paquetes de horquillas y el tarro de pomada. Además, como te excitas
tanto, estamos llamando la atención en medio de la calle. Arrimémonos a
esta rinconada\ldots{} Sólo puedo decirte hoy que el pobre Gracián no
debe de andar bien de salud. Parece que está enfermo, aburrido\ldots{}

---¡Ay, qué dolor! ¿Y se sabe\ldots{} esto sí podrás decírmelo\ldots{}
se sabe si sigue debajo del poder de la \emph{boticaria}?

---Eso no lo sé hoy, pero es seguro que lo sabré esta noche. Oye lo que
te digo. Vete mañana a mi casa. Vivo calle del Factor, número 6, piso
segundo. Apúntalo bien en tu memoria. Toda la mañana estoy
solita\ldots{} ¿No sabes dónde está mi calle? ¿Sabes la parroquia de San
Nicolás?\ldots{} Pues por allí. No tiene pérdida. Vas mañana\ldots{} me
encuentras sola, y hablamos\ldots{} Verás qué casa tan linda tengo, y
qué mueblaje\ldots{} todo nuevecito, acabado de comprar\ldots{} Y ahora,
chitón, que aquí viene ya papá Jerónimo. Te espero. Con él irás, y allí
nos le sacudiremos mandándole a casa de mi modista, que vive donde
Cristo dio las tres voces\ldots»

Nada más hablaron. Lucila volvió a su casa sin saber por dónde iba, ni
enterarse de lo que por el camino le contaba el buen Ansúrez, cosas
políticas de interés, que la inatención de la guapa moza convirtió en
insignificantes. Todo el alivio ganado perdíase súbitamente, y la honda
enfermedad del ánimo, sentimientos despedazados, dignidad ofendida,
ideas fuera de quicio, razón deshecha en locura, recobraba de golpe su
aterrador imperio. Por la noche, el insomnio renovó en ella los
suplicios de los días más tristes de su existencia, y el sueño la sumió
en las tenebrosas cavidades de la idea trágica. Cuchillo en mano, daba
muerte a la \emph{boticaria} una y cien veces, sin acabar nunca de
matarla\ldots{} Por la mañana, fatigada del insomnio y del sueño, que
tan vivamente reproducían su amor como sus odios, trató Lucila de
confortar su alma ideando alguna contingencia placentera, que bien podía
resurgir en los acontecimientos que se avecinaban. «Si encuentro a
Tomín---se dijo,---y me propone que huyamos sin pérdida de un instante,
me iré \emph{con lo puesto}\ldots{} a donde él quiera. Si fuese menester
que volviéramos al mechinal indecente de la calle de Rodas, iría sin
vacilar, apechugando con toda la miseria que Dios quisiera
mandarnos\ldots{} y si hubiéramos de ir lejos, a un monte cerrado, a una
cueva separada de todo el mundo, también iría con él\ldots{} como si me
llevara a un desierto, de esos en que hay tigres y leones\ldots{} No me
importa que haya leones y panteras, con tal que no haya Domicianas.»

Arregló las cosas y dispuso sus diligencias de aquel día en forma que su
salida y tardanza no inquietaran a Eulogia, y a hora conveniente, salió
con su padre en dirección de la parroquia de San Nicolás, en cuyas
cercanías vivía la endiablada Rosenda. Ávida de llegar pronto, aceleró
su marcha, y como Ansúrez, sofocado, la incitase a moderar la andadura,
díjole que urgía el arreglo de cierto vestido en el término de la
mañana, y que se preparara a llevar recados a puntos distantes\ldots{}
Entre los innúmeros desatinos, engendro de su loca pasión, que pasaban
vertiginosos por la mente de Lucila, prevalecía el que formuló de este
modo: «¡Estaría bueno que ahora se me presentara Tomín en casa de
Rosenda; que Rosenda le hubiera encontrado y allí le tuviera escondidito
para darme la gran sorpresa! Ello no será; pero bien podría ser\ldots{}
cosas más raras se han visto.»

Entró en la casa con sobresalto semejante al de las personas muy
nerviosas cuando saben que sonarán tiros, y por segundos esperan la
detonación y fogonazo. Apenas se fijó en la limpia vivienda de su amiga,
mujer arreglada y de gusto, que había tenido el arte de dar aspecto
risueño a una casa viejísima. Los muebles eran flamantes, de clase
barata con apariencia; las esteras de lo más fino, y la alfombra de la
sala y gabinete, del tipo industrial, a la moda, colores vivos que
durarían muy poco. Preparado había Rosenda la copa de aljófar con cisco
bien pasado, y a ella se arrimó Lucila para calentar sus manos ateridas,
con mitones. Aunque ya usaba manguito, no podía acostumbrarse a llevar
las manos metidas siempre en él\ldots{} Le costaba entrar por los
hábitos del señorío. Despachado Ansúrez a los recados distantes,
quedaron solas. Ponderaba Rosenda su casa y sus muebles, y aun quiso
llevar a su amiga a que viera la cocina, despensa y otras piezas. Pero
la guapa moza, impaciente y con su imaginación en esferas muy distantes,
lo dio todo por visto y admirado, diciéndole: «Luego lo veré. Ya
supondrás que vengo muerta de curiosidad, que he pasado una noche
terrible, que no viviré hasta saber\ldots»

---Pues aquí tienes a tu amiga---dijo Rosenda sentándose a su
lado,---con ganas de traerte al buen entender, y de apartarte de los
malos caminos. ¡Ay, hija! ayer tarde, cuando vine a casa, me pesaba,
créelo, haberte dicho lo que te dije\ldots{} Mejor habría sido
reservarlo para después, y echar por delante el consejo que ahora te doy
tocante al orden de las cosas. Por cien mil coros de arcángeles te pido
que te fijes, que me hagas caso, y te percates bien\ldots{} Allá
voy\ldots{} Lo primero que tienes que hacer es acelerar tu casamiento
por los medios que puedas\ldots{} Todo el tiempo que ganes en rematar la
suerte con Halconero, es tiempo ganado en tu bienestar y en tu
independencia\ldots{} Y ahora viene la segunda parte: en cuanto te
cases, y tengas a ese magnífico buey bien cuadrado, empiezas con él una
brega superior, muleta por aquí, muleta por allá, para que el hombre
abandone la vida del campo y venga a establecerse contigo en
Madrid\ldots{} Bien sé que por de pronto ha de cerdear. Es un viejo
gañán, que no podrá vivir lejos de los montones de estiércol\ldots{}
pero una mujer\ldots{} es una mujer\ldots{} y en luna de miel lo puede
todo\ldots{} Te aburre el campo, te entristece; las aguas gordas de
aquella tierra te revuelven los humores\ldots{} te pones malísima,
pierdes la salud, y hasta podría ser que se te malograra el
fruto\ldots{} Figúrate cuántas razones puedes emplear para convencer a
tu marido, cuántos mimos echarle y cuántas banderillas ponerle\ldots»

Absolutamente contrarias a estas ideas eran las de Lucila. Le gustaba el
campo, y en su soledad y augusto sosiego, esclavizando la atención con
amenos quehaceres, pensaba llevar su alma mansamente a un bienestar
tranquilo. Pero como Rosenda no quería satisfacer su curiosidad, si
antes no prometía someterse y adaptarse a las sabias reglas de la
filosofía del vivir, la guapa moza, como el sediento que entrega toda su
voluntad por un vaso de agua, le dijo: «Haré todo lo que me aconsejas,
Rosenda\ldots{} Y ahora, sepa yo pronto: ¿Han vuelto a verle? ¿Dónde le
han visto?\ldots{} ¿Qué ha pasado, qué más pasará?»

\hypertarget{xxviii}{%
\chapter{XXVIII}\label{xxviii}}

---Pues empiezo---dijo Rosenda poniéndose todo lo grave que podía,---por
darte una noticia que no sé si será buena o mala para ti\ldots{} El
amigo Bartolomé está en poder de la Socobio. Domiciana, que ha sufrido
varias derrotas, saliendo como Doña Victorina con las manos en la
cabeza, se ha quedado compuesta y sin novio\ldots{} No pudo dar al galán
lo prometido, que era el indulto, la rehabilitación y un ascenso, dos
con pase a Cuba\ldots{}

---¿Pero dónde está\ldots{} dónde? Quiero verle y que me vea.

---No pienses en eso\ldots{} Yo miro por ti más de lo que tú crees. Te
contaré una escena, mejor dicho una conversación que ayer hubo en
Palacio. La sé como si la hubiera oído yo misma. Eufrasia, que ahora no
se separa de la Reina\ldots{} ya sabes que Su Majestad ha entrado en
meses mayores: se espera su alumbramiento para Navidad\ldots{} Eufrasia,
digo, en una sala que está junto a la galería, entre el despacho de Su
Majestad y la oficina donde trabajan los de Secretaría particular, se
enchiqueró con un General joven, muy nombrado, D. Juan Prim. ¿Le
conoces? Da que hablar porque es de mucho sentido, y marrajo, de los que
dejan el trapo y van al bulto\ldots{} Hace días echó en las Cortes un
discurso tan fuerte que tembló todo el Ministerio, y a D. Juan Bravo se
le indigestaron los chorizos. Pues entre otras cosas, dijo el hombre que
hemos vuelto a los tiempos de \emph{Carlos II el embrujado}, que nos
están llenando la nación de frailes y monjas, que no hay libertad, y que
este moderantismo es una farsa para que se redondeen cuatro
mamalones\ldots{} No lo dijo así\ldots{} En fin\ldots{} pidió mil
gollerías, y declaró que él es partidario del \emph{naufragio
universal}, de la libertad \emph{disoluta} de la imprenta, del
\emph{ateísmo libre}, y del ciudadano libre, o del respeto al individuo
\emph{suelto del derecho particular}\ldots{} vamos, que no sé
decirlo\ldots{} Pues por este discurso y por lo mucho que se merece el
señor de Prim, Conde de Reus, se le tiene miedo, y se determinó mandarle
a Puerto Rico\ldots{} Como te digo, trató de ver a la Reina; no pudo
ser, por causa del estado\ldots{} Hizo por Eufrasia, que le recibió en
aquella salita, y allí le estuvo poniendo varas, y él tomándolas\ldots{}
que si ella es lista, él se hace el blando para pegar más fuerte. Estas
varas de que te hablo no son cosa de amores, no vayas a creer, sino de
política. Prim, asegurando que dará la vida por la Reina, amenazaba con
tramar una revolución, si no se entra por el camino libre, y no se da
carpetazo a ese proyecto\ldots{} tú lo sabrás, eso que llaman golpe de
Estado\ldots{} que es dar un bajonazo a la Constitución, y arrastrarla
con las mulillas\ldots{} Eufrasia, diciéndole que eso del golpe de
Estado no es más que conversación de Puerta de Tierra, trató de traerle
al bando Narvaísta\ldots{} Prim hacía fu\ldots{} decía que esto de
mandarle segunda vez a Puerto Rico es una partida serrana; pero que irá
para que no se diga. Entonces la Socobio le echó mucho incienso al
Conde: le dijo que la Reina estima su valor y su lealtad, y que cuenta
con él para una combinación progresista en cuanto tenga tiempo y ocasión
de desentenderse del moderantismo, polaquería, o como se llame. Y luego
de pasarle todas estas lindezas por los morros, le pidió al General un
favor\ldots{} y aquí entra lo que a ti te interesa\ldots{} el favor
consistió en que pidiese al Ministro de la Guerra el pase a Puerto Rico
del Capitán Gracián, no sé si como ayudante o como agregado al Cuartel
General. Prim dijo que con mil amores lo haría, y despidiéndose, tomó
soleta.

---¡A Puerto Rico!---exclamó Lucila levantándose de un brinco, y
despidiendo lumbre de sus lindos ojos.---Yo con él\ldots{} Me
llevará\ldots{} Quiero verle, Rosenda, quiero verle\ldots{} Haremos las
paces\ldots{} se olvidará todo; le perdonaré\ldots{}

---Eh\ldots{} ¿qué es eso? Yo no he de permitir que hagas ningún
disparate, ni que se te malogre el matrimonio, que ha de hacerte feliz,
libre. ¡Cigüela, chiquilla mal criada y sin juicio!, si una amiga
pérfida te metió en tantas amarguras, de ellas te sacará otra amiga que
no es pérfida, sino leal, y sabe mucho\ldots{} Siéntate, y no hables de
irte a Puerto Rico, pues para ti no hay más Puerto Rico que la Villa del
Prado\ldots{}

---Déjame que disparate, y que me ciegue y me trastorne. ¿Quién te
asegura que la vida feliz viene por el lado juicioso?---dijo Lucila, en
pie, desconcertada.---Debemos obedecer al corazón\ldots{} que nunca nos
engaña.

---¿Y si te dijera yo que nos engaña casi siempre? Toma ejemplo de mí,
que he sabido dar de lado a los loquinarios y cabezas de motín, haciendo
por los hombres de peso\ldots{} ¡Y tú que vas a casarte con un viejo
rico, tú que te has sacado el premio gordo de la Lotería, hablas ahora
de tirarlo por la ventana, porque te lo manda el corazoncito!

---Pues si no me dejas hacer el disparate gordo, déjame que hable con
esa señora de Socobio, y le diga\ldots{}

---¿Qué has de decirle tú, bobalicona?\ldots{} No te haría maldito
caso\ldots{} Para hablar con ella tendrías que ir a Palacio, donde está
casi siempre.

---A su casa iría yo\ldots{} A Palacio no voy por nada de este mundo.

---¿Ni por Tomín?

---Por Tomín quizás.

---¿Y por qué tienes ese horror a Palacio si no lo conoces?

---¿Que no lo conozco?---dijo Lucila sentándose de nuevo junto a la
copa.---Como tú tu propia casa\ldots{} ¡Ay, Rosenda! tú no has vivido en
la \emph{Casa Grande}, yo sí. Con los ojos cerrados subo y bajo yo por
todas sus escaleras, y me meto por todos sus pasillos, y voy de sala en
sala, como no sean las que habitan los Reyes. Conversaciones como esas
que me has contado entre la Eufrasia y el General Prim, he oído yo
muchas, porque también yo, aquí donde me ves, he sido un poco
duende\ldots{} a mí me han puesto escondidita entre cortinas para que
oiga las conversaciones, y he llevado y traído recados con cifra\ldots{}
Y para que acabes de convencerte de que he sido algo duende, y de que lo
soy todavía\ldots{} ¿quieres que te adivine una cosa?

---¿Qué\ldots?---murmuró Rosenda entre risueña y asustada.---Adivina lo
que quieras.

---Pues te digo que hoy, aquí, hablando contigo he descubierto quién es
la persona que te favorece\ldots{} Tú me has dicho que el nombre de esa
persona es un secreto\ldots{} Yo me lo guardaré; yo te aseguro que por
mí no se sabrá. Te diré tan sólo cómo lo he adivinado. He visto aquí una
sombra de ese sujeto, una sombra no más.

---¡Qué dices, mujer!---exclamó asustada la Capitana, mirando a las
paredes, creyendo que había, por descuido, algún indicio personal,
retrato tal vez.

---No te asustes, Rosenda---dijo Lucila risueña:---la sombra la he visto
en ti, en tu voz\ldots{} ¿Por qué empleas ahora una porción de términos
de toros, que antes no te oí nunca?\ldots{} Es que ahora tienes cerca de
ti, oyéndola sin cesar, a persona que habla con esos terminachos, y a
esa persona la conozco yo. Pero mi adivinanza no es completa\ldots{} Son
dos hermanos que se parecen en la figura, y más en el modo de hablar.
Uno de los dos tiene que ser. Con que\ldots{} ¿he acertado?

---Acabaras---dijo la Capitana soltando también la risa.---.. Tienes
razón\ldots{} se me ha pegado\ldots{} Vaya, pues sí\ldots{} es uno de
los dos hermanos. Aciértame ahora cuál.

---Ayúdame tú un poquito. Los dos son cazurros, beatos, rezadores,
esquinados, y muy amigos de meterse en lo que no les importa. De cara
ninguno de los dos es bonito; de cuerpo allá se van. Sólo se diferencian
en que el uno bizca un poco de los ojos y el otro un mucho de los
pies\ldots{} quiero decir, que anda como los loros\ldots»

Rompió a reír la maliciosa Rosenda con toda su alma, y entre las risas
pudo decir a borbotones: «Ese\ldots{} ese\ldots{} el que pisa como las
cotorras\ldots{} con los pies así\ldots{} para dentro\ldots{} ¡Ay qué
gracia me ha hecho!»

---Acerté: D. Francisco Tajón. Luego te daré señas que\ldots{} no son
mortales, pero pudieron serlo.

---Cuidado, chica, que no quiere que se sepa\ldots{}

---Descuida. Pues ese señor me conoce, lo mismo que su hermano. Háblale
de mí, y te dirá si sé yo andar por Palacio\ldots{} si conozco los
enredos y el laberinto de aquella casa. Déjame que te cuente: de esto
hace tres años, y fue en una de las épocas de mi vida que recuerdo con
más disgusto. Llegué a Madrid con mi padre y mis hermanos pequeños,
muertos todos de miseria y en el mayor desamparo, sin más esperanza que
una carta de recomendación para la monja Sor Catalina de los
Desposorios. La carta era de un caballero muy cumplido a quien conocimos
en Atienza. Pues la monjita fue nuestra salvación: por ella colocaron a
mi padre en la mayordomía de los Lavaderos del Príncipe Pío.

---Sí, sí, que eran del Sr.~Infante D. Francisco\ldots{} Administrador,
D. Enrique Tajón, el hermano mayor: son tres hermanos.

---Tres. El D. Enrique se parece poco a los otros dos\ldots{} Pues sigo:
mes y medio estuvimos allí. Luego llevaron a mi padre al servicio de los
Escolapios de Getafe, y a mí a Palacio, al servicio del Sr.~D. José
Tajón.

---El que bizca de los ojos.

---Casado él, empleado en la \emph{Etiqueta}. Con su esposa y dos hijos,
de los que yo era niñera, vivía en el piso segundo, subiendo por la
escalera de Cáceres, primer cuarto a mano derecha. Todo lo que diga de
lo buena que era la señora, es poco; todo lo que diga de lo falso,
enredador y embustero que era él, sería no decir nada. ¿Te incomoda que
hable de estos señores con tanta libertad?

---A mí no. Despáchate a tu gusto.

---Respetaré a tu D. Francisco, que también es de encargo, loco por los
toros, loquito por las hembras, en privado, que en público no hay
mojigato que le gane en hacer zalemas delante de un altar\ldots{}

---No me hagas reír, mujer---dijo Rosenda, más movida a regocijarse que
a incomodarse.---Estoy en que exageras un poquito. Tu tirria contra los
Tajones es señal de que te hicieron algún daño.

---Quisieron hacérmelo, sí\ldots{} Les aborrezco porque no tenían
miramiento para una muchacha sola y sin defensa de padre ni hermanos.
Los dos quisieron abusar de mí: fácilmente podía yo defenderme de D.
José, amparándome de la señora y de los niños; pero el D. Francisco,
que, como sabes, está separado de su mujer, me dio más guerra y cuidado
mayor, porque me llevaba con engaños a este o el otro lugar apartado, de
los muchos que hay en aquella casona. Una tarde me vi tan a punto de
perdición, que para salvarme no tuve más remedio que agarrar un
candelero de bronce que a la mano encontré, y darle con él en semejante
parte de la frente. Le pegué con tanta gana, que el hombre perdió el
conocimiento, y marcado quedó para toda su vida\ldots{}

---¡Ay! no me hagas reír\ldots{} Sí, sí: la señal del candelero tiene en
la frente, aquí\ldots{} en el sitio del asta derecha\ldots{} ¡Qué risa!
Me dijo que aquel golpe fue de una caída que dio en la sacristía de la
Encarnación, estando subido a una escalera para ponerle a San José vara
con azucenas naturales\ldots{} No es mala puya la que tú le
pusiste\ldots{}

---Yo también me río\ldots{} Bueno es que una se divierta un poco
después de tantas pesadumbres\ldots{} El pobre señor quedó escarmentado,
y luego decía: `¡Vaya unos derrotes que me gasta esta novilla!'.

---Saladísimo\ldots{} Adelante.

---Por haberme hecho Dios bien parecida, cuantos hombres había en
Palacio se propasaban, créelo. Todos me adoraban, todos me hacían mil
embelecos, todos me largaban declaraciones, todos por de pronto querían
fiesta\ldots{} En fin, que yo era buena, y muchos me tenían por
mala\ldots{} Si supiera yo distinguir bien los uniformes, te diría todas
las clases de hombres, desde señores a criados, que se emperraban en
hablar conmigo. Pero nunca llegué a conocer por los cintajos y colorines
los cargos de tanto farfantón. \emph{Jefes de oficio} me escribieron
cartitas, y también \emph{Ugieres de Cámara} y \emph{de Saleta};
\emph{Llaveros} y \emph{Porteros} de banda me tiraban besos al aire; un
\emph{Tronquista} me aseguró que se mataba si no le daba el \emph{Sí};
un \emph{Portero de vidrieras} y un \emph{Delantero de Persona} hicieron
lo mismo, y de rodillas se me puso delante un día uno a quien yo creí
\emph{Carrerista} vestido de paisano, y luego resultó que era
\emph{Sangrador de Cámara}.

---Pues, hija, no estarías poco orgullosa.

---Di que me tenían mareada y aburrida. Sigo mi cuento. Pues verás: mi
amo el señor Tajón, D. José, andaba en aquel estúpido enredo, que luego
se llamó del \emph{Relámpago}, y a mi padre y a mí nos traía de
correveidiles, cursando las órdenes. Dentro de Palacio fuí yo cartero,
espía, soplona; me mandaban a charlar con las azafatas, en sus ratos de
descanso, para saber quién entraba en las habitaciones reales a las
horas que no son de entrada, y quién salía cuando no se debe salir; me
obligaban a esconderme detrás de un tapiz o entre roperos para escuchar
conversaciones\ldots{} Y luego encomiendas y recados en la calle, por
ser yo quien con mayor disimulo podía llevarlos. ¡Hala! a la Escuela Pía
de San Antón, a San Ginés, con cartas para el coadjutor, a una casona de
la calle de Fuencarral, a la zeca 3 y a la meca, vestidita de moza de
rumbo, y con dinero para alquilar una calesa si me cansaba\ldots{}
También iba al Convento de Jesús, y de allí traía entre dulces alguna
carta bien disimulada\ldots{} Un día, fíjate en esto, que es lo más
gordo, y la más fea acción que por mandato de aquella gente tengo sobre
mi conciencia\ldots{} habían enviado las monjas carne de membrillo
dentro de una tarterita de plata. Al disponer la tartera para
devolverla, me llamaron los hermanos Tajón y una camarista, nombrada, si
no recuerdo mal, Doña Candelaria, y llevándome a un cuarto que está en
la galería principal, como se entra al comedor de ordinario, me dijeron
que llevase al Convento la tartera\ldots{} Envuelta la vi en un paño de
damasco, como solía venir. La descubrieron y destaparon para que yo
viese que estaba vacía\ldots{} Luego, el Sr.~Tajón, D. José, sacó del
bolsillo un paquetito forrado en papel y cruzado con cintas verdes.
Abultaba como un libro pequeño. Díjome que me guardara en el seno el
paquetito. La camarista, que como mujer podía meter la mano donde
meterla no pueden los hombres, me desabrochó y colocó el paquetito muy
bien acomodado entre mis pechos, de tal modo, que luego de abrochada no
se me conocía el contrabando. Hecho esto me leyeron bien la cartilla. Yo
tenía que ir al Convento a llevar la tartera, y al entregarla en el
torno pediría ver a Sor Catalina de los Desposorios. Se me abriría la
puerta, y una vez en presencia de la \emph{Madre}, en manos de esta
pondría lo que yo en mi sagrario llevaba. La \emph{Madre} me daría otra
vez la tartera con algo dentro, que era como señal o recibo de la
llegada feliz del embuchado. Volví a Palacio con la tartera llena de
tocino del cielo, y los Tajones, que me aguardaban con el alma en un
hilo, me felicitaron y diéronme cinco duros.

\hypertarget{xxix}{%
\chapter{XXIX}\label{xxix}}

---Duendecillo, ¿querrás hacerme creer que no supiste lo que llevabas?

---No lo supe. Verás: al caer de aquella tarde, cuando no hacía una hora
que yo había vuelto con la tartera llena de tocino del cielo, el
Sr.~Tajón me mandó a Getafe para que allí estuviese con mi padre hasta
que se me ordenara venir.

Mucho me dio que cavilar tal determinación. ¿Será por esto? ¿será por lo
otro? Yo sospechaba\ldots{} algo veía yo; pero nada con claridad. Pues
señor: viene de repente el gran tronicio de aquella mojiganga que
llamaron del \emph{Relámpago}\ldots{} Empiezan a prender gente, y los
primeritos que caen son mis señores y el tuyo, y me los mandan
desterrados qué sé yo dónde. Mi padre y yo nos vimos perdidos, porque a
los Escolapios de Getafe no les llegaba la camisa al cuerpo, temiendo
que allá podría llegar la quema\ldots{} A Madrid nos vinimos. Mi padre
se escondió en casa de unos boteros amigos suyos de la calle de Segovia;
yo, no sabiendo qué hacerme, pues a Palacio no había de volver ni atada,
pensé que no hallaría refugio mejor que el Convento, y allí me
metí\ldots{} Ya te contaré otro día mi vida en \emph{Jesús}, donde la
mayor desdicha fue hacer mi primer conocimiento con esa perra
boticaria\ldots{} Hoy, por completar esta historia mía palaciega, bien
triste, te diré que en el Convento, andando días, supe que la noche del
\emph{tocino del cielo}\ldots{} así marco yo aquella fecha
condenada\ldots{} hubo en Palacio rebullicio y mucho miedo, del cual
nada me tocó, gracias a Dios, por estar yo en Getafe\ldots{} Por orden
del señor Mayordomo Mayor se registraron muchas viviendas del piso
segundo\ldots{} Porteros y azafatas, y hasta damas fueron registradas,
obligándolas a enseñar el pecho y a levantarse las enaguas, mismamente
como registran a las cigarreras al salir de la fábrica, por si se llevan
tabaco escondido\ldots{}

---Ya era tarde para esos registros\ldots{} ¡ay qué risa! Hija, para
contrabandista no tienes precio.

---Te lo aseguro, Rosenda: no supe lo que llevaba\ldots{} pienso que no
sería cosa buena. Déjame que suspire un poco. El recordar mi vida de
Palacio me pone aquí un peso, una opresión\ldots! Nunca he sido más
inútil que en aquel tiempo; nunca me he sentido más sola; nunca me han
aburrido tanto las máscaras, pues máscaras me parecían cuantas personas
traté en aquella casa\ldots{} Tanto me amarga este recuerdo, que no he
contado los lances de aquella mi vida boba más que a dos personas: a
Tomín, a poco de conocerle, y hoy a ti. A la \emph{boticaria}, nada o
muy poco de esto le conté, porque con esa maldita nunca tuve yo
verdadera confianza\ldots{} siempre la temía, siempre de ella
desconfiaba\ldots{} No sirvo yo para esa vida de los palacios grandes,
grandes\ldots{} Las personas me parecen figuras que han salido de los
tapices, y que hablando y moviéndose siguen siendo de trapo\ldots{} En
todo no ves más que vanidad, mentira, y todo se te confunde y se te
vuelve del revés; llegas a no saber si los criados parecen señorones o
los señorones parecen criados.

---¡Quita allá, tonta\ldots!---dijo la Capitana con franco
regocijo.---Cada una debe mirar por su adelanto\ldots{} Pues a mí me
gustaría meterme en esa vida. Para eso he nacido yo, para vivir con
suposición entre personas encumbradas, para pasar el rato curioseando,
viendo lo que se traen estos y los otros, y poniendo mis manos en
cualquier enredillo\ldots{} Verían en mí un capeo superior\ldots{}
Pronto me buscarían para las suertes de más cuidado.

---No te compongas, Rosenda. Tu Don Paco no te llevará a la \emph{Casa
Grande}, si antes no enviuda y se casa contigo.

---Es de la Cofradía del \emph{Qué Dirán} y de la santísima Opinión.

---¡Quién les había de decir a los Tajones, cuando los desterraron, que
pronto habían de volver a sus puestos y a sus intrigas!---dijo Cigüela
cavilosa.---Esto prueba que en esa casa no hay idea de justicia, ni
formalidad para nada. Sólo una persona sería justa si la dejaran, y es
la Reina; pero no la dejan: la tienen metida en un fanal pintado de
mentiras para que no vea la justicia ni la verdad. Así anda todo\ldots»

Cayó en tristes meditaciones, de las que con trabajo la sacó su amiga.
«Ya ves tú si soy desgraciada---dijo la pobre mujer suspirando.---Ni en
Palacio hay justicia, ni yo me veo con fuerza para entrar en busca de
ella. ¡Valiente caso me harían!\ldots{} No hay salvación para mí.»

---Todo es posible, querida mía---le dijo Rosenda,---si sigues por el
caminito que yo te señalo. Lo primero, casarte, antes hoy que
mañana\ldots{} después estableceros en Madrid; después libertad\ldots{}

---¡No, Rosenda, no hay libertad que valga, ni casorio, ni nada de
eso!---exclamó Lucila en una erupción repentina de su pena latente.---Yo
no me caso\ldots{} No puedo, no quiero engañar a ese buen hombre\ldots{}
Prefiero la miseria, y todos los males que pudieran venir sobre mí.» Se
levantó, y con las manos en la cabeza recorrió la estancia con incierto
paso, diciendo: «Que no me caso, que no, que no\ldots{} Pues Tomín está
vivo, tengo que consagrarme a buscarle\ldots{} Has de decirme pronto si
es D. Francisco Tajón quien le ha visto, y dónde, y has de decirme
cuándo saldrá Tomín para Puerto Rico\ldots{} Tú sabes más, más de lo que
me has dicho, Rosenda; te lo conozco en la cara, te lo leo en los
ojos\ldots»

---Si quieres que yo sea tu amiga---dijo la otra, que para sosegarla fue
tras ella, y la enlazó del brazo,---no me pidas cosa ninguna contraria a
lo que creo tu bien. Y no vuelvas a decir disparate como ese de `no me
caso', porque\ldots{} Ya sabes que gracias a Dios soy de caballería; y
que las gasto pesadas\ldots{} Con que\ldots{} ándate con tiento.

---Dime dónde está Tomín; dímelo pronto---exclamó Lucila, con todo el
brío de voluntad que su renovada pena le daba.---Mira, Rosenda, que yo,
gracias a Dios, soy de artillería; mira que no veo, que no puedo ver
nada por encima de lo que es mi pasión, mi ser, mi vida\ldots{} Dímelo
pronto.

---No quiero; no sé nada\ldots{} A ver quién puede más.

---Rosenda, no eres amiga---gritó Lucila alzando la voz con tonos
iracundos,---o lo eres también falsa y traidora, como la
\emph{boticaria}\ldots{} Si no me contestas a lo que te pregunto,
hablaré con el Sr.~Tajón.

---¿Sí\ldots? Me parece bien---replicó Rosenda, que ideó desarmarla con
un chiste.---Pero ven prevenida: tráete un candelero de bronce\ldots{}
para igualarle el testuz, marcándole el sitio del pitorro
izquierdo\ldots»

No producía Rosenda con su humorismo todo el efecto que buscaba; pero
algo se amansó Lucila oyendo aquellos disparates. «No bromees---le
dijo,---que esto es muy serio.» Insistió la moza, con la terquedad de
los enamorados, tan parecida a la de los locos. No pudiendo la otra
calmar su ansiedad con negativas, se formó rápidamente un plan de
respuesta que al propio tiempo satisficiera los anhelos de su amiga, y
la desviara de la torcida senda. Mujer de cabeza ligera, o si se quiere
ligerísima, desmoralizada y sin otra mira ya que ir derivando su
frivolidad hacia el positivismo y el vivir regalado, no era mala persona
en el fondo, y su viciada naturaleza ocultaba un corazón bueno. Viendo
cuán fácilmente se levantaban en el alma de su amiga las llamadas del
mal extinguido incendio, sintió pesar de haber atizado el fuego con la
noticia referente a Tomín. La mejor enmienda de su error no era
desmentir o retirar lo dicho, sino agregarle alguna caritativa falsedad
que a la buena moza le quitara el gusto y la intención de arriesgadas
aventuras. Como Domiciana, levantó un artificio lógico, pero con idea
benévola y mirando al bien de la infortunada mujer. «Pues te empeñas en
saberlo---dijo,---en Palacio está el hombre, con destino, que ahora no
recuerdo; pero me informaré\ldots{} Ya ves que allí es mayor locura que
en parte alguna pretender cogerle, como se coge un perrito extraviado, y
llevártele contigo. Piensa en los estorbos que allí te saldrán, en el
sin fin de personas odiosas y antipáticas que encontrarías.»

Calló Cigüela, vencida de estas razones, y su dolor, imposibilitado de
manifestarse en actos, se condensó en lo íntimo\ldots{} A los sollozos
siguió un llorar ardiente, sin tregua. Rosenda la consolaba, ya con
nuevas razones, ya con cuchufletas\ldots{} «Si quieres, cambiamos: dame
a D. Vicente con Tomín detrás de la cortina, y yo te doy a mi D. Paco
con su pisar de loro\ldots»

---¿Ves, ves lo desgraciada que soy?---decía Lucila cuando el llanto le
permitió el uso de la palabra.---A donde quiera que voy, Dios me dice:
`alto; de aquí no se pasa\ldots{}'. Dos caminos tengo: o matarme o
casarme\ldots{} No sé cuál es peor.

---Yo no vacilaría\ldots{} Me casaría primero\ldots{} y después a pensar
en matarme\ldots{} pero sin prisa, que estas cosas deben hacerse después
de bien maduradas\ldots{}

---Pero antes de casarme ¿no te parece que debo dar algunos pasos, a la
calladita, por ver de ponerme al habla con Tomín?\ldots{} ¡Le escribiré
una carta!

---¡Escribirle!---contestole Rosenda con buena sombra.---No es mala
idea; pero debes aguardar a que tu maestro te enseñe la letra bien clara
y la perfecta ortografía\ldots{}

---No te burles\ldots{} ¿Y no será fácil cogerle cuando salga para
Puerto Rico?\ldots{} Todo está en averiguar la hora de salida, y\ldots{}
Pero nada de esto puedo hacer sin que me ayude alguien\ldots»

Interrumpidas por Ansúrez, que bruscamente llegó, las dos mujeres
callaron. Lucila limpió sus lágrimas, mientras Rosenda se enteraba de
los recados que traía el buen \emph{celtíbero}.

Despachó este en cuatro palabras, ávido de desembuchar las graves
noticias que de la calle traía. «Prepárense---les dijo en el tono
solemne que usaba,---para saber del grande suceso que a estas horas va
retumbando por todo el mundo, de pueblo en pueblo. ¿Están preparadas?
Pues oigan: El Sr.~D. Luis Napoleón, que era, como se dice, Presidente
de la República de los franceses, ha dado un puntapié a la Constitución
de allá, y se quiere nombrar a sí mismo\ldots{} aciértenlo\ldots{} pues
Emperador de la Francia\ldots{} que es como ser sucesor del otro
Napoleón, que fue \emph{Primero}\ldots{} y lo que yo no entiendo es que
no habiendo tenido \emph{Segundo}, tengan ahora \emph{Tercero.»}

Oyó Lucila con desprecio la noticia, pues maldito lo que le importaba
que cayesen Repúblicas y se levantaran Imperios; pero Rosenda, a quien
algo se le alcanzaba de tales cosas, dijo que si el Sr.~Ansúrez no venía
bebido, y era verdad la especie, ello era muy grave, y traería
cola\ldots{}

---¡Cola!---exclamó Ansúrez.---Tan grande será, que por mucho que
arrastre, no le veremos el fin. En la Puerta del Sol, junto al Principal
había tanta gente que aquello parecía el pregón de la Bula, y en los
corrillos leían un parte escrito que ha venido de París por los signos
de las torres, el cual dice que Emperador es ya el caballero, o lo será
pronto, porque falta todavía el requisito de ser votado por toda la
plebe de Francia\ldots{} Según lo que por ahí corre, es ahora seguro que
vuelve a mandarnos el de Loja, quiéranlo o no Palacio y las monjitas,
porque el Napoleón, D. Luis, es gran amigote de Narváez\ldots{} como que
a comer y cenar le convidaba todos los días, y andaban siempre de
bracete por los paseos y \emph{bolívares}\ldots{} Esto se dice, y si es
verdad, yo me alegro, porque ya se va poniendo esto muy al son de la
clerecía. Bueno es que se muden las tornas, y cambien las aguas, para
que lo seco se moje y lo mojado se seque; bueno será que se limpien
muchos comederos, y se llenen otros que ha tiempo están vacíos\ldots{}

---¡Ay! no, amigo Ansúrez---dijo Rosenda con cierta inquietud:---deje
usted los comederos como están\ldots{} ¿Pero se dice por ahí que
tendremos trastornos?

---Y tales serán que lo alto se suba más, y lo bajo se precipite hasta
los profundos abismos; pues sabido es que cuando Francia estornuda,
España dice \emph{Jesús}, como que las dos naciones están tan unidas por
fuera y por dentro como la nariz y la boca\ldots{} En fin, señora, ya
sabe lo que ocurre, y mi hija y yo nos vamos, que es hora ya de tocar a
retirada.»

Despidiose Lucila de su amiga y partió con su padre, abatida,
silenciosa, llevando en sí algo para ella de más peso y magnitud que el
nuevo Imperio que a punto estaba de levantarse. Recorrido habían ya
largo trecho, cuando Lucila, parándose, dijo al \emph{celtíbero}:
«Padre, cuando yo estaba en el Convento, siempre que venían noticias de
alguna trifulca en Francia, decían las \emph{Madres}: `esos demonios de
franceses nos van a traer acá un cataclismo'. Usted, que con su talento
natural ve tan claras todas las cosas, dígame: ¿cree que habrá en España
cataclismo?»

---Hijita, deja que pueda hacerme cargo de lo que resulte en Francia de
ese voto que ha de dar la plebe. El echar a rodar Napoleón el Trono de
la República, para poner las gradas del Imperio, quiere decir que no se
quieren las pasteleras libertades\ldots{} ¿Pues qué hará en vista de
esto el Progreso\ldots? Sacará clavos con los dientes antes que
humillarse\ldots{} Veremos, y vengan días, de donde podamos sacar el
juicio de las cosas.

---Porque yo quiero que haya cataclismo, padre, mucho cataclismo; que
los injustos caigan y sean pisoteados por los sedientos de justicia; que
los que cometieron tropelías sean hechos polvo, y que los buenos se
alegren. Justicia quiero, y habiendo justicia, habrá paz. ¿Esto cómo se
llama? ¿Se llama \emph{República}; se llama \emph{Imperio?»}

\hypertarget{xxx}{%
\chapter{XXX}\label{xxx}}

El efecto que causó en el alma de Lucila la noticia, dada por Antolín de
Pablo, de que Halconero llegaba, lo más tarde, al cabo de dos días, fue
de verdadera consternación. ¿Por qué volvía? ¿No era mejor que se
quedase por allá?\ldots{} La prometida esposa se conturbaba con la idea
de verle, y metiendo su exploradora mano en el corazón, tocaba frialdad,
aborrecimiento. Del anunciado regreso de D. Vicente la consolaba la idea
y presunción de que a su llegada hubiese un poco de cataclismo.

A su padre, que a verla iba diariamente, le dio un interesantísimo
encargo: «¿No tiene usted conocimientos en el Ministerio de la Guerra?
¿No conoce a un cabo que está en las oficinas?\ldots{} ¿Sí? Pues
averígüeme\ldots{} ello es muy fácil, padre, y hasta los gatos del
Ministerio deben saberlo\ldots{} averígüeme cuándo sale el General Prim
para Puerto Rico.»

---Va de Capitán General; le embarcan porque se pasa de valiente\ldots{}
Es, según se dice, hombre de mucha idea\ldots{}

---Y eso es lo que estorba.

---No sé por qué. Yo tengo mucha idea, y no me mandan a ninguna parte.

---Porque no temen a los humildes. El reino de los humildes está muy
lejos.

---¡Y tan lejos\ldots! Ni aunque uno se suba encima de los encumbrados
puede alcanzar a ver dónde está ese reino.»

Llegó Halconero: viéndole y tratándole, se calmó la fiebre de Lucila, y
las aberraciones disparatadas de sus sentimientos. No le aborrecía,
¡pobre señor! ¿Cómo aborrecer a quien le había hecho tantos beneficios,
y aun mayores e inapreciables se los prometía? Gustoso de aprovechar el
tiempo en la Villa y Corte, Halconero fue a visitar el nuevo Congreso,
llevando por delante, naturalmente, a Lucila y Eulogia, bien apañaditas.
Habíale dado las papeletas el Sr.~D. Matías Angulo, diputado por
Navalcarnero, como él propietario rico y persona sencilla y de las
mejores intenciones, así en política como en todo. En la admiración de
aquel lujoso monumento elevado a la Soberanía Popular, pasaron los tres
una mañana, y desde los salones de Sesiones y de Conferencias hasta la
Biblioteca, salas para Secciones, taquígrafos, etcétera, nada se les
quedó por examinar. Admiraba Eulogia con preferencia las ricas
alfombras, Lucila los altos techos con pinturas, y D. Vicente perdía el
tino ante la profusión de terciopelo encarnado\ldots{} Visitaron
asimismo el \emph{Museo de Artillería} y la Historia Natural, y no
continuaron por otros barrios de Madrid su instructivo zarandeo, porque
Lucila se resistió, sin dar de su negativa razones claras, a visitar las
\emph{Reales Caballerizas} y la \emph{Armería Real}\ldots{} Se fatigaba,
se le iba la cabeza, según dijo\ldots{} Pensando que el teatro la
distraería más que los Museos, propuso D. Vicente ir a ver \emph{la
Adriana}, obra muy hermosa de la que se hacían lenguas cuantos la habían
visto. Representábase en los \emph{Basilios}, y era el éxito mayor de la
temporada corriente. En efecto: allá fueron una noche, y no puede
describirse la emoción de los tres ante el interesante drama; con el río
de lágrimas que derramaron las mujeres, competían los pucheros del
hombre, queriendo echárselas de valiente. A Lucila le llegó al alma el
caso de la pobre cómica, tan bien representada por \emph{la Teodora}, a
quien envenena una princesa su enemiga (que también era un poco
\emph{boticaria}), con el simple olor de un ramillete. Le pareció la
comedia cosa real, y la emoción duró en su alma muchos días.

Siguió a esto un período de compras, en las cuales nada se hacía sin que
Lucila diera su \emph{exequatur}, previo examen de las cosas. De tienda
en tienda iban los tres; mirando y escogiendo lo que se diputaba mejor
dentro de la modestia, adquirió Halconero cama de matrimonio, de bronce
dorado, según los mejores modelos de una industria moderna, y colchón
\emph{de muelles elásticos}, que eran última novedad. Tras este tan
necesario y útil mueble, se compró un espejo grandecito, un juego de
reloj y floreros, un veladorcito \emph{maqueado}, vajilla de porcelana,
y juego de café, con maquinilla de reciente invención para hacerlo en la
misma mesa. Con estos goces inocentes de preparativo nupcial estaba el
buen señor en sus glorias. Antes de Navidad partió para su pueblo,
dejando determinado que volvería después de Reyes, \emph{ya para
casarse}. La boda sería entre San Antón y la Candelaria.

Ansiosa de sostenerse inexpugnable ante los arrebatos de su propio
corazón enamorado, Cigüela 4no salía más que para oír misa, en San
Andrés, y se propuso no volver a poner los pies en casa de Rosenda. No
aviniéndose esta con el desvío de su amiga, fue a verla, mostrándose en
la visita como la misma discreción y la prudencia en persona. A pesar de
no encontrarse presente Eulogia, la Capitana no nombró a Tomín, ni dijo
cosa alguna que con el perdido caballero tuviese relación. No se atrevió
Lucila a preguntarle; pero leyendo en los ojos de Rosenda, entendió que
algo sabía esta, y no quería decírselo por no perturbar el ánimo de su
amiga\ldots{} Lo agradecía, y al propio tiempo lo deploraba. Temía
saber, saber ansiaba. ¿Cómo armonizar deseos tan contrarios? Cuando
partió la maliciosa Capitana, la presunta esposa de Halconero se decía:
«Me ha dado olor a sepulcros\ldots{} En los ojos de Rosenda he visto una
cosa que se parece al último renglón de un libro triste\ldots{} Ya veo
claro. Tomín ha salido para Puerto Rico\ldots{} ¿Y dónde está ese
condenado Puerto Rico? De aquí allá ¡cuántas llanuras y montañas de
agua!»

Esta idea embargó su ánimo por muchos días, idea de duelo, seguida de
efusiones dolorosas de un cariño inextinguible, que derivaba hacia las
esferas de Ultratumba; porque en verdad, ¿qué cosa más parecida a la
muerte que un viaje a Puerto Rico? Y la cantidad de agua que entre Tomín
y su amada se extendía, era la expresión más sensible del infinito de la
ausencia. Lloraba Lucila sobre aquellas turbias aguas, que se movían con
ritmo y balanceo semejantes al navegar de las almas de este mundo al
otro\ldots{} En tal situación de espíritu, consolándose con el
desconsuelo, y meciéndose en lo infinito, sorprendieron a la infeliz
mujer sucesos de interés general, y otros de su particular incumbencia.
El feliz parto de la Reina, con público regocijo, fiestas,
iluminaciones, no fijó tanto su atención como las cuatro palabras que le
dijo el buen Ansúrez una tarde: «Querida hija, por fin te traigo
despachado el encargo que me diste, y es que\ldots{} tocante a la fecha
de salir para Puerto Rico el señor General Prim, no hay fecha ninguna,
porque el señor General ya no va a Puerto Rico.»

Palideció Lucila. Por las inmensas aguas no iba Tomín. ¿Pero quién
aseguraba que no fuera más tarde, con otro General, solo tal
vez?\ldots{} Examinando probabilidades, en sombría cavilación, vino a
parar en que todo era posible y todo imposible. No prestó atención a las
lamentaciones de su padre contra el clérigo Merino, que no acababa de
arrancarse al ofrecido préstamo, bien porque no hubiera realizado la
cobranza del crédito antiguo, bien por marrullería y ganas de fastidiar.
Esta última versión le parecía razonable, pues de sus conversaciones con
él, en los solitarios Paseos por la \emph{Tela}, había sacado la
presunción de que era D. Martín hombre cerrado a la benevolencia y malo
de por sí, amigo de martirizar: el único deleite de sus ojos era ver el
ajeno sufrir, y ninguna música le gustaba como el rechinar de dientes
del hombre desesperado\ldots{} Sin llegar a la desesperación, Ansúrez
deploraba que estando tan cerca el matrimonio de su hija, no pudiera él
festejarlo con tienda abierta, para que se dijese que el padre de la
novia era un comerciante establecido en la calle de las Maldonadas. ¡Y
que no haría poco servicio al Sr.~Halconero anunciando la venta en
comisión, y al por mayor, del fruto de sus feraces tierras!\ldots{}
Encomiando el rico \emph{género}, todo Madrid diría: «¡Cebada de
Halconero, huevos de Halconero, uvas de Halconero!\ldots»

En Navidad y en Reyes vio Lucila a Rosenda, y en los ojos de ella, así
como en su acento y actitudes, observaba la misteriosa reserva que
traducida con buena voluntad al lenguaje corriente, quería decir: «Sé
muchas cosas, pero las callo; mi deber es callarlas.» Por la delicadeza
y corrección que le imponía la proximidad de su boda, no se determinó a
preguntarle. Nada podía sacar del reservado escondrijo que llevaba en su
mente la Capitana, urraca codiciosa que escondía las ideas y noticias
que a Tajón robaba\ldots{} Pasó Cigüela en melancólicas dudas algunos
días, y razonaba su estado anímico en esta forma: «No quiero más que
saber, saber\ldots{} ¿Se habrá muerto \emph{Min}? ¡El silencio de
Rosenda dice tantas cosas! Dice muerte, dice vida y nuevas
traiciones\ldots{} Ya doy en creer que el traidor es él, y para
perdonarle, necesito saber la verdad\ldots{} ¿Cómo he de perdonarle, si
no sé\ldots?» Hervían estas ideas en su mente, cuando se encontró de
manos a boca con Ezequiel: ella salía de San Andrés, donde había oído
misa, y él entraba con un gran manojo de velas\ldots{} Requerida por el
mancebo, retrocedió la moza, y sentada en un banco próximo a la puerta,
esperó a que se desocupara de su carga para hablar con él.

---¿Qué querías decirme\ldots? Cuéntame\ldots{}

---¿No te has enterado de que Domiciana se ha ido a vivir a
Palacio?\ldots{} Allí la tienes de camarista suplente, con un
sueldazo\ldots{} Le han dado una habitación muy grande, subiendo por la
escalera de Cáceres, el primer cuarto a mano derecha\ldots{}

---Lo conozco, conozco ese cuarto. He vivido en él\ldots{} ¿Y qué
más?\ldots{} No me tengas en ascuas\ldots{} acaba pronto.

---Pues mi padre está cada vez peor de la vista.

---¡Pobrecito! Eso no me importa. ¿Se ha llevado tu hermana los muebles
de tu casa?

---Algunos\ldots{} Parece que le dan el cuarto amueblado. Se llevó, eso
sí, manojos de hierbas, y los morteros, los filtros\ldots{}

---Ya\ldots{} en Palacio practicará la \emph{botiquería}\ldots{} ¿Y qué
tal\ldots{} tiene la casa bien puesta?

---No la he visto; lo primero que nos encargó fue que no pareciéramos
por allá.

---¿Qué me dices, Ezequiel?

---¿Verdad que es una ingratitud\ldots? Mi padre está muy triste, pero
muy triste. Gracias que algunas tardes, en coche, viene Domiciana a
verle, y con esto se consuela el pobre.

---¿Ha llevado tu hermana a su servicio la criada que teníais?

---¿La Patricia? Allá se la mandamos; pero la despidió más pronto que la
vista\ldots{} No quiere a nadie de nuestra casa. ¿Ves qué esquiva y qué
testaruda? Ni que tuviéramos la peste\ldots{}

---No conoces tú a tu hermana, \emph{Zequiel}. Si os mantiene lejos de
su nueva casa, y no quiere que vayáis a visitarla, será que allí esconde
algo, algo que no debéis ver vosotros, ni nadie\ldots{}

---Puede que tengas razón. De algún tiempo acá, todo lo que hace mi
hermana es muy raro\ldots{} Mi padre suele decir como rezongando: `Dios
la perdone'.

---No la perdonará---exclamó Lucila con acento de ira, olvidándose de
que estaba en la iglesia.---\emph{Zequiel}, si me averiguas lo que
Domiciana oculta en su casa de Palacio, te doy\ldots{} no sé qué te
daría. Pídeme lo que quieras\ldots{}

---Lucila, sabes que te quiero mucho. ¿Qué no haría yo por ti? Sueño
contigo, y pienso que mi mayor felicidad sería tenerte siempre a mi
lado. El otro día, hablando de ti con mi padre, le dije que si ibas tú
por allí, te dijese, como cosa suya, lo mucho que te quiero\ldots{} Mi
padre se echó a reír y me contestó con una frase que me lastimó mucho.
Dijo, dice: `tú eres poco hombre para Lucila'. Eso es faltarle a uno. Yo
no seré todavía bastante hombre; pero voy siéndolo cada día más\ldots{}
Pues dime ahora qué tengo que hacer para averiguarte lo que deseas.

---Ir a la casa que habita tu hermana, en Palacio; entrar en ella
atropellando por todo, registrar bien las habitaciones, ver,
observar\ldots{}

---Sí que lo haré, y a todo el que quiera estorbarme el paso, le daré un
empujón\ldots{} Pues déjame ahora que te diga lo que tienes que darme en
pago de ese favor\ldots{} El caso es que aquí no puedo decírtelo, porque
estamos en la iglesia, y me da reparo\ldots{} Salgamos a la calle,
vámonos por la Costanilla, y te lo diré\ldots{} Aquí siento más
vergüenza que en la calle.»

Salieron. Lucila era una máquina que funcionaba inconsciente y con la
mayor rapidez en todo lo que condujera a la satisfacción de su
curiosidad. Al llegar al extremo de la Costanilla, entrando en la
plazoleta de San Pedro, Ezequiel, que iba silencioso junto a su amiga,
se paró, y más pálido que la cera de su taller le dijo: «Luci, yo
pensaba pedirte\ldots{} y perdóname si es desacato\ldots{} pensaba
pedirte por este favor\ldots{} que me dieras un beso; pero ahora veo que
es muy poco, Luci, es muy poco un beso: debes darme lo menos
tres\ldots{} o cinco\ldots»

---Y más, muchos más---dijo Lucila ardiendo en curiosidad, y movida
también a lástima intensa del pobre muchacho candoroso.---Si me traes la
verdad que busco, te daré tantos besos como palabras necesites para
contármelos, tantos como pasos has de dar de aquí a Palacio y de Palacio
aquí.

---¡Ay, qué buena eres, y qué agradecido quedaré, Luci!---dijo el pobre
chico casi llorando.---Iré corriendo. Pero\ldots{} para que yo vaya con
más ánimos, ¿por qué no me das uno a cuenta? Por ser el primero, ha de
saberme\ldots{} como el cuerpo de Nuestro Redentor, cuando uno comulga.

---Sí que te lo doy. Toma uno, toma dos, toma más\ldots---dijo Lucila
besándole, como besan las madres a los chicos para convencerles de que
deben ir a la escuela.

---No más---dijo al fin Ezequiel embebecido y asustado.---Pasa
gente\ldots{} pueden fijarse, y si lo sabe el que va a ser tu
marido\ldots{} ¡Jesús!

---Pues ve pronto\ldots{} yo te acompaño hasta la calle de
Segovia\ldots{} y en la subida de la Ventanilla, ¿sabes?\ldots{} allí te
espero\ldots{} No, no\ldots{} para que me encuentres más fácilmente, y
no haya equivocación, te espero en las Monjas del Sacramento.

---Allí\ldots{} Vamos, Luci.»

\hypertarget{xxxi}{%
\chapter{XXXI}\label{xxxi}}

Hízose todo conforme a programa. Media hora llevaba la moza de invocar
al Santísimo, a la Virgen y a todos los Santos, con fervoroso rezo, para
que en aquella terrible incertidumbre le concedieran el consuelo de la
verdad, cuando vio entrar a Ezequiel. Venía muy abatido, la
consternación y el miedo pintados en su angelical rostro. Con ansioso
mirar le devoró Lucila, y como notara en él cierta dificultad para la
articulación de la palabra, le sacudió el brazo, diciéndole: «Habla
pronto, tontaina\ldots{} ¿qué has visto?»

---Nada---balbució el cererillo.---Siento no traerte\ldots{} no poder
decirte\ldots{} Lucila, no me quieras mal porque no haya sabido\ldots{}
No pude, Lucila\ldots{} Tú sabes qué genio gasta Domiciana\ldots{}
Llegué, llamé\ldots{} Déjame que tome resuello. Del disgusto no puedo
respirar\ldots{} Pues\ldots{}

---En fin---dijo Lucila a punto de estallar en cólera,---que no has
hecho nada\ldots{} que has sido un ganso, un idiota, un avefría\ldots{}

---Déjame que te cuente\ldots{} Abriéronme la puerta, y cuando yo estaba
diciéndole a la criada que me abrió si podía ver a mi hermana,
salió\ldots{} ¿quién creerás que salió?

---¿Quién, quién, pavo del Paraíso?\ldots{} Acaba pronto.

---Domiciana; y apenas había yo abierto la boca para decirle\ldots{} lo
que tenía que decirle, me la tapó con estas palabras que me dejaron
yerto: «¿No te he dicho que aquí no tienes que venir para nada? ¿Harás
alguna vez lo que yo te mando? ¿No comprendes que si te digo: `Ezequiel,
haz esto', tu deber es callar y obedecerme?» Y diciéndolo, me cogía por
un brazo y me ponía de la puerta afuera\ldots{} Yo no sabía lo que me
pasaba.

---Vámonos de aquí---dijo Lucila, que se sintió leona, y temía que su
furor estallara en el recinto sagrado. Agarró al mancebo por un brazo, y
tirando de él, más bien arrastrado que cogido, le sacó a la calle.
Torciendo hacia el Sacramento, Ezequiel proseguía: «Me despidió con un
tira y afloja de palabras tiernas y de amenazas. `Hermanito mío, ¿qué
más quisiera yo que tenerte siempre a mi lado? Algún día será, y ese día
no está lejos\ldots{} Esta casa no es mía, y no siendo mía, menos puede
ser tuya\ldots{} Vete corriendo por donde has venido, y que no te vea yo
por aquí, mientras no se te llame\ldots{} Adiós, y a casa\ldots{} Anda,
hijo, anda'. Esto me dijo, y yo\ldots{} Lucila, perdóname por no haber
podido hacer tu encargo\ldots{} Yo no sirvo, yo no sirvo para
esto\ldots{} No he cumplido, y debo devolverte los besos que me diste.»

Llegaban ya a la Plazuela del Cordón. Despechada Lucila y fuera de sí,
viendo que el cererillo aproximaba su rostro al de ella en ademán de
besarla, le rechazó con vigoroso empujón, diciéndole: «Sinvergüenza,
vete de ahí\ldots{} Déjame, pavo de agua\ldots{} ¡Vaya que
atreverse\ldots! ¡Si te ve mi marido\ldots! ¡no era puntapié\ldots!»

El pobre chico permanecía frente a ella, suspenso, afligido\ldots{}
Mirándola con inmenso desconsuelo, sus labios se plegaron, se llevó los
cerrados puños a los ojos. «Echa a correr para tu casa, mostrenco---dijo
la moza amenazándole con la mirada fulgorosa y con el gesto.---Vete,
vete, si no quieres que te lleve yo por delante, sacudiéndote el polvo
de las costillas\ldots» Apenas dijo esto, y viendo la humildad y
amargura del pobre muchacho, aquel noble corazón que fácilmente pasaba
del arrebato fogoso a la piedad entrañable marcó un movimiento de
compasiva aproximación al pobre cerero. «Hijo mío, perdóname---le
dijo.---Como estoy tan rabiosa, he descargado contigo, que no tienes
culpa\ldots{} Vaya, no llores\ldots{} Ya me pagarás los besitos otro
día\ldots{} Aquí no puede ser\ldots{} Ya ves que pasa gente. Mira: dos
señores sacerdotes. ¡Qué dirían\ldots! Ea, a tu casa, y yo a la mía.»
Sin esperar a más razones ni cuidarse de si Ezequiel partía, se
precipitó velozmente por la bajada del Cordón. Ciega y disparada, fue al
taller de boteros donde trabajaba su hermano y vivía su padre, dejando a
éste recado urgente de que se avistara con ella en su casa lo más pronto
posible. Llamábale con premura sin saber claramente para qué. Su
pensamiento desbocado saltaba de las resoluciones más lógicas a las más
absurdas; y al propio tiempo, de su mente no se apartaban hechos y
personas de grande valor en la vida de la infeliz mujer. La boda estaba
próxima, pues corrían los últimos días de Enero, y aquel dichoso
acontecimiento se había fijado para el 3 de Febrero, día de San Blas.
Como el 3 caía en martes, y en ello no había reparado D. Vicente ni
Eulogia, seguramente trasladarían el casorio al miércoles 4. Todo esto
pensaba Lucila camino de su casa, haciendo un tremendo revoltijo de las
cosas positivas y las imaginarias. «Tengo que componer mi carátula---se
decía,---para no entrar en casa tan sofocada. Debo de ir como un
cangrejo; mis ojos serán lumbre\ldots{} Subiré despacio esta cuesta, y
luego, al llegar a Puerta Cerrada, compraré los clavitos dorados para
colgar láminas que me encargó Vicente, y compraré la cinta de seda y la
cinta de algodón\ldots{} ¡Buena se pondrá Eulogia si no llevo todo
eso!\ldots{} ¡Sabe Dios, sabe Dios si llegaré a casarme! Lo que puede
suceder, en la mente de Dios está. Dios me depara mi venganza\ldots»

Al entrar en su casa, disimulando lo mejor que pudo su turbación,
encontró a Don Vicente con un sacerdote, su amigo y algo pariente, a
quien había llevado con propósito de presentarle a su futura. Era D.
Francisco Pradel, párroco de San Justo, que se mostró con ella muy
amable y le dio mil parabienes. Ya la conocía de verla en su parroquia.
Al despedirse aseguró que sería para él muy satisfactorio imponerles el
santo yugo\ldots{} Poco después, de las hidalgas manos del novio recibió
Cigüela un alfiler de pecho con cuatro brillantitos y en medio un buen
rubí, una pulsera, pendientes con perlitas, y otras joyas lindas y
modestas. La gratitud y un temor que de lo hondo le salía inundaron de
lágrimas sus ojos. Halconero estuvo a punto de llorar también. Lo que
espantaba a Lucila era el miedo de ser ingrata\ldots{} «Voy creyendo que
soy un monstruo---se decía,---y yo no quiero ser monstruo: Señor,
justiciera sí, monstruo no.»

Con pretexto, ciertamente bien motivado, de probar un cuerpo en casa de
la modista, salió al siguiente día con su padre, a primera hora de la
tarde del sábado 31 de Enero. Llegando a la calle Mayor, junto a la
Almudena, preguntó Ansúrez a su hija si no sería conveniente, ya que de
pasear se trataba, bajar a la \emph{Tela}, donde estaría de fijo tomando
el sol el amigo D. Martín. Entre los dos le darían el último tiento.
Contestó Lucila que había salido con el propósito de ir a Palacio.
Subirían al segundo piso, donde habitaban personas a quienes ella tenía
que visitar.

---¿Y tardaremos mucho?---preguntó Ansúrez un tanto receloso.

---Eso sí que no lo sé---replicó ella.---Podremos despachar en un
santiamén, o tardar mucho, según\ldots»

Entraron en la Plaza de Armas, por el gran arco de la Armería: con paso
no muy vivo, porque Ansúrez iba sin gusto y como si le arrastraran,
recorrieron la línea entre el arco y la puerta lateral de Palacio.
Vacilaba el \emph{celtíbero}; su hija le cogió del brazo, y en esto,
vieron a un señor que de la \emph{Casa Grande} salía. Si ellos se
quedaron como alelados mirándole, el señor plantado en la puerta, les
echó la vista encima con esa curiosidad arrogante y descortés de quien
tiene por oficio atisbar las caras para descubrir las intenciones. Era
D. Francisco Chico, que por la estatura no merecía tal nombre, viejo,
seco y estirado, con patillas bordando la quijada dura, el pelo
entrecano, la actitud como de perro que olfatea. Lo más característico
de su rostro, lo que le hacía inolvidable para cuantos una sola vez le
veían, era la chafadura de su nariz en el arranque de ella, señal
indeleble de una tremenda pedrada que le dieron en Miguelturra, su
pueblo, por querellas locales de pandilla. Perteneció D. Francisco al
bando de los llamados \emph{Valerosos}, y cumplía como campeón terrible:
alguna vez, si a muchos pegó de firme, también hubo de tocarle la china.
Del bandolerismo villanesco pasó a las gestas del contrabando, en tierra
firme y mar salada, y ya mocetón le metieron en la policía de Madrid,
donde llegó por su astucia y su valor indomable al puesto de jefe, que
desempeñó más de cuarenta años. Era hombre terrible, de sagaz
inteligencia para tan ingrato servicio, y a los poderosos inspiraba
confianza, como a los débiles espanto. Llegó a ser al modo de
institución, personificando los arrestos insolentes de la Seguridad
Pública, y el odio con que el pueblo pagaba las vejaciones justas o
arbitrarias que sin cesar sufría.

Quedaron, como se ha dicho, suspensos Lucila y su padre, sin atreverse a
dar un paso más, invadidos del terror que Chico infundía: avanzó este
hacia ellos con firme paso, y en la forma destemplada que era en él
habitual interpeló al \emph{celtíbero}: «Hola, Jerónimo\ldots{} ¿se
puede saber qué buscas tú por aquí?» Volviole Cigüela la espalda, y se
llevó las uñas a la boca para mordérselas. Trémulo, descubriéndose,
Ansúrez contestó: «Señor, veníamos paseando, y como uno está tan
orgulloso de que nuestros queridos Reyes se alberguen en palacio tan
magnífico\ldots{} nos llegamos a ver y admirar ese gran patio\ldots{} Y
como españoles que adoramos a nuestra Reina, veníamos a visitarla y a
echarle nuestros homenajes. Triste pueblo somos, y nuestros homenajes y
visitas no pueden ser otros que mirar desde la calle las ventanas del
cuarto donde mora la perla de las Reinas.»

---Anda, que pareces la cabeza parlante---dijo Chico, requiriéndole, con
el movimiento marcado por su bastón, a que siguiera su paseo por lugar
distinto del patio.---Otro que mejor hile las palabras no
conozco\ldots{} ¿Y esta joven es tu hija?» Volviose Lucila hasta darle
de cara, pero sin mirarle. «¡Pues no es la niña poco vergonzosa! Anda,
¿qué te han hecho las uñas para que así las maltrates y te las
comas?\ldots{} Bonita eres; pero no hagas mañas, que se te va toda la
gracia. Paseen por la \emph{Tela}, o por la Virgen del Puerto, que aquí
no se les ha perdido nada\ldots{} Jerónimo, mucho cuidado conmigo; y tú,
pimpollo, no andes en malos pasos, que voy y se lo cuento al amigo
Halconero\ldots{} ¡Largo!»

Con una mirada, que en Ansúrez infundía más ganas de correr que una
carga de caballería, les echó hacia el arco grande. Al paso que tomó
Jerónimo hubo de ajustarse Lucila. Miraron hacia atrás, y vieron al
temido polizonte plantado en el propio sitio, atento al camino que
seguían. «Es mi D. Francisco un águila para las intenciones---dijo
Ansúrez medroso.---¿Qué se habrá creído ese prepotente?\ldots{} Pueblo
somos, pero pueblo honrado, y nada de más haría la Serenísima Señora
Reina en permitir que nos llegáramos a su trono para besarle la Real
mano.» Abrumada bajo la fatalidad, que cruel, o piadosamente, quién lo
sabe, atajaba sus propósitos, Lucila no decía nada, y siguió a su padre
hasta donde quiso llevarla; llegaron al Cubo de la Almudena, y andando,
andando cuesta abajo, por un portillo derrengado pasaron a una especie
de alameda, cuyos árboles raquíticos, enanos y sedientos parecían
increpar al sol con el gesto rígido de sus ramas desnudas. El suelo
blanqueaba de puro polvo. A un lado y otro, en trozos de sillería que
hacían oficio de bancos, se veían parejas de soldado y criada, o
solitarios y melancólicos paseantes. El sitio era desapacible, sin otros
encantos que el espléndido sol, y el despejado horizonte que mirando
hacia la parte del río, Casa de Campo y Sierra, se veía. Un cielo claro,
limpio, desesperante de extensión azul sin accidente de nubes, coronaba
la tristeza luminosa de aquel gran paisaje, del más puro Madrid.

---Mira, mira---dijo Ansúrez a su hija señalándole un bulto negro que
subía, figura tan escueta como los enfilados árboles:---aquí tenemos al
D. Martín de mis pecados.

---¿Y me trae usted aquí para ver a ese viejo loco\ldots?---dijo Lucila
desolada, colérica.---Yo me voy, padre\ldots{} ¿Por dónde salgo de este
páramo indecente, de este Infierno de polvo?

---Aguarda, hija\ldots{} Ya el Sr.~Merino nos ha visto. Viene hacia
nosotros.»

Acercábase el clérigo despacio, impasible, y su rostro adusto, pomuloso,
no expresaba más que el desdén de toda criatura. Su enorme sombrero de
teja, chafado y mugriento, obscurecía sus facciones, dándoles un tinte
terroso, de adobes viejos caldeados por el sol de cien años. Iba
levantando polvo, que le blanqueaba los zapatos y los bajos de la
sotana. Recogía el manteo en el brazo izquierdo, y con el derecho hacía
un pausado movimiento de sembrador.---Buenas tardes---dijo al ponerse al
habla.---Yo bien\ldots{} ¿y en casa?\ldots{} ¿Viene la moza de
paseo?\ldots{} Bueno. ¿Con que nos casamos, eh? Y con un hombre
rico\ldots{} No es mala suerte\ldots{} Aprovecharse, que todo se acaba,
y hombres ricos van quedando pocos.» Contestó la joven con las palabras
precisas para no ser descortés, y se sentó en un pedazo de sillería.
Había muchos por allí de forma curva, como pedazos del brocal o pilón de
una destruida fuente.

No tenía Lucila gana de conversación, y hasta le enfadaba oír lo que los
demás hablasen. No lejos de ella, en otro sillar, se sentó D. Martín;
Ansúrez permaneció en pie; y creyendo ver en el clérigo disposiciones a
la benevolencia, le instó a que de una vez se clareara, tocante al
préstamo, para saber a qué atenerse. «A eso voy, a eso iba---replicó el
cura extravagante;---pero antes os diré otra cosa. Ya sabéis\ldots{} y
con los dos hablo, hija y padre\ldots{} ya sabéis que estamos abocados
al cataclismo. Oiréis por ahí que vuelve Narváez. No lo creáis\ldots{}
Narváez no volverá más\ldots{} El maldito moderantismo es cosa
concluida. ¿Quién vendrá? Vendrán todos y no vendrá nadie. ¿Quién
mandará, quién obedecerá? Nadie y todos\ldots»

\hypertarget{xxxii}{%
\chapter{XXXII}\label{xxxii}}

---Si lo que anuncia D. Martín---declaró Ansúrez,---quiere decir que
veremos el fin de las rapiñas, bendígale Dios la boca. Pero a mí me dice
mi razón natural que la barredera de bolsillos no se acabará mientras
vengan tantos inventos nuevos de comodidades y regalo del vivir, porque
ellos traen las tentaciones, y los hombres de acá, que han visto cómo
triunfan y gastan los extranjeros ricos, quieren ser como ellos. La
tierra no lo da, que si la tierra lo diera, todos nadaríamos en la
bienandanza; y estando secos los pechos de la gran madre, el hombre fino
y agudo, que apetece buena vida porque el cuerpo y hasta la mesma
ilustración se lo piden, por ley natural deja crecer sus uñas todo lo
que se le merma la voluntad de trabajar. Loco es en España el que fíe
del trabajo para vivir a gusto, que de su sudor no ha de sacar más que
afanes, y ser el hazme reír de los que manipulan con lo trabajado. Tres
oficios no más hay en España que labren riqueza, y son estos: bandido,
usurero y tratante en negros para las Indias. Yo de mí sé decir que
habiéndome pasado la vida sobre la tierra, echando los bofes, sin fruto,
ahora no miro más que a reunir comerciando un capitalejo de mil duros:
me basta. Prestando dinero al interés de ciento por ciento, que hay
quien lo tome y quien lo pague, hágome con una renta igual a mi
principal, que será el mejor alivio de una vejez honrada.

---Alto ahí---dijo D. Martín, saliendo por un instante de su
impasibilidad,---que a interés mucho más módico que el ciento, he
colocado yo mis ahorros, y todo me lo han quitado los malos pagadores,
amparados por la curia maldita\ldots{} El usurero se cae también a los
profundos abismos, como caerán el militar insolente que oprime a la
Nación, el contratista que le chupa la sangre, el ministro que ampara
tantas contumelias; caerán la hipocresía y la falsedad que han
corrompido la honradez y buena fe de la Nación española\ldots{} y debajo
de todos, porque caerá el primero, veréis a Narváez, con toda su
infernal caterva de generales\ldots{} ¿Habéis oído contar las comilonas
y orgías de Palacio, y las que el sátrapa daba en su casona de la calle
de la Inquisición con el dinero que a manos llenas le regaló Isabel para
sus lujos? Pues mientras los cortesanos se hartan en banquetes, el
pueblo cena pan seco, y por no tener para carbón, que vale, como sabéis,
a catorce reales, no puede ni calentar agua para hacer unas tristes
sopas\ldots{} Desde que tomó Narváez las riendas, España no es más que
un laberinto de todos los males, y ahí tenéis al empleado que se
merienda al contribuyente, al policía que nos encarcela al menor
descuido, y al militar que por un triquitraque saca el chafarote y
acuchilla a los ciudadanos. Habéis visto que somos víctimas de tantos
vejámenes, atropellos y contumelias; que el robo es la suprema ley, pues
no sólo se roban riquezas, sino personas. Los hombres roban la mujer que
les agrada, y las mujeres al hombre que les peta. Y la Justicia para
castigar estos crímenes ¿dónde está?

---No se ve la Justicia, no se ve la ley---dijo Lucila, que gradualmente
se interesaba en la conversación.---Pero la Justicia ha de estar en
alguna parte, Sr.~D. Martín.

---A eso iba, a eso voy\ldots{} Coged todos los candiles que hay en el
mundo, encendedlos, recorred con ellos el suelo de España buscando la
Justicia, y no la encontraréis. Ella y la Verdad se han
escondido\ldots{} y para encontrarlas, más que candiles hace falta otra
cosa.

---La Verdad y la Justicia---dijo Ansúrez,---están en el corazón de los
poderosos; pero muy escondidas adentro, debajo de pasiones y de mil
cosas malas\ldots{}

---El corazón de los poderosos---agregó Merino agarrándose a la idea del
\emph{celtíbero},---tiene dentro la Justicia y la Verdad; pero como está
tan empedernido, no hay modo de llegar a él para sacar las virtudes.
Claro que tienen que salir, porque si no, se acabaría el mundo\ldots{}

---Peor que acabarse, porque sería el Infierno---dijo Lucila,---y siendo
el mundo Infierno, nos condenamos antes de morirnos.

---Condenados los que no delinquimos.

---Condenados malos y buenos: esto no puede ser.

---La Justicia y la Verdad tienen que salir---dijo Ansúrez;---pero ya
verán ustedes cómo no salen. Cuando más, asomará una puntita de
ellas\ldots{} A menos que venga un hombre tan grande y tan sabio que sea
como redentor que nos manden del otro mundo\ldots»

Sin perder su impasibilidad más que por segundos, D. Martín expresó esta
idea: «El hombre que por la Providencia venga destinado a desatar este
nudo, ha de reunir en sí solo el mérito que tuvieron Moisés, Numa y
Augusto\ldots{} y aún es poco. Hay que agregar el mérito de Ciro,
Semíramis y Alejandro\ldots{} No sabrán ustedes quién fue Numa, ni quién
Ciro y la gran Semíramis; pero poco importa que no lo sepan\ldots»

Ansúrez y Lucila le oían con la boca abierta. «Pienso---dijo el
\emph{celtíbero},---que al hombre, remediador de los males de España, o
sea médico de esta enferma Nación, no podemos imaginarlo reuniendo en un
sujeto a todos los talentos del mundo, pues aún sería poco material para
formar el gran seso que aquí necesitamos. Imaginarlo debemos como dotado
de santidad, de un fuego divino, que no puede encender más que el
Espíritu Santo, según reza la Sacra Teología.»

---La Teología---dijo Merino con marcado desdén,---será dentro de mil
años no más que lo que es hoy la Mitología para nosotros\ldots{} ¿Sabéis
lo que es la Mitología? Dioses, semidioses y héroes, todos movidos de
las pasiones del hombre. Pues en eso concluirá la Teología\ldots{} El
que a España regenere necesitará, más que talento y más que el brillo de
la llamada santidad, de un inmenso valor\ldots{} desprecio de la vida
propia así como de las ajenas\ldots{} Ea, yo me voy\ldots» Dio dos pasos
y se paró para completar su pensamiento: «Ese valiente que necesitamos,
bien merecerá el nombre de Mesías. Él traerá la Justicia y la Paz.
¿Cómo? Dichoso el que lo vea, y puede que vosotros lo veáis\ldots{} ¡Paz
y Justicia!, amigas siempre inseparables, porque donde no hay justicia
no hay paz\ldots{} y si lo dudáis, preguntádselo a Moisés, el cual, para
hablar de estas cosas, empezaba por invocar a los cielos y la tierra:
\emph{Audite cæi quæ loquor, audeat terra eloquia oris mei.}.. Si no
sabéis latín, es lo mismo. Quiere decir: \emph{Oiga el Cielo, óigame la
tierra.»}

---Oigamos lo que se le ha traspapelado en la memoria---díjole Ansúrez
cogiéndole del manteo, cuando ya iba en retirada.---Se olvida del
negocio de los dineros que ha de prestarme. ¿Es hecho o no es hecho?» Se
embozó Merino en el manteo; y dando la media vuelta casi sin mirar al
celtíbero, o mirándole de soslayo, le dijo: «Anda y que te dé los
cuartos tu yerno, que es bastante rico\ldots» Sin añadir palabra, mirada
ni gesto, siguió su pausada marcha hacia el Portillo.

---¿Sabes lo que se me está pasando por la intención?---dijo Ansúrez a
su hija, mirando los dos al clérigo que se alejaba.---Pues coger una
piedra\ldots{} recordar mis tiempos de muchacho\ldots{} y ¡ran! darle en
la misma corona\ldots{} ahora que se quita el sombrero\ldots{}

---Déjele, déjele\ldots{} que bien se ve lo perverso que es---replicó
Lucila,---y la poca o ninguna substancia que de él puede sacarse\ldots{}
Habla de traer la Verdad, y él que la tiene en el cuerpo ¿por qué no la
echa fuera?\ldots{} Vámonos de aquí\ldots{} Yo estoy mala\ldots{} no sé
lo que tengo\ldots{} Miedo, repugnancia\ldots{} ¿Por dónde vamos a casa?
¿Está muy lejos?

---Menos de lo que tú crees. Metámonos por el Portillo de las Vistillas,
que está dos pasos de aquí, y en un periquete subiremos hasta San
Francisco.»

Así lo hicieron. Lucila respiró con desahogo del alma al entrar en su
casa. «En este rincón humilde---se decía,---nunca, nunca, después que se
fue Tomín, me ha pasado nada desagradable. Personas y cosas, todo aquí
es bueno, y todo se sonríe al verme. Hasta los animales del corral, que
en aquellos días tristes me enfadaban, ahora son mis amigos: los
quiero.» Resultado de esta meditación fue el propósito de asentir a
cuanto resolviesen los que llamaba \emph{suyos}, Eulogia y Antolín, y
más suyo que nadie el bonísimo D. Vicente\ldots{} Por la noche, fue
Jerónimo convidado por Antolín a cenar, y de sobremesa le dijo Halconero
que abandonara todo proyecto de poner tienda; que llevara su vejez a un
trabajo sosegado, mirando a la salud más que a las riquezas; y pues era
hombre práctico en labranza, viniérase con su hija al pueblo, y allí se
le daría plaza descansada de mayoral o mayordomo, según la ocupación que
más le cuadrase. Conmovido Ansúrez, echó por aquella boca las retahílas
de su gratitud, y Lucila una lagrimita, de las dulces, ¡ay! que no
habían de ser amargas todas las que derramaba\ldots{} Tratando de la
boda, se puso a discusión el punto de si, descartado el martes, como día
nefasto, convenía retrasar al miércoles, o anticipar al lunes. «Que lo
decida la novia»---propuso Halconero; y ella, prontamente, sin vacilar,
decidió: «Mejor antes que después.» Tal idea vista por dentro en su
fatigada mente, era de este modo: «Si ello ha de ser, mientras más
pronto, mejor. Tengo miedo a estar libre.»

Pasaron el domingo en familia todos reunidos. Determinó Halconero que el
casorio se celebraría tempranito en San Justo, eligiendo esta iglesia
para complacer a su amigo, paisano y algo pariente, D. Francisco Pradel;
y aunque Lucila no veía con buenos ojos semejante elección de templo,
porque el recinto de San Justo estaba para ella plagado de tristezas, y
allí encontraría ideas suyas que deseaba perder de vista, no se atrevió
a votar en contra por no serle posible explicar las razones de su
repugnancia. Ampliando el programa, se acordó que después de la
ceremonia religiosa, y de oír misa y asistir a la función de las
Candelas, irían de gran almuerzo a casa de Botín, y de allí a Palacio a
ver la función de Corte en la Capilla Real. Esta parte del programa sí
fue rechazado por la novia en términos tan vivos que nadie se atrevió a
insistir en ello. Por nada del mundo se metería en las apreturas de
Palacio. «Total, ¿para qué? Para no ver nada.» Y pues la Reina con toda
su Corte habría de ir después a la iglesia de Atocha para la
presentación de la Princesita, mejor sería que desde la calle, en sitio
seguro o en un balcón, vieran el paso de la comitiva. Aceptada fue por
D. Vicente esta sensata proposición: también a él, por causa de no estar
nada flaco, le enfadaban las apreturas.

Las primeras luces del 2 de Febrero de 1852, día que había de ser
memorable por diferentes motivos, encontraron a Lucila despierta,
arreglándose: no le daba poca prisa Eulogia, que en madrugar dejaba por
perezoso al mismo sol. Las siete y media serían cuando vestida estaba ya
la novia; a las ocho le puso Eulogia la mantilla\ldots{} Celebrada fue
por cuantos en tal ocasión la vieron, la soberana hermosura de Lucihuela
Ansúrez. Con su trajecito negro, y en derredor del rostro pálido las
sombras del cabello fundiéndose con el nimbo obscuro de la mantilla, era
realmente una diosa del Olimpo con disfraz de española y
madrileña\ldots{} A las ocho y diez salieron\ldots{} A las ocho y media,
ya estaban en la sacristía de San Justo, y a las nueve menos minutos, la
diosa y mártir era ya, ante Dios y los hombres\ldots{}

\hypertarget{xxxiii}{%
\chapter{XXXIII}\label{xxxiii}}

\ldots esposa de Vicente Halconero, rico labrador de la Villa del Prado,
¡oh suerte, oh dicha, y admirable dictamen de la Providencia!

En la capilla de los Dolores oyeron los esposos misa rezada, que dijo D.
Martín Merino, y en verdad que necesitó Lucila de toda su voluntad para
oírla con devoción, porque entre su pensamiento y el oficiante, que al
mismo Cristo representaba, se interponían recuerdos, imágenes e ideas
que ella quería expulsar de sí para el resto de sus días. Siempre que el
adusto sacerdote al pueblo se volvía para decirnos que el Señor está con
todos, con el pueblo en fin, la recién casada bajaba los ojos\ldots{} En
una de estas, no los bajó tan a tiempo que dejara de ver la brillante
mirada del clérigo riojano que le decía: «Sé la historia\ldots{}
¿Quieres que te la cuente?\ldots» Cuando le vio partir para la
sacristía, Lucila daba vueltas en su cerebro a esta idea: «¡Vaya con mis
locuras! En todo pensará este pobre señor menos en mí y en Domiciana.»
Empezó luego la función de las Candelas, en la que vieron también a Don
Martín de asistente al culto, con sobrepelliz. Creyó Lucila que desde el
presbiterio la miraba el maldito cura\ldots{} mas no era para decirle
que sabía la historia, sino para repetir la terrible frase de otro día:
«Domiciana merecía la muerte. Zanguanga, ¿por qué no la aseguraste
bien?»

Terminada la función, vieron salir a Don Martín llevándose, como es
costumbre en tal día, la vela que había ostentado en la función. Pasó
junto al matrimonio sin saludar a Lucila ni a nadie, seco, ceñudo, con
una cara y gesto propiamente aterradores. Ansúrez se fue a él y le dijo:
«D. Martín, salude a los amigos, que por el maldito dinero no hemos de
indisponernos los que bien nos estimamos.» Y Merino: «¿Estáis de
bodorrio? Ahora iréis de comistraje.» Y Ansúrez: «Si quiere participar,
tendrá la presidencia.» Y Merino, en la cuerda más baja de la sequedad
amarga y del satánico desdén: «Que les aproveche\ldots{} Yo me voy a mi
casa\ldots{} Cada cual a lo suyo.»

Superior almuerzo les dio el amigo Botín. Ansúrez, que en aquel caso
venturoso veía la mejor ocasión y estímulo para su hablar bien hilado y
nutrido de ideas graves, les divirtió con amenos discursos. Contenta
estaba Lucila, viéndose rodeada de tanto cariño y respeto, y sintiéndose
a tan considerable altura en la escala social, que desde allí volvía los
ojos hacia su antiguo ser y apenas lo vislumbraba. Un trozo de su
existencia se iba quedando atrás, como siglo que muere para dejar a otro
siglo el puesto del tiempo. En la poquita Historia que le había enseñado
su maestro (que también con buenas tragaderas al banquete asistía), los
siglos eran diferentes unos de otros, y cada cual tenía su cariz,
carácter y mote singulares. Se heredaban y se sucedían, como cuando
muere el Rey y se corona Rey nuevo. Pues de este modo entendía Cigüela
que se le iba un pasado triste, y se le entronizaba un porvenir
risueño\ldots{} Consta en las crónicas de estos acontecimientos que
después de una larga sobremesa en que los plácemes en prosa y verso
halagaron los oídos y el alma de la hija de Ansúrez, vieron todos que la
ocasión llegaba de tomar sitio en la calle Mayor para ver el Cortejo
Real; y abandonados los manteles, llenos de migas de pan, de huesos de
aceitunas y de manchas de vino, el profesor de Lucila, hombre de luces y
un poquito pedante, tomó la delantera diciendo: «Vamos a ver pasar la
Historia de España.»

Buscando sitio donde pudieran ver bien, con retirada segura, se fueron a
la Plazuela de San Miguel, y aunque allí había ya gran muchedumbre de
mirones formando apretadas filas detrás del cordón de tropa, hicieron
cuña, metiéndose entre la masa, hasta llegar a donde tocar podían las
mochilas de los soldados\ldots{} Pasó tiempo, más tiempo del que en el
popular programa ponía límites a la paciencia, y la Historia de España
no pasaba. La hoja del inmenso libro no quería volverse. El pueblo, no
pudiendo ver la página nueva, se divertía inventándola\ldots{} Por toda
la masa corría un rumor de inquietud, de fastidio, rumor también de
conjeturas\ldots{}

Dadas y bien dadas las dos, y transcurridos después de la hora larga
serie de fugaces minutos, la impaciencia llegó a su colmo, y las
conjeturas tomaban giros disparatados. De improviso, a todo lo largo de
la carrera pasó una ráfaga\ldots{} Venía de la Plaza de Oriente, doblaba
la esquina de la Almudena y hacia la Puerta del Sol seguía, moviendo
todos los ánimos\ldots{} Las cabezas se volvían de un lado para otro, se
agrietaba la masa, se descomponían grupos para formar grupos nuevos, y
hasta la disciplinada fila de tropa osciló y se quebró en algunas
partes. ¿Qué ocurría? La ráfaga pasó silbando, y en cada trinca de
personas dejaba suposiciones absurdas. Se movió un gran oleaje, en
preguntas: «¿Qué pasa?\ldots{} ¿Qué ha pasado?\ldots{} ¿Verdad que pasa
algo?» Y con este oleaje chocaba otro de indecisas y turbadas
respuestas: «Nada: que al Rey le ha dado un síncope\ldots{} Nada: que la
Reina se ha puesto mala\ldots{} Nada: que ya no bajan a Atocha\ldots»

Nueva ráfaga, más vibrante, con sordo ruido de tormentas, de
estremecimientos del aire. El pueblo echaba chispas\ldots{} La masa se
resquebrajaba, buscando espacio para disolverse y correr; con su
tremenda expansión rompía el enfilado rigor de la carrera, como el agua
hinchada rompe sus cauces. En segundos corría la ráfaga enormes
distancias, y a su paso los miles de almas se daban y quitaban su
estupor, para transmitirlo con inaudita velocidad\ldots{} La afirmación,
la duda, la negación, el \emph{Dicen}, el \emph{¿Qué?}, el \emph{No
puede ser}, corrían como el restallar de un temporal de granizo.

---¡Que han matado\ldots{} a\ldots{} la Reina!---exclamó Halconero
volviéndose asmático del estupor, de la pena, de la indignación.---..
Imposible\ldots{} No lo creo.

En aquel punto, un hombre, que parecía de policía, soltaba tembloroso
una breve arenga en el círculo de gente que le rodeaba: «Señores,
calma\ldots{} no ha sido nada. Matarla no; no han matado a Su
Majestad\ldots{} Ha sido intento, como decimos, conato\ldots{} Herida
leve de Su Majestad\ldots»

Y un teniente, no lejos de allí, también arengaba: «Señores,
orden\ldots{} ha sido un sacerdote loco, un infame cura\ldots{}
Orden\ldots»

---Ha sido un cura, un cura\ldots---dijo la voz de la Historia corriendo
por toda la masa y encarnándose en ella.---Con un cuchillo\ldots{} ha
sido un cura, un cura\ldots{}

---¡D. Martín!---exclamó Lucila horrorizada llevándose las manos a la
cabeza; y el agudo \emph{celtíbero} repitió con firme acento: «¡D.
Martín!»

En medio de la llamarada de ardientes comentarios que la noticia levantó
en el grupo de su familia y amigos, echó Lucila con satisfactorio
convencimiento este combustible: «Aún no se sabe la verdad\ldots{}
Esperemos\ldots{} El cuchillo no iba contra la Reina, sino contra
Domiciana\ldots{} ¡A saber\ldots! ¡Muerta Domiciana! ¡Justicia al fin!»

Descuajada la muchedumbre, se desmenuzó en puñados de gente que querían
correr hacia Palacio. Era la gente mucha, estrechos los caminos. Al
rugido del pueblo se mezclaba el son de tambores y cornetas de la tropa
deshaciendo la formación. El \emph{¡Viva la Reina!} era un bramido
continuo, que prolongándose en las bocas, hacía vibrar el aire y
retemblar el suelo\ldots{} Y en tanto, el profesor de Lucila, hombre
agudo y un poco zahorí, aplacaba la curiosidad de su discípula y del
buen Halconero, asegurándoles que la narración del atentado y los
pormenores del castigo del infame cura se verían en las \emph{Memorias},
felizmente ahora continuadas, del simpático Fajardo-Beramendi.

\flushright{Madrid, Febrero-Marzo de 1903.}

~

\bigskip
\bigskip
\begin{center}
\textsc{fin de los duendes de la camarilla}
\end{center}

\end{document}
