\PassOptionsToPackage{unicode=true}{hyperref} % options for packages loaded elsewhere
\PassOptionsToPackage{hyphens}{url}
%
\documentclass[oneside,12pt,spanish,]{extbook} % cjns1989 - 27112019 - added the oneside option: so that the text jumps left & right when reading on a tablet/ereader
\usepackage{lmodern}
\usepackage{amssymb,amsmath}
\usepackage{ifxetex,ifluatex}
\usepackage{fixltx2e} % provides \textsubscript
\ifnum 0\ifxetex 1\fi\ifluatex 1\fi=0 % if pdftex
  \usepackage[T1]{fontenc}
  \usepackage[utf8]{inputenc}
  \usepackage{textcomp} % provides euro and other symbols
\else % if luatex or xelatex
  \usepackage{unicode-math}
  \defaultfontfeatures{Ligatures=TeX,Scale=MatchLowercase}
%   \setmainfont[]{EBGaramond-Regular}
    \setmainfont[Numbers={OldStyle,Proportional}]{EBGaramond-Regular}      % cjns1989 - 20191129 - old style numbers 
\fi
% use upquote if available, for straight quotes in verbatim environments
\IfFileExists{upquote.sty}{\usepackage{upquote}}{}
% use microtype if available
\IfFileExists{microtype.sty}{%
\usepackage[]{microtype}
\UseMicrotypeSet[protrusion]{basicmath} % disable protrusion for tt fonts
}{}
\usepackage{hyperref}
\hypersetup{
            pdftitle={O'DONNELL},
            pdfauthor={Benito Pérez Galdós},
            pdfborder={0 0 0},
            breaklinks=true}
\urlstyle{same}  % don't use monospace font for urls
\usepackage[papersize={4.80 in, 6.40  in},left=.5 in,right=.5 in]{geometry}
\setlength{\emergencystretch}{3em}  % prevent overfull lines
\providecommand{\tightlist}{%
  \setlength{\itemsep}{0pt}\setlength{\parskip}{0pt}}
\setcounter{secnumdepth}{0}

% set default figure placement to htbp
\makeatletter
\def\fps@figure{htbp}
\makeatother

\usepackage{ragged2e}
\usepackage{epigraph}
\renewcommand{\textflush}{flushepinormal}

\usepackage{indentfirst}

\usepackage{fancyhdr}
\pagestyle{fancy}
\fancyhf{}
\fancyhead[R]{\thepage}
\renewcommand{\headrulewidth}{0pt}
\usepackage{quoting}
\usepackage{ragged2e}

\newlength\mylen
\settowidth\mylen{...................}

\usepackage{stackengine}
\usepackage{graphicx}
\def\asterism{\par\vspace{1em}{\centering\scalebox{.9}{%
  \stackon[-0.6pt]{\bfseries*~*}{\bfseries*}}\par}\vspace{.8em}\par}

 \usepackage{titlesec}
 \titleformat{\chapter}[display]
  {\normalfont\bfseries\filcenter}{}{0pt}{\Large}
 \titleformat{\section}[display]
  {\normalfont\bfseries\filcenter}{}{0pt}{\Large}
 \titleformat{\subsection}[display]
  {\normalfont\bfseries\filcenter}{}{0pt}{\Large}

\setcounter{secnumdepth}{1}
\ifnum 0\ifxetex 1\fi\ifluatex 1\fi=0 % if pdftex
  \usepackage[shorthands=off,main=spanish]{babel}
\else
  % load polyglossia as late as possible as it *could* call bidi if RTL lang (e.g. Hebrew or Arabic)
%   \usepackage{polyglossia}
%   \setmainlanguage[]{spanish}
%   \usepackage[french]{babel} % cjns1989 - 1.43 version of polyglossia on this system does not allow disabling the autospacing feature
\fi

\title{O'DONNELL}
\author{Benito Pérez Galdós}
\date{}

\begin{document}
\maketitle

\hypertarget{i}{%
\chapter{I}\label{i}}

El nombre de \emph{O'Donnell} al frente de este libro significa el coto
de tiempo que corresponde a los hechos y personas aquí representados.
Solemos designar las cosas históricas, o con el mote de su propia
síntesis psicológica, o con la divisa de su abolengo, esto es, el nombre
de quien trajo el estado social y político que a tales personas y cosas
dio fisonomía y color. Fue O'Donnell una época, como lo fueron antes y
después Espartero y Prim, y como estos, sus ideas crearon diversos
hechos públicos, y sus actos engendraron infinidad de manifestaciones
particulares, que amasadas y conglomeradas adquieren en la sucesión de
los días carácter de unidad histórica. O'Donnell es uno de estos que
acotan muchedumbres, poniendo su marca de hierro a grandes manadas de
hombres\ldots{} y no entendáis por esto las masas populares, que rebaños
hay de gente de levita, con fabuloso número de cabezas, obedientes al
rabadán que los conduce a los prados de abundante hierba. O'Donnell es
el rótulo de uno de los libros más extensos en que escribió sus apuntes
del pasado siglo la esclarecida jamona doña Clío de Apolo, señora de
circunstancias que se pasa la vida escudriñando las ajenas, para sacar
de entre el montón de verdades que no pueden decirse, las poquitas que
resisten el aire libre, y con ellas conjeturas razonables y mentiras de
adobado rostro. Lleva Clío consigo, en un gran puchero, el colorete de
la verosimilitud, y con pincel o brocha va dando sus toques allí donde
son necesarios.

Pues cuenta esta buena señora que el día 23 de Julio de aquel año (aún
estamos en el 54) salía de la cerería de Paredes, calle de Toledo, el
enfático patricio don Mariano Centurión ostentando con ufanía el
sombrero de copa que estrenaba: era una prenda reluciente, de las
dimensiones más atrevidas en altura y extensión de alas que la moda
permitía, y en el pensamiento del buen señor tomaba su persona, con tan
airoso chapitel, una dignidad extraordinaria y una representación
pública que atraía las miradas y el respeto de las gentes. A dos pasos
de la cerería se tropezaron y reconocieron Centurión y un ciudadano
importante, Telesforo del Portillo, que también estrenaba sombrero, si
bien aquel cilindro no era tan augusto como el otro, sino artículo de
ocasión adquirido en el Rastro y sometido a un planchado enérgico. Se
saludaron, y Centurión entabló un vivo diálogo con su amigo, conocido
entre el vulgo por el apodo de \emph{Sebo}. No ha transmitido la
Historia los términos precisos de la conversación, limitándose a
consignar que ambos patricios se habían encontrado en lastimosa
divergencia en aquellas revueltas, por figurar don Mariano en la
\emph{Junta de salvación, armamento y defensa} que funcionó en la casa
del señor Sevillano, y \emph{Sebo}, en la que se denominó \emph{Junta
del cuartel del Sur}. La primera se componía de hombres templados y de
peso; en la segunda entraron los jóvenes levantiscos y la turbamulta
demagógica.

Según dijeron los dos respetables ciudadanos, las trapisondas entre
ambas asambleas dilataron más de lo preciso las anheladas paces entre
pueblo y tropa, y dieron tiempo a que asomara su hocico espeluznante
\emph{el monstruo de la anarquía}. Pero al fin, \emph{la salud pública}
se impuso, y las Juntas llegaron a una positiva concordia, gracias
\emph{al patriotismo del Trono}, que se inclinó del lado de la Libertad
llamando a Espartero. Sostuvo Centurión que ya teníamos Gobierno liberal
\emph{en principio}, y que era cuestión de días el determinar qué
hombres habían de formarlo. \emph{Sebo} los designó sin recelo de
equivocarse, nombrando las figuras más culminantes del \emph{elemento
progresista}. Espartero y O'Donnell entrarían en el nuevo Gobierno, y
los hombres civiles serían los que más sufrieron en los once años, y
probaron su entereza política con largos ayunos. Aseguró don Mariano que
su colocación en Estado dependía de que ocupase aquella poltrona el
señor Luján, y que si le daban a escoger, tomaría la plaza de jefe en la
Sección de \emph{Obra pía de Jerusalén}, que ya disfrutó por pocos días
en otra época. \emph{Sebo} se daba por empleado en Penales, si ponían en
Gobernación a don Manolo Becerra o a don Ángel de los Ríos. Esto era
dudoso, según Centurión, porque si bien ambos jóvenes descollaban por
sus talentos y acendrado patriotismo, no tenían el peso y madurez
convenientes para gobernar.

Sobre si eran aptos o no los tales, discutían Portillo y don Mariano,
cuando atrajo su atención un gran tumulto y escandaloso ruido de gente
que por la calle abajo venía. Ya estaba próxima la delantera de la que
parecía procesión, y el centro de ella, algo que descollaba sobre la
multitud como figuras del Santo Entierro conducidas en hombros,
desembocaba por el arco de la Plaza Mayor. Antes de que los dos
patricios se dieran cuenta de lo que aquello era, rodearon a \emph{Sebo}
unas hembras (no sé si tres o cuatro) con toda la traza de mozas del
partido, desgarradotas, peinadas con extremado artificio, alguna de
ellas reluciente de pintura en el marchito rostro. «Véanle,
véanle---dijeron.---Desde la Plazuela de los Mostenses lo
\emph{train}\ldots{} El \emph{Chico} es el que viene en andas, y el
\emph{Cano} a pie\ldots{} Que los afusilen, que les den garrote\ldots{}
que paguen las que han hecho.» Y Centurión, con grave acento,
arrimándose a la pared por no ser visto de la canalla delantera,
pronunció estas sesudas palabras: «¡Justicia del pueblo, mala
justicia!\ldots{} ¿Y don Evaristo no se ha enterado de esta
barbaridad?\ldots{} Decid, grandes púas, ¿vosotras habéis venido con
esta procesión infernal? ¿Pasasteis por Gobernación? ¿No estaba allí don
Evaristo? ¿Cómo habéis recorrido medio Madrid, o Madrid entero, sin que
algunos patriotas honrados os cortaran el paso, ralea vil?

---Cállese la boca, don Marianote---dijo la más bonita de ellas, la
menos ajada,---que pueden oírle, y corre peligro de que le chafen el
\emph{baúl} nuevo.

---Rafaela Hermosilla---replicó Centurión alardeando de entereza,---un
patriota honrado, un hombre de principios, no teme las coces de la plebe
indocta\ldots{} Pero arrimémonos a esta puerta para no dar lugar a
cuestiones, o metámonos en la cerería de Paredes, que será lo más
seguro\ldots{} \emph{Sebo}\ldots{} ¿dónde se ha ido \emph{Sebo}?»

Llamado por su amigo, se retiró también al arrimo de las casas el
ex-policía, seguido de otra de las pájaras. Lívido y tembloroso, no
podía disimular el terror que la plebeya justicia le causaba, y era en
verdad espectáculo que el más animoso no podía presenciar sin miedo y
compasión grandes. Detrás de la caterva que rompía marcha gritando, iban
dos hombres montados en jamelgos: vestían blusa de dril y cubrían su
cabeza con chambergo ladeado sobre una oreja, esgrimiendo sendos
chafarotes o sables. Seguíales un bigardón con un palo, del que pendía
un retrato al óleo, sin marco, acribillado ya de los golpes que por el
camino, en las paradas de la procesión, le daban con sus sables los dos
jinetes, en demostración de justicia popular. Al portador del retrato
seguía otro gandul con trazas de matarife, en mangas de camisa, esta
manchada de sangre, llevando una pértiga de la cual pendía muerto y sin
plumas un gallo colgado por el pescuezo. Tras este iba un hombre a pie,
empujado más que conducido por un grupo de bárbaros, también con aspecto
de matachines. Seguían las angarillas cargadas por cuatro, de lo más
soez entre tan soez patulea; las angarillas sostenían un colchón, en el
cual iba el infeliz Chico sentado, de medio cuerpo abajo cubierto con
las propias sábanas de su cama, de medio cuerpo arriba con un camisón
blanco, en la cabeza un gorro colorado puntiagudo, que le daba aspecto
de figura burlesca. Con un abanico se daba aire, pasándolo a menudo de
una mano a otra, y miraba con rostro sereno a la multitud que le
escarnecía, al gentío que en balcones y puertas se asomaba curioso y
espantado. Arrimándose a las angarillas todo lo que podía, iba la mujer
de Chico con una taza en la mano, revolviendo con un palo el contenido
de ella, que según decían era chocolate. Parecía loca: su rostro echaba
fuego; su cabeza recién peinada y con alta peineta, conservaba la
disposición de las matas de pelo armadas artísticamente. Digo que
parecía loca, porque el menear el palo dentro de la taza vacía era como
un movimiento instintivo, inconsciente, efecto de la máquina muscular
disparada y sin gobierno. Enrojeciendo más a cada grito, decía:
«¡Nacionales, no le matéis! ¡No le matéis, nacionales!»

Pasó todo este bestial aparato de venganza y muerte, que observaron
desde la cerería don Mariano y Telesforo, las dos muchachas de mal vivir
y don Gabino Paredes con su hijo Ezequiel. Rafaela Hermosilla, que había
visto el asalto de la casa de Chico, lo contó de esta manera:
\emph{«Lleguemos}; íbamos con idea de arrastrarle, que es la muerte que
merece\ldots{} El pillo del Cano nos dijo: \emph{Atrás, populacho}; y no
había acabado de decirlo, cuando Perico el lañador le echó mano al
pescuezo, y yo y otras le arañamos toda la cara. Daba risa\ldots{}
Después le amarraron bien amarradico con cordeles que prestó un mozo de
cuerda\ldots{} y \emph{entremos}; subimos dando patadas y gritos, y nos
\emph{desparramemos} por las salas llenas de muebles y cuadros\ldots{}
«A quemarlo todo.» Esta fue la voz. ¡Qué risa! Pero Alonso Pintado soltó
cuatro tacos, gritando: \emph{Pena de muerte al ladrón}\ldots{} Salió
esa gran tarasca llorando, acabadita de peinar, ¡qué risa!\ldots{} ¡Y
cómo chillaba la muy escandalosa! Que su marido estaba enfermo en cama
con la podagra, y que le había pedido el chocolate\ldots{} «Señoras y
caballeros---nos dijo Alonso Pintado subido en una silla,---venimos a
hacer justicia, no a faltar a \emph{naide}. Al ladrón \emph{busquemos},
no a las riquezas que robó\ldots{} No toquéis a estos \emph{faralanes y
cornucopios}\ldots{} Por el tirano de los pobres venimos. Justicia en
él, señoras y caballeros; pero sin alborotar, para que no digan\ldots»
Yo, me lo pueden creer\ldots{} no alboroté, ni cogí nada de lo que hay
en aquellas cámaras tan lujosas, donde el \emph{gachó} va metiendo lo
que rapiña\ldots{} Pues Alonso Pintado, \emph{Matacandiles},
\emph{Pucheta}, la \emph{Rosa} y la \emph{Pelos}, don Jeremías,
\emph{Chanflas}, \emph{Meneos}, \emph{la Bastiana} y otras y otros de
que no me acuerdo, empujaron puertas, rompieron fechaduras y se colaron
hasta la alcoba en donde estaba acostado el Chico\ldots{} No le valió a
su mujer decir que estaba imposibilitado, y que le iba a llevar el
chocolate. ¡Qué risa!\ldots{} «Espérense; no le maten\ldots{} me ha
pedido el chocolate\ldots{} está en ayunas\ldots{} se muere\ldots{} se
morirá solo\ldots{} Matarle, no.» Esto decía la tía \emph{Panderetona},
que no es mujer de él por la Iglesia, sino arrimada, como una, pongo el
caso, ¡qué risa!\ldots{} Total: que en vilo le levantaron, con colchón y
todo, y de una escalera hicieron las angarillas\ldots{} Pepe
\emph{Meneos} trajo un gallo, le retorció el pescuezo, y desplumándolo
delante del Chico, le echaba las plumas, diciéndole, dice: «Lo que hago
con este gallo haremos contigo, so ladronazo.» ¡Qué risa! Luego salió la
procesión que habéis visto\ldots{} Pues venía con \emph{muchismo} orden,
como se dice\ldots{} Pucheta mandaba, que es hombre que sabe del orden y
tal\ldots»

Oyendo estas referencias, Centurión tenía un nudo en su garganta, y no
acertaba ni a protestar contra el salvajismo del pueblo. «¡Ignominia,
barbarie! ---exclamaba dando palmadas en el mostrador.---La Libertad no
es eso, cojondrios, no es eso.» Y \emph{Sebo}, que en su consternación
se había calado el sombrero nuevo hasta las orejas, habló así: «Dime,
\emph{Rafa}, ¿iba \emph{Pucheta} en el \emph{entierro}? Porque yo no he
podido distinguir caras, del gran susto y sobrecogimiento que me entró
al ver lo que vi. Al tiempo que se me aflojaba el vientre, se me nublaba
la vista.

---Pues sí que iba---dijo Centurión.---El jinete de la derecha, el que
vimos por la parte de acá, era \emph{Pucheta}, con blusa de dril y un
plumacho en el sombrero. ¡En qué manos está la Libertad, cojondrios! Y
al lado de \emph{Pucheta}, a la parte de adentro, iba la Generosa
Hermosilla, hermana de esta buena pieza\ldots{}

---Mi hermana---dijo \emph{Rafa}---no se separa de \emph{Pucheta}: es la
que le mete en la cabeza el orden\ldots{} ¡Qué risa con ella! A todas
horas le canta la lección: «\emph{Pucheta}, orden\ldots{} Ándate con
orden, hijo.» Mi hermana iba al lado de él, terciado el manto, muy bien
peinadita, con un pompón en la peineta\ldots{}

---Tu hermana y tú---afirmó Centurión furioso,---sois unas solemnes
castañas pilongas, que después de llevar a los hombres al vicio, les
predicáis el orden. ¡Vaya un escarnio! Orden vosotras, que nunca
supisteis con qué se come eso. ¿Qué principios tenéis ni qué dogmas
profesáis para saber lo que es el orden? ¡Idos al infierno con cien mil
pares de cojondrios! Tu hermana Jenara y tú, \emph{Rafa} maldita, habéis
escandalizado en todo Madrid, después de escandalizar en las calles del
Humilladero, Irlandeses y Mediodía Grande\ldots{} A vuestro honrado
padre, el bueno de Hermosilla, le pusisteis a punto de morir de
vergüenza\ldots{} No os quitaréis nunca de encima el apodo de \emph{las
Zorreras}, que os aplicaron por ser hijas de un fabricante de zorros,
que también hace plumeros\ldots{} Vete, vete; sigue los pasos de tu
hermana, al lado de \emph{Pucheta}, de \emph{Meneos}, o de otro de esos
matarifes que deshonran la Libertad\ldots{} No te entretengas aquí,
entre gentes honradas y hombres de principios\ldots{} Corre, y verás
cómo ahorcan o fusilan o despachurran al desgraciado Chico.»

\hypertarget{ii}{%
\chapter{II}\label{ii}}

Echose a reír la moza con el airado discurso de Centurión, y llegándose
al dueño de la cerería, don Gabino Paredes, que arrobado la contemplaba,
los codos en el mostrador, el rostro en las palmas de las manos, le
dijo: «¿Verdad, Gabinico, que tú no me echas de tu casa?» Y el cerero,
revolviendo algo en su boca, completamente desdentada, le contestó: «Ni
yo ni el amigo Centurión te arrojamos de esta humilde tienda. Ha sido un
decir, rica: no te enfades\ldots{} Y para que veas que me acuerdo de ti,
toma este caramelito\ldots» Cuando los sacaba del hondo bolsillo de su
chaqueta, alargó Centurión la mano diciendo: «Deme otro a mí, don
Gabino, que del berrinche que he cogido con esta tragedia, se me ha
secado la boca.» Hizo el cerero ronda de caramelos, dando la mayor parte
a \emph{Rafa} y a su compañera, que con \emph{Sebo} platicaba, y
chuparon todos, refrescando sus secos paladares. La segunda pájara, de
apodo \emph{Jumos}, mujerona en el ocaso de la juventud, con restos
manidos de un gallardo tipo de majeza, tomó la palabra en contra del
señor de Centurión, desarrollando sus argumentos con razones no mal
concertadas: «Pues si el pueblo no hace la justiciada en ese capataz de
los guindillas, ¿quién la hará?\ldots{} ¡contra con Dios! ¿El Gobierno
nuevo que venga le había de castigar? Y \emph{vostedes} los patriotas
nuevos, ¿qué serían más que lameplatos del Chico? Hala con él, y
reviéntenle para que no haga más maldades\ldots{} Él comía con el
Gobierno, comía con el ladronicio\ldots{} ¿Que robaban a \emph{vostedes}
el reloj? Pues para recobrarlo, no tenían más que abocarse con don
Francisco, que devolvía la prenda por un tanto más cuanto, según el por
qué de la persona\ldots{} Alhajas muchas pasaron de sus dueños a los
ladrones, y de los ladrones a sus dueños, todo con su
\emph{porsupuesto}, menos cuando las alhajas le gustaban a Chico, que
tan fresco se quedaba con ellas. De sus ganancias prestaba dinero, a
seis reales por duro al mes, mediando el portero Mendas y uno de la
calle de la Palma, con trazas de clérigo, que le llaman don Galo, y
también \emph{el Chato de Pinto}, por ser de Pinto mismamente\ldots{}

---Invenciones de la plebe---dijo Centurión menos fiero que
antes;---malquerencia de los que Chico perseguía por revoltosos.

---Algo habrá de eso---observó en tonos de templanza el gran
\emph{Sebo},---sin que deje de ser verdad lo que cuenta esta
\emph{Jumos}. Testigos hay de que el pobre don Francisco no jugaba con
limpieza.

---Jugaba con cartas señaladas---afirmó la mujerona,---y era el primer
puerco del mundo. El Gobierno le pagaba para defender a cada hijo de
vecino, y él ¿qué hacía? cobrar el barato al vecino y al Gobierno y al
\emph{Sulsucorda}. A todos engañaba, y no era fiel más que con la
Cristina y su marido, el de Tarancón, porque estos, cuando los Ministros
estaban hartos de Chico y querían darle la puntera, sacaban la cara por
él\ldots{} Como que Chico era el hombre de confianza de los Muñoces, y
el que estaba al quite por si venían cornadas\ldots{} que el pueblo
hacía por ellos, ¡vaya!

---Exageraciones, mujer---dijo Centurión,---y desvaríos de la pasión
popular\ldots{} Algún día se hará la luz, y la Historia pondrá la verdad
en su punto.

---Historias ya tenemos---prosiguió la \emph{Jumos}:---pídaselas a don
José de Zaragoza y a don Melchor Ordóñez, que por saber bien de historia
han querido limpiarle el comedero a don Francisco Chico. Pero no podían,
que la Cristina le echaba un capote, y Chico tan fresco, se reía, se
reía, con aquella cara de sayón\ldots{} Pues el muy marrajo, para dar
gusto al Gobierno, se cebaba en los que caían en su mano, por mor del
conspirar y de la política. El que era masón y andaba en algún enredo
para echar proclamas o escribir contra la Reina, ya podía encomendarse a
Dios. A nadie metía en la cárcel sin darle antes un pie de paliza para
hacerle confesar la verdad, o mentiras a gusto de él, con las que se
abría camino para prender a otros, y abarrotar la cárcel\ldots{} A un
primo mío, Simón Angosto, zapatero en un portal de la calle de la
Lechuga, que los lunes solía ponerse a medios pelos y cantaba coplas en
la calle, con música del \emph{izno} de Espartero y letra que él sacaba
de su cabeza, le cogió una noche saliendo de la casa de Tepa, y tal le
pusieron el cuerpo de cardenales, que \emph{gomitó} el alma a los dos
días.

---No fue así, Pepa \emph{Jumos}, no fue así---dijo \emph{Sebo}
gravemente, poniendo en su acento todo el respeto a la verdad
histórica.---A Simón Angosto se le hicieron los cardenales y se le
aplicó de firme el vergajo, porque anduvo en aquellas
trapisondas\ldots{} bien me acuerdo\ldots{} cuando mataron a
Fulgosio\ldots{} Se le encontró una carta con garabatos masónicos y
razones en cifra que parecían\ldots{} así como un conato de atentado
contra Narváez\ldots{}

---Para conatos tú, reladronazo---replicó la mala mujer, roja de
ira.---¿Qué es conato?

---Es intento de delito, delito frustrado\ldots{}

---Me \emph{fustro} yo en ti y en el \emph{conato} de tu madre. Sales a
la defensa de Chico, porque tú eras de los del vergajo, que deslomaban
al infeliz que cogían. Tal eres tú como el otro, que ahora paga sus
\emph{conatas} y \emph{fustratas}\ldots{} y con él te debíamos llevar.

---Yo no estoy con él, ni estuve---dijo Telesforo palideciendo.---Pepa
\emph{Jumos}, mira lo que hablas: ten en cuenta que yo, si cumplí mi
deber en la Seguridad, luego me dio asco de aquel oficio, y me pasé al
partido de los señores generales de Vicálvaro, que nos han traído la
Libertad, verbigracia, la Justicia.

---Justicia contra ti, \emph{arrastrao}---dijo Rafaela Hermosilla,
terciando en la conversación.---Ándate con tiento, \emph{Sebito}, y no
pintes el diablo en la pared, que como te huela el pueblo, hará contigo
un \emph{conato}.

---El amigo Telesforo---indicó Centurión extendiendo una mano protectora
sobre el renegado de la Policía,---es hombre de principios, que jamás
atropelló al pueblo soberano. Si alguna vez impuso castigos, fue mirando
por el Ornato Público, que llamamos también Policía Urbana.»

Saltó al oír esto la \emph{Jumos} con briosa protesta, diciendo:
«¡Buenas \emph{ornatas} públicas nos dé Dios! Lo que hacía este tuno era
bailarle el agua a don Francisco Chico, y andar siempre agarrado a los
faldones de su \emph{levosa}\ldots{}

Y esto no me lo ha contado nadie, sino que lo han visto estos ojos,
porque yo, aunque no soy vieja, ni lo quiera Dios, he visto mucho mundo,
y pillería mucha; tanto, que de ver canalladas sin fin, cada lunes y
cada martes, paréceme que soy vieja, lo cual que no lo soy, sino que lo
viejo es el mundo y las malas partidas que se ven en él\ldots{} Pues el
día aquel, ya van para seis años, en que el pobre zapatero de la calle
de Toledo le tiró un ladrillo a don Francisco Chico, desde el primer
piso bajando del cielo, yo estaba en la acera de enfrente hablando con
mi comadre la Venancia, que tenía cacharrería donde hoy están los
talabarteros\ldots{} Pues como allí estaba una servidora, todo lo vi, y
nadie me lo cuenta\ldots{} Y digo que el ladrillo no fue ladrillo, sino
un pedazo de cascote, y que no le cayó a don Francisco en la
\emph{canoa}, como dijeron y mintieron, sino que se \emph{espolvoró} en
el aire, y sólo unas motas fueron a dar en el hombro del Chico, y otras
salpicaron al que le acompañaba, que era el señor de \emph{Sebo}, aquí
presente. Atrévase a decirme que esto no es verdad\ldots{} Se calla y
rezonga, como los perros\ldots{} Un perro fue entonces. ¿Quién subió
como un cohete a la casa de donde tiraron las \emph{mundicias}? ¿Quién
bajó en seguida trayendo al zapaterín cogido por el pescuezo?
¿Quién\ldots?

---Cierto que fuí yo\ldots{} no puedo negarlo---dijo \emph{Sebo} con
trémula voz.---Pero como ha declarado el señor Centurión, lo hice por
Ornato Público, o por \emph{Policía} y \emph{Buen Gobierno}, que era el
\emph{Ramo} en que yo servía entonces. Y dice el bando de 1839, en su
art. 5.º: «Los que arrojen a la calle basuras, cascos de loza o ceniza
de braseros, pagarán cuarenta reales de multa, sin perjuicio de las
penas en que incurran en el caso de causar daños a los
transeúntes\ldots»

---¿Y por qué bando fusilasteis al zapaterito\ldots?

---Eso no es cuenta mía, ni tuve nada que ver. ¿Que el hombre fuera
masón, y guardara papeles que le comprometían, y una estampa indecente
de Fernando VII con orejas de burro\ldots{} es acaso culpa mía?

---¿Y de que por eso le fusilaran---agregó Centurión,---es culpa de
nadie\ldots{} más que del sicario de Narváez?

---Sobre pintarle al Rey orejas que no eran las suyas---dijo \emph{Sebo}
defendiéndose con timidez,---el susodicho dibujó un letrero saliendo de
la boca de \emph{Narizotas}, que a la letra decía: «Marchemos, y yo el
primero, por la senda borrical de la reacción.»

El cerero don Gabino Paredes cortó con su desentonada voz la disputa
histórica, sosteniendo que ninguno de los señores presentes tenía culpa
de las barbaridades del 48. Todo ello se hizo para guarecernos de las
revoluciones y tempestades que venían de Francia, de Italia y de
Hungría, y cerrarle la puerta al maldito Socialismo. No se entendían las
graves razones del buen Paredes, porque, deshabitada absolutamente de
huesos su boca, el aire conductor de la voz hacía dentro de aquella
caverna extraños pitidos, gorjeos y cambios de tono, que quitaban a las
palabras su verdadero sentido, o las dejaban escapar con silbos
desapacibles. Más claramente habló Centurión, despachando a las dos
pajarracas con estas desahogadas expresiones: «Seguid vuestro camino,
tú, \emph{Zorrera}, y tú, \emph{Jumos}, y no alternéis con hombres de
principios, que os compadecen, pero no os escuchan. Id a ver cómo mata
el pueblo a esos desgraciados, y si llegáis a tiempo, sed piadosas, ya
que no podéis ser honradas, y decid al pueblo que no envilezca su
patriotismo con el asesinato. Influye tú, \emph{Rafa}, con tu hermana la
otra \emph{Zorrera}, para que a su vez interceda con ese \emph{Pucheta}
condenado, a ver si el hombre se ablanda, y evita ese crimen de leso
Pueblo\ldots{} Vosotras, \emph{zorreras}, a quienes debo llamar, para
daros más categoría, \emph{plumeros}, que algo más vale el plumero que
el zorro, y si lo dudáis preguntádselo a vuestro padre; vosotras, digo,
y tú, \emph{Jumos}, id hacia abajo en seguimiento de la chusma, y haced
una buena obra. Sois lo que sois; pero no malas de mal corazón\ldots{}
creo que me entendéis\ldots{} El diablo que lleváis dentro vuélvase
compasivo, o escóndase para que un ángel se meta en vosotras por un
ratito no más. Salvad a esos infelices, y después seguid escandalizando
por el mundo; practicad la liviandad pública, hasta que os llegue la
hora del arrepentimiento\ldots{} Idos, dejadnos en paz.»

Risas desvergonzadas provocó en ambas cortesanas del pueblo el agrio
sermoncillo de Centurión, endulzado por cariños del cerero, que rasgando
toda su boca hasta las orejas, y ahuecándola y haciendo buches con las
palabras, decía: «\emph{Zorrerita}, no te vendas tan cara. Ven mañana y
te daré almendras de Alcalá.» Presente estaba Ezequiel Paredes, arrimado
a su padre, y el pobre chico miraba con encandilados ojos a las dos
culebronas, sin expresar horror del infamante oficio de las tales.
«\emph{Zequilete}---dijo la Pepa \emph{Jumos} acariciando con sus dedos
ensortijados la barbilla del mancebo,---¡qué callado estás!\ldots{} Ven
con nosotras, \emph{cara e cielo.»} De estas confianzas protestó don
Gabino cogiendo al chico por un brazo: «No, no; dejadle, que es todavía
una criatura. No os entiende\ldots{}

---Sois libros que el pobrecito no sabe leer---dijo Centurión.

---Deletrea---indicó \emph{Sebo} jovial;---pero más vale que no pase del
\emph{a b c}. En fin, idos al matadero y no volváis por aquí.

---Lo que sentimos---declaró la \emph{Jumos}---es no llevarte por
delante, para que los fusiles hagan boca con tu cabeza pindonguera.» Y
la otra: «Con Dios, abuelo y \emph{Zequiel}\ldots{} Don Mariano,
conservarse\ldots{} \emph{Sebo}, no ande hoy por esta calle, no sea que
lo derritan.»

Diciendo esto la \emph{Zorrera}, se oyeron tiros lejanos. Don Gabino se
santiguó; Centurión soltó un terno; se echaron a la calle despavoridas
\emph{las del partido}, ansiosas de alcanzar algo de la función, y
\emph{Sebo} humilló su cabeza y encogió su cuerpo como si quisiese
meterse debajo del mostrador. En esto pasaba por la calle tropel de
gente con aspecto medroso. Salió Ezequiel a la puerta, y oyó decir: «En
la Fuentecilla les han despachado.» Oyéndolo, redobló Centurión sus
apóstrofes declamatorios, y proclamó la supremacía de los principios
sobre las pasiones. \emph{Sebo} callaba, y como su amigo le propusiera
emprender la retirada hacia los barrios del centro, se fue derecho a la
trastienda murmurando con ahilada voz: «También yo principios\ldots{}
hombre de principios\ldots{} hombre de bien\ldots{} ¿Pero cómo salgo a
la calle?\ldots{} ¡Me ven, se fijan en mí\ldots! Amigo Paredes,
escóndame en su casa hasta la noche\ldots» Esto dijo acariciando el
sombrero, que en la mano llevaba, e internándose por el pasillo. Tras
él, Centurión trataba de aliviarle el miedo: «No hay cuidado,
Telesforo\ldots{} Yendo conmigo, podrá usted salir\ldots{} Mi persona es
la mejor fianza\ldots{}

---¡Fíese usted de fianzas!\ldots{} ¿Fianzas contra el pueblo? ¡Ni de la
Virgen!\ldots{} Aquí me quedo.»

Retirose don Mariano, dejándole al cuidado de Ezequiel y de Tomás, el
encargado de la cerería, pues don Gabino, completamente chocho ya del
agobio de sus años, no hacía más que acopiar caramelos para obsequio de
toda mujer que entraba en la tienda por cirios, agraciándola con su
sonrisa lela, sin distinguir señoras de sirvientas, ni honradas de
públicas, que para él todo ser con faldas, salvo los curas, era lo
mismo. Cuando a don Mariano en la puerta despedía, vieron pasar al
General San Miguel, con su séquito de militares y patriotas, a trote
largo calle abajo. «A buenas horas, mangas verdes,» dijo Centurión; y
don Gabino daba toda la cuerda de sonrisa a su boca sin dientes,
persignándose como cuando habían oído los tiros. Entraron luego dos
señoras, hija y madre, ambas muy guapas, a comprar cerillos y mariposas,
y como venían asustadas del tumulto de la calle, no se detuvieron más
que el tiempo preciso para su negocio, y tomar los caramelos con que las
obsequió baboso y risueño el bueno de don Gabino. Este las despidió
enjuagándose la boca con palabras que ellas no entendieron, haciendo la
señal de la cruz y besándose los dedos. «Angelote---dijo a Ezequiel
apenas se quedaron solos,---¿cuándo aprenderás a no ser huraño con las
señoras? A tu edad yo no las dejaba salir de la tienda sin decirles
alguna palabra fina y con aquel\ldots{} Eres un ganso, y en cuanto ves a
una mujer, se te alarga el hocico, te pones colorado y no sabes decir
más que \emph{mu}, \emph{mu}, como un buey que no ha salido de la
dehesa\ldots{} ¡Y que no son poco lindas la madre y la hija!\ldots{} No
sabría uno con cuál quedarse si le dieran una de las dos\ldots{} La
madre es hija de un señor de Pez que tuvo la contrata de conducción de
caudales. Casó con el coronel Villaescusa, que ahora irá para
General\ldots{} Conozco bien a esta familia\ldots{} El coronel y su
hermana Mercedes, casada con Leovigildo Rodríguez, son primos carnales
de nuestro amigo Centurión, que acaba de salir de aquí\ldots{} Pues la
niña es una flor\ldots{} ¿no te parece que es una flor?\ldots{} Se llama
Teresita. Ya viste con qué ojos tan tiernos me miraba, y qué cuchufletas
tan graciosas me decía, ji, ji, ji\ldots{} Y tú, grandísimo pavo, te
quedaste lelo como un poste cuando la madre te pasó los dedos por la
cara y te dijo: «\emph{Zequiel}, qué guapín eres.»

\hypertarget{iii}{%
\chapter{III}\label{iii}}

No vuelve a mentar Clío a nuestro buen Centurión hasta la página en que
nos cuenta la entrada de Espartero en Madrid, por la Puerta de Alcalá,
entre un gentío loco de entusiasmo, que le bendecía, le aclamaba y le
llevaba medio en vilo con coche y todo. A pie iba Centurión junto a la
rueda trasera, puesta la mano en la plegada capota, dando al viento, con
toda la violencia de su voz estentórea, los gloriosos nombres de
Luchana, Peñacerrada y Guardamino, emprendiéndola luego con la Libertad,
la Soberanía del Pueblo y otras invocaciones infalibles para enardecer a
las multitudes. El caudillo de los patriotas, cuando los vaivenes del
océano de personas detenían el coche en que navegaba, se ponía en pie,
sacaba y esgrimía la espada vencedora, y soltando aquella voz tonante,
sugestiva, de brutal elocuencia, con que tantas veces arrastró soldados
y plebe, lanzaba conceptos de una oquedad retumbante, como los ecos del
trueno, con los cuales a la turbamulta enloquecía y la llevaba hasta el
delirio\ldots{} Reaparece luego Centurión cuando Espartero y O'Donnell
se dieron el célebre abrazo en el balcón de la casa donde fue a vivir el
primero, plazuela del Conde de Miranda. Detrás de los dos Generales
invictos se veía, entre otros paniaguados, la imagen escueta de
Centurión, derramando de sus ojos la ternura, de sus labios una alegría
filial, dando a entender que allí estaba él para defender a su ídolo de
cualquier asechanza. Cuenta la Musa que el buen señor se constituyó en
mosca de don Baldomero, acosándole sin piedad a todas horas, hasta que
su pegajosa insistencia logró del caudillo el anhelado nombramiento en
la Obra Pía de Jerusalén.

Daba gusto ver la \emph{Gaceta} de aquellos días, como risueña matrona,
alta de pechos, exuberante de sangre y de leche, repartiendo mercedes,
destinos, recompensas, que eran el pan, la honra y la alegría para todos
los españoles, o para una parte de tan gran familia. Capitanes
generales, dos; Tenientes generales, siete, y por este estilo avances de
carrera en todas las jerarquías militares, sin exceptuar a los soldados
rasos, aliviados de dos años de servicio. ¡Pues en lo civil no digamos!
La \emph{Gaceta}, con ser tan frescachona y de libras, no podía con el
gran cuerno de Amaltea que llevaba en sus hombros, del cual iba sacando
credenciales y arrojándolas sobre innumerables pretendientes, que se
alzaban sobre las puntas de los pies y alargaban los brazos para
alcanzar más pronto la felicidad. La \emph{Gaceta} reía, reía siempre, y
a todos consolaba, orgullosa de su papel de Providencia en aquella
venturosa ocasión. Y no era menor su gozo cuando prometía
bienaventuranzas sin fin para el país en general, anunciando proyectos,
y enseñando las longanizas con que debían ser atados los perros en los
años futuros. La \emph{Gaceta} tenía rasgos de locura en su semblante
iluminado por un gozo parecido a la embriaguez. Diríase que había bebido
más de la cuenta en los festines revolucionarios, o que padecía el
delirio de grandezas, dolencia muy extendida en los pueblos dados al
ensueño, y que fácilmente se transmite de las almas a las letras de
molde.

Era de ver en aquella temporadita el súbito nacimiento de innumerables
personas a la vida elegante o del bien vestir. Se dice que nacían,
porque al mudar de la noche a la mañana sus levitas astrosas y sus
anticuados pantalones por prendas nuevecitas, creyérase que salían de la
nada. La ropa cambiaba los seres, y resultaba que eran tan nuevos como
las vestiduras los hombres vestidos. El cesante soltaba sus andrajos, y
mientras hacían negocio los sastres y sombreros, acopiaban los
mercaderes del Rastro género viejo en mediano uso. Y a su vez, pasaban
otros de empleados a cesantes por ley de turno revolucionario, que no
pacífico. Alguna vez había de tocar el ayuno a los orgullosos moderados,
aunque fuera menester arrancarles de las mesas con cuchillo, como a las
lapas de la roca.

El observador indiferente a estas mudanzas entreteníase viendo pasar
regocijados seres desde la región obscura a la luminosa, entonando
canciones anacreónticas o epitalámicas, y sombras que iban silenciosas
desde la claridad a las tinieblas. Al gran \emph{Sebo} le veíamos salir
de su casa después de comer, bien apañadito de ropa, llevando entre dos
dedos de la mano derecha un puro escogido de cuatro cuartos, que fumaba
despacio, procurando que no se le cayera la ceniza, y a su oficina de
Gobernación se encaminaba, saludando con benévola gravedad a los amigos
que le salían al paso. Poco trecho recorría Centurión desde su casa de
la calle de los Autores hasta Palacio, bajando por la Almudena y
atravesando el arco de la Armería, sin encontrar amigos o comilitones
que en tan desamparado lugar le saliesen al encuentro para pedirle
noticias de la cosa pública. Mejor era así, pues se había impuesto
absoluta discreción\ldots{} Atento a la dignidad más que a vanas pompas,
limitose, en la cuestión indumentaria, a lo preciso y estrictamente
decoroso, y pensó en mejorar de vivienda, cambiando el mísero cuarto de
la Cava de San Miguel por una holgada habitación en la calle de los
Autores, casa vieja, pero de anchura y espacio alegre, con vista
espléndida al Campo del Moro. Allí se instaló por gusto suyo y
principalmente por el de su mujer, que como andaluza hipaba por las
casas grandes bañadas de aire y luz. El primer cuidado de la mudanza fue
la conducción de tiestos. Los dos balcones de la Cava de San Miguel
remedaban los pensiles de Babilonia; diversidad de plantas en macetas,
cajones y pucheros, entretenían a doña Celia, que tal era el nombre de
la señora, ocupándole horas de la mañana y de la tarde en diversas
faenas de jardinería y horticultura. Los cuatro balcones de la calle de
los Autores, abiertos al Oeste, dieron amplitud y mayor campo a su dulce
manía, y lanzándose a la arboricultura, con el primer dinero que le dio
Centurión para estos esparcimientos compró una higuera, un aromo y un
manzano, que con la arbustería formaban, a las horas de calor, una
deliciosa espesura de regalada sombra.

En su nueva casa, visitado de pocos y buenos amigos, veía Centurión
pasar la Historia, no sin tropiezos y vaivenes en su marcha, a veces
precipitada, a veces lenta; vio la salida de la Reina Cristina, de
tapujo, pues los demagogos querían, si no matarla, darle una pita
horrorosa, homenaje a su impopularidad; vio cómo se establecía la
Milicia Nacional, de lo que sacaron fabulosas ganancias los fabricantes
y almacenistas de paños por la enorme confección de uniformes; vio y
leyó el Manifiesto que hubo de largar Cristina desde Portugal,
quejándose de que la Nación la había tratado como a una mala suegra, y
augurando calamidades sin fin; vio entrar en España huésped tan molesto
como el cólera morbo; asistió a la apertura de las nuevas Cortes, que
eran, para no perder la costumbre, Constituyentes y todo; vio a Pacheco
salir del Ministerio de Estado, sustituyéndole don Claudio Antón de
Luzuriaga, lo que no le supo mal, por ser este un buen amigo que le
estimaba de veras; y lamentó, en fin, los motines con que el loco año 54
se despedía, desórdenes provocados en unos pueblos por la inquieta
Milicia, en otros por ella reprimidos.

A medida que prosperaban los árboles en los balcones de doña Celia,
Centurión se iba sintiendo más inclinado al orden, y más deseoso de la
estabilidad política, tomando en esto ejemplo del reino vegetal y de la
Madre Naturaleza, que con lenta obra arraiga las plantas, protege la
savia y asegura flores y frutos. La moderación se posesionaba de su
alma, y garantida por el empleo la vida física, se sentía lleno de la
dulce y fácil paciencia, que es la virtud de los hartos. Quería que
todos los españoles fuesen lo mismo, y renegaba de los motines, no
viendo en ellos más que una insana comezón, conatos de nacional
suicidio. ¡Cuánto mejor y más práctico que estuviéramos tranquilos los
españoles, disfrutando de las libertades conquistadas, y esperando en
calma la Constitución nueva que iban a darnos los conspicuos!\ldots{}
Pensando en esto todo el día, por las noches solía tener el hombre
pesadillas angustiosas; soñaba que Espartero y O'Donnell se tiraban al
fin los trastos a la cabeza, como decían los profetas callejeros, y
venía el temido rompimiento. Con imaginario peso sobre el buche y tórax,
don Mariano no podía respirar. Era una barra de plomo, y la barra de
plomo era la espada de Lucena, vencedora de la de Luchana. O'Donnell
triunfante reía como un diablo de los infiernos irlandeses, con glacial
cinismo, entreteniéndose en limpiar los comederos de todos los
esparteristas habidos y por haber. Despertaba el hombre sobresaltado,
clamando: «¡Ay, que me ahogo!\ldots{} ¡Quítate\ldots{} O'Donnell!\ldots»
Y aun despierto persistía la sensación de horrible pesadumbre sobre el
pecho. A los gritos del buen señor se despabilaba doña Celia, y
sacudiendo a su esposo por el brazo de este que tenía más próximo, le
decía: «Mariano, ¿qué es eso?\ldots{} ¿El dolor en el vacío\ldots{} la
opresión en el pecho?

---Sí, mujer\ldots{} es este O'Donnell\ldots{}

---¿Qué O'Donnell?

---La opresión, hija. La llamo así porque\ldots{} ya te lo expliqué la
otra noche\ldots{} Dame friegas\ldots{} aquí\ldots{} la opresión se me
va pasando, pero el miedo no\ldots{} Veo la gran calamidad del Reino, el
rifirrafe entre estos dos caballeros. El uno tira para la Libertad, el
otro para el Orden\ldots{} Adiós, revolución bendita; adiós, principios;
adiós, España\ldots{} Y todo para que vuelva el perro
moderantismo\ldots{} el atizador de estas discordias\ldots{} por la
cuenta que le tiene\ldots{} Vaya, no friegues más. Duérmete, pobrecilla.

---Cuando me despertaron tus ayes---dijo doña Celia requiriendo el
rebozo,---soñaba yo que uno de mis jacintos echaba un tallo muy largo,
muy largo\ldots{}

---¡Muy largo!---murmuró don Mariano cerrando los ojos y arrugando su
faz.---Ese largo es O'Donnell.

---¿Sueñas otra vez?

---No sueño\ldots{} pienso.

---No pienses\ldots{} Oye, Mariano: treinta y dos capullos tiene mi
rosal pitiminí\ldots{} y ya han echado la primera flor los ranúnculos de
Irlanda.

---¡Irlanda\ldots{} O'Donnell!

---¿Qué tiene que ver?\ldots{} Duerme\ldots{} yo también\ldots{} Me
levantaré temprano para limpiar los rosales, sembrar más extrañas, y
recortar el garzoto blanco.

---Blanco es O'Donnell\ldots{} el hombre blanco y frío\ldots{} Duerme,
Celia. Yo no puedo dormir\ldots{} Pronto amanece. Oigo cantar
gallos\ldots{} su grito dice: «¡O'Donnell!\ldots»

\hypertarget{iv}{%
\chapter{IV}\label{iv}}

Modesto y sencillo en sus costumbres, Centurión recibía en su casa, las
más de las noches, a familias amigas, unidas algunas con lazos de
parentesco a doña Celia o a don Mariano. Eran personas de trato
corriente, de posición holgada y obscura dentro de los escalafones
burocráticos. Con gente de alto viso se trataban poco, no siendo en
visitas de etiqueta, y aunque sus relaciones habían llegado a ser
extensas en el curso del 54 al 55, no cultivaban más que las de cordial
intimidad o las de parentesco. Asiduos eran el comandante Nicasio Pulpis
y su mujer Rosita Palomo, sobrina de doña Celia; Leovigildo Rodríguez,
con su esposa Mercedes, hermana del Coronel Villaescusa, primo de
Centurión, y María Luisa Milagro de Cavallieri, hermana de la Marquesa
de Villares de Tajo \emph{(Eufrasia)}. También frecuentaban la tertulia
el comandante don Baldomero Galán y su señora, doña Salomé Ulibarri
\emph{(Saloma la Navarra)}; Paco Bringas, compañero de Centurión en la
oficina de Obra Pía; don Segundo Cuadrado y don Aniceto Navascués,
empleados en Hacienda. De personas con título, no iba más que la
Marquesa de San Blas, camarista jubilada, y de personas pudientes, las
culminantes en aquella modesta sociedad eran don Gregorio Fajardo y su
esposa Segismunda Rodríguez, que del 48 al 54 habían engrosado
fabulosamente su fortuna. La Coronela Villaescusa y su linda hija
Teresa, tenían rachas de puntualidad o abstención en la tertulia.
Durante un mes iban todas las noches, y luego estaban seis o siete
semanas sin aportar por allí. Razón le sobraba a doña Celia, que
calificó de alocadas o locas de remate a la madre y la hija.

Redichas y despabiladas eran María Luisa del Milagro, Rosita Palomo y la
vetusta y mal retocada Marquesa de San Blas; espléndida y maciza
hermosura bien conservada en sus cuarenta años, tarda en el hablar y muy
limitada en sus ideas, era Salomé Ulibarri de Galán; despuntaba
Segismunda por su tiesura y por el tono que se daba, no perdiendo
ocasión de aludir incidental y discretamente a sus improvisadas
riquezas. Más de una noche, cuando traía la actualidad asunto político,
digno de ser tratado por todos los españoles que entendían de estas
cosas, los caballeros, dejando a las señoras que a sus anchas picotearan
sobre modas o sobre lo caro que estaba todo en la plaza, se agrupaban en
un rincón de la sala. Era este como abreviatura del Congreso, donde todo
problema se ventilaba, entendiendo por ventilación que saliesen al aire
opiniones poco diversas en el fondo, y que aleteando estuviesen entre
bocas y oídos, volviendo al fin cada opinión a su palomar. Tratose allí
por todo lo alto y todo lo bajo el gravísimo asunto de la
Desamortización Civil y Eclesiástica, votada por las Cortes en Abril.
¿Por qué se obstinaba la Reina en no dar su sanción a esta ley?
Desdichado papel hacían O'Donnell y Espartero cabalgando un día y otro
en el tren de Aranjuez, con la Ley en la cartera, y volviéndose a Madrid
cacareando y sin firma. Leovigildo Rodríguez y Aniceto Navascués no se
mordían la lengua para sacar a la vergüenza pública, con sátira cruel,
las cosas de Palacio. A la colada salieron el Nuncio, Sor Patrocinio, y
clérigos palaciegos o gentiles hombres aclerigados.

Por aquellos días, empeñado el Gobierno en que Su Majestad sancionara la
ley, y obstinada Isabel en negar su firma, vieron los españoles una
prodigiosa intervención del cielo en nuestra política. Fue que un
venerado Cristo que recibía culto en una de las más importantes iglesias
del Reino, se afligió grandemente de que los pícaros gobernantes
quisieran vender los bienes de Mano Muerta. Del gran sofoco y amargura
que a Nuestro Señor causaban aquellas impiedades, rompió su divino
cuerpo en sudor copioso de sangre. Aquí del asombro y pánico de toda la
beatería de ambos sexos, que vio en el milagro sudorífico una tremenda
conminación. ¡Lucidos estaban Espartero y O'Donnell y los que a
entrambos ayudaron! ¡Vaya, que traernos una Revolución, y prometer con
ella mayor cultura, libertades, bienestar y progresos, para salir luego
con que sudaban los Cristos! La vergüenza sí que debió de encender los
rostros de O'Donnell y Espartero, hasta brotar la sangre por los poros.
Por débiles y majagranzas que fuesen nuestros caudillos políticos,
incapaces de poner a un mismo temple la voluntad y las ideas, la
ignominia era en aquel caso tan grande, que hubieron de acordarse de su
condición de hombres y de la confianza que había puesto en ellos un país
tratado casi siempre como manada de carneros. El de Luchana y el de
Lucena se apretaron un poco los pantalones. Y la Reina firmó, y Sor
Patrocinio y unos cuantos capellanes y palaciegos salieron desterrados,
con viento fresco; al buen Cristo se le curaron, por mano de santo, la
fuerte calentura y angustiosos sudores que sufría, y no volvió a padecer
tan molesto achaque.

Siempre que de este y otros asuntos semejantes se trataba en la tertulia
de Centurión, decía este que el mayor flaco de nuestros caudillos era
que no se atacaban bien los pantalones, y solían andar por el Gobierno y
por las salas palatinas sin la necesaria tirantez del cinturón que ciñe
aquella prenda de vestir. Hombres que en los campos de batalla se
cinchaban hasta reventar, y arrostraban impávidos los mayores peligros
con los calzones bien puestos, en cuanto se ponían a gobernar,
aflojábanse de cintura y desmayaban de riñones sólo con ver alguna
compungida faz de persona religiosa, llamárase Nuncio o simple monjita
seráfica. La vista de un cirio les turbaba, y cualquier exorcismo de
varón ultramontano les hacía temblar. Pero, en fin, aquella vez se
habían portado bien y merecían alabanzas de todo buen español.
Conservárales Dios en tan buen temple de voluntad y con los pantalones
bien sujetos.

Cuando desmayaban los temas políticos de actualidad, pasaban el rato los
amigos de Centurión entreteniditos con los burocráticos temas: se
trabajaba de firme en tal oficina; el jefe de la otra era un vago que
permitía hacer a cada cual lo que le viniera en gana. En Rentas
Estancadas les había tocado un Director que era una fiera; la Caja de
Depósitos disfruta cinco días de estero y desestero, y el Director
obsequiaba con dulces a los empleados el día del santo de la señora y de
las niñas\ldots{} Luego invertían largo tiempo en designar sueldos
efectivos o sueldos probables, y la conversación era un tejido de frases
como estas: «El trabajo que me ha costado llegar a doce mil, sólo Dios
lo sabe\ldots» «Heme aquí estancado en los catorce mil, y ya tenemos a
Mínguez, con sus manos lavadas, digo, sucias, encaramado en veinte
mil\ldots» «Vean ustedes a Pepito Iznardi, con el cascarón pegado
todavía en semejante parte, disfrutando ya sus diez mil, que yo no pude
obtener hasta pasados los treinta años\ldots» «Madoz me ha dado palabra
solemne de que tendré pronto diez y ocho mil\ldots» «Pues yo, si entra
en Hacienda, como parece, mi amigo don Juan Bruil, los veinte mil no hay
quien me los quite.»

El ser empleado, aun con sueldos tan para poco, creaba posición: los
favorecidos por aquel Comunismo en forma burocrática, especie de
imitación de la Providencia, eran, en su mayoría, personas bien educadas
que, por espíritu de clase y por tradicional costumbre, vestían bien,
gozaban de general estimación, y alternaban con los ricos por su casa.
Fácilmente podían procurarse una o más novias los chicos que lograban
pescar credencial de ocho mil en sus floridos años, y se consideraba
buen partido casar a la hija predilecta con un mozo de catorce mil, que
gastaba guantes, y cubría su cabeza, bien peinada, con enorme
\emph{canoa} de fieltro. Llegaba a una ciudad de corto vecindario un
caballerete con destino de ocho mil en Administración Subalterna, y sólo
con presentarse, volvía locas a todas las señoritas de la población. En
tropel se asomaban a las ventanas para verle pasar, y fácilmente
introducido en las mejores casas, tomaba el papel de \emph{lion}
irresistible, a poco desenfado y cháchara que gastase. Vestía bien,
usaba guantes, y un sombrero de copa que eclipsaba con su brillo a todos
los del pueblo. En este, que era de los de pesca, se daba un tono
inaudito: de Madrid contaba maravillas y rarezas que embobaban a sus
oyentes; en la Corte tenía innumerables relaciones; conocía marquesas,
camaristas, actores célebres, caballerizos y gentiles hombres de
Palacio\ldots{} Era sobrino de un tío que cobraba cuarenta mil. Todo
esto y su agradable figurilla bastaban para que se le estimase, y para
que su alianza con cualquier familia de la localidad se considerara como
una bendición.

Tales desproporciones entre la pobreza y el falso brillo de una posición
burocrática, componían el tejido fundamental de aquella sociedad.
Jóvenes existían que cautivaban con su fino trato y el relumbrón de una
superficial cultura, y, no obstante, ganaban menos dinero que un
limpia-botas de la calle de Sevilla. Pelagatos mil existían, bien
apañados de ropa y modales, que se alimentaban tan mal como los
aguadores; pero no tenían ahorrillos que llevar a su tierra. Verdad que
también había gran desproporción entre la prestancia social de muchos y
su valer intelectual. Licenciados en Derecho, con ocho o diez mil
reales, que entendían algo de literatura corriente, y poseían la fácil
ciencia política que está en boca de todo el mundo, ignoraban la
situación del istmo de Suez, y por qué caminos van las aguas del
Manzanares a Lisboa\ldots{} De lo que sí estaban bien enterados todos
los españoles de levita, y muchos de chaqueta, era de la guerra de
Oriente, o de Sebastopol, como ordinariamente se la nombraba. Los
caballeros ilustrados, las señoras y señoritas, hasta las chiquillas,
hablaban de la torre de Malakoff con familiar llaneza. El Malakoff y los
\emph{offes}, los \emph{owskys} y los \emph{witches} de las
terminaciones rusas servían para dar mayor picante a los conceptos y
giros burlescos. Ejemplo: «¿Qué pasa, amigo \emph{Centurionowsky}, para
que esté usted tan triste? ¿Se confirman los temores de que
\emph{Leopoldowitch} le juegue la mala partida al gran
\emph{Baldomeroff?»}

En el círculo de señoras, solía dar doña Celia conferencias sobre el
cultivo de plantas de balcón, en que era consumada profesora; y cuando
no había en la tertulia solteras inocentes, o que lo parecían, las
casadas machuchas y las viudas curtidas tiraban de tijeras, y cortaban y
rajaban de lo lindo en las reputaciones de damas de alta clase, pasando
revista a los líos y trapicheos que habían venido a corromper la
sociedad. ¡Bonita moral teníamos, y cómo andaban la familia y la
religión! La sal de estos paliques era el designar por sus nombres a
tantas pecadoras aristocráticas, y hacer de sus debilidades una cruel
estadística. Véase la muestra: «La Villaverdeja está con Pepe Armada; la
Sonseca con el chico mayor de Gravelina; a pares, o por docenas, tiene
sus líos la de Campofresco; la Cardeña habla con Manolo Montiel, y con
Jacinto Pulgar la de Tordesillas\ldots» Poniendo su vasta erudición en
esta crónica del escándalo la veterana Marquesa de San Blas, el seco
rostro se le iluminaba debajo de la pintura que lo cubría. Ella sabía
más que sus oyentes; conocía todo el personal, y no había liviandad ni
capricho que se le escapase\ldots{} Muchas le revelaban sus secretos, y
los de otras, ella los descubría con sólo husmear el ambiente. Óiganla:
«Ya riñó la Navalcarazo con Jacinto Uclés; ahora está con Pepe Armada:
se lo quitó a la Villaverdeja, que se ha vengado contando las historias
de la Navalcarazo y enseñando cartas de ella que se procuró no sabemos
cómo. La Belvis de la Jara, que presumía de virtud, anda en enredos con
el más joven de los coroneles, Mariano Castañar, y la Monteorgaz se
consuela de la muerte del chico de Yébenes, entendiéndose con Guillermo
Aransis. La aristocracia de nuevo cuño no quiere quedarse atrás en este
juego, y ahí tienen ustedes a la Villares de Tajo aproximándose a ese
andaluz pomposo, Álvarez Guisando\ldots» Y por aquí seguía. Las honradas
señoras pobres, o poco menos, que se cebaban con voraces picos en esta
comidilla, no maldecían la inmoralidad sin poner en su reprobación algo
de indulgencia, atribuyendo al buen vivir tales desvaríos. En la
estrechez de su criterio, creían que la mayor desgracia de las altas
pecadoras era el ser ricas. Doña Celia resumía diciendo: «Véase lo que
trae tener tanto barro a mano, y criarse en la abundancia, madre de la
ociosidad y abuela de los vicios.»

Por la mente de Centurión pasaban, sin alterar la normalidad de su
existencia, los sucesos que habían de ser históricos. Casi en los días
en que el Cristo sudaba, murió en Trieste don Carlos María Isidro; mas
con la muerte del santón del carlismo, no murió su causa: en Cataluña y
el Maestrazgo aparecieron las tan acreditadas partidas, y casi tanto
como de rusos y turcos, se habló de Tristany, Boquica y Comas\ldots{}
Sin que ningún Cristo sudara, se retiró el Nuncio, y las relaciones con
el Vaticano quedaron rotas. El verano arrojó sus ardores sobre la
política. Una calurosa mañana de Julio, hallándose doña Celia en la
dulce faena de regar sus tiestos y limpiar las plantas, entró don
Segundo Cuadrado con la noticia de que habían estallado escandalosos
motines en Cataluña y Valladolid, y de que O'Donnell, al saberlo, se
tiró de los pelos y maldijo a la Milicia Nacional como raíz y fundamento
de la brutal anarquía. Don Mariano, que en mangas de camisa se paseaba
por la habitación, dijo pestes del irlandés, y le acusó de estar
confabulado con los \emph{eternos enemigos} de la Libertad, para
producir alborotos y desacreditar la Revolución. «\emph{Maquiavelismo},
puro \emph{Maquiavelismo}, querido Cuadrado. Ese hombre frío nos
perderá. Acuérdese usted de lo que anuncio\ldots» Se puso a temblar, y
daba diente con diente, como si le atacara pulmonía fulminante. Trájole
su mujer un chaquetón, que él endilgó presuroso, diciendo: «En medio de
un ambiente abrasador, yo tirito\ldots{} ¡Oh frío inmenso! Es O'Donnell
que pasa.»

\hypertarget{v}{%
\chapter{V}\label{v}}

Linda era como un ángel Teresita Villaescusa, como un ángel a quien Dios
permitiese abandonar la solemne seriedad del Cielo, adoptando el reír
humano. Porque, según los doctores en belleza, la de Teresita
Villaescusa no habría sido tan completa sin aquel soberano don de
sonrisa y risa que le iluminaba el rostro y le descubría el alma. A
todos encantaba su gracia ingenua, y la amistad y el amor se le rendían.
La tez de un blanco alabastrino, el cabello castaño, los ojos negros:
¿verdad que no pudo idear combinación más bonita el Supremo Autor de
toda hermosura? Pues espérense un poco, y verán qué obra maestra. Hizo
el cuerpo de proporciones discretas, ni largo ni corto; el talle
esbelto, los andares graciosos, el pecho lozano. Y decían admiradores de
Teresa que se había esmerado en la dentadura, haciéndola tan bella y
nítida como la de los ángeles, que ni ríen ni comen. La inocente niña,
que en sociedad era el hechizo de cuantos la trataban, en la intimidad
doméstica se encerraba, según decía su madre Manolita Pez, en una
gravedad taciturna, con tendencias a la melancolía. Educada en completa
libertad de lecturas, Teresa devoraba cuantos libros caían en sus manos,
novelas sentimentales o de enredo, obras picarescas, y hasta tratados
ascéticos y místicos. A los diez y ocho años gustaba menos del teatro
que de la iglesia, y se dejaba llevar de sus tías, las señoras de Pez, a
novenas y triduos. Daba cuenta de los ritos y solemnidades eclesiásticas
a que asistía, bien compuesta y acicalada con sencilla elegancia, pues
el gusto de arreglarse bien era otro de los dones con que quiso
agraciarla el Soberano Fabricante de toda belleza. Su apacible dulzura y
su querencia de lo espiritual, y aun su pulcritud modesta, daban motivo
a que la madre dijese: «Esta hija mía acabará por ser monja.»
Confirmábala en tal creencia el tesón con que Teresita, después de
sonreír y reír con cuantos muchachos se le acercaban, no entraba con
ninguno. Admitía bromas galantes; pero en cuanto le hablaban de
relaciones y de noviazgo, se metía en la concha de su seriedad, y
desaparecían de la vista de sus admiradores los maravillosos dientes.

El coronel don Andrés de Villaescusa, excelente militar, era hombre poco
doméstico. Pesábale el techo de su casa; ardía el suelo bajo sus pies:
las altas horas de la noche le encontraban en tertulias de cafés o
casinos. Liberal en política, lo era más aún con su mujer, a quien
dejaba en la plenitud de los derechos, sin ningún rigor en los deberes.
Las pasiones que al Coronel dominaban eran los caballos, el juego y el
continuo disputar en casinos, cafés y tertulias de hombres, llevando
siempre la contraria, embistiendo con impetuosa dialéctica los problemas
más difíciles. Menos sus obligaciones militares, todo lo dejaba por
hablar, y discutir, y defender las opiniones más apartadas del sentir
general: era la eterna oposición. En estos placeres de la charla
maniática, contrariábale un crónico padecimiento del estómago, que de
tiempo en tiempo con violencia le acometía, haciéndole atrabiliario y
por demás impertinente. Se dejaba cuidar por su esposa en la crudeza de
los accesos; pero cuando estos pasaban, volvía estúpidamente al vivir
desordenado, toda la noche en febriles disputas, comiendo mal y a
deshora, renegando del Verbo. De su matrimonio con Manolita Pez no tuvo
más sucesión que Teresa. De niña la mimaba. Viéndola mujer, no pensó más
que en librarse del cuidado que exige la doncellez, casando pronto a la
chica, que para eso nacen las hembras. «No andemos con
remilgos---decía.---Es locura esperar a que le salgan marqueses,
banqueros o accionistas de minas. El primer teniente que pase, o el
primer oficinista con diez mil, se la lleva, y a vivir.» A risa tomaba
lo del monjío, y pensaba que las tristezas de su hija en casa no eran
más que ganas de novio, y cavilación en las dificultades para
encontrarle bueno.

A fines del 55, en la tertulia de Centurión, le salió a Teresa un novio,
que parecía del agrado de ella. Era un teniente muy simpático, de la
familia de Ruiz Ochoa. Pero los sangrientos desórdenes de Valladolid
interrumpieron el tanteo de amor, porque el joven oficial salió de la
Corte con las tropas destinadas a contener aquel movimiento. Teresa, con
fría inconstancia, aceptó los obsequios de otro, Rafaelito Bueno de
Guzmán, de familia bien acomodada; pero a los tres meses de telégrafos
en el balcón y de cartitas, fue despedido el jovenzuelo, y suplantado
por un estudiante de Caminos que sabía sinfín de matemáticas y hablaba
el francés con perfección. Al matemático sucedió un poeta; al poeta, un
chico del comercio alto, Trujillo y Arnaiz; a este, un médico novel, y
un pintor, y un hijo del Marqués de Tellería, y un sobrino del
contratista de la Plaza de Toros, con poca bambolla y muchos cuartos, y
un joven filósofo medio cegato, y otro, y otro, en cáfila interminable,
peregrinación de criaturas hacia el Limbo.

Rodaba el Tiempo, rodaba la Historia, sin que Teresita encontrase novio
de que ahorcarse. Quería, sin duda, que el árbol fuese muy alto, o no
había tejido aún cuerda bastante sólida para el caso. Radiante de
belleza, y dislocando a cuantos la veían y más aún a los que la
trataban, entró la señorita en los veinte años. La Historia, en aquellos
días fecundos, traía hoy una novedad, mañana otra, menudencias del vivir
público que anunciaban sucesos grandes. Ausente el coronel Villaescusa,
que operaba en Andalucía contra milicianos desmandados, y contra otros
que se apodaban Republicanos o Socialistas; desentendida Mercedes de su
hermano, Centurión y doña Celia eran los encargados de recordar a la
niña la obligación de decidirse pronto. Ya se iba haciendo célebre por
la descarada seducción con que al paso de los novios los enganchaba, así
como por la fría displicencia con que los despedía. Esta conducta de
Teresa, que se interpretaba de muy distintos modos, era causa de que se
retrajeran muchos candidatos que venían con el mejor de los fines, y de
que otros, desairados a las primeras de cambio, hablaran pestes de ella
y de su madre y de toda la familia.

Manolita Pez, la verdad sea dicha, no se cuidaba de dar a su hija
ejemplo de seriedad ni de constancia, y en su frívola cabeza no dejaban
las ligerezas propias espacio para los sanos pensamientos que debía
consagrar a la guía y dirección de la desconcertada joven. Doña Celia
prestaba más atención a sus tiestos que al cultivo de su parentela, y
don Mariano, sobresaltado noche y día por el mal sesgo que iba tomando
la cosa pública, no tenía tranquilidad para poner mano en aquel negocio
de familia. «Déjalas, Celia---decía,---que harto tengo yo que pensar en
las cosas del procomún, y en las desdichas que vienen sobre esta pobre
patria nuestra. Si la madre es loca y la hija necia, y ninguna de las
dos sabe hacerse cargo de las realidades de la vida, ¿qué adelantaremos
con meternos a consejeros y redentores? Arréglense como quieran, y que
se las lleven los demonios.»

Tomaba las cosas el buen señor muy a pechos y era su impresionabilidad
demasiado viva. Lo que debía disgustarle, le causaba hondísima pena; lo
que para otro sería molestia o desagrado, para él era una desgracia, y
su ánimo turbado convertía las ondulaciones del terreno en montes
infranqueables. Detestaba el papel satírico llamado \emph{El Padre
Cobos}, considerándolo como la más fea manifestación de la desvergüenza
pública. Se había impuesto la obligación de no leerlo nunca, y fielmente
la cumplía. Pero no faltaba un amigo indiscreto y maleante que en la
oficina o en el café le recitase alguna cruel \emph{indirecta} del
maligno fraile, o graciosas coplas y chistes sangrientos, todo ello sin
otro fin que denigrar al vencedor de Luchana y pisotear su figura
prestigiosa. Ponía sus gritos en el cielo don Mariano, y tomaba entre
ojos para siempre al amigo que tales bromas se permitía. No era buen
español quien se recreaba con el veneno de aquel semanario y con la
suciedad asquerosa de sus burlas. Leer públicamente \emph{El Padre
Cobos} era \emph{hacer cínico alarde de moderantismo}; llevarlo en el
bolsillo, de ocultis, para leerlo a solas, era hipocresía y traición
cobarde, \emph{indigna de los hombres del Progreso}.

Los desmanes de la plebe en ciudades de Castilla, sacaban a don Mariano
de quicio. En todo ello veía \emph{la oculta mano de la reacción}
moviendo los títeres demagógicos y comunistas. ¿Qué se quería? Pues
sencillamente, desacreditar el régimen liberal, y presentarnos a
Espartero como incapaz de gobernar pacíficamente a la Nación. Los
\emph{retrógrados de todos los matices}, y los facciosos y clérigos,
andaban en este fregado, y, para engañar al pueblo y arrastrarlo a los
motines, alzaban \emph{maquiavélicamente} la bandera de \emph{La
carestía del pan}\ldots{} ¡Farsantes, politicastros de tahona, y
entendimientos sin levadura! ¡Qué tendrá que ver la hogaza con los
principios!\ldots{} «Pero, Señor---decía,---si tenemos Cortes legalmente
convocadas, que sin levantar mano se ocupan en darnos una Constitución
nueva, pues las viejas ya no sirven, ¿por qué no esperamos a que esa
nueva Constitución se remate, se sancione y promulgue, para ver cuán
lindamente nos asegura, \emph{a clavo pasado}, los principios de
Libertad, resolviendo para siempre la cuestión del pan y del queso, y de
los garbanzos de Dios?»

En el café de Platerías se reunían a media tarde, después de la oficina,
media docena de progresistones chapados y claveteados, como las
históricas arcas que en los pueblos guardan las viejas ejecutorias y los
desusados trajes. Alzaba el gallo en la reunión el buen don Mariano,
como el orador más autorizado y sesudo. Había que oírle: «Hasta los
ciegos ven ya las intenciones de O'Donnell. Con sus intrigas, ese
irlandés maldito nos pone al borde del abismo\ldots{} ¿Qué creerán que
ha inventado el tío para dar al traste con el \emph{Progreso}? Pues esa
gaita del justo medio, y de que se vaya formando un nuevo partido con
gente de la Libertad y gente de la Reacción\ldots{} o lo que es lo
mismo, que seamos \emph{progresistas retrógrados}, o \emph{despóticos
avanzados}\ldots{} ¡Vaya un pisto, señores! ¿Saben ustedes de algún
cangrejo que ande hacia adelante, o de lebreles que corran hacia
atrás\ldots? ¿Quieren decirme qué significa el habernos metido en el
Ministerio a ese jovencito burgalés? El tal es un modelo vivo de lo que,
según O'Donnell, han de ser los hombres futuros: hombres con un pie en
el \emph{Retroceso} y otro en el \emph{Adelanto}. No le niego yo el
talento a ese Alonsito Martínez, o Manolito Alonso, que a estas horas no
sé bien su nombre\ldots{} pero lo que digo: ¿tan escaso anda el Partido
de hombres graves y experimentados, que sea preciso echar mano de
criaturas recién salidas de la Universidad para que nos gobiernen?»

Y otra tarde: «¡Cómo se va realizando todo lo que dije! Ya ven ustedes:
el Olózaga nos va saliendo grilla, y aunque parece que tira contra
O'Donnell, tira contra el Duque. Uno y otro estorban a su ambición sin
límites\ldots{} ¿Y qué me dicen del Ríos Rosas, ese a quien ha dejado
tan mal sabor de boca el deslucido papel que hizo en el \emph{Ministerio
metralla}? Cuidado que el hombre tiene bilis y malas pulgas. Dicen que
es moral; pero yo sostengo que Moralidad y Reacción rabian de verse
juntas. Ya sabemos cómo estos señores del escrúpulo acaban tragándose
medio País. Ríos Rosas tira contra Espartero y la Libertad desde el
campo \emph{cangrejil}, y desde el campo del democratismo tira
Estanislao Figueras\ldots{} otro que tal\ldots{} Figueras, Fernando
Garrido y Orense quieren llevarnos a la anarquía, con esa maldita
república que no admite Trono\ldots{} ¡Como si pudiera existir la
Libertad sin Trono!\ldots{} En fin, que al Duque le tienen aburrido. Él
no dice nada; pero bien se le conoce que está más que harto de este
paisanaje, y que el mejor día se nos atufa, lo echa todo a rodar, y
adiós Libertad, adiós Trono, adiós Milicia. Despidámonos de los buenos
principios, y de la Moralidad\ldots»

Y otras tardes, allá por enero del 56 y meses sucesivos: «El nuevo
Ministerio no me disgusta, porque sale de Fomento el joven burgalés, y
entra en Gobernación Escosura. Observen ustedes que con Escosura, Santa
Cruz y Luján tenemos tres progresistas en el Gabinete; pero no son de
los \emph{puros}, pues estos se quedan, por lo visto, para vestir
milicianos, digo, imágenes. Ya no es un secreto para nadie que el
irlandés se entiende con Palacio para barrernos. En Palacio le dan la
escoba\ldots{} ¿Conque tenemos de Capitán General al \emph{general
bonito}? ¿Y ese modo de señalar qué significa? Bobalicones del Progreso,
¿no habéis reparado que todos los mandos militares están en manos de
amiguitos y compinches de O'Donnell? Ros de Olano, Director de
Artillería; Hoyos, de Infantería\ldots{} ¿Qué tal, Escosura? ¿Qué dices?
El Duque, como personificación de la lealtad y de la consecuencia,
desprecia las personalidades y se atiene a los principios\ldots{}
Espartero es Cristo; O'Donnell, Iscariote\ldots{} ¿Y Palacio?\ldots{}
Palacio es la Sinagoga.»

\hypertarget{vi}{%
\chapter{VI}\label{vi}}

Concuerdan todos los historiadores en que fue un día de Febrero del 56
cuando Teresita Villaescusa despidió a su vigésimo sexto novio,
Alejandrito Sánchez Botín, joven elegante, con buen empleo en Gracia y
Justicia, y además medianamente rico por su casa. Tan bellas cualidades
no impidieron que Teresa le diese el canuto con la fórmula más
despectiva: «Alejandrito, su figura de usted me empalaga, y su elegancia
se me sienta en la boca del estómago. Va usted por la calle mirándose en
los vidrios de los escaparates para ver cómo le cae la ropa\ldots{} y
cuando no hace esto, hace otra cosa peor, que es mirarse los pies
chiquitos que le ha dado Dios, y las botitas bien ajustadas. Ea, ni
pintado quiero ver aun hombre que gasta pies más chicos que los
míos\ldots{} ¿Que tiene usted una tía Marquesa, y en La Habana un tío
que apalea las onzas?\ldots{} Bueno: pues déles usted memorias\ldots{} y
que escriban\ldots{} ¿Que su papá le ha prometido comprarle un caballo,
y que cuando lo tenga me paseará la calle, y hará delante de este balcón
piruetas muy bonitas? Ándese con cuidado, no se le espante el animal y
se apee usted por las orejas, como aquel otro que conmigo
hablaba\ldots{} No le valió ser de Caballería\ldots{} Créame: no le
conviene andar en esos trotes. Usted a patita, pisando hormigas con ese
calzado tan mono, o en el coche de su tía la \emph{señá}
Marquesa\ldots{} Y otra cosa, Alejandrito: ¿de dónde ha sacado usted que
es elegante dejarse crecer una uña como esa que usted lleva, larga de
una pulgada, y emplear en cuidarla y limpiarla tanto tiempo y tanta
paciencia? ¡Bonito papel hace un caballero mirándose en la uña como si
fuera un espejo, y acompasando los movimientos de la mano para que no se
le rompa esa preciosidad!\ldots{} esa porquería, digo yo, por más que la
limpie con potasa y la tenga como el marfil\ldots{} Por todas estas
cosas, me es usted antipático, y si admití sus relaciones fue porque
mamá se empeñó en ello, y no me dejaba vivir\ldots{} Alejandrito por
arriba, Alejandrito por abajo, como si fuera Alejandrito la flor de la
canela\ldots{} En fin, diviértase, y cuide bien la uña, que esas cosas
tan miradas, y en las que se ponen los cinco sentidos, se rompen cuando
menos se piensa\ldots{} Agur\ldots{} y no se acuerde más de mí\ldots»

No constan las protestas que debió de hacer el galán de la uña despedido
con modos tan expeditos y desusados. Ello es que tomó la puerta, y que
Manolita Pez se lió con su hija en furioso altercado por aquella brutal
ruptura, que en un instante destruía los risueños cálculos económicos de
la egoísta mamá. Entró poco después de la disputa Centurión: iba no más
que a preguntar por su primo Villaescusa, que aquellos días había tenido
un fuerte y alarmante acceso de su mal en provincia lejana. Manuela le
tranquilizó, mostrándole una carta de Andrés de fecha reciente\ldots{}
Hablaron un poco de política, que era el hablar más común en aquel
revuelto año, y Teresa, con jovial malicia, se entretuvo en mortificar a
su tío con las bromas que más en lo vivo le lastimaban\ldots{} Cogió de
la mesa un número de \emph{El Padre Cobos}, como si cogiera unas
disciplinas, y sin hacer caso del gesto horripilante de Centurión y de
la airada voz que decía: «¡no quiero, no quiero saber!» leyó esta cruel
sátira: «Se conoce a la Moralidad progresista por el ruido de los
cencerros\ldots{} tapados.»

---¡Déjame en paz, chiquilla!\ldots{} Lee para ti esas infamias.»

Se tapaba los oídos, retirábase al otro extremo de la sala; pero tras él
iba Teresa con el papel enarbolado, y risueña, sin piedad, soltaba esta
cuchufleta: «Adoquín y camueso\ldots{} son la sal y pimienta del
Progreso.»

---Te digo que calles, o me voy de tu casa\ldots{} Una señorita bien
educada y de principios no debe repetir tales indecencias. Manuela,
llama al orden a esta niña loca.»

Pero la señora de Villaescusa encontrábase aquel día en una situación de
sobresalto y ansiedad que la incapacitaba para el conocimiento de los
hechos comunes que a su alrededor ocurrían. Distraída y con el
pensamiento lejos de su casa, no decía más que: «Niña, niña, juicio.»
Pero Teresita no hacía caso de su madre, y acosó a Centurión, que
huyendo de ella y del maldito fraile procaz, se había refugiado en el
gabinete próximo. La diabólica mozuela repetía, poniéndole música, un
dicharacho del periódico: «Muchacho, ¿qué gritan?---¡Viva la
libertad!---Pues atranca la puerta.» Poco valían tales chistes, que como
todos los del famoso papel, con menos sal que malicia, eran desahogo de
sectarios, dispuestos a cometer en doble escala los pecados políticos
que censuraban. Pero en los oídos de don Mariano sonaban a \emph{de
profundis}, y antes muriera que encontrar gracioso lo que en su criterio
inflexible era depravado y canallesco. El hombre bufaba, y le faltó poco
para poner sus dedos como garras en el blanco pescuezo de la casquivana
señorita. Esta volvió a la sala riendo a todo trapo. Su madre,
súbitamente asaltada de una idea y propósito que podían ser solución
venturosa de la crisis que agobiaba su ánimo, cogió a Teresita por un
brazo, y adelgazando la voz todo lo posible, le dijo: «Bribona, me estás
poniendo a Mariano en la peor disposición\ldots{} Yo le necesito
cordero, y con tus tonterías está el hombre como los toros
huidos\ldots{} ¡A buena parte voy!\ldots{} En vez de preparármele y
cuadrármele bien, o de entontecerle con finuras y zalamerías, me le has
puesto furioso\ldots{} En fin, quita de aquí ese maldito papelucho;
lárgate a tu cuarto, o al comedor, y déjame sola con tu tío\ldots{} con
la fiera\ldots{} No sé cómo embestirle\ldots{} no sé cómo
atacarle\ldots{}

¡Infeliz don Mariano! Aquel día se tuvo por el más infortunado de los
mortales, dejado de la mano de Dios y maldito de los hombres, porque la
niña, azotándole y escarneciéndole con \emph{El Padre Cobos}, y la
lagartona de la madre levantando sobre su cabeza el corvo sable de
cortante filo, le corrompieron los humores y le ennegrecieron el alma.
¡Vaya un día que entre las dos le daban! En vez de entrar en aquella
casa de maldición, ¿por qué, Señor, por qué no se escondió cien estados
bajo tierra? No se cuentan, por ser ya cosa sabida, los circunloquios,
epifonemas, quiebros de frase, remilgos, pucheros y palmaditas con que
Manuela Pez formuló y adornó la penosísima petición de dinero para
urgentes, inaplazables atenciones de la familia\ldots{} A Centurión se
le iba un color, y otro se le venía. Suspiraba o daba resoplidos echando
de su pecho una fragorosa tempestad\ldots{} Sintiendo su cráneo partido
en dos por el tajante filo, no sabía qué determinar. Acceder era grave
caso, porque tres meses antes le saqueó Manolita sin devolverle lo
prestado. Negarse en redondo no le pareció bien, porque Andrés, al
partir, le había dicho: «Querido Mariano: te ruego que, si fuese
menester, atiendas, \emph{etcétera}\ldots{} que a mi regreso yo\ldots{}
\emph{etcétera}\ldots» En tan horrible trance, pensó que amarrado al
pilar donde le azotaban, no padeció más nuestro Señor Jesucristo\ldots{}
Por fin, cayó el hombre con mortal espasmo en el consentimiento, bañado
el rostro en sudor frío de angustia\ldots{} No era bastante firme de
carácter para la negativa, ni bastante hipócrita para disimular su dolor
inmenso ante la catástrofe. Al retirarse diciendo con lúgubre voz
\emph{volveré con el dinero}, parecía un ajusticiado a quien el verdugo
manda por el instrumento de suplicio\ldots{}

Hallábase doña Celia en el gratísimo pasatiempo de arreglar sus
vergeles, cuando vio entrar al buen don Mariano con cara de amargura y
consternación. «¿Qué tienes, hijo? ¿Ocurre alguna novedad?» le dijo
destacándose del umbrío follaje para llegarse a él y ponerle sus manos
en los hombros. Por no afligir a su bendita esposa, Centurión cultivaba
el disimulo y se tragaba sus penas, o las convertía en contrariedades
leves. Dejándose caer en el sofá y componiendo el rostro, tranquilizó a
la señora con estas apacibles razones: «Nada, mujer: no me ocurre nada
de particular\ldots{} No es más sino que\ldots{} ese maldito \emph{Padre
Cobos}\ldots{} Un amigo de estos que no tienen sentido común, ni
delicadeza, ni caballerosidad\ldots{} me enseñó el último número. De
nada me valió protestar\ldots{} Yo bufaba, y él me leía un parrafillo
asqueroso donde dicen que los del Progreso somos inmorales, que los del
Progreso defraudamos y hacemos chanchullos\ldots{} Ya ves\ldots{} ¡Y
esto se escribe, esto se propaga por los que\ldots! Me callo, sí, me
callo; no quiero incomodarme. Es tontería que me sulfure; tienes
razón\ldots{} Punto en boca; pero antes déjame que repita lo que cien
veces dije: de estas burdas infamias tiene la culpa O'Donnell\ldots{}
Él, él es el causante\ldots{} Bajo cuerda, nuestro maldito irlandés
azuza, pellizca el rabo a estos sinvergüenzas, todos ellos moderados y
realistas, para que hablen mal de nosotros y pongan al Duque en el
disparadero\ldots{} Es mi tema. ¿Que nos insultan? La lengua de
O'Donnell. ¿Que estallan motines? La mano de O'Donnell. ¿Que nos piden
dinero y tenemos que darlo? El sable de O'Donnell.»

En los días siguientes, cuando arreciaban, según Centurión, los manejos
del de Lucena para deshacerse de Espartero, y cuando Escosura lucía su
galana elocuencia en las Cortes, la Coronela Villaescusa y su hija
subieron un grado en el escalafón social, concurriendo a las reuniones
íntimas que Valeria Socobio daba los lunes en su linda casa, calle de
las Torres. Halláronse Manolita y Teresita en un ambiente de elegancia
muy superior al de la humilde tertulia de Centurión; y si por virtud de
la llaneza de nuestras costumbres, algunas figuras concurrentes a la
morada de la calle de los Autores se dejaban ver en la de Valeria, como
la Marquesa de San Blas, Gregorio Fajardo y su mujer Segismunda, también
iban allí personas de pelaje muy fino, como Guillermo de Aransis, y
otros que irán saliendo. Es lo bueno que tenía y tiene nuestra sociedad:
en ella las clases se dislocan, se compenetran, y van prestándose unas a
otras sus elementos, y haciendo correr la savia social por las ramas de
diferentes árboles que, injertados entre sí, llegan a constituir un
árbol solo.

Guapísimas eran Manuela y Teresita, cada una según su tipo y edad; la
madre, un Verano espléndido derivando hacia los tonos naranjados de
Otoño; la hija, plena Primavera rosada y luminosa. A la vera de ambas
iban a buscar sombra y frescura los amadores finos, o los timadores y
petardistas de amor. Coqueteaba la mamá con arte exquisito, colocándose
al fin en un reducto de honradez hipócrita que no engañaba a todos, y
Teresilla jugaba al noviazgo con risueña desenvoltura, pasando los
galanes de la mano de admitir a la mano de rechazar, como en el juego de
\emph{Sopla, que vivo te lo doy}.

Con franca simpatía se unieron Valeria y Teresita. Comunes eran los
secretos de una y otra, todavía de poca importancia y gravedad. Juntas
paseaban los más de los días, y juntas iban al mayor recreo de Valeria,
que era el recorrido de tiendas, comprando, revolviendo, examinando el
género nuevo acabadito de sacar de las cajas llegadas de París. El furor
de novedades había producido dos efectos distintos: embellecer la casa
de Valeria hasta convertirla en un lindísimo muestrario de muebles y
cortinas, y esquilmar el bolsillo de don Serafín del Socobio, hasta que
el buen señor y doña Encarnación pronunciaron el terrible \emph{non
possumus}. De aquí resultó que Valeria, por gradación ascendente de su
fiebre suntuaria, que atajar quería sin voluntad firme para ello, se fue
llenando de deudas, cortas al principio, engrosadas luego, hasta que,
creciendo y multiplicándose, la tenían en constante inquietud. Para
colmo de desdicha, Rogelio Navascués, en vez de llevar dinero a casa, se
gastaba en el Casino toda su paga, y era además insaciable sanguijuela
que desangraba horriblemente el bolsillo de la esposa, nutrido por la
pensión que daban a esta sus padres. Tales razones y el absoluto
enfriamiento del amor que tuvo a su marido, labraron en el ánimo de
Valeria la idea y el propósito de desembarazarse de tan gran calamidad.
No había más que un medio: mandarle a Filipinas, con lo cual ella se
veía libre de él, y él cortaba por lo sano la insostenible situación a
que le habían llevado sus estúpidos vicios.

Iniciado el proyecto por la esposa, el marido lo encontró de perlas.
Quería pasarse por agua, y salir a un mundo nuevo donde no le
conocieran. Manos a la obra. Valeria trabajó el asunto con febril
actividad en Febrero y Marzo, tecleando las amistades y relaciones de su
familia con personajes del Progreso. Moncasi, Sorní, Montesinos, Allende
Salazar ofrecían; mas todo quedaba en agua de cerrajas. Dirigiose luego
a los amigos de O'Donnell, a Vega Armijo, Ulloa, Corbera, y ello fue
mano de santo. No había, no, hombre como O'Donnell: su sombra era
benéfica, y en ella encontraban su paz las familias. A principios de
Abril recibió Navascués el pase a Filipinas, con ascenso, y no esperó
muchos días para ponerse en marcha, porque Valeria, modelo de esposas
precavidas, le tenía ya dispuesta toda la ropa que había de llevar: las
camisas ligeras como tela de araña, los chalecos de piqué, levitines de
crudillo\ldots{} Todo lo adquirió la dama en las mejores tiendas, y del
género superior, por aquello de \emph{al enemigo que huye, puente de
plata}. ¡Qué descansada se quedó la pobre! No podía con su alma de
fatiga y ajetreo de arreglarle en tan pocos días el copioso surtido de
\emph{ropa para países tropicales}.

Horas después de aquella en que la diligencia de Andalucía se llevó a
Rogelio, Valeria dijo a su cordial amiga Teresita: «¡Ay, qué
descanso!\ldots{} Si en España tuviéramos Divorcio, no necesitaríamos
tener Filipinas.»

Y la otra: «¡Filipinas! Alargar la cadena miles de leguas, ¿no es lo
mismo que romperla?»

\hypertarget{vii}{%
\chapter{VII}\label{vii}}

Consecuentes en su fraternal amistad, Valeria y Teresita pasaban juntas
días enteros, muy a gusto de ambas, y a gusto también de Manolita Pez,
que podía campar sin ninguna traba, y espaciar sus antojos por el libre
golfo de la vida matritense, poniendo a su niña bajo la custodia de una
señora casada de buena conducta, que era lo prevenido por los cánones
sociales. Cumplía Manolita con la moral por lo tocante a su hija, y
aliviada quedaba con esto su conciencia para poder cargar con los
pecadillos propios. Muchos días almorzaba y comía Teresa con su amiga, y
algunas noches también allí dormía, por la inocente causa de volver muy
tarde del teatro, y no tener persona mayor y de respeto que tan a
deshora la llevase a casa de su madre. Al poco tiempo de esta intimidad,
observó la niña de Villaescusa que las atenciones con que Guillermo de
Aransis a la señora de Navascués distinguía, iban perdiendo su colorido
platónico. Era Teresita una de estas vírgenes que, por asistir demasiado
cerca al batallar de las pasiones, están privadas de toda inocencia: no
bien ocurridos los hechos, los comprendía y apreciaba en toda su real
gravedad, sin asustarse de cosa alguna. Viendo las visitas de Guillermo
a horas desusadas, y las salidas extemporáneas de la dama, se hizo dueña
de la verdad. Su confianza con Valeria la llevó a una sinceridad ingenua
de \emph{enfant terrible}, y como quien no hace nada, sin asomos de
severidad ni dejo malicioso, interrogó a su amiga sobre tan escabrosos
particulares. En su acento vibraba un candor que en su alma no existía.
Respondiole Valeria con cierto embarazo, empezando diferentes frases que
quedaron sin terminar, y concluyó así: «¿Para qué quieres tú más
explicaciones?\ldots{} Estas cosas no las entienden las solteras\ldots»

Saliendo aquel mismo día las amigas al jaleo de tiendas, vio Teresita
con asombro que Valeria pagaba cuentas atrasadas, lanzándose a nuevas
compras de telas y faralaes de vestir. Generosa y amable, la dama
obsequió a su amiga con un corte de vestido para verano, elegantísimo,
de extremada novedad y con el más puro sello parisiense, regalándole de
añadidura un canesú y un miriñaque de pita de hilo, última novedad. Con
sincera gratitud acogió Teresa estos obsequios, y los estimó más porque
su madre la tenía bastante desairadita de ropa, con sólo dos trajes
nuevos, y uno del año mil, transformado ya tres veces.

No estaba descontenta Teresa en aquellos días, que ya eran de franco
Verano, y el conocimiento del enredo de Valeria con Aransis despertaba
en ella tanto interés como una novela de las mejores que entonces se
escribían. Novela era, viva, de estas que entretienen y no asustan.
Personaje de novela le pareció Aransis, guapo, joven, condiciones
precisas para la figuración poética, la cual era más grande y sutil por
sus maneras exquisitas, y el derroche de dinero que suponían sus trajes,
coches y todo el tren de su dorada existencia. Y no fue Guillermo el
único personaje novelesco que por entonces mantenía el espíritu de
Teresa en continua soñación. Desde los comienzos de Mayo se personaba en
los \emph{Lunes} de Valeria un joven muy guapo, de belleza distinta de
la de Aransis, pero no menos atractiva. Era rubio, de azules y dulces
ojos, con una barba ideal, de corte y finura semejantes a la de Nuestro
Señor Jesucristo, tal como le representan Correggio y Van Dyck. Dominaba
en sus pensamientos la melancolía, como en su voz los tonos apacibles.
Era extremeño; se llamaba Sixto Cámara. A Teresa cautivó desde el primer
día por su conversación fina, por el atrevimiento de sus ideas, y la
noble lealtad que su trato, como toda su persona, revelaba. Gozosa le
veía llegar a la reunión, y con mayor gozo veía preferencia que por ella
mostró desde la primera noche, entrando al poco tiempo por la senda
florida del galanteo. Creyó Valeria que en aquel noviazgo sería Teresa
más perseverante que en los anteriores, y de ello se alegraba; Manuela
Pez, en cambio, no parecía gustosa de que su hija se insinuase con el
galán de la barba bonita, y así se lo manifestó con razones de peso, la
noche de un lunes, al volver a casa rendidas de tanto charlar y de un
poquito de bailoteo.

«Mira, Teresa---le dijo:---te he reñido por tu ligereza en admitir y
despachar novios, y ahora, que te veo más sentadita, también te riño,
porque das en ser consecuente con uno que no te conviene poco ni mucho.
Ya debes decidirte, fijándote en aquellos que puedan sacarte de pobre, y
reservando tus despachaderas para los barbilindos que no traen nada de
substancia. Los tiempos están malos, vendrán otros peores, y como no te
cases con un rico, no sé qué va a ser de ti. Despreciaste al que yo te
propuse, Alejandrito Sánchez Botín, y ahora te veo entontecida y
acaramelada con el don Sixto, del cual me han dicho que con todo su
saber, y su hablar modoso, y su vestir elegante, y su barbita, no es más
que un triste pelagatos, con lo comido por lo servido, y los pocos
reales que saca de algún periódico. ¿Te parece a ti que es buen porvenir
un papel público y las rentas que pueda dar?\ldots{} Y hay otra cosa:
del don Sixto me han dicho que es \emph{demagogo}. ¿Sabes lo que es
esto? Pues tener ideas disolventes, querer derribar el Trono, y puede
que también el Altar, y traernos un Gobierno de anarquía, que es, como
quien dice, la gentuza. No, hija mía: apártate de esto, y no te me hagas
demagoga, la peor cosa que se puede ser. Figúrate el porvenir de un
hombre que jamás desempeñará un destino del Gobierno, porque estos no se
dan a tales tipos\ldots{} No des a demagogos, y si me apuras, ni a
progresistas, el sí que te piden, pues harías trato con el hambre y la
desnudez. Ten juicio y fíjate en alguno que sea resueltamente del
partido de O'Donnell, el hombre que muy pronto ha de coger la sartén por
el mango\ldots{} Con que, fuera el don Sixto, o entretenle hasta que
venga el bueno\ldots{} que vendrá, yo te aseguro que vendrá.»

Oyó estas razones y sabios consejos Teresita, fingiendo admitirlos como
palabra divina; mas en su interior se propuso hacer su gusto, que en
esto iba a parar siempre con maestra de tan poca autoridad como su
madre. Al día siguiente la llamó Valeria; fue, charlaron\ldots{}
Tratábase de organizar una temporadita en la Granja, donde se
divertirían mucho, si la Coronela daba permiso a Teresa para ir con su
amiga. Examinaban las dificultades que para esto podían surgir, y la
resistencia que había de oponer Manuela si no la invitaban también a ser
de la partida, cuando entró Aransis inquieto, y contó que en el Consejo
con Su Majestad, aquella mañana, O'Donnell y Escosura habían rifado de
una manera solemne y ruidosa. La Reina se decidía por O'Donnell, y
Espartero, desairado en la persona del Ministro que representaba su
política, había dicho: \emph{vámonos}. El \emph{vámonos}, o el \emph{yo
también me voy} del Duque de la Victoria, era una proclama
revolucionaria. Si Espartero, apoyado en las Cortes y al frente de la
Milicia Nacional, daba a don Leopoldo la batalla, ardería Madrid. Había
que desistir del viaje a la Granja mientras no se aclarase el horizonte.
No se asustaron la señora y señorita tanto como Guillermo esperaba;
antes bien, dijeron que les gustaban las trifulcas, y que si había de
venir revolución gorda, viniera de una vez para ver si se quedaban con
España los Nacionales, o se quedaba O'Donnell, con su personal de
caballeros elegantes, limpios y vestidos a la última moda. Esto era lo
más probable y lo más revolucionario, pues la ramplonería y ordinariez
debían ser desterradas para siempre de este hidalgo suelo.

Observó Teresa que Aransis no estaba contento, y que las anunciadas
revueltas le contrariaban. Sintiendo acaso preferencias por estas o las
otras ideas políticas, ¿temía verlas derrotadas en la próxima lucha?
Esto no podía ser, pues harto sabían Valeria y Teresita que el ocioso
galán, aunque inclinado en su espíritu a las tendencias liberales, era
en la práctica un gran escéptico, y no se dignaba empadronar su nombre
ilustre en el censo progresista ni en el moderado. Las gloriosas espadas
no le llevaban tras sí, y con igual indiferencia veía los resplandores
de la de Luchana, de la de Lucena o de Torrejón. Sin duda, el endiablado
humor de Aransis provenía de algún contratiempo relacionado con la
política por extraños engranajes, pero que no era la política misma. Así
lo pensaba Valeria; así también Teresa, que, aunque más talentuda que su
amiga, érale inferior en el conocimiento del mundo. Ninguna de las dos
penetró el arcano. La Historia lo sabe, y lo revelará, pues no sería
Historia si no fuese indiscreta.

Guillermo de Aransis, Marqués de Loarre por sucesión directa, Conde de
Sámanes y de Perpellá por su parte en la herencia de San Salomó, era un
joven de excelentes prendas, corazón bueno, inteligencia viva; prendas
¡ay! que se hallaban en él ahogadas o por lo menos comprimidas debajo
del avasallador prurito de elegancia. Resplandor de la belleza es la
elegancia, y como tal, no puede negársele la casta divina; pero cuando
al puro fin de elegancia se subordina toda la existencia, alma, cuerpo,
voluntad, pensamientos, sobreviene una deformación del ser, horrible y
lastimosa, aunque, en apariencia, no caiga dentro del espacio de la
fealdad. Dotado de atractivos, hermosa figura, palabra fácil y
seductora, no vivía más que para agregar a su persona todos los
ornamentos y toda la exterioridad que había de darle brillo y supremacía
evidentes entre los individuos de su clase. Exaltado su amor propio, no
reparaba en medios para obtener tal supremacía y hacerla indiscutible;
sus trajes habían de ser los más notorios por el sello de la
personalidad, siguiendo la moda con el precepto sutil de acatarla sin
parecerse a los que ciegamente la seguían. Había de ser lo suyo distinto
de lo general, sin disonancia, o con sólo una disonancia que, por muy
discreta, llevaba en sí la deseada y siempre perseguida superioridad. Se
preciaba, o de inventar algo en el arte de vestir, o de ser el primero
que importase de los talleres parisienses las formas nuevas, cuidando de
presentarlas modificadas por su gusto propio antes que el uso de los
demás las generalizara. En todo esto, para que resultase verdadera
elegancia, la naturalidad sin estudio alejaba toda sombra de afectación.

A estos primores del vestir seguían los del andar en coche. Muy santo y
muy bueno, legítimo a todas luces, es que no salgan a pie los ricos, y
que gasten coche para su comodidad, decoro y recreo; pero que se pasen
el día ostentando formas y estilos nuevos de carruajes, guiándolos con
más trabajo de cocheros que descanso de señores, es un extremo de
vanidad rayano en la tontería. El elegante toma con esto un carácter
profesional; siente sobre sí la mirada crítica y exigente del público;
ha de divertir antes que divertirse; los bonitos caballos de tiro y de
silla pregonan su riqueza y buen gusto, y al fin se estima y alaba más
la gallardía de sus bestias que la suya propia.

Naturalmente, las vanidades del orden suntuario iban a resumirse y
coronarse en la vanidad amorosa. Aransis llegó a creer que uno de los
principales fines de la Humanidad era que se prendasen de él todas las
mujeres hermosas que en Madrid había. Lo consideraba en ellas como una
obligación, y en sí como un cumplimiento de las leyes de su destino. Con
todas entraba, alcurniadas y plebeyas, más afortunado tal vez en las
zonas altas que en las medias de la sociedad, por venir esta corrupción
de arriba para abajo, cosa en verdad que no es nueva en la Historia de
los pueblos. Imposible referir todas las proezas de amor con que ilustró
su juventud el Marqués de Loarre, y sobre difícil, la estadística sería
poco interesante, por carecer estas aventuras, en el prosaico siglo XIX,
de la poesía erótica y caballeresca que en edades de más duras
costumbres tuvieron. La tolerancia de hecho encubierta con la gazmoñería
pública, la flexibilidad moral y el culto frío y de pura fórmula que la
virtud recibía, quitaban toda intensidad dramática a las transgresiones
de la ley. Salían de los palacios estas historias, sin que al pasar de
la realidad a las lenguas, movieran ruidosamente la opinión, ni
escandalizaran en grado más alto que el común de los sucesos privados y
públicos. Como los pronunciamientos y motines, como las revoluciones a
tiros o a discursos por ganar el poder, estas inmoralidades del mundo
heráldico iban tomando carácter crónico que apenas turbaba la paz de las
conciencias amodorradas.

Si en los amoríos de garbosa vanidad, y en otros de pasional demencia,
se iba dejando Aransis vellones de su fortuna, el vellón más grande lo
perdió con la Marquesa de Monteorgaz, dama en extremo dispendiosa, con
menguada riqueza por su casa. Era un zarzal con tantas púas, que el
Marqués de Loarre perdió en él toda su lana. Los estados de Sámanes y
Perpellá quedaron como si dijéramos desnudos, en fuerza de hipotecas. No
era en total la fortuna de Guillermo de las más altas de la grandeza:
podía con ella vivir holgada y noblemente, sujetándose a un orden
estrecho de administración. Pero con la vida que llevaba quedaría todo
el caudal liquidado en media docena de años. Tarde vio el \emph{lion} el
abismo en que había de caer; pero aún podía salvar una parte del haber
patrimonial si se plantaba en firme y ponía un freno a sus desórdenes.
Sobre esto le habló con cariñosa severidad un día su amigo Beramendi:
tan instructivo fue el sermón, exégesis de aquella sociedad y de otras
más próximas a la nuestra, que la Historia se dignó traerlo acá y
hacerlo suyo.

\hypertarget{viii}{%
\chapter{VIII}\label{viii}}

«Estás arruinado, Guillermo, y sólo trazando una raya muy gorda en tu
vida con propósito de cambiar esta radicalmente, podrás salvar lo
preciso para vivir con decencia, sin locuras. Dices que aún cuentas con
la herencia de tu tío el Marqués de Benavarre, y con ese monte de la
sierra de Guara, que denunciado ya como terreno carbonífero, puede ser
para ti un monte de oro. No te fíes, Guillermo: tu tío puede cambiar de
propósito, si llega a enterarse de los humos que gastas, y en el monte
no pongas tus esperanzas: una vez entre mil dejan de salir fallidas las
ilusiones de los mineros. Déjate, pues, de montes de oro y de tíos de
plata, y hazte cargo de la realidad, y oye bien lo que voy a decirte,
que es duro, muy duro, pero saludable. Por algo soy el amigo que más te
quiere.

La vida que vienes haciendo del 50 acá es enteramente estúpida; tu
conducta es la de un idiota. Imbecilidad pura es tu vida, y así la llamo
pensando que todavía no la califico tan severamente como merece. Y voy
más allá, Guillermo: sostengo que no hay derecho a vivir así. Se dice
que cada cual hace de su dinero, de su tiempo y de su salud lo que
quiere; y yo afirmo que eso no puede ser. En el dinero, en el tiempo y
en la salud de cada persona hay una parte que pertenece al conjunto, y
al conjunto no podemos escatimarla\ldots{} Una parte de nosotros no es
nuestra, es de la totalidad, y a la totalidad hay que darla. ¿Qué? ¿te
asombras? ¿No entiendes lo que digo? Pues lo repito, y añado que están
por hacer las leyes que determinen esa parte de nosotros mismos
perteneciente al acervo común, y que ordenen la forma y manera de que
los demás, todos, le quiten a cada cual esa partija que indebidamente
retiene. Las leyes que faltan se harán: ni tú ni yo lo veremos; pero
cree que se harán\ldots{} Y mientras las leyes vienen, debemos anticipar
su cumplimiento con algo que se parezca a la ley nonnata. Tú, Guillermo,
eres idiota y criminal, porque gastas todo tu dinero, todo tu tiempo y
toda tu salud en no hacer nada que conduzca al bien general. El que no
hace nada, absolutamente nada, debe desaparecer, o merece que le tasen
los bienes que derrocha sin ventaja suya ni de los demás. Me dirás que
yo soy lo mismo que tú, que vivo en grande sin trabajar ni producir cosa
alguna. Estás equivocado: yo hago algo, no todo lo que debo; pero con un
poquito de acción útil cumplo la ley, y no soy como tú, materia inerte
en la Humanidad. Yo gasto parte de las rentas de mi mujer en vivir bien
y decorosamente, sin escarnecer con un lujo desfachatado a esta familia
española compuesta de pobres en su gran mayoría. Yo no cultivo mis
tierras, no ejerzo ninguna profesión ni oficio; pero no puede decirse de
mí que nada produzco. Yo he producido un hijo, y en criarle y educarle
para que sea ilustrado, saludable y hombre de bien, pongo todo mi
espíritu y empleo casi todas las horas del día. ¿Qué\ldots{} te ríes?
¿Te parece poco?

No me interrumpas\ldots{} déjame seguir. Voy a contar por los
dedos\ldots{} por los dedos no, pues son pocos para tan larga
cuenta\ldots{} Voy a recordarte los crímenes de imbecilidad que has
cometido, para que te horrorices: Cubrir de piedras preciosas el seno
hiperbólico de la Navalcarazo, que te lo agradeció diciendo, al mes de
romper contigo, que eras un \emph{niño de la Doctrina Cristiana}. Para
pagarle a Samper toda aquella quincalla fina, tuviste que hipotecar dos
dehesas\ldots{} a dehesa por pecho. Sigo: no fue menor imbecilidad
regalarle a \emph{Pepa la Sevillana} una casa de tres pisos en la calle
de Belén. Habrías cumplido con una casa de muñecas\ldots{} para jugar a
los compromisitos\ldots{} Imbecilidad de marca mayor, los convites de
doscientas personas que dabas en tu finca de Aranjuez, con tren
especial, comilonas servidas por Lhardy, y champaña de la señora Viuda
de Clicquot a todo pasto\ldots{} En tus chapuzones con la de Cardeña no
pudiste deslumbrar a esta con alardes de lujo insensato, porque ella es
más rica que tú, como diez veces más rica. Pero de aquella fecha data tu
furor de coches y caballos, que luego llevaste al delirio en tiempo de
la Villaverdeja, grande apasionada de las cosas hípicas y cocheriles. El
colmo del idiotismo veo en tu afán de pasear por Madrid trenes lujosos,
y la misma Villaverdeja o la Belvis de la Jara, no estoy bien seguro, te
hizo justicia poniéndote el apodo del \emph{Faetonto}\ldots Te han hecho
un daño inmenso tus viajes anuales a París, y el flujo de imitar las
opulencias que has visto en aquella capital. Bien podías haberte lucido
discretamente en este coronado villorrio, sin importar las grandezas que
allí son proporcionadas y aquí desmedidas. Añadiendo a estas locuras el
boato de tu casa, tus almuerzos y cenas, tu protección a innumerables
vagos que, adulándote, te trastornan, y con astutas socaliñas te
saquean, tenemos, mi querido Guillermo, que el Bobo de Coria es un sabio
comparado contigo.

Pero el punto en donde llegas a la suprema imbecilidad y al idiotismo
más perfecto, lo vemos en tu enredo con la Monteorgaz. Si en otros
amoríos te arruinabas neciamente, al menos veías satisfecha tu vanidad.
Los brillantes de la Navalcarazo, la casita de \emph{Pepa la Sevillana},
los coches de la Belvis de la Jara, y tus faetones, tus caballos
normandos o cordobeses o del Demonio, te daban fama de esplendidez y el
diploma de hombre de buen gusto. ¿Pero qué ibas ganando con la
Monteorgaz, más graciosa que bonita y más elegante que joven, que tiene
detrás de sí un familión famélico, capaz de tragarse el dinero de media
España y de digerirlo sin que se le resienta el estómago? Carolina te
hacía pagar sus cuentas rezagadas de diez años, y las del Marqués, que
debía sumas fabulosas a Utrilla y a los dependientes del Casino. Seguían
los hermanos de ella, los hermanos de él, todos unos perdidos, con
hambre atrasada de dinero y de protección\ldots{} Caían sobre ti como
nube de langosta, y tú, que no sabes negar nada y eres un fenómeno
morboso de generosidad; tú, Guillermo, que si hubieras sido mujer,
habrías entregado tu honor al primer pedigüeño que se te pusiera
delante; tú, Guillermo, a todos consolabas, creyendo rodearte de
agradecidos, y lo que hacías era enseñar la ingratitud a los
viciosos\ldots{}

Sigo, y aguanta el nublado\ldots{} Dime, gran majadero: ¿qué
satisfacción del amor propio sentías viéndote de número veintitantos en
el índice amoroso de Carolina Monteorgaz? ¿Qué ilusión te fascinó, qué
desvarío te disculpa? Si no puedes vivir sin hacer perpetuamente el don
Juan; si tu fatuidad necesita el rendimiento de mujeres, búscalas en
esfera más humilde: dedícate a las costureras, que las hay muy lindas,
más hermosas que las de arriba, y algunas más ilustradas, con mejor
ortografía que la Belvis de la Jara, que escribe \emph{ir} con \emph{h}
(yo lo he visto); cultiva las viudas de empleados o viudas de
cualquiera, en clase modesta; y entre estas, tu personalidad de
\emph{lion fashionable} alcanzaría triunfos facilísimos y de reducido
coste. Imita al noble Marqués de la Sagra, hermano de la Villaverdeja,
que con mundana filosofía se ha dedicado a las cigarreras (entre las
cuales las hay muy monas), y gracias a lo económico de sus vicios, ha
podido fomentar sus propiedades de Griñón, Alameda y Villamiel\ldots{}
Ahí tienes un modelo de próceres que sabe divertirse mirando por la
prosperidad del país\ldots{} Aprende, abre los ojos\ldots{}

No tomes esto a broma; no argumentes, no te defiendas, que defensa no
tiene tu estolidez, y escucha un poco más. He señalado el mal,
mostrándolo en toda su magnitud fea para que te cause espanto, y ahora
voy a proponerte, si no el remedio, que es difícil y ya vendría tarde,
al menos el alivio. Óyeme, Guillermo: si yo te propusiera que cambiaras
de improviso tu modo de vivir, sujetándote al modesto pasar de un
empleado de catorce o de veinticuatro mil, sería tan necio como tú.
Nunca serías capaz de tanta abnegación, ni está tu alma templada para
sacrificios grandes del amor propio\ldots{} Lo que has de hacer, ante
todo, es balance general de tu hacienda, y saber lo que debes, las
obligaciones hipotecarias que has contraído, lo que aún posees libre,
\emph{etcétera}; en fin, que pongas ante tus ojos la realidad escueta,
descartando todo lo ilusorio. Para esto necesitas valor, necesitas
disciplina\ldots{} No perdones ningún dato verdadero, no te engañes a ti
mismo\ldots{} Luego que sepas lo que has perdido y lo que te resta,
trata de impedir que ese resto se te escurra también, para lo cual has
de hacer propósito firme de poner punto final en tus aventuras
\emph{donjuanescas} con señoras de copete\ldots{} Inmediatamente de
esto, antes hoy que mañana, pensarás en buscar novia con buen fin; una
heredera rica, riquísima. El santo matrimonio, de que tú has sido
burlador, es lo único que puede salvarte\ldots{} Por la cara que pones,
comprendo que esta idea no te parece mal. Como que no hay para ti otra
salida del atolladero en que estás.

Te veo meditabundo. Piensas, como yo, que una heredera rica millonaria y
de clase igual a la tuya no es tan fácil de encontrar en los tiempos que
corren\ldots{} Casi todas las que había se han ido colocando. Las de
banqueros y capitalistas, que fácilmente adquieren hoy título
nobiliario, también escasean. Algunas conozco que te convendrían; pero
aún son muy niñas; tendrías que esperar, y esperar es envejecer\ldots{}
A ver qué te parece esta otra idea que ahora se me ocurre\ldots{} Pon
atención, y no te enfades si para plantear esta idea, precisado me veo a
proponerte algo que seguramente no será de tu gusto, algo que hiere tu
dignidad\ldots{} Lo digo, aunque al oírme des un brinco en la
silla\ldots{} Ya sabes que en España tenemos un medio seguro de aliviar
la desgracia de los que por su mala cabeza, por sus vicios o por otra
causa, pierden su hacienda. Se les manda a la isla de Cuba con un buen
destino, y allá se arreglan para recobrar lo que aquí se les fue entre
los dedos. España goza de esta ventaja sobre los demás países: posee un
heroico bálsamo ultramarino para los males de la patria europea\ldots{}
No te sulfures, ten calma, y óyeme hasta el fin. Ya sé que considerarás
denigrante el tomar un empleo en Cuba; ya sé que tú, si lo tomaras, no
irías allá con el fin bajo de ensuciarte las manos en la Aduana, o de
especular con los desembarcos fraudulentos de carne negra\ldots{}
No\ldots{} ya sé que no harás esto, y que si vas pobre, volverás puro
con los ahorros de tu sueldo, y nada más.

Si te propongo este arbitrio\ldots{} pasado por agua, es porque calculo
que el casamiento redentor que aquí no encontraríamos fácilmente, allí
\emph{te saldría} en cuanto llegaras, por la virtud sola de tu
esplendorosa persona, por tu elegancia y nobleza, y la fama que has de
llevar por delante. El género de ricas herederas abunda en aquella
venturosa Isla, créelo; no tendrás más trabajo que \emph{l'embarras du
choix}. Véate yo, Guillermo, llegar aquí corregido de tus ligerezas y
aumentado con una guajirita muy mona, de hablar lento, dengoso, que
recrea y enamora. Será bonita, tierna, leal, amante, y con más inocencia
y rectitud de principios que el género de acá, un tantico dañado por
influjo del ambiente y de la proyección de las clases altas sobre las
medias. Pues en el aquel de la instrucción femenina, no sé si te diga
que irás ganando. Allá se van estas con aquellas en nociones científicas
y de vario saber; pero sí te aseguro, refiriéndome al arte inicial, o
sea, la escritura, que las cubanitas gastan una letra inglesa limpia y
gallarda, y una ortografía que ya la quisieran nuestras elegantes para
los días de fiesta. En fin, hijo, que no te me subas a la parra de la
dignidad por esto de la cubanita. Mira las cosas por el lado práctico,
que suele ser el lado más bonito; no desprecies los ingenios, los
potreros y cafetales que para ti reserva la virgen América; piensa en el
genio de Colón; considera los cientos de miles de cajas de azúcar que
podrás verter en el Océano de tus amarguras para endulzarlo\ldots{}

\hypertarget{ix}{%
\chapter{IX}\label{ix}}

Veo que si te subes a la parra de la dignidad---prosiguió
Beramendi,---no trepas tan alto como yo creía\ldots{} Calma, y ojo a los
hechos reales. Ponte en el exacto punto de mira, y aléjate del
sentimentalismo, que te alteraría las líneas y color de los
objetos\ldots{} Ahora, dando por hecho que trazas en tu existencia la
línea gorda de que antes te hablé, establezcamos el sano régimen
económico en que de hoy en adelante has de vivir. Para librarte de la
usura que en poco tiempo te dejaría sin camisa, es forzoso que levantes
un empréstito, en grande, no para salir del día y del mes, sino para
salvar definitivamente los restos de tu patrimonio. Entre tú y yo
tenemos que buscar un capitalista o banquero que recoja todo el papel
emitido por ti en condiciones usurarias, y además te cancele en tiempo
oportuno la escritura de retro que en mal hora hiciste a mi hermano
Gregorio. De este no esperes piedad ni blanduras, pues aunque él
quisiera ser fino y blando, por lo que queda de nativa indulgencia en su
corazón, Segismunda no se lo permitiría. Esta es implacable, feroz en
sus procedimientos adquisitivos, como lo es en su ambición. Si
encontramos el capitalista que quiera salvarte, pactarás con él lo
siguiente: tú le entregas todas las fincas de los estados de Loarre y
San Salomó, con facultad de vender las que se determinen y de
administrar las restantes. Él, al otorgarse la escritura, cancelará las
cargas hipotecarias y los créditos pendientes. Tu propiedad inmueble
queda en poder suyo hasta la amortización de tu deuda, y en ese tiempo
recibirás de él trimestralmente la cantidad que se estipule para que
puedas vivir con decoro y modestia, ajustando estrictamente tus
necesidades a esa rigurosa medida.

Y ahora digo yo: ¿a qué capitalista debemos acudir? Piensa tú, recorre
tus conocimientos; yo pasaré revista en los míos. ¿Qué te parece don
José Manuel Collado? De Rodríguez y Salcedo, ¿qué me dices? ¿No eres tú
amigo del Duque de Sevillano? Yo lo soy de don Antonio Guillermo
Moreno\ldots{} Cerrajería y Pérez Hernández, me consta que han hecho
negocios de esta índole\ldots{} ¿Quieres que mi suegro y yo hablemos a
don Antonio Álvarez y a don Antonio Gaviria, o crees tú que podrás
entenderte fácilmente con Casariego? ¿Has pensado en Udaeta, en Soriano
Pelayo? ¿Podríamos contar con Zafra Bayo y Compañía, si habláramos a
nuestro amigo Adolfo Bayo?

Debo advertirte, para que no te adormezcas en una confianza optimista,
que nuestros hombres de dinero no se aventuran en ningún negocio que no
vean claro y seguro desde el momento en que se les plantea. Por rutina y
por comodidad, van tras las ganancias fáciles, con poco riesgo y sin
quebraderos de cabeza. Han tomado el gusto a las gangas que nos ha
traído la transformación social; se han acostumbrado a comprar bienes
nacionales por cuatro cuartos, encontrándose en poco tiempo poseedores
de campos extensos, feraces, y no se avienen a emplear el dinero en
operaciones aleatorias de beneficio lento y obscuro. No les censuremos
por esto: es condición humana.

Que nuestros ricos están a las maduras y no a las agrias, lo ves
palpablemente en que pudieron agruparse y acometer con dinero español
empresa tan nacional y útil como el ferrocarril de Madrid a Irún, y se
han echado atrás, dejando esta especulación en manos de extranjeros. No
sienten estos señores el negocio con espíritu amplio y visión del
porvenir: ven sólo lo inmediato, y se asustan de la menor sombra.
Carecen de la virtud propiamente española, la paciencia. Verdad que esta
virtud no la tenemos más que para el sufrimiento\ldots{} Otra cosa. Es
fácil que un solo capitalista no se atreva solo con tan grande
operación, y que se reúnan dos o tres en reata para tirar de ti, pobre
carro atascado en los peores baches de la existencia. En fin, sea lo que
fuere, tú por tus relaciones, yo por las mías, buscaremos un Creso,
entre los pocos Cresos españoles que tengan el sentido de la
reconstrucción, en vez del sentido de la destrucción. Porque no lo
dudes: un principio negativo les ha hecho ricos\ldots{} Grandes casas
son, levantadas con material de ruinas\ldots{} Han contratado el derribo
de la España vieja. ¿La nueva quién la construirá?»

Sensible al grande afecto que el sermón revelaba, Guillermo manifestó su
conformidad con los claros razonamientos de su amigo, y lanzándose con
ardor a las primeras iniciativas, pasó revista fugaz a los próceres del
dinero. «¿Te parece que desde luego hable yo con Cerrajería?\ldots{} Y
entre tanto, tú tanteas a Collado, a Sevillano\ldots{} Este me parece el
más capaz de comprender la operación y sus ventajas. Sólo una vez he
hablado con él. ¿Sabes dónde? En el baile que dio la Montijo para
celebrar los días de su hija Paca, a fines de Enero. Pues Miguel de los
Santos me presentó a Sevillano, que estuvo conmigo amabilísimo\ldots{}
Tengo idea de que me dijo algo del arrendamiento de los pastos de mis
dehesas de Perpellá\ldots{} Si no me equivoco, sus ganados trashuman de
la provincia de Guadalajara a la de Huesca. Luego le he visto dos o tres
veces en la calle; nos hemos saludado\ldots{} Créelo: me resulta
respetable este hombre, que de la paja ha extraído el oro.»

Quedaron, en fin, los dos amigos en trabajar el asunto cada uno por su
lado, y así se hizo, siendo más activo Beramendi que el propio
interesado, cuyo espíritu fácilmente se escapaba de las cosas graves
para volar hacia las frívolas. La primera noticia de que su amigo
gestionaba, la tuvo Aransis una noche en la casa del Duque de Rivas,
adonde concurría con preferencia por gusto de la distinción, buen tono y
amenidad que allí reinaban. Eran las salas del Duque terreno en que lo
mejorcito de las Letras y la flor y nata de la Aristocracia se juntaban,
sin que ninguna de las dos Majestades se sintiera humillada ante la
otra. Arte y Nobleza hacían allí mejores migas que en ninguna parte,
bajo los auspicios del que era Grande de la Poesía y Grande de España,
dos grandezas que no suelen andar en un solo cuerpo. La noche de
referencia, Guillermo Aransis encontró a Martínez de la Rosa charlando
con Romea, y a Escosura con Nocedal, el agua y el fuego. Aquel era, sin
duda, el reino de la transacción y de la tolerancia, porque la de
Madrigal y la de Monvelle, damas respetabilísimas, celebradas por sus
virtudes, alternaban con la Navalcarazo y la Villaverdeja, reputaciones
de calidad muy distinta. Molíns, Bretón de los Herreros, Alcalá Galiano
y Federico Madrazo, llevaban la representación de las Letras y de la
Pintura. Con otros próceres arruinados como él, o en camino de serlo, el
de Loarre representaba la Grandeza holgazana, distraída y sin ningún
ideal serio de la vida, preparándose a un buen morir, o a un morir
deshonroso\ldots{} Le llamó la Navalcarazo, para decirle secreteando:
«Guillermo, ya sé que estás en \emph{pourparlers} con los capitalistas
para el arreglo de tu casa. Me lo ha dicho Collado\ldots{} Yo ando
detrás de Felipe (este Felipe era el Marqués de Navalcarazo) para que
haga una cosa semejante; pero nada consigo. Felipe es un hombre
imposible\ldots{} el eterno sonámbulo que dormido tira el dinero, y no
despierta sino cuando se le acaba y viene a pedírmelo a mí\ldots{} Aún
estás a tiempo, Guillermo. Entiéndete con esos señores. Me ha dicho
Collado que hará el negocio a medias con Udaeta\ldots» Así dijo la dama
frescachona, y cuando salían, cogiéndole el brazo, añadió esto: «Vas por
buen camino, Guillermo. Luego buscas una heredera rica, aunque sea del
ramo de Ultramarinos, y ya eres hombre salvado.»

Claramente vio Aransis que Beramendi trabajaba por él. Fue a verle al
siguiente día, y juntos visitaron a Collado, quien les dijo que tenía el
negocio en estudio y que pronto daría contestación. Pero la respuesta se
hizo esperar. Hablaron a Bayo y a Casariego, que de plano rechazaron la
proposición, y una noche, ya bien entrada la primavera, hallándose
Aransis en casa de Osma, tuvo inesperada noticia de su asunto por otra
dama de historia, muy corrida, y de extraordinario y sutil ingenio. Era
la Campofresco, a quien la Marquesa de Turgot, Embajadora de Francia,
llamaba \emph{Madame Diogène}, expresando así muy bien el gracioso
cinismo de aquella señora que, sin tonel ni linterna, creaba con sus
célebres dichos la filosofía mundana más adaptable a la sociedad de
aquel tiempo. «Guillermito---le dijo, sentada junto a él a la mesa,---yo
le tenía a usted por un loquinario, y ahora resulta que es uno de
nuestros primeros razonables. Bien, hijo, bien: así me gustan a mí los
hombres. Lo he sabido por Sevillano, que es mi banquero, y hoy estuvo en
casa y me preguntó si me parecía bien el negocio. Yo le contesté que
sí\ldots{} Dígame: ¿quién le aconsejó su salvación? De fijo no ha sido
la Navalcarazo, ni la Monteorgaz\ldots{} Apuesto a que ha sido
\emph{Pepa la Sevillana}, que estas \emph{de cartilla} son las que
tienen más talento\ldots» Reían\ldots{} \emph{Madama Diógenes} habló de
otras cosas.

En efecto: Sevillano estudiaba el asunto, y en tales estudios pasó
tiempo largo, con grande impaciencia y desazón del Marqués de Loarre,
que cada día se iba hundiendo más, y que, incapaz de parar en firme los
estímulos de su vanidad donjuanesca, buscó en Valeria Socobio un
enredillo modesto, creyendo, sin duda, que podría sostener su imperio
sobre la mujer en condiciones poco dispendiosas. Cansado de esperar el
fin de los prolijos cálculos que hacían los aristócratas del dinero, se
lanzó a proponer su asunto a otras casas. Habló con Weissweiller y
Baüer, los cuales, por conducto del simpático y bondadoso don Ignacio,
le dijeron que la cantidad del empréstito no les asustaba; pero que en
España no hacían ninguna operación sobre foncière. Tratárase de fondo
\emph{mobiliario}, y llegarían a entenderse. Ya desesperaba el aburrido
galán de encontrar su remedio, cuando Collado y Carriquiri unidos
formularon unas bases que, si alteraban algo el primitivo proyecto y
fijaban condiciones un tantico onerosas, resolvían la cuestión con más o
menos ventajas, y el caballero no podía menos de conformarse con ellas.
Eran su única esperanza, su salvación infalible, si aseguraba los
efectos de la medicina con una perfecta higiene. Empezaron los
preparativos, examen de escrituras y ejecutorias, contratos, hipotecas,
préstamos, y en ello estaban cuando sobrevino la ruptura entre Espartero
y O'Donnell y el derrumbamiento de la situación política. En puerta una
nueva revolución, la Milicia Nacional en armas, \emph{Baldomeroff}
rabioso, \emph{Leopoldowitch} apoyado por Palacio, Palacio decidido a la
resistencia, se obscurecían los horizontes, y sobre la sociedad, sobre
el Trono mismo y su compañero el Altar, venían tempestades cuyo fragor
en lontananza se percibía. Tal fue el motivo del repentino y doloroso
desengaño de Aransis, cuando ya creía tener en la mano su regeneración.
Collado, a quien vio aquel día en el Congreso, le dijo en tono plácido,
que a Guillermo le sonó a \emph{Dies iræe}: «Amigo mío, no podemos hacer
nada por ahora. ¡Quién sabe lo que va a venir aquí!\ldots{} ¿Estallará
el volcán?\ldots{} Yo me temo que estalle\ldots{} Esperemos.»

Ved aquí por qué se presentó aquel día el Marqués de Loarre con tan
mohíno rostro y decaimiento del ánimo en casa de Valeria, y por qué
relató los graves sucesos políticos con acento de pesimismo fúnebre.
Como se ha dicho, Valeria no penetraba la causa de la sombría tristeza
de su amigo; Teresita, menos conocedora del mundo que Valeria, pero
dotada de mayor perspicacia, no sabía, pero sospechaba; no veía el fondo
del abismo, pero algo vislumbraba asomándose a los bordes\ldots{} No era
aquel día el más propio para entretenerse en vanas pláticas con dos
mujeres, que no daban pie con bola en nada referente a la cosa pública:
desfiló el galán volviéndose al Congreso; de allí pasó a casa de Vega
Armijo, ávido de noticias. Por desgracia, estas eran malas, y en todas
las bocas aparejadas iban con negros presagios. Comió en casa de
Beramendi, y fueron luego juntos al Príncipe, a ver \emph{El Tejado de
Vidrio}, linda comedia de Ayala. En el teatro no se hablaba más que de
política, de esa política febril y ansiosa, natural comidilla de las
gentes en los días que preceden a las grandes agitaciones; fue después
al Casino, hervidero de disputas, de informes falsos y verdaderos, de
ardientes comentarios, y al retirarse a su casa de la calle del Turco,
cuando apuntaba la rosada claridad de la aurora, sintió el hombre lo que
nunca había sentido: desdén de sí propio y de su patria. Su pesimismo se
concretaba en esta frase que dijo y repitió mil veces, hasta que sus
ideas fueron anegadas por el sueño: «Ni ella ni yo tenemos compostura.»

\hypertarget{x}{%
\chapter{X}\label{x}}

Sorpresa y disgusto causó al Marqués de Loarre la primera noticia que al
despertar, el día 14, le llevó a la cama su criado con el
\emph{Extraordinario de la Gaceta}. Leyó la lista de los Ministros del
flamante Gabinete de O'Donnell, y al ver \emph{Collado, Fomento, con la
dirección de Ultramar}, la impresión fue por demás penosa. Ya no debía
contar con el millonario, que chapuzándose en la política y en los
afanes de dos importantes ramos de Administración, pondría un paréntesis
en los negocios. No habría más remedio que proseguir arando la tierra en
busca del escondido capital, que para la compostura de su hacienda
necesitaba. Dinero había de sobra; mas no quería venir a la reparación
de las casas históricas, ocupado sin duda en demoler las que aún no se
habían caído. Al salir en busca de su amigo Beramendi para pedirle
sostén moral y consejos, atormentado iba por esta endiablada conjetura:
«¡A ver si ahora se le ocurre a Pepe Fajardo aprovechar la entrada de
Collado en la Dirección de Ultramar para mandarme a Cuba!\ldots{} ¡Qué
humillación!\ldots{} Mucho puede Pepe Fajardo sobre mí; pero no hará de
Guillermo de Aransis un vista de Aduanas\ldots»

Reuniéronse los dos amigos. Loarre propuso prescindir de Collado, y
continuar las diligencias del empréstito en otras casas; la misma idea
expresó Beramendi, y nada dijo del extremo recurso de Ultramar. Al
Congreso fueron los dos, creyendo encontrar allí grande animación,
concurrencia extraordinaria de diputados y charladores de política; mas
no vieron sino contadas personas, y en ellas, como en todo el ambiente
de la casa, desaliento y tristeza, con olor a miedo\ldots{} Así lo dijo
Fajardo, aproximándose a dos amigos suyos que platicaban con cierto
misterio arrimados a la pared del pasillo de entrada. «¿Se puede saber
qué pasa o qué pasará hoy?» Los dos señores, desconocidos para
Guillermo, respondieron a Fajardo que nada positivo sabían, y que lo
mismo podía venir en la tarde y noche próximas una descomunal batalla
entre el Progreso y la Reacción, que una ignominiosa tranquilidad. Todo
dependía de que el Duque se pusiera las botas, obediente a las
instancias de su partido y al estímulo de las ideas que representaba.
Uno de los señores que Guillermo desconocía era de edad avanzada, largo
de estatura y un si es no es agobiado de espaldas, de rostro áspero y
displicente, la mirada como de hombre a quien abruman las
contrariedades, sin hallar en su ánimo fuerzas para resolverlas o
sortearlas. Joven era el otro, de mediana talla, con barba negra y
corta, la boca extremada en dimensiones y como hecha para rasgarse
continuamente en un sonreír franco tirando a diabólico, el mirar vivo y
ardiente, el pelo bien compuesto, con raya lateral, y un mechón
arremolinado sobre la frente formando cresta de gallo.

«¿Quiénes son esos?---preguntó Aransis a su amigo, apartándose de aquel
grupo para pegarse a otro.

---El alto y viejo es un fanático progresista---replicó Fajardo,---de
los de acuñación antigua, y que ya van siendo raros, como las monedas de
veintiuno y cuartillo. Se llama Centurión, y no tiene más dios ni más
profeta que San Espartero. El otro es Sagasta, ¿no le conoces?; diputado
creo que por Zamora, hombre listo y simpático, que perorando ahí dentro
es la pura pólvora, y entre amigos una malva.»

Apenas llegaban los dos marqueses al primer grupo que veían, entrando en
el Salón de Conferencias, llegó Escosura, que al punto fue asaltado de
curiosos. Parecía enfermo; venía de mal temple. Aransis le oyó decir:
«Se lo he pedido casi de rodillas, y nada. No quiere ponerse al frente
de la Revolución\ldots{} Esto es entregar el País y la Libertad a
O'Donnell y a los del \emph{Contubernio.»} Centurión dio sobre esto, a
Beramendi y a su amigo, más claras explicaciones. El Duque, vencido por
O'Donnell en la guerra de intrigas, y desairado por la Reina, desmentía
su fogosidad y bravura, encerrándose en un quietismo incomprensible.
¿Qué significaba esta conducta? ¿Por qué procedía en forma tan contraria
a su historia el hombre que personificaba la Libertad, precisamente en
la ocasión en que tenía más medios de defenderla? «¿Qué dirán, Señor,
qué dirán los diez y ocho mil milicianos que están arma al brazo,
esperando oír la voz que ha de conducirles al barrido y escarmiento de
toda esta pillería del justo medio?\ldots{} Fíjese, Marqués, ¡diez y
ocho mil hombres! decididos a morir por la Libertad\ldots{} Y el Duque,
nuestro Duque, se cruza de brazos, ve impasible que la Revolución es
pisoteada, que el nuevo Código Político se queda en el claustro materno,
y nosotros, los buenos, desamparados y a merced de O'Donnell, que no
piensa más que en traernos ese ganado hambriento, ese pisto, Señor, de
moderados y apóstatas, cuyo ideal no es más que comer, comer,
comer\ldots»

Escosura dijo a Sagasta: «Vayan usted y Calvo Asensio a ver si le
convencen\ldots{} yo nada he podido.» Ya en este punto y hora, que era
la de las tres, iban llegando más diputados, y los divanes del Salón de
Conferencias, que desde la inauguración del edificio eran cómodo asiento
de gobernadores cesantes, de pretendientes crónicos o charladores por
afición y costumbre, se poblaban de vagos. Creyérase que los tales
habían nacido allí, o que no tenían más oficio ni otros fines de vida
que petrificarse sobre aquellos blandos terciopelos. Cuando el número de
diputados en la casa pasó de seis docenas, dispuso abrir la sesión el
vicepresidente don Pascual Madoz. Desairada, tirando a ridícula,
resultaba la reunión de los representantes del Pueblo, y fúnebres los
discursillos que allí se pronunciaron. Las Cortes Constituyentes
agonizaban. O'Donnell ni aun quería hacerles el honor de disolverlas
\emph{manu militari}. Se votó una proposición, en la que unos ochenta
caballeros declaraban que el Gobierno de don Leopoldo no les hacía
maldita gracia, y los que fueron en comisión a Palacio para llevar el
papelito volvieron con las orejas gachas, diciendo que O'Donnell, Ríos
Rosas y los demás Ministros nuevos les habían despedido con un cortés
puntapié\ldots{} Las Cortes se acababan, morían sin lucha y sin gloria,
abandonadas del caudillo que tenía el deber de defenderlas, y lloraban
su desdichada suerte frente a dieciocho mil hijos ingratos, que no
sabían disparar un tiro en defensa de su madre.

Los votantes de la proposición de censura iban desfilando hacia la
calle, con la idea de que más seguros estarían en su casa que allí, por
si a O'Donnell le daba la ventolera de meter tropas en el
\emph{establecimiento} con objeto de \emph{asegurar} al moribundo. Unos
treinta o cuarenta quedaban, firmes en los escaños, arrogantes ante su
menguado número, y votaron una proposición que en puridad decía:
«Hallándose amenazada la inmunidad de las Cortes\ldots{} confiamos a don
Baldomero Espartero el mando de las fuerzas necesarias a su defensa, a
cuyo fin se comunicará este decreto a todos los Cuerpos del Ejército y
Milicia Nacional, \emph{caeteraque gentium}\ldots» Y a los pocos
instantes de que fuera votado este acuerdo, a estilo de Convención, se
oyó claramente en todo el edificio ruido lejano de tiros, con lo que
algunos se alegraron viendo justificada la actitud de los firmantes de
la proposición, y celebraban la lucha, prólogo quizás de un airoso
morir, mientras otros, revistiéndose de prudencia, se escabullían hacia
las puertas de Floridablanca y el Florín, para ir a buscar el seguro de
sus casas.

Entró Centurión en el pasillo largo gritando: «Ya se armó. La Milicia se
bate, señores\ldots{} ¡En la Plaza de Santo Domingo, un fuego
horroroso!\ldots{} La Libertad puede morir; pero no deshonrarse en este
trance supremo, metiéndose debajo de las camas.

---¿Está el Duque al frente de los milicianos?---le preguntó Eugenio
García Ruiz, que era el más caliente de los diputados fieles a la
Representación Nacional; y Centurión dijo: «No lo sé; no puedo
afirmarlo\ldots{} lo presumo, sin más dato que el coraje con que ha roto
el fuego\ldots{} Tenemos Duque. Si aún dudara, la bravura de nuestro
pueblo armado le decidiría.» A este optimismo casi pueril opuso Sagasta
una de sus más delicadas sonrisas, y rascándose la barba, dijo a García
Ruiz: «No nos hagamos ilusiones; el Duque no se mueve más que para irse
a Logroño. Hemos estado a verle Calvo Asensio y yo, y nos ha
dicho\ldots{}

---¿Qué os ha dicho?\ldots{} ¿El \emph{cúmplase} de siempre? Es burlarse
de nosotros; es arrojar la Libertad, atada de pies y manos, a los pies
de los caballos de O'Donnell y Serrano. \emph{¡Cúmplase!}\ldots{} ¿Y a
cuándo espera?

---No sé---murmuró Sagasta acariciándose de nuevo la barba, cuyas hebras
sonaban levemente al rasgueo de sus uñas.

---¿Qué razón hay para esa calma increíble, para ese abandono de los
principios?\ldots{} ¡Él\ldots{} Espartero!---preguntaba García Ruiz
lleno de confusiones. Y el gran Centurión, no tan confuso como
indignado, reforzó la pregunta en la forma más colérica: «¿Qué razón
hay, cojondrios?

---Alguna razón hay---dijo Calvo Asensio ceñudo, frío.---No puede
ponerse el Duque en esa actitud sin alguna razón\ldots{} y razón de
peso, Eugenio\ldots{} Ya te la diré.»

Aransis y Beramendi, oyendo el fragor lejano de tiros a cada instante
más intenso, salieron a la puerta de Floridablanca y allí deliberaron
qué camino tomarían para la retirada. Proponía Guillermo que fueran a su
casa, calle del Turco, de la cual muy poco distaban. Pero como
insistiera Fajardo en ir a la suya, por no estar ausente de su familia
en días de trifulca, allá corrieron los dos, tomando la vuelta que
creían menos peligrosa. En el Congreso quedó Centurión, que si no era
diputado lo parecía, por el ardiente celo que mostraba, mirando la
dignidad de la Representación Nacional como la suya propia, y
desviviéndose porque fuese de todos honrada y enaltecida. En la misma
idea y tensión estaba García Ruiz, castellano viejo con toda la seca
testarudez de la raza, hombre de voluntad más que de fantasía,
calificado entonces entre los sectarios furibundos, y que no lo era
realmente, pues en él lucía la claridad del buen sentido, y habría dado
cuerpo a las ideas dentro de los moldes de la realidad, si se le
presentara ocasión de hacerlo. Nicolás Rivero, otro de los que allí
permanecían, trataba de infundir con su presencia un aliento más de vida
a las Cortes moribundas. Poca fe tenía ya en que la Institución saliera
bien de aquel soponcio, y como a difunta la miraba.
*«Zeñores---decía,---¿qué hacemos aquí? Velar el cadáver.» Y Madoz,
vehemente y práctico, como mestizo de catalán y aragonés, respondía:
«Pues velaremos por si le da la gana de resucitar, y estaremos al
cuidado de que no lo profanen.» Fernando Garrido, revolucionario
ardiente, partidario de los remedios heroicos, salía y entraba con
Centurión, trayendo noticias consoladoras: «La cosa va de veras. Hemos
visto a Manolo Becerra y a Sixto Cámara que van a ponerse al frente del
5.º de Ligeros\ldots{} En la Plaza de Santo Domingo se está levantando
una barricada formidable, que ha de dar algún disgusto a los de
Palacio\ldots{} Cuentan que en Palacio el pánico es horroroso\ldots{}
Hay tropa en Chamberí, tropa detrás del Retiro; pero muy
desalentada\ldots{} nos dicen que muy desalentada\ldots» El General
Infante, Presidente, ponía en duda lo del desaliento, y cuando llegó la
noche dormitaba en un sillón de su despacho. Seoane y Montemar volvieron
a la persecución de Espartero, que abandonando su casa se había
trasladado a la de Gurrea; y Sagasta y Calvo Asensio se mostraban
tristes y resignados, como hombres que, viendo con claridad las causas,
esperaban en calma los tristes efectos.

Así pasó la mayor parte de la noche, en expectación melancólica y
amodorrante, pues no se oían tiros próximos ni lejanos, ni llegaban al
Congreso indicios de haberse trabado una formal batalla entre nacionales
y tropa. Los diputados fieles, apegados por respeto y amor a la casa
paterna, con los aficionados políticos que les acompañaban en el duelo,
velaban dispersos aquí y allí, en grupos que se juntaron locuaces y se
disgregaban soñolientos. Las voces se extinguían; el salón de Sesiones y
el de Conferencias, alumbrados como para grandes escenas parlamentarias,
ostentaban su espléndida soledad de capilla ardiente\ldots{} Por fin, a
las últimas horas de la noche, que en aquella estación era muy corta,
empezó a manifestarse en los grupos alguna animación, por aires que
entraban de la calle, y personas que acudían al recinto
mortuorio\ldots{} De cuatro a cinco, el bullicio y animación crecieron
hasta el punto de que pudo decir Madoz: «¿Resucitaremos? ¡Vaya que si
resucitáramos!\ldots» A las seis, un intenso ruido, como el de las olas
del mar, indicó que grandes masas de gente ocupaban las calles próximas.
Oyéronse los mugidos de vivas y mueras, que son la espuma que salta en
el hinchado tumulto de las muchedumbres. Por las puertas de
Floridablanca y del Florín entraron hombres uniformados, con armas, y
otros que las llevaban sobre la ropa ordinaria de paisano, como los
cazadores que van al monte. Eran milicianos y guerrilleros de campo y
calle, que venían a ofrecerse a la Representación Nacional para su
custodia y defensa. Se dijo que las tropas mandadas por Serrano ocupaban
Recoletos; seguramente ocuparían el Prado. Venían a disolver, empresa
sencillísima dos horas antes, pues las Cortes no tenían a su lado más
que a los maceros; pero no muy fácil ya, con tanta gente decidida en su
recinto, y alguna más que vendría pronto y tomaría posiciones. El
interés del suceso histórico pasó del interior a las inmediaciones del
Congreso. Los milicianos, obedientes a jefes con uniforme o sin él, se
dirigían en secciones a las casas de Vistahermosa y Medinaceli, que
ocuparon, situándose en los aposentos de planta baja y desvanes\ldots{}
Tomó el mando de ellos el menos militar de los hombres, el de más
pacífica y bonachona estampa: don Pascual Madoz.

Ya el rubicundo Febo esparcía sus rayos por todo Madrid, cuando entre
las multitudes que invadían y cercaban el Palacio de las Cortes apareció
Espartero, no a caballo, con arreos y jactancia de caudillo que conduce
a sus prosélitos al combate, sino pedestremente, en traje civil. Dentro
y fuera de las Cortes echó breves peroratas con menor ahuecación de voz
que la comúnmente usada por él frente al pueblo, y terminaba con vivas a
la Libertad y a la Independencia nacional. Todo era una vana fórmula,
dedada de miel para entretener el ansia popular, o escape instintivo de
los cariños de su alma, que no podía contener\ldots{} A sus
exclamaciones respondió la patriotería con otras, y luego dio media
vuelta para tomar la calle de Floridablanca, en compañía de Montemar,
Gurrea y Seoane. Iría tal vez a ponerse las botas, a montar a caballo, a
sacar de la funda la espada gloriosa, panacea infalible contra las
enfermedades de la España Libre\ldots{} Esto creyeron algunos. Los
desconsolados ojos de los milicianos le vieron partir, y él desde lejos
espaciaba sobre la multitud una mirada triste. Se despedía para Logroño.

A Centurión faltábale poco para llorar; García Ruiz maldecía su suerte.
Calvo Asensio y Sagasta, melancólicos, arrojaban estas gotas de agua
fría sobre el ardiente afán de sus amigos: «No puede, no puede\ldots{}
Ya comprendéis que valor no le falta.

---Y con ponerse a la cabeza de la brava Milicia, y soltar cuatro tacos,
¡cojondrios! arrollaría fácilmente a nuestros enemigos, \emph{a los
eternos enemigos de la Libertad}.

---Sí, los arrollaría\ldots{} Caerían hechos polvo; pero con ellos
vendría también al suelo, rompiéndose en mil pedazos, el Trono,
señores\ldots{}

---¿Y qué?\ldots{}

---¡Oh!\ldots{} es pronto\ldots{} es grave\ldots{} Espartero no quiere
tal responsabilidad.

---¡Desgraciado país!\ldots»

Diciendo esto el que lo dijo, los cañones que Serrano había puesto en el
Tívoli empezaron a vomitar metralla contra Medinaceli, y granadas contra
las Cortes.

\hypertarget{xi}{%
\chapter{XI}\label{xi}}

Tenía Serrano, Capitán General de Madrid, lo que en Andalucía llaman
\emph{ángel}. Más que a su guapeza, por la que obtuvo de Real boca el
apodo de \emph{General bonito}, debía los éxitos a su afabilidad,
ciertamente compatible, en el caso suyo, con el valor militar temerario,
en ocasiones heroico. Fascinaba a las tropas con alocuciones
retumbantes, como las de Espartero, y las llevaba tras sí con el ejemplo
de su propia bravura, dando el pecho al peligro. Era, pues, un valiente,
no inferior a ninguno de los demás caudillos de nuestras luchas civiles,
perfecto guerrillero más que general, y con su valor, su buena estampa,
y la \emph{suerte}, que suele acompañar a los atrevidos en épocas de
revueltas y en países cuya legislación y costumbres no están
fundamentadas sobre sólidas instituciones, llegó muy joven a la cumbre
de la jerarquía militar\ldots{} Entiéndase que el valor de Serrano era
exclusivamente del orden guerrero, pues fuera de los dominios de Marte,
su voluntad desmayaba, haciéndose materia blanducha, fácilmente
adaptable a las formas sobre que caía. En él se marcaban con gran
relieve los caracteres de la generación política y militar a que le tocó
pertenecer. Todos en aquella especie o familia zoológica eran lo mismo:
los militares muy valientes, los paisanos muy retóricos, aquellos
echando el corazón por delante en los casos de guerra, estos enjaretando
discursos con perífrasis galanas o bravatas ampulosas, y cuando era
llegada la ocasión de hacer algo de provecho, todos resultaban fallidos,
y procedían como mujeres más o menos públicas.

No había lucido hasta entonces en Serrano ninguna cualidad de hombre
político. En este punto, nada tenía que envidiar a Narváez, que fuera de
algunos rasgos de energía, brotes repentinos de su temperamento, nada
estable había producido; ni a Espartero, que inició alguna suerte
lucida, puso en ella la mano, mas no supo o no pudo rematarla; ni a
O'Donnell, que hasta entonces no era más que un enigma. Quizás se
aproximaba el día en que la esfinge de Vicálvaro hablase, y de sus
palabras saliese algo práctico que nos trajera permanentes beneficios.
Serrano debió de creerlo así; fiaba en la eficacia de lo que llamaban
\emph{Unión Liberal}, la concentración de los hombres más listos y
presentables de los dos bandos históricos, y ofrecía su concurso a esta
obra fecunda. En su mano había puesto O'Donnell las tropas que debían
aniquilar a los diez y ocho mil milicianos mal contados. \emph{¡Santiago
y a ellos!} Serrano, ayudado por Dulce, hombre de coraje también, no
dudaba de la pronta dispersión de la chusma uniformada. Y al entrar en
los jardines del Tívoli, pensando en la seguridad de su triunfo, el
simpático General fue asaltado de escrúpulos y temores que no carecían
de lógico fundamento. «¡Estaría bueno---se decía---que después de dar
nosotros la cara para echar al Duque y de cargar con la impopularidad
del desarme del Pueblo, nos salga Palacio con alguna mala partida, y nos
mande a paseo, y llame \emph{al divino Narváez}, para que nos ponga a
todos el \emph{Inri!»}

Conocía muy bien el salado General la veleidosa condición de la Reina,
sus sarcasmos y disimulos, heredados de Fernando VII, y sus preferencias
por la política moderada; conocía también, y mejor que nadie, la
flaqueza del corazón de Isabel ante las taimadas sugestiones de una
beata embaucadora; sabía que fácilmente se ganaba la Real voluntad, no
siendo en aquel nebuloso terreno. Isabel podía desechar el temor del
Infierno por sus personales culpas; pero no por el pecado de consentir
que su pueblo cayese en los abismos del descreimiento y la corrupción
masónica. En esto tan sólo era consistente su voluntad; en lo demás se
desmenuzaba, reduciéndose a migajas que el viento esparcía. Constábale
asimismo a Serrano que Isabel II, en sus juicios aguda y cruel, mordaz
en sus calificativos, se había dejado decir que \emph{unos cuantos
malhechores y rufianes jugaron a cara o cruz la dinastía en el Campo de
Guardias}\ldots{} Y el General discurrió así: «Yo no estuve en el Campo
de Guardias; pero de fijo me comprende en el número de los rufianes que
jugaron\ldots{} En fin, ya sabremos en qué parará esto. ¡Ay, O'Donnell
de mi alma! Si hemos de hacer algo de provecho, es menester que al
soltar las espadas tomemos cada cual un cirio\ldots{} Transacción es
esto, que no fanatismo\ldots{} O transigir, o\ldots»

Quedó en el aire el pensamiento del Capitán General de Madrid. La
realidad que traía entre manos absorbió por completo su atención.
Pensando juiciosamente que la mejor táctica era infundir terror, así en
los nacionales, como en los diputados que aún sostenían en el Congreso
una farsa de representación, mandó situar en puntos convenientes la
artillería que acababa de llegar del vecino parque, y dio órdenes de
fuego. Apenas iniciados los terribles zambombazos contra el Congreso y
la Milicia, se retiró al fondo del jardín. En hora tan temprana, pues
aún no eran las ocho, el calor sofocaba. Habían dispuesto los ayudantes,
sobre una mesa de despintado pino, agua, refrescos y aguardiente de
Chinchón. Los oficiales que estaban en pie desde antes de media noche,
acudían allí a tomar la mañana y a calmar su sed. Otros, en pie junto a
los árboles, se desayunaban con fiambres que sacaban de papeles
grasientos.

Dio Serrano concluyentes órdenes a varios Jefes de Cuerpo, que partieron
al punto. Uno de ellos, el Coronel Villaescusa, acompañado de un
Teniente Coronel de su regimiento, pasó al patio grande del Buen Retiro,
donde los dos habían dejado sus caballos: montaron; picaron espuelas
hacia la calle de Alcalá, atravesando por las arboledas del Retiro. Iba
el buen Coronel, no digamos de mal talante, porque esto no expresará su
rabiosa desazón, sino dado a los demonios, que en su cuerpo furiosamente
se habían metido. Atacado el infeliz señor de su mal crónico del
estómago, sentía que en esta víscera tenía su instalación todo el
infierno, por el tormento que le daban dolores agudísimos y el fuego que
en sus entrañas ardía. Necesitaba de una entereza, más que heroica,
sobrehumana, para sostenerse en el caballo y dar cumplimiento a las
órdenes del General. Estas fueron así: «Con el batallón que tiene usted
en el Ministerio de la Guerra, cuidará de mantener libre la calle de
Alcalá. Dos piezas de artillería que he mandado situar entre el Palacio
de Alcañices y la Inspección de Milicias, cañonearán a los milicianos
que enredan por la calle de Alcalá, y hacen fuego desde los tejados de
algunas casas. Cierre usted las entradas del Barquillo, de las Torres y
Peligros; ocupe el Caballero de Gracia si no le hostilizan mucho desde
los balcones; ocupe también la Plaza de Bilbao\ldots{} Los efectos de la
artillería nos lo darán todo hecho. A los milicianos que se retiren
hacia los barrios del Norte, se les desarma tranquilamente. Creo que no
han de oponer resistencia. Si se resistieran, usted sabe lo que tiene
que hacer. Si en las Vallecas o en Calatravas sacaran algún cañoncillo,
de esos que les sirven de juguete, quitárselo, cueste lo que cueste, que
mucho no costará\ldots{} El segundo batallón, que siga en Santa Bárbara,
Fábrica de Tapices y la Ronda, no permitiendo que salgan milicianos
armados, ni que entren víveres de ninguna clase\ldots{} Adiós, y
aliviarse, que eso no será nada.»

No digamos que trinaba el Coronel, sino que del alma le salían rayos y
truenos, y que furioso los masticaba, tragándoselos después envueltos en
horrible amargura. Era un hombre de buena presencia, de faz morena y
curtida, que con la terrible enfermedad había tomado color terroso; los
ojos negros, el pelo y bigote con canas prematuras. En el Ministerio de
la Guerra dio sus órdenes con la mayor concisión posible, apretando los
dientes, como si cortando las frases pudiese partir en dos el dolor que
le atenazaba. Salió a recorrer las posiciones de Caballero de Gracia y
Plaza de Bilbao, mostrando a sus subordinados un rostro de severidad
aterradora, y una tiesura embalsamada, como la del cadáver del Cid
cuando lo montaron en la silla para que a los moros dispersara,
remedando en la muerte el miedo que vivo infundía su presencia. Daba
cumplimiento exacto a las disposiciones del General, reservándose la
facultad de alterarlas con libre iniciativa, si las circunstancias así
lo reclamaban; exigía la observancia fiel, con maldiciones secas; la
crudeza militar ponía en su boca rayos del cielo y resplandores de los
abismos\ldots{} Viendo a sus tropas tirotearse, en la parte baja de la
calle de San Miguel, con los milicianos que ocupaban una casa en el
Caballero de Gracia, infirió groseras ofensas a Dios, a la Virgen y a
venerables Santos\ldots{} Pasó tiempo\ldots{} Al saber que los suyos
habían dejado pasar un cañoncillo de mala muerte, en la calle de
Peligros, pronunció frases altamente ofensivas para la Santísima
Trinidad, para el Copón y las Once mil vírgenes. De estas sacrílegas
exclamaciones no era responsable el pobre Don Andrés, pues las
pronunciaba como una máquina, en las horribles embestidas del demonio
que dentro de sí llevaba.

Despejada de enemigos la calle de Alcalá, la recorrió Villaescusa desde
el Depósito Hidrográfico hasta donde estaban los cañones, mudos ya. Allí
supo la eficacia de la metralla y bombas disparadas contra los
milicianos de Vistahermosa y Medinaceli, y contra el Congreso. Una
granada, penetrando por la claraboya del Salón de Sesiones, pidió la
palabra con horrendo estallido en medio del hemiciclo, diciendo a los
buenos señores allí presentes que se fueran a sus casas y no se metieran
en más dibujos parlamentarios.

«No dijo eso, no dijo eso---clamó rabioso el Coronel, arrojando toda
clase de inmundas materias sobre el Verbo Divino, sobre el Arca de Noé,
y también sobre las Once mil vírgenes, por quienes, en sus furibundos
desahogos, tenía una predilección especial.

---¿Pues qué dijo, mi Coronel?

---Lo contrario, enteramente lo contrario---replicó, cual si en aquel
doloroso estado no tuviera más consuelo que la contradicción\ldots{}

---¿Pero se acaba esto? ¿Estaremos aquí hasta mañana, por estos títeres
de la Milicia?»

Oyendo decir luego que el Presidente de las Cortes, General Infante,
había pedido parlamento a Serrano, Villaescusa no dio crédito a la
noticia, y como le aseguraran por testimonio \emph{de visu} que en aquel
momento trataban Serrano y Dulce, con Infante y los Jefes de la Milicia,
de la suspensión de hostilidades, el Coronel trincó los dientes, se alzó
un poco sobre los estribos, y con voces iracundas, entre las cuales no
faltaban feas alusiones a San Pedro, a San Basilio y a otros personajes
de la Corte celestial, dijo y repitió: «No puede ser; sostengo que no
puede ser\ldots{} Esto no acabará más que matando al perro, para que se
acabe la rabia. Despoblar el mundo, digo yo, y así no habrá
tontos\ldots»

Los sufrimientos del pobre señor, que toda la mañana habían sido
intolerables, se aplacaron un poco después de mediodía. Corto era el
alivio; pero aun así lo acogió el pobre enfermo con regocijo y gratitud,
no dejando por eso de apostrofar suciamente a todas las potencias del
cielo y de los abismos\ldots{} Tronaba también contra el Gobierno,
inculpándole por la prisa con que le trajo a Madrid, y le metió en fuego
sin darle ni aun horas de descanso. Tanta fatiga y ajetreo provocaron el
ataque, de una violencia superior a cuantos había sufrido. Al llegar a
Leganés en la noche del 15, se iniciaron los dolores, y pasó una cruel
noche, creyendo que se moría y deseando la muerte, único remedio, a su
parecer, de tan inveterado y perverso mal. Aliviado a la mañana
siguiente, fue a Madrid con objeto de ver a su familia y aun de
abrazarla, que en su decaimiento le halagaba la idea de los abrazos; por
el camino acarició el propósito de presentarse a O'Donnell, exponerle el
mal que le atormentaba, y pedirle que le relevase de las obligaciones
militares por unos días, los necesarios para reponerse. Llegó a su casa
serían las diez, y cuando a la puerta llamaba con la ilusión de
encontrar allí consuelo y alegría, fue sorprendido por este jicarazo con
que le recibió la criada: «La señora y la señorita no están.»

Entró, dio varias vueltas por el recibimiento y sala, diciendo: «¿Y a
dónde se han ido esas\ldots?» Terminó con grosería cruel, a la que
siguieron los acostumbrados anatemas contra las cosas divinas.

\hypertarget{xii}{%
\chapter{XII}\label{xii}}

«Han ido de campo con la señorita Valeria, y no volverán hasta mañana
por la noche---dijo la muchacha, acostumbrada ya, por su largo servicio,
al bárbaro estilo del señor en sus ratos de ira. Preguntole después si
quería acostarse, si almorzar quería, y añadió que si le molestaba el
dolor de estómago, le haría una taza de la hierba que el señor quisiera.
A todo contestó con formidable negativa, y con mandar a la moza que se
fuera corriendo a semejante parte\ldots{} Salió el Coronel de estampía,
y de la fuerza del coraje sobre los nervios y de estos sobre otras
partes del organismo, se le calmó el dolor. Bajando la escalera,
rabioso, y aliviado hasta sentirse bien, pensó que no debía pedir
descanso al Ministro de la Guerra. Era poco airoso y de mal gusto estar
enfermo en día de combate. Cumpliría los deberes que el honor le
imponía, y confiaba en la remisión del ataque por lo de \emph{similia
similibus}, o sea por la virtud de un enérgico berrinche.

Dos horas después entraba en Madrid y se acuartelaba en San Francisco el
Regimiento mandado por Villaescusa. Este se puso al frente. Algunas
horas de descanso en el cuarto de banderas le aseguraron, al parecer, el
alivio. Pero a las doce de la noche, al montar a caballo para situarse,
según orden superior, en el Ministerio de la Guerra, se vio nuevamente
acometido con mayor violencia y sufrimientos más agudos. Hizo de tripas
corazón, y del riguroso deber fortaleza, en la cual se encastillaba,
tratando de engañar el dolor físico con la satisfacción de conciencia.
Así estuvo todo el día, firme en su puesto, atormentado, mas no vencido,
por las mordeduras del monstruo que llevaba en sus entrañas. Al caer de
la tarde, cuando ya la insurrección, o lo que fuese, parecía dominada,
los sufrimientos de Villaescusa eran tales, que apenas podía ya contra
ellos la entereza militar. Difícilmente se sostenía en el caballo, y las
tremendas imprecaciones, las injurias a lo divino y lo humano, que
ayudaban a robustecer la voluntad, perdían ya su eficacia. Con
sobrehumano esfuerzo recorrió la extensa línea que el primer batallón
ocupaba, Plaza de Bilbao, Red de San Luis, Jacometrezo, Postigo de San
Martín, hasta la Plazuela de las Descalzas, y viendo que todo iba bien y
que los milicianos entregaban aquí y allí sus armas con menguada
resistencia en algunos puntos, mansamente en otros, todo lo miraba como
si fuera mal, y a los que debía elogiar los reñía, y su cara parecía el
símbolo de la suprema severidad y de la fiereza.

En la Red de San Luis conferenció Villaescusa con el Coronel
Mageniz\ldots{} Minutos después de la conferencia no recordaba lo que
hablaron; persistía en la mente de don Andrés la idea de que las Cortes
se habían suspendido con la fórmula de \emph{se avisará a
domicilio}\ldots{} y recordando esto, decía: «No puede ser\ldots{} yo lo
pongo en duda, yo lo niego\ldots» Bajó hacia la Cibeles, casi sin darse
cuenta de la dirección que a su caballo señalaba con las riendas. Allí
se encontró al Coronel Berruezo, de Artillería, el cual, conociendo en
el rostro de su amigo los sufrimientos que le abrumaban, le recomendó el
sosiego. Bien podía resignar el mando en el Teniente Coronel Zayas, y
retirarse a su casa. «¡A mi casa, sí!» balbució Villaescusa, que en el
paroxismo de sus dolores sentía ganas de llorar como un niño\ldots{}
Berruezo añadió que a enfermos y sanos convenía tomar algo de alimento,
pues no hay cosa peor que entregar nuestro cuerpo al desgaste orgánico
sin reparar de algún modo las pérdidas, y terminó con este récipe
substancioso: «Hemos preparado ahí, en la sala baja de la Inspección, un
tente en pie, comida pobre, de plaza sitiada\ldots{} poca cosa. Amigo
Villaescusa, contamos con usted. Pues nada o muy poco tenemos que hacer
ya, apéese usted, que yo haré lo mismo. Las nuevas órdenes de Serrano
las recibiremos aquí, y puede que venga él mismo a dárnoslas, comiendo
con nosotros. Con que\ldots{}

---Comer, comer\ldots---murmuró Villaescusa rabiando.---¿Y sé yo acaso
cómo se come, con este infierno que llevo aquí, en el buche, y estos
rayos que me suben al pecho, y este acíbar en la boca?» El dolor
lacerante del estómago era tan pronto mordedura de dientes agudísimos,
como chisporroteo de las entrañas taladradas por un hierro candente.
Trincando las encías con fuerza, apretando las piernas contra la silla,
y conteniendo la respiración, el paciente lograba por un instante
adormecer al monstruo. Este recobraba su imperio, mordiendo y quemando
por el esófago arriba, o bajándose hasta desgarrar con sus afiladas uñas
la vejiga. El corazón aterrado negábase a funcionar; temblaba toda la
máquina; recibía el cerebro olas de sangre fugitiva, y anegado se
quedaba sin pensamiento y sin memoria. Duraba segundos no más el efecto
congestivo, y luego venían otros penosos efectos. El dolor, el monstruo
llamaba a sí toda la sangre\ldots{} hormigueaban las manos; la lengua se
pegaba al paladar, seca y estropajosa\ldots{} Al delirio llegaba el
aborrecimiento del paciente a la Divinidad, así cristiana como gentil, y
el desprecio de todo el Género Humano era en él un amargo sentimiento
que por su intensidad en placer casi se convertía. A su hija y a su
mujer no las exceptuaba Villaescusa de este menosprecio y desestimación.
Las veía como dos pobres pulgas que andaban brincando de cuerpo en
cuerpo, en busca de un poco de sangre con que nutrirse.

Se apeó el Coronel, asistido de un ordenanza de la Inspección, el cual
le echó mano al cuerpo para que no se desplomase antes de poner el pie
en el suelo. No agradeció al parecer el pobre Villaescusa este cuidado,
porque en breves y cortados términos, confundidos con el nombre de Dios
en mala guisa, reprendió al subalterno por haberle casi cogido en
brazos\ldots{} ¡Le había lastimado un muslo, le había hundido una
costilla, dos\ldots{} mala peste con las Once mil vírgenes!\ldots{}
Entró tambaleándose\ldots{} A fuerza de metodizar sus pasos, guardaba un
imperfecto equilibrio, atento a las paredes para ampararse de ellas con
una o con otra mano, en caso de necesidad. Traspasó al fin el portal;
entró luego en una estancia, a mano derecha, donde vio claridad de
bujías (ya era casi de noche), una mesa puesta con más botellas que
platos y adorno de flores mustias, y algunos oficiales que hablaban
agrupados en un rincón. Saludó Villaescusa agarrándose a la primera
silla que encontró a mano, para disimular el peligro en que estaba de
caer al suelo\ldots{} Una vez salvado de aquel riesgo, pensó si se
sentaría o no. Decidiose por lo primero, y al desplomarse sobre el
asiento, los dolores horrorosamente se avivaron\ldots{} Apretó los
dientes; fingió cansancio, calor; se limpió el sudor del rostro\ldots{}
Un Oficial se le acercó. Debía de ser un amigo; pero tal estaba
Villaescusa, que a nadie quería conocer ya. Como ruido de moscardón sonó
en sus oídos la voz del Oficial, refiriéndole el fin de la página
histórica de aquel día. La Milicia estaba ya sin armas, salvo algunos
elementos levantiscos, \emph{los eternos enemigos de la tranquilidad
pública}, que sostendrían durante la noche una lucha estéril en los
barrios del Sur\ldots{} O'Donnell era ya el amo de la situación.
Serrano, el saladísimo General Serrano, y el bizarro Dulce, con las
fuerzas del Ejército a sus órdenes, acababan de prestar un gran servicio
a la Libertad y al Trono\ldots{} Habría forzosamente recompensas\ldots{}
Terminada felizmente la Revolución de este año, podríamos decir:
«Señores, hasta el año que viene.»

De este vano sermón histórico poco o nada entendió el mártir. Miró al
Oficial queriendo decir algo, pero sin poder articular sílaba\ldots{}
Las palabras, temerosas de ser pronunciadas con torpeza, se quedaban de
labios para adentro. Sorprendiose el Oficial de ver que en los ojos del
Coronel brillaban lágrimas, y que hinchadas estas, y no cabiendo en los
párpados, rodaban por las rugosas mejillas de color de tierra\ldots{}
Villaescusa no decía nada. Daba rienda suelta a sus ganas de llorar,
como un niño afligido y mudo. El Oficial, inclinándose sobre él, le
dijo: «Mi Coronel\ldots{} ¿dolor de muelas?» Respondió el mártir con un
movimiento de cabeza. El Oficial le ofreció vino, aguardiente, agua.
Cualquiera de estas cosas que bebiese, pensó don Andrés que se
convertirían en fuego al pasar por su boca: lo sabía por dolorosa
experiencia. Pero tuvo el antojo de tomar agua con vino: con signos lo
manifestó al que tan galanamente le servía. Bebió gran cantidad de vino
aguado, y al dejar el vaso en la mesa con golpe furibundo, una vivísima
flexión del monstruo que llevaba dentro le hizo ponerse en pie. Algo que
estaba doblado en las entrañas se desdobló, con juego de muelles que
horrorosamente dolían\ldots{} Viéndole tan demudado y con cierto
desvarío en los ojos, que ya se habían secado de lágrimas, el Oficial le
indicó que podía descansar en un sillón de cuero colocado a la otra
parte de la mesa. Villaescusa, andando con paso lento y bien marcado
hacia la puerta próxima, entrada de un largo pasillo, dijo con no poca
dificultad: «Sí\ldots{} Vuelvo.»

Internose el mártir por el pasillo, tocando la pared más próxima con una
de sus manos, y encontró a un ordenanza que al paso le saludó; luego a
un Oficial\ldots{} después a un perrito que le cedió el paso. Sentía un
calor tan sofocante en todo su cuerpo, como si llamas corrieran por sus
venas. La fiebre intensa le dificultaba la respiración, le turbaba el
entendimiento, quería también imposibilitarle el paso; pero él, con
extremada erección de la voluntad, se sostuvo. Ya no sólo era mártir,
sino héroe. En su turbación mental, no pensaba más que esto: «Todo menos
caerme\ldots{} caer nunca\ldots» Encontrose en una estancia sombría y
anchurosa, en la cual no vio más que libros, rimeros de tomos verdes,
todos iguales, como colección de \emph{Gacetas} o cosa tal, y en la
pared retratos viejos de generales con peto rojo cruzado de bandas, el
rostro afeitado, la cabeza cana. No había luz de lámparas ni de bujías,
ni otra claridad que la del moribundo rayo crepuscular que por dos
grandes balcones penetraba. Hacia uno de ellos se encaminó el Coronel,
que ya veía los objetos desfigurados por su trastornada mente, y sólo
pensaba que sus acerbos dolores se adherían más a él con feroces dientes
para devorarle y consumirle. Vio al través de los cristales árboles
raquíticos; no vio que, al pie de ellos, unos cuantos caballos de jefes
y oficiales generales comían tranquilamente su pienso, colgado el saco
de sus propias cabezas. Entre ellos andaban ordenanzas y carreteros, que
reían y parloteaban frívolamente. Caballos y hombres tomaron a los ojos
del desdichado enfermo figura y voz distintas de las reales. Sus
extraviados sentidos hiciéronle ver a su esposa y a su hija, que de un
bosquete salían, más que risueñas, riendo a carcajadas, y hacia él se
encaminaban con paso que parecía de danza más que andar decoroso de
personas formales. Lo que las quiméricas imágenes de las dos hembras le
dijeron o quisieron decirle, no lo oyó don Andrés\ldots{} lo adivinaba
quizás por el mover de labios y el gesto expresivo. Ello es que arrimó
su rostro a los cristales, desgranando sobre ellos sílabas balbucientes
que, interpretadas por derecho, podrían decir: «¡Mujeres de Madrid! aquí
estoy. Vosotras reís\ldots{} yo también, porque me voy y os dejo el
dolor, mi dolor\ldots{} Aquí os lo dejo\ldots{} Venid por él\ldots{} Ya
veis que yo también me río\ldots{} ¡Qué gusto quitarme este
perro\ldots{} dejároslo!\ldots{} Pobrecitas, reíd, reíd.» No podía matar
a su enemigo, el terrible monstruo que le devoraba; pero sí desprenderse
de él, obligándole a que abriera la feroz boca y soltara su presa. El
instrumento de abrir bocas de monstruos era la pistola que el Coronel
llevaba al cinto, y que cogió con mano firme. Aplicado el cañón a la
sien, salió el tiro, y el mártir dejó de serlo.

\hypertarget{xiii}{%
\chapter{XIII}\label{xiii}}

En gran desolación y necesidad quedaron Manolita y Teresa con la trágica
muerte del Coronel. Por muchos días, su casa fue un jubileo de visitas;
las personas doloridas o que fingían el dolor desfilaban vestidas de
negro, dejando en los oídos de la huérfana y la viuda suspiradas
frasecillas, con rumor semejante al del vuelo de las moscas. La
situación económica de la familia era poco halagüeña, porque la viudedad
de la Coronela, unos quinientos reales al mes, no resolvía ni el
problema primario de alimentarse y vestirse las dos mujeres, ni menos
los secundarios problemas que a casa traía la viuda con sus trapicheos,
y los despilfarros consiguientes. En vida de don Andrés ya eran grandes
los atrasos, y Manolita empleaba todo su arte y astucia para ocultarlos
a su marido. Después de la desgracia, la gravedad de la situación se
centuplicaba, por las derivaciones de la desgracia misma en el orden
social. La desamparada familia no tenía más remedio que vestirse de
cerrado y decoroso luto. El papel en que escribían alguna carta había de
tener orla negra, y negras habían de ser asimismo las cartulinas que
para visitas y otras mundanas etiquetas eran necesarias. ¡Qué diría la
sociedad si no veía en derredor de la familia todo aquel aparato de
negrura y tristeza! La huérfana y la viuda, que apenas tenían para
comer, y obligadas vivían a una representación pública incompatible con
su menguado haber, eran en realidad más infelices y más pobres que las
últimas vendedoras de hortalizas en medio de la calle.

Gran desdicha fue que Teresa no se hubiera casado antes del desastre, y
casarla después, ya tan baqueteada y manoseada de novios, había de ser
obra de romanos. Por de pronto, hija y madre tenían que vestir y
calzarse como Dios mandaba, pues no era cosa de andar por la calle mal
trajeadas y con los zapatos rotos. Manolita, pasándose de previsora, no
bien cobró la primera paga de viudedad quiso proveerse para los meses
futuros, y solicitó de Gregorio Fajardo que le hiciera un empréstito,
reteniendo su pensión. No quiso meterse en ello Gregorio (que si estos
negocios feos habían sido la base de su engrandecimiento, ya picaba más
alto), y endosó el asunto a un machacante de estas cosas, el cual fue a
ver a Manolita, y trató con ella en condiciones tan duras, que la
desconsolada señora no quiso aceptarlas. A Centurión no recurría ya,
porque agotadas estaban la paciencia y el bolso del primo de
Villaescusa, que sobre tantas socaliñas anteriores a la muerte de
Andrés, había tenido que atender, haciendo de tripas corazón, a las más
urgentes necesidades en los días de la tragedia. Y la razón que daba
para llamarse Andana era de las que no tenían réplica. «Ya ves,
hija---le decía:---estoy como el alma de Garibay, entre el ser y el no
ser, esperando a cada instante la cesantía, pues sé que O'Donnell me
tiene una tirria espantosa. Y aunque mi jefe, el señor Pastor Díaz,
parece que algo estima mis servicios en la Obra Pía, no me llega la
camisa al cuerpo. La cesantía, nueva espada de Damocles, pende sobre mi
pobre cabeza\ldots{} Ahorros no hay. ¿Cómo quieres que te socorra, si el
mejor día no tendré para dar a mi pobre Celia una triste taza de caldo?
Ten paciencia, hija, y arréglate como puedas.»

Así lo hizo Manolita, que aun sin consejos tan sabios, buscaba su
arreglo como y donde podía, gracias a su diligencia y a lo bien que
brujuleaba fuera de casa en obscuras campañas tras el dinero, teniendo
que pignorar su agradable persona con la mayor ventaja posible, según
las condiciones del mercado. Mala época era el estío para ciertos
arreglos, porque casi todos los ricos estaban en baños, o recluidos con
sus honestas familias en alguna casa de campo. Pero aun luchando con los
rigores de la estación, la viuda supo allegar para vestirse bien y
vestir a su hija, y comer ambas con menos miseria de la que su triste
soledad les imponía.

Muy solita estuvo Teresa todo el verano, y acometida de tristezas
lúgubres, porque Valeria, su íntima amiga, se fue a la Granja. Los
novios con buen fin que en aquella sosa temporada le propuso su madre,
eran todos de mal pelaje, esmirriados y pobres\ldots{} Pensaba en aquel
don Sixto, el de la bonita barba rubia; pero no extrañaba su
desaparición, porque ya sabía que anduvo en las calles batiéndose como
un tigre contra las tropas del Gobierno. Probablemente, o le habían
llevado a un presidio, o andaba oculto entre polvo y telarañas. Pero a
ninguno de sus conocimientos echaba tan de menos Teresita como a
Guillermo de Aransis, que también se había largado a tomar el fresco a
San Ildefonso. ¡Vaya un verde que se estaban dando Valeria y él! ¡Qué
paseítos por los pinares; qué subiditas a los montes, en amor y compaña,
sin testigos, y qué bajaditas a los profundos, solitarios barrancos!
Agua se le hacía la boca pensando en esto, y no dejaba de considerar que
no era la señora de Navascués mujer de mérito proporcionado a tanta
dicha\ldots{} Soñando, más que pensando, decía Teresa: «¿Por qué no
tendré yo también un marido en Filipinas, ya que aquí está visto que no
puedo tenerlo?»

El regreso de Valeria y del Marqués de Loarre puso fin a estas
nostalgias. Volvieron las dos amigas a su cariñosa intimidad, y en ella
vivieron algunos días hasta que llegó uno desgraciado en que aquella
venturosa concordia tuvo su término. Sucedió que Valeria, ordinariamente
muy habladora y con bastante desahogo para tratar todos los asuntos, dio
una mañana en hablar de moral privada y pública, de sobremesa del
almuerzo, y allí sacó unas teorías y unos escrúpulos que a Teresa le
parecieron el colmo de la sutileza. Todo a las casadas se podía
perdonar; nada a las solteras\ldots{} Protestó Teresita, dándose por
aludida y exigiendo a su amiga que declarase si la tenía por soltera
escandalosa. Contestó Valeria que no; pero que no bastaba ser buena;
había que parecerlo, y acabó por decir: «Eres honesta; pero tu madre
arroja sobre ti una sombra mala, que te hace pasar por lo que no eres, y
con esa sombra no podrás encontrar marido que no sea un perdulario sin
vergüenza.» Palideció Teresa; luego se puso muy colorada, y acabó por
echarse a llorar. Quiso la otra enmendar su impertinencia con
expresiones agridulces; pero ya era tarde. Teresa, que tenía su alma en
su almario, y no se mordía la lengua, tronó contra Valeria en esta
destemplada forma: «Mi madre es una pobre viuda sin recursos\ldots{} Ya
sé que no es buena\ldots{} Por desgracia mía, conozco todos los malos
pasos de mi madre. Ella, de algún tiempo acá, no se cuida mucho de
ocultarlos\ldots{} La pobre no tiene valor, no tiene virtud para
resignarse a la miseria\ldots{} Yo no puedo acusarla: soy su
hija\ldots{} Pero sí puedo decir que peor que ella eres tú\ldots{} Mi
padre, atormentado de un cáncer, se mató\ldots{} Si hubiera vivido, ni a
mi madre ni a mí se nos habría ocurrido mandarle a Filipinas, para
quedarnos libres\ldots{}

---Mira lo que dices---clamó Valeria descompuesta, cogiendo un plato y
amenazando con él la cabeza de la que momentos antes era su amiga.»

Animosa y creciéndose al castigo, Teresa cogió la cafetera y el
azucarero, una cosa en cada mano, y con flemático valor apuntó a la
dueña de la casa, diciendo: «Mira lo que haces, Valeria. Deja ese plato,
o no quedará en la mesa un solo chirimbolo que no vaya contra tu cabeza.
Me has ofendido y tengo que ofenderte\ldots{} Pues digo que eres peor
que mi madre, porque eres rica, y no tienes que luchar contra la
miseria. En la miseria quisiera yo ver lo que tú hacías\ldots{} Mi madre
enviudó por una desgracia, y tú te has enviudado a ti misma embarcando a
tu marido para \emph{el país de las monas}.

---Eso no es cuenta tuya---dijo Valeria, batiéndose en retirada,
haciendo pucheros\ldots---Y por lo otro, Teresa; por lo que dije de la
moral y de la sombra de tu madre, haz cuenta que yo no creía nada malo
de ti\ldots{} No fue eso lo que dije.

---Podías haber añadido que más que la sombra de mi madre me ha dañado
la tuya, Valeria: te lo digo sin resquemor\ldots{} Ya se me está pasando
el berrinche\ldots{}

---Siento que mi sombra haya sido mala para ti---dijo Valeria en pie,
atufándose otra vez, pero sin agarrar plato ni taza.---Bien te he
querido, Teresa; bien de sacrificios he sabido hacer por ti\ldots{}

---Y yo te lo agradezco---respondió Teresa, que ya no pensaba más que en
coger su mantilla para salir de la casa.---Pero antes que me recuerdes
tus favores, tus regalitos, quiero retirarme\ldots{} Yo soy pobre y no
he podido corresponderte; pero tanto como pobre soy orgullosa y no me
gusta que me humillen.

---Haces bien\ldots{} busca mejor sombra que la mía\ldots{} No dudo que
la encontrarás.

---¡Vaya si la encontraré!\ldots{} Yo te juro que no he de tardar
mucho\ldots{} Entre los favores que te debo, los más de agradecer son
tus lecciones\ldots{} las lecciones que me has dado para buscar
sombras.»

Frente a frente las dos, separadas por la mesa, que un campo de
Agramante parecía, con el azucarero volcado, las cucharillas dispersas,
las tazas ennegrecidas interiormente por el poso del café, el mantel
arrugado, se disparaban su ira con flechazo irónico, imitando a las
mujeres de rompe y rasga que se injurian graciosas antes de venir a las
manos. Valeria mandó a su criada que trajese la mantilla de la señorita
Teresa, y a esta dijo con retintín: «Vete, vete, sí; no se te escape la
sombra que buscas\ldots{}

---No se escapa. Lo que temo es que sea yo más torpe como discípula que
tú como maestra\ldots{} No tengo costumbre\ldots{}

---La niña inocente no sabe nada\ldots{} ¡Si será torpe!\ldots{} ¡Con
toda la Universidad en casa\ldots!

---Puede que esté allí la Universidad; pero me falta el libro de
texto\ldots{}

---El tuyo, los tuyos, Teresa, en la calle los encontrarás.

---O no\ldots{} Cállate, Valeria, si quieres que yo me calle. He sido tu
amiga; ya no lo soy.

---Volverás cuando me necesites.

---No digo que no. Puede que vuelva y no te encuentre. ¡Quién sabe a
dónde irás tú a parar!\ldots»

Decía esto la de Villaescusa nerviosa y trémula, de la ira y confusión
que removían toda su alma. No acertaba a ponerse la mantilla. Creyérase
que sus manos no encontraban la cabeza en el sitio de costumbre: la
buscaban más arriba\ldots{} Por fin, puesta como Dios quiso la mantilla,
y pronunciando un \emph{adiós} seco, tomó la puerta del comedor y luego
la de la escalera, no sin tropezar con algún mueble en su carrera
desmandada. A saltos bajó la escalera y se puso en la calle, con paso de
fugitiva o de esclava que rompe sus cadenas. Sorprendidos los porteros
de verla partir con andares y viveza tan contrarios al encogimiento
señoritil, salieron a la puerta para ver qué dirección tomaba. Fue hacia
la calle de Alcalá, camino de su casa sin duda, pues vivía en la calle
de las Huertas. Era la primera vez que salía sola, contraviniendo la
española costumbre que prohíbe a las solteras dejarse ver en público sin
compañía de alguno de la familia, o de servidores de confianza. Siempre
que iba de la casa de Valeria a la suya, llevaba una criada vieja o
moza, que cualquier edad servía para esta función. Pero ya, por decreto
del Destino, se había roto la rancia costumbre, motivada del poco
miramiento que en nuestra raza suelen guardar al sexo débil los
individuos del que llamamos fuerte.

Atravesada la calle de Alcalá para embocar a la del Turco, respiró
fuerte Teresita: era la sensación de libertad, que entraba con ímpetu en
su alma. ¡Y qué agrado le causaba el discurrir sola de calle en calle,
sin la enojosa guardia de una fregona cerril que comúnmente desempeñaba
su papel con sequedad policíaca!\ldots{} En la calle del Turco se detuvo
ante la casa de Guillermo de Aransis; miró al portal, decorado con
leones, y luego a las ventanas, poniendo un interés particular en
pasarles revista, y en distinguir las que tenían cerradas las persianas
de las que mostraban el cristal bien limpio, vestido por dentro con
elegantes visillos. «Ya se ha levantado---decía.---Andará por ahí,
conversando con los amigos que ha convidado a almorzar, o leyendo los
periódicos, a ver qué mentiras traen.» Conocía las costumbres del ocioso
caballero por lo que a menudo le oía contar en casa de Valeria. Siguió
después de esta observación su camino, y al atravesar la Plazuela de las
Cortes para entrar en la calle del Prado, vio venir el coche de Aransis,
bajando la Carrera de San Jerónimo. De lejos le conoció por el cochero;
de cerca por la elegancia y pulcritud del vehículo, por los blasones,
por algo que no era común a todos los coches. Aguardó el paso,
poniéndose casi en medio del arroyo. En el carruaje iba Guillermo con el
Marqués de Beramendi. Ambos la vieron: Guillermo, con viva curiosidad y
sorpresa, sacó la cabeza por la portezuela para mirarla bien, como si
dudara de lo que veía.

Pasó el coche, y Teresa siguió, ya sin parar hasta su vivienda, ni
apartar la vista de las piedras y baldosas. Tuvo la suerte de no
encontrar a su madre, con lo que se libró de las necesarias
explicaciones del trueno gordo con Valeria. Con la criada Felisa, en
quien ponía toda su confianza, se entendió para ocultar a Manuela el
inaudito caso de haber venido sola, y acto continuo se encerró en su
cuarto y se puso a escribir. Tan metida en sí misma estaba, que no paró
mientes en que escribía conservando puesta y liada en su cabeza la
mantilla. No se la quitó hasta que una fuerte sensación de calor, tan
molesta como su torpeza para expresar con la pluma lo que sentía, atrajo
su atención hacia aquel estorbo. ¡Qué tonta, Señor; qué simple! Sin duda
no acertaba en la fiel reproducción de sus ideas en el papel, por causa
del sofoco de la mantilla. Resultó luego que ni aun despejada su cabeza,
y con la cabeza su magín, de la espesa nube negra, lograba dar a los
conceptos la debida claridad. Seis cartas escribió, y todas fueron rotas
para empezar de nuevo. Pero, agotada con la última su paciencia, se
declaró incapaz de aquel empeño\ldots{} No contenta con romper las
cartas, llevó los pedacitos a la cocina para quemarlos en el fogón,
cuidando de que ni el fragmento más menudo se le escapase en aquel auto.

Nada digno de ser contado ocurrió en la tarde de aquel día ni en la
mañana del siguiente, como no sea que Teresa apuró todos los disimulos
para que su madre ignorase el ya irreparable rompimiento con la de
Navascués. Temía los enfadosos interrogatorios de Manolita, las
disposiciones que tomaría para privarla de libertad, o imponerle nueva
esclavitud contraria a los gustos de la esclava. Aprovechando una de las
salidas de su madre, que solían ser de larga duración, tomó al fin
Teresita la calle y fue con libertad a su objeto, el cual no era otro
que acechar el paso de Aransis para tener con él unas palabritas. Al
dedillo conocía los hábitos del caballero, los cuales obedecían a un
cierto método dentro del desorden. Sabía que muchas tardes, sobre las
seis, a pie salía de la casa de Valeria, y por las calles de Alcalá y
Cedaceros se iba a la querencia del Casino; sabía que pasaba algunos
ratos en la sala de armas de la calle de la Greda, tirando al florete; y
con estos datos y su paciencia, dio con él una tarde, no consta si la
primera o la segunda de su tenaz espionaje callejero. Tuvo la suerte de
cogerle solo, sin la compañía de amigos impertinentes, al salir de la
lección de esgrima. Pero se turbó tanto al verle, y tal miedo le entró
de aquel paso, viendo su ridiculez e inconveniencia en la realidad, que
se habría echado a correr si el caballero no mostrase mayor deseo que
ella de las cuatro palabritas, avanzando a su encuentro con rostro
alegre. Teresa no sabía por dónde empezar; lo que pensó para exordio se
le había escapado de la memoria. Rompió el galán el silencio y cortó la
cortedad diciendo: «Ya sé, ya sé\ldots» Y ella se turbó más. Sus
primeras palabras, entregando al caballero sus dos manos, fueron de
arrepentimiento, de vergüenza: «Déjeme, Guillermo\ldots{} No he debido
venir a buscar a usted\ldots{} Se me ocurrió este desatino, por no saber
a quién volverme\ldots{} Aunque tengo madre, estoy sola en el
mundo\ldots»

Medias palabras de una y de otro, expresiones vagas, de esas que nada
dicen y lo dicen todo, siguieron a las primeras manifestaciones
incoherentes y turbadas de la señorita de Villaescusa. Aransis le dijo:
«En la calle no podemos hablar con libertad. Ni se oye lo que se dice ni
se dice todo lo que se siente\ldots{} ¿Vámonos a mi casa?»

Teresa dudó\ldots{} parecía que dudaba; pero se dejó llevar. ¡Era tan
cerca!\ldots{} Cuatro pasos no más.

\hypertarget{xiv}{%
\chapter{XIV}\label{xiv}}

Debe decirse, para mejor conocimiento del proceder y fines de Teresita,
que esta, en los últimos días de su intimidad con Valeria, se había
hecho cargo con sutil adivinación de que el Marqués de Loarre declinaba
rápidamente hacia el cansancio en sus relaciones con la hija de Socobio.
No lo advertía la dama; su amiga sí, por virtud de una ciencia no
aprendida, a la que daban viveza su admiración del caballero y su
ardiente anhelo de serle grata. Y algo más sabía Teresa, que en aquel
aprendizaje sacaba, como quien dice, los pies de las alforjas, probando
y ejerciendo su nativa aptitud para las artes de amor. Sabía que su
persona penetraba en los gustos del Marqués: se lo revelaron ciertos
medios de experimentación existentes en el alma de toda mujer, y
principalmente en la suya, que era de las más afinadas y conspicuas para
estas cosas. Por encima de todas las hipocresías y de las conveniencias
que ambos guardaban en la casa de Valeria, Teresa sabía que agradaba al
Marqués, y que este se lo habría manifestado si no se lo vedara su
exquisita delicadeza. ¿Qué invisible enlace psicológico, qué magnetismo
pudo establecer entre ellos este preliminar estado de amistad que tuvo
repentino acuerdo en medio de una calle? Ni frase furtiva ni mirada
indiscreta pudieron delatar la volubilidad del amante o la traición de
la amiga. Miradas y frases hubo de gran sutileza, sólo de los criminales
comprendidas por clave misteriosa, y con tales antecedentes no más, se
lanzó Teresa a la busca y captura del Marqués de Loarre. Acometió la
señorita con fe ciega y ardor esta persecución cinegética, y el éxito
fue tan rápido como decisivo.

A los diez días o poco más de estos sucesos, que maldito lo que tienen
de históricos, habitaba Teresa un pisito muy mono, calle de Lope de
Vega, amueblado con elegante sencillez. Mañana y tarde invadía la casa
una caterva de tapiceros, modistas y prenderas, que iban a completar el
decorado, a tomar medidas a la señora para diferentes vestidos, o a
ofrecerle objetos diversos, gangas y \emph{proporciones} con que
especula el corretaje a domicilio. Gozosa estaba Teresa, la verdad sea
dicha, por verse libre, o en esclavitud que no lo parecía, y con ancho
camino por delante para correr tras de la risueña Fortuna que desde
rosados horizontes le decía: «Ven; aquí estoy.» Rota la cadena que la
sujetaba al desabrido estado señoritil, ya podía campar a sus anchas, y
dar el debido valor a su belleza y a las demás prendas que poseer creía:
inteligencia, bondad de corazón, finura social. Bastante tiempo había
perdido en la tienta de novios sin encontrar ninguno que le sirviera: el
que no era tonto, era malo; el listo pecaba de pobretón, y si algún feo
resultaba despejadito, los guapos se caían de bobos. Bien los había
examinado ella en el veloz desfile; breve y superficial trato le bastaba
para catarlos y calarlos. Si no lo encontró en las condiciones
necesarias para fundar un sólido edificio matrimonial con la honradez y
ventura consiguientes, no era culpa suya. Su destino le marcaba los
caminos irregulares, y por ellos se lanzaba, afirmada su conciencia en
la persuasión de que no podría andar por otros. Cada ambición tiene su
espacio propio para volar. Que el de la suya era de los más extensos, se
lo probaba la grandeza y poder de sus alas.

Del Marqués de Loarre debe decirse que en aquella nueva caída de su
voluntad inválida, tuvo más parte la pasión que la vanidad. Infundíale
Teresa un amor travieso, juvenil, de continua ilusión, que
constantemente se renovaba empalmando lo más espiritual con lo que al
parecer no lo es. Ninguna mujer, como aquella, le había llevado al puro
éxtasis contemplativo de la humana belleza, y a la poesía del amor, que
inspira elevados pensamientos y gallardas acciones. Preciosa era
Teresita antes de meterse en aquel enredo; metida en él, y habiendo
soltado ya la compostura y encogimiento de señorita \emph{del pan
pringado}, como las culebras sueltan su piel gastada quedándose con la
nueva reluciente, su persona resplandecía en todos los grados y matices
de la belleza, desde los más delicados a los más incitantes. Era un
libro de poesía incomparable, tan superior en los pasajes de absoluta
seriedad, como en los amenos y graciosos\ldots{} libro satánico,
encuadernado en piel de serafines.

Sabía muchas cosas de la vida y de la sociedad la despabilada Teresa,
añadiendo los descubrimientos que hacía su natural penetración a lo que
la experiencia le enseñaba. Pero sabiendo tanto, no se había dado clara
cuenta de su situación ante el mundo, y sobre este particular tan
interesante la ilustró Guillermo con discretas explicaciones: «Tu
libertad está limitada al interior de tu casa; fuera de ella has de
andar con mucha cautela y disimulo para que de la libertad no te resulte
el escándalo. De poco te valdrá tener trajes lindos y variados, los
sombreros más elegantes, y los prendidos y adornos más a la última,
porque no podrás lucirlos en ninguna parte donde haya lo que llaman
buena sociedad, y la otra sociedad, la de las que viven como tú, es muy
reducida y no se muestra en público con alardes de riqueza. Coches no
debo ponerte, y bien sabe Dios que lo siento, porque no está bien visto
que las mujeres de vida irregular gasten otra clase de vehículos que los
simones. Al teatro puedes ir, y como no has de ir sola, tienes que
acompañarte de otras tales, y esto llama la atención. Has de presentarte
muy modestamente en todo sitio público, dándote tus mañas para que nadie
te conozca. Esto es difícil: tu belleza te delata, y la sencillez, la
pobreza misma en el vestir, no te disfrazarían. Para que pudieras ir
libremente a todas partes y echar facha con trajes bonitos y carruajes
de lujo, necesitarías ser casada\ldots{} ¡ya ves qué grande anomalía! Si
hubieras entrado en esta vida con marido, o lo adquirieras después
casándote con cualquier calzonazos, que te diera nombre y pabellón, ya
podrías hacer tu contrabando libremente, y hasta te tratarían muchas
señoras que hoy primero se cortan la cabeza que saludarte. Ya ves,
chiquilla, qué diferencias tan absurdas en el proceder del mundo con las
que no se ajustan a la moralidad. Eres soltera: \emph{vade retro}. Que
tuvieras un maridillo, pararrayos de las burlas y de las iras de la
opinión, y ya sería otra cosa. No gozarías la consideración de persona
de ley; pero serías tolerada, y tu presencia en los teatros y paseos,
desafiando con tu lujo, a nadie chocaría\ldots{} Con que ya sabes,
Teresa: dentro de tu casa eres reina; fuera, esclava, sobre quien tiene
puesto el pie la opinión y no te deja respirar.»

Asimilándose al punto estas ideas, Teresa contestó que se conformaba con
andar siempre de trapillo fuera de casa, pues si para engalanarse hacía
falta marido, más parecido a un trasto portátil que a un hombre, se
quedaba muy a gusto en su soltería mal mirada. Como estaban en la luna
de miel, o poco menos, siempre que hablaban del porvenir daban por punto
indiscutible que no habían de separarse nunca, y que serían los eternos
amantes, eternamente embobados el uno con el otro. Lo malo fue que a
poco de instalarse Guillermo y Teresa en aquel rincón de los dominios de
Afrodita, enterose de ello Beramendi, y si se dice que al saberlo cogió
el cielo con las manos, no se expresa bien toda su pena y cólera. Y
razón tenía el enojo del caballero y fiel amigo. Sépase que a fines de
Agosto revolvió a Roma con Santiago para conseguir la realización del
tantas veces aplazado empréstito de Loarre y San Salomó. Gracias a su
perseverancia y actividad, apencó al fin con el negocio del señor
Sevillano, sin participación de otro alguno. Se firmó la escritura el 10
de Septiembre. Aransis quedó libre de la pesadumbre y esclavitud de
onerosas deudas, y recibía el primer plazo de la renta que se le
señalaba para vivir en decorosa medianía. ¿No era un dolor que casi en
los mismos días de esta felicísima solución, que debía ser fundamento de
nueva vida y principio de enmienda, recayese Guillermo en las mismas
culpas, en los mismos desórdenes que habían motivado su ruina?

A la dura filípica de Beramendi, contestó con estos artificiosos
argumentos: «Tienes razón, Pepe: yo reconozco que no merezco tu
amistad\ldots{} Quiero conservarla, y la fatalidad no me deja. Un poder
superior me arrastra: contra él nada puedo. Cada uno lleva en sí desde
el nacer el germen de la enfermedad de que ha de morir\ldots{} Me he
convencido de una cosa: la medicina que intenta curar estos males, que
son la vida misma, es peor y más dolorosa que la enfermedad. Déjame
vivir con mi muerte, Pepe\ldots{} Te digo también que este delirio de
ahora no es vanidad, sino pasión; la única de mi vida quizás\ldots{} Ver
pasar esta pasión, ver pasar estos rábanos y no comprarlos, ya
comprendes que no puede ser\ldots{} Tener el ideal cogido en la mano y
dejarlo escapar, es locura tan grande, que no la tendrías igual sumando
las locuras de todos los locos que están en Leganés\ldots{} Y aunque me
injuries, Pepe; aunque me mates, te diré que me apesta el orden
acompasado; que odio la administración, y que ese \emph{desideratum} de
la vida práctica, al modo inglés, al modo extranjero, como decís, se me
sienta en la boca del estómago\ldots{} Morir, Pepe, morir en la cruz
de\ldots{} ¿cómo llamaré a esta cruz?\ldots{} en la cruz del ideal
único, del que sólo nos visita una vez\ldots»

Esto, y algo más en el propio sentido sin sentido, dijo el de Loarre,
provocando al de Beramendi a burlonas risas. Despidiéronse, asegurando
Fajardo que era para no verse ni hablarse más. «Eres hombre perdido---le
dijo,---y cansado de luchar inútilmente por ti, te abandono. Cuando te
bailes en las últimas, cuando vayas a un hospital, o cuando mal trajeado
y con las botas rotas te pasees en la acera del Casino, pidiendo un
napoleón a cualquier transeúnte desdichado, volverás a verme; antes no,
Guillermo. Quédate con Dios.»

A pesar del severo propósito, como le amaba tan de veras, pasados
algunos días volvió Beramendi a la carga con arsenal nuevo de razones y
un plan que creía de grande eficacia. En su casa, recién salido del
lecho, oyó Aransis con calma el nuevo rapapolvo de su amigo: «Ya sé que
has agotado en tres semanas o poco más el primer trimestre de tu
pensión, y que has tenido que acudir otra vez a los usureros para el
sostén de la Villaescusa\ldots{} Olvido lo que te dije aquella tarde en
el Casino, y vuelvo a ti considerándote como un niño enfermo. No tendría
yo perdón de Dios si te abandonara. Te salvaré, aunque para ello tenga
que sacarte de Madrid entre guardias civiles, y encerrarte luego en un
castillo, en una torre o casa de campo, como se encierra a los locos
furiosos que se golpean a sí mismos y muerden a sus enfermeros.
Prepárate, chico. Ahora verás cómo las gasto. Pedí a Pastor Díaz un
puesto diplomático para ti, con el interés que puedes suponer. Atenas,
Bruselas, Turín, lo mismo da. Me contestó que hay vacante, pero que nada
puede hacer sin una indicación de O'Donnell. Fui a ver al General, que,
como sabes, es mi amigo. En la Granja he tenido ocasión de tratarle con
frecuencia. Vinyals y Vega Armijo tienen gran empeño en llevarme a la
\emph{Unión Liberal}. Don Leopoldo parece estimarme más de lo que yo
merezco\ldots{} Pues como te digo, fui a verle y le solté a boca de
jarro mi pretensión. ¿Sabes lo que me contestó? «Siendo cosa de usted,
Beramendi, es cosa mía, y, por tanto, cosa hecha. Parece que una
plenipotencia quedará vacante pronto. Se hará una combinación\ldots»
Quedé en volver a Buenavista dentro de pocos días, y allá me voy mañana,
pero no solo: irás conmigo, y darás las gracias al General por el honor
que te hace.»

El de Loarre nada dijo: creyérase que levantar repentinamente el vuelo
hacia un país lejano, con airosa investidura diplomática, no le parecía
mal. Antes que formulara una objeción tímida, más sugerida tal vez del
disimulo que del convencimiento, Beramendi se precipitó a completar su
plan: «Falta la segunda parte. Verás: mañana mismo escribes una carta a
esa linda serpiente que te ha trastornado el seso. Ya comprenderás lo
que tienes que decirle\ldots{} Que no puedes seguir, que dé por
terminado este chapuzón, pues a ti te saco yo a flote, y ella que busque
otro imbécil con quien ahogarse\ldots{} A la carta acompañarás una
cantidad prudencial, que determinaremos, y si no la tienes, que no la
tendrás, no has de pedirla a los usureros: yo te la doy\ldots{} Con que
ya ves que te estimo de veras. Te participo, querido Guillermo, que por
si cerdeas tú, o se sale tu sílfide con algún ardid para retenerte, ya
tengo preparado un lindísimo artificio judicial para meterla en la
Galera, o mandarla desterrada lejos, muy lejos\ldots{} Nada, nada. Hoy
me he levantado con la idea y propósito de convertirme en sátrapa. No
queda otro remedio. Contra la tontería y la inmoralidad reunidas; contra
un loco y una perdularia, ambos sin conciencia, sin idea del honor, sin
ninguna rectitud, no hay más que el palo absolutista\ldots{} Aquí me
tienes dispuesto a hollar todas las libertades, y a convertir en
pajaritas las hojas del libro de la Constitución. Declaro que desde este
momento has perdido todos los derechos del ciudadano, y eres mi vasallo,
mi siervo. Aquí vengo a tu conquista y captura. Vístete, arréglate, y te
llevo conmigo a mi casa, de donde no saldrás hasta que demos tú y yo
cumplimiento a todo mi programa.»

Oyó estas conminaciones Guillermo entre atontado y risueño, como si a
veces las tomase a broma, a veces con harta seriedad y recelo. El tono
brioso de Fajardo le persuadió al fin de que se las había con una
voluntad enérgica, y sintió miedo. La suya, floja y pasiva, no sabía
mantenerse en pie contra la razón erguida y brutal de su amigo\ldots{}
Más que nada temía la convivencia con su tirano. Siempre al lado suyo,
acabaría por obedecerle, por ser un niño\ldots{} Como pidiera más
explicaciones de aquel cautiverio que le esperaba, Beramendi le dijo:

«Desde hoy vivirás en mi casa. Que no te suelto, que no te escapas.
Verás con mis ojos, andarás con mis piernas y respirarás con mis
pulmones. Pensaba yo que fuéramos hoy a ver al Presidente del Consejo,
para que quedases cogido y amarrado en el compromiso de tu nombramiento
de Ministro de España en una corte extranjera. Pero ahora caigo en que
estamos a 10 de Octubre, cumpleaños de la Reina. Gran gala, besamanos;
por la noche baile en Palacio. No hay que pensar hoy en visitar a gente
política y militar. Para no perder el día, después de almorzar
redactarás en mi despacho la carta explosiva que has de mandarle a tu
coima\ldots{} explosiva digo, a ver si revienta cuando la lea\ldots{}
Verdad que irá acompañada de los maravedises, y el topetazo será con
algodones\ldots{} Cree, Guillermo, en la virtud de los maravedises, que
vienen a ser colchón blando para la caída de las que se derrumban de
desesperación\ldots{} Ea, vístete y vámonos\ldots{} ¡Silencio! no se
permiten observaciones. No hay derecho a protestar, no y no. Sólo
concedo un derecho, el del pataleo\ldots{} Arréglate, digo, y en casa
patalearás a tu gusto.»

\hypertarget{xv}{%
\chapter{XV}\label{xv}}

Esto pasaba en la mañana del 10 de Octubre. En la madrugada del 11
ocurrían otras cosas igualmente insignificantes en apariencia, pero que
aquí se refieren porque su simplicidad se nos presenta enlazada, horas
después, con hechos de evidente complicación y gravedad. Empezaban a
salir los invitados a la fiesta de Palacio; arrimaban los coches a la
colosal puerta, por la Plaza de la Armería; entraban en ellos,
chafándose en las portezuelas, los hinchados miriñaques, dentro de los
cuales iban señoras; entraban plumas, joyas, encajes, bonitas o vetustas
caras compuestas, y apenas un coche partía, otro cargaba\ldots{} De los
primeros, más que de los últimos, fue un carruaje sin blasones, de un
tipo medio entre los elegantes y los de oficio, alquilados por año, y en
él entró doblándose un largo cuerpo, un dilatado capote que por arriba
remataba en tricornio con plumas, por abajo en botas de charol con
espuelas. Tras el sujeto larguirucho no entró en el coche señora, sino
dos militares, que por la traza distinguida y cargazón de cordones
debían de ser ayudantes\ldots{} El coche partió, y ninguno de los tres
señores en él embutidos pronunció palabra en todo el trayecto desde
Palacio al Ministerio de la Guerra. El Presidente del Consejo, General
O'Donnell, el más largo de los tres en estatura y en todo, que nunca
ejerció la comunicatividad baldía, fue en aquella ocasión arca cerrada.
Llegaron a Buenavista; subieron en callada procesión, algo parecida a la
del cura y acólitos que llevan el Viático, y en las habitaciones del
General, rompió este el silencio ante su digna esposa, que jamás se
acostaba cuando él iba de fiesta palatina, las únicas que le hacían
trasnochar, y aquella noche le esperó como de costumbre, para informarse
de si volvía contento y en buena salud, con algo más que nunca omite en
estos casos la curiosidad femenina.

Contestando a doña Manuela, luego que se acomodó en un sillón y estiró
las piernas, el gran O'Donnell dijo: «¿El baile? Precioso. Allí teníamos
todo el lujo y toda la elegancia que hay en Madrid\ldots{} No hay más.
¿Señoras? No faltaba ninguna: allí estaban las de la sangre y las del
dinero\ldots{} ¿Calor? Bastante, y poco espacio, por el volumen
exagerado de los miriñaques. ¿La Reina? Deslumbradora\ldots{} amable con
todos\ldots{} Traje riquísimo de gasa\ldots{} el adorno, guirnaldas de
violetas\ldots{} elegantísimo\ldots{} Soberbio alfiler de
brillantes\ldots{} Bailó conmigo el primer rigodón; luego\ldots»

Volviéndose a los ayudantes, como para pedirles testimonio de un
recuerdo, dijo que la novedad del baile había sido la presentación en
Palacio de la Condesa de Reus\ldots{} «¿Verdad que es muy mona la mujer
de Prim? Morenita y simpática\ldots{} En fin, buenas noches.» Ansiaba el
descanso, la soledad. Algo de íntimo interés tenía que referir a su
esposa; pero por lo avanzado de la hora, determinó dejarlo para el día
siguiente. Poco después de esto se hallaba don Leopoldo en manos de su
ayuda de cámara, que desenfundó su cuerpo del uniforme, sus desmedidas
piernas, de las botas sin fin\ldots{} Algunos minutos más, y ya le
teníamos tendido y estirado en su cumplido lecho, en postura supina, más
dispuesto a la meditación que al sueño, porque del baile había traído un
resquemor, que hasta el amanecer había de ser cavilación fatigante.
Aunque era O'Donnell hombre más reflexivo que apasionado, que sabía
mirar con calma los graves acontecimientos y las contrariedades de la
vida o de la política, la misma pujanza y frialdad de su razón apartaban
su mente del descanso para aplicarla al examen de los hechos, y cuando
estos despertaban su enojo, no dejaba de correr por los nervios del
grande hombre el hormigueo que determina el insomnio. De la devanadera
que en aquella madrugada giró dentro del cerebro del héroe de Lucena, se
han podido extraer con no poco trabajo estos fraccionados pensamientos:

«Es por la Desamortización, por la pícara Desamortización\ldots{} Ya lo
veía yo venir\ldots{} Pero no creí, no, que tan pronto\ldots{} Ni pensé
que me pusiera en la calle por tal motivo\ldots{} Narváez llegó hace
tres días; fue a Palacio y dijo: «Señora, sepa Vuestra Majestad que yo
no desamortizo. Mi política es tener contentos a los curas y al Papa.»
Así le dijo, y las consecuencias bien claras las he visto esta
noche\ldots{} Ha sido una impertinencia, un rasgo de mala
educación\ldots{} No jugar, Señora, no jugar con los hombres ni con los
partidos\ldots{} Con estos juegos y estas humoradas, las coronas se caen
de las cabezas\ldots{} y menos mal que estamos en España, un país de
borregos; que hay países donde por estas bromitas caen las cabezas de
los hombros\ldots{} Cuidado, ¿eh?\ldots»

Dio una vuelta, cargando sobre el lado izquierdo su formidable osamenta.
La devanadera echaba esto de sí: «No hay manera de crear un país a la
moderna sobre este cementerio de la Quijotería y de la Inquisición.
España dice: «Dejadmh como soy, como vengo siendo: quiero ser bárbara,
quiero ser pobre; me gusta la ignorancia, me deleitan la tiña y los
piojos\ldots» Y yo digo: Modo de arreglar a esta nación: saco del
partido Moderado y del Progresista los hombres que en ellos hay
inteligentes, limpios, bien educados; los cojo, con ellos me arreglo,
dejando a los fanáticos y a los tontos, que para nada sirven\ldots{} Con
esta flor de los partidos amaso mi pan nuevo\ldots{} \emph{Unión
Liberal}\ldots{} Reunimos y organizamos lo útil, lo mejor, lo más
inteligente; y lo demás, que se descomponga y vuelva al montón\ldots{}
¿Cuántas veces, Reina mía, he tratado de meterte en la cabeza esta
idea?\ldots{} Trabajo perdido. La comprendes\ldots{} ¡como que no tienes
un pelo de tonta! pero entra por un oído y sale por otro\ldots{} Sale
porque hay dentro de tu cerebro ideas viejas, heredadas,
petrificadas\ldots{} ¿Y esas ideas, qué son? Reinar fácilmente y sin
ninguna inquietud sobre un pueblo, mitad desnudo, mitad vestido de paño
pardo\ldots{} Esto no puede ser\ldots{} Y tú, Reina, ¿qué piensas
trayendo a Narváez con la Constitución del 45, neta, y el palo por única
ley, y el \emph{tente tieso} por única política? Tú, Reina, mira lo que
haces. Tú, Reina, no olvides que para mantenerse en esas alturas, hay
que tener educación política, educación social, principios,
formas\ldots{} tú me entiendes; tú\ldots»

El hablar de \emph{tú} a Su Majestad era señal de que se dormía. Por un
momento, la onda del sueño estuvo a punto de anegarle\ldots{} De
improviso volvió sobre sí: despabilándose y volteando su corpachón hacia
el lado derecho, dio nuevo impulso a la devanadera, que decía:
«Desamorticemos\ldots{} País nuevo\ldots{} Salaverría, que sabe sacar
estas cuentas mejor que nadie, ha calculado la Mano Muerta en siete mil
millones. Yo digo que debe de ser más\ldots{} ¡Siete mil millones! Ello
es nada: caminos carreteros, ferrocarriles, puertos, faros, canales de
riego y de navegación\ldots{} Y vale más que todo el gran aumento de la
propiedad rústica\ldots{} Serán propietarios de tierra muchos que hoy no
lo son, ni pueden serlo\ldots{} aumentará fabulosamente el número de
familias acomodadas; los que hoy tienen bastante, tendrán mucho más; los
dueños de algo, lo serán de mucho, y los poseedores de la nada, poseerán
algo\ldots{} ¿Qué es esta España más que \emph{un hospicio suelto}? Esas
nubes de abogadillos que viven de la nómina, las clases burocráticas y
aun las militares, ¿qué son más que turbas de hospicianos? El Estado,
¿qué es más que un inmenso asilo? Dice Salamanca que en toda España hay
dos docenas de millonarios, unos quinientos ricos, unos dos mil
pudientes o personas medianamente acomodadas y ocho millones de
pelagatos de todas las clases sociales, que ejercen la mendicidad en
diferentes formas. En esta cuenta no entran las mujeres\ldots{} Pues
bien, digo yo: Amigo Salaverría\ldots{} vendamos la Mano Muerta, hagamos
miles de hacendados nuevos, facilitemos el pago de las fincas que se
vayan desprendiendo de esa masa territorial muerta\ldots{} A los pocos
años, tendremos agricultura, tendremos industria, y la mitad por lo
menos de los hospicianos que forman la Nación, dejarán de serlo\ldots{}
Digan lo que quieran, el español sabe trabajar. No le faltan aptitudes,
sino suelo, herramientas, estímulo y mercado que les compre lo que
producen\ldots{} ¡Siete mil millones, que hoy existen en el fondo de un
arcón cerrado con llaves que la Iglesia tiene en su mano, y no quiere
soltar ni a tiros!\ldots{} A tiros sí que las soltaría\ldots{} Pero,
señora Reina, ¿hemos de armar otra guerra civil por esas dichosas
llaves? ¿No derramamos bastante sangre en la primera, para defender tus
derechos y asegurarte en el Trono?\ldots{} ¡Y los vencidos en aquella
lucha, Reina mía, son ahora los que detrás de una cortina te aconsejan y
te dirigen!\ldots{} ¡Y no pudiendo dar el poder a los vencidos de
aquella guerra, lo das a Narváez, que entra en Palacio diciendo: «Yo no
desamortizo\ldots!» Cuidado, Reina: no se juega con la vida de un
pueblo\ldots{} de una Nación viril, por más que sea la gran \emph{Casa
de Caridad}. El hospiciano sigue diciendo: «quiero ser bárbaro, quiero
ser pobre;» pero lo dice por rutina\ldots{} Detrás de ese estribillo
suena un querer oculto, suenan otras voces, que apenas se
entienden\ldots{} Tú no sabes oír estas voces; yo las oigo\ldots{} las
oímos muchos\ldots{} A Palacio no llegan sino cuando nosotros te las
decimos y tú no las escuchas\ldots{} Abre los oídos, Reina; abre los
ojos, para que oigas y veas\ldots{} Estás a tiempo aún\ldots{} Algún día
dirás: ¿qué ruido es ese?\ldots{} Pues ese ruido, ¿qué ha de ser más
que\ldots?»

Otra vez la trataba de \emph{tú}, otra vez se dormía\ldots{} Por fin
cogió el sueño, y la devanadera cedió lentamente en su veloz volteo
hasta quedar inmóvil\ldots{} De día no funcionaba la devanadera, y los
pensamientos del General se producían con ponderación y sensatez, en
perfecta consonancia con el pensar común y el ambiente intelectual de su
tiempo. Se mantenía en el justo medio, y no se apartaba un ápice de la
realidad. El libre y atrevido pensamiento quedábase para los instantes
que preceden al sueño, o para los que inmediatamente le siguen, cuando
aún no ha entrado la plena luz en la alcoba, ni se ha oído más acento
que el de los gallos que cantan en la vecindad.

Levantose el General temprano, como de costumbre: despachada su
correspondencia con el Secretario particular, vistiose para ir a
Palacio. A punto de las doce, hora de las visitas de confianza, recibió
la de dos caballeros, el Marqués de Beramendi y el de Loarre. Al salón
pasaron, y ofrecían sus respetos a doña Manuela, que charlaba con su
amiga la Duquesa de Gamonal, cuando entró O'Donnell con uno de sus
ayudantes, dispuesto ya para ir a Palacio. Saludó a los dos
aristócratas; después cogió de un brazo a Beramendi, y llevándole
aparte, le dijo risueño: «Nada puedo hacer ya\ldots{} ¡Estamos caídos!

---¡Caídos, General!\ldots{} ¿Por qué?\ldots{} ¿De veras hay crisis?

---La plantearemos de hoy a mañana\ldots{} Caídos\ldots{} Nos
echan\ldots{}

---¿Pero esa señora está desatinada, o\ldots?

---De lo prometido no hay nada, Marqués. En testamento, no podemos
proveer vacantes del personal diplomático\ldots{} Pero ahora tendrá
usted en el poder a su amigo Narváez, que le dará eso y cuanto usted le
pida\ldots{}

---¿Narváez\ldots?

---Ea, que no puedo entretenerme. Dispénseme. Voy a la Casa grande.»

Mientras duró este aparte, Loarre y la Gamonal hablaron de la
inauguración del teatro de la Zarzuela, erigido en la calle de
Jovellanos, hermoso coliseo que resultaba como el hermano menor del
teatro Real. Inquieto y caviloso Beramendi por lo que el General acababa
de decirle, trató de llevar la conversación al terreno político para
esclarecimiento de sus dudas, y a la menor indicación que sobre crisis
hizo a doña Manuela, esta señora, a quien sin duda se le atragantaba la
noticia, se precipitó a echarla fuera en esta forma: «Pues sí\ldots{} lo
digo, porque hoy ha de saberlo todo Madrid. La Reina estuvo en el baile
de anoche muy inconveniente. Bailó el primer rigodón con O'Donnell: la
etiqueta manda que Su Majestad rompa el baile con el Presidente del
Consejo. Terminado el primer rigodón, la Reina le dijo a mi marido: «¿Te
parece que baile el segundo con Narváez?» Mi marido, que es la pura
corrección, le respondió: «Señora, Vuestra Majestad me dispense; pero la
etiqueta y las conveniencias más elementales mandan que ahora baile
Vuestra Majestad con un individuo caracterizado del Cuerpo
diplomático\ldots» ¿Pues qué creerán ustedes que hizo la Reina? Sonreír,
alzar los hombros, y \emph{sacar a bailar} a Narváez\ldots{} Esto es un
desprecio para mi marido\ldots{} es decirle, no con la boca, sino con
los pies: «O'Donnell, tú\ldots» En fin, que tenemos crisis.»

Condenaron enérgicamente los dos próceres la forma anticonstitucional y
pedestre de cambiar de Gobierno, no sin que Beramendi hiciera gala de su
erudición encareciendo la seriedad y rectitud de la Corona de Inglaterra
en los procederes constitucionales. La Gamonal, dama que había sido de
la Reina, y Duquesa de las de nueva emisión, oía estas cosas de alta
política como si fueran cuentos traídos de la China. «Pues yo no sé, no
sé\ldots---dijo abanicándose con mayor viveza de ritmo.---¡Estaría bueno
que la Reina, con ser Reina, no pudiera bailar con quien le diera la
gana!

---Hija, no puede ser\ldots---observó Doña Manuela sin cambiar de ritmo
en el abaniqueo.---Las Reinas, por serlo, están obligadas a mirar bien
lo que hacen, lo que dicen y lo que bailan\ldots»

¡Y vuelve por otra!\ldots{} Era doña Manuela más lista y aguda de lo que
parecía. Su figura insignificante, sus vulgares facciones afeadas por
una expresión desabrida, y la tez de un moreno harto subido, no
predisponían comúnmente en su favor. La cualidad suya dominante, que era
el amor intenso a su esposo, no tenía carácter social y de extenso
relieve. Para ella no había más Dios ni más Rey que O'Donnell, ni
tampoco mejor y más venerado profeta. O'Donnell, hombre de una dulzura
grande y de sencillez patriarcal en sus afectos, la amaba tiernamente y
la ponía en las niñas de sus ojos azules. Decían gentes maliciosas que
la temía. Temía todo lo que pudiera desagradarla, que es el temor de los
enamorados.

Volvió de Palacio don Leopoldo tranquilo, impenetrable. Ya los Marqueses
se habían ido, y sólo permanecía en el salón de Buenavista la Duquesa de
Gamonal. La presencia de esta señora, de cuño tan reciente, que aún no
se había enfriado el troquel que estampara su título, contuvo al General
dentro de la mayor reserva: lo que a ella le dijese se haría tan público
como si saliera en los periódicos. Entró luego más gente: dos amigos del
General, don Santiago Negrete y el Gobernador de Madrid, Alonso
Martínez, almorzaron con él. Por lo que hablaron de política, la crisis
era inevitable: ya se había citado a los ministros a Consejo, del cual
seguramente saldría la dimisión total. ¿Qué había dicho Isabel II a su
primer Ministro en la entrevista de aquella mañana? Algo referente a la
Ley de Desamortización. Sólo la Condesa de Lucena conocía el texto
exacto de las palabras de Su Majestad: «Mira, O'Donnell: te dije que no
me gustaba la Desamortización, y ahora digo y repito que en conciencia
no puedo admitirla; que no la quiero, vamos, que no puede ser\ldots»

\hypertarget{xvi}{%
\chapter{XVI}\label{xvi}}

\emph{Paulo minora canamus}, y de otra crisis hablemos, menos resonante
que aquella, porque a menor número de personas afectaba, pero no de
inferior interés psicológico. Teresa Villaescusa, sin darse cuenta del
valor y significado de las palabras, \emph{quería desamortización}. Si
alguna vez oyó hablar de la Ley a su tío don Mariano, en la memoria no
le quedó rastro del nombre ni de las ideas que expresaba. Tenía, sí, un
sentimiento vago de la detestable petrificación de la riqueza en manos
inmóviles, y una visión confusa del remedio de esta cosa mala, el cual
no era otro que coger todo aquel caudal, fraccionarlo, repartirlo en mil
y mil manos que supieran hacerlo fecundo. No sería propio decir que
Teresa pensaba en esto, sino que por su pensamiento a ratos pasaban como
sombras de estas ideas, en abstracción completa, sin que con ellas
pasaran los términos usuales con que los entendidos y los ignorantes las
designaban en aquel tiempo. Menos abstracto era en el alma de Teresita
el aborrecimiento de la pobreza. Por las escaseces que había sufrido, o
por ingénito gusto de las comodidades y de los goces, la miseria le
causaba horror. Egoísta y al propio tiempo magnánima, no quería ser
pobre ni que lo fueran los demás: su anhelo era que hubiese muchos
ricos, más ricos de los que había, y mayor número de millonarios\ldots{}
pensando, naturalmente, que de todo este bienestar algo le había de
tocar a ella.

Y sépase ahora que resuelto el buen Fajardo a sacar a Guillermo del
nuevo pantano en que había caído, no perdonó medio para este meritorio
fin. El destierro del pródigo, disimulado por una posición diplomática,
si no se conseguía por O'Donnell, caído ya, se conseguiría seguramente
por Narváez. Pero esto no bastaba, y era forzoso impedir a todo trance
que Teresa y Aransis volvieran a unirse. Reteniendo a éste cautivo en la
casa de Emparán, obligole a escribir la carta notificando a su amada el
definitivo rompimiento. Mas no seguro de los efectos de la epístola, ni
confiado en la resignación de la cortesana, determinó abordar ante esta,
descaradamente, el delicado asunto. No la conocía; deseaba explorarla y
sondear su voluntad. Bien podía suceder que fuese bastante discreta y
razonable para prestar su auxilio al salvamento del caballero. Casos de
abnegación semejante había en el mundo. Dejando, pues, a su amigo en
casa, una mañana, bien custodiado por María Ignacia y D. Feliciano, se
fue derecho al bulto, se encaminó a la gruta de la fascinadora ninfa,
solicitó verla, accedió la ninfa sin recelo, y poco tardaron en
encontrarse sentados vis à vis en la elegante salita.

Sorprendido quedó Beramendi de la tranquilidad con que la hermosa mujer
oyó la exposición preliminar, hecha con habilidad pasmosa de explorador.
Procurando no causar a su interlocutora la menor ofensa, la trataba como
amigo. Guillermo y él eran, más que amigos, hermanos. Teresa se hacía
cargo de todo; mostrábase atenta, mirando el caso como medianamente
grave en el aspecto moral, gravísimo en el económico. En sus réplicas,
mostraba dignidad, aplomo y un interés casi fraternal por Guillermo de
Aransis. Cuando Beramendi, alentado por el buen giro que a su parecer
tomaba el asunto, hizo a Teresa referencia clara de la situación de su
amigo, de sus locuras dispendiosas, de la pérdida de su caudal, del
embrollo de sus intereses; cuando le contó que él (el propio Beramendi)
había revuelto el mundo por salvar una parte al menos del patrimonio de
Loarre y San Salomó; cuando le expuso el contrato con Sevillano y el
estado presente de Aransis, que era el de un caballero cautivo de su
administrador, y sujeto a una pensión, suficiente para vivir con
modestia, cortísima para el vivir grande, con trenes de lujo y la
diversión de caballos y mujeres; cuando, por fin, le hizo ver que si
Guillermo seguía embarcado con ella, su naufragio era seguro, y no
habría de pasar mucho tiempo sin que se viese miserable, degradado, sin
dinero y sin dignidad, Teresa palideció, y con arranque dio esta briosa
respuesta:

«No siga usted, Marqués\ldots{} No necesito saber más. Mucho quiero a
Guillermo\ldots{} y por quererle tanto me aterra la idea de que sea
pobre. Aunque me esté mal el decirlo, la pobreza me da horror. No la
quiero para él ni para mí. Usted me ha convencido de que le favorezco
separándome de él. Bien está que vaya de Embajador o cosa así; bien está
que no me vea más. Soy la primera en reconocer que no debemos
seguir\ldots{} que él debe irse por un lado, yo por otro\ldots{} Ya la
carta suya, que recibí anoche acompañada de una cantidad muy lucida, me
dio que pensar. He dormido mal pensando que Guillermo me dejaba por no
poder sostenerme\ldots{} Marqués, no me asombre usted; no se enfade
conmigo, no vea en mí una mujer mala si te digo que me repugna el
\emph{contigo pan y cebolla}. Esto es pura imbecilidad y cosas ridículas
que han inventado los poetas para engañar el hambre\ldots{} No, no: yo
quiero a Guillermo, le querré siempre\ldots{} pero que por mí no se
degrade ni se arruine\ldots{} Queda usted complacido, Marqués. Su amigo
y yo hemos roto para siempre\ldots{} Cuídese usted de que no venga a
buscarme, y yo cuidaré de que no me encuentre si acaso viniera\ldots»

Dijo esto último con empañada voz y el consiguiente tributo de ternura y
lágrimas. Eran sinceras, pues si su aborrecimiento de la pobreza podía
considerarse como primer móvil de tal resolución, detrás o debajo de
este sentimiento había también cariño, gratitud y una dulce adhesión al
hombre, al caballero\ldots{} A él debía su libertad, la iniciación en
alegrías y goces que le fueron desconocidos; debíale las primicias del
bienestar humano, hasta entonces no disfrutado por ella. Por Guillermo
se le abrían horizontes tras de los cuales creía vislumbrar espacios de
felicidad. Había sido su revelador y el primero que dio realidad a su
grande ambición\ldots{} Bien le quería, sí. Bien merecía el homenaje de
sus lágrimas\ldots{} Dejándolas correr, dijo a Beramendi: «No hay que
hablar más, Marqués. En seguidita me marcho, me escondo\ldots{} No, no
voy a casa de mi madre, donde Guillermo daría conmigo si en ello se
empeñara. Es testarudo; me quiere\ldots{} Puede usted estar tranquilo.
Yo le aseguro que me esconderé bien, y que no volveré a esta casa hasta
saber que Guillermo se ha ido a esa Embajada \emph{de extranjis}\ldots{}
Leeré algún periódico para enterarme. Adiós, adiós\ldots{} ¡Pobre
Guillermo! Pobre, no; no le quiero pobre\ldots{} que sea feliz, que sea
caballero noble, que conserve la dignidad; y usted, tan buen amigo suyo,
consuélele\ldots{} haga porque me olvide. Yo no le olvido, no. Crea
usted que Guillermo se pondrá muy triste\ldots{} ¡Y qué bueno sería que
al volver de la Embajada se encontrara su capital sacado de todos esos
embrollos, limpio y\ldots{} En fin, adiós\ldots{} Dígale usted que me he
muerto; no, que me han robado\ldots{} robado mi persona; que\ldots{}
dígale usted lo que quiera, y ya sabe que tiene en mí una servidora.
Adiós, adiós\ldots»

Salió Beramendi encantado de la sinceridad de Teresa, y de la honradez
relativa con que proclamaba su afición a las riquezas y su culto del
bienestar. Tenía el mérito de decir lo que otros hacen diciendo lo
contrario, con hinchadas protestas de falsa delicadeza. Pensó el
caballero que su amigo estaba salvado, no contribuyendo poco a tan
lisonjero fin el buen sentido de la coima, cualidad rara en esta clase
de mujeres. Ya no había más que esperar el cambio de Gobierno para caer
sobre Narváez y no dejarle vivir hasta que diera los pasaportes al
Marqués de Loarre para una Corte extranjera, cuanto más distante mejor.
Y el cambio de Gobierno fue un hecho al siguiente día, tal y como
\emph{Don Leopoldo el Largo} lo había previsto. Doña Isabel, imitando a
su señor padre, dispuso que las cosas volvieran al estado que tenían
antes de lo de Vicálvaro, declarando nulo todo lo ocurrido en los dos
\emph{llamados años} de dominación progresista. Resultaba que las
\emph{lamentables equivocaciones} de Su Majestad volvían a cometerse, o
a constituir la efectiva normalidad política. Los hechos decían que el
Gobierno de liberales y progresistas era el verdadero equivocarse
lamentablemente, según el Real criterio, y que Isabel II hablaba con su
pueblo en lenguaje socarrón, abusando de la contragramática y del
maleante aforismo chispero: \emph{al revés te lo digo, para que lo
entiendas}.

Fue la subida de Narváez como un trágala de toda la gente arrimada a la
cola, que se preciaba de ser la dueña de nuestros destinos. ¡No era mal
puntapié el que la España vieja, momificada en sus rutinas absolutistas
e inquisitoriales, daba en semejante parte a la España nueva, tan
emperejilada y compuesta entonces con su Justo medio, su Unión de
hombres listos y pulcros, y su poquito de Desamortización, para mejorar
siquiera el rancho que veníamos repartiendo en el \emph{hospicio
suelto!} Y Narváez entraba como en su casa, tosiendo fuerte y trayéndose
cogiditos de la mano, como muestra de liberalismo, a Nocedal, a Pidal y
a otros \emph{ejusdem fúrfuris}. ¡Qué país tan dichoso! ¿Quién duda que
hemos nacido de pie los españoles? Apenas enfermamos del dengue
revolucionario, sale una Providencia benignísima que Dios destina
paternalmente a nuestro remedio, y en dos palotadas corta el mal, y por
lo sano, dejándonos como nuevos, en el pleno goce de nuestra
barbarie\ldots{} Y apenas entraron los \emph{providenciales} al mangoneo
político y administrativo, empezó el desmoche oficinesco, y la matanza
de empleados de la situación caída, para resucitar a los de la
imperante, que venían muertos desde el 54. Todo \emph{el elemento
progresista}, que arrimado estuvo a los pesebres desde aquella fecha de
las \emph{lamentables equivocaciones}, fue arrojado a la calle con
menosprecio, y entraron a comer los pobrecitos que no lo habían catado
en todo el bienio. Los unionistas, amarrados al presupuesto por
O'Donnell, también cayeron con los ilotas del Progreso, y a llenar el
inmenso hueco entró la caterva moderada, con alegre alarido de triunfo,
como si ejerciera un derecho sagrado. Eran los pobres a quienes se había
hecho creer que la bazofia nacional les pertenecía, y que no debía comer
de ella ninguna otra casta de hospicianos.

Otra vez el alza y baja de ropa; otra vez el vertiginoso
\emph{triquitrín} de las tijeras de los sastres; otra vez \emph{La
Gaceta} cantando los nuevos nombramientos con grito semejante al de las
mujeres que pregonaban los números de la Lotería; otra vez la procesión
triunfal de los que subían por las empolvadas escaleras de los
Ministerios, y el lúgubre desfile silencioso de los que bajaban. En el
coro lastimero y fúnebre de los cesantes, descollaba una voz campanuda
que dijo: «¡Cojondrios, ya está aquí la muerte!» Era Centurión
recibiendo el oficio en que, con formas de sarcástica urbanidad, se le
decía que \emph{cesaba}\ldots{} Y el cesar en sus funciones de la Obra
Pía, era como suspender las funciones orgánicas de asimilación y
nutrición\ldots{} ¡Comer, comer! De eso se trataba, y toda nuestra
política no era más que la conjugación de ese sustancial verbo. El
nacional Hospicio no podía mantener a tan grande número de asilados,
sino por tandas\ldots{} Veíase el buen hombre condenado a una nueva
etapa de miseria. ¿Por qué, Señor? Porque a nuestra Soberana se le había
metido en la cabeza que no debía desamortizar, y el \emph{espadón de
Loja} recogió al vuelo la idea, y con la idea las riendas y el látigo,
subiéndose de un brinco al pescante del desvencijado carricoche del
Gobierno.

Pues, siguiendo paso a paso la Historia integral, dígase ahora que al
tiempo que Isabel de Borbón decía con desgarrada voz de maja: \emph{yo
no desamortizo}, la otra maja, Teresa Villaescusa, gritaba: «juro por
las Tres Gracias que a mí nadie me gana en el desamortizar.» No usaba
esta palabra, ni daba concreta forma a sus atrevidos pensamientos; pero
en la rigurosa interpretación de la idea no fallaba la despejada hembra.
Aún persistía en su corazón el duelo de Aransis, cuando puso fundamento
al nuevo trato de amor con que debía sustituir al trato roto. Base de su
criterio en estos graves asuntos era el principio de que la peor cosa
del mundo es la pobreza; de que el vivir no es más que una lucha
sistemática contra el hambre, la desnudez, la suciedad y las molestias,
y partiendo de esto, eligió entre los tres o cuatro individuos que la
solicitaron aquel que ofrecía más templadas armas para luchar contra el
mal humano. Ya en los últimos días del breve reinado de Aransis, llegó
una emisaria con varias proposiciones que no quiso aceptar. Teresa era
leal: no cometería una traición por nada de este mundo. Pero sacada,
como si dijéramos, a concurso por la abdicación de Guillermo, no quiso
precipitarse, sino antes bien hacer el debido examen y selección de
candidatos. No tenía prisa; el dinerillo del testamento de Guillermo le
permitía tomarse todo el tiempo que fuera menester para elegir con
calma. Cuidó en aquel tiempo de dar mayor realce a su belleza, cada día
más interesante; coqueteaba graciosamente con los remilgos mejor
copiados del modelo de la honradez; acentuaba su gracia, su donosura,
hacia la gran señora; se daba un tono fenomenal\ldots{} La resolución o
sentencia vino por fin informada en esta idea: los grandes fardos de
riqueza deben ser manoseados y sacudidos con alguna violencia, para que
de ellos se desprenda el exceso, que es carga perniciosa; y si no se
dejan sacudir, debe quitárseles lo más que se pueda para remedio de los
que van sin ninguna carga por estos mundos de Dios. Aligerar a los
demasiado ricos es obra meritoria\ldots{} \emph{et cætera}\ldots{} no lo
decía así, pero lo hacía.

\hypertarget{xvii}{%
\chapter{XVII}\label{xvii}}

Eligió con exquisita cautela y previsión Teresilla la persona que más le
convenía para sus fines estratégicos, consistentes en levantar
formidables baluartes contra la pobreza, y para llegar a la final
decisión empleó diversas artes, sometiendo al preferido a pruebas de
lealtad, de sinceridad, de esplendidez y de otras virtudes que la pícara
mujer estimaba condiciones \emph{sine quibus non}. Era el nuevo
contratista de amor un francés de mediana edad, ni joven ni viejo, más
gordo que flaco, alto, rubio, sonrosado, de correctísima educación y
finos modales, que había venido a Madrid al establecimiento del
\emph{Crédito Franco-Español}, núcleo de capitalistas extranjeros que
debía emprender en España negocios colosales, como \emph{Los Caminos de
Hierro del Norte}, el monopolio del Gas de las principales poblaciones,
la explotación de Riotinto\ldots{} Dándose mucho tono, reservándose,
como quien aspira por sus propios méritos a una elevada cotización,
celebró Teresa más de una conferencia con Isaac Brizard, y mientras
exploraba el terreno, su perspicacia descubrió que el tal traía dinero
fresco y abundante, harto más lucido que las escatimadas riquezas
territoriales de nuestros nobles, los cuales viven comúnmente empeñados,
y son esclavos de sus administradores, o del precio que en cada año
alcanzan la cebada y el trigo. La importación de capitales extranjeros
limpios de polvo y paja estimábala Teresa como una de las mayores
ventajas para la Nación. Que aquí se quedara, derramado en cualquier
forma, todo el dinero que viene para negocios, era una bendición de
Dios.

Cuando Teresa se hallaba en los días de resistencia, de coquetería, de
pruebas, redoblaba Isaac sus galanteos, que a menudo llevaban séquito de
regalitos costosos y del mejor gusto. Como dijera un día la moza que su
niñez había sido muy desolada y triste, que jamás tuvo una muñeca
bonita, el francés le mandó por la noche dos elegantísimas, de la tienda
de Scropp, una y otra vestidas con tanto primor como cualquier señorita
de la más alta nobleza. La una decía papá y mamá; la otra movía los
ojitos, y ambas tenían articulaciones, con las que se les daban
graciosas posturas. Agradeció Teresa este obsequio como el más delicado
que podía ofrecérsele, y todo el santo día se lo pasó jugando con sus
nuevas amiguitas y diciéndoles mil ternuras a estilo maternal, entre
caricias y besos. Deseaba Isaac obsequiar a Teresita con un espléndido y
delicado banquete, sin más compañía que la de uno o dos buenos amigos,
de lo más selecto de la sociedad. Dos comederos elegantes había entonces
en Madrid: Farruggia y Lhardy. Pero en ninguno de los dos veía Brizard
la disposición de aposentos que la reserva exige. Gabinetes con efectiva
independencia no había en ninguna de las dos casas. Como no era cosa de
llevar a la sin par Teresa al \emph{Colmado de Rueda}, en la calle de
Sevilla, o a la \emph{Tienda de los Pájaros}, discurrió el bueno de
Isaac un arbitrio que resolvía dos problemas: el del convite y el de la
instalación de Teresa, con cuyo rendimiento contaba ya como hecho
indudable. Con tanto barro a mano, fácil le fue al extranjero alquilar
un bonito piso en casa nueva, calle de Santa Catalina, y amueblarlo, si
no totalmente, en la parte de sala y comedor. Lo demás de la casa se
completaría pronto: ya estaba todo encargado a Prévost, el mueblista más
caro y elegante de aquellos tiempos. Dispuestas así las cosas, Isaac
encargó a Farruggia la comida para cuatro personas. Había, pues, dos
invitados.

Si los periódicos pudieran dar cuenta de estas cosas, habrían dicho, en
Octubre de aquel año (no consta el día): \emph{«Verificose} el anunciado
banquete\ldots{} tal y tal\ldots» Pero lo que no dice el periódico lo
dice el libro. Bella sobre toda ponderación, y elegante como las propias
hadas, si estas se ajustaran a la moda, estaba Teresa, que con seguro
instinto sabía combinar en su atavío el lujo y la modestia, y con
infalible puntería daba siempre en el blanco de agradar a los hombres de
gusto. Admirable era su tez, de blancura un tanto marfilesca, sin ningún
afeite ni polvos, ni nada más que lo que al pincel de Naturaleza debía;
hechicera su boca fresca, estuche de los mejores dientes del mundo;
arrebatadores sus ojos negros, con un juego de miradas que recorrían
todos los registros, desde el más burlesco al más ensoñador; deliciosas
las dos matas de pelo castaño que se partían sobre la frente,
extendiéndose en bandas, no con tiesura pegajosa, sino con cierta
ondulación suave, un trémolo del cabello que iba a parar tras de la
oreja, bordeándola graciosamente. El cuello era un presentimiento de la
garganta y seno, que no se dejaban ver, pues la pícara tuvo la sutil
marrullería de no presentarse escotada. La tela vaporosa contaba en
lenguaje estatuario todo lo que dentro había. El traje, de color malva
claro, apenas lucía sus cambiantes entre una niebla de finos encajes; la
cintura delgadísima enlazaba el abultado pecho con la ampulosa
magnificencia del bulto inferior, todo hinchazón de telas alambradas. En
la jaula del miriñaque desaparecían de la vista las caderas y toda la
demás escultura infracorpórea de la mujer. La moda exhibía la mitad de
una señora colocada sobre la mitad de un globo.

Presentados por Isaac los dos amigos, Teresita les acogió con graciosa
sonrisa; ocupó su sitio, diciendo a los tres que se sentaran, que no
anduvieran con ceremonia, que hablasen con libertad, pues tanto le
gustaba a ella la libertad como le cargaban los cumplidos, y los criados
de Farruggia, limpios y estirados, empezaron a servir. El más joven de
los convidados, Ernestito de Rementería, esposo de Virginia Socobio,
poco había cambiado en figura y acento desde la época de su matrimonio,
como no fuera que eran algo más orondos sus mofletes, y más chillona y
delgada su voz. Desde la desaparición de su mujer, que se escapó con un
pintor de puertas, llevaba Ernestito una vida serena, cachazuda y
metódica, distribuyendo su tiempo entre los trabajos de \emph{La
Previsión}, junto al papá, el honesto recreo de regir un cochecillo en
la Castellana, y la monomanía de coleccionar objetos diversos, que un
día fueron bastones, luego petacas y fosforeras, y por último, se había
dado a las celebridades europeas en fotografía y grabado. Conservaba
\emph{el joven Anacarsis} el tipo de sacerdote francés con melenita, la
escasez de pelo de barba, la finura empalagosa de su trato, y la
absoluta insubstancialidad de cuanto decía. El otro convidado era en
realidad un grande hombre, figura de primera magnitud en la historia
social del siglo XIX, y tan notable por su facha, que era la de un
perfecto aristócrata, como por su trato, el más afable y seductor que
imaginarse puede. Viéndole una vez, ¿quién olvidaba la corpulenta y
gallarda estatura de aquel señor, su cuerpo bien distribuido de carnes y
más grueso que flaco, su faz risueña que declaraba el contacto y
serenidad de una vida consagrada a los goces, sin ningún afán ni
amargura? Don José de la Riva y Guisando era un hombre que parecía
simbolizar la posesión de cuantos bienes existen en la tierra, y el
convencimiento de que nos ha tocado, para pacer en él y recrearnos, el
mejor de los mundos posibles.

Hay tanto que decir de Riva Guisando (para los íntimos, Pepe Guisando),
que no conviene decirlo todo de una vez, sino soltar el personaje en
esta historia, para que él mismo hablando se manifieste, y sea fiel
pintor de su persona y el intérprete más autorizado de sus ideas\ldots{}
Cuatro palabras ahora para describir el físico y algo del ser moral de
Isaac Brizard: Casi tan alto como Riva Guisando, no podía comparársele
por la nobleza y arrogancia de la figura. Podía Guisando servir de
modelo a todos los duques y aun a los más estirados príncipes de Europa.
Isaac, igualando a su amigo en la intachable limpieza, no podía ser
modelo de próceres, sino de apreciables sujetos, hijos de negociantes y
educados en los mejores colegios de Francia. Guisando fue un elegante
genial, que todo lo había aprendido en sí mismo, y nació con la
presciencia de cuantas ideas y formas constituyen elegancia en el mundo.
Brizard era un producto de la educación, un hombre distinguido y
pulquérrimo, de un excelente fondo moral, con tendencias al vivir cómodo
y sin bambolla, ni envidioso ni envidiado\ldots{} Y por fin, para que se
vea todo en su propio color y sentido, el tipo de Isaac Brizard revelaba
la hibridación franco-germánica o franco-flamenca, un admirable tipo
engendrado por trabajadores, sano, leal, ordenado hasta en los
desórdenes a que le empujaba su riqueza, de ojos azules que delataban al
hombre confiado y bondadoso, la boca risueña, sobre ella un bigote
menudo, del más fino oro de Arabia. Hablaba un español incorrecto, mal
aprendido en la conversación y sin principios, con modulaciones
guturales que le resultaban más feas por su afán de corregirlas o
disimularlas. Al reproducir aquí su lenguaje, se tiene con este
simpático extranjero la caridad de enmendarle las desafinaciones del
acento.

«Eh, señores, ¿cómo se llama esta sopa?---dijo Teresa riendo con
deliciosa sinceridad.---Ya irán ustedes notando que soy muy
bruta\ldots{} Me parece que me pondría más en ridículo dándomelas de
fina, y queriendo ocultar mi ignorancia\ldots{} Pues esta sopa, yo no sé
lo que es ni la he comido en mi vida. Casi nada sé de comidas francesas;
no entiendo los motes raros que ponen a cada plato\ldots{} ¿Verdad que
soy muy bruta?

---Usted es hechicera, y esta sopa es, o quiere ser, \emph{potage a la
Montesquieu}---dijo Guisando, erudito y galante.---Cualquier otro nombre
le cuadraría mejor.»

Acordándose de su colección de celebridades, Ernesto quiso amenizar la
reunión con este comentario: «¿Montesquieu\ldots? Tengo dos retratos del
gran francés: uno de ellos en talla dulce, de la época\ldots{}

---Está buena la sopa---observó Teresa.---¿Pero a qué sabe? ¿de qué
legumbres está hecha?»

La opinión de Isaac no pudo ser más sensata: «En culinaria, el cocinero
debe saber mucho, y el que come ignorarlo todo. Así come uno más
tranquilo.

---Perdóneme, mi querido Brizard---dijo Riva Guisando,---que no le
acompañe en esos distingos. Saboreamos mejor los productos de la
culinaria, cuando sabemos a qué saben, y con qué ingredientes han sido
compuestos\ldots{}

---¿Pero es esto un puré de pepinos, de patatas, o qué demonios
es?---preguntó Teresa, sin que las dudas mermaran su apetito.

---No es más que una mixtificación, a la que ponen el primer nombre que
se les ocurre---afirmó Guisando.---Cuando Isaac me hizo el honor de
invitarme a esta comida, que, entre paréntesis, sería deliciosa aunque
la prepararan los cocineros más malos del mundo, volví a mi cantinela de
siempre con el amigo Farruggia: «Las sopas caldudas y crasas pasaron a
la historia\ldots{} Ya que usted se propone enseñar a los españoles a
comer, trate de propagar, de popularizar los \emph{consommés} finos, tan
substanciosos como transparentes\ldots» Le propuse para esta interesante
comida el \emph{Consommé a la creme de faisán}, que es delicioso,
verdaderamente delicioso, Teresa\ldots{} y sencillísimo\ldots{} verá
usted.

---¡Ay, enséñeme!\ldots{} Me gusta cocinar algo\ldots{} Poco sé\ldots{}
Quisiera poseer el secreto de algún platito delicado\ldots{}

---Sencillísimo, como digo. Todo el arte está en preparar los huevos,
que se sirven aparte\ldots{} Se cuecen huevos bien frescos, de polla
precisamente, de gallina joven\ldots{}

---¡Ay!\ldots{} ¡Lástima no tener gallinero en casa! Adelante.

---Luego se les vacía\ldots{} se saca la yema por medio de un
tubito\ldots{}

---Tanto instrumento ya es por demás.

---Con las yemas y el picadillo de pechugas de faisán, se hace la
\emph{farce a la creme}.

---¿Y esa farsa, qué es?

---El relleno\ldots{} Se rellenan los huevos\ldots{} se ponen al baño
María\ldots{}

---¡María Santísima!

---En vez de faisán, puede usarse perdiz, bien fresca\ldots{}

---Yo sí que estaría fresca si me metiera en esos trajines tan
enredosos.

---Amiga mía, no necesita usted cocinar. Bien se ve que lo haría con
mucha gracia si a ello se pusiera\ldots{} Ya tendrá usted un buen
\emph{jefe} que la libre de esos quebraderos de cabeza, y del deterioro
de sus manos lindísimas.»

Viendo que le servían Jerez después de la sopa, protestó Teresa con
sincero desenfado: «¡Eh, caballeros! que el Jerez se me sube al quinto
piso\ldots{} Repito que soy muy bruta\ldots{} no tengo costumbre de
beber tanto, ni de variar de bebidas\ldots{} ¿Quieren verme peneque?»
Aseguró Isaac que todo era cuestión de costumbre, y que debía poco a
poco educarse en el comer fuerte, acompañado de bebida confortante.

«¡Ay, ay! eso no va conmigo\ldots---dijo Teresa, probando el
Jerez.---Porque ustedes no me crean demasiado palurda, bebo un poquito;
pero no se asombren si me ven perdida de la cabeza, y diciendo algún
disparate.»

Ernestito, dando ejemplo de buen tono, equivalente a la poca sobriedad,
se atizó dos copas, comentándolas en esta forma: «La sopa y el Jerez no
tienen en las comidas otro objeto que preparar el estómago, darle
fortaleza\ldots{}

---¿Para qué?

---Para comer, para seguir comiendo\ldots{} Ahora empezamos, señora mía.
Yo, ya lo irá usted notando, como bien. Desde que entré en el Colegio
Flaminio, en Saint-Denis, aprendí a comer bien, dando al cuerpo todo lo
que pedía. Es un gran sistema para tener siempre la cabeza\ldots{}

---¿Cómo?

---Despejada\ldots{} y las ideas claritas. Es lo que yo recomiendo
principalmente a todos mis amigos: que coman fuerte\ldots{}

---Y con la recomendación les mandará usted la comida, porque si
no\ldots{}

---Eso es cuenta de ellos, y de que quieran tener salud o no tenerla.
Repare usted, Teresita, que todos los grandes hombres han sido de buen
diente. Federico el Grande, de cuyos retratos poseo la colección más
lucida que hay en España, profesaba la doctrina de Rabelais: cinco
comidas y tres siestas. Talleyrand consagraba toda su atención a la
buena mesa. Mi padre, que es hombre muy entendido en todos los adelantos
extranjeros, no cesa de predicar a los españoles que se den buena vida,
la mejor vida posible, y sostiene que uno de los mayores atrasos de este
país consiste en que aquí no saben comer.

---Es verdad---dijo Guisando:---reconozcamos una de las deficiencias que
nos ponen a la cola de las demás naciones. Los españoles no saben
comer.»

\hypertarget{xviii}{%
\chapter{XVIII}\label{xviii}}

Sirvieron pastelitos de \emph{foie-gras}\ldots{} después un plato de
pescado que Guisando tradujo al francés: \emph{Turbot bouilli, garni,
sauce Colbert}, y entre tanto, los cuatro comensales apuraron el tema de
si saben o no comer los españoles. Ingenioso y ameno, Riva Guisando se
despachó a su gusto en esta forma: «No podemos dudar que, de algunos
años acá, nuestro país viene entrando en la civilización, y asimilándose
todos los adelantos. Eso lo vemos en diferentes órdenes. Nuestras casas
adquieren el \emph{confort} de las casas extranjeras. Verdad que falta
el agua, pero ya vendrá; la tenemos en camino. Nuestros teatros no
desmerecen de los de otros países; y en ópera creo yo que estamos a la
altura de las capitales más aristocráticas. Nuestras mujeres, bien a la
vista está, visten con tanto gusto y elegancia como las parisienses, y
nuestros hombros no tienen nada que envidiar a los caballeros ingleses
mejor vestidos\ldots{} Sólo en el comer estamos atrasados\ldots{} Fuera
de unas pocas casas, hasta las familias más ricas no saben salir del
cocido indigesto, y de los estofados, pepitorias y fritangas\ldots{} Y
en la manera de comer guardan la tradición: se atracan y no comen
realmente; no saben lo que es la variedad, la composición artística de
las viandas para producir sabores especiales y excitantes; no han
llegado a penetrar la filosofía del condimento, que es una filosofía
como otra cualquiera\ldots{} En el beber, tragan líquidos, sin apreciar
el rico \emph{bouquet} de cada uno, sin distinguir los innumerables
acentos que forman el lenguaje de los vinos. Cada uno dice algo distinto
de lo que dicen los demás\ldots{}

---¡Alto ahí!---exclamó Teresa cortándole el discurso con delicioso
tonillo y ademán de burlas;---perdone usted, señor Guisando, que le
interrumpa. Si los vinos son cada uno una palabra, un acento, y todos
juntos como lenguajes; si los de España hablan español, francés los de
Francia, y así los demás, ustedes quieren introducirme a mí en el cuerpo
la torre de Babel\ldots{} Vamos, que a poco más, salgo hablando todos
los idiomas.

---No, no, Teresa---dijo prontamente Brizard;---no se bebe para
embriagarse, ni se embriagan los que saben beber\ldots{} La bebida fina
y variada es un signo de civilización. En eso estoy con el amigo
Rementería y con Guisando\ldots{} ¡Oh! en Guisando hay que reconocer un
gran civilizador.

---Civilizador usted---replicó el elegante caballero,---que nos trae la
más grande forma del Progreso, los ferrocarriles.

---Es verdad; de eso trato, y mi mayor gloria será vestir a España de
país civilizado\ldots{} Usted y yo civilizamos; pero permítame que
marque entre los dos una diferencia\ldots{} una diferencia en que yo
salgo favorecido. Usted empieza la campaña civilizadora por el fin, mi
querido Guisando, porque quiere enseñar a los españoles cómo se come; yo
la empiezo por el principio, enseñándoles a buscar lo que han de comer.

---¡Eso\ldots{} eeeeso!---gritó Teresa risueña, con desbordada alegría,
las mejillas echando fuego, el gesto más expresivo y acentuado de lo que
pedía la compostura.---El señor Guisando se trae aquí la filosofía de la
buena mesa, y quiere enseñársela a un pueblo que no tiene sobre qué
caerse muerto. ¿Cómo quiere usted que sepa comer el que no come? Y esas
salsas Colbert, esas \emph{besamelas}, esas \emph{farsas} o rellenos,
esos \emph{rosbifes}, y \emph{chatobrianes}, y gigotes, y esas trufas y
esos jugos, ¿de dónde han de salir? ¿Reparte usted diariamente un par de
monedas de cinco duros por barba a todos los españoles?\ldots{} ¡Ay, ay!
Yo les suplico, señores míos, que me den licencia para callarme\ldots{}
Siento que el disparate se me viene a la boca, y a poco que me descuide,
oyen ustedes una barbaridad. Es mucho comer este, es mucho beber, para
que una tenga la cabeza despejada. Perdónenme; estoy un poquito a medios
pelos\ldots{} Me callo\ldots{} Ustedes me agradecerán que cierre el
pico.»

Dejó el tenedor, y requiriendo el abanico, empezó a darse aire con
viveza. Los caballeros le reían la gracia; celebraban que se trastornase
un poquitín, y asegurando que el encendido color y el chispeante mirar
la embellecían extraordinariamente, incitábanla a beber del rico
\emph{Borgoña} que a la sazón servían. Pero ella no hacía caso, y jovial
agitaba el abanico con verdadero frenesí, diciendo: «Yo, punto en boca:
no vayamos a salir con alguna patochada. Me conozco. Hablen ustedes y yo
escupo, digo, yo callo y otorgo\ldots»

Tan modesto como ingenioso, Guisando se mostró conforme con las ideas de
Isaac, reconociendo en el magisterio civilizador de este más sentido
práctico que en el suyo. «Es cierto, Brizard: usted trae a España los
primeros elementos del bienestar. Por ahí se principia. Yo empiezo por
el fin, porque no sé otra cosa. Cada uno comienza sus lecciones por
aquello que más sabe\ldots{} En la mente del discípulo siempre queda
algo de la enseñanza que se le da, por más que esta sea prematura. Yo
digo a los españoles: «No sabéis comer;» usted les dice: «Trabajad y
comeréis.» Claro es que usted está en lo firme. Yo, si bien se mira, soy
un profesor extravagante que coge a los chicos cerriles que no saben
leer ni escribir, y se pone a explicarles las asignaturas del
doctorado\ldots{} Pero todo es enseñanza, amigo. Algo quedará\ldots»

Sirvieron el plato de legumbres, que Guisando y Ernesto celebraron
mucho, definiéndolo así: \emph{concombres farcis à la demiglace}. Pidió
Isaac su opinión a Teresa, la cual se dejó decir: «Señores míos, la
turca que estoy cogiendo, no por mi gusto, sino por el empeño de ustedes
en que yo empine más de lo regular, no me deja ser hipócrita. Quiero
mentir con finura y no puedo\ldots{} Esos \emph{concombros} me parecen
una porquería. Si mi cocinera me presentara este comistraje, yo le
tiraría la fuente a la cabeza.» Servido el asado, Teresa se resistió a
comer más. Obstinose Guisando en servirle una bien cortada lonjita del
\emph{Chapon à la financière}; regateó Teresa; cedió al fin con salados
remilgos.

Debe decirse que la hermosa mujer, cuya iniciación en la vida grande
aquí se describe, exageraba su torpeza o su ignorancia de los
refinamientos sociales. No los desconocía en absoluto; pero dotada de
grande agudeza, calculó, antes de personarse en el banquete, que la
afectación de finura podría llevarla, sin que de ello se diera cuenta, a
una situación algo ridícula. Mejor y más airoso era la contraria forma
de afectación, hacerse la palurda, la novata, todo ello desplegando su
natural donosura. Y el resultado de esta táctica fue tal como ella lo
pensó, admirable y decisivo. Isaac parecía extasiado; celebraba con
entusiasmo las donosas salidas y sinceridades de la que pronto había de
ser suya, y gozaba con la idea de educarla y darle un curso de todas las
leyes y toques del buen gusto. Bien comprendía la muy ladina que a los
extranjeros agrada lo que llaman \emph{carácter}, \emph{color local}, y
que se enamoran de lo que menos se parece a lo de su tierra\ldots{}
Isaac, prendado locamente de la española, en ella simbolizaba la
conquista de esta tierra, mirándola con amor y sembrando en ella ideas
fecundas y fecundos capitales.

Una de las condiciones propuestas por Teresa en el trato de amor con
Brizard, era que este había de llevarla a París y tenerla allí una
temporadita, aprovechando el primer viaje que tuviera que hacer a la
capital vecina. Con alegría dio Isaac su aprobación a esta cláusula. De
ello y de los encantos de París en el segundo Imperio hablaron los tres
caballeros en la comida, dando pie a Teresa para que se despachara a su
gusto y con desenvoltura en este tema: «Mucho me gustará París. Tantas
maravillas he oído contar, que ya me parece que las he visto\ldots{} De
seguro me divertiré y aprenderé; pero todas las cosas buenas de París no
me quitarán el ser española neta\ldots{} Española voy, y más española
vuelvo\ldots{} ¿Que aprenda yo francés? Imposible, Ernestito\ldots{}
\emph{Tarde piache}. Cuatro palabras aprendí en mi colegio, y con esas
cuatro palabras y otras cuatro que allá me enseñen, me arreglaré\ldots{}
Dicen que la Emperatriz Eugenia, con ser nada menos que Emperatriz, no
ha querido afrancesarse\ldots{} Y yo pregunto: ¿por qué usará Napoleón
esos bigotes engomados tan largos y tan tiesos?\ldots{} No me hagan
caso; estoy perdida de la cabeza\ldots{} París, con todos sus
monumentos, no vale lo que Madrid, que tiene las grandes plazas\ldots{}
Puerta Cerrada, la Red de San Luis, y como \emph{bulevares}, ¿dónde me
dejan ustedes el Postigo de San Martín y la Costanilla de los
Ángeles?\ldots{} París es bonito, alegre, y con cuatro magníficas
fachadas al Mediodía, como quien dice, al Amor\ldots{} todas las
fachadas dan al Amor\ldots{} En París hay mucho dinero, es la ciudad del
dinero\ldots{} y por ser aquel pueblo tan rico, hay allí más honradez
que en los pueblos pobres\ldots{} En los pueblos tronados viven todos
los vicios\ldots{} No me hagan caso\ldots{} ¿Verdad que estoy diciendo
sin fin de disparates? No sé lo que digo\ldots{} Me han hecho ustedes
beber más de lo que bebe una señora fina\ldots{} No tengo
costumbre\ldots{} Soy lugareña y tonta\ldots{} Las tontas se emborrachan
antes que las listas\ldots{} y a las honradas se les va la cabeza más
pronto que a las disolutas\ldots{} Yo me callo\ldots{} Estoy
avergonzada.»

Protestaron los caballeros de esta falsa vergüenza, y Guisando le dijo:
«Está usted adorable, y el mareíto se le quitará bebiendo esta copa de
\emph{Champagne}\ldots» Isaac le rogó que bebiese, y ella sin melindres
accedió. Le gustaba el \emph{Champagne}: si pudiera, no bebería en las
comidas más que \emph{Champagne}\ldots{} La variedad de vinos le
repugnaba: uno solo y superior. Guisado celebró esta opinión de Teresa,
la más conforme con el gusto de él y de toda persona verdaderamente
refinada. «Bebo---dijo Teresa tomando la copa larga, por cuya boca
estrecha se escapaba la espuma,---bebo a la salud de mis buenos amigos;
bebo a su felicidad, y a\ldots{} a que tengan lo que desean\ldots{}
Usted, Isaac, que le salga bien el negocio que ahora le trae tan
preocupado\ldots{} ya me entiende\ldots{} Usted, Guisando, que sea
pronto Grande de España, por título\ldots{} que ya lo es grandísimo por
su magnificencia\ldots{} y usted, Ernesto, que haga muchas conquistas,
pues ya sabemos que es usted muy enamorado\ldots{}

---¡Oh, no, no!---dijo el plácido Anacarsis, presuroso en desmentir una
suposición que, a su parecer, le desconceptuaba.---¿Enamorado yo? No es
cierto, Teresa\ldots{} Bien se ve que se le ha ido el santo al
cielo\ldots{} Exceptuando lo presente, tengo del bello sexo la peor
idea\ldots{}

---Pues perdóneme usted, Ernestito: no he dicho nada. Somos muy
malas\ldots{} Usted puede decirlo\ldots{} y probarlo\ldots{} Es usted un
ángel\ldots{} por eso tiene esos colores tan bonitos y esa frescura en
el rostro\ldots{} Señores, el \emph{Champagne} me ha matado. ¿He dicho
muchas gansadas?

---No, no, no\ldots{}

---Ya no puedo más\ldots{} Se me cierran los ojos\ldots{} El comedor da
vueltas\ldots{} la mesa baila\ldots{} Guisando tiene dos caras: con las
dos me mira y se ríe. Ernestito se pone sobre la cabeza el ramo del
centro de la mesa\ldots{} Me duermo, me\ldots{} eclipso; me envuelve la
noche. Isaac, por favor, deme usted la mano; ayúdeme a levantarme, y a
llegar al sillón\ldots{} al sillón que allí veo\ldots{} Así, así\ldots{}
ya estoy a mi gusto\ldots{} Aquí me desmayo\ldots{} aquí me
desvanezco\ldots{} Por Dios, Isaac, mi buen Isaac, abaníqueme usted,
deme aire; pero fuerte\ldots{} Ya no veo más cara que la de usted,
Isaac\ldots{} El aire que usted me da me consuela, me anima\ldots{} ¡Qué
aire tan bonito, digo, tan fresco\ldots{} tan\ldots! No sé: es un aire
extranjero\ldots{} aire rico, muy rico\ldots{} Isaac, deme más
aire\ldots{}

---Café bien fuerte---dijo Guisando proponiendo el mejor específico
contra las borracheras de señora de buen tono.»

Con la ventilación enérgica que le administró Isaac, y el café y la
dulce conversación, sin ruido, se fue despabilando Teresa y venciendo la
somnolencia. Terminó la comida sin ningún incidente digno de figurar en
la Historia integral ni en la fragmentaria, pues el hecho de arreglarse
y cerrar trato aquella misma noche Teresita y Brizard es de esos que,
por descontados y claramente previstos, no piden más que una
mención\ldots{} menos aún, una raya de cualquier color trazada en la
página sin letras de esa historia que llamamos \emph{Chismografía}.

\hypertarget{xix}{%
\chapter{XIX}\label{xix}}

Y esa historia sin letras dice que Teresita se instaló en la misma casa
del ya referido banquete, días después de la partida de Aransis para la
gloriosa y coruscante Atenas, como Encargado de Negocios de la Católica
Majestad de Isabel II en aquel Reino. Obra fue del buen amigo Beramendi
este destierro, ayudado por Narváez, quien tomó el asunto como propio y
lo resolvió con diligencia. Llamado a París Isaac Brizard por el reclamo
de sus negocios, determinó partir en Noviembre, llevándose a Teresa,
conforme a lo convenido. Ni a esta causaba temor el viaje en pleno
invierno, ni quería separarse de Isaac, que era para ella el mejor de
los hombres, extremado en la bondad y en la largueza, prodigando sin
tasa su dinero como su cariño. Sobre el punto interesante del estipendio
de amor, Teresita veía colmadas sus ambiciones. El gozo de ver
satisfechos todos sus gustos se completaba con la dicha de tener
sobrantes y de atender con ellos a necesidades ajenas, empezando por su
madre, que era una boca no fácil de tapar. Pero en aquella venturosa
etapa para todo había.

Con sus íntimas amigas tuvo Manolita Pez algunas confianzas que merecen
ser consignadas en estos papeles: «A Teresa la ha venido Dios a ver con
ese francés tan frescachón y tan caballero. Ya quisieran los nobles de
aquí parecérsele en la lozanía del rostro, que es lo mismo que una rosa,
y en la mano siempre abierta para complacer a su adorada. Yo le he dicho
a Teresa que no aparte sus ojos del porvenir\ldots{} Además del tanto
fijo que \emph{Musiú} Brizard le señale para la vida corriente, debe mi
hija poner todo su talento en sacarle \emph{un millón}\ldots{} ¿Qué es
un millón para una mujer de tanto mérito? Y con este capitalito ya puede
la niña echarse a dormir\ldots{} El día de mañana, si ese señor pasa a
mejor vida, lo que no quiera Dios, o si por envidia le arman algún
enredo para que rompa con mi hija, esta podrá bandearse sola, sin tener
que aguantar las pejigueras de un vejete baboso, de un puerco, de un tío
cargante; y aun podría encontrar proporción de matrimonio. Con el
milloncito todo se olvidaría, ¡vaya!\ldots{} ¡Y que tendría mi Teresa
mal gancho para pescar marido; y este no había de ser un cualquiera,
sino persona de algún viso, y quizás con el pecho cargado de cruces y
bandas!»

Con Centurión no se trataba Teresa directamente, y bien lo sentía, que
para ella no habría mejor gusto que poder acudir al remedio de las
escaseces que a don Mariano le trajo su cesantía. Sabía de él y de doña
Celia por su tía Mercedes, la mujer de Leovigildo Rodríguez, con quien
reanudó el trato después de una temporadita de moños. También Leovigildo
estaba cesante, situación lastimosa en aquel honrado matrimonio, cargado
de familia. La pobre Mercedes, al poco tiempo de desembarazarse de una
cría, ya se mostraba con los evidentes anuncios de otra. Y creyérase que
en los períodos de cesantía procreaban más los desgraciados cónyuges. La
sociedad quería matarlos de hambre, y ellos aumentando sin cesar el
número de bocas. No faltaban, afortunadamente, personas caritativas que
se condoliesen de su desamparo y fecundidad, entre ellas Teresa, que les
enviaba surtido de zapatos para toda la cáfila de criaturas, o repuesto
de arroz y garbanzos para muchas semanas. Don Mariano, que había tomado
entre ojos a los Villaescusas de una y otra rama, no quería tratarse con
la esposa de Leovigildo; pero doña Celia, más benigna, la visitaba
algunas tardes a hurtadillas de su marido. La señora de Centurión y
Manolita Pez se encontraban algún día en un terreno neutral, la casa de
Nicasio Pulpis, esposo de Rosita Palomo, y allí, rompiendo doña Celia la
consigna que su marido le diera de no tener trato con la Coronela ni con
su depravada hija, hablaban de sus respectivas desazones. La curiosidad
más que el afecto, movía comúnmente a doña Celia Palomo a preguntar por
Teresa; respondía Manuela, tratando de dorar la deshonra de su hija con
hábiles artificios de palabra.

Con la de Navascués no había vuelto a tener Teresita ningún trato.
Traidora y desleal llamaba Valeria a la que fue su amiga, y no le
perdonaba el solapado ardid que empleó para sustraerle \emph{el libro de
texto}. Mala partida como aquella no se había visto nunca. Dos o tres
veces se cruzaron las dos hembras en la calle, y se dispararon miradas
rencorosas. No desconocía Valeria que para ella había sido un bien la
retirada de Aransis, que arruinado ya, no era partido de conveniencia
para ninguna mujer. Pero esta consideración no le quitaba el reconcomio
contra Teresa, en quien, por otra parte, reconocía un magistral talento
para conducirse en sus empresas de amor, y prueba de ello era la
reciente pesca del opulento francés Isaac Brizard. Sin duda por llevar
tan buena parte en los favores de la suerte, Teresa no se cuidaba de
aborrecer a su víctima. Más bien le tenía lástima, sabedora de que la
pobrecilla andaba mal de intereses. Por las prenderas que \emph{corrían}
trajes de lujo en buen uso, supo que Valeria lanzaba al mercado de
ocasión, malbaratándolas, algunas piezas de valor, abrigos, cachemiras,
mantón de la China. Supo también que a la famosa corredora \emph{Paca la
Bizca} debía un pico de consideración por dos sortijas y un alfiler que
adquirió antes del destierro de Navascués. De esto tomó pie Teresa para
lanzar contra Valeria una bomba en la que había de todo, burla y
compasión. Era la travesura de la enemiga vencedora, que sintiéndose
fuerte, quería mortificar a su rival en una forma que le expresara su
lástima desdeñosa, su generosidad, quizás el deseo de hacer las paces.
El día antes de su partida para Francia, Teresa escribió esta carta:
«Estimada maestra y amiga: Un pajarito me trajo el cuento de que la
respetable corredora \emph{Paca la Bizca} te hizo dos mil y tantas
visitas para que le pagaras dos mil y tantos reales de aquel alfiler y
sortijas de marras\ldots{} Sé que cuantas veces fue la corredora a tu
casa con este objeto, salió con las manos vacías\ldots{} Pues bien; para
que veas si te estimo, Valeria, hoy he dado a Pepa los dos mil y pico,
encargándole que no vuelva a molestarte por esa bicoca. Acepta este
favor de la que fue tu amiga, y no te atufes ni salgas ahora con pujos
de una dignidad que habría de ser fingida\ldots{} No tienes que
devolverme esos cuartos, que ahora los tengo de sombra, gracias a
Dios\ldots{} Abur, bobita. Mañana salgo para París, donde me tienes a tu
disposición para todo lo que gustes mandarme.---Tu fiel compañera,
\emph{Therese Brizard.»}

Mostró Teresa esta carta al bueno de Isaac, para que después de leerla
le dijese cómo había de poner su nombre en francés. Hallábase presente
Riva Guisando, y ambos amigos celebraron el rasgo generoso y la gracia
zumbona, que de todo había. Partieron los amantes a París al día
siguiente; despidioles Guisando al arrancar la silla de postas, de la
propiedad de Brizard, y por la tarde se fue a visitar a su amiga la
marquesa de Villares de Tajo (Eufrasia), a quien contó lo de la carta de
Valeria, repitiéndola casi textualmente. Bien conocía la dama los
enredos de la sobrina de su esposo, y la depravación que se iba marcando
en ella. Después de comentar y reír al caso de la carta, \emph{la
moruna} rompió en este bien entonado epifonema: «¡A qué extremo llegan
ya, Dios mío, los desvaríos de esta sociedad!\ldots{} ¿A dónde vamos a
parar por tal camino? Mentira parece que esas dos chiquillas, tan monas,
tan inocentes cuando vine yo de Roma casada con Saturno, se hayan
perdido escandalosamente, cada cual a su modo. Virginia, con las
antorchas de Himeneo aún encendidas, se escapa con un chico menestral, y
anda por esos pueblos hecha una salvaje, y esta Valeria corre a la
perdición amparada del formulismo matrimonial, con lo que me resulta más
perversa que su hermana.»

Dijo a esto Guisando que Valeria claudicaba por espíritu de imitación,
sin arte ni riqueza para cohonestar sus incorrecciones. Dos cosas
redimían del pecado, según la filosofía guisandil: el buen gusto y la
opulencia. Las maldades parecían peores cuando eran feas\ldots{} y
pobres. Todo era relativo en el mundo, hasta los vicios. De estas
opiniones casuísticas no participaba Eufrasia, que en aquel punto de su
existencia (los treinta y cinco años) se dedicaba con ahínco a señalar a
la juventud los caminos derechos, únicos que conducen a la virtud y a la
paz del alma. Era, en aquel período histórico, la conducta de la
Villares de Tajo mejor y más limpia que su fama. El mundo, que en la
plenitud de tantos escándalos exageraba los desvaríos de la sociedad
matritense, la suponía en amores con Riva Guisando. ¡Falsa y calumniosa
especie! ¿Mas quién destruye un errado juicio en tiempos en que el aire
viciado divulga, no sólo la corrupción, sino las vibraciones de ella
manifestadas en el lenguaje? Entre \emph{la moruna} y el espléndido
caballero y \emph{gourmet} Riva Guisando, no había más que una sincera y
noble amistad fundada en la armonía de pareceres sobre algunas materias
sociales, y por parte de él, ligero matiz de adoración platónica, que
tenía su origen en la gratitud, como a su tiempo se demostrará.
Preguntado el caballero por la distribución de sus comidas, dijo: «Esta
noche como en casa de Navalcarazo; mañana, en la Legación de los Estados
Unidos.

---Aunque tenga usted---le dijo Eufrasia,---que renegar una vez más de
la cocina española, el viernes comerá usted con nosotros\ldots{} Ya le
pondremos algo de su gusto: las famosas chuletitas de cordero \emph{à la
Bechamel}, y la tan ponderada \emph{Salade celeri et betterave}.

---Con esos ojos que ahora me miran---replicó el \emph{gourmet},---tengo
bastante\ldots{} Ya sabe usted que \emph{los ojos a la española} son mi
delicia\ldots{} Quedamos en que el viernes\ldots{}

---Apúntelo usted para que no se le olvide.»

Era Riva Guisando, como se ha dicho, un artista genial del buen porte,
de la buena vida, del buen comer\ldots{} Y esto debe repetirse al
consignar que su abolengo no fue tan humilde como la gente decía; ni
vendió pescado su madre, como propalaron los que querían denigrar su
arrogante persona. Nació en una capital andaluza, de familia decente,
privada de bienes de fortuna, y desde su más tierna infancia se
distinguió el muchacho por la compostura y aseo de su persona, por lo
afinado de sus gustos y su fácil asimilación de todo lo que constituye
la personalidad externa, y los medios del bien parecer. Vino a Madrid
muy joven en busca de fortuna, y a poco de llegar, su exquisita
educación y su prestancia no aprendida le proporcionaron relaciones
excelentes. Alternó con la juventud elegante; sabía ganar amigos, porque
a todos encantaba con su trato, y a ninguno con destempladas jactancias
ofendía. Era tan modesto en su alma como fastuoso en su cuerpo; su
orgullo no pasaba de la ropa para dentro. El primero en el vestir, no
anhelaba confundir a los demás por otra clase de superioridad, y poseía
el supremo arte de no lastimar a nadie, de contentar a todos,
conservando su dignidad. No creo que haya existido en Madrid más
consumado maestro de las buenas formas: por esta cualidad Madrid le debe
gratitud. De todo hemos tenido modelos admirables. ¡Lástima grande que
con modelos perfectísimos de cada una de las partes, no hayamos tenido
nunca el modelo sintético, integral!

Para vivir con tanto boato, introducido en la sociedad de los ricos,
Guisando no disponía de más caudal que su sueldo en Hacienda, y por los
años a que este relato se refiere, no cobraba el hombre arriba de diez y
seis mil reales. De su honradez daban testimonio algunos hechos que como
irrefutable verdad histórica deben consignarse aquí. ¿Qué era el buen
Guisando más que un milagro, el milagro español, ese producto de la
ilógica y del disparate que sólo en esta maravillosa tierra puede
existir y ha existido siempre? Ya se irá viendo esto, y por ahora,
léanse aquí los motivos de la gratitud de Guisando a la Marquesa de
Villares. Desde que esta le conoció en casa de los Condes de Yébenes, y
pudo enterarse de la formidable disonancia entre el \emph{Coram vobis}
de aquel sujeto y sus menguados medios de subsistencia, le miró con
interés y curiosidad. Aficionada \emph{la moruna} a las
generalizaciones, y ducha en buscar la entraña de las cosas, vio en él
como una imagen sintética de la sociedad de aquel tiempo. No podía
imaginarse nada más español que Guisando, debajo de sus apariencias
europeas. Tratándole después con cierta asiduidad, tuvo ocasión Eufrasia
de apreciar en él cualidades que al pronto le parecieron inverosímiles,
pero que al fin, por especiales circunstancias, pudo comprobar
plenamente. Ascendió Riva Guisando a Jefe de Negociado en la Dirección
de Rentas. Un amigo de los Socobios, don Cristóbal Campoy, ex-diputado,
tenía en aquella oficina un embrollado expediente, de esos que se
atascan en los baches de la administración, y no hay cristiano que los
mueva. Se recomendó el asunto a Guisando: este lo sacó del montón, lo
estudió y resolvió, como se pedía, en menos de una semana. Maravillado y
agradecido el señor Campoy, creyó que procedía recompensar la diligencia
del funcionario con un discreto obsequio en metálico, y sin detenerse
entre la idea y el hecho, dejó algunos billetes del Banco metidos en una
carta, sobre la mesa del arrogante andaluz, quien no tuvo sosiego hasta
remitirlos con atenta epístola a las manos del propio donante. ¿Era esto
moralidad intrínseca, o un \emph{bello gesto} de elegancia, un rasgo más
de gran artista social? De todo había. Honradez y arte perfeccionaban la
figura del caballero.

Al saber esto Eufrasia, se decía: «¿Pero cómo vive un hombre que sólo en
planchado de camisas ha de gastarse todo su sueldo, y aun puede que no
le baste?» Hablando de esto con algún amigo muy conocedor del mundo, oyó
\emph{la moruna} explicaciones aceptables de aquel milagroso vivir: «Se
pasa la madrugada en el Casino, y hace sus visitas a las mesas del 30 y
40. Hay muchos que de este modo se ayudan\ldots{} van viviendo.» Otros
casos, semejantes al de Campoy, que llegaron a conocimiento de la
Villares de Tajo, persuadieron a esta de la rectitud y caballerosidad
del atildado señor. Además, el trato frecuente le reveló en él otra
cualidad, rarísima en la esfera social de aquel tiempo. Poseía el
secreto de la conversación amena sin hablar mal de nadie. A todo el
mundo encantaba, sin emplear la ironía maliciosa. Defendía gallardamente
a los que en su presencia recibían daño de las malas lenguas, y cuando
la defensa era imposible, callaba\ldots{} Pues estas excelentes
cualidades del sujeto agradaron a la dama y la movieron a protegerle.
Cesante en el bienio, repuesto el 56 por influjo de Ros de Olano, le
puso en peligro un malhadado arreglo del personal de Hacienda; pero
Eufrasia acudió a Cantero, y no fue menester más para sostenerle. A la
caída de O'Donnell y elevación de Narváez, temió el \emph{gourmet} que
le perjudicara el haber sido recomendado por un general de la Unión;
pero la Marquesa habló expresivamente a Barzanallana, ponderándole la
capacidad y el celo del empleado andaluz, y esto bastó para que quedara
bien seguro en la nueva situación. El vulgo avieso y mal pensado vio en
esta protección lo que no había, pues si \emph{la moruna} endulzaba
entonces su existencia con algún pasatiempo amoroso, iba su capricho por
órbita muy distinta de la de Riva Guisando, y si en pasos de amor andaba
este, por querencia desinteresada o por estímulos de su ambición, no
pisaba los caminos de Eufrasia, su incomparable amiga y protectora. La
lógica de tal protección era que \emph{la moruna} admiraba al caballero
del milagro español, el único milagro que admitían tiempos tan
irreligiosos y corruptos, la suprema maravilla de ser grato a todos
ejerciendo la elegancia como virtud, y la virtud como arte. Era D. José
de la Riva algo nuevo y grande en nuestra sociedad: la esperanza del
reino del bienestar y de la alegría, destronando a la miseria triste.

\hypertarget{xx}{%
\chapter{XX}\label{xx}}

¡No había caído mala nube sobre nuestra pobre España! Los moderados, con
el brazo férreo de Narváez y la despejada cabeza de Nocedal, estaban
otra vez en campaña, comiéndose los niños crudos, y los buenos platos
guisados del presupuesto. Todo para ellos era poco: ni una plaza dejaron
para los infelices del Progreso y la Unión. A los españoles que no eran
borregos del odioso \emph{moderantismo}, les miraban como clase
inferior, esclava y embrutecida. ¿Era esto gobernar un país? ¿Era esto
más que una feroz política de venganza? A la Ley de Desamortización
dieron carpetazo, y en cambio sacaban nueva Ley de Imprenta, que no era
más que un régimen de mordaza, de Inquisición contra la grande herejía
de la verdad. Temblaban los ciudadanos que en su vida tenían algún
antecedente liberal; otros defendían sus personas y haciendas con el
ardid de la adulación. El alma de España cubríase de las nieblas del
miedo y en sí misma se recogía, como los inocentes acusados y
perseguidos que al fin llegan a creerse criminales.

Ya no se atrevía el iracundo Centurión a soltar en público sus honrados
anatemas. Temeroso de que sobre él o sobre sus buenos amigos recayese
algún duro castigo, licenció la tertulia del café de Platerías. Los
leales que le escuchaban como a un oráculo hubieron de congregarse en la
propia casa o templo de don Mariano, que al quedar cesante, tuvo que
cambiar la dispendiosa vivienda en la calle de los Autores por otra más
reducida y barata en la de San Carlos, esquina a Ministriles. Lo más
doloroso de la mudanza fue el transporte de jardines balconeros, y la
precisión de deshacerse de corpulentos árboles y enredaderas vistosas
que no tenían espacio en la nueva casa. Sobrellevó con cristiana
paciencia doña Celia este desmoche de su riqueza forestal, y don
Mariano, en un arranque de amargo pesimismo, entristeció más el alma de
su esposa con estos lúgubres conceptos: «Abandona, Celia, todas tus
plantas aromáticas y floridas, y trae a tus balcones un ciprés y un
llorón, únicos árboles que ahora nos cuadran. Cadáveres o poco menos
somos, y nuestra casa cementerio.»

A darle conversación iban algunas tardes el bajo Cavallieri, que se
defendía míseramente cantando en las misas solemnes y en los funerales
de primera; don Segundo Cuadrado, que con tétrico humorismo trataba de
regocijar los abatidos ánimos; Nicasio Pulpis, que iba pocas veces, casi
de tapadillo, con el solo fin de hablar pestes del Gobierno y
desahogarse, pues ya los militares ni en los rincones más obscuros de
los cafés podían aventurar una palabra de política. Iba muy de tarde en
tarde Baldomero Galán, y no aparecían ya por allí ni la Marquesa de San
Blas, ni Aniceto Navascués, ni Paco Bringas, estos dos últimos vendidos
al Gobierno y adulones de Nocedal.

Si en política no transigía Centurión por nada de este mundo con sus
enemigos, en otros órdenes de la vida era menos inflexible, y daba paz a
su fiereza. Amansado por la desgracia, volvió a tratarse con la
Coronela, viuda de Villaescusa, y recibía de ella alguno que otro
obsequio. Por Manolita sabía las buenas andanzas de Teresa en París, lo
alegre que estaba y el mucho dinero de que disponía. La madre y la hija
se escribían a menudo, y en ninguna de sus cartas dejaba Manolita de
recordar a Teresa el cuidado de allegar el consabido millón, que le
asegurara la existencia por el resto de sus días. Para hablar de esto,
tenía la Coronela que emplear una clave, escondiendo la idea del millón
debajo de la figura y nombre de un santo muy venerado. «No se aparte de
tu mente---leía Teresa,---ni de día ni de noche la devoción que debes a
nuestro santo tutelar el bendito San Millán. Que ese glorioso santo guíe
tus pasos, que sea contigo siempre, y que te acompañe cuando vuelvas al
lado de tu madre.»

Refería Manolita cuantas impresiones le comunicaba Teresa, los
monumentos que veía, las preciosidades sin número que Isaac le compraba,
y cuando se le iba concluyendo la realidad, metíase a inventar nuevos
prodigios. Una tarde, no teniendo cosa positiva que contar, relató un
sueño que tuvo la noche antes, el cual, si fuese verdad, había de traer
grande trastorno al mundo. Desgraciadamente, no era más que sueño, si
bien de los más lógicos y verosímiles. Pues Señor, Manolita había soñado
que su hija llamaba la atención en París\ldots{} Iba por la calle, y
todos se paraban para mirarla. Millonarios franceses y príncipes rusos
le enviaban ramos de flores y cartitas pidiéndole relaciones. Tanto de
ella se hablaba, que Napoleón quiso verla. De la ocasión y lugar en que
la vio, nada decía la señora: este punto interesante quedaba envuelto en
las neblinas del sueño\ldots{} Total: que al Emperador le entró la niña
por el ojo derecho. Locamente enamorado, iba de un lado para otro en las
Tullerías clamando por Teresa, y pidiendo que se la llevaran\ldots{}
Aquí terminaba el sueño, y era lástima. ¡Sabe Dios la cola que traería
el capricho imperial, y las complicaciones europeas que podían
sobrevenir si\ldots! En fin, no hay que reírse de los sueños, que a lo
mejor resultan profecías o barruntos vagos de la realidad.

Para Centurión, que no tenía derechos pasivos, era la realidad bien
triste, sin que la embelleciera ningún ensueño. La situación
reaccionaria, reforzada por el innegable talento de Nocedal, llevaba
trazas de perpetuarse. Había moderados para un rato. Y aun cuando la
Reina, con otra repentina veleidad, les pusiese en la calle, sería para
traernos a O'Donnell, con su caterva de señoretes tan bien apañados de
ropa como desnudos del cacumen. No había, pues, esperanzas de
colocación, los escasos ahorros se irían agotando, y la miseria que ya
rondaba, vendría con adusto rostro a prepararles una muerte tristísima.
Como si las propias desgracias no fueran bastantes, las ajenas llamaban
a la puerta de don Mariano con desgarrador acento. Leovigildo Rodríguez,
que en la desesperación de su miseria solía recurrir a las casas de
juego, arriesgando un par de pesetas para sacar un par de napoleones,
tuvo un percance en cierto garito de la Plaza Mayor, junto a la
Escalerilla. Por un tuyo y mío surgió pendencia soez, y arrastrado a
ella Leovigildo por su genio arrebatado, recibió un navajazo en el
costado derecho, que a poco más le deja en el sitio. La herida era
grave, pero no mortal. Lleváronle a una botica próxima; de allí, a su
casa; Mercedes se desmayó, y los chicos entonaron un coro angélico que
partía los corazones. Acudió Centurión al clamor de la vecindad, pues
Leovigildo vivía en la calle de Lavapiés muy cerca de la de San Carlos,
y viendo que en la casa se carecía de todo, y no había medios de hacer
frente a la gran calamidad que se entraba por las puertas, acudió a
Segismunda, hermana del herido. Esta fatua señora se limitó al
ofrecimiento de sufragar los gastos de médico y botica. No podía más,
según dijo, y harta estaba ya de socorrer a su hermano, que con su mala
cabeza y peor conducta llamaba sobre sí todos los infortunios. Tan
bárbaro despego puso al buen don Mariano en el compromiso de atender a
la manutención de toda la chiquillería y de la madre, mientras el herido
se restableciese, que ello sería muy largo. ¿Qué había de hacer el
hombre?

Y menos mal si las calamidades vinieran solas; que solas ¡ay! no venían,
sino trabadas entre sí con enredo de culebras que retuercen la cola de
una en la cabeza de otra. A la entrada de primavera tuvo doña Celia un
ataque de reúma que empezó con agudos dolores en la cintura, acabando en
una completa invalidez y postración de ambas piernas. Creyó Centurión
que el cielo se le desplomaba encima. Habría tomado para sí la
enfermedad de su esposa, si estos cambios pudieran efectuarse. Se
avecinaban días horrorosos, requerimientos de médicos, que uno y dos no
habían de bastar; dispendios de botica, y, sobre todo, el dolor de ver
en tan gran sufrimiento a la bonísima Celia. ¡Y este traspaso, estas
angustias, venían en tiempo de maldición, que maldición es la cesantía y
azote de pueblos!\ldots{} Antes castigaba Dios a la Humanidad con el
Diluvio; a Sodoma y Gomorra con el fuego: ahora, descargando sobre los
países corruptos una nube de \emph{moderados}, en vez de castigar a los
malos, les da de comer, y a los buenos les mata de hambre. «¿Quién
entiende esto, Señor; qué cojondrios de justicia es la que mandan los
cielos sobre la tierra?»

Ya sabía Dios lo que hacía. Proponiéndose tal vez dar a la Humanidad
otro Job que fuera lección y ejemplo de paciencia ante la rigurosa
adversidad, dispuso que cayeran sobre el poco sufrido don Mariano nuevas
y más atroces desventuras, que se referirán a su debido tiempo. Sépase
ahora que las demasías del Gobierno Narváez-Nocedal tenían
constantemente al infeliz cesante en un grado de exaltación que le
amargaba la existencia. Cuando se hicieron públicos los graves sucesos
del Arahal, una revolución más agraria que política, no bien conocida ni
estudiada en aquel tiempo, no podía el buen hombre contener su ira, y en
medio de la calle con descompuestos gritos expresaba su protesta contra
la bárbara represión de aquel movimiento. Cuadrado, que con él venía
calle abajo por la de Lavapiés, le recomendó que adelgazara la voz y
reprimiera su justa cólera, pues no estaban los tiempos para vociferar
en público sobre tan delicadas materias. Pero él no hacía caso: a
borbotones le salían los apóstrofes, y la justicia y la verdad que
proclamaba no se avenían a quedarse de labios adentro. En la puerta de
la tienda de un sillero, conocido en todo el barrio por sus fogosas
ideas, puso cátedra Centurión, y ante el auditorio que pronto se le
formó, el sillero y su mujer, el zapatero de un portal próximo, dos
transeúntes que se agregaron y cuatro chiquillos de la calle, rompió con
estas furibundas declamaciones:

«¿Qué pedían los valientes revolucionarios del Arahal? ¿Pedían Libertad?
No. ¿Pedían la Constitución del 12 o del 37? No.~¿Pedían acaso la
Desamortización? No.~Pedían pan\ldots{} pan\ldots{} quizás en forma y
condimento de gazpacho\ldots{} Y este pan lo pedían llamando al pan
Democracia, y a su hambre Reacción\ldots{} quiere decirse que para matar
el hambre, o sea la Reacción, necesitaban Democracia, o llámese pan para
mayor claridad\ldots{} No creáis que aquella revolución era política, ni
que reclamaba un cambio de Gobierno\ldots{} era el movimiento y la voz
de la primera necesidad humana, el comer. Bueno: ¿pues qué hace el
Gobierno con estos pobres hambrientos? ¿Mandarles algunos carros
cargados de hogazas? No. ¿Mandarles harina para que amasen el pan?
No.~¿Mandarles cuartos para que compren harina? No.~Les manda dos
batallones con las cartucheras surtidas de pólvora y balas. La tropa,
bien comida, pone cerco al pueblo, embiste, penetra en las calles y
acosa con tiros a la multitud revolucionaria para que se entregue. ¿Por
ventura los soldados apuntan a la cabeza? No.~¿Apuntan al corazón?
No.~Apuntan a los estómagos, que son las entrañas culpables. El corazón
y el cerebro no son culpables\ldots{} No van los tiros a matar las
ideas, que no existen; no van a matar los sentimientos, que tampoco
existen: van a matar el hambre\ldots{} Dominada la insurrección y
cogidos prisioneros sin fin, el jefe de la fuerza escoge para
escarmiento los que han sido más levantiscos\ldots{} En las caras se les
conoce su perversidad: fíjanse en los más pálidos, en los más
demacrados. Aquellos, aquellos son los que gritaron Democracia, que fue
un disimulo del grito de Pan\ldots{} Pues escogidos cien democráticos, o
dígase cien estómagos vacíos, los llevaron contra unas tapias que hay a
la salida del pueblo, y allí les sirvieron la comida, quiero decir, que
los fusilaron\ldots{} Y ya se les cerró el apetito, que abierto tenían
de par en par. No hay cosa que más pronto quite la gana de comer que
cuatro tiros con buena puntería\ldots{} Esos cien hambrientos pronto
quedaron hartos\ldots{} Ya lo veis, señores: cien hombres fusilados por
el delito de no haber almorzado, ni comido, ni cenado en muchos días. ¡A
esto llaman Narváez y Nocedal gobernar a España! España pide sopas:
¡tiros! España pide Justicia: ¡tiros! Yo pregunto: ¿tiene hambre
Narváez? No tiene hambre, sino sed de sangre española. Pues démosle
nuestra sangre; que acabe de una vez con todos los buenos liberales.
¿Preferís vivir sin comer a morir de un tiro? No\ldots{} ¿De qué os
duele el estómago, de empacho de Libertad, o de vacío de alimentos? De
vacío de alimentos. ¿Creéis que con ese horrible vacío se puede vivir?
No.~Pues pedid al Gobierno que os mate, que bien sabe hacerlo\ldots{} Y
para abreviar, digo yo: ¿no sería más sencillo que al decretar las
cesantías en un cambio de Gobierno nos reunieran en un patio o en la
Plaza de Toros a todos los cesantes con sus familias respectivas, y
poniéndonos en fila delante de un pelotón de soldados, nos vendaran los
ojos y nos mandaran rezar el \emph{Credo}\ldots? El jefe de la fuerza
daría las voces de ordenanza: \emph{«¡Preparen!\ldots{}
¡apunten!\ldots{} ¡cesen!\ldots»} y pataplum\ldots{} cesábamos\ldots{}
Todas las penas se acababan de una vez\ldots{} Con Dios, señores, y a
casa, que huele a pólvora\ldots{} y sopla un aire tempestuoso cargado de
Nocedales\ldots{} Con Dios.»

\hypertarget{xxi}{%
\chapter{XXI}\label{xxi}}

Aunque debía su puesto a \emph{los hombres de Julio}, el gran
\emph{Sebo} era una excepción venturosa en nuestra política, y no estaba
cesante bajo la dominación moderada. Decía de él Centurión que era una
de esas lapas que no se desprenden de la roca sino hechas pedacitos. El
caso fue que en la crisis de Octubre del 56, la subida de Narváez hirió
a Telesforo en lo más sensible de su dignidad. ¿Con qué cara continuaría
en su empleo, él, que bien podía contarse, y a mucha honra, entre los
\emph{hombres de Vicálvaro}? ¿Presentaría la dimisión antes que un
ignominioso puntapié le lanzara a la calle? En tales dudas estaba,
cuando su protector, el Marqués de Beramendi, confortó su turbado
espíritu con estas razones: «Usted no dimite, ni le dimiten, porque es
un funcionario irreemplazable en el organismo de la Administración. Y
para que el amigo Nocedal así lo comprenda, y detenga la mano aleve que
a estas horas emborrona las cesantías, voy a prevenirle al instante,
diciéndole quién es \emph{Sebo} y lo que significa y vale.» Así lo hizo
Fajardo, y no fue preciso más para que las \emph{narices de perro
pachón} se salvaran del desmoche, y ejercieran su olfato en servicio del
nuevo Ministro.

Un año después de esto, en Octubre del 57, tuvo que ver Beramendi a
Nocedal para un asunto que vivamente le interesaba; mas antes de ir a
Gobernación, habló con Telesforo, habilísimo en descubrir hechos
ignorados y en encontrar la relación de ellos con otros conocidos. De él
sacó Beramendi cuantos datos podían servirle, y se fue derecho a
Nocedal, cogiéndole en su despacho a la hora en que le creyó menos
agobiado de visitantes políticos y de pretendientes jaquecosos.

Apreciaba realmente Fajardo al Ministro de la Gobernación, aunque las
ideas de uno y otro rabiaban de verse juntas; le tenía en gran estima
por su talento, por su cultura y amenidad, y hasta por el gallardo
cinismo con que había pasado de la exaltación progresista a los furores
ultramontanos. No veía en esto defección ni apostasía, creyendo que
ningún hombre está obligado en edad madura a respetar su propia
juventud. La juventud es aprendizaje, ensayo de medios de vida, tanteo y
calicata de terrenos. Cada cual sabe a dónde va, y por dónde va más
seguro, según sus aspiraciones y fines. El pensar, al vivir debe
subordinarse. Nocedal comprendió que por el Progresismo, terreno a media
formación y surcado de zanjas peligrosas, no se iba a ninguna parte. Los
caminos de la reacción podían llevarle más pronto a resolver los
capitales problemas de la existencia. La Libertad era, en verdad, cosa
espléndida y sugestiva; pero aventurarse por sus senderos tortuosos y de
extremada longitud, era locura no teniendo doscientos o trescientos años
por delante. La vida es corta. ¿A qué malograrla en lo inseguro, en lo
discutido, en lo variable? ¿No es más práctico apoyarla en lo
indiscutible y eterno, en la base sólida de las cosas dogmáticas?
Beramendi se ponía en su caso, y hallaba muy natural que hubiese tomado
postura política al arrimo de la Iglesia. Era un gran talento que
gustaba de comodines. Fácil es la política en que todo se arregla
echando a Dios por delante: no es preciso argumentar mucho para esto,
porque en el ultramontanismo todo está pensado ya. ¡Qué cómodo es tener
la fuerza lógica hecha y acopiada para cuantos problemas de gobierno
puedan ocurrir!

Entró Beramendi en el despacho del Ministro; este se fue a su encuentro
con rostro alegre, y al estrecharle ambas manos tiró de él para llevarle
junto a un balcón donde podían hablar con más reserva. Contra las
presunciones de Fajardo, había gente, aunque no mucha ni la más enfadosa
del ganado político. «Ya sé a qué viene usted---dijo el Ministro.---Y
usted sabe también que este cura, Cándido Nocedal, ha hecho en el asunto
cuanto humanamente podía\ldots{}

---No, amigo, no: usted puede y debe hacer mucho más. Déjeme recordarle
el caso y agregar algunos antecedentes que usted ignora.

---Me parece que no ignoro nada. La hija de Socobio y su amante vinieron
a Madrid el mes pasado\ldots{} creo que de un pueblo próximo a Villalba.
Traían un niño enfermo, el único hijo que han tenido, creo yo.

---El único. El niño tenía poco más de dos años. Por quien le ha visto
sé que era una criatura ideal\ldots{} Enfermó en el pueblo, y no
sabiendo sus padres cómo curarle, le trajeron a Madrid. Se alojaron en
la calle de la Ventosa, miserablemente; buscaron médico\ldots{} Ni el
médico pudo hacer nada, ni Dios quiso salvar al niño. Imagínese usted,
mi querido Nocedal, la tribulación de aquellos infelices, privados de
todo recurso\ldots{} Y en esta situación, la infame policía les rondaba.

---Y qué quiere usted, amigo mío. La policía tiene que cumplir con su
deber. No deja de ser lo que es porque los criminales se encuentren en
una situación patética, digna de piedad, de misericordia\ldots{}

---Déjeme seguir. Muerto el pequeñín, había que enterrarle. Leoncio se
procuró un ataúd blanco. Entre los dos amortajaron al pobre
ángel\ldots{} Sé todo esto por quien lo vio\ldots{} le vistieron con sus
trapitos remendados, le pusieron flores y ramitos de albahaca\ldots{}
Leoncio cogió la caja para llevarla al cementerio\ldots{} salió, tomó su
camino por el Paseo Imperial. Figúrese usted si iría desolado el hombre.

---Sí\ldots{} desoladísimo, y la situación algo novelesca\ldots{} Ya sé
lo que usted me va a decir ahora\ldots{} Que los policías escogieron
aquel momento de emoción tan grande y bella para echar el guante a
Leoncio\ldots{} Sí, sí: es tremendo; pero qué quiere usted, la ley es la
ley. Observe, querido Pepe, que los policías no fueron insensibles a la
tribulación de un padre que va camino del cementerio con su hijo debajo
del brazo: respetaron aquel dolor inmenso\ldots{}

---Pero lo seguían\ldots{} Esperaban a que el niño quedara en la tierra,
para caer sobre el padre\ldots{}

---Y eso prueba que no son los agentes de seguridad tan inhumanos como
se cree\ldots{} Luego que Leoncio cumplió sus últimos deberes de padre,
salió del cementerio\ldots{}

---Y no había dado veinte pasos, cuando se abalanzaron a él como perros
de presa\ldots{}

---Cumplían las órdenes que se les dieron. El otro sacó una pistola de
esas que llaman \emph{giratorias}, y empezó a tiros con los agentes: a
uno le metió una bala en la clavícula; al otro le habría dejado en el
sitio si con tiempo no se hubiera puesto en salvo\ldots{} Él mismo ha
referido que corría más que el viento.

---¡Lástima que Leoncio no hubiera matado a esos canallas! En fin, el
valiente chico escapó de milagro\ldots{} Locos andan los guindillas
buscándole.

---Y le encontrarán, créalo usted.

---Antes de que le encuentren, querido Nocedal, yo vengo a pedirle a
usted que dé órdenes a don José de Zaragoza o al inspector Briones para
que dejen en paz a ese hombre infeliz\ldots{} Leoncio no es más criminal
que usted ni que yo, ni que otros mil, burladores de matrimonios y de
toda ley religiosa y social.

---Por Dios, mi querido Beramendi, nosotros seremos eso y algo
más\ldots{} allá usted con la responsabilidad de lo que dice; pero ni a
usted ni a mí, gracias a Dios, se nos ha formado causa por adulterio y
rapto, con agravante de abuso de confianza\ldots{} ¿Qué quiere usted que
haga yo, yo, que habré sido el pecado, paso por ello, pero que ahora soy
la ley?\ldots{} Es uno pecado y es uno ley cuando menos lo piensa. Yo
haría fácilmente, en este caso, lo que el amigo me pide: coger la ley y
meterla donde nadie la viese\ldots{} ¿Pero no sabe el amigo que tengo
sobre mí la mosca de don Serafín del Socobio, que no me deja vivir, que
viene a mí con sus pretensiones, asistido del Arzobispo, del Nuncio, del
Presidente del Consejo, de la Reina y del Verbo Divino, para que yo coja
y encierre y haga picadillo al lobo que se llevó la oveja del
\emph{Joven Anacarsis}? ¿Si el juez me pide que le busque y le capture y
le traiga atado codo con codo, qué he de hacer yo?

---Pues nada: mandar a paseo al juez, y a don Serafín, y a todas las
personas altas que apoyan esa barbarie\ldots{} Yo pregunto: ¿Leoncio
Ansúrez se llevó a Virginia contra la voluntad de esta?\ldots{} ¿Por
ventura empleó engaño para llevársela, o recursos de magnetismo, o algún
brebaje maléfico?\ldots{} ¿Cree usted que en la situación presente de
Virginia y Leoncio, es legal y moral separarles? Ya sabe usted, Nocedal
amigo, que entre sacristanes, la efigie milagrosa pierde mucho de su
veneración. La moral labrada toscamente y vestida de colorines, ante la
cual el vulgo se arrodilla y reza, a nosotros poco o nada nos dice.
Quitémonos la máscara, Nocedal, y hablemos claro. Ponga usted la mano
sobre su conciencia, y dígame si cree que ese hombre, el hombre del niño
muerto y de la pistola giratoria, debe ser perseguido como un criminal.

---¿Quién lo duda, Marqués? ¡A dónde iríamos a parar si aplicáramos al
pueblo la moral que usted llama de los sacristanes!»

Dijo esto con su habitual gracejo, mirando al amigo y turbándole un
tanto con la fina sonrisa que solía poner en su rostro volteriano. Muy
serio contestó Beramendi: «Iríamos a parar a donde estamos: a la
relajación de toda ley, al libre ambiente de una sociedad en la cual
todos somos unos grandes bribones que nos pasamos la vida perdonándonos
nuestras picardías y barrabasadas. Si no tuviera esta sociedad el perdón
y la indulgencia, no tendría ninguna virtud. Toda la moral que viene de
arriba, en cuanto toca al suelo, queda reducida a un Prontuario de
reglas prácticas para uso de las personas pudientes\ldots{} Elevémonos
un poco sobre estos absurdos; levantemos nuestros corazones, que usted
puede hacerlo como nadie: su gran talento le ayudará. Tras de usted voy
yo, y con usted subo\ldots{} Seamos un poquito indulgentes con ese
humilde ladrón de mujer casada, ya que con ladrones mejor vestidos hemos
derrochado tanta indulgencia\ldots{} ¿No lo cree usted así?

---¿Yo qué he de creer?---replicó Nocedal echándolo todo a
risa.---Ingenioso es lo que usted me dice, y yo le oigo con mucho
gusto\ldots{}

---Pero oyéndome con mucho gusto, en cuanto yo vuelva la espalda tomará
usted sus medidas para cometer la gran iniquidad. No me mire con esos
ojos, que no sé si son asombrados o burlones\ldots{} La intención del
Ministro bien comprendida está\ldots{} Han hecho ustedes una \emph{Ley
de vagos}\ldots{}

---Sí, señor. Ley de higiene social, de policía política\ldots{}

---Está bien. Esa Ley, que ya es inicua por facilitar la persecución y
destierro de la gente política de oposición, lo es mucho más porque con
ella se desembarazan los amigos del Gobierno de toda persona que les
estorba. ¿Que don Fulano o don Mengano, personaje o fantasmón
influyente; que la Zutanita o la Perenzejita, damas, o menos que damas,
querindangas tal vez de cualquier cacicón, tienen algún enemigo a quien
desean apabullar con razón o sin ella? Pues aquí está la Ley de vagos
para socorrer a los bien aventurados que tienen hambre y sed de
venganza.

---¡Eh\ldots{} poco a poco, Marqués!---dijo don Cándido con gravedad
sincera.---Eso podrán hacerlo otros\ldots{} no lo sé. Lo que aseguro es
que yo no lo hago.

---Pero como en el caso de Leoncio Ansúrez hay causa criminal pendiente,
el señor Ministro lo hará, y se quedará tan fresco, y ni aun se lavará
las manos con que ha dado el golpe. ¡Qué manera tan sencilla y fácil de
dar satisfacción a esos malditos Socobios! Coge la policía al desdichado
Ansúrez, y por el doble delito de robar a Virginia y del desacato
reciente a la autoridad, me le mandan a Leganés atado codo con codo. De
allí, sin dejarle respirar, sin que nadie se entere, ni puedan
socorrerle los que le aman, saldrá para Filipinas o para Fernando Poo en
la primera cuerda\ldots{} ¡Qué bonita, qué rápida sentencia! ¡Y la pobre
mujer, que por fas o por nefas tiene puesto en él todo su cariño,
esperándole hoy, esperándole mañana, esperándole quizá toda la vida!

---Es triste\ldots{} sí\ldots{} Ya ven que el amor libre tiene sus
quiebras\ldots{}

---El amor atado las tiene mayores\ldots{} Y ya que hemos nombrado a
Virginia, sabrá usted que la he recogido, la he puesto en lugar
seguro\ldots{} no me pregunte usted dónde\ldots{} y me la llevaré a mi
casa, donde Ignacia y yo la tendremos y miraremos como hermana, si
nuestro buen amigo persiste en aplicar a Leoncio la Ley de vagos.

---Verdaderamente---dijo el Ministro fingiéndose sorprendido para
disimular su inclinación a la benevolencia,---no sé, no entiendo, mi
querido Marqués, los móviles de ese interés de usted por un quídam, por
un zascandil\ldots{}

---Los móviles de este grande interés---replicó Beramendi con acento
grave,---no son otros que un ardiente amor a la justicia. La justicia
esencial me mueve\ldots{} Y esto que digo, bien lo comprende usted. En
el fondo de su espíritu, usted piensa y siente como yo\ldots{} Pero
desde el fondo del espíritu de Nocedal a la exterioridad del hombre
público, del ultramontano por conveniencia, del Ministro de la
Gobernación, hay distancia tan grande, que los sentimientos no tienen
tiempo de llegar a los ojos, a los labios\ldots{} ¿Qué?

---No he dicho nada. Siga usted.

---Sólo me queda por decir que si el amigo no me hace caso, si no
satisface este anhelo mío de justicia, perderemos las amistades.

---¿Así como suena?\ldots{} ¿Perder las amistades?\ldots{} Y amistades
que no son políticas, sino de puro afecto y simpatía.

---Afecto y simpatía se desvanecerán. Además de eso, yo perderé una
ilusión: el convencimiento de que Nocedal no es tan fiero como le
pintan.»

Tanto y con tanto ardor insistió Fajardo en su pretensión humanitaria,
que el otro, si no se dio a partido resueltamente, bien claro mostraba
en su rostro la flexibilidad inherente a todo político español; blandura
de voluntad que si en el común de los casos que afectan al interés
público es defecto grande, en algún particular caso, como el que ahora
se cuenta, era hermosa virtud. Un poco más de matraca del bravo
Beramendi, y ya podría Leoncio reírse de la trampa que le tenían
armada\ldots{} No era, en efecto, el Ministro de la Gobernación tan
fiero como se le pintaba. Su destemplado ultramontanismo, manifiesto en
la vaguedad de los principios y en la retumbancia de los discursos,
apenas tenía eficaz acción en la vida práctica, y si en la general
esfera política funcionaba con estridente ruido el potro de tormento, en
la esfera privada y en los casos particulares, todos los garfios y
ruedas de la tal máquina se volvían completamente inofensivos. Era
Nocedal un hombre culto, de trato amenísimo, que había tomado la postura
ultramontana porque con ella descollaba más fácilmente entre sus
contemporáneos. Si los caracteres son producto y resultancia de
elementos éticos que, difusos y sin conformidad entre sí, se ramifican
en el fondo social, el complejo ser de don Cándido había tomado su
fundamental savia de yacimientos morales muy desperdigados y diferentes.
Sensible como pocos al amor, la ternura de su corazón ante el sexo débil
le inspiraba la piedad en la vida política. Por eso, si no presenta su
conducta privada el modelo perfecto del hombre, tampoco hay en su
gestión pública actos de crueldad; si por la doctrina ultra-reaccionaria
que profesó fue odioso a muchos que no le conocían, su trato encantador
y afabilísimo le hizo simpático a cuantos le trataban. Juzgándole por el
aspecto declamatorio y vano que lleva en sí todo papel político, aparece
como un discípulo de Torquemada, o como Gregorio VII redivivo; pero si
le hacemos bajar a las llanezas de la Administración, vemos en él un
excelente gobernante, que supo llevar el orden, la actividad y la
rectitud al departamento que regía.

Seguro ya de haber conquistado el corazón del Ministro, despidiose
Beramendi con extremos de afecto y gratitud\ldots{} Algún recelo le
asaltó al partir; ya próximo a la puerta, retrocedió, diciendo a su
amigo: «No me voy tranquilo, Nocedal\ldots{} y es que\ldots{} me temo
que usted, con toda su buena voluntad, no pueda ocuparse de este
asunto\ldots{} por falta de tiempo\ldots{} Déjeme que le
explique\ldots{} La gran tensión de espíritu que he puesto en salvar a
Leoncio, me quitó de la memoria\ldots{} algo que quería decir a
usted\ldots{} Es una noticia de sensación. Allá va: están ustedes
caídos.»

Riendo, contestó Nocedal algo que expresaba dubitación no exenta de
intranquilidad.

«Lo sé por el conducto más auténtico. La Regia prerrogativa, que hemos
convenido en comparar a una veleta, ha dado una vuelta en redondo.

---Cuentos, amigo, chismajos de la Puerta del Sol. Su Majestad está en
meses mayores y no se ocupa de política.

---Su Majestad está fuera de cuenta, y ha decidido que la noticia de su
alumbramiento no la dé al país el Ministerio Narváez-Nocedal. Veo que
usted no lo cree\ldots{} tal vez lo duda. Pues in \emph{dubiis
libertas}. La libertad de ese Leoncio me arreglará usted sin tardanza.
Hoy mismo, por lo que pueda tronar\ldots{}

---Arreglado quedará hoy.

---Hágalo usted por mí, por la Justicia\ldots{} y por el feliz
alumbramiento de doña Isabel II.»

\hypertarget{xxii}{%
\chapter{XXII}\label{xxii}}

En la Puerta del Sol se encontraron Beramendi y el \emph{Joven
Anacarsis}, ¡oh fatalidad cómica de los encuentros personales en el
laberinto de las poblaciones!, y después de los saludos, cambiáronse las
preguntas que infaliblemente se hacían siempre que la casualidad les
juntaba. Ernesto preguntó por Aransis, y Beramendi por Teresa
Villaescusa. Ved aquí las respuestas: continuaba en Atenas el Marqués de
Loarre; pero fatigado ya de la vida helénica y algo resentida su salud,
había pedido licencia para venir a Madrid y gestionar su traslado a
Bruselas o Stockolmo. Teresa volvía de París, después de ausencia larga
y de no pocas peripecias, según le habían contado a Ernestito sus amigos
del Crédito Franco-Español. Ya no \emph{hablaba} con Brizard; los
motivos del acabamiento de relaciones, \emph{Anacarsis} los ignoraba.
Sólo sabía que la hermosa mujer había cogido en sus redes a un Marqués o
Conde andaluz, tan cargado de años como de dinero, según decían, y no
libre de los achaques que anublan el ocaso de una vida de continuos
goces\ldots{} De algo más habló Ernesto; pero en la memoria de Beramendi
no quedó rastro de ello, y con indiferencia le vio partir y desvanecerse
en aquella muchedumbre de la Puerta del Sol, compuesta de desocupados
expectantes y de transeúntes sin prisa.

El mismo día en que Isabel II dio a luz con toda felicidad un Príncipe
que había de llamarse Alfonso, llegó a Madrid Teresa Villaescusa.
Recibíala su patria con tumulto de alegría y esperanzas, y con
preparativo de festejos: hasta en esto había de tener Teresa buena
sombra. En su paso desde la frontera a Madrid, las impresiones que
recibió fueron asimismo muy gratas, según contó meses adelante a sus
amigos de esta Corte. Ello fue que, viniendo de un país tan bello como
Francia y de ciudad tan opulenta y fastuosa como París, al embocar a
España por Behovia no sintió la tristeza que deprime el ánimo de la
mayoría de los viajeros cuando pasan de la civilización a la incultura,
y del vivir amplio a la estrechez mísera; sintió más bien alborozo y
verdadero amor de familia. Atravesando en la diligencia las estepas de
Castilla, no se cansaba Teresa de contemplar las tierras pardas, sin
vegetación, a trechos labradas para la próxima siembra; entreteníase
mirando y distinguiendo los tonos diferentes de aquella tierra
esquilmada, madre generosa que viene dando de comer a la raza desde los
tiempos más remotos, sin que un eficaz cultivo reconstituya su savia o
su sangre. Miraba los pueblos pardos como el suelo, las mezquinas casas
formando corrillo en torno a un petulante campanario\ldots{} Ni
amenidad, ni frescura, ni risueños prados veía, y, no obstante, todo le
interesaba por ser suyo, y en todo ponía su cariño, como si hubiera
nacido en aquellas casuchas tristes y jugado de niña en los ejidos
polvorosos. Las mujeres vestidas con justillo, y con verdes o negros
refajos, atraían su atención. Sentía piedad de verlas desmedradas,
consumidas prematuramente por las inclemencias de la naturaleza en suelo
tan duro y trabajoso. Las que aún eran jóvenes tenían rugosa la piel.
Bajo las huecas sayas asomaban negras piernas enflaquecidas. Los
hombres, avellanados, zancudos, con su seriedad de hidalgos venidos a
menos, parecían llorar grandezas perdidas. Todo lo vio y admiró Teresa,
ardiendo en piedad de aquella desdichada gente que tan mal vivía,
esclava del terruño, y juguete de la desdeñosa autoridad de los
poderosos de las ciudades. Por todo el camino, al través de las llanadas
melancólicas, de las sierras calvas, de los montes graníticos, iba
empapando su mente en esta compasión de la España pobre, a solas, muy a
solas, pues la persona que la acompañaba esparcía sus pensamientos por
otras esferas.

En Madrid permaneció Teresa algunos días en completa obscuridad.
Advirtieron los amigos y parientes de la familia que la Coronela no
echaba las campanas a vuelo por la llegada de su hija, sin duda porque
esta no había rezado bastante al bendito San Millán para que le
concediera el millón, objeto de las ansias maternales. Según
indicaciones de Manolita, el rompimiento con Brizard no había sido por
culpa de Teresa, cuyo comportamiento con el caballero francés fue
siempre correctísimo. Los padres de Isaac le prepararon matrimonio con
una opulenta señorita alsaciana, que debía de ser hebrea por el
sonsonete del nombre, algo así como \emph{Raquel} o
\emph{Rebeca}\ldots{} Lo que le supo peor a Manolita fue que Brizard, al
despedirse de Teresa, no le dio más que la porquería de diez mil
francos. ¡Quién lo había de creer de un hombre tan rico, tan rico, que
sólo en un punto que llaman Mulhouse tenía tres fábricas de hilados, y
en otro punto que llaman Charleroi, allá por los Países Bajos, poseía
minas de carbón muy grandes, muy ricas! En fin, no había más remedio que
tener paciencia. Daba a entender asimismo la Coronela que no era muy de
su devoción aquel \emph{embalsamado} con quien Teresa volvía de París,
un señor flaco, atildado y mortecino, que parecía un Cristo retirado de
los altares. Limpio era y de maneras finísimas el Marqués de Itálica,
que así le llamaban; pero algo tacaño, y además hurón: venía con el
propósito de llevarse a Teresa a un pueblo de Sevilla donde tenía gran
casa y hacienda mucha. ¡Vaya, que meter a la niña en un villorrio y
esconderla como cosa mala!\ldots{} Nada pudo contra esto Manolita, y vio
pasar a su hija por Madrid como una sombra triste, después de socorrer a
Centurión con algún dinero y a doña Celia con cuatro hermosas macetas de
flores.

Hallábase Beramendi en aquellos días muy debilitado de memoria y con los
ánimos caídos. Pasaban hechos y personas por delante de su vista sin
dejar imagen ni apenas recuerdo, y la vida externa le interesaba poco,
como no fuera en la esfera familiar y de las íntimas afecciones. Una vez
que aseguró la libertad y sosiego de \emph{Mita} y \emph{Ley}, y les vio
partir para el pueblo donde tenían su habitual residencia y modo de
vivir, quedó tranquilo y no se ocupó más que de sus propios asuntos.
Paseando solo una mañana por la calle de Alcalá, vio a Eufrasia que
salía de San José con Valeria. Ambas venían de trapillo eclesiástico,
vestiditas modestamente, y con rosario y libro. Ya sabía Beramendi que
\emph{la moruna} andaba en la meritoria empresa de corregir a la
Navascués de sus locos devaneos, aplicándole la medicina infalible:
frecuentar los actos religiosos. Consigo a diferentes iglesias la
llevaba, eligiendo aquellas formas de culto que más pudieran cautivar
por su solemnidad a la descarriada joven. Y no estaba Eufrasia
descontenta. Valeria, mujer de indecisa y floja personalidad, se dejaba
modelar fácilmente por toda mano que la cogía. Saludó a las dos damas el
buen Fajardo, que después del cambio de cortesanías, oyó de labios de la
Marquesa estas palabras afectuosas: «¡Ay, Pepe, qué caro se vende
usted!\ldots{} ¿Nosotras? Ya lo ve\ldots{} venimos de la iglesia,
venimos de comulgar\ldots{} Aprenda usted, hereje, mal cristiano\ldots{}
Adiós, adiós, y vaya usted alguna vez por casa, que allí no nos comemos
la gente.»

Siguió cada cual su camino. Beramendi las vio pasar como sombras, y no
pensó más en ellas. Así había visto pasar y caer el Ministerio
Narváez-Nocedal, cuya política arbitraria y dura llegó a inspirar miedo
en Palacio, y así vio venir el Gabinete Armero-Mon-Bermúdez de Castro,
que no era más que una cataplasma simple aplicada al tumor nacional; vio
después desvanecerse y morir con su último día el año 57, y aparecer con
risueño semblante el 58; y vio cómo trajo también este año nuevo su
correspondiente Ministerio anodino, que se llamó Istúriz-Sánchez Ocaña,
y tan sólo se hizo memorable porque, dentro de él, unos tiraban a
liberales templados, otros al absolutismo rabioso\ldots{} En la mente de
Fajardo se fijó la idea de que el alma de la Nación, como la de él,
sufría un acceso de pesada somnolencia. Todo dormía en la sociedad y en
la política; todo era gris, desvaído; todo insonoro y quieto como la
superficie de las aguas estancadas. Pasaban meses, y las querellas entre
las distintas fracciones moderadas, la \emph{liga blanca}, la \emph{liga
negra}, no sacaban a la política de su sombría catalepsia\ldots{} Por
fin, un hombre agudísimo y de cuidado, don José Posada Herrera, astur,
largo de cuerpo y de entendederas, puso fin a todo aquel marasmo y
atonía de las voluntades.

Antes de ver cómo se movieron las dormidas aguas, sépase que una mañana
de fines de Mayo fue sorprendido Beramendi por la súbita presencia de
Guillermo de Aransis, que apenas llegó de Marsella corrió a los brazos
de su entrañable amigo. Doce días había tardado del Pireo a Madrid,
rapidísimo viaje en aquellos tiempos de lentitud en todas las cosas.
Encontrole Fajardo envejecido, canosa la barba, ralo el pelo, y los ojos
privados de aquel alegre resplandor que tuvieron en España. «¿Qué tal
las griegas? ¿Te han tratado bien las griegas?» le dijo. Sonrió el de
Loarre; y como el otro pidiera con insistencia informes del bello sexo
en aquel clásico país, hizo Guillermo un resumen étnico y social de todo
el mujerío ateniense, lacedemonio, beocio y tesálico\ldots{} Luego, en
el almuerzo, a instancias de Ignacia y de don Feliciano, dio noticias
interesantes de Atenas, de la Acrópolis, del Partenón, de los montes
Pindo, Himeto, y hasta del mismísimo Parnaso. Con todas sus hermosuras,
más reales en el conocimiento humano que en la propia Naturaleza, Loarre
quería dar un solemne adiós a la patria de Homero solicitando la
representación de España en un país del norte de Europa. «Pide por esa
boca, hijo mío, y no te quedes corto---dijo Beramendi,---que prontito
vamos a tener en candelero a nuestro grande amigo \emph{don Leopoldo el
Largo}, y a él nos vamos como fieras cuando gustes\ldots{} ¿Quieres
mañana, quieres hoy mismo?»

Respondió Aransis que no había tanta prisa, y que si estaba en puerta
O'Donnell, debían esperar a la efectiva entrada. «¡Ay, chico, cómo se
conoce que vienes de Grecia, de un país alelado, de un país dormido
sobre ruinas! Hay que tomar vez, hijo mío. No permitamos que el aluvión
de pretendientes nos coja la delantera. Seamos nosotros aluvión de
madrugadores. Iremos mañana. ¿No sabes lo que pasa? En el Ministerio de
este pobrecito Istúriz han puesto una bomba, que se llama don José
Posada Herrera, la cual estallará el día menos pensado, y vas a ver
volar por los aires los restos despedazados del Moderantismo. Y hay más,
querido Guillermo. Me consta, por revelación directa y verbal de un
amigo mío que tiene alas para entrar en Palacio, y entra por los
balcones, por las chimeneas, por las rendijas\ldots{} vamos, por donde
quiere; me consta, digo, que la pobrecita Isabel está desde hace un año
muy pesarosa de haber despedido a O'Donnell\ldots{} Fue un verdadero
tropezón y torcedura de pie en aquel baile famoso\ldots{} Su Majestad no
tiene consuelo, y elevando sus Reales ojos a las bóvedas pintadas por
Tiépolo, dice que no hay hombre más insufrible que Narváez; que se vio
precisada a darle el canuto antes de tiempo, porque con sus malas pulgas
y sus intemperancias sacaba de quicio a toda la Nación; que ha traído
estos Gabinetes de cerato simple para calmar los ánimos, apurar las
Cortes y ganar días, hasta que lleguen los de O'Donnell, que serán
largos y felices\ldots{} Esto y algo más que aquí no puedo decir, tengo
yo que contarle al Conde de Lucena\ldots{} A poco que él apriete, España
es suya y para mucho tiempo. ¡Arriba la Unión!\ldots{} Dime tú: ¿has
leído el discurso que en el Senado pronunció don Leopoldo en Mayo del
año último?\ldots{} \emph{No, padre}. Pues a tu Legación había de llegar
la \emph{Gaceta}. Pero tú, entretenido con las grietas, no ponías la
menor atención en las cosas de tu patria. En aquel discurso memorable,
sin fililíes oratorios, salpicado de frases pedestres y de alguno que
otro solecismo, se nos revela O'Donnell como el primer revolucionario y
el primer conservador. Él transformará la familia social; él ennoblecerá
la política para que esta, a su vez, ayude al engrandecimiento de la
sociedad\ldots{} ¿No me entiendes? Pues ya te lo explicaré mejor.
¡Arriba la Unión, arriba O'Donnell!»

Fueron a visitar al grande hombre, a quien hallaron frío y reservado en
la conversación política, afabilísimo y jovial en todo lo que era de
pensamiento libre. Algo de las referencias de intimidad palatina que
Beramendi le llevó, ya era de él conocido: algo había que ignoraba o que
afectaba ignorar, añadiendo que le tenía sin cuidado. Dejaba traslucir
la persuasión de que el poder iría pronto a sus manos; pero esperaba sin
impaciencia la madurez del hecho. En su íntimo pensar, se decía
Beramendi que esta actitud de flemática pasividad no carecía de
afectación, finamente disimulada. Era un recurso más de arte político,
casi nuevo entre nosotros. Variando graciosamente la conversación,
O'Donnell pidió a Guillermo noticias de la política griega, de cómo eran
allá las Cámaras, el parlamentarismo, de la forma en que se hacían las
elecciones y se mudaban los Gobiernos. Aransis le explicó la política
helénica con extremada precisión narrativa, y con detalles pintorescos y
ejemplos anecdóticos que daban la impresión justa de la realidad. El
General y todos los presentes alabaron la pintura, y doña Manuela
sintetizó su juicio con esta seca frase: «Lo mismo que aquí.

---Lo mismo, no---dijo don Leopoldo.---Peor, mucho peor. Nos imitan, y
los imitadores valen siempre menos que sus modelos.»

Hablose esto en la modesta casa (calle del Barquillo) y en la
modestísima tertulia del General, después de comer. Los íntimos que
asiduamente concurrían no pasaban de media docena, y el tiempo se
invertía en conversaciones familiares, o en alguna partida de tresillo
casero, a tanto ínfimo. El juego favorito de O'Donnell era el ajedrez;
pero no quería jugarlo sino cuando la ocasión le deparaba un adversario
digno de su maestría. Conviene hacer constar los hábitos sencillísimos
del gran don Leopoldo. Por las mañanas solía consagrar largas horas a la
lectura de libros y revistas profesionales, que le ponían al tanto de la
ciencia militar de su tiempo. Después de almorzar recibía visita de
gente política, con la cual charlaba discretamente sin dar largas a su
espontaneidad. Paseaba por las tardes, en buen tiempo, con la Condesa;
no iba jamás a reuniones, y a teatros rarísima vez. Por las noches,
después de la tertulia, en la cual se daba el \emph{rompan filas} a hora
temprana, tenía largas pláticas con su mujer, que, por sufrir pertinaces
insomnios, procuraba entretener los instantes hasta que llegase el del
deseado sueño. Gustaba doña Manuela de la lectura de folletines, y se
deleitaba y divertía con los más excitantes, de acción enmarañada y
liosa, que mal traducidos del francés eran la sabrosa comidilla que daba
la prensa de aquel tiempo a \emph{sus amables suscritoras}. Con igual
interés se internaba la Condesa de Lucena en los asuntos enredosos y en
los sentimentales, sin que se le escapara ningún lance ni perdiera jamás
el hilo que por tales laberintos la guiaba.

Pues la noche aquella de la visita de Beramendi y Loarre, que debió de
ser allá por Junio del año 58, retirose como de costumbre doña Manuela a
su estancia apenas terminada la tertulia. Tras ella fue don Leopoldo, y
como las anteriores noches, la invitó a que se acostara. ¿Qué necesidad
tenía de calentarse la cabeza, vestida, leyendo junto al velón? «Yo leo,
y tú escuchas hasta que te entre sueño.» Así se hizo: dispuso la
doncella el velador junto a la cama después de acostar a la señora; el
gran O'Donnell ocupó a la vera de la mesita su sitio, y gozoso del papel
familiar que desempeñaba, tiró de periódico y dio comienzo a la lectura,
en el pasaje que su buena esposa le indicaba: Capítulo tantos de
\emph{El último veterano}; \emph{La Condesa de Harleville y el
Mayordomo}, por \emph{E. M. de Saint-Hilaire}.

Guiando su vista con el dedo índice que de línea en línea resbalaba, el
gran O'Donnell leía:

«Uno de los testigos prestó su sable a nuestro joven, que no decía una
palabra; pero apenas se pusieron en guardia, cuando Monsieur Massenot
conoció que el artillero, a pesar de ser boquirrubio, sería para él un
adversario temible. En efecto: en el momento en que Mr.~Massenot se
aprestaba a introducir con una estocada recta seis pulgadas de hoja en
el estómago del rubio, este ejecutó con su sable un molinete tan rápido,
que se hubiera dicho que era un sol de fuegos artificiales.»

---¡Qué bien!---exclamó doña Manuela con júbilo.---Ese rubio, ya te
acuerdas, es aquel artillerito que vino de la Bretaña disfrazado de
buhonero. Por las trazas es hijo natural de la Condesa\ldots{} Adelante.

---«Mariscal en jefe de los alojamientos, recoged vuestra nariz---le
dijo el artillero con tranquilidad,---y otra vez sed más amable con
vuestros inferiores.»---Estas fueron las únicas palabras que pronunció
el rubio.

---Según eso---observó doña Manuela,---¿le cortó la nariz?

---Así parece\ldots{} Y bien claro lo dice: «El rostro de Massenot se
cubrió de sangre, que corría como dos arroyos sobre sus mostachos
grisáceos.»

---Me alegro, Leopoldo\ldots{} Ande, y que vuelva por otra. Ahora veamos
lo que sigue contando Harleville.

---A eso vamos: «Pues bien, mi querido acuchillado---dijo
Harleville,---esa desgraciada aventura no corrigió al mayor Massenot,
porque en 1815, antes del regreso de nuestro Emperador, se encontraba
una tarde en el café Lamblin, en Palais Royal, sentado enfrente de un
oficial de Dragones\ldots»

Interrumpió doña Manuela la lectura incorporándose y atendiendo a ruidos
que venían del interior de la casa.

---No han llamado---dijo el de Lucena;---sigamos: «enfrente de un
oficial de Dragones, a medio sueldo como él\ldots»

Sí que habían llamado, y también habían abierto. Oyeron doña Manuela y
su marido los pasos de la doncella, que después de un discreto golpe con
los nudillos, entró con un pliego en la mano, y dijo: «Esto trae un
señor de Palacio\ldots»

Levantose O'Donnell, y cogido el pliego abriolo despacio, y leyó para
sí. Impaciente doña Manuela, quería echarse de la cama con esta ardorosa
pregunta: «¿Qué, Leopoldo?\ldots{} ¿Ya\ldots?

---Sí, ya---replicó el grande hombre imperturbable.

---¡A esta hora! ¿No son ya las doce?

---Su Majestad no quiere que pase la noche sin hablar conmigo\ldots{}
Pronto\ldots{} A Matías, que saque mi uniforme. Voy a vestirme.»

Hizo doña Manuela por levantarse, movida de la gran vibración nerviosa y
del cerebral tumulto que aquel repentino suceso en ella promovía. Mas el
General le ordenó que siguiese en la cama, y con tranquilo acento le
dijo al despedirse: «Creo que volveré pronto. Si cuando yo vuelva estás
desvelada, seguiremos leyendo\ldots{} Hay que ver si recobra su libertad
la Condesa, y en qué para ese boquirrubio\ldots{} Hasta luego.»

\hypertarget{xxiii}{%
\chapter{XXIII}\label{xxiii}}

¡Arriba la \emph{Unión Liberal!} ¡Viva don Leopoldo! Al fin se ponía el
cimiento al edificio político que aliaba las expansiones del espíritu
moderno con el recogimiento y la majestad de la tradición. ¡Al poder los
hombres de juicio sereno, no extraviados por el proselitismo sectario,
ni petrificados en bárbaras rutinas! Entren en la vida pública todos los
hombres que al saber de cosas de Gobierno reúnen la distinción y el buen
empaque social. Vengan la riqueza y los negocios a desempeñar su papel
en la política, y ensánchese la vida nacional con la desvinculación de
las comodidades, del bienestar y hasta del buen comer. ¡Abajo la Mano
Muerta! Desamorticemos y repartamos, no con violencia revolucionaria,
sino con parsimonia y suavidad conservadoras, concordando con el Papa la
forma y modo de conciliar los intereses de la Iglesia con los de la
sociedad civil. Hágase política sinceramente constitucional y
parlamentaria. Venga libertad y venga orden, el orden augusto que
engendran las leyes bien meditadas y bien cumplidas. Creemos una
poderosa Marina, un Ejército potente dentro de nuestros medios, y con
este modo de señalar, Ejército y Marina, pidamos un puesto en la
diplomacia europea. Salga de su infancia la ciencia, florezcan las artes
y despójense nuestras costumbres de toda rudeza y salvajismo. Seamos
europeos, seamos presentables, seamos limpios, seamos, en fin,
tolerantes, que es como decir limpios del entendimiento, y desechemos la
fiereza medieval en nuestros juicios de cosas y personas. Transijamos
con las ideas distintas de las nuestras y aun con las contrarias, y
pongamos en la cimera de nuestra voluntad, como divisa, la bendita
indulgencia.

Esto decía Beramendi, ardiente propagandista de la Unión, en todas las
casas adonde solía ir, que no eran pocas, y extremaba sus entusiasmos y
el brío de su declamación en la morada de uno y otro Socobio, don
Saturno y don Serafín, a las cuales concurría después de algunos años de
absentismo. Con la Marquesa de Villares de Tajo, cada día más talentuda
y perspicaz, tenía Fajardo las grandes pláticas de política. Era una
persona con quien daba gusto discutir, disputar y aun pelearse, porque
conocía muy bien el mundo, y manejaba con igual donosura las ideas
propias y las contrarias. Sin abdicar de sus opiniones narvaísticas,
ocasionales sin duda, \emph{la moruna} reconocía la inmensa fuerza con
que O'Donnell entraba en campaña, llevando a su lado lo mejor de los dos
partidos históricos. Del moderado le seguían nada menos que Martínez de
la Rosa, don Alejandro Mon, Istúriz, y otros muchos que estaban ya con
un pie dentro de la Unión. Del \emph{Progreso} había tomado a Prim, a
Santa Cruz, a Infante, a don Modesto Lafuente, a Lemery, a don Cirilo
Álvarez y otros que vendrían detrás. No tenía O'Donnell perdón de Dios
si con tales elementos y la grande autoridad adquirida con su sensato
proceder en la oposición, desde el 56 al 58, no realizaba una obra
memorable de paz y florecimiento en este país. Pronto se vería si España
había encontrado al fin su hombre, o si el que a la sazón la tenía entre
sus manos era una figura más que añadir a nuestra galería de
fantasmones.

El principal móvil de las asiduidades de Beramendi en las casas de uno y
otro Socobio, era que se había impuesto la caballeresca empresa de
reconciliar a Virginia con sus padres, trabajosa, descomunal aventura.
En Mayo de aquel año, antes del triunfo de la Unión, dio principio a la
campaña poniendo cerco a la terquedad de don Serafín, voluntad maciza,
baluarte atávico defendido por ideas contemporáneas del Concilio de
Trento. La expugnación de esta formidable plaza era difícil; mas no
arredraron al gran batallador Beramendi ni la fortaleza de los muros, ni
el vigor de las rutinas que los defendían. Con la táctica del
sentimiento obtuvo las primeras ventajas, y desde el recinto sitiado se
le llamó a parlamentar. Don Serafín y doña Encarnación manifestaron al
caballero que perdonarían a Virginia; que estaban dispuestos a
reintegrarla en su amor, a recibirla en su casa, ya viniese sola, ya con
la añadidura de algún chiquillo, habido en su deshonesta vagancia. Con
ella transigían y con el fruto de su vientre, que ya era mucho
transigir, sacrificando sus ideas y su recta moral al irresistible amor
de padres. Pero jamás, jamás transigirían con \emph{él} (no le
nombraban, no querían saber su nombre); era imposible toda concordia con
semejante pillo: antes morir que admitirle al trato de una familia
honrada. Para que Virginia pudiese tornar junto a sus padres y estos
devolverle su cariño, era menester que el hombre maldito desapareciese,
bien por acto de la ley, bien por consentimiento propio, retirándose a
un punto lejano, \emph{más allá} de los antípodas. Dispuestos estaban a
subvencionar con fuerte suma la fuga del mil veces maldito ladrón, si
este consentía en\ldots{} Beramendi no les dejó concluir. Virginia
deseaba la paz con sus padres; pero por encima de esta paz y de todas
las paces del mundo estaba la inefable compañía del hombre que amaba. No
había, pues, avenencia si don Serafín y doña Encarnación no se quitaban
algunos moños más\ldots{} Protestaron los señores: bastantes moños
habían arrancado ya de sus venerables cabezas; bastante ignominia
soportaban\ldots{} no podían ir más allá.

Rechazado con esta ruda intransigencia, el sitiador se propuso emplear
nuevos y más eficaces ingenios de guerra que abatieran la rígida
entereza socobiana. Confiado en el tiempo, dejó pasar días esperando las
ocasiones favorables que en el curso del verano seguramente se
presentarían. El verano del 58 fue alegre, por los chorros de alegría
que la subida de la Unión derramó sobre el país reseco. O'Donnell vencía
con sólo su nombre y los nombres de los que iban tras él. Creyérase que
por la superficie social corría una ola de frescura, de juventud. La
limpieza y gallardía de tantos jóvenes, o viejos rejuvenecidos, que
subían a oficiar en los altares de la patria con vestiduras nuevas,
infundían confianza y evocaban imágenes de bienestar futuro. Anticipaban
o descontaban algunos las bienandanzas del porvenir, procurándose corto
número de comodidades a cuenta de las muchas que habían de traer los
próximos años, y adoptaban el mediano vivir a cuenta del vivir en grande
que los horóscopos para todos anunciaban. Fuerza es reconocer que con
esta prematura expansión de la vida, obra de los risueños programas de
la Unión, se resquebrajó más el ya vetusto edificio de la moral privada,
reflejo de la pública. Cundían los ejemplos y casos de irregularidades
domésticas y matrimoniales, y se relajaba gradualmente aquel rigor con
que la opinión juzgaba el escandaloso lujo de las guapas mujeres que
eran gala y recreo de los ricos. Descollaba entre estas Teresa
Villaescusa, que en Octubre vino de Andalucía \emph{contratada} por un
rico ganadero de aquel país, tan opulento como sencillo, facha un si es
no es torera, y aires de franqueza campechana; obsequioso con todo el
mundo, con las hembras galante, según el viejo estilo español, que
ordena la frase hiperbólica y el rendimiento sin medida. El hombre
quería darse lustre en Madrid, cosa no difícil trayendo dinero fresco:
era gran caballista, gran bebedor si se ofrecía, cuentista gracioso, y,
en fin, se llamaba Risueño, que es lo mejor que podía llamarse un hombre
de sus circunstancias y condiciones.

Caballos bonitos de casta andaluza, rivales en arrogancia de los que
inmortalizó Fidias en el friso del Partenón, ostentaba en paseos, calles
y picaderos; pero ninguno de sus bellos animales, enjaezados a lo
príncipe, igualaba en arrogancia y primor a Teresa, que por entonces
apareció en la culminante esplendidez de su hermosura, vestida, para
mayor pasmo de los que la veían, con una elegancia tan selecta, tan
suya, que difícilmente la superarían las señoras más encopetadas. ¡Vaya
con la niña, y qué bien se le había pegado París, en el año que allí
tuvo su residencia! Pues viéndola tan reguapa que a los mismos
guardacantones enamoraba, y tan bien trajeadita que era el primer
figurín de la Villa y Corte, todos decían: \emph{esa es la de
Salamanca}, o \emph{el número uno de las de Salamanca}, error que se
explicaba por no ser Risueño bastante conocido en Madrid. En aquel
tiempo, el vulgo señalaba como de Salamanca todo lo superior: las
poderosas empresas mercantiles, los cuadros selectos y las estatuas, las
mujeres hermosas, los libros raros y curiosos\ldots{} Homenaje era este
que tributaba la opinión a uno de los españoles más grandes del
siglo{\textsc{xix}}.

Aunque parezca disonante pregonar las virtudes de personas sobre quienes
recae la maldición pública, la verdad obliga al historiador a decir que
tanto como escandalosa era Teresa caritativa. Tenía medios abundantes de
ejercer la liberalidad; su mano no era una hucha, sino ánfora o tonel
construido por el mismo que hizo el de las Danaides. Lo que entraba por
un lado, no tardaba en salir por otro. Enterada de la miseria en que
estaban los Centuriones, les mandó por Manolita lo necesario para vivir,
y a su madre encargó que les pusiera en libertad toda la ropa que
empeñada tenían. Los dos gabanes de don Mariano, la capa, un pantalón
gris perla que lucía en las grandes solemnidades, las mantillas y el
traje de seda de doña Celia, salieron del cautiverio. Al principio de su
desdicha, repugnaban al buen señor las larguezas y protección de
Teresita; pero el rigor mismo del infortunio le hizo bajar la cresta.
Estábamos en tiempos de tolerancia, de transacción, pues la Unión
Liberal ¿qué era más que el triunfo de la relación y de la oportunidad
sobre la rigidez de los principios abstractos? Se transigía en todo; se
aceptaba un mal relativo por evitar el mal absoluto, y la moral, el
honor y hasta los dogmas, sucumbían a la epidemia reinante, al
\emph{aire de flexibilidad} que infestaba todo el ambiente. Después de
remediar a sus tíos, fue la buena moza a visitarles: doña Celia la
recibió con lágrimas; don Mariano temblaba y sentía frío en el espinazo
oyendo decir a Teresa: «Ya que nadie quiere colocarle a usted, le
colocaré yo, tío; yo, yo misma. Entrará usted en la Unión Liberal, cosa
muy buena según dicen, y que hará feliz a España librándola del peor mal
que sufre, o sea, la pobreza. Créalo usted, don Mariano: todos los
Gobiernos son peores si no dan curso al dinero para que corra de mano en
mano. El Gobierno que a todos dé medios de comer, será el mejor\ldots{}
Lo que yo digo: desamortizar; coger lo que aquí sobra para ponerlo donde
falta\ldots{} igualar\ldots{} que todos vivan\ldots{} ¿Es esto un
disparate?\ldots{} Puede que lo sea por ser mío\ldots{} En fin, adiós;
ánimo, que ya vendrá la buena.»

De allí se fue a casa de Leovigildo Rodríguez, donde hizo de las suyas,
vistiendo las desnudas carnes de tanto chiquillo, y proveyendo a su
alimentación, pues daba lástima ver sus lindas caras macilentas y sus
ojos sin brillo. Mercedes no se hartaba de bendecir a su bienhechora,
prodigándole los elogios que a su parecer debían halagarla más, los de
su belleza y elegancia. Leovigildo, que no tenía escrúpulos, y
transigía, no con el mal relativo, sino hasta con el absoluto, le dijo:
«¡Por Dios, Teresa! colóqueme usted, que bien podrá hacerlo\ldots{} y a
usted le sobran relaciones\ldots{} No tiene más que decir: `esto
quiero', para que todos, de O'Donnell para abajo, se despepiten por
medir su boca y darle cuanto pida.»

En una de las visitas que hizo la Villaescusa a la morada de Leovigildo,
que entonces vivía en un piso alto de la calle de Ministriles, supo que
en los desvanes de la misma casa se moría de hambre una familia.
¡Morirse de hambre! Esto se dice; pero rara vez existe en la realidad.
Subió la guapa mujer y a sus ojos se ofreció un cuadro de desolación que
por un rato la tuvo suspensa y angustiada. No había visto nunca cosa
semejante: mil veces oyó referir casos de la extremada miseria que en
los rincones de Madrid existe. Pero la evidencia que delante tenía,
superaba en horror a todos los cuentos y relaciones. Una mujer de
mediana edad, apenas vestida, yacía entre pedazos de estera y jirones de
mantas, sin alientos ni aun para llorar su desdicha; dos niñas como de
ocho y diez años, la una sentadita en un taburete desvencijado, la otra
de rodillas arrimada a la pared, se metían los puños en la boca, luego
se restregaban con ellos los ojos, exhalando un plañidero quejido sin
fin, como ruido de moscardones. Avanzó Teresa, venciendo su terror y
repugnancia; la suciedad, la pestilencia ofendían la vista tanto como el
olfato. Interrogó a la mujer, observando al verla de cerca que no era
bien parecida; pronunció la mujer frases entrecortadas como las que
emplean con artificios los que pordiosean en la calle; pero que de la
boca de ella salían con el acento de la pura y terrible verdad\ldots{}
«¿No tienen ustedes ningún recurso?---dijo Teresa traspasada de
aflicción.---¿En qué se ocupa usted?\ldots{} ¿Es que no han comido hoy?
¿No hay ninguna persona caritativa en este barrio?» Respondió la infeliz
mujer que personas buenas había, pero ya se habían cansado de
socorrerla\ldots{} No comían sino cuando les llevaba de comer otro
desgraciado que con ellos vivía.

---Y ese desgraciado, ¿dónde está?

---Aquí\ldots{} Mírelo---dijo la medio muerta de hambre, señalando a un
hombre que en aquel instante entraba.---Si \emph{Tuste} nos trae,
comemos; si no, lloramos.»

El llamado \emph{Tuste} permanecía junto a la puerta, respetuoso. En una
mano tenía la gorra que acababa de quitarse, en otra dos lechugas
manidas. «Usted, buen hombre\ldots---dijo Teresa volviendo sus miradas
hacia el tal, y encarándose con la figura más desastrada y haraposa que
podía imaginarse.---¿Trae algo que coma esta pobre gente?

---Esto nada más, señora---replicó \emph{Tuste} mostrando las dos
lechugas.---Me las han dado unas vendedoras en la plazuela de
Lavapiés\ldots{}

---¡Valiente porquería!---dijo Teresa, que gustando de mirarlo todo, por
repugnante que fuese, examinó de pies a cabeza la facha de \emph{Tuste},
en quien se reunían los más tristes y desagradables aspectos de la
miseria. Lo que se veía de la camisa era la misma suciedad; la chaqueta
y calzones, prendas de ocasión que debieron de ser viejísimas antes que
él las usara, eran ya jirones de tela mal cosidos, llenos de agujeros y
desgarraduras\ldots{} El calzado lo componían dos zapatos diferentes: el
derecho a medio uso; el izquierdo informe, retorcido, suelto de
puntos\ldots{} Observado con rápida vista todo esto, miró Teresa el
rostro, y espantada de la suciedad espesa que lo cubría, no pudo
distinguir las líneas hermosas, ni la noble expresión que debajo de la
inmunda costra se escondía. El cabello era una maraña en que no había
entrado el peine desde la invención de este instrumento de limpieza. La
mugre de toda la cara se hacía más densa metiéndose por los huecos de
las orejas; en el cuello de la camisa se apelmazaba el sudor; la tela y
la piel se confundían en su morbidez pegajosa. Desgarraduras de la
camisa dejaban ver una parte del pecho menos sucia que lo demás, tirando
a blanca.

Arrebató Teresa de las manos puercas de \emph{Tuste} las dos lechugas;
sacó de su bolsillo el poco dinero que le quedaba; dio una parte a la
mujer, otra al hombre sucio, diciéndole: «Corra usted a la tienda y
traiga lo más preciso para que coman hoy; traiga carbón, encienda
lumbre\ldots» Y a las niñas acarició, y de ellas y de la que parecía su
madre se despidió con estas afectuosas expresiones: «Vaya, no lloren
más. Hoy es día de estar contentas, ¿verdad que sí? \emph{Tuste} les
traerá para que almuercen. En seguida que aquí despache, le mandan a mi
casa\ldots{} les dejaré las señas en este papel\ldots{} Pues que vaya
corriendo, y por él recibirán un par de mantas\ldots{} ropa mía de
desecho, y alguna golosina para estas criaturas. Vaya, adiós: alegrarse.
Ya no se llora más.»

\hypertarget{xxiv}{%
\chapter{XXIV}\label{xxiv}}

Fue \emph{Tuste} a casa de Teresa, y la criada le anunció de este modo:
«Ahí está un pobre muy asqueroso: dice que la señorita le mandó venir.
Si la señorita tiene que hablar con él, echaré un poco de sahumerio.» Ya
había escogido Teresa las ropas usadas que debía mandar a la calle de
Ministriles. Salió presurosa al recibimiento, donde la esperaba el más
miserable de los hombres, quien al verla se inclinó respetuoso, mudo,
pues toda palabra le parecía insuficiente para expresar su gratitud. «Ha
venido usted demasiado pronto---le dijo Teresa.---La ropa mía de
desecho, aquí está; lo demás, tengo que salir a comprarlo.

---Volveré cuando la señora me mande. ¿Qué tengo que hacer más que
obedecer a la señora?» Esto dijo \emph{Tuste}. La voz del pobre no era
como su facha, sino una voz espléndida, de timbre sonoro, dulce,
varonil. Así lo advirtió Teresa la segunda vez que la oía; en la primera
no advirtió nada. En aquel punto de apreciar la bella voz del sujeto, un
ligero brote de curiosidad en el espíritu de Teresa la movió a formular
esta pregunta: «¿Usted cómo se llama? ¿Su nombre de pila\ldots?

---Yo me llamo Juan. Mi apellido es Santiuste. La mala pronunciación de
aquellas niñas me ha convertido en \emph{Tuste}\ldots{} Lo mismo da,
señora. He venido tan a menos, que ya no me detengo a recoger ni las
letras de mi nombre que se caen al suelo.» Avivada con esto la
curiosidad de Teresa, se acercó a él para verle mejor; apartose al
instante, y dijo: «Cuénteme usted: ¿qué familia es esa y cómo ha venido
a tanta postración? Y usted, ¿qué relación tiene con esa familia?

---Se lo contaré en pocas palabras para no cansar a la señora\ldots{}

---Aguárdese un poco\ldots{} Antes tiene que decirme por qué es usted
tan sucio\ldots{}

---No lo soy, lo estoy\ldots{} Permítame decirle que no debe juzgarme
por lo que ve. Dentro de estas apariencias inmundas hay otra persona. De
algún tiempo acá vivo, si esto es vivir, como si me hubiera entregado a
la tierra para que me descomponga. Mis desgracias me han inspirado el
horror del aseo. Abandonado de todo el mundo, sin nadie que me socorra,
el tener una facha desagradable ha sido para mí como un desquite, como
una venganza\ldots{} ¿Quería usted que saliera a pedir limosna vestido y
peinado como un señorito? Nadie me hubiera hecho caso. ¡La miseria!
Quien no conoce la miseria, quien no ha vivido en ella, quien no se ha
revolcado en ella, no puede apreciar el goce de ser repugnante\ldots»

La curiosidad de Teresa, con cada uno de los extraños dichos del sucio
se avivaba. Quería saber más. Tuste le ofreció un resumen de su
infortunada existencia. Nació en la Habana, de padre burgalés y madre
andaluza; dos años tenía cuando le trajeron a la Península; pasó su
niñez en Alicante, donde quedó huérfano; recogiéronle unas tías
residentes en Chiclana; allí corrió su adolescencia, allí estudió todo
lo que estudiarse podía en un pueblo de escasa cultura; casi hombre, le
llevaron a Cádiz, donde siguió estudiando y adquirió ardiente afición a
la lectura; hombre ya, y no cabiendo en aquella ciudad su espíritu
ambicioso, se vino a Madrid, solo, con escasísimo dinero que le dieron
sus tías. Estas habían empobrecido, y él no quiso serles gravoso\ldots{}
A la mitad del camino se quedó sin blanca y tuvo que continuar a pie. En
Madrid buscó el amparo de un pariente de su madre a quien las tías le
recomendaron: era un impresor llamado Quintana, que le acogió muy bien,
ocupándole en su establecimiento como corrector; le daba de comer, le
vestía pobremente, porque no podía más, y le matriculó en la Universidad
para que estudiara las dos Facultades de Derecho y Filosofía y Letras.
Trabajaba Juan y leía con insaciable anhelo cuanto libro caía en sus
manos, que no eran pocos. Tres años cursó en la Universidad, donde hizo
amistades con chicos aplicados y con otros que no lo eran. El 56 murió
el bueno de Quintana de un repentino mal del corazón, y esta desgracia
fue como el preludio de las innumerables que estaban aguardando al pobre
Juan para devorarle y consumirle\ldots{} Ya no hubo para él un día de
reposo, ni una hora que no le trajera inquietudes y fatigas. Trató de
buscar algún recurso con su trabajo; pero difícilmente allegar podía un
pedazo de pan. En diferentes periódicos solicitó colocación; en algunos
escribía de materias diferentes: Política extranjera, Toros, Literatura,
Música, Salones, Hacienda\ldots{} No le leían ni le pagaban. Escribió
después aleluyas, compuso versos para novenas\ldots{} Todo resultaba
trabajo perdido, infecundo. Aunque ya no iba a la Universidad porque no
tenía ropa presentable, solicitó de alguno de sus amigos estudiantes, y
de otros que ya no lo eran, apoyo y recomendación para obtener algún
destino. Nada consiguió: ni moderados ni progresistas le hacían maldito
caso. Trató de meterse a hortera; pretendió plaza en una Sacramental; se
arrimó a un memorialista. Nada: no había manera de luchar contra el
hambre y la muerte. De patrona en patrona iba rodando por Madrid,
tolerado en algunas casas, rechazado en otras por su irremediable
insolvencia, hasta que fue a poder de la más infeliz de las pupileras,
Jerónima Sánchez, que tenía su hospedería en la calle de Mesón de
Paredes. Era el marido de esta señora un incorregible borrachín que
espantaba a los huéspedes con sus groserías y malas palabras. La casa
iba de mal en peor\ldots{} Desertaron los demás pupilos, dos chicos de
Veterinaria y uno de Medicina; sólo quedó Juan, que por entonces pudo
allegar algunos cuartos llevando las cuentas en una tienda de patatas y
huevos, y en un establecimiento de ataúdes y mortajas. Así las cosas, en
Marzo del año mismo en que esto refería Santiuste, reventó Cuevas, el
bebedor esposo de la patrona, muerte que fue como incendio del alcohol
que llevaba en sus entrañas, y Jerónima, descansada ya de aquella cruz,
tomó otra casa; puso papeles llamando huéspedes, y estos no picaban.
Perdió Juan su colocación miserable en los dos establecimientos
referidos; pero Jerónima no le despidió, esperando mejor suerte para el
desamparado joven. La suerte ¡ay! no vino para él ni para ella, porque
Juan cayó enfermo de calenturas y estuvo a la muerte, siendo tan
desgraciado que hasta la muerte le despreció y no quiso
llevársele\ldots{} y la pobre Jerónima, cuando él iba saliendo adelante,
resbaló en la cocina (encharcada del agua de jabón que rebosaba de la
artesa), y cayendo torcida y en mala disposición, se rompió una pierna
por bajo de la rodilla\ldots{} Era Juan agradecido, y no abandonó a la
que a él le había tan noblemente amparado. Reunidas quedaron desde
entonces ambas desdichas, y recíprocamente se apoyaron, corriendo juntos
el temporal. La rotura de pierna de Jerónima excluía todo trabajo
patronil. Se acabaron los recursos, y empezó el rápido descender de
escalón en escalón hasta la miseria lacerante y angustiosa. Por
diferentes casas pasaron, y de una en otra iban llevando su mala sombra,
su pavoroso sino; siempre a peor, a peor: cada día más desnudos, cada
día más hambrientos, hasta llegar al horrible extremo en que les vio y
descubrió la señora. Cuando ya les faltaba poco para morir, se les
apareció un ángel que les dijo de parte de Dios: «Vivid, pobres
criaturas, que también para vosotros existo.»

---Haga usted el favor, señor Santiuste---dijo Teresa, que con algo de
broma quería disimular su emoción,---de no llamarme a mí ángel, pues no
lo soy ni por pienso, y paréceme que se burla usted de mí\ldots{} Pero
dejemos eso, que es tarde y tengo que salir de tiendas. Lleve usted
ahora esta ropa para Jerónima; también le van medias y un par de zapatos
de mi madre, que tiene el pie mucho mayor que el mío\ldots{} No vuelva
usted hoy por lo demás, sino mañana, que así tendré yo más tiempo de
reunir lo que quiero mandarles\ldots{} Vamos, que algo habrá para usted
también, grandísimo Adán.

Alelado de gratitud y admiración, Santiuste no dijo nada. Teresa
prosiguió así, más burlona que compasiva: «¿Pero por pobre que esté un
hombre, Señor, ha de faltarle un real para cortarse esas greñas?\ldots{}
y en último caso, buscar un barbero caritativo, que ya los habrá.
Felisa, trae un peine tuyo\ldots{} Empecemos desde hoy a desenmascarar
este esperpento. Cuidado que es usted horroroso\ldots{} Ea, tome el
peine y métalo en ese bosque\ldots» Después de besar el peine, Juan lo
guardó entre el pecho y la camisa, único bolsillo practicable en su
astrosa vestimenta\ldots{} Y partió balbuciendo expresiones de exquisita
ternura, que ama y criada apenas entendieron. Creían que lloraba\ldots{}
¡y ellas le compadecían riendo, pobres mujeres que no conocían más que
la superficie del mal humano!

Volvió puntual Santiuste a la mañana siguiente, y al salir Teresa al
recibimiento, se maravilló de ver extraordinaria transformación en la
cabeza y rostro del infeliz hombre. Se había lavado la cara, pescuezo y
manos, sin duda con muchísimas aguas y con fuertes restregones, porque
no quedaba ni el más leve rastro de la suciedad que le desfiguró. Era
como una resurrección. De las tinieblas salía una cabeza admirable, un
rostro hermoso, grave, tan escaso de barba y bigote, que con un ligero
pase de navajas quedaba limpio; salía también la juventud. De asombro en
asombro con tales descubrimientos, Teresa decía: «A mí no me engaña
usted, señor Tuste. No es usted el de ayer, sino otro\ldots{} o el mismo
con distinta cabeza\ldots{} Hoy trae la cabeza joven\ldots{} y la boca
joven, y joven toda la carátula\ldots{} que me parece viene también
afeitadita.

---Sí, señora---replicó Juan con infantil orgullo, confundido por los
elogios de la hermosa mujer.---Del dinero que la señora dio a Jerónima,
Jerónima apartó un real para que yo me afeitara. Ayer compramos
jabón\ldots{} El jabón es un ingrediente que no habíamos podido ver en
mucho tiempo.

---¡Vaya, que no se ha lavoteado usted poco!\ldots{} Así, así me gusta a
mí la gente\ldots---decía Teresa, acercándose a él con menos repugnancia
que el día anterior.---Y otra cosa veo, que me deja atónita. ¡La camisa
limpia! ¡Qué lujo! Bien, bien. La habrá lavado Jerónima.

---No, señora: la he lavado yo mismo, anoche\ldots{} ¡qué noche, señora!
No hemos dormido\ldots{} las niñas tampoco han dormido. La aparición de
usted en aquel mechinal indecente nos ha trastornado a todos. Ni
Jerónima, ni las niñas, ni yo, acabábamos de convencernos de que la
señora es persona humana\ldots{} Todavía hoy\ldots{} las niñas hablan de
usted como de un ser sobrenatural\ldots{} Es el hada de los cuentos de
niños, o el ángel de las leyendas cristianas.

---Vuelvo a decirle que a mí no me llame usted ángel ni hada\ldots{}

---No es usted, no, como las demás personas---dijo Tuste, soltando poco
a poco su timidez.---Bajo esa vestidura mortal se esconde un ser que
tiene por morada la inmensidad de los cielos, un ser que en su aliento
nos trae el propio hálito del Padre de toda criatura\ldots{}

---¡Ay, ay, ay! cállese por Dios. ¿Pero es usted también poeta?

---No, señora: cuando Dios quiso, yo no escribía versos, sino prosa.

---Prosa con la cara sucia, versos con la cara limpia\ldots{} No me haga
reír\ldots{} ¿Pero usted cree que se puede ser poeta ni prosista con
esas botas? ¿No le da vergüenza de andar por el mundo con calzado tan
indecente?

---Antes de que la señora se nos apareciera, me daba vergüenza de
acicalarme: la fealdad y el desaseo eran la mueca con que yo hacía burla
del mundo que me abandonaba. Ahora, deslumbrado por el ángel de
luz\ldots{} perdone usted\ldots{} por la divina mensajera del Dios de
piedad\ldots{} es todo lo contrario\ldots{} Me avergüenza mi facha
repugnante, y toda el agua del mundo me parece poca para mi limpieza, y
cien Jordanes no me bastarían para purificarme.

---¡Ya escampa!\ldots{} Basta de poesía, y venga, véngase a la
prosa---dijo Teresa, conduciéndole con Felisa a una estancia inmediata,
el despacho de la casa convertido en guardarropa.---Pase y verá lo que
tiene usted que llevarse. Irá cargadito como un burro; pero ¿qué le
importa?\ldots{} Mire, mire: un trajecito para cada niña\ldots{}
camisas, delantalitos, medias y zapatos\ldots{} Para usted dos mudas
completas de ropa interior\ldots{} La exterior quedará para más
adelante, que no se puede todo de una vez\ldots{} ¿Qué le parece todo
esto? Y dos cajas de galletas finas para las chiquillas\ldots{} para
Jerónima un refajo\ldots{} para todos dos mantas\ldots{} ¿Qué
dice?\ldots{} Eche, eche poesía\ldots{}

---Si pudiera traer a mi mente la inspiración de Homero---dijo Tuste con
arrobamiento no afectado,---expresaría una parte no más de la gratitud
que debemos a nuestra bienhechora. Nuestra bienhechora reúne en sí toda
la belleza de las divinidades paganas y toda la esencia sublime de la
Ley evangélica.

---Pues pagana y evangélica, ¿sabe usted lo que se me ocurre? Pues que
le voy a obsequiar con unas botas\ldots{} Usted mismo se las comprará.
Aquí tiene cuatro napoleones\ldots{} Ha de prometerme que no empleará
este dinero en otra cosa\ldots{}

---Si yo contraviniera las órdenes de nuestra deidad tutelar, merecería
la muerte; algo peor que la muerte, el desprecio de la señora.

---No se remonte tanto y tome los napoleones\ldots{} ¿Esa costumbre de
besar las monedas, la adquirió usted cuando pedía limosna?

---Beso el metal que ha sido tocado por la mano caritativa. La caridad,
hija del cielo, es la cadena de oro que une al Criador con la criatura.

---Bueno, bueno. Usted siempre tan poético\ldots{} Por unas tristes
botas baratas que le regalo, saca a relucir a Dios y a los
santos\ldots{} ¿Dónde ha aprendido usted, Juanito, a expresarse de esa
manera tan superfirolítica?

---Se lo explicaré, si tiene paciencia para oírme un rato.

---Sí que le escucho. Siéntese en ese banco\ldots{} A ver\ldots{}
¿Cómo\ldots?

---Pues este lenguaje mío es el reflejo del espíritu de la elocuencia
sobre mi pobre espíritu. Tres años ha, el 55, estudiando yo en la
Universidad, y reunido siempre con otros chicos, ávidos de saber y
amantes de la literatura, me metía\ldots{} nos metíamos en todo sitio
público donde hubiera lectura de versos, explicación de doctrinas nuevas
o viejas, discursos\ldots{} Un día caímos en el teatro de
Oriente\ldots{} gran fiesta de la inteligencia\ldots{} concurso de
oradores para cantar la Democracia. ¡Qué día, señora! Lo tengo por el
más memorable de mi vida; día solemne, día grande, porque en él vi salir
el sol de la elocuencia, el Verbo del siglo XIX, Emilio Castelar\ldots{}
Habían hablado no sé cuántos oradores, que nos parecieron bien\ldots{} Y
concluía la sesión, cuando pidió vez y palabra un joven regordete,
tímido, a quien nadie conocía. El buen público, ya cansado de tanta
oratoria, remuzgaba con murmullo de impaciencia, casi casi de
burla\ldots{} Pues, Señor, rompe a hablar el hombre, y a las primeras
cláusulas ya cautivó la atención de la multitud\ldots{} ¡Qué voz, qué
gesto oratorio, qué afluencia, que elegancia gramatical, qué giro de la
frase, qué aliento soberano, qué colosal riqueza de imágenes, encarnadas
en las ideas, y las ideas en la palabra! El público estaba absorto; yo,
embelesado, creía que no era un hombre el que hablaba, sino un mensajero
del cielo, dotado de una voz que a ninguna voz humana se parecía.
Avanzaba en la oración aquel hombre bendito, y el público electrizado le
seguía, sin poder seguirle; iba tras él cuando se remontaba a las cimas
más altas de la elocuencia, y desde aquella altura caía deshecho en
aplausos, quebrantado de tanta emoción\ldots{} Yo estaba como loco; yo
adoraba la Democracia, cantada por el orador con la infinita salmodia de
los ángeles, y cuando acabó, me sentí anonadado\ldots{} me sentí grano
de arena, que por un instante había estado en la cima de aquel
monte\ldots{} ¡y ya me encontraba otra vez en el llano!\ldots{}
¡Castelar! Este nombre llenaba mi espíritu. Por muchos días siguieron
retumbando en mi cerebro ideas, imágenes que le oí, y mi memoria
reconstruyó trozos de aquella oración superior a cuanto han oído hasta
hoy los hombres\ldots{} Desde entonces, yo leía cuanto publicaba
Castelar en los periódicos, y las reproducciones de sus discursos. Nunca
le hablé\ldots{} Si le veía en la calle, iba tras él hasta que se me
perdía de vista\ldots{} era mi ídolo, y lo será siempre, porque si en
los días de mi atroz miseria se me borraron del espíritu las cláusulas
arrebatadoras que yo recordaba, y todo se me obscureció, como si mi
asquerosa naturaleza no fuera digna de contener tales hermosuras, en
cuanto la mano de la señora me sacó de aquella inmundicia, volvieron a
mi mente Castelar y su elocuencia sublime, y ya lo tengo otra vez en
mí\ldots{} Es mi sol, mi oxígeno, y el alma de mi alma.

---Cállese ya---dijo Teresa un poco sofocada de la emoción.---¿Pues no
me ha hecho llorar con esa cantinela? Vea, vea mis ojos\ldots{} No me
gusta llorar, no quiero afligirme por nada. En el mundo no estamos para
eso.»

Levantose Santiuste, creyendo sin duda que permanecía demasiado tiempo
en la visita, y recogiendo los líos y paquete que había de llevarse,
soltó así la vena de su facundia: «Anoche, el contento de verme
redimido, por esa divina mano, de la esclavitud de esta pobreza
embrutecedora, hizo renacer en mi alma toda la poesía castelarina,
soberano monumento oratorio de la Democracia triunfante, de la Libertad
iluminada por la idea cristiana. Mientras lavaba y fregoteaba, primero
mi rostro, después mi camisa, yo, como todo el que está muy alegre,
cantaba y rezaba, que rezo y canto era todo lo que salía de mi
boca\ldots{} Recitaba con amor y fe aquel pasaje del advenimiento del
Redentor: «El que había de venir, viene; el que había de llegar, llega;
pero no viene ni en el seno de la sonrosada nube ni en el de las
estrellas, sino manso y humilde en el seno de la pobreza y de la
desgracia. No viene acompañado de numeroso ejército, sino de su bendita
palabra y de su eterno amor; no viene seguido de esclavos, sino ansioso
de acabar con toda esclavitud; no viene blandiendo la espada del tirano,
sino pronto a quebrantar todas las tiranías; no viene a levantar un
pueblo sobre otro pueblo, ni una raza sobre los huesos de otra raza,
sino a estrechar contra su pecho y a bendecir con el infinito amor de su
corazón todos los pueblos y todas las razas\ldots»

---Basta, basta, Juanito---le dijo Teresa interrumpiéndole y casi
echándole con un gesto.---¿No ve que se me saltan las lágrimas?\ldots{}
Retírese ya\ldots{} ¡No quiero lágrimas, no las quiero, ea!\ldots{}
Adiós, adiós\ldots»

Y el gran Tuste traspasó la puerta y descendió los pocos escalones que
conducían al portal, cantando más que repitiendo con briosa voz el final
de aquella sonora melopea: «Dios de paz y de amor, que después de haber
extendido los inmensos cielos azules y haber derramado en los cielos,
como una lluvia de luz, las estrellas, y haber hecho salir del obscuro
seno del caos la tierra coronada de flores, ¡él! causa de toda vida,
autor de toda existencia, se despoja de su vida, de su existencia, por
la salud y la libertad de los hombres en el altar sublime del Calvario.»

\hypertarget{xxv}{%
\chapter{XXV}\label{xxv}}

Vivía Teresa en la calle del Amor de Dios, piso bajo. La casa era
hermosa y desahogada, de altos techos. Cuatro ventanas con rejas le
daban luz por la calle; por el interior, los huecos abiertos a un patio
anchuroso y limpio. El día en que Tuste recibió los cuatro napoleones
para unas botas, Teresa le dijo: «Quiero yo enterarme de que usted no se
gasta el dinero en otra cosa que el calzado. No venga usted a casa; pero
pásese por la calle\ldots{} yo estaré en la ventana. La mejor hora es
por la tarde, de tres a cuatro.» Obediente y puntual, hizo el hombre su
aparición, y al tercer recorrido por la acera de enfrente, vio a Felisa
en la enrejada ventana. A poco apareció Teresa, y ambas sonriendo le
llamaron. Acercose Juan, y oyó de labios de su bienhechora estas dulces
palabras: «Bien, señor Tuste: así se portan los caballeros. ¡Y qué bien
le van las botitas!\ldots{} ¡Lástima que el traje no
corresponda!\ldots{} En fin, retírese ya, y diga usted a Jerónima que
esta, Felisa, le llevará el socorro para la semana.» Saludó el hombre, y
respetuoso se alejó con la cabeza baja, el andar lento.

Dos días después, asomada casualmente Teresa, le vio aparecer doblando
la esquina de la calle de Santa María. Aguardó un poco, le llamó con
gracioso gesto, y cuando le tuvo debajo de la reja, le dijo: «Pobrecito,
tú has salido hoy a pedir limosna. ¿Quieres que te eche dos cuartos?

---No vengo a pedir limosna, señora---respondió Juan doblando el
pescuezo, como para mirar al cenit;---vengo porque no hay día que no
pase yo por esta calle\ldots{} Esta calle es mi religión.

---No te entiendo, bobito---dijo Teresa, sin darse cuenta de que por
primera vez le tuteaba.

---Me entiendo yo.

---Te echaré los dos cuartos, para que no se te olvide que eres pobre.
Aguárdate un instante, que no tengo aquí calderilla.»

Volvió al poco rato, y sacando la mano fuera de la reja en ademán de
arrojar algo, dijo al que parecía mendigo bien calzado: «Pon tu gorra
más acá\ldots{} a plomo de mi mano\ldots{} no se caigan los dos cuartos
a la calle.»

Puso Tuste la gorra como se le mandaba; tomó bien la puntería Teresa, y
la moneda cayó dentro de aquel casquete asqueroso de forma
indefinible\ldots{} Brilló en el aire la moneda, y antes de que cayera
vio Santiuste que era un doblón de a cuatro. No pudo hacer ninguna
observación, porque Teresa desapareció de la reja cerrando los
cristales. Minutos después, sonaba la campanilla de la puerta; abrió
Felisa, y se encaró con el pobre, que le dijo: «Quiero ver a la señora
para devolverle una cosa que se le ha caído a la calle.» No había
concluido la frase, cuando apareció Teresa en el recibimiento, risueña,
y replicó al joven con esta graciosa burla: «Es verdad: me equivoqué.
Eché oro en vez de cobre. Venga mi monedita\ldots{} Gracias\ldots{} Eres
un mendigo honrado\ldots{} Dios te lo premie.

---Es que---murmuró Juan---me dio vergüenza de\ldots{} de eso, de que el
oro fuese para mí. Bastante ha hecho la señora por este infeliz\ldots{}
Si yo abusara sería un malvado; empañaría el resplandor de la bendita
caridad, hija del Cielo, con el aliento de mi egoísmo\ldots{} La caridad
obliga al que la recibe a ser tan bueno como el que la hace.

---Echa más poesía, hijo\ldots{}

---Esto no es poesía\ldots{} es mi corazón, que habla con el lenguaje de
su delicadeza, de su gratitud\ldots{}

---Pues has de saber que yo soy muy prosaica, Juan, y no gusto de verte
con esos andrajos tan\ldots{} poéticos---dijo Teresa echando mano al
bolsillo.---Mira, mira toda la calderilla que aquí tenía yo guardada
para vestirte de prosa\ldots{} ¿No has querido un doblón? Pues mira,
cuenta: dos, tres, cuatro, cinco. Voy entendiendo que te gusta ser muy
cochino, muy zarrapastroso y muy nauseabundo, para que te tengamos
lástima\ldots{} Yo te pregunto: si así te viese tu ídolo Castelar, ¿qué
diría?\ldots{} Ves tu ropa como una vestidura poética, que te hace muy
interesante\ldots{} Te las das de anacoreta o de santo. Pues esos moños
te los voy yo a quitar.»

Atónito, asaltado de diferentes emociones, Santiuste no sabía si reír o
llorar. Mayor fue su turbación cuando oyó estas palabras de Teresa:
«Coges ahora mismo estos doblones; vas a una tienda de ropas hechas de
la calle de la Cruz o de cualquier calle, y te compras un terno,
pantalón, chaqueta, chaleco, todo modestito; no vayas a creerte de la
Unión Liberal y a vestirte a lo grande\ldots{} Añades corbata\ldots{}
añades un sombrero, mejor gorra\ldots{} No es tiempo todavía de que te
emperifolles demasiado. Con que\ldots»

No hizo ademán de tomar las monedas. Su inmovilidad era la de una
estatua; su hermoso rostro, su mirar perdido revelaban los efectos de la
fascinación de imágenes lejanas. Díjole Teresa que abandonara los
espacios poéticos a que miraba y descendiese al mundo. Bajó Tuste,
protestando de la nueva limosna con expresiones balbucientes; Teresa
sacó las uñas, sacó su autoridad: «O me obedeces, Juanito, en todo lo
que te mando, o no vuelvas a mirar esta cara mía\ldots{} Te digo que si
tú me miras, yo doy un giro rápido a todo el cuerpo ¿ves?, para decirte
sin palabras: «Quítate de mi vista, \emph{democrático}\ldots{} poético y
castelareño\ldots{} Vete con tus músicas a otra parte.» Aplacados con
esta amenaza los escrúpulos del hombre mísero, tomó el dinero, y con
paso lento, con visajes de asombro y algún gesto que revelaba su esclava
sumisión a la bienhechora traspasó la puerta. Antes de que Felisa
cerrase tras él, volvió Juan presuroso diciendo: «Señora, señora,
¿cuando compre la ropa y me la ponga, he de pasar por aquí? ¿Quiere la
señora verme?\ldots{}

---No---dijo Teresa,---no es preciso. Ni vengas a casa, ni pases por la
calle. Haz lo que te mando, Juan.» Afirmaba él con la cabeza; salió
suspirando\ldots{}

Por aquellos días, que eran los que precedieron a las elecciones, el
feliz poseedor de Teresita, Facundo Risueño, andaba muy metido en
enredos electorales, pues como hombre de gran propiedad en una comarca
de Andalucía y de no poca influencia, le bailaban el agua don José
Posada Herrera y el Marqués de Beramendi, candidato cunero designado
para representar en Cortes aquel distrito. Tras un sinfín de pláticas
con el cunero y con el Ministro, dio gallardamente todo su apoyo el buen
Risueño, ofreciendo que sin necesidad de trasladarse a Andalucía, y sólo
con escribir cartas imperativas a diferentes personas de allá, se
aseguraba la elección. Así lo hizo, y al hombre se le cansó la mano de
tanto plumear, atarugando diariamente el correo con el fárrago de su
correspondencia. Por todo ello y por su activo proceder, estaba Fajardo
muy agradecido al andaluz, y quedaron uno y otro enlazados en sincera
amistad. Vivía Risueño con un hermano suyo, rico también, establecido
aquí desde el año 50 en negocio de aceites. Llamamos vivir al tener allí
un cuarto bien provisto y arreglado, en el cual rara vez dormía. Sus
comidas eran siempre fuera de casa, bien en los colmados y fondas, o
bien en casa de Teresita, que algunos días veía en torno de su mesa, con
cierto tapadillo, a personajes políticos de viso, y a caballeros
aristócratas, aficionados a caballos o a toros. La asiduidad de Facundo
en la vivienda de su linda coima aflojó un poco en los días del trajín
electoral; pero una vez llenas las urnas con el nombre de Beramendi, y
proclamado su triunfo, restableció el andaluz la normalidad de sus
costumbres, y el primer convite que organizó en casa de Teresa fue para
obsequiar al nuevo diputado y a otros amigos, auxiliares en la electoral
batalla.

La novedad de aquel banquete fue que Teresa contó su aventura de caridad
en la calle de Ministriles, y el descubrimiento que había hecho de un
horripilante caso de la miseria humana. Cautivaban estas historias al
buen Beramendi, que era muy amante del pueblo, y sabía, como nadie,
condolerse de sus desdichas. Dio a entender Teresa que si el
\emph{contratista} la dejaba explayarse en sus aficiones benéficas,
trataría de restaurar a los hambrientos de la calle de Ministriles en la
situación o estado que tuvieron antes de su desgracia; restablecería la
casa de huéspedes, en la cual sería primer punto el hombre raro, el
hombre poético, que hablaba como Castelar. Más vanidoso que caritativo,
Facundo Risueño la autorizó, delante de los amigos, para que aplicase a
socorrer al prójimo parte de la \emph{guita} que él le daba para
alfileres.

Así lo hizo Teresa, y apenas entrado Diciembre, tenía Jerónima su casa
de pupilos en la calle de Juanelo, amuebladita con modestia y provista
de todo; las niñas iban a un colegio, y el famoso Tuste hallábase en el
pleno goce de un cuartito decente en la casa, y de algunas prendas de
ropa para salir decorosamente en busca de colocación o trabajo. De vez
en cuando iba Teresa a contemplar su obra y a oír las alabanzas y
bendiciones de los favorecidos. A Santiuste le encontraba como en
éxtasis, mirándose en su ropa, satisfecho y un tanto presumido;
cuidándose el rostro y el pelo, que ya llevaba cortado y a la moda;
esmerándose en el aseo y corrección de la persona. A su bienhechora
mostraba un respeto que rayaba en devoción fanática. En la casa
expresaba su culto con retóricas de un espiritualismo sutil, y
declamaciones hiperbólicas, parafrásticas, imitadas del gran modelo de
oratoria; en la calle, alguna vez que se encontraban casualmente,
saliendo Teresa de la casa de Jerónima, no se atrevía el buen Tuste a
darle convoy, temeroso de que la compañía de un hombre humildísimo
mermara el decoro de tan gran señora; y a propósito de esto tuvieron en
cierta ocasión unas palabras que merecen transcribirse.

«Déjate de pamplinas, Tuste---le dijo Teresa, entrando los dos en la
Plaza del Progreso,---y no me llames a mí gran señora ni nada de eso,
pues soy la menor cantidad de señora que se puede imaginar. O eres un
inocente que no conoce el mundo, o crees que yo me pago de nombres vanos
y de palabras sin sentido.

---No será usted gran señora para los demás---dijo Tuste con efusión
caballeresca;---para mí lo es, y yo hablo por mí, no por el mundo que me
condenó a la miseria\ldots{} y en la miseria estuve hasta que me sacó un
ángel del Cielo.

---Pamplinas, vuelvo a decir, recomendándote por milésima vez, pobre
Tuste, que no seas pamplinoso, y que todas las faramallas bonitas que
has aprendido de Castelar las guardes para pasar el rato. En la vida
real, eso no sirve para nada. Yo no soy señora, aunque como las señoras
me visto; yo, para decirlo de una vez, soy una mujer mala, una\ldots{}
que se ha dejado poner en la frente el letrero de mujer mala\ldots{}
Llevo ese letrero, que leen todos los que me conocen\ldots{} No conviene
que me vean contigo por la calle; pero no es porque yo me avergüence de
ti, ni porque tu compañía me deshonre, sino porque en mi condición de
mujer mala, si me ven contigo creerán lo que no es\ldots{} El hombre con
quien ahora estoy, Facundo, ya sabes\ldots{} es bueno y no repara en que
yo gaste lo que quiera\ldots{} pero tiene la contra de que es algo
celoso, y por cualquier cuento, por cualquier chismajo que le lleve un
adulón o un mal intencionado, se pone insufrible\ldots{} Con que\ldots{}
da media vuelta, Tustito, y déjame sola\ldots{} Las señoras de mi
\emph{categoría} van mejor solas que bien acompañadas. Abur.»

Esto pasó y esto se dijeron. Santiuste buscaba la soledad para dar libre
rienda a su espiritualismo vaporoso; Teresa, si no podía recrearse en la
meditación solitaria, dejaba libre el pensamiento en las ocasiones en
que era más esclava, y hablando con este y con el otro se recogía en el
sagrado de su alma para mirarse en ella\ldots{} Dice la Historia
psicológica que la guapa moza cayó en grandes tristezas por aquellos
días de Diciembre del 58; que sus esfuerzos para disimular las murrias
que la devoraban casi le costaron una enfermedad. En las fiestas de
Navidad, el bullicio y alegría de la gente la mortificaban; las personas
que a su lado veía constantemente, \emph{el contratista} sobre todo,
éranle odiosas. Para aislarse, exageró sus leves indisposiciones,
quedándose en cama no pocos días. Risueño no abandonaba por acompañarla
su sociedad de caballistas, ni el recreo de las innumerables amistades
que endulzaban su existencia. Cuenta también la Historia íntima que una
tarde que Facundo tenía gran cuchipanda con sus amigos en la Alameda de
Osuna, Teresa se echó a la calle, de trapillo, y se fue a casa de
Jerónima, donde le dijeron que Tuste no iba más que a comer y a dormir;
que aún no había encontrado colocación; pero que en tanto, se había
puesto a aprender el oficio de armero en el taller de un amigo. ¿Dónde?
En las Vistillas. Allá se fue Teresa, movida de un irresistible anhelo
de hablar con Tuste, de oírle sus poéticos disparates y de contarle ella
sus intensísimas tristezas, que sin duda tenían por causa un error
grande de la vida: el haber equivocado los caminos de la felicidad. No
le había dado Jerónima, por ignorarla, la dirección exacta del taller
donde Tuste trabajaba; pero ya lo encontraría preguntando, y al entrar
en las Vistillas puso atención a los ruidos del barrio, esperando
escuchar el son vibrante de los martillos sobre el yunque, o los
chirridos de las limas raspando el metal. Nada de esto oyó. Viendo al
fin en una tienda negrura y aparatos de ferrería, pero ningún hombre que
trabajase, interrogó a una mujer que sentada en la puerta estaba. «Sí,
señora, es aquí: pero el maestro armero y el aprendiz no están; se han
ido a la compostura de unas máquinas. Si quiere la señora saber cuándo
vendrán, pregúntele a la maestra\ldots{} ¿Ve aquella mujer que está
sentadita en un sillar dando de mamar a su niño? Pues es la maestra.»

Vio Teresa desde lejos a la mujer señalada: se distinguía de las otras
dos, que en el mismo sillar se sentaban, por ser más joven y tener
chiquillo en brazos. Fuese allí derecha. Al verla llegar, las tres se
sobrecogieron y se levantaron, pues aunque Teresa iba vestida con la
mayor sencillez, su aire señoril en nada se desmentía. A la urbanidad de
las pobres mujeres correspondió la Villaescusa con amable sonrisa,
mandándolas sentar; y poniendo su mano cariñosa en el hombro de la que
amamantaba, le preguntó\ldots{} La pregunta no llegó a ser formulada,
porque Teresa quedó suspensa a la mitad de la frase; miró a la mujer, se
apartó un poco, acercose luego como si quisiera besarla\ldots{}
dudó\ldots{} volvió a creer\ldots{} al fin no había duda\ldots{}
«¡Virginia!\ldots{} ¡Usted es Virginia!

---Sí, señora---dijo la otra, mirando y poniendo en su mirada toda la
memoria,---y usted es\ldots{} Conozco la cara; la cara no se me
escapa\ldots{} pero el nombre\ldots{}

---Soy Teresa Villaescusa. ¿No se acuerda usted? Éramos amigas\ldots{}
de esto hace algunos años\ldots{} No digo que tuviéramos gran intimidad;
pero nos conocíamos\ldots{} nos hablábamos\ldots{}

---Sí, sí\ldots{} Era usted más joven que nosotras\ldots{} me acuerdo
bien\ldots{} ¡Oh, Teresa! era usted entonces muy linda, y hoy\ldots{}
hoy más. ¿Quiere usted que subamos a mi casa?\ldots{} Es una pobre
casa\ldots{}

---No importa: vamos.»

\hypertarget{xxvi}{%
\chapter{XXVI}\label{xxvi}}

En el templo más hermoso y venerado no entraría Teresa con más respeto
que entró en la humilde casa de Virginia. Desnudas paredes vio, muebles
viejos en buen uso, cama, cómoda, cuna, en todo una pobreza
decorosamente conllevada, y un vivir modesto y sin afanes. Allí no había
nada bello ni superfluo; nada tampoco que indicase la penuria
angustiosa, la inquietud del día siguiente. La mano hacendosa se veía en
todas partes, y cierta entonación alegre de las cosas, en conformidad
con la claridad de la estancia.

Virginia, después de mostrar el chiquillo a Teresa, dormido ya, y de
dársele a besar, le acostó en la cuna. En un sofá de Vitoria con
colchoneta de percal encarnado, se sentaron las dos. El sol penetraba en
el aposento, dando a los objetos vigoroso colorido. «Mi casa es
pobre---fue lo primero que dijo Virginia;---¿pero verdad que es alegre,
muy alegre?

---Ya lo creo: más que la mía\ldots---afirmó Teresa, espaciando su vista
por todo.

---¿Cómo ha de ser este tabuco más alegre que su casa de usted\ldots{}
que será un palacio?

---No, hija, no\ldots---dijo la señora echándose a reír.---No es
palacio\ldots{} ¡quia!

---¿No tiene usted niños?

---No\ldots»

La pregunta referente a los niños envolvía en el espíritu de Virginia la
persuasión de que Teresa era casada. No podía ser de otro modo. Ninguna
soltera sale sola, y toda señora de aquel empaque tenía forzosamente
marido por la Iglesia. Debe decirse que las ideas de cada una frente a
la otra eran totalmente distintas. Teresa conocía perfectamente la
historia de \emph{Mita} y \emph{Ley}, y hasta los trabajos de Beramendi
con los Socobios para negociar las paces. En cambio, Virginia no sabía
nada de Teresa: entre la señorita que había visto y tratado en tiempos
remotos en alguna reunión, y la señora que tenía delante, había un
enorme vacío de conocimientos. Dígase también que Virginia, en su vida
salvaje, y después en aquel vivir apartado del trato de personas de
viso, había perdido toda la picardía mundana, quedándose en una
ingenuidad enteramente pastoril. Con sencillez digna de la Arcadia,
preguntó a la otra si era Marquesa.

«¡Marquesa yo! No, hija mía.

---Dispénseme: me he vuelto muy bruta. Lo he preguntado porque\ldots{}
Verá: alguna vez hablamos Pepe Fajardo y yo de la sociedad de mis
tiempos de soltera. Yo le pregunto: «¿Y Fulana\ldots{} y Zutana\ldots?»
Y él casi siempre me responde: «Es Marquesa.» Resulta que de poco tiempo
acá, todos los que tienen algún dinero son Marqueses, Condes o algo
así\ldots{} Por eso yo pensé\ldots{} Dispénseme.

---Sí, hija mía: la pregunta es de lo más natural\ldots{} Hay, en
efecto, sin fin de títulos de nuevo cuño, unos con dinero, otros
buscándolo\ldots{}

---Marquesa o no---dijo Virginia echando fuera toda su
ingenuidad,---usted es de esas damas de la Beneficencia que vienen a
estos barrios pobres a ver dónde hay miseria, para remediarla.

---Sí, soy benéfica---replicó Teresa confusa.---En otros barrios he
socorrido yo a muchos pobres\ldots{}

---Pues en este los hay también. Yo podré decir a usted dónde
encontraría grandes desdichas\ldots{} ¡y qué desdichas!

---Sí, sí\ldots{} pero no he venido a eso\ldots»

Al pronunciar esta frase, contúvose Teresa bruscamente, invadida de un
sentimiento que participaba de la vergüenza y el temor. ¿Cómo decir que
su presencia en aquel barrio y en aquella casa no tenía otro móvil que
buscar a un hombre? ¡Ella, que despreciaba la moral corriente, como
desprecia el gran artista las formas comunes del amaneramiento, sentíase
cohibida, vergonzosa ante la pobre Virginia, que era sin duda la primera
de las inmorales! Habíala tomado Virginia por gran señora. ¿Qué pensaría
cuando la gran señora le dijese: «Vengo tras del aprendiz de armero que
está en tu casa?» Esta idea caldeó el rostro de Teresa, y la puso en
gran turbación. De alguna manera tenía que justificar su visita. ¿Qué
diría, Señor? Afortunadamente para Teresa, la desbordada ingenuidad de
Virginia la sacó de tan embarazosa perplejidad, señalándole este camino:
«Ahora recuerdo\ldots{} Me dijo Pepe no hace muchos días que algunas
señoras de la mejor sociedad se interesaban por mí en la cuestión que
traemos ahora mis padres y yo\ldots{} Mis padres quieren tenerme a su
lado; yo también lo deseo; pero exigen que sacrifique a\ldots{}

---Ya sé\ldots{} Estoy bien enterada\ldots{} Y ese sacrificio es
imposible---afirmó Teresa, gustosa del pie que le daba la otra para
fundamentar racionalmente su visita.---En mí tiene usted una partidaria
acérrima\ldots{} Buena es la ley; pero cuídese mucho la ley, digo yo, de
no pisotear los corazones.

---¡Ay\ldots{} también yo digo eso!---exclamó Virginia suspirando
fuerte.---Los corazones por encima de todo\ldots{} No me engañaba el mío
cuando la vi llegar a usted. Usted no me conocía\ldots{} preguntaba por
mí a las vecinas\ldots{} quería informarse\ldots{}

---Informarme, sí, Virginia. Yo he dicho a Pepe que dada la testarudez
de los señores de Socobio, no hay más que una solución\ldots{} Solución
propiamente no es\ldots{} Yo le indiqué a Beramendi, y convino en ello
conmigo, que a falta de solución se arreglaría un\ldots{} Ya no me
acuerdo cómo se llama eso\ldots{} Es un término en latín\ldots.
\emph{modus vivendi}\ldots{} o cosa tal\ldots»

En aquel momento, dos jilgueros aprisionados que formaban parte de la
familia, y habían sido puestos al sol, jaula sobre jaula, en el batiente
de uno de los balcones, rompieron a cantar con tal algarabía de trinos,
que las mujeres tenían que alzar la voz para entenderse. Gustaba Teresa
de aquella música, que cubría su propio acento, permitiéndole ser poco
explícita en lo que hablaba. La idea del \emph{modus vivendi} no era
invención suya para salir del paso. Del asunto de Virginia se habló días
antes en su casa, de sobremesa; pero no recordaba bien Teresa lo que
Pepe Fajardo había dicho de la solución o arreglo provisional que
pensaba proponer a los Socobios; mas obligada, por su equívoca situación
en la visita, a manifestar algo concreto sobre aquel punto, apeló a su
imaginación, y entre el estruendoso cantar de los pájaros, como otro
pájaro que también cantaba, salió, a su parecer airosamente, por este
registro: «El arreglo consiste en que sus padres le señalen a usted una
cantidad para alimentos, que por el pronto debe ser corta, lo preciso y
nada más. Irán ustedes a vivir a casa del Marqués de Beramendi, en un
pisito que tiene arriba, y que ahora está desocupado, pues la
servidumbre vive casi toda en el principal. La respetabilidad de la casa
será el mejor ambiente para la reconciliación que deseamos. Allí podrán
los señores de Socobio visitar a su hija, que ya parece que pierde gran
parte de culpa poniéndose bajo el techo de los Emparanes. Hay que ir
poquito a poco, amiga mía. Al principio, recibirá usted la visita de su
padres cuando no esté Leoncio en casa\ldots{} Será preciso para esto
fijar horas determinadas. Los papás se volverán locos de alegría con el
chiquillo, con su nieto\ldots{} Le devolverán a usted su cariño, y así,
día tras día\ldots{} podrá llegar el de la completa benignidad de esos
señores con toda la familia, con el propio Leoncio\ldots» Dijo esto
Teresa, y al concluir su inventada solución o \emph{modus vivendi}, vio
que la obra era buena, y descansó como Dios después de haber hecho el
mundo.

Oyó Virginia la donosa mentira, con intensa curiosidad primero, con
arrobamiento y grande admiración al fin, y acogió la propuesta de Teresa
como uno de esos maravillosos descubrimientos que después de conocidos
nos asombran por su sencillez\ldots{} «Pues sí que es un arreglo
magnífico, una idea preciosa\ldots---dijo cruzando las manos y
descruzándolas luego para coger una de las de Teresa y besarla.---¿Y
esto se le ha ocurrido a usted? Verdaderamente se interesa por
nosotros\ldots{} ¡Y ha venido a enterarme del arreglo\ldots! ¡Qué
idea!\ldots{} es la mejor, la única\ldots{} ¿Lo sabe ya Pepe? ¿Se lo ha
dicho usted a Pepe\ldots?»

Teresa, creyendo que no podía menos de afirmar, afirmó ligeramente con
la cabeza. Los pájaros cantaban ya con frenesí, alzando tanto sus agudas
voces, que Teresa no habría podido hacerse entender si algo dijese. Así
era mejor\ldots{} En aquel momento el chiquillo remuzgó en su cuna.
Acudió Virginia diciendo: «Estos diablos de pájaros me le han despertado
con su música\ldots{} Y creo yo que lo hacen adrede. El niño es su
amiguito, se vuelve loco con ellos, y cuando se me duerme ellos le
llaman, le dicen: «Ven, ven, rico; te estamos cantando, y no nos haces
caso\ldots» Cogió en brazos al niño, que malhumorado se restregaba los
ojos con los puños, y prosiguió hablándole así: «Te han despertado estos
parlanchines\ldots{} Es que quieren charlar contigo, mi sol. Ven, ven, y
diles tú cosas; diles cosas\ldots»

Cogió Teresa el chiquillo, que no la extrañó; antes bien, se dejó
zarandear por ella frente a los pájaros, desarrugando el tierno ceño, y
con sus manos gordezuelas quería tocar las jaulas en que sus amiguitos
trinaban desaforadamente. Virginia, en tanto, mirando a su hijo en
brazos de Teresa, y a ésta gozosa, apuntándole al niño lo que tenía que
decir a los jilgueros en contestación a sus amantes clamores, entretuvo
algunos segundos con este ingenuo monólogo: «Buena señora es Teresa
Villaescusa\ldots{} Viene a verme y a contarme el arreglo que ha
inventado\ldots{} ¡Famosa idea!\ldots{} Pero yo digo: Teresa
Villaescusa, ¿quién es ahora? ¿Será la Navalcarazo, esa de quien tanto
se habla? ¿Será la Cardeña, esa de quien también se habla mucho? No, no
es ninguna de estas, porque me ha dicho que no es Marquesa. Será
entonces mujer de algún banquero, sin título ni corona. ¿Y qué clase de
amistad tiene con Pepe? ¡Oh, quién lo sabe!\ldots{} ¡Vaya, que no saber
yo con quién casó Teresita Villaescusa!\ldots{} Me da en la nariz que es
una de estas casadas ricas, a quienes Pepe corteja\ldots{} porque es
tremendo ese hombre\ldots{} El venir ella aquí a decirme lo que ha
discurrido, revela dos cosas: un gran interés por mí, una confianza
grande con Pepito\ldots»

El chiquillo, distraído un momento con los jilgueros, volvió los brazos
y el rostro hacia su madre, poniendo en sus lindas facciones un mohín
displicente. «Ahora pide teta---dijo Teresa.---Désela pronto\ldots{} que
si no, se va a incomodar con usted y conmigo.

---Ven acá, sol\ldots{} Es de lo más malo que usted puede
figurarse\ldots{} Mire, mire las carantoñas que me hace para que le dé
lo que tanto le gusta\ldots{} ¡Si será pillo\ldots! Me pasa la mano por
la cara, me mete los deditos en la boca\ldots{} y luego, véale
usted\ldots{} va derecho a sacar lo suyo\ldots{} ¡Ah, ladrón\ldots!»

En esto, ruidos que venían del piso bajo, voces confusas de personas y
chocar de hierros, anunciaban que habían llegado el maestro y el
aprendiz. Al entenderlo así, sacó Teresa de su cacumen otra donosa
invención, que debía cubrirle la retirada y permitirle realizar el
propósito que la llevó a las Vistillas. «Yo me voy---dijo, afectado la
inquietud de las graves ocupaciones olvidadas.---Esta casita humilde;
usted, Virginia, y su niño precioso, me han encantado\ldots{} El tiempo
se me ha ido sin sentirlo\ldots{} Ya no puedo entretenerme más.
Presénteme usted a Leoncio; quiero conocerle\ldots{}

---Le llamaré, le mandaré que suba.»

Antes que la maestra llamara, detúvola Teresa: «No, no. Le saludaré
abajo, al salir\ldots{} No le llame usted\ldots{} Él tendrá que hacer, y
yo me voy corriendo, corriendito\ldots{} ¡Dios mío, qué tarde! Pues
ahora tengo que ir a la calle de la Solana, que ni siquiera sé dónde
está.» Dijo Virginia que Leoncio la acompañaría, y ella, con rápida
visión de su plan estratégico: «No, hija\ldots{} Que venga conmigo el
chiquillo, el aprendiz.

---No es chiquillo, es un hombre.

---Lo mismo da. Bastará con que me indique el camino\ldots»

Bajaron\ldots{} Leoncio y Juan, ambos en traje de mecánica, con blusa
azul, la cabeza al aire, se quedaron como quien ve visiones ante la
mujer que iluminó el taller con su hermosura. Presentó Virginia al que
llamaba su marido, que invalidado por la cortedad no supo qué decir, ni
qué hacer con el cañón de escopeta que en la mano tenía: al fin lo dejó
sobre el banco. Palideció el aprendiz al ver la \emph{celeste
aparición}, y luego se puso muy colorado, permaneciendo en perfecta
inmovilidad, tan mudo como las tenazas que en la mano tenía\ldots{} Al
fin vio Teresa cumplido su estratégico plan de retirada tal y como lo
imaginara en rápida concepción. No había tenido poca suerte; que si se
empeña Leoncio en acompañarla, de nada le hubiera valido su ingeniosa
mentira. Cogió su manguito, despidiose afectuosamente del matrimonio
libre o liberal, llevándose al aprendiz para que en el laberinto de las
calles la guiase. ¡No era ella mal laberinto! El aprendiz la siguió
callado y respetuoso, y \emph{Mita} y \emph{Ley} quedáronse comentando
la visita, que tenían por venturosa, sin poder discernir quién era, en
la sociedad del 59, la que de soltera fue Teresita Villaescusa. Casada y
rica debía de ser; Marquesa no; amiga de Beramendi sí.

Hasta que entró con su guía en la calle de los Santos, no rompió el
silencio Teresa. Comprendiendo Tuste que su deidad tutelar quería
hablarle a solas, la desvió hacia la calle de San Bernabé. Tomó Teresa
un tonillo algo displicente para decirle: «Quiero saber por qué te has
metido en este oficio de armero sin decirme una palabra. ¿Acaso no soy
nadie para ti? ¿No merezco ya que me consultes todo lo que piensas?

---Usted lo merece todo, Teresa---replicó Juan.---Pero ¿cómo había yo de
consultar a mi bienhechora, si no la he visto en dos semanas?

---Estuve enferma. Ni siquiera has tenido la atención de ir a preguntar
por mí.

---Usted me prohibió que fuera a su casa y hasta que pasara por la
calle.

---Es verdad; pero ya debiste comprender\ldots{} En fin, Tuste, ¿por qué
te has metido en ese oficio sin decirme nada?\ldots{} Yo te hablé de
conseguirte un destino\ldots{} ¡Bonitas se te están poniendo las
manos!\ldots{}

---Pues yo\ldots---balbució Tuste, no sabiendo cómo aplacar aquel enojo
que no comprendía.---Verá usted\ldots{} Pensé que de este modo sería más
grato a la señora\ldots»

Teresa se plantó en medio de la calle, y con súbita energía le echó sus
dos manos a los hombros, diciéndole en un tono que lo mismo podía ser de
reconvención que de súplica: «Juan, la última vez que te vi te mandé que
no me llamases señora; yo no soy señora\ldots{} soy una mujer y nada más
que una mujer. Sigamos, y hablaremos andando\ldots{} Suprime lo del
señorío, vuelvo a decirte.» Antes de que Tuste pudiera formular sus
protestas de obediencia incondicional, volvió a plantarse Teresa, y con
el mismo tono que revelaba la firmeza de su voluntad, le dijo: «Tuste,
hasta me incomoda que me trates de usted\ldots{} Es ridículo que
hablemos tú y yo como se habla en las visitas. Tutéame\ldots{}

---¡Yo\ldots{} Teresa!

---Que me tutees, digo\ldots{} Yo lo quiero, yo lo mando.»

\hypertarget{xxvii}{%
\chapter{XXVII}\label{xxvii}}

«Bueno---dijo Tuste, guiándola hacia Gilimón.---Al llamarte de
\emph{tú}, me entra en el alma una frescura deliciosa\ldots{}
que\ldots{} No sé cómo expresarlo\ldots{} Tratándote de \emph{tú}, soy
otro, crezco, me agiganto\ldots{} Me siento capaz de las acciones más
hermosas, y hasta me parece que mi inteligencia levanta el vuelo\ldots{}
Teresa, ¿qué ideal sublime se encarna en ti?

---Tuste, dime, dime esas cosas, aunque sean mentira, y bien sé que lo
son\ldots{} Antes me daba de cara la poesía, y ahora no.

---Pues déjame que te cuente cómo me metí en este aprendizaje sin
consultar contigo. Pasados dos o tres días desde la última vez que te
vi, me encontré a Leoncio, que es amigo mío: le conocí cuando Jerónima
tenía la casa en Mesón de Paredes\ldots{} Me dijo: «Si quieres aprender
el oficio de armero, yo te enseñaré.» Le respondí que lo
pensaría\ldots{} Pues aquella noche soñé contigo, Teresa, como todas las
noches\ldots{} Te me apareciste coronada de rosas, vestida de un blanco
traje que relucía como plata, los pies con zapatos azules\ldots{}

---Estaría bonita\ldots{}

---Alargaste un pie y me dijiste que te descalzara\ldots{} así lo hice.

---Pues eso podía significar que aprendieras el oficio de zapatero.

---No, porque andando descalza en derredor de mí, me dijiste: «Vete con
tu amigo y que te enseñe a construir las bonitas armas\ldots{} armas con
que matar a los egoístas, a los perversos\ldots» Teresa, me puedes creer
que vi y oí todo esto como si fuera la misma realidad\ldots{} Al
siguiente día tuvo tal fuerza en mí la idea de ponerme a trabajar con
Leoncio, que no vacilé un momento más. Tú me lo dijiste, me lo mandaste.
Yo te veía como te estoy viendo ahora, puedes creerlo\ldots{}

---¿Y en las noches siguientes no me viste también?

---Sí\ldots{} venías a mi lado, tal como estás ahora\ldots{} Yo callaba;
tú me decías: «Juan, abandona la idea de seguir estudiando Filosofía y
otras garambainas que nunca te sacarán de pobre. No pienses en destinos
del Gobierno, que no son más que pan para hoy y hambre para
mañana\ldots{} Métete en el comercio; compra y vende patatas, fruta,
madera, cal, huevos, cualquier cosa; aprende un oficio; ponte a hacer
cosas, a fabricar algo, jabón, ladrillos, clavos, peines, velas, relojes
o demonios coronados\ldots{} el cuento es que ganes dinero\ldots»

---¿Eso te decía? ¿Pues sabes que esas noches estaba yo muy prosaica,
después de aquella otra noche en que me llegué a ti con zapatos azules?

---Prosaica estuviste, y yo, que siempre fuí rebelde al prosaísmo, me
sentí tocado del tuyo. ¿Por ventura, digo yo, tus consejos prosaicos no
eran la quinta esencia de la poesía?

---Es fácil que sí, Juan\ldots{} Dime: ¿y cuando te aconsejaba que
comerciaras o aprendieras un oficio, cómo iba yo calzada?

---De ninguna manera, porque venías a mí con los pies desnudos.

---¡Ay, Juan! eres un soñador tremendo\ldots{} Ten cuidado\ldots{}

---¿Cómo no soñar estando a veces cerca de ti, a veces tan lejos como lo
estoy de las estrellas? Teresa, si después de lo que te he dicho,
encuentras mal que yo aprenda el oficio de armero, lo dejaré\ldots{} Tú
mandas.

---No, Juan, no: si hace un rato te reñí por esto, fue\ldots{} qué sé
yo\ldots{} Tenía yo ganas de pelearme contigo\ldots{} el motivo
importaba poco\ldots{} la cuestión era decirte cosas\ldots{} que tú me
las dijeras a mí\ldots{} Ya no te riño por lo que has hecho. Déjate
llevar de tu inspiración. Puede que esto sea el principio de una gran
fortuna para ti\ldots{} Juan, busca donde nos sentemos, que yo estoy tan
cansada como si hubiera andado leguas\ldots{} Allí veo una piedra grande
que es como un banco. Vamos allá.

---Vamos\ldots{} siéntate\ldots{} Estoy pensando una cosa: Leoncio y
Virginia dirán que tardo mucho en llevarte a la calle de la Solana;
¿pero qué nos importa?

---Dices bien: ¿qué nos importa?\ldots{} Has medido el tiempo que debías
tardar en volver a tu taller, y en eso, Juan, demuestras ser más
prosaico que yo, que de tal cosa no me acordaba.

---Pero medí ese tiempo, Teresa, sin que se me diera cuidado de que
fuera largo. Obedeciendo a mi corazón, yo entraría en casa de Leoncio
diciendo: «Esa mujer es para mí más hermosa que los ángeles, más alta
que las estrellas, y más benigna y generosa que la propia Caridad que
Dios envió al mundo\ldots»

---¡Ay, ay, ay, Juanito\ldots! ¡Cómo se reirían de ti Leoncio y su mujer
si entraras diciendo eso\ldots! Esos disparates no debes decirlos más
que a mí. Aun sabiendo lo mentirosos que son tus dichos, me gusta
oírlos, Juan. Hoy es día de libertad, casi casi de embriaguez. Sigue,
Juan, sigue. Las almas cogen un día cualquiera y hacen en él su
carnaval.

---No son mis dichos mentirosos, sino la verdad misma---afirmó Tuste
fundiendo su mirada en la mirada de ella,---sino la misma verdad. Cosas
y personas no son lo que ellas creen ser, sino lo que son en el alma del
que las mira y las siente.

---Quiere decir eso que yo puedo ser para los demás lo que quieran; pero
que para ti siempre seré como debo ser\ldots{} No sé si lo entiendo
bien, ni cómo se ha de explicar esto.

---Para mí, las denominaciones de señora y caballero son motes que este
y el otro gustan de ponerse en un juego social parecido al de las cuatro
esquinas\ldots{} Yo no pongo motes; no clavo tampoco letreros infamantes
en la frente de ningún ser humano. Cristo me ha enseñado el perdón; la
Democracia me ha enseñado la sencillez, la igualdad\ldots{} Yo miro al
alma, no miro a la ropa.»

Dicho esto por Tuste, ambos callaron. Había Teresa encontrado en el
banco unas briznas secas, despojo de un árbol plantado no lejos de allí.
El árbol, nada robusto y con su ramaje en completa desnudez, daba sombra
al banco en estío; en invierno soltaba sobre él y sobre el suelo próximo
los residuos muertos de su verdor pasado, para dar lugar a los nuevos
brotes. Cogió Teresa dos, tres o más de aquellas briznas y se las llevó
a la boca: eran amargas, pero el amargor no la desagradaba. La pausa que
hicieron Teresa y Juan en su diálogo fue larguísima: él, apoyando en las
rodillas los codos, miraba al suelo; ella, teniéndole a su izquierda,
volvió su rostro hacia el opuesto lado, y clavaba sus miradas en una
larguísima y fea pared que como a veinte pasos se extendía, triste
superficie con letreros pintados anunciando alguna industria, y otros
escritos debajo con carbón por mano inexperta\ldots{} Sobre aquellas
letras y garabatos dejaba correr sus ojos Teresa sin ver nada, sin darse
cuenta de lo que allí estaba escrito\ldots{} El lugar o fondo de la
escena no podía ser más prosaico; el suelo era todo polvo. La pared
escrita limitaba el espacio por esta parte; por aquella, a distancia de
pedrada corta, una fila de casas pobrísimas\ldots{} Como seres vivos,
daban animación a tan feo lugar perros flacos, chiquillos sucios y
mujeres desmedradas.

De improviso volvió Teresa el rostro chocando su mirada con la de Tuste,
y mordiendo los amargos palitos le dijo: «Según eso, Juan, podrás tú
quererme a mí, sin llamarme ángel, ni diosa, ni nada de eso, queriéndome
con todo lo que tengas en tu alma, sin acordarte para nada de lo que fuí
ni de lo que soy\ldots»

Abrumado Tuste por la gravedad de esta proposición, que le cogió algo
desprevenido y despertó en su alma un furioso tumulto, no tuvo arrestos
para resistir la mirada de Teresa; bajó los ojos, y pasando el dedo por
la superficie áspera y polvorosa del sillar en que se sentaban, iba
soltando con lentitud la respuesta, como si anotara las palabras, o si
leyera lo que escribía. Fue así: «¡Quererte yo, Teresa! ¡Si desde que te
vi y me socorriste, te estoy queriendo, más que con amor, con adoración!
Yo no veo en ti señora, ni veo la que otros hombres llamaron o llaman
suya. Para mí, tú no eres de nadie, no puedes ser de nadie\ldots{} Yo no
he mirado a tu cuerpo tanto como a tu espíritu. ¿Por qué te he llamado
ángel? Porque he visto en ti el corazón generoso, la frente noble, y las
alas para subir a donde no subió ninguna mujer\ldots{} Cuando supe tu
pasado y la vida que hacías, lloré y rabié lo que no puedes
figurarte\ldots{} Pero luego mis ideas separaron tu espíritu de toda la
broza material, y limpia y purificada te guardé en el sagrario de mi
corazón, donde te doy culto con el espíritu mío. ¡Que si te quiero,
Teresa! El amor mío por ti es grande, como todo amor que ha nacido y
crecido sin esperanza\ldots{} El no esperar nada, aviva el fuego de
amor. Si me despreciaras, te querría lo mismo\ldots{} Queriéndome tú, lo
mismo que si no me quisieras\ldots{} ¡Vivir, amar, morir!\ldots{}
términos absolutos\ldots{}

---¿Me amarás de veras?---dijo Teresa en lenguaje llano.---¿Como yo a
ti?

---No me lo preguntes con palabras, sino con la imposición de algún
sacrificio, o sometiéndome a la prueba más terrible. ¿Que es preciso
morir por ti? Dímelo, y verás qué pronto\ldots»

Siempre que Tuste hablaba este lenguaje de vaporosa espiritualidad,
Teresa se conmovía y se le aguaban los ojos. Aun en los casos en que las
declamaciones de su amigo la movían a risa, no dejaba de sentir emoción,
y confundía la risa con el llanto. Aquella tarde hubo de extremar el
esfuerzo de su voluntad para contener las lágrimas. «Oye, Juan---dijo
después de una corta pausa:---vete a tu maestro y a tu maestra,
diles\ldots{} cuéntales esto. ¿Sabes cómo se nombran en la intimidad
Leoncio y Virginia? \emph{Mita} y \emph{Ley}. Pues diles que te quiero y
me quieres. No se asombrarán poco, y\ldots{} ¡quién sabe! puede que no
se asombren nada.

---Virginia y Leoncio se quieren y hacen de sus dos almas un alma sola.

---Pregúntales por su vida salvaje\ldots{} que te cuenten cómo fueron a
esa vida, y la felicidad que tuvieron en ella.

---No están unidos más que por la ley de sus corazones.

---Sus corazones fueron la ley\ldots{} Pregúntales cómo han vivido, cómo
han soportado las penas\ldots»

El recuerdo de \emph{Mita} y \emph{Ley} determinó repentinamente en
Teresa un estado de espíritu semejante al que tuvo en la entrevista con
Virginia. Sintió vergüenza y miedo. ¿Qué pensaría Virginia si supiera
que había sacado del taller con engaño al aprendiz para irse con él de
bureo por las calles? Esto era incorrectísimo. ¿Por quién la tomaría
Virginia, después de haberla tomado por señora y hasta por Marquesa? No,
no podía soportar los juicios desfavorables de \emph{Mita}\ldots{} Veía
en ella, ¡qué cosa tan rara! la cifra y compendio de la moralidad. La
salvaje había venido a ser como una personificación de toda la virtud
humana.

«Tuste---le dijo levantándose resueltamente, llena su alma de un
sentimiento de pudor,---no quiero que \emph{Mita} y \emph{Ley} piensen
mal de nosotros\ldots{} No está bien que les engañemos. Dirán que para
enseñarme dónde está esa calle has empleado mucho tiempo.

---Sí: pensarán mal\ldots{} y no quiero yo eso.

---Pues vete pronto, Juan, vete. Si te riñen por la tardanza, cuéntales
la verdad\ldots{} les dirás quién soy: ¡qué vergüenza!\ldots{} Pero sí,
deben saberlo\ldots{} Anda, no tardes. Yo también me entretengo
demasiado. Todavía no soy libre.

---Les diré la verdad, la verdad. Quizás me conozcan en la cara que tú
me quieres, y nada tendré que decir.

---Quizás, Juan\ldots{} ¿Y a mí me lo conocerán también en la cara? No:
yo sé disimular\ldots{} Es preciso que nos separemos.

---Se separan nuestras personas, como dos sombras que han estado
confundidas en una; pero tu espíritu y el mío permanecen juntos, juntos
como un solo espíritu.

---Como un solo espíritu\ldots{}

---Adiós. Déjame esas briznas que llevas en tu boca.

---Tómalas. Máscalas un poco, y las guardas luego.

---Son amargas. Toda la vida es amarga; pero contigo el amargor es
dulzura. Teresa, déjame besar tu mano\ldots{} Así\ldots{} Déjame ahora
que meta mi mano en tu manguito para que se me pegue el calor de las
tuyas.

---Así\ldots{} deja tu mano metida aquí un ratito, para que te lleves el
calor\ldots{} Vaya, no más.

---No más. Adiós\ldots{} yo me voy por aquí.

---Yo por aquí\ldots{} Adiós.»

Desde lejos, embocando diferentes calles, uno y otra se pararon para
saludarse. Alzaba ella la mano en que tenía el manguito, él la suya sin
nada. No llevaba bastón ni sombrero. Por fin, cada calle se llevó lo
suyo, y entre los dos aumentaba el oleaje de calles, de
transeúntes\ldots{}

Fue Teresa por todo el camino hasta su casa en completa abstracción de
cuanto pudiera apreciar por los sentidos. Toda su compañía la llevaba
dentro. Al llegar a su casa, ya de noche, la primera pregunta que hizo a
Felisa fue: «¿Ha venido \emph{ese}?\ldots» Diciendo \emph{ese} chocó con
la realidad\ldots{} Se sentía fatigadísima; le dolían los pies de andar
por calles empedradas con guijarros puntiagudos. Despojada de mantilla y
zapatos, se tendió en un sofá, y a solas, recordando y pensando, su
cerebro entró en blanda sedación. De la calle traía, para decirlo claro,
una borrachera de espiritualidad. ¿Pero todo lo ocurrido en aquella
tarde, la excursión a las Vistillas, la entrevista con Virginia, la
conversación con Tuste, era verdad? ¿Estaba ella en su cabal sentido
cuando dijo al aprendiz lo que aún recordaba, pues cada palabra suya,
cada palabra de él, estampadas permanecían en su mente?\ldots{} ¿No se
había excedido un poco en el abandono de su voluntad, comprometiéndose a
más de lo que debiera?\ldots{} En estos pensamientos se mecía y aun se
adormecía, cuando un fuerte ruido de voces y de pisadas en la escalera
le anunció que volvía Risueño con los expedicionarios de la Alameda de
Osuna. No sintió Teresa que Facundo viniese acompañado, trayendo a cenar
a dos o más amigos, pues cuando venía solo eran más difíciles de
conllevar las brusquedades e impertinencias del \emph{contratista}.

Entraron en la casa con alegre vocerío Risueño y otro señor, luciendo el
empaque andaluz de marsellés y calañés, y dos más en traje corriente de
caballeros que van de campo. Estos eran el Marqués de Beramendi y José
Luis Albareda, el más arrogante, salado y ceceoso de los señoritos
andaluces que por entonces se abrían camino en la política. Dirigía
\emph{El Contemporáneo}, órgano de los conservadores que llamaban
\emph{de guante blanco}, los más atildados y conspicuos, chapados a la
inglesa, que era la \emph{dernière} en punto a política y arte
parlamentario.

Hubo Teresa de violentarse para sonreír a toda la cuadrilla, y disertar
con ella de asuntos que no la interesaban. Allí estuvieron hasta media
noche, charlando, contando cuentos andaluces, consumiendo la manzanilla
y otras bebidas de que tenía Risueño grande acopio en aquella casa.
Resistiéronse a cenar: tan repletos venían del comistraje en la Alameda;
pero bebían, algunos moderadamente, otros empinando de lo lindo, sin
embriagarse o sólo poniéndose alegres y decidores. Risueño cogió la
guitarra, y tras un preludio de ayes y jipidos lastimosos, se arrancó el
hombre con playeras. No lo hacía mal: alguno le jaleaba con palmaditas;
Beramendi tenía más sueño que ganas de música. Las doce serían cuando
desfilaron dos, quedando solos Facundo y el compañero que, como él,
traía ropa y sombrero al estilo de la tierra de María Santísima. Era un
caballero joven a quien las aficiones a la jácara y a las cañitas no
privaban de la exquisita distinción en sociedad. A todo hacía, mostrando
igual superioridad en ambos papeles; era el primero en las zambras
andaluzas, el primero en la cortesanía que podremos llamar europea,
terreno común de la civilización. Disputaron un rato Risueño y Manolo
Tarfe (que tal era el nombre del caballero), sobre si braceaban los
potros cordobeses mejor que los jerezanos o viceversa; pero ello quedó
en las primeras escaramuzas, porque Risueño fue bruscamente acometido de
tan intensa modorra, que con media palabra entre los labios, dejó caer
al suelo la guitarra, y su cuerpo se estiró en el sofá, convirtiéndose
en plomo. Tarfe le llamó con fuertes gritos, como podría llamarle si se
hubiera caído en un pozo\ldots{} «¡Pobrecito!---dijo Teresa, gozosa de
ver anulada la personalidad de su \emph{contratista},---hay que dejarle
dormir la manzanilla.» Entre ella y Felisa le sacaron con no poca
dificultad las botas\ldots{} El marsellés no pudieron quitárselo, por la
extraordinaria pesadumbre del cuerpo de Risueño.

Cogió en esto el caballero Tarfe su calañés para retirarse, y haciendo
ademán de poner el gracioso sombrero andaluz en la cabeza de Teresita,
le dijo: «Ahí te dejo con ese fardo\ldots{} Mejor para ti que se haya
convertido en lo que ves, en un saco de patatas\ldots»

Solía tutear a Teresa, viéndola sola, por arranque nativo de su
temperamento, y por expresarle mejor sus atrevidas pretensiones. Tiempo
hacía que la enamoraba con disimulo, aprovechando toda buena coyuntura
para convencerla de que debía entenderse con él, rescindiendo la
contrata con Risueño. Pero Teresa, blasonando de virtud relativa, no
daba oídos a la sugestión del caballero, y se mantenía leal a su
compromiso. Sin esperanza de ser más afortunado aquella noche, Tarfe,
cuando Teresa salió a despedirle hasta la sala, dejando en el gabinete
el inanimado cuerpo de Facundo, que más bien cadáver parecía, le soltó
la milésima declaración y propuesta de amores. Echose a reír la guapa
moza; pero más benigna que otras veces, deslizó una frase de esperanza.

«Manolito, tenga paciencia\ldots{}

---¿Te decides a despachar al fardo?

---Paciencia, Manolo.

---Di una palabra\ldots{} ¿Quieres que hablemos\ldots?

---Sí: algo tengo que decir a usted.

---¿Dónde podríamos\ldots? Teresa, no me engañes. Has dicho que tienes
algo que decirme\ldots{} Pues aquí mismo\ldots{} Cuenta que Facundo es
una pared.

---Las paredes oyen\ldots{} Aquí no puede ser.

---¿Pues dónde, cuándo? ¿Me citarás?

---Sí: para ponerle a usted banderillas.

---No bromees. ¿Me citarás?\ldots{}

---Que sí\ldots{} Y basta. Tome el olivo pronto.

---Bueno: me voy. Pero quedamos en que me citas, Teresa.

---Para que hablemos\ldots{}

---Eso, para que hablemos. ¡Si es lo que deseo: hablar contigo! Adiós,
gracia del mundo.»

\hypertarget{xxviii}{%
\chapter{XXVIII}\label{xxviii}}

Se ignora el día, el mes no es seguro\ldots{} ello pudo ser en Febrero,
en Marzo del 59, cuando en todo su apogeo lucía su espléndido plumaje
nuevo la Unión Liberal, empollada por el gran O'Donnell. La indecisión
de la fecha no quita valor histórico a la comida con que los Marqueses
de Villares de Tajo obsequiaron a sus amigos. Estos eran cuatro: el
Marqués de Beramendi, Manolo Tarfe, Nocedal y el pomposo Riva Guisando,
\emph{la première fourchette de Madrid}. Asistían también a la comida
don Serafín del Socobio y su hija Valeria; pero como el pobre señor
estaba medio paralítico, no gustaba de sentarse a la mesa grande,
temiendo el desagradable espectáculo que a los convidados daba con su
torpeza para el manejo de cuchillo y tenedor. En una estancia próxima le
habían puesto una mesita, donde Valeria le acompañaba, le partía la
comida cuando era menester, y se la iba metiendo en la boca con tenedor
o cuchara.

Bromeaba Eufrasia graciosamente con Guisando, diciéndole que su
patriotismo le ordenaba la proscripción de todo estilo francés en su
cocina. Mal día le esperaba al \emph{gourmet}. Afirmaba Guisando que él
era internacional, y que adoraba los buenos platos españoles
condimentados \emph{secundum artem}. Dejándose llevar de su galantería,
llegó a decir que mantenidos en España los buenos principios
gastronómicos de la raza, él sería el primer enemigo de la invasión, y
pondría en todas las cocinas una copia en pastaflora del grupo de Daoiz
y Velarde. Don Saturno del Socobio, que estaba ya casi lelo, no decía
más que: «España es la primera nación del mundo por el valor y por la
sobriedad. ¿Qué mayor gloria para un país que vivir sin comer? Los
españoles han hecho en ayunas su brillante historia.» Apoyó esto
Nocedal, diciendo que España no había cultivado nunca las artes que no
eran espirituales, y que entre todas las filosofías había preferido el
ascetismo, que resuelve de plano y sin quebraderos de cabeza la cuestión
de subsistencias. La Economía Política que a la sazón estaba tan en
boga, era desconocida de los españoles del gran siglo\ldots{} La
decadencia empezó cuando entraron las ideas económicas\ldots{} La vida
española es, por naturaleza, vida de inspiración, vida de hazañas en la
esfera humana, y de milagros en la esfera religiosa\ldots{} Es España la
cristalización del milagro: vivir sin trabajar, trabajar sin comer,
comer sin arte y hacer una historia que así revela el poder de las
voluntades como el vacío de los estómagos\ldots{} Con estas paradojas a
los comensales entretenía, y corroboraba su fama de decidor agudo.

Beramendi preguntaba si había noticia histórica de cómo se alimentaban
el Cid, Nuño Rasura y Laín Calvo, y Guisando afirmó que estos caballeros
no comían más que pan y sus derivados; migas o sopas de ajo, caza y
cotufas, con lo que se podía componer un excelente \emph{Timbale de
perdreaux aux truffes}\ldots{} Don Saturno, reiterando su patriotismo,
sostuvo que la cocina francesa era una alquimia indecente y una
botiquería repugnante, y Tarfe se declaró internacional como Guisando,
preconizando la concordia y armonía entre los dos sistemas culinarios,
tomando lo bueno de uno y otro para formar lo excelente y superior;
vamos, una verdadera Unión Liberal del comer.

¡Unión Liberal! Estas mágicas palabras llevaron la conversación a la
comidilla política, que era la más incitante y sabrosa.

«Tiene razón Tarfe---dijo Eufrasia.---¿Qué es la Unión Liberal más que
una mixtura de sistemas gastronómicos?

---Trátase de un sistema político---apuntó Nocedal,---que no tiene más
que un dogma, o si se quiere, tres dogmas: comer, comer, comer.

---Trátase, señor don Cándido Nocedal---dijo Tarfe, el más convencido y
frenético de los unionistas,---de traer a España la vida nueva y grande,
la vida del progreso, de la cultura, y poner fin a la política sectaria
y facciosa.»

Apoyado por Beramendi, hizo Manolo Tarfe el ardiente panegírico de la
Unión y de su creador y jefe don Leopoldo O'Donnell. Con ser profunda la
fe del caballero en las excelencias del nuevo partido y en las venturas
que al país traería, más que la fe en las ideas le alentaba su amor y
respeto al Conde de Lucena y a toda su familia. Por el General tenía
verdadera adoración; con tal vehemencia ponderaba su valor, su talento y
su sencillez y bondad, que en el Casino solían llamarle \emph{O'Donnell
el Chico}. Provenía más bien este apodo de su estatura, harto menguada
en comparación con la de su ídolo, y de su semejanza fisonómica con
personas de la familia del General. Era rubio, de azules ojos,
simpático, y de hablar expedito y donoso. Rico por su casa, Tarfe quería
lucir en el terreno político, y no carecía de bien fundadas ambiciones.
Ya era diputado, y con la protección de O'Donnell sería todo lo que
quisiese. Su frivolidad y los hábitos de ocio elegante en los altos
círculos, o en los pasatiempos y deportes andaluces (pues esta doble
naturaleza era en él característica), se iban corrigiendo con el trato
de personas graves y con la constante proyección de la seriedad de
O'Donnell sobre su espíritu. No era un derrochador como Aransis, ni
había llegado al reposo y madurez de juicio del gran Beramendi. Algunos
le tenían por \emph{cuco}, y veían en sus jactanciosas actitudes, dentro
de las dos naturalezas, un medio de hacerse hombre y de abrirse camino
en la política.

Ameno y fácil hablador, \emph{O'Donnell el Chico} se disparaba en la
conversación, estimulado por su propia facundia y por el agrado con que
le oían. Decía la Villares de Tajo que no había caja de música más
bonita y menos cansada que Manolo Tarfe, y siempre que a su mesa le
tenía, dábale cuerda, variándole al propio tiempo la tocata. Todas las
de \emph{O'Donnell el Chico} iban a parar siempre a la exaltación y
apoteosis de la Unión Liberal. Nocedal y don Saturno creyeron que Tarfe
deliraba o se ponía peneque cuando le oyeron decir: «Don Leopoldo es el
primer revolucionario, porque al par de los derechos políticos para
todos los españoles, trae los derechos alimenticios. Viene a destruir la
mayor de las tiranías, que es la pobreza. Su política es la regeneración
de los estómagos, de donde vendrá la regeneración de la raza. Sin buenos
estómagos, no hay buenas voluntades ni cerebros firmes. De Mendizábal
acá, nadie ha pensado en que España es un pobre riquísimo, un vejete
haraposo, que debajo de las baldosas del tugurio en que vive tiene
escondidos inmensos tesoros\ldots{} Pues O'Donnell levantará las
baldosas, sacará las ollas repletas de oro, y con ese oro, que es a más
de riqueza talismán, le dará al vejete unos pases por todo el cuerpo, a
manera de friegas, devolviéndole la juventud, la fuerza física y
mental.»

Tronó don Saturno contra esto; Eufrasia y Beramendi rieron; Nocedal, más
desdeñoso que indignado, dijo que la figura podía pasar, pero que la
idea era detestable y masónica. La palabra Desamortización corrió de
boca en boca, y en la de Riva Guisando provocó esta opinión escéptica:
«Hágase la prueba\ldots{} Sáquese del subsuelo un poco de pasta, dénsele
las friegas al vejete\ldots{} véase qué cara pone, y si le entusiasma la
idea de recobrar la juventud\ldots{} Porque si después de desamortizar
salimos con que el viejo requiere sus andrajos y clama por que no le
quiten de la cara sus benditas arrugas, no hemos hecho nada\ldots»

Y tal alboroto levantaron las ideas de Tarfe, que hasta la salita donde
comía don Serafín llegó el eco de los apóstrofes, réplicas duras y
burlonas risas. El pobre señor se afligió enormemente cuando Valeria le
dijo que hablaban de Mendizábal y de la Mano Muerta, y con la suya, que
no estaba muy viva, dio sobre la mesa no pocos golpes, diciendo: «Tarfe
masón\ldots{} Perdónele Dios.» Tan excitado se puso, que Valeria pasó al
comedor para rogar que se variase la tocata.

«¿Qué hay, hija mía?

---Papá está furioso por lo que dice Manolito Tarfe. Manolito, haga el
favor de no ser aquí tan masónico.

---¿Qué ha dicho mi buen amigo don Serafín?

---Que toda política que va contra Dios, es una política infernal.

---No he dicho nada\ldots{} Valeria, aunque venga usted en clase de
inquisidora, nos alegramos de verla.

---No nos abandone, Valeria. Está usted monísima; nos embelesa su
rostro; su mirada y su sonrisa nos encantan, aunque vengan cargadas de
anatemas y excomuniones.»

Halagada en su vanidad por tales piropos, dijo Valeria que no podía
separarse de su papá, pero que aprovecharía cualquier ocasión para dar
un saltito al comedor y echar un palique con los buenos amigos, siempre
que estos prometieran ser muy poquito herejes y muy poquito masónicos.
La ocasión para zafarse del cuidado de don Serafín, y campar un rato a
sus anchas en el comedor, la determinó Fajardo, que cuando servían el
café se fue a tomarlo en compañía del paralítico, relevando a Valeria.
Esta voló al comedor, y solos el Marqués y don Serafín, ofrecieron a la
Historia una memorable conversación: «Mi noble amigo, de hoy no pasa que
usted me dé su conformidad con el plan que le he propuesto para el
perdón de Virginia\ldots{}

---¡Oh, Virginia, hija del alma!---exclamó Socobio lloriqueando, pues en
cuanto aquel tema se tocaba, sus ojos eran fuentes.

---¡Hija del alma, dice usted, y no le abre sus brazos!\ldots{} ¡Hija
del alma, y le niega su cariño, le niega el pan!\ldots{}

---El pan no\ldots{} Todo el sobrante que hay en casa, que no es poco,
será para ella\ldots{} ¡Hija mía\ldots{} tan pobre y lactando!\ldots{}
Yo le aseguro a usted que si Virginia criara dentro del santo
matrimonio, yo pagaría con gusto las mejores amas asturianas y
pasiegas\ldots{} Pero ella lo ha querido, ella rompió todos los lazos y
pisoteó todas las leyes\ldots{} Castigo de Dios: darás el pecho a tus
hijos, porque no tendrás dinero para pagar ama\ldots{}

---Virginia goza de buena salud, y no necesita alquilar la leche para su
hijo. Virginia es la mujer fuerte, la mujer que va derecha por el camino
de la vida.

---¡Ay! no, no, Pepe\ldots{} no me aflija usted más de lo que
estoy\ldots{} Vea, vea cómo corren mis lágrimas\ldots{} Ya tengo este
pañuelo que se puede torcer\ldots{} Pero traigo otro\ldots{} y otro.
Siempre que salgo de mi casa llevo tres pañuelos, porque me aflijo por
la menor cosa y\ldots{} ya ve usted\ldots{} Hágame el favor, Pepito, de
no disculpar a Virginia ni llamarla mujer fuerte. Podré perdonarla; pero
disculparla nunca\ldots{} Es la mujer débil, la mujer extraviada\ldots{}
Póngame usted más azúcar\ldots{} me agrada el café dulcesito\ldots{}
Pues no ensalce usted a Virginia, pecadora y adúltera; no la comparemos
con este ángel, con mi Valeria, la hija fiel, la hija discreta que ha
preferido las asperezas del deber a los deleites de la libertad\ldots{}
Ahí la tiene usted, casada honesta y viuda honestísima, que viudez
efectiva, aunque pasajera, es el alejamiento de su marido\ldots{} Ahí
está, firme en sus deberes, intachable en la virtud, ajustando
estrictamente su conducta a lo que dice San Dionisio Areopagita acerca
de la forma y manera con que han de guardar su recato las viudas. ¿Lo ha
leído usted?

---No, señor\ldots{} Pero sin leer nada de eso, reconozco que Valeria es
un modelo de viudas ocasionales y de amantes hijas.

---No tiene más que un defecto, que es su loco devaneo por los muebles
elegantes y las cortinas de última novedad\ldots{} Pero este defecto no
atañe a la virtud propiamente, ni la menoscaba. ¡Oh, qué diera yo por
que a Virginia no se le pudiera echar en cara otro pecado que el
mueblaje suntuoso y el gusto exagerado del vestir a la moda!\ldots{} Los
pecados de Virginia van contra Dios, son la negación de Dios y de su
maravillosa obra en la humanidad\ldots{} Yo lloro esos pecados, querido
Pepe; los lloro por ella, y los estaré llorando mientras viva\ldots{}

---Serénese un poco, don Serafín; tómese su cafetito, que está muy
bueno, y sin lloriqueos ni suspiros, deme su conformidad con el proyecto
de reconciliación\ldots{} ¿Quiere que le recuerde las bases? Usted
señalará a su hija pensión de alimentos, cantidad razonable, la que le
correspondería si no existieran estas discordias\ldots{} Virginia y su
familia vivirán en mi casa; podrán visitarla usted y doña Encarnación a
la hora que se determine para encontrarla sola con el chiquillo\ldots{}
¿No es esto lo tratado?

---Eso es\ldots{} déjeme que llore\ldots{} eso y algo más. Viéndome ya
tan caduco y de tan torpe andadura, propongo que puedan venir a mi casa
Virginia y su nene; pero nunca pretender vivir con nosotros\ldots{} De
su casa de usted vendrán a la mía, y de la mía volverán allá, sin que el
hombre en ningún caso les acompañe por la calle\ldots{}

---Muy bien. Mi mujer o yo nos encargaremos de la traslación\ldots{}
Todo irá bien. Yo he hablado con Ernestito\ldots{} ya se lo dije a usted
ayer. El dulce \emph{Anacarsis} está en la disposición más conciliadora,
y no le importa ni poco ni mucho su mujer. Se hace la cuenta de que
Virginia no existe, de que está viudo, situación que le agrada en
extremo. No echa de menos el matrimonio, ni tampoco el divorcio, porque
si lo hubiera y él recobrara por la ley la facultad de volver a casarse,
no lo haría\ldots{} Con que todo va como una seda, mi querido Socobio, y
sólo falta que pongamos en ejecución nuestro convenio lo más pronto
posible\ldots{}

---Sí, pronto\ldots{} De pensar que veré a Virginia soy un río de
lágrimas\ldots{} ¿Dice usted, Beramendi, que el chiquillo es lindo? Bien
podrá ser que haya sacado toda mi cara, mi expresión\ldots{}

---Paréceme que sí\ldots{} Usted le verá\ldots{}

---Y es una bendición que no hable todavía\ldots{} Me sabría muy mal
oírle nombrar a su padre\ldots{} No sé quién me dijo que el padre es
guapo, y yo me resistí a creerlo\ldots{} Ya sabe usted lo que dice Santo
Tomás\ldots{}

---No me acuerdo.

---Pues dice que nada puede ser bello si no es bueno.

---Hay excepciones\ldots{} Pero, en fin, dejemos eso\ldots{}

---Dejémoslo\ldots{} que, en último caso, la belleza física poco importa
y poco vale\ldots{} La belleza moral es reflejo de la Divinidad\ldots{}
Vea usted reunidas las dos bellezas, la moral y la física, en esta
angelical Valeria, ¡ay!\ldots{} que sería la perfección misma sin esa
flaqueza por la futilidad de las consolas, por la absoluta vanidad de
los entredoses, y por la frágil opulencia de las porcelanas\ldots{}
Déjeme usted que llore, mi querido Beramendi\ldots{} mi llanto es una
mezcla de alegría y de pena, porque veré junto a mí a mis dos hijitas,
la buena y mala, y a ratos, a ratos\ldots{} me forjaré la ilusión de que
las dos son buenas, piadosas, y las querré a las dos lo mismo, lo mismo;
y el chiquitín de Virginia me figuraré que es de Valeria, y creeré, como
creen los niños, que no lo engendró varón, sino que lo han traído de
París\ldots{} de París, Pepito, para recreo de mis dos hijas y mío.
Jugarán ellas, jugaremos todos con él\ldots{} De París ha venido en una
caja con mucho papel picado, como las que recibía Valeria con aquellas
lámparas elegantísimas y aquella loza de Sevres, que me costaban un
dineral\ldots{} De París ha venido el niño, sí, sí, y yo estoy muy
contento, yo lloro de contento\ldots{} y le estoy a usted muy
agradecido\ldots{} Beramendi, deme usted un abrazo fuerte,
fuerte\ldots{} y de mi parte este para María Ignacia\ldots{} Déselo
usted bien fuerte, bien fuerte\ldots{} ¡ay, ay, ay!»

\hypertarget{xxix}{%
\chapter{XXIX}\label{xxix}}

Salvó a Beramendi de aquel sofoco Cándido Nocedal, que fue a dar sus
plácemes a don Serafín por la feliz aprobación del convenio de paces, y
tuvo que aguantar los abrazos con salpicadura de lágrimas efusivas. El
ex-ministro reaccionario no había contribuido poco a domar la testarudez
socobiana, desmintiendo en aquel caso, como en otros de la vida privada,
el rigor de sus principios dogmáticos. En el comedor, todo era luz,
animación y alegre bullicio. Valeria se derretía con los finos galanteos
de Manolo Tarfe, y afectaba sorpresa burlona cuando el caballero hacía
descaradas alusiones a los flamantes amoríos de ella con Pepe
Armada\ldots{} Beramendi cogió al vuelo estas frases. «¡Qué tonto, qué
malo!\ldots{} Con usted no hay reputación segura.» Y en el otro corrillo
oyó a don Saturno: «Querido Guisando: eso que usted dice es un insulto a
la Divina Providencia, y una burla de los designios del Altísimo. Porque
el Altísimo permite que haya pobres, y los pobres y miserables lo son
porque así les conviene\ldots{} Permite también que haya ricos que no
necesitan trabajar\ldots{} Naturalmente, les conviene la ociosidad en
medio de la abundancia; pero el Hacedor, al permitir estas desigualdades
por conveniencia de unos y otros, no consiente que los ricos inventen
manjares absurdos por lo costosos. Eso es ya sibaritismo, y el
sibaritismo es pecado.

---El desenfreno de la gula---dijo Eufrasia,---llevará a los infiernos a
nuestro simpático \emph{gourmet.»}

Riva Guisando, encendido de satisfacción el rostro, respondió con
sonrisa olímpica que él no era más que un experimentador, un espíritu
teórico que ponía sus conocimientos al servicio de la Humanidad.
«Repetiré lo que ha escandalizado a esta señora---dijo,---para que se
enteren Beramendi y Tarfe. Yo sé hacer un caldo tan superior, que cada
taza sale por diez duros\ldots{} cinco tazas cincuenta duros, y no
rebajo ni un maravedí.» Grande escándalo y risotadas de incredulidad.
«¿Pero qué demonios echa usted en ese caldo?..» «Será que, en vez de
carbón, emplea usted billetes de Banco\ldots» «Lo hará con alones de
ángel, o con huesecitos de santos milagrosos\ldots» «Cuando uno tome ese
caldo, verá desde aquí el rostro del Padre Eterno.» Fiando a la
comprobación real su problema gastronómico, Guisando emplazó y convidó a
los presentes para la prueba. Él haría su caldo en casa de Farruggia.
Luego que los amigos cataran y saborearan tan extraordinario condumio,
él les daría exacta cuenta de las carnes, especias y demás ingredientes
que habían entrado en la composición\ldots{} y se vería si era o no
razonable calcular en diez duros el coste de cada taza.

Aceptaron todos el extraño convite. En opinión de don Saturno, Guisando
tenía que pedir pagas adelantadas, vender sus camisas y empeñar toda su
ropa, si daba en regalarse a menudo con su famosa invención. «Pues
sí---dijo Eufrasia,---quiero probar ese caldo y saber cómo se
hace\ldots{} para no hacerlo en mi vida, y declarar loco a su inventor.»
Con esto se puso fin a la reunión, que en aquella casa venerable
terminaba siempre temprano. Salió el primero don Serafín, acompañado de
su hija, y luego los demás convidados, agradecidísimos a las atenciones
de los Marqueses. Nocedal y Beramendi se fueron a sus casas, y Tarfe y
Guisando al Casino.

No se le cocía el pan a Manolito hasta no avistarse con Teresa, y a la
mañana siguiente se fue a su casa, esperando verla ya en completa
liberación del fardo. Aunque el rompimiento era seguro, aún no había
dicho Risueño la última palabra, según contó a su amigo la moza,
inquieta y malhumorada. «Hemos tenido más de un arrechucho. Está el
hombre imposible\ldots{} Ni él puede aguantarme a mí, ni yo a él. A
decir verdad, no ha estado muy violento\ldots{} lo que significa que no
me tiene ninguna estimación. Yo a él tampoco le estimo desde que sé que
quiere proteger a una tal Genara, alias \emph{la Zorrera}, que estuvo
con Pucheta\ldots{} y es de lo último de la calle\ldots{} Lo que más me
carga de Facundo es su gusto pésimo y su ordinariez\ldots{} Veo que me
despedirá como se despide a una criada que no guisa con aseo. Ayer me
dijo: «Sé que tomas varas de un chico, aprendiz de armero, que pedía
limosna\ldots{} Ya veo que tus caridades no son más que una tapadera
indecente.» Yo le contesté con medias palabras, de las que ni afirman ni
niegan\ldots{} siempre con dignidad. Sé tener dignidad; él no\ldots{}
Vaya con Dios\ldots{} No me importa: ya lo deseo.»

Reiteró Tarfe su proposición de recoger la herencia de Risueño, y la
guapa mujer, agraciándole con sonrisas y seductores melindres, le ordenó
que tuviese paciencia, y escuchase lo que a decirle iba. Sacó del
bolsillo un mal escrito papel, sentose frente a \emph{O'Donnell el
Chico}, y le dijo mostrando sus garabatos: «Estas notas, Manolo,
escritas por mí, que no estoy fuerte en ortografía, las pondrá usted en
limpio. Tome, entérese. Verá tres nombres de personas, y otros tantos
destinos, que quiero, Manolo, que necesito\ldots{} Lo hago cuestión de
gabinete: o me trae usted las tres credenciales, o no se presente más
delante de mí. Usted es poderoso; el General O'Donnell no le niega nada.
En todos los Ministerios tiene usted gran metimiento. Se va usted a
Posada Herrera, o a Calderón Collantes, o a Salaverría\ldots{} si no
prefiere irse a la cabeza, a su padrino don Leopoldo, diciéndole:
«Padrino, esto quiero\ldots{} Mis compromisos políticos me exigen,
me\ldots» En fin, usted sabrá lo que tiene que decirle.»

Leyó Manolito la nota, y suspirando dijo que lo haría, lo intentaría,
sin asegurar que lo consiguiese, pues había pedido ya y obtenido de don
Leopoldo y de los demás Ministros excesivo número de credenciales. Pero,
en fin, él lo tomaría con gran empeño, presentando los tres casos como
graves compromisos políticos, de los ineludibles y que no admiten
espera. Pedía Teresa para su tío don Mariano plaza de Jefe de
Administración, que era el ascenso que le correspondía, si era posible
en Estado, y si no en cualquier parte. Para Leovigildo Rodríguez pidió
plaza de la misma categoría que tuvo en Hacienda, y otra igual para don
Segundo Cuadrado. Tragose la nota el buen Tarfe, viendo que con Teresa
no valían palabritas engañosas, y se fue dispuesto a marear a medio
Ministerio y a su cabeza visible hasta lograr las tres plazas. Cosas más
difíciles había en este mundo. Él de nada se asustaba, fiado en su buena
estrella y en su ángel. Era el niño mimado de la Unión. Adelante, pues,
y a trabajar por Teresa, por aquel París que bien valía una misa, y aun
tres misas.

Mientras andaba \emph{O'Donnell el Chico} en la campaña que había de
producir el remedio de tres cesantes infelices, Teresa no mantenía
ociosa su mano liberal. Creía llegado el caso de repartir todos los
bienes que a ella le sobraban. Su idea desamortizadora y de distribución
del bienestar nunca brilló en su mente con luz tan viva. A su madre dio
dos trajes muy buenos para que los arreglara, y dos miriñaques; a
Mercedes Villaescusa, una bata, camisas, enaguas, zapatos; a doña Celia
envió macetas con las mejores plantas que entonces se conocían en
Madrid, y además loza de vajillas descabaladas, un par de cortinas,
cuatro botellas de manzanilla, un calentador para los pies, tabaco y
otras menudencias; a Jerónima, provisiones de boca, galletas finas y un
jamón, amén de unos visillos para las ventanas\ldots{} Y entre otros
pobres que en sus excursiones por los barrios del Sur había encontrado,
repartió diferentes especies de ropa y comestibles, y algún dinero. En
esta caritativa ocupación la sorprendió el \emph{ultimatum} de Risueño,
que se despidió de ella con una carta muy mal escrita, concediéndole la
propiedad de todo lo existente en la casa, y enviándole mil reales de
\emph{plus}\ldots{} Alegrose Teresa de que la madeja de aquel lío se
desenredase tan suavemente, y dio por buena la mezquindad del socorro
final, perdonando el coscorrón por el bollo. Nunca le fue tan grata la
libertad; nunca tan a sus anchas respiró, a pesar del alarmante vacío de
sus arcas. Ya vendría dinero de alguna parte; vendría tal vez la franca
resolución de despreciarlo, y el recurso supremo de no ver su necesidad.
Hallábase después de la carta de Risueño en gran perplejidad, cavilosa,
echando ahora su alma por un camino, ahora por otro. Pocos días después
de encontrarse libre, recibió la visita de una señora, o con apariencias
de tal, que alguna vez se personaba en su casa; mujer de peso, de
historia y de mucha labia, de estas que vienen a menos por desgracias de
familia, o por picardías de hijos desnaturalizados. Había sido famosa
\emph{cuca}; vestía decentemente, sin borrar de sí la inveterada traza
\emph{celestinoide}.

Secretearon las dos algo que merece referirse. Con extremados
encarecimientos habló Serafina, que así la tal se llamaba, de un
opulento señor, en buena edad, que por la calle había visto a Teresa y
deseaba obsequiarla en alguna forma delicada\ldots{} «Usted se habrá
fijado tal vez\ldots{} y ya comprende a quién me refiero\ldots{} Sólo le
diré, por si lo ignora, que ese señor tiene la contrata de todo el
tabaco que en España se consume, y que no sabe qué hacer del
dinero\ldots{} Pero sí sabe, sí. ¿Ve usted la Puerta del Sol, con todas
las casas derribadas para hacerlas de nuevo, ensanchando la plaza? Pues
dicen que él levantará todas las casas nuevas. Imagine usted qué
fincas\ldots{} Es de estos hombres que de chicos se van descalzos a la
Habana y vuelven con las botas puestas\ldots{} Pero este no trabajó en
calzado, sino en sombreros, con más suerte que mi difunto esposo, que
después de ganar en Cuba muchísimo dinero, allá se dejó las onzas y la
pelleja\ldots{} Pues como le digo, es persona sentada, tan limpio que da
gloria verle; la cara bonachona, los cabellos entrecanos\ldots{} figura
hermosa en sus años maduros\ldots{}

---Le conozco de vista---dijo Teresa poco interesada en el asunto,---y
algo me han contado de la facilidad con que gana el dinero. Yo, si he de
decirle la verdad, Serafina, estoy cansada de esta vida\ldots{} ¿Sabe
usted lo que pienso de algunos días acá? Se va usted a reír\ldots{}
Ríase lo que quiera. Pues se me ha metido en la cabeza dedicarme a la
honradez pobre, o a la pobreza honrada\ldots{} que es lo mismo\ldots{}
¿Qué le parece?

---¡Ay, hija mía: si es cuestión de conciencia, yo nada digo; no me meto
a dar consejos a nadie, mediando la conciencia! ¡Ay, no, no!\ldots{}
¿Pero de qué le ha dado a usted esa ventolera? ¿Es cierto lo que oí, que
le ha salido a usted un obrerito? Hija mía, ándese con cuidado con los
obreritos, que esos\ldots{} a lo mejor la pegan, y salen unos
perdularios o unos borrachines. Las clases bajas de la sociedad, me
decía Bravo Murillo, son dignas de que se las socorra, de que se las
aliente; pero líbrenos Dios de meternos entre ellas\ldots{} No, no: ni
usted ni yo, por nuestra educación, podemos hacernos a la grosería del
pueblo.

---Yo no pienso así. Al contrario, se ha fijado en mí la idea de que no
hay cosa mejor que no poseer nada, absolutamente nada. ¡Fuera
necesidades, fuera obligaciones! Tener una un hombre que la
quiera\ldots{} casarse con él, vivir con vida sencilla,
descuidada\ldots{} ganando el pedazo de pan necesario para cada
día\ldots{}

---¡Ay, ay, Teresa, qué gracia me hace usted! ¡Salir con eso del
bocadito de pan, ahora, ahora, cuando tenemos a la Unión Liberal, que
viene con la idea de hacer de España otro país, como quien dice,
fomentando, fomentando\ldots! Yo no sé expresarlo bien; pero \emph{este
es el momento histórico}\ldots{} así me lo ha dicho don Francisco
Martínez de la Rosa\ldots{} \emph{el momento histórico} de multiplicar
en España las comodidades y el bienestar de tantos miles de
almas\ldots{} Tendremos más ricos, pudientes muchos, y menos
pobres\ldots{} Vendrá la venta de la Mano Muerta\ldots{} saldrán miles
de millones\ldots{} y verá usted a España cubierta de
\emph{ferroscarriles}, que traerán a Madrid todo el género de las
provincias casi de balde\ldots{} Así me lo decía esta mañana Salustiano
Olózaga, y del mismo parecer es el Infante don Francisco, con quien
hablé la semana pasada, y me dijo: «Serafina, mucha riqueza que está
guardada veremos salir pronto de debajo de la tierra.» ¡Y \emph{en este
momento histórico} cambia usted de rumbo, y vuelve su lindo rostro hacia
la pobreza!\ldots» La condenada tenía la perversa costumbre de citar
personas respetables, que le daban confianzudamente un autorizado
parecer, con el cual fortalecía su opinión propia. Teresa, en verdad sea
dicho, había tenido con ella poco trato, y este fue casi siempre
puramente comercial, por la compra o cambalache de joyas, encajes,
abanicos, y otras prendas que cautivan a las señoras. Sin hacer ningún
negocio la despidió aquella mañana, y fue tan discreta Serafina que no
reiteró su proposición, limitándose a decir: «Volveré, Teresita\ldots{}
quiero verla a usted en su nuevo papel\ldots{} ¡Compartir la vida pobre
con un obrerito! ¡Qué pronto se dice, y qué bonito parece pensado y
dicho!\ldots{} En el hecho ya es otra cosa\ldots{} Aquí donde usted me
ve, yo, en mis quince y en mis veinte, tuve, mejor diré, padecí, ese
bello ideal, y\ldots{} ¡ay!\ldots{} En fin, no quiero quitarle las
ilusiones. Váyase usted, Teresita, a la pobreza honrada, que si es
cuestión de conciencia, yo seré la primera que le aconseje ir por ese
camino. La conciencia sobre todo: así me lo decía, sin ir más lejos,
ayer tarde, el Cardenal Fray Cirilo de Alameda\ldots{} Adiós, hija mía.»

\hypertarget{xxx}{%
\chapter{XXX}\label{xxx}}

Atacó el buen Tarfe con loco empeño a su protector don Leopoldo, de
quien obtuvo una repulsa cariñosa. Ya le dolía la mano de dar a su
protegido tantas credenciales. De Posada Herrera, a quien ya tenía frito
con sus peticiones, nada sacó en limpio. Más feliz fue con Salaverría y
con el Marqués de Corbera, que al menos le dieron esperanzas. El hueso
más duro de roer era el destino de Centurión, en Estado; y no viendo
medios de salir airoso con O'Donnell ni con Calderón Collantes, que se
llamaban Andana, dirigió sus tiros contra doña Manuela, que le quería,
le mimaba y se divertía con su graciosa cháchara. No la encontró muy
propicia, por tener bastante gastada ya su poderosa influencia; pero
Tarfe insistió, y para ganar el último reducto de la voluntad de la
señora, le llevó folletines nuevos, que ella no conocía: \emph{Isaac
Laquedem}, por Alejandro Dumas, y luego \emph{Los Mohicanos de París},
del mismo autor. Esta larga y complicada obra fue muy del agrado de la
Condesa. Tarfe sacrificaba por las noches sus más agradables ratos de
casino y teatros para leerle a doña Manuela pasajes de febril interés.
Total: que con esto y sus hábiles carantoñas, y los elogios que hacía
del gran mérito administrativo de Centurión (no le conocía ni de vista),
logró interesar a la señora, y el buen don Mariano tuvo su destino, no
en Estado, sino en Fomento, que para el caso de comer era lo mismo.

Para mayor ventura de Manolo Tarfe, el mismo día que le dieron la
credencial de Centurión, entregole Salaverría la de Leovigildo
Rodríguez. Sólo faltaba la de Cuadrado; pero de esta colocación se
encargó Beramendi, gozoso de favorecer al que había sido su desgraciado
jefe en la \emph{Gaceta}. Con estas bienandanzas, corrió Tarfe a ver a
Teresa. Le llevaba todo lo que le había pedido. Tan contento estaba el
hombre de poder satisfacerla en sus deseos generosos, que al darle las
credenciales, se dejó decir: «Pide por esa boca, Teresa. A ver si
encuentras otro que con tanta diligencia te sirva.» Muy agradecida, y
loca también de contento, Teresa no dio al caballero el sí que este
anhelaba; difirió su acuerdo para dentro de algunos días. «Estoy ahora
en grandes dudas, Manolo, y dispense si no le contesto a lo que
desea\ldots{} Mil y mil gracias, amigo: es usted la flor de la canela
para estas cosas. ¡Viva la Unión Liberal! y viva \emph{O'Donnell el
Chico}, que es el vicario del grande\ldots{} Crea usted, Manolo, que le
aprecio de veras\ldots{} Pero estoy en la crisis del alma, en la
terrible duda, Manolo. ¿Me voy hacia arriba, o me voy hacia abajo? ¿La
felicidad dónde está? ¿En la honradez pobre y sin cuidados, con sólo un
hombre para toda la vida, o corriendo, arrastrada de muchos hombres, y
metiendo mano a los millones de la Desamortización?» Decía esto muy
nerviosa, poniéndose la mantilla. Creyó Tarfe que no estaba buena de la
cabeza, o que de él donosamente se burlaba. Salió la guapa moza, sin
permitir que el caballero la acompañase por la calle. En la esquina de
Antón Martín le dejó plantado, corriendo con paso ligero a llevar las
buenas nuevas al infortunado Centurión.

Interesantísima fue la escena de la presentación de la credencial a don
Mariano, quedándose el buen señor tan absorto y turulato que no daba
crédito a lo que veía\ldots{} Leyó doña Celia; corrió Centurión a sacar
del cajón de la mesa unos anteojos de gran fuerza que usaba para leer
documentos de letra borrosa\ldots{} Ni aun leyendo con aquellas potentes
gafas se convencía\ldots{} Ordenó a su mujer que leyese de nuevo\ldots{}
¡Colocado y con ascenso! No podía ser. «Teresa, o eres tú un demonio,
que gasta conmigo bromas harto pesadas, o Dios me confunde por haber
hablado mal de don Leopoldo O'Donnell.» Díjole a esto Teresa que el Jefe
de la Unión Liberal estaba bien al tanto de lo que valía don Mariano, y
que de su \emph{motu proprio} había ordenado la reposición\ldots{} El
gozo de ver terminada su horrible cesantía inundó el alma del buen
señor; mas por entre los espumarajos del gozo asomó la dignidad adusta
diciéndole: «Hombre menguado, aceptas tu felicidad del hombre público
más funesto\ldots{} y por mediación de tu pública sobrina\ldots{} Lo que
no lograron los principios de un varón recto, lo consigue la hermosura
de una mujer torcida\ldots{} ¡En qué manos está el Poder!\ldots» Viendo
que doña Celia mostraba su gratitud a Teresilla besándola con ardiente
cariño, se escabulló del alma la dignidad de don Mariano. «Será preciso
que yo vaya personalmente a dar las gracias al General---dijo paseándose
en la habitación con grandes zancajos. Replicó Teresa que no deseaba
O'Donnell más que conocerle, y felicitarle por su mérito administrativo.

Una vez derramados los chorros de alegría en aquella casa, corrió Teresa
a la de Mercedes Villaescusa. Al darle la credencial añadió estas graves
palabras: «Dile a Leovigildo que ahí tiene eso, la mejor prueba de que
Teresa Villaescusa es buena cristiana y sabe devolver bien por mal. Tu
marido escribió el anónimo diciéndole a Facundo que yo tenía algo que
ver con Santiuste\ldots{} Es una canallada, de la cual me vengo
sacándoos de la miseria\ldots{} No, no me niegues que tu marido escribió
el anónimo. Por mucho que quiso disimular la letra, no logró disimular
su infamia\ldots{} Conocí la mano que escribió el anónimo por el trazo y
por dos faltas de ortografía que son suyas\ldots{} suyas son\ldots{} Se
me quedaron en la memoria desde que me escribió una carta pidiéndome
doscientos reales, que por cierto le di\ldots{} Pone \emph{berdad} con
\emph{b} alta, y \emph{prueva} con \emph{v} baja\ldots{} No, no le
defiendas: mi ortografía es mala; allá se va con la de él\ldots{}
Pero\ldots{} convence a tu marido de una cosa: la falta más fea de
ortografía es\ldots{} la ingratitud\ldots{} Adiós; que lo paséis bien.»
No esperó la réplica, y bajó muy terne por la empinada escalera.

Con la satisfacción de haber producido el bien, Teresa no pensó ya más
que en frecuentar el trato de \emph{Mita} y \emph{Ley}, a quienes había
tomado gran cariño. Mientras vivieron en las Vistillas, a los dos les
veía casi diariamente; pero una vez que los esposos libres se
trasladaron a la casa de Beramendi, no encontraba en el taller más que a
Leoncio y al espiritual Santiuste, ardoroso en el trabajo por
instigación constante de la guapa moza. Ya esta no era un enigma para
\emph{Mita} y \emph{Ley}, que la conocían por lo que era y lo que había
sido, y ambos ponían gran empeño en atraerla mansamente a las vías de la
virtud, conforme al sentir general, no al sentir suyo; que no se
atrevían a proponer su libertad como modelo de vida. Pero ya se uniesen
Juan y Teresa por lo libre, ya por lo religioso, tropezarían con un
grave problema: los medios de la vida material. Pensando en esto,
\emph{Mita} y \emph{Ley} no veían clara solución, porque con el mezquino
jornal que daban a Juan (y más no le daban porque no podían), no era
posible sostener una casa por humilde que fuese. Quizás Teresa, pensaban
ellos, que tan buenas plazas había obtenido para infelices cesantes,
conseguiría para su futuro un buen puesto en la Administración pública,
quitándole del oficio. En esto discurrían torpemente Leoncio y Virginia,
pues nada más lejos de la fantasía de Teresa que vulgarizar y
empequeñecer la personalidad del buen Tuste, confiriéndole la
investidura de vago a perpetuidad, sin horizontes ni ninguna esperanza
de gloriosos destinos. En la crisis que removía su espíritu, la cual era
como si todo su ser hubiese caído en ruinas, y de entre ellas quisiera
surgir y sobre ellas edificarse un ser nuevo, extrañas ambiciones al
modo de centellas la iluminaban. Por momentos veía que la más hermosa
solución era imitar el arranque intrépido de \emph{Mita} y \emph{Ley}
cuando se arrojaron solos en brazos de la Naturaleza, sin recursos, con
lo puesto, volviendo la espalda a la sociedad y encarándose con la
severa grandeza de los bosques inhospitalarios\ldots{} ¿Serían Teresa y
Juan capaces de repetir el paso heroico de sus amigos?

En esto pensaba la Villaescusa sin cesar, desde que se sintió enamorada
de Tuste, y miraba con desdén, casi con repugnancia, los ordinarios
arbitrios de vida pobre, el jornal, el empleíto, y el encasillado
inmundo en un mechinal urbano. En estas ideas fluctuaba cuando ocurrió
lo que a continuación se cuenta.

Tan hechos estaban \emph{Mita} y \emph{Ley} al vivir campestre, que no
podían pasarse sin salir los domingos a ver grandes espacios luminosos,
tierra fecunda o estéril, árboles, siquiera matas o cardos borriqueros,
la sierra lejana coronada de nieve, agua corriente o estancada,
avecillas, lagartos, insectos, todo, en fin, lo que está fuera y en
derredor del encajonado simétrico que llamamos poblaciones. Desde que
Tuste entró en el taller, les acompañaba en sus domingueras expansiones,
y cuando Teresa cultivó la amistad de los armeros por querencia de Juan,
fue también de la partida una vez o dos, y por cierto que se recreó lo
indecible, tomando gusto a lo que parecía ensayo de vida suelta. Salían
los expedicionarios por diferentes puntos, la Mala de Francia, la
Moncloa, las Ventas; pero cuando no querían andar demasiado, y esto
ocurría siempre que llevaban a Teresa, incapaz de largas caminatas,
preferían un lugar próximo, la llamada \emph{huerta del Pastelero},
grande espacio cercado, en las afueras del Barquillo, junto al camino
viejo de Vicálvaro, ni huerta, ni solar, ni campo, ni jardín, aunque
algo de todo esto era, y restos quedaban de las diversas granjerías que
existieron en aquel vasto terreno. Teníalo arrendado Tiburcio Gamoneda
para establecer en él en grande las famosas industrias de \emph{obleas},
\emph{lacre} y \emph{fósforos}, que tuvo su padre en la calle de
Cuchilleros. Había una casa o almacén que debió de parecer palacio a los
que estaban hechos a los chinchales del interior de Madrid; había dos
estanques de quietas y limpias aguas, con pececillos; algunos árboles,
entre ellos cuatro cipreses magníficos junto a los estanques, que
reproducían, vueltas hacia abajo, sus afiladas cimeras de un verde
obscuro y triste. Eran Tiburcio y su mujer hacendosos, y habían
compuesto una noria vieja, con la cual podían sostener un fresco plantío
de hortalizas. Tenían gallinas y palomas que se albergaban en la casa;
un perro y un burro completaban su arca de Noé, amén de un tordo
enjaulado.

Pues un sábado de Abril, pocos días después de la entrega de
credenciales a Centurión y Leovigildo, recibieron \emph{Mita} y
\emph{Ley}, a punto que anochecía, la visita de Teresa. Invitáronla para
el día inmediato, domingo, en la \emph{huerta}, que así llanamente
decían. Aceptó Teresa gozosa. ¿Quería que Tuste fuese a buscarla? No:
ella iría sola; bien sabía el camino. No convenía que Juan fuese a
buscarla, porque si se enteraba la madre de ella, furiosa enemiga de
Juan, podría inventar cualquier enredo para impedir que acudiera a la
cita. Fueran los tres temprano, que ella, solita, recalaría por la
huerta sobre las diez\ldots{} Así se convino\ldots{} Partió
Teresa\ldots{} En Puerta Cerrada tomó un coche para llegar pronto a su
casa, y al entrar en ella temerosa, dijo a Felisa: «Si vuelve
\emph{O'Donnell el Chico}, le dices que no estoy, que he ido a casa de
mi madre\ldots{} No, no, eso no, que el muy tuno allí se
plantaría\ldots{} Le dices que me he marchado fuera de Madrid\ldots{} a
un pueblo\ldots{} inventa el pueblo que quieras\ldots{} y que no volveré
hasta pasado mañana\ldots{} o hasta cuando a ti te parezca.» Adviértase
que desde que le dio las credenciales, Tarfe la perseguía sin descanso,
y a su puerta llamaba sin conseguir ni una vez sola ser recibido.

\hypertarget{xxxi}{%
\chapter{XXXI}\label{xxxi}}

Se acostó Teresa, y desvelada estuvo gran parte de la noche, sintiendo
la voz de Tarfe en la puerta y las mentiras con que Felisa por centésima
vez le despachaba. Engañó el insomnio, pesando y midiendo los términos
candentes de la resolución que había de tomar el día próximo. Se llaman
candentes estos términos, porque le quemaban el cerebro cuando
alternativamente o los dos juntos entraban en él. Grave era esto, grave
lo otro\ldots{} tan difícil el sí como el no, y el ser no menos
escabroso que el no ser\ldots{} Levantose temprano, después de un corto
sueño; se arregló y vistió; cuando tomaba su chocolate, cayó en la
cuenta de que su portamonedas estaba flaquísimo. Sólo le quedaban dos
napoleones y alguna peseta del dinero que le dejó Facundo.
Afortunadamente, tenía innumerables objetos de valor que vender o
empeñar\ldots{} Apartada un momento del estado económico, voló a más
alta esfera, y de esta descendió, para pensar que debía ir a ver a su
madre. Si no la entretenía y embaucaba con cualquier embuste, Manolita
vendría en busca de ella; la perseguiría como a una criminal, si no la
encontraba\ldots{} Hasta era capaz de coger un coche y plantarse en la
huerta, que bien sabía dónde estaba. La primera vez que Teresa fue a la
merienda, hizo la tontería de contárselo a su madre, y describirle el
sitio, y darle cuenta de las personas que la invitaron\ldots{} Por ser
ella tan necia, se veía sin libertad. «No se puede ser
libre---pensaba,---sino con sombra de hombre.»

Encaminose a la calle de Cañizares, donde vivía doña Manuela Pez, y tan
recelosa iba de que su madre la detuviera birlándole la merienda en el
campo, que de la escalera pensó volverse. No se determinó a retroceder.
Subió despacio. Manolita, que desde el balcón la había visto entrar, le
abrió la puerta, y llevándola con cierto misterio al cuarto más próximo,
le dijo: «¿Tienes algo que hacer hoy por la mañana? ¿Has venido con la
idea de quedarte a almorzar conmigo?\ldots» «De aquí---replicó Teresa,
encontrando con rápida inspiración la mentira,---pensaba ir a casa del
tío Mariano. Quiero zambullir mi espíritu en la alegría de aquella casa,
nadar en ella. ¡Cómo están los pobres!

---Pues vete al instante---dijo Manolita, con el delicado y sutil acento
que empleaba en los casos de gran oficiosidad.---Espero por la mañana la
visita de una persona que viene a tratar conmigo de un asunto\ldots{} No
te lo digo\ldots{} Ya has comprendido que se trata de ti, ¿verdad? Pues
por lo mismo que se trata de ti, no quiero que estés en casa.

---¿La persona que usted espera es hombre o mujer?

---¡Ah, picaruela! sospechas que es Serafina. No\ldots{} Serafina estuvo
anoche dos veces; hoy volverá\ldots{} Pero no es ella la persona que
espero\ldots{} Y lo repito: no quiero que estés aquí cuando
venga\ldots{} Necesito estar sola\ldots{} ¡Ay, hija, cuánto tienes que
agradecerme!\ldots{} Otra cosa: como al mediodía tendré que salir y no
sé cuándo volveré, no vengas acá hasta la tarde\ldots{} mejor a la
noche\ldots{} ¡Ay! concédame Dios el poder darte esta noche un notición
tremendo\ldots{} Mucho vales tú, Teresa\ldots{} pero una suerte tan
grande, tan grande, no podrías soñarla\ldots{}

---Bueno, mamá: ya me lo dirás\ldots{} ¿De modo que me mandas que me
vaya?\ldots{}

---Sí, sí\ldots{} pronto\ldots{} Vete a casa del tío Mariano\ldots{} Que
te convide a almorzar, que bien te lo has ganado\ldots{} Bueno\ldots{}
no te entretengas\ldots{} Adiós, hija; hasta la noche.»

Vio Teresa el cielo abierto, y no se hizo rogar para tomar el
portante\ldots{} ¡Qué suerte había tenido! Su madre no sólo no la
retenía, sino que la echaba\ldots{} ¿Y qué negocio arduo era el que la
viuda tenía que tratar con el desconocido visitante? ¡Ay, ay,
ay!\ldots{} ¿Y por qué no podía estar ella en la casa mientras Manolita
conferenciaba? ¡Ay, ay!\ldots{} ¡Y era ella el objeto de la
conferencia!\ldots{} ¡Y a la noche, notición tremendo! ¡Ay!\ldots{}
Aunque todo esto le resultaba odioso, no cesaba de pensar en ello,
siguiendo presurosa su camino en dirección de la calle de Alcalá\ldots{}
Y era la curiosidad lo que la hacía pensar, pensar en lo mismo, apurando
toda la lógica para descubrir el pensamiento de su señora madre.
Curiosidad era sin duda, y no gusto de aquellas intrigas ni de sus
consecuencias\ldots{} Verdad que el amargor de ciertas cosas no quita el
picor del deseo de conocerlas. «Sabiendo lo que es esto---se decía,---lo
aborreceré mejor.» Gozosa de haber encontrado esta fórmula que
armonizaba la virtud con la curiosidad, desembocaba por la calle del
Baño para tomar la de Cedaceros, cuando chocó con un objeto duro\ldots{}
tal efecto le hizo ver a \emph{O'Donnell el Chico}, que venía en
dirección contraria.

«¡Teresilla\ldots{} alto! Ya no te me escapas\ldots{} Trescientas veces
he llamado inútilmente a tu puerta\ldots{} y ahora\ldots{} la casualidad
te trae a mí.

---Si me hubiera bajado al Prado desde mi casa---dijo Teresa, sin
disimular lo que el encuentro la contrariaba,---mejor cuenta me habría
tenido\ldots{} Manolo, por Dios, déjeme seguir mi camino\ldots{} Vaya;
un saludito y cada uno por su lado.»

No se avino el joven a esta forma tan simple de separación, y siguió
junto a ella protestando de que no era su idea molestarla. Aceleró
Teresa el paso, fingiendo mucha prisa; pero Tarfe no se rendía
fácilmente, y amenizaba la carrera con estas bromas: «¿Vas a apagar un
fuego? Mejor: yo llevo las bombas.»

---Manolo, por la Virgen Santísima---dijo Teresa parándose,
sofocada:---es usted caballero, y no se obstinará en seguirme cuando yo
le suplico que no me siga\ldots{} ¿Qué quiere de mí?

---Verte, oír tu voz\ldots{} Hicimos un trato que yo he cumplido
fielmente, tú no.

---Yo no prometí nada, Manolo, ni era preciso, porque usted, al
conseguirme las credenciales, hacía una obra de caridad, y no quería más
recompensa que la satisfacción de socorrer a los desgraciados.

---Teresilla, sabes más que Aristóteles. Si no te quisiera por tus
encantos, por tu talento te adoraría, por el salero con que sabes ser
traidora, pérfida, ingrata.

---No desvaríe, Manolo, y déjeme seguir.

---Vas aprisa como los que han hecho una muerte: el muerto soy yo.

---Voy aprisa, sí señor; voy fugada.

---¡Fugada!\ldots{} Llamas tú fugas a las escapatorias de la mujer
caprichosa que un día sale a correrla\ldots{}

---No es escapatoria de un día, Manolo---dijo Teresa con gravedad que
dejó suspenso a \emph{O'Donnell el Chico};---es para siempre.

---¡Para siempre!

---Y no me verá usted más\ldots{}

---Si fuera cierto, sería lo más desagradable que pudieras
decirme\ldots{} Pero no es verdad, Teresa. Tú no eres capaz de seguir la
senda por donde fueron Mita y Ley. Eres cortesana\ldots{} Parece que has
abandonado tu puesto en la Corte de Venus, y lo que haces es alejarte
hoy para volver mañana, y ocupar tu sitio\ldots{} con ascenso\ldots{}
Cuando parece que bajas, subes, Teresa, y has de ponerte al fin tan alta
que desprecies a los pobres como yo\ldots{} y no podremos ni mirarte
siquiera.» Dijo esto \emph{el pequeño O'Donnell} con tristeza. Teresa no
le entendía; esperaba que hablase más claro.

«Todo lo que usted me dice, Manolo, es para mí como si me hablara en
chino\ldots{} ¡Yo despreciarle a usted\ldots{} y por pobre!\ldots{}
¡Jesús! ¡Vaya con el pobrecito, el hombre de la influencia, el niño
mimado de la Unión Liberal, el primero de los hombres públicos. Muy
agradecida estoy a \emph{O'Donnell el Chico}, pues apenas abrí la boca
para interceder por tres cesantes, fuí atendida\ldots{}

---Pero ya no necesitarás recurrir a mí, Teresa. Lo que obtuve yo para
tus parientes, te ha de parecer pronto a ti la última de las
bicocas\ldots{} Porque tú, con más facilidad que ninguna otra persona,
darás credenciales de Directores Generales, de Gobernadores, ascensos al
Generalato y propuestas de Obispos; tú, Teresa, tú\ldots{} No pongas
ojos espantados\ldots»

Decían esto bajando por la calle de Alcalá. La curiosidad que, en forma
de brasas, sentía Teresa en su mente, ya levantaba llama. «Explíqueme
eso de modo que yo lo entienda---dijo a Tarfe;---y explíquemelo pronto,
porque tengo prisa. Voy lejos, y en la Cibeles he de tomar coche, o
tartana si la encuentro.

---Yo tomaré la tartana, y te llevaré a donde quieras.

---Eso no\ldots{} Acompáñeme a pie un ratito, y después cada lobo por su
senda\ldots{} Quiero saber de dónde voy yo a sacar ese poder que usted
supone\ldots{} ¡Qué gracioso!

---¿De dónde?\ldots{} De tu hermosura, de tu gracia\ldots{} Eres la
mayor farsante que conozco, y la cómica más perfecta. No sé para qué
gastas conmigo esos disimulos. ¿Cómo has de ignorar tú que alguna
persona de grandísimo poder y de riqueza desmedida te solicita\ldots{}
vamos, pide tu mano para llevarte al altar que no tiene santos?\ldots{}
Hazte la tonta. ¿Crees que me engañas?\ldots{}

---Pues, hijo, gracias por la noticia de la petición de mano\ldots{}
Pero puede creerme que no sabía nada\ldots{} ¡Qué risa! ¿Pues así se
piden manos sin que los ojos se hayan dicho algo antes? Usted ha perdido
el juicio, Manolo.

---Lo perderé por ti, viéndote en manos de las que no podré quitarte.
Soy fuerte si me comparas con Risueño, débil si con otros me
comparas\ldots{} ¿Quieres que te diga una cosa, una idea que desde
anoche se me ha metido aquí y no puedo soltarla? Pues tú eres el numen
de la Unión Liberal, la encarnación de esas ansias de bienestar y de
esos apetitos de riqueza que van a ser realizados por mi partido. Tú
eres la evolución de la sociedad, que transforma sus escaseces en
abundancias con los tesoros que saldrán de la tierra; tú\ldots{}

---Cállese, por Dios, Manolo\ldots{} Me trastornaría la cabeza si no la
tuviera yo bien firme. ¿Qué tengo yo que ver con tesoros enterrados, ni
con nada de eso?

---No diré que por tus propias manos; pero sí que por manos que estarán
muy cerca de las tuyas, han de pasar los millones, los miles de millones
de la Desamortización.

---¡Jesús, Manolo!

---Sé lo que digo\ldots{} A tu lado verás nacer y crecer las maravillas
del siglo, los caminos de hierro\ldots{} verás el remolino que hace el
oro, girando en derredor de los que lo manejan y hacen de él lo que
quieren\ldots{} ¿Qué mujer podrá, como tú, darse el gusto de ser
dadivosa?»

Pudo creer Teresa, en los comienzos de la conversación, que Tarfe
bromeaba\ldots{} Ya creía que mezclaba las burlas con las veras, y que
algo había de verdad en aquel fantástico vaticinio. Sin duda, en el
conocimiento de Manolo había una certidumbre que en el ánimo de ella
sólo era un presagio, más bien sospecha. Cierto que hombres de gran
poder político y financiero gustaban de ella; pero ¿en qué se fundaba
\emph{O'Donnell el Chico} para sostener que entre el deseo y la
realización había tan poca distancia? Con esto, la curiosidad, que desde
la rápida entrevista con su madre prendió en su mente, era ya incendio
formidable. Las llamas le salían por los ojos, y por la boca este vivo
lenguaje:

«Párese un poquito, Manolo, y, dejando a un lado las bromas, dígame si
es verdad eso de la Desamortización; si es un hecho ya, vamos. Porque
Risueño decía que la Desamortización es un \emph{mito}, que es como
decir una guasa.

---No es mito, sino dogma, Teresa: pronto será un hecho, y la gloria más
grande de O'Donnell y de la Unión Liberal\ldots{} Con la ventaja de que
ya el desamortizar no traerá trifulcas ni cuestiones, porque se hará de
acuerdo con el Papa\ldots{} Ya está negociado el nuevo Concordato. Ríos
Rosas y Antonelli han quedado ya conformes. ¿Me entiendes? Concordato es
un convenio con la Santa Sede.

---¿Para que desamorticemos todo lo que queramos?

---Para que se venda la Mano Muerta, favoreciendo a la Mano Viva\ldots{}

Algunos segundos estuvo Teresa como alelada, mirando al suelo\ldots{}
Luego dijo: «Bien, Manolo; me parece bien\ldots{} Razón tiene usted en
adorar a O'Donnell\ldots{} yo también le admiro, y declaro que es el
primer hombre de España.

---Como tú la mujer más simple de todo el Reino, si no me confiesas que
eso que has dicho de fugarte y no volver, es un bromazo que quisiste
darme. ¡Cómo he de creer yo que desmientes la ley de tu destino en el
mundo, y que ahora, cuando la fortuna te da todo lo que
ambicionabas\ldots{} no lo niegues: tú me has dicho que lo
ambicionabas\ldots{} cuando la fortuna viene en busca de ti, huyes tú de
ella!\ldots{}

---Cierto es que tuve mis sueños de grandeza y poder\ldots{} ¿Quién no
sueña, viviendo como vivimos en medio de tantas necesidades?\ldots{}
Pero ya esa racha pasó, ya estoy curada de esos desatinos\ldots{}

---Esa cura no puede hacerla más que el amor\ldots{} Pero hay casos en
que la salud es tan mala como la enfermedad\ldots{} o peor\ldots{} Y yo
te digo, con toda la efusión de mi alma: «Teresa, ¿no podrías conciliar
la ambición y el amor? Ello es sencillísimo: aceptas lo que los ricos te
dan, y me quieres a mí. La riqueza mía es corta, Teresa; lo suficiente
para la vida de mediano rumbo que yo me doy\ldots{} No puedo satisfacer
tu ambición\ldots{} Pero los que pueden satisfacerla no te darán un
corazón amante como el mío.» Rebelose Teresa contra la profunda
inmoralidad que esta proposición envolvía\ldots{} «Manolo---le
dijo,---no es nada caballeroso lo que usted pretende\ldots{} ¿Quiere que
le diga toda la verdad, confesándome con usted como con Dios? Pues
sentémonos. Estoy cansada. Se cansa una de andar, de pensar cosas raras:
cansa la duda, y cansa el no entender bien las cosas\ldots{} ¿Con que
dice usted que podré yo desamortizar? ¡Qué risa!

---¿No lo crees?

---Juro a usted que no lo creo.»

Teresa juraba en falso. Aunque no conocía la tragedia de Macbeth, en su
íntimo pensamiento se decía: \emph{La ciencia de aquellas mujeres (las
brujas) es superior a la de los mortales}. Y las brujas del tiempo de
estas historias se llamaban \emph{O'Donnell el Chico}.

\hypertarget{xxxii}{%
\chapter{XXXII}\label{xxxii}}

Se sentaron en un rústico banco, próximo al jardín de la Veterinaria.
Habló Teresa la primera: «No tengo ya esa ambición, Manolo. Me la quitó
el amor. Por primera vez en mi vida puedo decir que quiero a un hombre.
Dispénseme si le lastimo: ese hombre que yo quiero no es usted.

---Ya sé\ldots---murmuró Tarfe sombrío, quejumbroso\ldots.---Es el
aprendiz de armero. Le conozco\ldots{} Sigue, Teresa.

---¿Qué más quiere usted que le diga? Que a los pocos días de tratar a
Juan sentí por él una piedad y un respeto, que pronto, sin pensarlo, se
me convirtieron en cariño. Vi en él la conformidad con la desgracia,
cosa nueva para mí; pude ver y conocer que el pobrecito tenía por mí un
amor muy grande, sin cuidado de la opinión, y que con su pensamiento me
limpiaba, y borraba todas mis faltas para volverme pura y poder adorarme
a su gusto. ¿Comprende usted lo que esto vale, Manolo? Por el hombre que
así me quiere, por el que en mí ve la mujer, la madre, la hermana, y
todos estos amores reunidos en uno, ¿qué menos puedo yo hacer que
consagrarle mi vida?

---Comprendo, sí, que desees consagrarle tu vida; pero no que lo hagas.
La cantidad de abnegación que necesitas para descender hasta él es
enorme\ldots{} Más que virtud, sería santidad, y esta no existe hoy en
el mundo. Creo en tu amor; no creo en tu santidad. Si yo te viera
consumar el gran sacrificio, si te viera precipitarte a la pobreza y al
estado de vulgar estrechez que te traería la unión con el armero, fuera
por matrimonio, fuera de otro modo, yo te admiraría, Teresa, y
respetaría tu caída\ldots{} Caer de ese modo es alcanzar la mayor
elevación moral. ¿Entiendes lo que quiero decirte?

---Sí lo entiendo, y desde hoy puede usted empezar a respetarme y
admirarme---dijo Teresa levantándose.---A la pobreza honrada voy\ldots{}
ahora mismo\ldots{}

---Vas a la huerta llamada \emph{del Pastelero}, donde te esperan
\emph{Mita} y \emph{Ley}, y con ellos tu novio\ldots{} Ya puedo llamarle
así\ldots{}

---Así debe llamarle. Voy a la huerta\ldots{} Ya me ha detenido usted
bastante\ldots{} Si es usted caballero, déjeme seguir mi camino.

---Tienes razón. No te estorbo en tu camino. Pero te digo que si vas hoy
a lo que llamas la pobreza, volverás de ella mañana. ¿Qué has de poder
tú contra tu Destino, tontuela?\ldots{} Sobre tus resoluciones, sobre
esos arranques fantásticos, de momento, prevalecerán las dos grandes
fuerzas que hay en ti. ¿No las conoces? Pues son la pasión del buen
vivir y la pasión de repartir el bien humano\ldots{} En la pobreza, ni
una ni otra de estas pasiones puede tener realidad\ldots{} Vete, corre a
donde te esperan el hombre enamorado y los amigos que fueron salvajes.
Ya no te detengo\ldots{} Anda, sigue tu camino. Sé que volverás\ldots{}
Media palabra, un recadito, una mirada de cualquiera de nuestros
dioses\ldots{} ¿No sabes qué dioses son estos? Los ricos, Teresa, los
inmensamente ricos, que te rondan sin que lo sepas tú\ldots{} Pues
cualquier insinuación de uno de estos te sacará de la pobreza honradita
y sosita para traerte a la deshonra brillante, tolerada\ldots{} Si lo
dudas, haz la prueba. Te acompaño hasta indicarte la vereda más corta
para ir a tu objeto.»

En pie, mirando a su amigo con cierto espanto, Teresa no se movía.
Tarfe, llevado ya por el hervor de sus ideas y de sus apetitos al punto
de la inspiración, de la sugestiva elocuencia, prosiguió así: «No
olvides lo que te he dicho, no una vez, sino veinte o más. Te lo dije en
tiempo del fardo, y después del fardo. Tú eres, Teresa, sin darte cuenta
de ello, el numen de la Unión Liberal; eres la expresión humana de los
tiempos\ldots{} Los millones de la \emph{Mano Muerta} pasarán por tu
mano, que es la \emph{Mano Viva}\ldots{} Mueve tus deditos, Teresa. ¿No
sientes en ellos el frío de los chorros de oro que pasan\ldots?

---No siento nada, Manolo; no siento nada---dijo Teresa, ceñuda,
estirando y encogiendo sus dedos como Tarfe le mandaba.

---Pues es raro. Los nervios de los ambiciosos se anticipan a la
sensación real, y el alma a los nervios. Eres tú la fatalidad histórica
y el cumplimiento de las profecías\ldots{} ¿No lo entiendes? Sin
entenderlo lo sentirás en ti, como sentimos el correr de la sangre por
nuestras venas\ldots{} Tú serás la ejecutora de lo que decimos y
predicamos yo y los de mi cuerda, los de mi partido, los que
evangelizamos el verbo de O'Donnell, que es el verbo de
Mendizábal\ldots{} No pongas esos ojos espantados, esos ojos que están
diciendo: «Yo desamortizo; yo quito del montón grande lo que me parece
que sobra, para formar nuevos montoncitos\ldots{} Yo soy la niveladora,
yo soy la revolucionaria\ldots{} Yo desplumaré a los bien emplumados
para dar abrigo a los implumes; yo quitaré el plato de la mesa de los
ahítos para ponerlo en la mesa de los hambrientos\ldots» Esto dices tú
sin saber que lo dices, y esto piensas creyendo pensar en las
musarañas\ldots{} Si otra cosa sientes hoy, es una humorada, un sentir
pasajero\ldots{} Vete a la pobreza; vete a ese juego inocente,
Teresilla, que de allí volverás, y si no vuelves pronto, alguien irá en
tu busca, y te traerá con sólo cogerte de un cabello y tirar de
ti\ldots{} Si tardas en volver, te buscarán los que te rondan, y dirán:
«¿Dónde está esa loca?\ldots» Y esta loca está jugando a la honradez
pobre, uno de los juegos más inocentes de la infancia. Juega a las
comiditas, a ir a la compra, y a remendarle los trapos al ganapán que la
llama su mujer. Vete, vete pronto, Teresa. Cuando vuelvas, me
encontrarás\ldots{} Yo te espero: iré a tu casa\ldots{} a tu nueva casa.
Adiós, gran revolucionaria, adiós.»

Dicho esto con el hechizo que reservaba para ciertas ocasiones, se fue,
dejándola sola en una vereda por donde sin cansancio podía llegar pronto
a su objeto. Desde lejos la saludó, y ella tuvo fijos en Tarfe sus ojos
hasta que le vio desaparecer. Siguió entonces por la vereda, cabizbaja:
lo que le había dicho \emph{O'Donnell el Chico} levantaba en su alma un
tumulto borrascoso. ¡Y qué cosas se le ocurrían, tan bien dichas y con
tan hondo sentido! Sin duda era Manolo un diablillo simpático, tentador,
que con permiso de Dios le sugería las ideas ambiciosas cuando ella
anhelaba ser modesta y despreciar las vanidades del mundo.

A cada rato se paraba Teresa y volvía sus ojos hacia Madrid. Poníase de
nuevo en marcha lenta, arrastrando sus miradas por los surcos del campo,
en que verdeaba la cebada raquítica para pasto de las burras de leche.
¿Quién era, o quiénes eran los magnates del dinero que la solicitaban?
Esto se decía, mirando a los surcos, y relacionando las indicaciones de
Tarfe con las vaguedades de Manolita Pez, y todo esto con la indirecta
que le soltó Serafina días antes. Fuera de ella y de su voluntad, había
sin duda una conspiración cuyo fin bien claro veía: faltábale sólo
conocer la persona. Según Tarfe, no se trataba del candidato de
Serafina, sino de otro de mayor vuelo y poder más brillante\ldots{} Loca
la habían vuelto entre todos; pero ella debía persistir en sus sanos
impulsos de moralidad, apresurando el paso para llegar pronto a la
presencia de Juan y de \emph{Mita} y \emph{Ley}, que confortarían su
alma turbada.

Pasaron junto a ella, y se le adelantaron, algunas familias pobres que
iban de merienda. Groseros le parecieron los hombres, desgarbadas las
mujeres, flacuchos y pálidos los niños. ¡Oh! ¿llegaría Teresa a verse
así, sin garbo ella, bárbaro su hombre, y degenerados sus hijos, si los
tenía? ¿El hambre y la privación de todo bienestar la llevarían a tan
triste estado? No quería ni pensarlo\ldots{} Entráronle súbitas ganas de
volverse a Madrid, y aun dio algunos pasos hacia atrás, movida de un
ardiente deseo de encararse con su madre y decirle: «¿Pero quién\ldots?»
Pronto se rehízo de esta instintiva inclinación al retroceso; siguió su
camino y\ldots{} pensando en el hombre aceleró el paso, como aceleran
las aves el vuelo cuando van al nido. ¡Vaya, que el pobrecito Juan
esperándola! ¡Qué impaciente estaría, qué inquieto, qué ansioso! ¿Y
\emph{Mita} y \emph{Ley}, qué pensarían de aquella tardanza? Ya eran las
doce, las doce y media. Tendrían ganas de comer; pero la
esperarían\ldots{} Sólo en el caso de que ella tardase mucho,
comerían\ldots{} ¡pero qué tristeza tener que ponerse a comer sin ella!

Llegó hasta donde veía las tapias de \emph{la huerta}, y lo mismo fue
verlas que sentir que los pasos se le acortaban por sí solos, hasta
llegar a detenerse en firme. Tuvo miedo; sintió la urgencia de resolver
y ordenar en su mente un aluvión de ideas que en ella entraron como
huéspedes alborotadores. Grande era el amor que sentía por Juan; mucho
le quería, mucho. Era bueno, sencillo, inteligente, capaz de todo lo
bello y noble\ldots{} Merecía la felicidad y cuantos bienes ha puesto
Dios en el mundo\ldots{} Pero si ella se metía en la vida pobre, ¿quién
había de dar estos bienes al honrado y amante Santiuste? ¿Quién cuidaría
de su alimento, quién le socorrería en sus desgracias? ¿Quién le
costearía las más brillantes carreras en el caso de que quisiese
dedicarse a la sabiduría? ¿Quién le pondría la gran tienda de armero en
el caso de que optase por la industria? ¿Quién le proporcionaría las
mejores ropas, los libros más instructivos, la casa cómoda y elegante, y
las mil frivolidades y pasatiempos que engalanan la vida?\ldots{} Tenía
que pensar en esto antes de lanzarse resueltamente en la vida pobre, y
para pensarlo despacio y poner cada idea en su punto, se apartó del
camino. La cosa era muy grave. Necesitaba recoger su espíritu\ldots{}
Tanto quiso recogerlo, que se fue a un altozano donde se alzaba un
artificio que parecía noria, entre pelados olmos. Sentose allí y meditó.

Pensando, se fijó en los grupos que merendaban en el prado próximo a la
huerta. ¿Quién cuidará de socorrer a tanto pueblo infeliz, si ella se
metía en el árido reino de la pobreza?\ldots{} ¡Cuánta miseria que
remediar, cuánta hambre que satisfacer, y cuánta desnudez que cubrir!
Ella, ella sola podía con mano solícita y diligente acudir a todo,
cogiendo a puñados lo que sobraba del montón grande, y\ldots{} No había
duda, no, de que era verdad lo que Tarfe le dijo. Como que Manolo era el
espíritu mismo y la esencia de O'Donnell el Grande, trasvasados a un ser
familiar, un tanto diablesco, rebosante de ingenio y de gracia.

¿Pero no era discreto y razonable que todas estas cosas se las dijese al
propio Santiuste, su amor único desde que vivía? Seguramente, cuando se
lo dijera, Santiuste le daría la razón, y le aconsejaría que se dedicase
pronto a las funciones de intérprete del verbo de O'Donnell, que era el
verbo de Mendizábal. La Humanidad aguardaba con ansia los beneficios que
la \emph{Mano Viva} de Teresita había de derramar sobre ella\ldots{}
Púsose en camino hacia \emph{la huerta}, cuyo tapial bien cerca veía;
pero a los pocos pasos la obligaron a nueva detención estas ideas: «Si
digo esto a \emph{Mita} y \emph{Ley}, no me comprenderán. Si lo digo a
Tuste, me comprenderá, pero después de explicaciones muy largas, que no
pueden hacerse en un día ni en dos. Juan tiene mucho talento, y ve las
cosas desde lo alto, desde lo más alto; pero idea como esta, ni Tuste,
con todo su entendimiento y su saber castelarino, la puede penetrar,
así\ldots{} de primera intención. Yo se la pondré bien clarita\ldots{}
pero no puede ser ahora\ldots{} ahora no\ldots»

Vaciló un instante, frunció el ceño, y al fin determinó que no pudiendo
decir lo que pensaba, debía volverse a Madrid. Frente a ella se extendía
la tapia de \emph{la huerta}, por el Este. Veía los tejados irregulares
de la casa, los chopos, los cuatro cipreses, de igual altura con muy
poca diferencia. El del extremo de la derecha subía un poquito más que
sus tres hermanos. Acercose Teresa aguzando el oído con intento de
percibir algún ruido del interior de \emph{la huerta}\ldots{} Oyó voces
confusas, pasos, cantos del gallo\ldots{} Su viva imaginación le fingió
imágenes precisas de lo que allí dentro pasaba. Juan, muerto ya de
impaciencia y desconfiado de que a tan avanzada hora llegase, se había
retirado del portalón, donde estuvo en acecho desde las diez, y abrumado
de tristeza se sentaba en el brocal del estanque, mirando las aguas
verdosas y el reflejo de los cuatro cipreses, tan rígidos y melancólicos
vueltos hacia el cielo bajo, como lo eran señalando al cielo alto con
afinada puntería. \emph{Mita}, sentada en la puerta de la casa,
expresaba con su inmovilidad, el codo en la rodilla, la cara recostada
en la palma de la mano, el aburrimiento de una larga espera. \emph{Ley}
paseaba por entre los chopos al niño, y le zarandeaba para alegrarle; el
perro corría tras ellos fingiendo alborozo, sin más objeto que aligerar
el tiempo\ldots{} Por fin, \emph{Mita} llamaba: ya no podían esperar
más. ¿Qué habrían llevado para comer? La imaginación de Teresa vaciló
entre figurarse la tortilla y un buen arroz, o el par de pollos
precedidos de ruedas de merluza\ldots{} Vio, sin dudar un punto, el
postre de polvorones que tanto gustaban a la amiga invitada; vio también
que, arrimados \emph{Mita} y \emph{Ley} al mantel tendido a la sombra
del moral, Juan negábase a comer\ldots{} Su tristeza le ponía un nudo en
la garganta y no podía tragar bocado. Los amigos le consolaban,
discurriendo las explicaciones más racionales de la tardanza de Teresa.
Los consuelos quedábanse en los oídos de Juan sin llegar al alma; esta,
empapada en amargura, agrandaba su pena hasta lo infinito, viendo en la
ausencia de la mujer amada algo tan solitario y desesperante como el
vacío de la muerte\ldots{} Mientras los otros comían, Juan, volviendo a
la puerta, asaltado de una débil esperanza, declamaba mentalmente
cláusulas altísonas, que lo mismo podían ser suyas que de Castelar.
Teresa las reproducía en su imaginación y en su memoria como si las
oyera: «Muerto el paganismo, el humano espíritu levanta el vuelo y corre
tras el cumplimiento de la ley de amor\ldots{} Amor le brindan los
cálices de las flores, amor la dulce onda de los sagrados ríos, amor la
conciencia pura de la mujer cristiana, Eva restaurada, virgen renacida
de las cenizas de la inmolada \emph{Venus\ldots»}

La idea de que Juan saliese a explorar el camino y la encontrara en
aquel acecho angustioso, le infundió tal vergüenza y terror, que
instintivamente se alejó a buen paso. Alborotada su conciencia, no
quería ver ni aun con la imaginación los rostros de sus inocentes
amigos, ni oír sus amantes voces. ¿Qué entendían ellos de los graves
conflictos del alma en lucha con todo el artificio social, y solicitada
de poderosas atracciones?\ldots{} Por el amor mismo que a Juan tenía, y
por la piedad intensa con que miraba el presente y el porvenir del
interesante mozo, amigo de su alma, no debía verle en tal
ocasión\ldots{} ¡A Madrid, a Madrid otra vez! Anduvo largo trecho muy
aprisa, siguiendo la mejor dirección para cambiar pronto de
perspectiva\ldots{} Al fin vio casas mezquinas y tapias de corrales, que
a cada paso aparecían en mayor número, como si ante ella surgieran del
suelo. Por un boquete de aquellas rústicas construcciones distinguió la
Plaza de Toros\ldots{} Como no había comido nada desde el desayuno, que
tomó muy temprano, sentía, sin tener apetito, los desmayos propios de un
cuerpo exhausto en día de tantas emociones. Una vieja, vendedora de
rosquillas, torrados y cacahuetes, le salió al paso. ¡Hallazgo feliz!
Con tres o cuatro rosquillitas y un poco de agua, pensó Teresa que se
sostendría muy bien hasta la noche. Cuando esto pensaba, vio aparecer
una aguadora. Ya tenía su lista de comida completa. En un banco de
mampostería del Paseo de la Ronda se sentó, una vez hecha la provisión
de rosquillas, que hubo de ser harto mayor de lo presupuesto, porque se
le acercaron multitud de chiquillos que le pedían \emph{chavos}, o pan,
y a todos obsequió. De la cesta de la vendedora pasaban las rosquillas a
la falda de Teresa, que las repartía graciosamente y con perfecta
equidad entre aquella mísera chusma infantil. Y cuanto más daba, mayor
número de criaturas famélicas y haraposas acudían, hasta formar en torno
a la guapa mujer una bandada imponente. La más contenta de esta invasión
fue la rosquillera, que viendo la pronta salida del género decía: «¡Ay,
señorita, hoy casi no me había estrenado, y con usted me ha venido Dios
a ver! Bien pensé yo, cuando la vi venir, que la señora se parecía a la
Virgen Santísima.»

Sin dar paz a su mano generosa, Teresa iba consolando a toda la
chiquillería. «Desnuditos y hambrientos estáis---les dijo.---Malos
vientos corren en vuestras casas\ldots» Contaban algunas chiquillas las
miserias de su orfandad, y las viejas vendedoras metieron baza,
lamentándose de lo malo que estaba todo. Si los hombres no tenían dónde
ganar para una libreta, ¿qué habían de hacer las pobrecitas mujeres? Con
gravedad bondadosa les dijo Teresita, dirigiéndose igualmente a las
ancianas y a los niños: ¿Pero no sabéis que ahora van a venir tiempos
buenos, muy buenos?» Ante la incredulidad de las viejas, Teresa repitió:
«Vendrá una cosa que llaman la Desam\ldots» No siguió, porque su
auditorio no entendería tal palabra\ldots{} «Señora, como eso que venga
no sea un alma caritativa, no sabemos lo que podrá ser\ldots» «Pues
eso---añadió la guapa mujer:---vendrán manos piadosas que cojan lo que
sobra de los montones grandes, y lo lleven a remediar tantas
miserias\ldots{} Creed que vendrá esa mano\ldots{} ya está cerca\ldots{}
casi está aquí ya.»

Con estos consuelos que daba a los menesterosos, se le fue a Teresa el
tiempo sin sentirlo\ldots{} Más de dos horas había permanecido en aquel
lugar, entre mocosos y viejas; la tarde declinaba; se veían grupos de
familias pobres que volvían ya de paseo con dirección al centro de
Madrid. Buscando la soledad, Teresa se metió por un callejón que a su
parecer debía de conducirla a la Veterinaria y al mismo sitio donde
estuvo sentada con Tarfe. Pero se había equivocado de sendero, pues el
callejón la condujo al Taller de coches, y costeando este, fue a parar
junto al Palacio de Salamanca, cuyo grandor y artística magnificencia
contempló largo rato silenciosa, midiéndolo de abajo arriba y en toda su
anchura con atenta mirada. En esto la sorprendió un movimiento de
ternura en lo más vivo de su alma, y acongojada apartó del palacio sus
ojos, que empezaron a llenársele de lágrimas: fue que se acordó del
pobre Juan y de los excelentes amigos, de honesta, sencilla y
semisalvaje condición. Trató de encabritar su espíritu abatido,
espoleándolo con esta idea: «Pobre Juan mío, yo haré por ti más de lo
que pudieras soñar\ldots»

Afirmando esto, vio multitud de carruajes que volvían de la Castellana.
Antes que en acercarse para ver bien a los que pasaban, pensó en
retirarse para no ser vista\ldots{} Entre una ligera neblina polvorosa,
Teresa vio pasar a la Navalcarazo, que llevaba en su coche a Valeria; a
caballo, al vidrio, iba Pepe Armada. Pasó después la Belvis de la Jara;
tras ella la Cardeña, tan linajuda como ricachona, en una berlina de
doble suspensión, elegantísima, de gran novedad\ldots{} Pasaron otras
que Teresa no conocía, y otros a quienes conocía demasiado. La Villares
de Tajo iba en el coche de la Gamonal, de la aristocracia de poco acá,
que deslumbraba con el brillo chillón del oro nuevo. Ambas señoras iban
muy emperifolladas, y llevaban en el asiento delantero de la berlina al
pomposo y magnífico Riva Guisando. Detrás iban a caballo, con toda la
gallardía andaluza, Manolo Tarfe y Pepe Luis Albareda. «¡Ay---pensó
Teresa, volviendo el rostro,---si llega a verme \emph{O'Donnell
Chiquito}, me luzco!\ldots» La Villaverdeja, la Monteorgaz, dejáronse
ver en la rauda procesión de vanidad; y por fin\ldots{} \emph{O'Donnell
el Grande}, en una vulgar berlina con doña Manuela\ldots{} Vio Teresa el
rostro del irlandés en la ventanilla, y en su imaginación le consideró
rodeado de un glorioso nimbo de oro y luz como el que ponen a los
santos. «Maestro, Dios te guarde---dijo la guapa moza con vago
pensamiento.---Toquemos a desamortizar\ldots{} Ya está aquí la
\emph{Mano Viva.»}

\flushright{Madrid, Abril-Mayo de 1904.}

~

\bigskip
\bigskip
\begin{center}
\textsc{fin de o'donnell}
\end{center}

\end{document}
