\PassOptionsToPackage{unicode=true}{hyperref} % options for packages loaded elsewhere
\PassOptionsToPackage{hyphens}{url}
%
\documentclass[oneside,8pt,spanish,]{extbook} % cjns1989 - 27112019 - added the oneside option: so that the text jumps left & right when reading on a tablet/ereader
\usepackage{lmodern}
\usepackage{amssymb,amsmath}
\usepackage{ifxetex,ifluatex}
\usepackage{fixltx2e} % provides \textsubscript
\ifnum 0\ifxetex 1\fi\ifluatex 1\fi=0 % if pdftex
  \usepackage[T1]{fontenc}
  \usepackage[utf8]{inputenc}
  \usepackage{textcomp} % provides euro and other symbols
\else % if luatex or xelatex
  \usepackage{unicode-math}
  \defaultfontfeatures{Ligatures=TeX,Scale=MatchLowercase}
%   \setmainfont[]{EBGaramond-Regular}
    \setmainfont[Numbers={OldStyle,Proportional}]{EBGaramond-Regular}      % cjns1989 - 20191129 - old style numbers 
\fi
% use upquote if available, for straight quotes in verbatim environments
\IfFileExists{upquote.sty}{\usepackage{upquote}}{}
% use microtype if available
\IfFileExists{microtype.sty}{%
\usepackage[]{microtype}
\UseMicrotypeSet[protrusion]{basicmath} % disable protrusion for tt fonts
}{}
\usepackage{hyperref}
\hypersetup{
            pdftitle={LA REVOLUCIÓN DE JULIO},
            pdfauthor={Benito Pérez Galdós},
            pdfborder={0 0 0},
            breaklinks=true}
\urlstyle{same}  % don't use monospace font for urls
\usepackage[papersize={4.80 in, 6.40  in},left=.5 in,right=.5 in]{geometry}
\setlength{\emergencystretch}{3em}  % prevent overfull lines
\providecommand{\tightlist}{%
  \setlength{\itemsep}{0pt}\setlength{\parskip}{0pt}}
\setcounter{secnumdepth}{0}

% set default figure placement to htbp
\makeatletter
\def\fps@figure{htbp}
\makeatother

\usepackage{ragged2e}
\usepackage{epigraph}
\renewcommand{\textflush}{flushepinormal}

\usepackage{indentfirst}

\usepackage{fancyhdr}
\pagestyle{fancy}
\fancyhf{}
\fancyhead[R]{\thepage}
\renewcommand{\headrulewidth}{0pt}
\usepackage{quoting}
\usepackage{ragged2e}

\newlength\mylen
\settowidth\mylen{...................}

\usepackage{stackengine}
\usepackage{graphicx}
\def\asterism{\par\vspace{1em}{\centering\scalebox{.9}{%
  \stackon[-0.6pt]{\bfseries*~*}{\bfseries*}}\par}\vspace{.8em}\par}

 \usepackage{titlesec}
 \titleformat{\chapter}[display]
  {\normalfont\bfseries\filcenter}{}{0pt}{\Large}
 \titleformat{\section}[display]
  {\normalfont\bfseries\filcenter}{}{0pt}{\Large}
 \titleformat{\subsection}[display]
  {\normalfont\bfseries\filcenter}{}{0pt}{\Large}

\setcounter{secnumdepth}{1}
\ifnum 0\ifxetex 1\fi\ifluatex 1\fi=0 % if pdftex
  \usepackage[shorthands=off,main=spanish]{babel}
\else
  % load polyglossia as late as possible as it *could* call bidi if RTL lang (e.g. Hebrew or Arabic)
%   \usepackage{polyglossia}
%   \setmainlanguage[]{spanish}
%   \usepackage[french]{babel} % cjns1989 - 1.43 version of polyglossia on this system does not allow disabling the autospacing feature
\fi

\title{LA REVOLUCIÓN DE JULIO}
\author{Benito Pérez Galdós}
\date{}

\begin{document}
\maketitle

\hypertarget{i}{%
\chapter{I}\label{i}}

\textbf{Madrid}, \emph{3 de Febrero de 1852}.---En el momento de
acometer Merino a nuestra querida Reina, cuchillo en mano, hallábame yo
en la galería del Norte, entre la capilla y la escalera de Damas,
hablando con doña Victorina Sarmiento de un asunto que no es ni será
nunca histórico\ldots{} La vibración de la multitud cortesana, un
bramido que vino corriendo de la galería del costado Sur, y que al
pronto nos pareció racha de impetuoso viento que agitaba los velos y
mantos de las señoras, y precipitaba a los caballeros a una carrera loca
tropezando en sus propios espadines, nos hizo comprender que algo grave
ocurría por aquella parte\ldots{} «Ha sido un clérigo,» oí que decían; y
en efecto, recordé yo haber visto entre el gentío, poco antes, a un
sacerdote anciano, cuyas facciones reconocí sin poder traer su nombre a
mi memoria\ldots{} Hacia allá volé, adelantándome a los que iban
presurosos, o tropezando con damas que aterradas volvían, y lo primero
que vi fue un oficial de Alabarderos que a la Princesita llevaba en alto
hacia las habitaciones reales. Luego vi a la Reina llevada en
volandas\ldots{} ¡Atentado, puñalada\ldots{} un cura! ¿Había sido herida
gravemente? Muerta no iba. Creí oírla pronunciar algunas palabras; vi
que movía su hermoso brazo casi desnudo, y la mano ensangrentada. Rápida
visión fue todo esto, atropellada procesión de carnes, terciopelos,
gasas, mangas bordadas de oro, tricornios guarnecidos de plata,
Montpensier lívido, el infante don Francisco casi llorando\ldots{} Al
Rey no le vi: iba por el lado de la pared, detrás del montón
fugitivo\ldots{} Vi a Tamames; creo que vi también a Balazote\ldots{}

Mi fogosa curiosidad de lo anormal, de lo extraordinario, de lo que
borra y destruye la vulgar semejanza de todas las cosas, me abrió paso,
a codazo limpio, hacia el grupo donde esperaba ver al criminal. No sé
cómo llegué: vi la cabeza cana de Merino, a la altura en que vemos la
cabeza del que está de rodillas; la vi luego subir, y tras ella negras
vestiduras nada pulcras\ldots{} Apenas distinguí el rostro\ldots{}
Llevaban al reo hacia la Sala de Alabarderos, por detrás empujado, por
delante a rastras. Entre tantas manos que querían conducirle, y al son
confuso de las imprecaciones y denuestos, se me perdió aquella figura
que yo quería ver en los instantes que siguen al punto trágico, ya que
en este punto mismo no logré verla. Quise entrar; no me dejaron. En
aquel momento me sentí cogido por el brazo, y volviéndome encaré con mi
suegro, el señor don Feliciano de Emparán, en quien reconocí la imagen
del terror: su boca era como la de una máscara griega, de la
guardarropía de Melpómene, y sus cabellos, si no los empobreciera la
calvicie, habrían estado en punta como las crines de un
escobillón\ldots{} «Figúrate---me dijo,---que lo he visto tan de cerca,
tan de cerca, que más no cabe\ldots{} Pasó Su Majestad\ldots{} la vi
pararse, la vi sonreír mirando hacia atrás, como si llamara a una
persona de la comitiva: esta persona era el Nuncio\ldots{} el Nuncio de
Su Santidad, que se adelantó pegándome un codazo por esta parte. Y
cuando me volví, por esta otra parte me dieron otro codazo. Era el
maldito clérigo, que se abalanzó, se arrodilló como para dar un
memorial\ldots{} le vi asestar la puñalada\ldots{} Creí que la tierra se
abría para tragarnos a todos\ldots{} No sé si la Reina cayó o no
cayó\ldots{} Nos abalanzamos al criminal\ldots{} Yo le oí decir\ldots{}
no sueño, no; yo le oí decir, no una vez, sino dos: \emph{Ya tienes
bastante.»}

Llegose a nosotros un gentilhombre regordete y chiquitín, a quien no
conozco. Hoy me ha dicho mi suegro su nombre; pero ya se me ha ido de la
memoria. Conservo en ella lo que aquel buen señor, tan corto de
presencia como largo de alientos vengadores, nos dijo con caballeresca
indignación: «Yo no entiendo estas pamplinas de la ley\ldots{} ¡Cuidado
con los trámites! ¿Procedía, sí o no, que le descuartizáramos aquí
mismo? ¿Pues no le vimos todos asestar el golpe, como una fiera? ¿Qué
duda puede haber? ¿A qué vienen esos interrogatorios y esos dimes y
diretes? ¡Si él no niega sus perversas intenciones! ¿Saben lo que dijo
cuando le levantábamos del suelo? Pues dijo: \emph{¡Oh, si hubiera en
Europa doce hombres como yo!} Por lo visto, su idea es matar a todos los
Reyes y al mismo Papa\ldots{} ¡Qué vergüenza, señores, para nuestra
Nación, donde jamás hubo regicidas!

---Perdone usted---estuve por decirle.---Regicidas hemos tenido en
nuestra Historia, y regicidas que han sido reyes, de lo cual resulta
algo que parece como un suicidio del Principio Monárquico.» Digo que
estuve a punto de expresar esta idea; pero me la guardé, observando que
no era prudente apear al buen señor de su remontada fiereza. Y él siguió
así: «No sé qué daría por que ese hombre no resultara español. Un
español puede ser todo lo depravado que se quiera; pero jamás atentará
con mano aleve a la vida de sus queridos Monarcas\ldots{} Y al fin,
contra un Rey, pase; pero contra una Reina, contra esta bondadosa Reina,
toda candor\ldots{} Lo que yo digo: es una furia del Averno vestida de
cura\ldots{}

---¡Y qué deshonra para el sacerdocio!---exclamó entonces mi suegro
echando toda su alma en un suspiro.---Daría yo\ldots{} no sé qué, porque
resultaran disfraz la sotana y hábitos de ese bandido; disfraz también
la corona que lleva en su cabeza. No pierdo la esperanza de que el
asesino haya tomado figura eclesiástica para poder engañarnos a todos, y
asestar el golpe con la más sacrílega de las traiciones. Y si no es
extranjero, téngolo por extranjerizado. Lo que yo vengo diciendo,
señores; lo que a ti te he dicho mil veces, Pepe: \emph{he aquí} el
fruto de tanto folleto, de tanto \emph{virus} demagógico; \emph{he aquí}
lo que nos traen esos malditos periódicos, donde meten la pluma
pelafustanes cuya ciencia no es más que unas miajas de francés\ldots{}
eso\ldots{} y vengan acá cuantos delirios corren por el mundo\ldots{}
todo ello sin censura, sin permiso del Ordinario ni nada\ldots{} Así
está España medio loca ya, y así nos llega cada día una calamidad,
primero \emph{enciclopedistas}, luego la gaita esa de que \emph{la
propiedad es un robo}; y, por fin, estos monstruos\ldots{} el
Apocalipsis\ldots»

Cedió la palabra don Feliciano a un alabardero, que con noticias frescas
del asesino, por haber oído sus primeras declaraciones, fue acometido
por los curiosos insaciables. «Es español---nos dijo,---riojano por más
señas, y cura. Se llama Martín Merino; dijo misa esta mañana. Al salir
de su casa juró que no volvería sin matar a la Reina, o a la Reina
madre, o a Narváez\ldots» Nada consternó tanto a mi señor suegro como
que el asesino fuera real y efectivo sacerdote, con la agravante
sacrílega de haber celebrado aquella mañana; y cuando el alabardero, y
otro que vino detrás, dijeron que Merino era exclaustrado y había vivido
en Francia muchos años, desempeñando un curato, rompió en estas o
parecidas exclamaciones: «¿No lo decía yo? ¡Enciclopedia, demagogia, con
su poco de \emph{Espíritu del Siglo}, cosas que no existían en España
cuando ésta era una Nación de caballeros, que no mataban a sus reyes,
sino que por ellos morían!»

Nos dirigimos luego a la Saleta, y en ella el mismo gentilhombre
iracundo y enano, de cuyo nombre no puedo acordarme, vino a decirnos que
la herida de la Reina no era de cuidado; que el puñal sólo penetró tanto
así\ldots{} que habiendo sufrido Su Majestad un desvanecimiento, los
médicos procedieron a sangrarla, y\ldots{} en suma, que tendríamos Reina
para un rato. Con esto nos volvió el alma al cuerpo a mi padre político
y a los que con él estábamos. Frenéticos vivas a Isabel sonaron en la
Saleta y Antecámara, y a la consternación sucedieron esperanza y
regocijo, sólo turbados por el anhelo que a muchos abrasaba de la
inmediata matazón del malvado clérigo.

Vimos llegar jadeantes y ceñudos al Presidente del Consejo, Bravo
Murillo, y a González Romero, Ministro de Gracia y Justicia, que estaban
en Atocha esperando a Su Majestad; y recibido allí por veloces correos
el jicarazo de tan descomunal crimen, corrieron a Palacio en ansiedad
mortal. Fue su primer cuidado visitar a la Reina en su cámara; y una vez
informados de que mayor daño había recibido de la emoción del lance que
del puñal de Merino, se encaminaron a donde éste aguardaba que le cayera
encima la nube de jueces y escribanos para decirles: «Caballeros,
mátenme de una vez, pues yo no he venido a otra cosa, y cuanto menos
conversación, mejor.» Poco después de ver entrar a los dos Ministros en
la Sala de Alabarderos, corrió de boca en boca, por la galería, una
opinión que pronto tuvo adeptos, inclinándose a ella los mismos
partidarios de la venganza instantánea. «No se le puede matar sin
proceso, y el proceso no puede ser corto, porque ha de haber
cómplices\ldots{} Esto no es un hecho aislado\ldots{}

---\emph{Yo abundo} en esa idea---me dijo mi suegro,---y no dudo que en
ella \emph{abundarás} tú también. Aquí hay complot\ldots{} y complot de
ramificaciones muy obscuras.» Con el honrado objeto de adquirir mayores
luces sobre el particular, quise penetrar en la Sala, pegado a los
faldones de un alabardero amigo mío. Pero se me negó la entrada, y de
aquí tomó pie don Feliciano para decirme: «Ya nos lo contarán, hijo.
Vámonos a casa, que a estas horas habrá corrido por todo Madrid, como
reguero de pólvora, la noticia de esta \emph{hecatombe}, y Visita y tu
mujer estarán con cuidado.»

\emph{5 de Febrero}.---He creído siempre que el pueblo español ama
verdaderamente a su Reina. Pero hasta hoy, ante el reciente suceso que
mi suegro llama \emph{hecatombe}, no había yo visto clara la exaltación
de ese cariño, que raya en idolatría. Hay que leer las manifestaciones
de los pueblos, que nos trae la \emph{Gaceta}, y el lenguaje que emplean
algunos alcaldes en sus protestas contra el atentado. Uno empieza
diciendo: «¡Qué horror! ¿Y aún vive el regicida?» Luego llama a éste
«monstruo vomitado del Infierno,» y pide que le maten a escape. Domina
en todas las protestas, al lado de las imprecaciones violentísimas, la
implacable sentencia popular: «Matarle, descuartizarle, hacerle polvo.»
Y otro funcionario exclama, dejando caer sus lagrimones sobre el papel
de oficio: «¡Querer quitarnos la mejor de las Reinas, la joya, la prenda
más querida de todos!» Y esto es sincero, esto sale de los corazones, y
nos retrata al pueblo español como un enamorado de su Reina: Isabel es
hija, hermana y madre en todos los hogares, y como a un ser querido y
familiar se le rinde culto. Sábelo, Isabel; hazte cargo de que este
sentimiento lo tienes por ti sola, no por tu padre, que se pasó la vida
haciendo todo lo posible para que le aborreciéramos, ni por tu madre,
más admirada que amada; acoge en tu corazón este sentimiento y
devuélvelo, como un fiel espejo devuelve la imagen que recibe. Consagra
tu vida y tus pensamientos todos a satisfacer a este sublime enamorado y
a tenerle contento. Aprovecha este amor purísimo, el mejor de los
innumerables dones que has recibido de la Divinidad, y no lo
menosprecies ni lo arrojes en pedazos, como la cabeza y las manos de las
lujosas muñecas con que jugabas cuando eras niña. Esto no es cosa de
juego. Eres muy joven, y tu juventud vigorosa te anuncia un largo
reinado. Mira lo que tienes, mira lo que haces, y mira con lo que
juegas.

Pues en Madrid no hay más tema de conversación que los partes de la
Facultad de Palacio, anunciando que la Reina se restablece del sofoco y
de la puñalada, y la sabrosa comidilla del proceso de Merino, sobre cuya
criminal cabeza sigue la opinión arrojando los anatemas más horribles.
Hasta los niños le llaman ya \emph{monstruo abortado} y \emph{oprobio de
la Naturaleza}. Todos sus dichos y actitudes en la cárcel son comentados
como nuevos signos de perversión; a su serenidad se la llama
\emph{insolencia} procaz; a sus graves ratificaciones de
responsabilidad, \emph{brutal cinismo}. Al juez instructor respondió,
entre otras cosas abominables, «que había ido a Palacio a lavar el
oprobio de la Humanidad, y a demostrar la necia ignorancia de los que
creen que es fidelidad \emph{aguantar} la infidelidad y el perjurio de
los Reyes.» A su abogado le dijo que no buscara motivos de defensa,
porque no los había; que si se empeñaba en defenderle por loco, él se
encargaría de demostrar lo contrario. No estaba arrepentido; no tenía
cómplices, ni recibía inspiraciones más que de su propia inquina, del
aborrecimiento de toda injusticia y del mal gobierno de la Nación. Era
una víctima de las leyes mentirosas que desamparan al débil; había
sufrido ultrajes y reveses sin que nadie le amparase; detestaba toda
autoridad; no tenía rencores contra Isabel; pero sí contra la Reina por
serlo, y contra Narváez, que nos había traído innumerables ignominias y
desventuras. No temía la muerte, y al notificársele la sentencia, decía:
«Pues encarguen que el patíbulo sea muy alto.» Al subir a él diría:
«Imbéciles, os compadezco, porque os quedáis en este mundo de corrupción
y de miseria.»

Estos dichos y réplicas comenta la gente, dándoles un sentido de
infernal depravación. Ya echan también su cuarto a espadas los poetas.
Uno de éstos nos habla del \emph{Tártaro}, el cual, no sabiendo qué
hacer un día, se distrae \emph{abortando al sacrílego}, el cual sale de
allí, \emph{armada la mano impía}, sin más objeto que arruinar a
España\ldots{} Otro ve venir a un \emph{tigre} disfrazado---\emph{con el
sacro vestido---del sacerdote del Señor Eterno}; y sospechando por su
actitud sus dañadas intenciones de matar a la \emph{tierna cordera},
empieza a dar gritos llamando al \emph{León de España} para que
\emph{saque la garra} y\ldots{} etc. Al son de la Poesía, aunque no con
acentos tan roncos y desatinados, viene la Política, que ante este grave
suceso, que parece un aviso de la Fatalidad, ha borrado la vana
diferencia y mote de partidos, fundiéndose todos en la emoción unánime
por la Reina en peligro, por Isabel amenazada de un puñal
alevoso\ldots{} Da gusto ver los periódicos clamando contra el
delincuente, y ofreciendo al ídolo nacional los homenajes de respeto y
amor más ardientes y sinceros. Sobre todo interés de bandos o grupos
está la salud y la vida de la Soberana. Por esto dice muy bien \emph{El
Heraldo} que \emph{se ha suspendido la oposición}.

¿Ves, Isabel? Todos te quieren, así los que están de servilleta prendida
en la mesa del Presupuesto como los que ha largos años contemplan lejos
del festín las ollas vacías. Todos te aman; en todo corazón español está
erigido tu altar. Míralo bien y advierte lo que esto significa, lo que
esto vale. Considera, Isabel, a cuánto te obliga ese amor, y con qué
pulso y medida has de ejercer el poder, la autoridad y la justicia que
tienes en tus bonitas manos. Aviva el seso, Reina, y no juegues.

\hypertarget{ii}{%
\chapter{II}\label{ii}}

\emph{7 de Febrero}.---A mi dulce amiga invisible, la indulgente
Posteridad, doy anticipada explicación de los vacíos o faltas que notará
en mis vagas \emph{Memorias} cuando llegue a leerlas, si tal honra
merezco al fin. Creerá que es mi correo el viento; que a él las confío
en descosidas hojas, y que algunos puñados de éstas se le van cayendo en
su carrera por los espacios. Pues no es así, que buen cuidado pongo en
que todo vaya bien ordenadito, no por caminos del aire, sino por manos
de depositarios y conductores diligentes. La causa de estos vacíos debe
buscarse en la propia morada y época del autor, que ha visto perseguido
y condenado a destrucción su trabajo, fruto de tantas observaciones y
vigilias. Sepa la Posteridad que ha dos años padecí alteraciones de mi
salud, cuyo proceso y síntomas fueron gran confusión de los médicos de
casa; y tan desconcertado me puse, que mi amada esposa y mi bendita
suegra llegaron a creer que yo había perdido el juicio, o que mis
tenaces melancolías y desgana de todo me llevarían pronto a perderlo.
Inquietísimas las dos señoras, como el buen don Feliciano y las damas
mayores, no sabían qué hacer para mi asistencia; todo su tiempo y su
atención eran para vigilarme y no perder de vista la más insignificante
acción mía, por donde pudieran descubrir mis alocados pensamientos. En
aquellos trances me vino una crisis de flojedad de todo mi cuerpo y de
fatigas intensas, que me tuvieron preso y encamado largos días; y en lo
que duró mi quietud hubo tiempo sobrado para que María Ignacia y doña
Visita, que veían en mis persistentes lecturas y en mis nocturnas
encerronas para escribir la causa inmediata de mis achaques,
discurrieran algo semejante a lo que el ama y sobrina de don Quijote
imaginaron para cortar de raíz el morboso influjo de los libros de
Caballerías. Registraron mi cuarto, y una vez sustraídos bastantes
libros de los que más me deleitaban, abrieron con traidora llave uno de
los cajones en que guardaba yo mis papeles, y todo lo que allí
encontraron perteneciente a mis Memorias fue reducido a cenizas. Me ha
dicho después María Ignacia que no fue ella, sino su madre, la verdadera
inquisidora de aquel auto; que había intentado salvar algunas piezas de
mi escritura; pero que doña Visita y don Feliciano se las arrebataron al
instante, pronunciando la terrible sentencia: «¡Al fuego, al fuego!»

Sin tratar de averiguar quién tuvo más culpa en aquel desaguisado, me
limité a llorar la pérdida de todo lo que escribí en el 50 y en parte
del 51, porque en ello puse, a mi parecer, pensamientos muy míos que no
merecían fin tan desastrado. Lo restante del 51 lo pasamos en Italia, y
allí nada escribí, porque mi mujer me quitaba de la mano la pluma
siempre que yo intentaba contarle algún cuentecillo a mi amiga la
Posteridad. Permita Dios que esta nueva ristra de memorias sea más
afortunada, y permanezca segura de incendios. Así lo espero, alentado
por María Ignacia, que, oídas mis explicaciones, me ha prometido
respetar mi trabajo siempre que observe yo dos reglas de conducta por
ella impuestas. La primera es que no consagre a este recreo cerebral más
que hora y media, a lo sumo dos horas en cada veinticuatro; la segunda,
que no reserve de su curiosidad mis papelotes, reconociéndole el derecho
de revisión, censura y aun de enmienda si fuere menester\ldots{} Mi
amada mujer, a quien he confiado mis pensamientos más íntimos, no me
tiene por lunático, y a cuantos en la Posteridad me leyeren les aseguro
que no lo soy ni jamás lo he sido. Divago a mis anchas, eso sí, y digo
todo lo que me sale de dentro, sin que me asuste la chillona inarmonía
entre mis ideas y las de mis contemporáneos.

Si con los más suelo estar en desacuerdo, con mi señor padre político
desentono horrorosamente, pues jamás dice él cosa alguna que a mí no me
parezca un disparate. Al propio tiempo, cuanto sale de mi boca es para
él herejía, delirio, necedad garrafal. Vaya un ejemplo: ayer mismo,
hallándonos de sobremesa del almuerzo, con dos convidados, mi hermano
Agustín y don Clemente Mier, dignidad de Capiscol de la catedral de
Toledo, sacó mi suegro un papel y nos leyó la sentencia del cura Merino:
«Fallamos: que por fundamentos y artículos tal y tal\ldots{} debemos
condenar y condenamos\ldots{} tal y tal\ldots{} a la pena de muerte en
garrote vil\ldots{} que el reo \emph{sea conducido al patíbulo con hopa
amarilla y un birrete del mismo color, una y otro con manchas
encarnadas}\ldots»

Al oír esto, dije tales cosas que don Feliciano me quería comer, y salió
con la tecla de que no sigo bien del caletre. «¿Qué razón
hay---añadí,---para que se vista de máscara, con escarnio repugnante, a
un pobre reo, que bastante castigo tiene ya con la muerte?» Y como el
clérigo comensal y mi hermano afirmasen que ello era formas y ritualismo
de la ley, para inspirar más horror del crimen que se castigaba, y mi
suegro triunfante nos leyese que lo de la hopa amarilla con llamas rojas
lo disponía el Código en su artículo 91, sostuve que somos un país
bárbaro, donde la justicia toma formas de Inquisición, y los
escarmientos de pena capital visos de fiesta de caníbales. Dentro de
cada español, por mucho que presuma de cultura, hay un sayón o un
fraile. La lengua que hablamos se presta como ninguna al escarnio, a la
burla y a todo lo que no es caridad ni mansedumbre. Aún despotriqué más;
pero ahogaron mis expresiones con risas, saliendo por un registro que
iniciaba siempre mi mujer: «Cosas de Pepe.» Yo tengo cosas, y con este
comodín puedo dar suelta, sin gran escándalo, a cuantos absurdos bullen
en mi mente. El canónigo Mier, hombre ilustrado y tolerante, fue de los
que más celebraron mis ocurrencias, y a renglón seguido me dijo que,
designado por el arzobispo Bonell y Orbe para asistir a la ceremonia de
la degradación del cura Merino, la cual había de ser muy interesante y
patética, me proponía llevarme consigo, si yo lo deseaba. Aunque ordena
el Concilio de Trento que estos ejemplares actos sean públicos, en el
caso presente no se abrirán las puertas de la cárcel más que a los que
asistan por ministerio eclesiástico, y a contadas personas que quieran
presenciar la ceremonia, no por curiosidad, sino por edificación. Me
apresuré a contestarle afirmativamente, y quedamos de acuerdo en hora y
punto de cita para el mismo día.

Agustín, que a más de hermano mío lo es de la Paz y Caridad, contonos
ayer que, habiendo visto en su calabozo \emph{al monstruo del Averno},
salió de allí escandalizado y horrorizado de \emph{un cinismo tan
infernal}. No se contenta Merino con repetir que quiso matar a la Reina
por \emph{vengar en ella las iniquidades de los que mandan, y por
aversión al género humano}, sino que ha declarado con el mayor descoco
que desde su entrada en el convento, siendo aún niño, leyó cuantas obras
prohibidas le vinieron a las manos, filosofismos y herejías de lo peor
que \emph{abortan} las prensas francesas. Luego se dejó decir que en su
juventud \emph{estuvo enamorado de la Libertad}; que por huir de
persecuciones se largó dos veces a Francia el 41, empleó en préstamos el
dinero de sus ahorros y algo que ganó en la Lotería, siendo tan
desgraciado en sus negocios, que los acreedores, sobre no pagarle, le
pegaban\ldots{} Sufrió vejaciones, malos tratos, estafas y vituperios
mil, con lo que se le fue corrompiendo la sangre, y se llenó de hiel.

En sus últimos años, no tenía trato de gentes; se pasaba el día echando
amargos ayes de su boca; quejábase continuamente de enfermedades
efectivas y de otras imaginarias. Su genio era tan agrio, que no había
cristiano que le aguantase\ldots{} Dormía tan poco, que sus descansos
salían a hora y media no más por noche, y entretenía los insomnios con
lecturas continuas de cuanto papel en sus manos caía\ldots{} En la
cárcel afecta fría tranquilidad, desprecio de la vida, desdén de
escribanos y jueces, de su propio defensor, y hasta del señor Presidente
del Tribunal Supremo, don Lorenzo Arrazola, lo que verdaderamente revela
un \emph{orgullo más que satánico}\ldots{}

Mucho agradecí al buen amigo señor Mier que me facilitara ocasión de ver
al preso en acto tan imponente y severo. Consistía la degradación en
despojarle de la investidura carácter sacerdotal, para que pudiera ceñir
sin mengua de la Iglesia la hopa amarilla que ordena la etiqueta del
cadalso, según los artículos 160 y 91 de nuestro benigno Código penal. A
la hora designada para degradar entramos en el Saladero el señor Mier y
yo, y nos encaminaron a una sala baja con rejas a la calle: en el
testero principal vimos un altar, y sobre éste ropas litúrgicas, un
cáliz, un crucifijo y dos velas. No tardó en llegar el señor Cascallana,
Obispo de Málaga, con media docena de graves sacerdotes, que habían de
asistirle, y casi al mismo tiempo se personaron el juez señor Aurioles,
los Gobernadores civil y militar, el Fiscal, escribanos y algunas
personas que no llevaban más cargo que el de mirones, ni otro fin que el
de saciar su curiosidad ardiente. En la calle, numeroso gentío ansiaba
ver cosa tan extraordinaria. Pocos eran los que algo podían vislumbrar
pegados a las rejas; muchos los que empujaban disputando sitio a los que
habían madrugado para cogerlo; muchísimos los que renegaban de no ver
más que la pared, detrás de la cual pasaba algo terrible. Juntándose al
murmullo y risotadas de los menos el mugido displicente de los más,
resultaba un coro de crueldad y grosería que nos daba la sensación de
los autos de fe.

El Obispo se revistió de medio pontifical rojo, con báculo y mitra, y
ocupó un sillón de espaldas al altar; los demás curas situáronse a
izquierda y derecha; yo me agazapé en sitio donde pudiera ver quedándome
casi invisible, y ya no faltaba más que el reo, parte o figura
indispensable del edificante espectáculo que debíamos presenciar.

Tras una breve espera, vimos aparecer la figura escueta y pavorosa de
don Martín, alto, rígido, el cuerpo todo negro de la sotana, amarillo el
rostro de las hieles que le andaban por dentro, la mirada viva, la
expresión desdeñosa. Traía las manos atadas atrás, y del nudo que las
enlazaba partían dos cuerdas, una para cada pie. Con esta sujeción su
paso era lento, como el de un gran buitre que, inutilizado de las alas,
se viera en la penosa obligación de hacer su camino por el suelo. Cuando
pude verle de perfil y de frente, reconocí la fisonomía del clérigo que
en 1848 prestó los auxilios espirituales a la pobre Antoñita en la
triste casa de la plaza Mayor. Él no me vio a mí, y aunque me viera, no
me habría reconocido. Diré con toda verdad que su presencia en la sala
del Saladero levantó en mi espíritu el terror más que la compasión; casi
casi encontré apropiadas a su persona las calificaciones de
\emph{monstruo abortado}, \emph{tigre}, y demás remoquetes que la Prensa
había hecho populares. El maestro de ceremonias, con su libro en una
mano y el puntero en la otra, se adelantó hacia el reo y desabridamente
le dijo: «Tiene usted que vestirse.» Y el reo, más desabrido aún,
contestó: «¿Y cómo? ¿con las manos atadas?» Los alguaciles desliaron la
cuerda de sus manos; y en cuanto éstas estuvieron libres, llegose el
hombre al altar y empezó a vestirse con pausa y método, sin la menor
alteración en los ademanes, lo mismo que si se vistiera para decir misa,
pronunciando con voz segura la frase de ritual que el celebrante dice a
cada prenda que se pone. El amito, el alba, el cíngulo, la estola, el
manípulo, tienen simbólica significación, que el sacerdote va expresando
al tomar la figura de Cristo en aquella oblación pura, que no se puede
manchar por indignos y malvados que sean los que la hacen. Sereno estaba
el hombre, repitiendo en tan lúgubre ocasión lo que hacía todas las
mañanas en San Justo o en otras iglesias; y como el acólito se
equivocara queriendo ponerle el manípulo en el brazo derecho, le dijo
pronta y secamente: «Al brazo izquierdo.»

Vestido, el reo parecía otro. Su rostro huraño y repulsivo recibía no sé
qué vislumbre apacible de la casulla que cubría su cuerpo. Se le mandó
que se arrodillara, y obedeció al instante, hincándose frente al
Prelado. «Más cerca, más cerca,» le ordenó el maestro de ceremonias.
Obedeció tan vivamente Merino, andando de hinojos hacia Cascallana, que
llegó hasta tocarle las rodillas. Asustado el Obispo de aquel bulto que
se le iba encima con salto parecido al del cigarrón de zancas aceradas,
rebotó en su silla, se puso en pie, tuvo miedo. Pensó quizás que el
asesino de Isabel sacaba de la casulla otro puñal de Albacete\ldots{} El
Gobernador, don Melchor Ordóñez, se arrimó a Su Ilustrísima, que
tranquilizado recobró su asiento. En esta parte de la escena advertí un
ligero matiz cómico, que anoto aquí para que nada se me escape. Pasó
como una fugaz mueca de Melpómene, si en el momento de su actitud más
trágica la picara una pulga. Inmediatamente después de esto, Merino se
fijó en las caras de chiquillos, de descocadas mujeres y pálidos hombres
que aparecían en las rejas, y en el siniestro rumor del pueblo ansioso.
«¿Hay alguna rúbrica---dijo---que disponga que estos actos se celebren a
la luz del día y con los balcones abiertos?» Nadie le dio respuesta ni
explicación. Un señor que a mi lado estaba, viendo que el reo se encogía
de hombros y alargaba el labio inferior con un expresivo \emph{¿a mí
qué?}, me dijo: «Pero ¿ha visto usted qué monstruo de frescura?»

Empezaron sus terribles funciones los que degradaban, y lo primero fue
ponerle a don Martín en las manos un cáliz con vino y agua, que al punto
le arrebató el Obispo, dándole después el copón, que con la misma
prontitud le fue quitado. El señor Cascallana pronunció la fórmula en
latín, que traducida fielmente dice: \emph{Te quitamos la potestad de
ofrecer a Dios sacrificio, y de celebrar la misa tanto por los vivos
como por los difuntos}. Inmediatamente cogió Su Ilustrísima un
cuchillito que le dieron los acólitos, mandó a don Martín que alargase
los dedos y se los raspó suavemente, acompañando el acto de estas
desconsoladoras palabras: \emph{Por medio de esta rasura te arrancamos
la potestad de sacrificar, consagrar y bendecir, que recibiste con la
unción de las manos y los dedos}. Luego se le quitó la casulla, y el
Obispo dijo: \emph{Te despojamos justamente de la caridad, figurada en
esta sacra vestidura, porque la perdiste, y al mismo tiempo toda
inocencia}. Y al arrancarle la estola: \emph{Arrojaste la señal del
Señor, figurada en esta estola: por esto te la quitamos}\ldots{}

Como el ceremonial que describo es a la inversa de la imposición del
Sacramento del Orden, deshaciendo y desbaratando todo lo que éste
significa, hasta privar al condenado de la dignidad, carácter y oficio
sacerdotales, para entregarlo abiertamente \emph{al fuero de los legos},
luego que se le quitó a Merino la calidad de Presbítero la emprendieron
con el Diácono. Revestido con la dalmática, y puesto el libro de los
Evangelios en las manos pecadoras, lo arrebató de ellas el Obispo con
estas aterradoras palabras: \emph{Te quitamos la potestad de leer el
Evangelio, porque esto no corresponde a los indignos}. Y al despojarle
de la dalmática: \emph{Te arrancamos con justicia la cándida vestidura
que recibiste para llevarla inmaculada en la presencia del Señor, porque
no lo hiciste así conociendo el misterio, ni diste ejemplo a los fieles
para que pudieran imitarte como consagrado a Dios}. Al desnudar al
Subdiácono, la tremenda voz de la Iglesia dijo: \emph{Te desnudamos de
la túnica subdiaconal, porque el casto y santo temor de Dios no domina
tu corazón y tu cuerpo}. Arrebatado le fue el manípulo con esta
cláusula: \emph{Deja el manípulo, porque no combatiste las asechanzas
del enemigo por medio de las buenas obras}; y el amito con ésta:
\emph{Porque no refrenaste tu voz, te quitamos el amito}.

Aún no había concluido la terrible escena. Vi que por las mejillas del
Prelado resbalaban dos gruesos lagrimones. Las de Merino estaban secas:
su cara, como una escultura tejida con esparto, imitaba la impasibilidad
del cadáver que no chilla ni remuzga cuando le pinchan y le sajan en la
sala de disección. El señor que a mi lado estaba me dijo: «Esto no es un
hombre, ni siquiera una fiera: esto es un árbol. Fíjese usted: Su
Ilustrísima llora; yo, que no soy aquí más que \emph{mero espectador},
lloro también\ldots{} no puedo contenerme, y él como si tal cosa\ldots{}
Vea usted, es un árbol\ldots» Nada pude contestar a mi vecino: tales
eran mi emoción y el ansia que yo sentía de que la desgarradora escena
terminase.

\hypertarget{iii}{%
\chapter{III}\label{iii}}

Como dije, aún faltaban los últimos trámites para despojar al reo de las
insignias de los grados menores y de la primera tonsura. El despiadado
simbolismo era largo como toda la carrera eclesiástica\ldots{} al revés.
La Iglesia había de borrar hasta la última señal de unción divina en el
réprobo que expulsaba de su seno. Cuando, puestas y quitadas las
insignias de estos grados, quedó el reo con sobrepelliz, al despojarle
de ésta se levantó el Obispo, y entonando la voz todo lo que le permitía
su emoción vivísima, pronunció este tremendo anatema: \emph{Por la
autoridad de Dios Omnipotente, Padre, Hijo y Espíritu Santo, y la
muestra, te deponemos, te despojamos y te desnudamos de todo orden,
beneficio y privilegio clerical; y por ser indigno de la profesión
eclesiástica, te devolvemos con ignominia al estado y hábito seglar}.
Acto seguido, los acólitos le quitaron la sotana y el alzacuello, y el
hombre quedó en chaquetón, figura lastimosa. Permaneció inmóvil, como
esperando que le arrancaran también el pellejo. El señor Cascallana,
armado de tijeras, le cortó un mechón de cabellos, y al punto uno de los
alguaciles trasquiló la parte superior de la cabeza, hasta borrar en lo
posible el redondel de la corona. El chis-chas de las tijeras me daba
frío, como si me estuvieran trasquilando a mí. Aún no se aplacaba la
terrible indignación de la Iglesia, que con iracundo estilo pronunció
este anatema al compás de los tijeretazos: \emph{Te arrojamos de la
suerte del Señor, como hijo ingrato, y borramos de tu cabeza la corona,
signo real del sacerdocio, a causa de la maldad de tu conducta}.

Ya parecía que todo terminaba. Oí suspiros y toses de los concurrentes,
autoridades o curiosos; oí el mugido del pueblo que no veía nada desde
la calle (salvo algunos privilegiados adheridos a las rejas) y que se
conformaba con aspirar la trágica emoción rezumándola al través del
espeso muro del Saladero. Merino, requiriendo las solapas de su
chaquetón, dijo de mal talante: «Despachemos, que me voy quedando frío.»
El mísero reo no podía abrocharse, porque al ingresar en la prisión, los
guardianes le habían arrancado los botones del chaquetón: parece que es
costumbre carcelaria, para evitar el suicidio, que algunos reos han
consumado tragándose los botones. Apenas dijo esto, resonó un
estruendoso \emph{¡viva la Reina!} en la plaza, y después otro en los
patios del edificio. Don Martín permaneció impasible ante las
exclamaciones: sólo sentía frío\ldots{} Cuando le vi salir, de nuevo
maniatado, el horror que al entrar me había inspirado dio paso a la
compasión más viva. En su impavidez no vi cinismo, ni en su frialdad
insolencia, sino más bien una entereza estoica de que yo no he conocido
hasta hoy ningún ejemplo. Ansiaba ya dar espacio refrigerante a mi
espíritu lejos de aquel ambiente inquisitorial, patibulario. Los
eclesiásticos degradadores, y los acólitos y alguaciles que desnudaban y
trasquilaban al reo, traían a mi mente imágenes, no sé si soñadas o
reales, de las más siniestras figuras de la Edad Media. Salí con un
remolino de confusiones en mi cabeza, y tan pronto me parecía natural,
justo y humano que a Merino se le indultase después de la feroz
ejecución espiritual de aquel día, tan pronto anhelaba su muerte,
viéndola como un holocausto grande y bello; pero no se le había de matar
en garrote ni por los medios usuales, sino con hacha\ldots{} La
comparación con un árbol expresada por el caballero desconocido, no se
apartaba de mi mente. Yo quería ver si el estoico, como el tronco
herido, crujía y soltaba un ¡ay! al recibir el hachazo.

Despedime del señor Mier, a quien el Obispo llevaba en su coche, y a mí
me ofreció el suyo y su grata compañía don Melchor Ordóñez, que iba al
Gobierno Civil. Acepté, y rodando por el pedernal de estas malditas
calles me dijo el simpático Gobernador: «Pero ¿ha visto usted qué tío?
No creo que exista en el mundo otro con más agallas. El Obispo se hacía
cruces viendo la fibra de este hombre. Me ha contado que en tiempos
antiguos hubo clérigos delincuentes que ante la espantosa sofoquina de
la degradación perdieron el conocimiento, y uno hubo, en Italia o no sé
dónde, que cayó patas arriba y le recogieron cadáver\ldots{} Pero este
tío, ya usted le vio: como si estuviera el sastre tomándole medida de un
traje nuevo.» Dije yo que, en efecto, es un caso estupendo de dominio
sobre sí mismo. Y él a mí: «¡Ah! Pero ¿no sabe usted lo mejor? Esta
peña, este tronco de acebuche, era un manojito de sensibilidad cuando
estuvo enamorado del ama que le sirvió hace algunos años\ldots{} ¡Oh!
había usted de ver las cartas, Aurioles las tiene. En algunas hay frases
tan apasionadas que no las escribiera más sentidas el mejor poeta. Oiga
usted: «Cuando en la misa me vuelvo a decir \emph{Dominus vobiscum} y no
te veo, como antes, ni la Virgen en su soledad pasaba la tristeza que
yo.» ¿Qué tal? ¡En qué estaría pensando el hombre cuando
celebraba!\ldots{} Pues otra: ¿no sabe usted que el año 22 estuvo al
lado de los milicianos en la acción del 7 de Julio? Sí, hombre, y en ese
mismo mes quiso matar al Rey; al menos se abalanzó al coche con todas
las de Caín, gritando: «¡Mueran los perjuros!» Sí, hombre: ahora lo
hemos descubierto\ldots{} Bragado es el tío, como hay Dios, y de un
temple que ya no se estila\ldots{} ¡Vaya, que si llega a darle de veras
a Fernando VII, la que se arma! ¿Qué sería hoy de España? \emph{Acá para
inter nos}, creo que le habríamos quedado muy agradecidos\ldots{} Pues
verá usted lo que me contó anoche. El año pasado, solía ir al gabinete
de lectura de San Felipe Neri, y allí se daba unas atraquinas de
periódicos españoles y extranjeros que Dios temblaba\ldots{} Dice que, a
pesar de sus amarguras y de su odio al género humano, se mantenía
tranquilo y sin idea de matar; pero que al enterarse del golpe de Estado
de Napoleón, y ver la nube de despotismo que se venía encima en toda
Europa, se fue del seguro y dijo: «aquí hay un hombre\ldots» Querido
Beramendi, yo he visto locos en la política; pero como éste ninguno, ni
creo que haya venido al mundo un alma más fanática.»

Terminó Ordóñez así: «Tengo que poner un bando en las esquinas; está el
pueblo muy excitado contra el asesino, y tan condolido de nuestra Reina,
que ni aun sabiendo que la herida es leve se da por satisfecho. Me dice
la policía que entre la gente del bronce hay \emph{elementos} decididos
a dar un golpe el día de la ejecución, arrancando al reo de manos de la
justicia para \emph{escabecharlo} por manos de la plebe\ldots{} Figúrese
usted, ¡qué carnicería, qué barbarie! Esto no es propio de un pueblo
culto\ldots{} Conozco yo a esos \emph{elementos}: son los que alborotan
siempre, hoy en este sentido, mañana en otro, y al fin en el sentido de
la poca ilustración\ldots{} Pero ellos también conocen a Melchor
Ordóñez. Pregunten al pueblo de Madrid quién es Melchor Ordóñez, y dirá:
\emph{un hombre que sabe respetar y hacer respetar.»} Y era verdad. Por
la fama de su probidad y rigidez, acreditadas en otras provincias, le
trajeron a este Gobierno civil, en el cual ha emprendido con fortuna el
escarmiento de pícaros, el acoso de vagabundos y la corrección de
revolucionarios de oficio. Pero quien manda, manda. No obstante la
rectitud y nobles alientos de Melchor Ordóñez, en algún caso, que he de
contar si Dios me da salud, no le han dejado ser tan rígido como él
quería.

Llegué a mi casa con dolor de cabeza, desconcertado de todo el cuerpo,
amarga la boca y los espíritus muy caídos. El frío que cogí en la odiosa
cárcel me molestaba menos que el recuerdo de lo que allí vi, la vileza y
procederes bajunos del brazo secular, por una parte; por otra, el
bárbaro formalismo del brazo eclesiástico. ¡Con tales brazos, valiente
tronco social nos hemos echado!\ldots{} Prolijamente lo referí todo a
María Ignacia, que, al verme arrumbado en un sofá, no se separó de mí en
lo restante de la tarde. Horrorizada con mi relato, me autorizó para que
lo escribiese, recomendándome que en lo sucesivo huya de impresiones
patibularias, y consagré mis \emph{Memorias} a cuadros y tipos
placenteros, proscribiendo todo lo dramático. La misma sociedad me
indica el camino que debo seguir, pues ella no quiere ya cuentas con el
género trágico, y se ha hecho pura comedia, con sus puntas de sátira, y
la exhibición de pasiones tibias, de caracteres excéntricos o graciosos.
Esto vino a decirme mi cara esposa, aunque no con los términos que yo
empleo, sino más a la pata la llana. La tragedia no existe ya más que en
el pueblo bajo, y en los ladrones y bandidos. Debo, pues, concretarme a
las clases superiores, que no quieren ver sangre más que en casos de
recetar el médico sangría o sanguijuelas. Para mi salud es conveniente
que yo ponga un freno a esta recóndita querencia mía de las cosas
trágicas, volviendo mis ojos a la sociedad alta y media, y a la
política, que también es ya comedia pura, de enredo muchas veces, otras
de figurón. Prometí a María Ignacia seguir el camino que su buen sentido
me indicó, y aquí me tenéis en plena vulgaridad social.

¿Recuerdas ¡oh Posteridad benigna! a las dos lindas muchachas, Virginia
y Valeria, hijas de mi amigo don Serafín del Socobio, con las cuales
honestamente me divertía yo allá por los años 48 y 49, jugando con ellas
a los novios, y tratándolas siempre como si fuesen una sola mujer con
dos cuerpos distintos, aunque muy semejantes? Creo haberte dicho también
que les salieron efectivos novios, uno para cada una, dos tenientes, que
también a mí me parecieron duplicadas imágenes de un teniente solo. Pues
se casan; uno de estos días serán llevadas al altar, no por aquellos
pretendientes que las cortejaban el año 50, sino por otros, militar el
de Valeria, civil el de Virginia. Ambas, según me cuenta mi mujer, están
rabiando por cambiar de estado, ansiosas de pasar de señoritas a
señoras, con casa propia, libertad, y hombre a quien poner las enaguas
para hacer de él un monigote. Me figuro que estas dos bodas son algo
precipitadas, y que los padres, aunque aparezcan satisfechos, han
consentido en colocar a las niñas, por no poder aguantar ya sus
vehementes ganas de emancipación. Valeria ha escogido por sí su hombre,
el cual es un capitancito de buenas prendas, hijo del coronel don Felipe
Navascués, que figuró en la guerra civil; entra Virginia en la coyunda,
más que por designio propio y libre, por la persuasión amorosa y tenaz
de sus padres, que han visto la felicidad de la niña en el orondo y
fresco joven Ernesto de Rementería, hijo de un señor que pasa por
millonario. Dios las haga felices, y a ellos también, pues, aunque
apenas los conozco, merecen mi respeto y la sana compasión que debemos a
todo cristiano que se embarca para cruzar el engañoso piélago del
matrimonio. Así lo llamo, porque si a mí me ha salido este mar
totalmente limpio de sumideros y escollos, otros que entraron en la nave
con el corazón lleno de alegrías, navegan desesperados entre bravísimas
olas, y no saben en qué peña irán a estrellarse.

La educación de mis amiguitas Virginia y Valeria no las eleva mucho, por
más que otra cosa creyera yo, sobre el común nivel de nuestras señoritas
de la clase media tirando a superior. Poseen, eso sí, su caudal de saber
religioso, todo de carretilla, sin enterarse de nada; escriben muy mal,
con una ortografía que parece el carnaval del Alfabeto; en Aritmética no
pasan de las cuatro reglas, practicadas con auxilio de los rosados
dedos; en Historia, fuera de la de José vendido por sus hermanos, y de
la de Moisés recogido en el Nilo, están rasas, y sólo saben que hubo
aquí godos muy brutos, y después moros que eran derrotados por Santiago.
Todo lo que saben de Geografía no vale un comino: se reduce a nociones
vagas de la superficie del planeta, y al conocimiento de que es forzoso
embarcarse para ir a las Américas descubiertas por Colón. En Literatura
moderna y clásica están a la altura de su cocinera; no les ha entrado en
el entendimiento más que la comedia o el drama del día que han visto en
el teatro, y algún novelón sentimental, tal vez empalagosa leyenda de
caballeros tontos y sultanas redichas, que han leído en el
\emph{Semanario Pintoresco}, o en el folletín del periódico de la casa.
Poseen unas cuantas fórmulas de francés \emph{para sociedad}, y en el
piano aporrean furiosamente valses y polcas. No conocen nada de la vida;
no se ha permitido que en sus espíritus, amañados para la elegancia,
penetre parte alguna del prosaísmo con que tenemos que luchar. No
conocen ni el valor de la moneda, ni las pesas y medidas; no tienen idea
de lo que es una legua, un celemín, un quintal; apenas se hacen cargo de
cómo se convierte el trigo en pan, las uvas en vino, y de cómo salen del
cascarón los polluelos. Su corta vida y sus ingenuos caracteres se han
desarrollado entre las primarias labores domésticas, y entre novenas y
funciones de teatro, perfilando la educación social en tertulias
insustanciales, academias de toda humana tontería.

Hablando yo de esta pobreza educativa con las propias Virginia y Valeria
delante de su señora madre, ésta, que es una idiota muy honrada y muy
buena, dijo que para ser mujeres de su casa no necesitaban las niñas
saber más Historia Natural que la precisa para distinguir un canario de
un burro, y que los que llamados \emph{Principios} quedáranse para los
que habían de ganarse la vida como catedráticos. Quizás aquella
apreciable mula tiene razón, pensé yo al oírla, y traje a mi memoria el
ejemplo de María Ignacia, que, si en estudios no estaba menos cerril que
Virginia y Valeria, me salió excelente mujer, y ha sabido cultivar por
sí, en la vida más que en los libros, sus nativas dotes, fundando
fácilmente el nuevo saber sobre el raso de su ignorancia. Esto pienso
que harán mis amiguitas, guiadas por su despejo natural y por la sana
índole de sus corazones. \emph{Amén}.

Anoche tuvimos a comer a don Mariano José de Rementería, padre del joven
que pronto será feliz esposo de Virginia. Es hombre de posición, según
dicen, y de una cultura más brillante que sólida, elaborada en los
viajes y en el trato social más que en el estudio. Suelen ser los cultos
mundanos menos enfadosos que los eruditos, mayormente si éstos
descuellan en la especialidad de la sabiduría rancia y del atavismo
histórico y arqueológico; pero don Mariano desmiente esta regla, porque
es el señor más molesto, más prolijo y más pedante (en el ramo de
cultismo europeo) que yo he podido echarme a la cara en esta vida
triste, valle de lágrimas\ldots{} no, no, valle, vivero más bien de
imbéciles. Cuando se pone a contar sus odiseas y las maravillas de la
civilización, se creería que él solo las ha visto y gozado, porque a
nadie deja meter baza, ni permite que otras bocas alaben cosa distinta
de lo que pondera hiperbólica y neciamente la suya. Pues sucedió que el
pasado año tuvo este señor la ocurrencia, y nosotros la desgracia, de
ir\ldots{} vamos, de que fuera él, no a escardar cebollinos, sino a
visitar la Exposición Universal de Londres. Los que le alentamos a ese
viaje, y yo fui uno de ellos, con rabia lo confieso, bien lo hemos
pagado, bien, porque ahora, con sus enfáticas descripciones del
\emph{Crystal Palace} y de los peregrinos \emph{adelantos} que vio en
él, nos trae a todos locos, a mí particularmente, que tengo la cabeza
débil, y el humor fácilmente irritable contra los habladores. ¡Jesús me
valga y Santa Librada bendita, patrona de Sigüenza! Es un hombre que
empieza a contar algo que le ha pasado en sus viajes, y desde los
primeros conceptos pega un brinco y se mete en una digresión, de ésta en
otra, y en otra, hasta que, viéndonos a todos mareados, se para y
pregunta: «¿En dónde estaba yo?» «Pues estaba usted---le contesté
anoche---en \emph{Oxford Street}, queriendo darnos una idea aproximada,
nada más que aproximada, de lo grande que es esa calle.

---Justamente---prosiguió él.---Pues verán ustedes: salía yo de
\emph{Hyde Park} con el famoso Losada, ya saben ustedes, el primer
relojero del mundo, y nos encontramos a Carreras, el primer tabaquero de
Londres\ldots{} Hablamos de España, de este país tan pobre y tan
atrasado\ldots{} Entre paréntesis, aquí no tienen idea de la penosa
impresión que a los que venimos del extranjero nos causa el llegar a
Madrid, y ver el \emph{sistema} primitivo de recoger las basuras\ldots»
De esta digresión pasó a otra, y a mil, y fue a parar ¡a Egipto! a los
carneros de cuatro cuernos que ha presentado Egipto en la Exposición
Universal\ldots{} ¡Cuatro mil cuernos había puesto ya en nuestras
cabezas aquel condenado narrador!\ldots{} Sin el menor cargo de
conciencia, digo que le detesto. Su palabra fácil, sus períodos
gramaticales muy pulidos, inflados por las amplificaciones, me atacan
los nervios. Se oye cuando habla, y se recrea en el efecto que hace. El
vértigo de sus digresiones adormece a muchos, y a mí me pone en un grado
de furor que difícilmente puedo disimular en su presencia. Y para mayor
desgracia, mi suegro, que ahora se pirra por aprender todos los
\emph{adelantos}, con tal que no salgan de la esfera material, le trae a
su mesa un día y otro para proveerse de ideas sueltas, y \emph{ponerse
al tanto de las conquistas} más notorias que debe la industria a la
ciencia extranjera. Escúchale con devoción, y acaba siempre por desearle
una larguísima existencia para que pueda viajar mucho y contarnos tantas
maravillas. Lo que yo le deseo es que se muera, que le maten, que le
salga un asesino y nos le quite de en medio\ldots{} Mi mujer me riñe
cuando me oye tan despiadados disparates. «Es que me encocora este buen
señor---respondo yo,---y me hace desgraciado siempre que viene a casa.
Es un tonto, de la clase de los dorados, que son los peores. ¡Luego me
dices tú que me consagre a los tipos cómicos de nuestra sociedad! ¡Ay,
mujer mía!, me divierten mucho más los trágicos.»

\hypertarget{iv}{%
\chapter{IV}\label{iv}}

\emph{8 de Febrero}.---Ya no existe Merino. Ayer por la mañana, según
dicen, hizo protestación de fe, y dictó un escrito pidiendo perdón a la
Reina. Las dos serían cuando le condujeron al suplicio, en burro, con su
hopa amarilla llameada de rojo, para que la grosería de la cabalgadura y
la horripilante fealdad del empaque, disfraz sustraído a las máscaras de
la Muerte, llevaran más fácilmente la ejemplaridad al pueblo. Luego, por
la noche, le hicieron exequias a la romana: dieron fuego al cadáver,
para que no quede hueso, ni momia, ni despojo alguno a que agarrarse
pueda la memoria de los venideros. Así lo ha determinado el Gobierno de
Su Majestad, sospechando que la corrupción de los corazones nos traiga
una nueva demagogia, tan devota del regicidio que dé en la manía de
adorar el zancarrón de este desgraciado sujeto. Ello ha sido un
simulacro del Santo Oficio en la mitad del siglo XIX, para que puedan
echar una canita al aire los muchos que aquí conservan el gusto de la
quemazón de gentes, y se remocen viendo arder a un muerto, ya que no
pueden asar a los vivos.

¡Por Cristo, que sin la prohibición terminante de mi mujer, a quien
obedezco en todo, aunque me esté mal el decirlo, hubiera yo vuelto al
maldito Saladero! Hubiera, sí, cedido a la tentación de acompañar al
cleriguito señor Puig y Esteve, que llevó a la prisión de Merino el
encargo de examinar las profundidades del espíritu del criminal con la
sonda del conocimiento de Humanidades y de los clásicos latinos.
Brindome a esta visita don Serafín del Socobio, presentándome en su casa
al propio Puig y Esteve, quien reiteró el ofrecimiento con exquisita
urbanidad. «Pues está usted fuerte en latinidad clásica---me
dijo,---vamos juntos, y entre los dos haremos lo que podamos.» En un
tris estuvo que yo aceptara; pero me acordé de mi costilla, y más pudo
el temor de disgustarla que el estímulo de mi curiosidad. Al día
siguiente, oyendo contar al curita el resultado de su misión, me
maravillé del saber profundo y del buen gusto del asesino. Yo le tenía
por buen latinista; pero no sospeché que lo fuera en grado eminente. Y a
más de asombrarme, me desconcertó un poco la exacta concordancia de las
preferencias de Merino con mis preferencias en el gusto de los clásicos.
Como él, he tenido yo siempre marcada predilección por la Sátira X de
Juvenal. En mis tiempos de vida romana la recitaba de memoria, sin que
se me escapara un solo verso; y cuando arreglaba la biblioteca de
Antonelli en Albano, emprendí la traducción de la Sátira en verso libre:
no llegué a terminarla por culpa de Barberina, que se sobrepuso al gusto
de Juvenal. Aún puedo recitar algunos trozos, y entre otros el que dice:
\emph{Ad generum Cereris sine caede et vulnere, pauci---Descendunt
reges, et sicca morte tyranni}. Yo lo traducía de este modo:

\small
\newlength\mlena
\settowidth\mlena{\quad Son que a los reinos de Plutón descienden}
\begin{center}
\parbox{\mlena}{\textit{\quad Pocos los reyes, pocos los tiranos        \\
                        Son que a los reinos de Plutón descienden       \\
                        Sin ser heridos por puñal aleve.}}              \\
\end{center}
\normalsize

Fácilmente adapto al alma y a los pensamientos de Merino, en los últimos
años de su vida lo que piensa y dice Juvenal en esta admirable Sátira:
la turbación de las ideas en Roma, tan semejante a la turbación nuestra;
la indiferencia del pueblo a las cosas públicas en cuanto se ha enterado
de que la política es oficio de unos pocos; la degradante cobardía de
los que pisotean el cadáver del favorito de Tiberio para que no les
acusen de haber sido amigos suyos; la ingratitud de la opinión con los
grandes hombres; el triunfo de los osados y perversos; la tristeza de la
vida, y la vanidad de todas las cosas\ldots{} Encuentro muy lógico que
el elocuente pesimismo de Juvenal se infiltrara en el espíritu de
Merino, dispuesto por sus melancolías y desgracias a ser el vaso más
propio de tantas amarguras\ldots{} La voz y el ritmo del poeta latino
inspiró sin duda al enemigo de nuestra Reina su ansia de morir, y de
morir \emph{públicamente}, entre el escarnio de la plebe y las iras de
los poderosos, ostentando ante todo el Universo una gallarda postura de
muerte.

En otras predilecciones literarias del humanista criminal, no difiere su
gusto del mío. También prefiero entre los poemas bíblicos el de Job y de
él conservo en mi memoria algunos pasajes, de sublime grandeza. Y cuando
yo, estudiante en la \emph{Sapienza} y en San Apolinar, me ejercitaba en
el análisis exegético y retórico de los Evangelistas, San Mateo me
cautivaba más que los otros por su evidente cultura, y delicado arte. En
todo lo clásico estábamos conformes el regicida y yo, y si el regicidio
me parece una atrocidad, más que a perversión moral lo atribuyo al
empuje de las ideas negativas en un cerebro donde han perdido las
afirmativas toda su resistencia. Desprecio de la vida, querencia de la
muerte: ésta es la clave. El morir es bueno, aun para los tiranos; el
vivir es malo, aun para los oprimidos.

Lo que el joven Esteve y otros testigos presenciales contaban de la
reconciliación de Merino con la Iglesia, horas antes de subir al
cadalso, no altera mis ideas acerca de su estoicismo, sino más bien las
confirma. Quiso ser entero hasta el fin, y afianzarse en la calidad y
nombre de cristiano, como el que se sube a la mayor altura para
despeñarse con más admiración y sorpresa de los que contemplan su caída.
Una vez cumplido aquel deber elemental, pudo Merino permitirse
desdeñosas burlas de los que le llevaban al suplicio en tren de
mascarada de la Muerte, con ropa de autos de fe y gemidos de una
multitud enconada, aunque al fin compasiva. Parte de esta horrible
procesión patibularia pude yo ver, valiéndome de cierto casuismo para
quebrantar las saludables órdenes de mi buena María Ignacia. «Pepe mío,
te suplico, te mando que no vayas a la ejecución.» Así lo prometí.

Pero al renunciar al espectáculo de la ejecución, pensé que a la
obediencia no faltaría observando si se confirmaban o no las inquietudes
de Melchor Ordóñez. Con ánimo de ver si el pueblo nos daba una
interpretación trágica de su decantada soberanía, me fui hacia Santa
Bárbara, y cuando me escabullía entre la multitud, atento a las voces y
pensamientos de hombres y mujeres, tropecé con un alguacil, José
Risueño, que me tiene ley, porque yo le conseguí la plaza, siendo
Gobernador don José Zaragoza. Creyendo Risueño que la mejor prueba que
de su gratitud podía darme en aquella ocasión era introducirme en la
lúgubre casa, me dijo, asimilando su rostro a su apellido: «Venga, don
José, y podrá ver con toda comodidad al cura cuando salga al patio.»
¿Cómo resistir a esta tentación? Entré con mi protegido Risueño, y vi a
Merino a punto que montaba en el burro. La hopa amarilla le daba un
aspecto aterrador. Cuando le ataban los pies por debajo de la cincha,
dijo en tono agresivo: «¡Eh, brutos, que me lastimáis! ¿Creéis que me
voy a caer? Traedme un caballo y veréis si soy buen jinete.» Cuando el
asno daba los primeros pasos, miró don Martín al verdugo y al pregonero
que iban a su lado, y con flemático gracejo les dijo: «Buen par de
acólitos me he echado;» y volviendo el rostro, se despidió con este
familiar laconismo: «Abur, señores, abur.»

Vi la oscilación del pueblo, y oí su inmenso clamor de curiosidad
satisfecha, el goce del horror gustado en visión teatral y objetiva. No
advertí nada que indicase movimiento sedicioso para arrebatar a la
Justicia su presa. Más que pueblo, me pareció público aquel mar
ondulante de cabezas espantadas, de ojos ávidos del menor detalle, de
alientos contenidos, de bocas abiertas sin ninguna sonrisa. En miles y
miles de pensamientos humanos brotaba en tal instante la idea de que el
pescuezo de aquel hombre vivo, amortajado de amarillo, iba a ser muy
pronto triturado dentro de un cepo de hierro, y esta idea ponía en todos
los rostros una gravedad y palidez de rostros enfermizos. Decidido a no
seguir la pavorosa procesión, me escabullí por la Ronda con ánimo de
tomarle las vueltas al gentío, para observar su actitud. De lejos vi que
el paso del reo iba levantando la exclamación trágica, y que ésta le
seguía por una y otra banda, como siguen las nubes de polvo al
torbellino de viento que las eleva.

No vi más al condenado: de lejos distinguí un punto amarillo que se
perdía entre bayonetas y sobre la movible crestería de las muchedumbres.
Contáronme aquella misma tarde que, en todo el camino, don Martín no
dejó de \emph{guasearse} de la Justicia, del verdugo, de los clérigos
asistentes y de los respetables Hermanos de la Paz y Caridad. Todo este
interesante personal se veía defraudado en el ejercicio de sus
caritativas funciones; por los suelos estaba el programa patibulario,
pues el reo faltaba descaradamente a sus obligaciones de tal, negándose
a llorar, a besuquear la estampa, y a dejar caer su cabeza sobre el
pecho con desmayo que anticipaba la inacción de la muerte. Al sacerdote
que le exhortó a recitar salmos y a besar la estampita, le dijo: «Ya
rezo, señor. Quiero ver al pueblo y que él me vea a mí.» Y como de nuevo
le incitara el clérigo a mirar la estampa, sus palabras: «Ahora estoy
mirando la nieve de la Sierra. ¡Qué hermoso espectáculo!» Al conductor
del asno reprendió en esta forma: «Torpe eres tú para criado mío, con mi
genio\ldots{} Creo que no vas a servir ni para ahorcar.» Y luego siguió
así: «¡Cuánto tiempo que no doy un paseo tan largo, y de balde! ¡Qué
buena borrica es ésta!» Llegó al cadalso, subió con aplomo la escalera,
y acercándose al banco, tocó y examinó los instrumentos de suplicio para
ver si estaba todo en buen orden. Besó el crucifijo, sentose para que el
verdugo le atara, y mientras lo hacía, le encargó que no apretase mucho,
que él prometía moverse lo menos posible en el momento de morir. Se le
probó la argolla, y como notara que le lastimaba un poquito de un lado,
hizo un mohín de disgusto. Pero no era cosa mayor la molestia. Expresado
el deseo de hablar, permitiéronle pronunciar sólo algunas palabras,
repitiendo que no tenía cómplices, y terminó con la fórmula: «He dicho.»
El verdugo volvió a colocarle la argolla; acomodó Merino su
pescuezo\ldots{} Sus últimas palabras fueron: «Ea, cuando usted quiera.»

Cumplió el verdugo\ldots{} En mi memoria reviven estos versos de la
Sátira de Juvenal, que toscamente traducidos dicen:

\small
\newlength\mlenb
\settowidth\mlenb{\quad Pide un ánimo fuerte que no tema}
\begin{center}
\parbox{\mlenb}{\textit{\quad Pide un ánimo fuerte que no tema         \\
                        Morir, y que la corta vida mire                \\
                        Como precario don de la Natura.}}              \\
\end{center}
\normalsize

\emph{9 de Febrero}.---Y voy con lo urbano y apacible, con lo que mi
mujer llama comedia, y es la trama vulgar y descolorida de la
existencia, mundo medianero entre la risa consoladora y el llanto
dolorido, entre el sainete y el drama\ldots{} Allá voy, allá voy. ¡Pues
no se pondría poco enojada la Posteridad si me descuidara yo en
informarla de que hoy lunes se han casado mis amiguitas Virginia y
Valeria! Ya lo saben las presentes y futuras generaciones; sepan también
que hubo gran gentío, y después un copioso refresco en casa de Socobio,
y que los recién casados se fueron por la tarde a ocultar su vergüenza,
una pareja a San Fernando, orillas del manso Jarama, de regaladas
truchas; la otra a Canillejas, donde parece que el señor Rementería
tiene un \emph{cottage}, así lo dice él, muy para el caso. Sepan cuantos
las presentes vieren y entendieren que las dos chicas lloraron a moco
libre cuando al término de la ceremonia las abrazó su madre; que esta
voluminosa dama sufrió, de la emoción, un repentino desmayo, y que yo
fui, por mi proximidad al sitio de la catástrofe, el desgraciado mortal
a quien tocó la china de recogerla en sus brazos. Feo me vi para
sostenerla, y con hábil maniobra, como quien no hace nada, pude arrojar
toda aquella pesadumbre sobre mi vecino Rementería.

Sabrán asimismo que Rogelio Navascués, marido de Valeria, es un militar
nada bonito, pero simpático y airoso. Creo que bastaría su hoja de
servicios a darle crédito y fama de valor si ya no lo acreditara
casándose. Es despierto, picado de viruelas, delgado y rígido. Del de
Virginia, Ernesto Rementería, se hace lenguas la gente ponderando sus
buenas cualidades y su finísima educación a la extranjera. Es gordito,
sonrosado, de rostro pulido, limpio totalmente de bigote y barbas, la
melena lustrosa y ahuecadita sobre las orejas. Vestido con traje talar
podría pasar por una mujer metida en carnes, o por un lindo clérigo
francés. Viste muy bien, y sus maneras no pueden ser más atildadas.
Habla tres o cuatro idiomas, según dicen, que yo siempre le oigo
expresarse en un castellano premioso, arrastrando las \emph{erres} con
sones de gargarismo. Educose en el Mediodía de Francia. Su padre, antes
de traerle a España, le ha dado una pasada por diferentes naciones
cultas, teniéndole seis meses en Francfort, otro tanto en Londres, y año
y medio entre distintas poblaciones de Austria, Suiza y Holanda. Han
procurado instruirle principalmente en el alto comercio y en la magna
industria. Por su distinción, su gravedad y el aquél de tanto viajar con
fin educativo, yo le llamo \emph{El Joven Anacarsis}. Él se ríe,
enseñando unos dientes blancos como la leche, y poniéndose un tanto
colorado.

En fin, yo les deseo a todos mucha felicidad, e invoco a la diosa
tutelar del matrimonio, la honrada Juno de brazos de alabastro, y a la
prolífica Cibeles, para que les conceda\ldots{} \dotfill

\hypertarget{v}{%
\chapter{V}\label{v}}

\textbf{Sigüenza}, \emph{20 de Abril}.---Al reanudar con tanta distancia
de espacio y tiempo estas vagas Memorias, mi primer plumada será para
explicar por qué quedó interrumpida y suspensa la última hoja de lo que
emborroné con fecha 9 de Febrero. A punto que me acercaba a la
terminación de aquel escrito, fui sorprendido por una carta de Sigüenza
que, con los emolientes de costumbre, me notificaba la muerte de mi
buena madre. Días antes habíamos recibido carta de ella poco firme de
pulso, en los conceptos vigorosa: sólo nos hablaba de sus inveterados
alifafes sin importancia, y se mostraba gozosa, muy esperanzada de
vernos pronto por allá. Como nada temía, la triste nueva fue para mí un
escopetazo, y María Ignacia, que amaba a mi madre tanto como a la suya,
se afectó en tal manera que la tuvimos ocho días en cama. Con el vivo
dolor mío y la dolencia de mi mujer, imagínese qué gusto tendría yo para
redactar memorias\ldots{} El fallecimiento de mi adorada madre parece
que desató sobre mi familia todos los infortunios, y desde aquel aciago
día de Febrero ya no hubo para nosotros tranquilidad. A poco de
restablecida mi mujer, empezó nuestra niña a ponerse muy descaecida,
y\ldots{}

Aguárdense un poco: ahora caigo en la cuenta de que, por la quemazón de
mis papeles del 51, no sabe la Posteridad que el Cielo me concedió una
niña, la cual no nació tan robusta como su hermanito; que la pusimos por
nombre Librada, como mi madre, y que desde los primeros meses de su
existencia nos dio no poca guerra con el quita y pon de leches, pues
atribuíamos el desmedro a las malditas amas. Ya la teníamos al parecer
metidita en caja, cuando de improviso se nos echó de nuevo a perder, y
luchando hemos estado, hasta que al fin, hartos de doctores y
tratamientos, resolvimos venirnos con las dos criaturas a esta saludable
tierra y a estos purísimos aires. Tanto a Ignacia como a mí nos probó
muy bien el cambio de suelo, de ambiente, y de paisanaje sobre todo,
pues en este sentido Madrid me iba pareciendo ya tierra clásica de
majaderos. La niña también ha mejorado, y el primogénito está hecho una
fiera de apetito y codicia vital. He llegado a creer que la sombra de mi
madre, vagando entre nosotros, nos arregla la vida del modo más
llevadero y fácil; misteriosa tutela de ultratumba, que uno cree y
admite como se creen otras cosas sin someterlas al contraste de la
razón. Doy en pensar que la santa señora nos trae, en forma de
consuelos, proyecciones de la Bienaventuranza con que Dios ha premiado
sus virtudes\ldots{} Y de \emph{Memorias} nada, porque aquí no hay vida
pública; ningún acontecimiento sonoro rompe el plácido runrún de la
existencia. Ecos llegan acá del rebullicio político que anda en Madrid
por la Reforma Constitucional; pero como nada me importa que nos quiten
la vigente Constitución para ponernos la que más guste a la Reina
Cristina, a los señores eclesiásticos y a los realistas disfrazados de
liberales; como pienso que con libertad y con despotismo siempre seremos
los invariables ciudadanos de \emph{Majaderópolis}, dejo pasar la racha,
y, venga lo que viniere, aquí me tienen, como el impávido varón de
Horacio, mirando las ruinas de ayer\ldots{} y las fáciles construcciones
de hoy, añadiré que son las ruinas de mañana.

\textbf{Madrid}, \emph{13 de Enero de 1853}.---¡Vaya una
lagunita!\ldots{} Para saltar de la orilla en que dejé mis
\emph{Memorias} a esta ribera en que ahora las reanudo, tengo que dar un
brinco tan grande, que es fácil me caiga en medio del agua, o sea, en el
cenagoso abismo de mis calamidades. ¿Saben que se nos murió la niña a
fines de Septiembre del año pasado? ¿Saben que mi padre está si cae o no
cae, pues desde la muerte de su esposa no ha levantado cabeza el pobre?
¿Saben que a nuestra hija dimos tierra en el mismo sepulcro de mi madre,
para que juntas esperen la resurrección de los muertos? Ignacia y yo nos
hemos consolado con la idea de que la Librada grande y la chiquita gozan
abrazadas la dicha eterna\ldots{} ¿Saben que cayó Bravo Murillo, y que
se llevó la trampa todo aquel tinglado de la Reforma Constitucional; que
María Cristina y los demás realistones que patrocinaban esta idea se
echaron atrás asustados, dejando sólo al extremeño don Juan con sus
economías para fuera y sus chorizos para dentro de casa? ¿Saben que ha
venido, como quien viene de Belén, un Ministerio Roncali, con Federico
Vahey, Alejandro Llorente y otros que no recuerdo? ¿Saben que me afecta
tanto la emergencia de esta trinca de gobernantes como si vinieran a
decirme que se han descubierto mosquitos en la Luna? ¿Y saben, en fin,
que he perdido en absoluto las ganas de continuar pergeñando estas
deslavazadas \emph{Memorias}, y que me cosquillea en la voluntad el
prurito de quemar todo lo escrito desde Febrero del año anterior, con lo
que vendré a ser inquisidor de mí mismo? Pues saben todo lo que yo sé, y
no necesito escribir más.

\textbf{Madrid}, \emph{Noviembre de 1853}.---A instancias de mi mujer,
intento reanimar mi espíritu con el enredo de contarle cositas a la
Posteridad; pero ello ha de serme difícil en grado sumo, pues ya parece
que de mi mente se alejan esquivas las formas literarias, negándose a
entrar en ella, como en venganza del largo desuso a que las he tenido
condenadas. ¿Cómo empiezo? ¿Qué materia social o política cogeré del
montón de la vida presente para probar en ella mis fuerzas mentales
embotadas? Nada encuentro que me sirva; y sin materia rica en elementos
psicológicos, ¿qué ha de hacer el pobre ingenio mío, que veo acortarse
de hora en hora, como la piel de lija que sirvió a Balzac para su famoso
simbolismo de la humana vida? ¿Hablaré de la muerte de mi padre? Esto a
mis lectores poco interesa, y a mí, por dolerme tanto, me lastima traer
asunto tan íntimo a estas páginas frívolas, amargas, que sin quererlo
suelen salirme irónicas. Y lo malo es que, apartando la mente de mi
desgracia para llevarla a la vida general, no encuentro más que muertes,
muertes célebres, como podríamos llamar a las que no circunscriben su
duelo al término de una familia. Murió Mendizábal. Tres días ha le
llevamos al cementerio con gran multitud de pueblo y señores, tributo
tardío y menguado a un hombre que estimo como de los más altos de
nuestro siglo por las ideas grandes y la voluntad poderosa. Últimamente,
los políticos de tanda le hacían poco caso. Muerto, se ha visto su talla
gigantesca, y hemos empezado a mirar y a medir su obra colosal,
incompleta, porque aquí siempre ha de perderse en el tiempo el remate de
las cosas: así lo dispone la envidia. Los envidiosos callaron al ver
pasar su entierro; el pueblo, que, por ser tan poco envidiable, es quien
menos envidia, le siguió con respeto y emoción, comprendiendo con seguro
instinto todo el valor de la figura política que ya teníamos arrumbada,
y que ahora revive, aunque sólo en nuestra memoria.

Murió también Jenara, la viuda de Navarro, mujer de larga historia
propia y de grandísimo ingenio para contar la de los demás. Era la dama
más guapetona y más salada que nos legó el ominoso reinado; su trato
cautivaba; su sinceridad era la mayor de sus virtudes, con haber tenido
muchas, aunque no todas las que manda el Decálogo. Desde el azaroso
tiempo de José hasta las dos Regencias inclusive, precursoras de Isabel
II, y aun un poquito más acá, no había fragmento anecdótico relacionado
con la vida pública que no existiera en sus archivos, y estos solía
mostrarlos, cuando estaba de vena jovial, a sus buenos amigos. Deja una
memoria dulce sin sombra de rencores. En su testamento ha repartido
entre las amistades algunas joyas y objetos de valor. A María Ignacia le
ha dejado una pulsera que le regaló doña Francisca de Braganza, y a mí
dos cartas de Chateaubriand en que habla del Congreso de Verona de
Alejandro de Rusia, y particularmente se sí mismo. R. I. P.

El que no se ha muerto es Rementería, ni trazas tiene de óbito cercano;
al contrario, disfruta de una salud aterradora, que, según dice, debe a
las abluciones con agua fría\ldots{} Este azote de la humanidad no ha
concluido de contarnos todo lo que vio en la Exposición de Londres, y
siempre acaba diciendo: «estamos muy atrasados,» o «vivimos en un grande
atraso.» Pasa por hombre de posibles, y así lo manifiesta su casa, que
es también la de sus hijos, Ernesto y Virginia. Sus alfombras, sus
cortinajes, su comedor \emph{vieux chêne}, sus dos salones con tapices,
imitación de Gobelinos el uno, el otro de severo estilo inglés, son la
admiración de Madrid\ldots{} Y además de hombre adinerado, es muy
entendido en negocios. España le debe, si no la implantación, el
perfeccionamiento de las Sociedades de Seguros sobre la Vida, pues la
Sociedad suya, la que fundó y dirige, nombrada \emph{La Previsión}, ha
empezado sus operaciones con un éxito loco, según dicen. No pocas
Empresas de esta clase se han fundado en España de algunos años acá, y
parece que todas prosperan\ldots{} Los milagros de la asociación y del
mutuo auxilio, en Inglaterra y en Francia, por diversas plumas han sido
explicados aquí en periódicos y boletines. ¡Si llegaremos algún día, con
la ayuda de Dios y el concurso de estos entendidos negociantes, a la
categoría y significación de pueblo rico y civilizado!\ldots{} Guizot
dijo a los franceses: \emph{enriqueceos}, y nuestros aseguradores de la
vida contra la pobreza, de la propiedad urbana contra incendios, y de
las naves contra los riesgos de la mar, dicen a los españoles:
«asociaos; traedme vuestras economías, y os haré poderosos.» Los
españoles, borregos \emph{a nativitate}, llevan, sí, sus
economías\ldots{} «¡La previsión, el ahorro, el mutuo auxilio!\ldots{}
¡ah! Estamos muy atrasados\ldots»

A instancias de mi mujer, leo los párrafos antecedentes, y ella me dice:
«Parece que tomas a broma la sociedad de don Mariano José\ldots{} y en
eso no eres justo, Pepe. Rementería tiene mucho talento, ha visto medio
mundo, aprendió en el extranjero estas cosas de juntar los ahorros de
todos para hacerlos crecer y\ldots{} Pero ¿qué gestos haces?

---Estamos muy atrasados\ldots{}

---No te rías, ni tomes esto a broma. Aquí tengo el prospecto.

---Lo sé de memoria. Trae primero la retahíla del Consejo de Vigilancia,
en el cual han metido a tu padre; después dice: \emph{Director, don
Mariano José, etc\ldots{} caballero de la Legión de Honor}\ldots{}

---Y por ser de la Legión de Honor, ¿crees que le llevarán más pronto el
dinero?

---Naturalmente. Ya sabe el pie de que cojea todo español que tenga
cuartos. El español con ahorros camina ciegamente hacia un hombre
estirado, que se los pide mostrando un cintajo en el ojal de la levita.

---¡Qué exagerado eres, y qué disparates dices! Explícame esto de las
imposiciones y del 3 por 100. Dime cómo se aumenta el capital o fondo;
qué significa esto de la \emph{amalgamación} de intereses, y cómo y por
qué se enriquece el asegurado que conserva la vida\ldots{}

Di a mi mujer explicación clara de las bases y mecanismos de la
Sociedad, que son excelentes, y añadí: «La idea es en sí fecundísima:
desconfío de las personas que la ejecutan.» Tronó María Ignacia contra
mi pesimismo, y apuntando el propósito de asegurar a nuestro hijo, me
leyó este incitativo párrafo del prospecto: «La combinación de
imposición de fondos en títulos del 3 por 100 y de la herencia mutua
entre los asegurados, produce resultados sorprendentes, en términos que
una imposición de 1.000 reales anuales que un padre hace encabeza de un
niño recién nacido, promete a este hijo un capital de 470.000 reales si
alcanza su vida la edad de veinticinco años.»

---¡Parece milagro! Aseguraremos al pequeño, si quieres. Somos ricos. Si
en esta lotería no nos cae premio, poco perderemos\ldots{} ¡Adelante las
Sociedades de Seguros! Con desconfianza de sus manipuladores, yo las
admito y enaltezco. El principio económico en que se fundan no puede ser
mejor. La base del negocio es la muerte, infalible cosecha, y lo que en
otros países ha dado buenas ganancias, aquí debe darlas triples, porque
los españoles, en su gran mayoría, se mueren antes de tiempo, por la
falta de higiene, por las guerras civiles, por la miseria. Yo te concedo
que las tales Sociedades son buenas y que debemos alentarlas hasta ver
en qué paran; pero no me mandes incluir estas vulgaridades en mis
\emph{Memorias}, que o no serán nada, o deben transmitir desde mis días
a los venideros los graves hechos políticos y militares\ldots{}

---Ven acá, tonto de capirote---replicó mi mujer, poniendo en sus
ojuelos claros toda la agudeza de su espíritu:---¿quién te dice que esto
no es un tema social, un tema político, el más político de cuantos
pueden existir\ldots{} histórico además, por ser cosa que va de un día
para otro y de un año para otro año?\ldots{} Ciego estás si no ves lo
interesante que ha de ser este capítulo de las Sociedades para los que
te lean dentro de medio siglo. Piénsalo, Pepe, y hazme el favor, te lo
suplico, de no romper lo que has escrito de don Mariano José y de la
flamante \emph{Previsión}, que es una mina, créelo, el grande hallazgo
de todos los españoles que se tomen tiempo para morirse. Piénsalo, Pepe,
y no rompas\ldots»

No rompí, pensé\ldots{}

\hypertarget{vi}{%
\chapter{VI}\label{vi}}

\emph{Diciembre de 1853}.---He comprendido que mi mujer, en quien cada
día reconozco más claras dotes de profetisa y adivinadora, me señala la
verdadera fuente de la histórica filosofía, y su talento admirable
arroja mayor brillo cuando me dice: «En los actos más insignificantes
encontrarás el filón de pensamientos que buscas.» A los pocos días de la
conversación relatada en mi anterior confidencia, salimos de compras.
Algo necesitábamos para nuestra casa; pero íbamos acompañando a Valeria,
que aún no había completado el ajuar de la suya, y nos llevaba de
asesores inteligentes. En muebles importados de Francia vimos
maravillas. De algún tiempo acá se han establecido en Madrid no pocos
marchantes que nos traen las formas gallardas y cómodas de la
ebanistería y tapicería francesas. ¿Qué sillones y qué sofás de blando
asiento! Los huesos duros del español de raza, hechos a toda incomodidad
y dureza, caen en ellos embelesados y no saben levantarse. Todavía
tenemos espíritus ascéticos que se escandalizan de esta blandura y la
llaman ¡sibaritismo!\ldots{} Pues en muebles de puro adorno hay
preciosidades que quitan el sentido a las señoras de buen gusto y de
aficiones suntuarias. Los veladores \emph{maqueados}, las sillas del
mismo estilo, hacen el agosto de estos buenos mercaderes, en quienes
advierto, con grande asombro, que se han asimilado la relamida finura
francesa para enseñar el artículo, regatearlo, venderlo y cobrar su
importe, si es que lo cobran. En juegos de habitaciones completas,
comedores de nogal tallado, alcobas de palo santo, salas y gabinetes, vi
gran variedad, a precios razonables, al alcance de los que tienen poco
dinero y aun de los que no tienen ninguno\ldots{}

Pues mayor fue mi sorpresa cuando me llevaron (Valeria era la que
guiaba) al 17 de la calle Mayor, \emph{La Exposición Extranjera},
suntuoso almacén de objetos de laca, de bronces para regalos, y de mil
bagatelas elegantes, graciosas, útiles, obra del inagotable ingenio
parisiense. No me había yo enterado de que nos traen acá toda esta
superfluidad bonita, y menos de que se vende como pan, echando por
tierra la leyenda de las austeras costumbres españolas. Vi gente
innumerable que compraba, o al menos que veía y regateaba. Aquel género
de pura distinción y lujo también se va poniendo al alcance de los que
no tienen sobre qué caerse muertos. Compran los ricos, los que disfrutan
un modesto pasar, y los empleados de catorce mil reales que dan
reuniones en su casa, y se prometen mayor ostentación cuando logren el
ascenso a dieciséis mil\ldots{} ¡El mundo está perdido! Algunos cuartos
dejamos en \emph{La Exposición Extranjera}, y no volvimos a casa sin
echar un vistazo a otros establecimientos: género blanco y mantelería,
encajes, \emph{plata Ruolz, etcétera}\ldots{} Cuando María Ignacia y yo
comentábamos a solas nuestra correría por las tiendas de tan grande
novedad, me dijo ella: «¿Tú qué te creías, que Madrid no progresa? Pues
déjate que pongan los ferrocarriles; verás cómo se cuelan aquí todos los
adelantos.» Y yo: «Ya veo, ya: nuestro pueblo se asimila los progresos
del lujo y de la comodidad más pronto de lo que yo pensaba. Tenías razón
en decirme que estas cosas insignificantes y comunes merecen que se les
indague el busilis. Escribiendo yo de ellas, escribo Historia \emph{sans
m'en douter.»} Y ella: «Yo no sé si escribes Historia o no: sólo sé que
esto es comedia y de las entretenidas, es sátira, es pintura de
costumbres.»

«Relaciono estos hechos---dije---con la epidemia reinante, que llaman
\emph{pasión de riquezas, fiebre de lujo y comodidades}. Así nos lo
cuentan y así lo vemos con nuestros propios ojos. Un día y otro nos
hablan de los \emph{escandalosos agios}, de los negocios y contratas con
que el Gobierno premia a los que le ayudan. Ya viene de atrás este
tole-tole; pero D. Juan Bravo Murillo fue quien más abrió la mano en las
concesiones de vías férreas, de explotación de minas, de obras para
nuevos caminos y para puertos y canales. Esto es muy bueno, esto es
vivir a la moderna, esto es progresar. No hemos de ser un eterno
Marruecos petrificado en la barbarie y en la pobreza\ldots{} Aunque sigo
aborreciendo a nuestro amigo Rementería, por hablador insufrible, pienso
que este hombre enfadoso y cargante es un mesías que viene a traernos
vida nueva. Poco vale el mesías; pero sin duda no merecemos otro. Ahora
falta ver qué regeneración nos trae, y cómo la recibimos.» Y ella: «No
hay duda de que los españoles quieren entrar por el camino de la
ilustración, madre del bienestar.» Y yo: «Pero no empiezan por el
principio, que es instruirse y civilizarse, para después gozar. Dicen:
\emph{gocemos, y luego nos civilizaremos}. Ven todo ese material bonito
y elegante que los extranjeros han inventado para su goce, para su
descanso y recreo; y tomando el fin por el principio, piden que vengan
acá esas maravillas, las compran, las usan, quieren gozar de ellas,
creyendo que con adquirirlas y poseerlas son tan civilizados como los
que las inventaron y luego las hicieron. Signo de cultura son las ricas
alfombras, las tapicerías, los sillones de muelles en que se hunde el
cuerpo perezoso. Pues tráiganmelo, dicen: decoraré con ello mi casa, me
daré tono de hombre culto, y ya se verá luego de dónde saco los dineros
para pagarlo. No ha de faltar un buen negocio, un repentino hallazgo de
veta minera, un cambio político, un premio de lotería, una herencia de
tíos de América.»

\emph{Enero de 1854}.---Mi simpatía por Sartorius, motivada quizás de su
cumplida urbanidad y de las atenciones que de él merecí en otra época,
no se amengua con el zarandeo en que le traen los innumerables cesantes
de alta categoría, moderados inclusive; los que ministraron el pasado
año con Roncali y con Lersundi, y toda la caterva progresista y
democrática. Ni entiendo este remoquete de \emph{polacos} y
\emph{polaquería} con que se designa toda corruptela, los verdaderos o
imaginarios chanchullos de que nos habla la vocinglera opinión. Ello es
que desde que entró San Luis a dirigir el cotarro, en Septiembre del año
anterior, se ha desatado un viento de huracán, que conmueve el cimiento
del poder público. Las polvaredas que a todos nos ciegan, no nos dejan
ver la mentira ni la verdad.

A la nariz me llegan olores de revolución sin que sepa precisar de dónde
salen; pero ya puedo presumirlo, porque les acompaña tufo de cuarteles.
Se nota en el vecindario madrileño esa especial alegría del pueblo
español cuando hierve dentro de él el caldo de las conspiraciones, algo
como preparativos de bodorrio plebeyo. Hasta me parece que noto en las
personas de afición filarmónica el prurito de componer himnos, y en las
de armas tomar, ojeadas estratégicas para el emplazamiento de
barricadas. Comparten con Sartorius el vilipendio de la impopularidad el
Ministro de Hacienda, don Jacinto Félix Domenech, ayer progresista, hoy
\emph{polaco}, y Agustín Esteban Collantes, contra el cual los
maliciosos no fulminan ningún cargo concreto: que fue redactor de
\emph{La Postdata}, Secretario del Gobierno Civil de Madrid, Director de
Correos, Ministro después; motivos suficientes para que le aborrezcan
los que no han sabido ser nada. Me enfadan estos aspavientos de la
ineptitud, que se disfraza de catonismo para que la oigamos. A los
hombres que con vigorosa voluntad han sabido encumbrarse, les tengo
siempre por mejores, en todo sentido, que los entecos que sólo saben
tirar de los pies al prójimo que sube. Hablo de Collantes con más
extensión que de Domenech, porque a éste apenas le conozco, y aquél es
amigo mío. Pienso que ambos están llamados a grandes amarguras, y por
anticipado les compadezco\ldots{}

Mi mujer, firme en la idea de que un constante y metódico empleo de mis
facultades anímicas ha de ser muy provechoso para mi salud, me
recomienda que ponga mi atención en la política, ahora que está cual
nunca interesante, \emph{preñada}, como dice algún órgano de la Prensa,
de \emph{formidables acontecimientos}. Anhelo yo que esos
acontecimientos vengan, y que me traigan aspectos y emociones
dramáticas, con algún perfil cómico que dé humana realidad a mis
historias. Anhelo también que, si los sucesos políticos toman vuelo y se
hinchan con trágica grandeza revolucionaria, salga del seno agitado de
los tiempos algún privado suceso de los que se miden y confunden con los
públicos, formando una conglomeración sintética. Revolución quiero y
necesito: revolución en los cerebros y en los corazones, revolución
arriba y abajo, dentro y fuera\ldots{}

\emph{17 de Enero}.---¡Vaya que esto parece brujería! Cuando con tanta
fe y devoción pedía yo al Destino, a las Musas o al Demonio coronado,
una revolucioncita privada o pública para mi solaz y entretenimiento
(con tal que no venga por mi casa), ¿quién me había de decir que tan
pronto sería complacido por las ocultas divinidades celestiales o
demoníacas que me protegen? Todo suceso, sin excluir los políticos de
mayor monta, palidece ante el que me ha traído una carta que esta mañana
recibí, sin que me haya sido posible averiguar qué mano la entregó a mi
portero con encargo de que me la subiese pronto, pronto\ldots{} ¡Vaya
unas prisas!\ldots{} Lo primero que hice al desdoblar el papel fue
buscar la firma, y con estupor leí: \emph{Virginia}\ldots{} Pues veamos
lo que Virginia cuenta. Corrigiendo su criminal ortografía, para que la
Posteridad no vitupere a esa criatura más de lo que merece, copio lo que
ya he leído cien veces, sin que tantas lecturas me curen de mi asombro.

«Querido Pepe: a ti que eres tan bueno, y sabes apreciar las cosas como
son, te digo lo que te digo, a ti solo antes que a nadie\ldots{} Y te lo
digo sin remilgos, de escopetazo, como deben decir las personas
valientes lo que hacen con firme voluntad. Te asustarás, Pepe; pero ya
se te irá pasando el susto.

Sabrás, querido Pepe, que me he escapado de mi casa con el hombre que
amo, con el que es primero y único amor mío. Dios sabe que nunca amé a
otro, y que mi corazón estuvo muerto hasta que conocí al que ha de ser
mi pasión y mi felicidad ahora y siempre. Con él me lanzo por el camino
de la vida. De mi casa he salido tranquila y animosa, sin llevarme más
que alguna ropa y las alhajitas que tuve de soltera.

Querido Pepe: no te digo el nombre de él, porque no quiero que nos
descubran ni nos persigan. Te diré que es joven, que es bueno, y que me
ama con delirio, como yo a él.

No siento dejar mi casa, que me era odiosa: en ella se queda Ernesto, de
quien te diré que el mismo día de la boda, y al siguiente, ya vi bien
claro que no le quería, ni podría quererle nunca. Ernesto no es un
marido, ni sabe más que hacer cuentas. No te escribo para que vayas a
consolarle. ¡Como que se habrá quedado tan fresco! él, y el ladronazo de
su padre, y su casa, y toda la sociedad me importan a mí un rábano.

«Te escribo porque mis padres y mi hermana sí que me importan, y
pensando en el sentimiento que tendrán por mi fuga, se me amarga el
júbilo de mi estado libre. De tu buena condición y de tu amistad espero
el favor de que vayas a ver a mis padres y a Valeria, y les digas que
aunque me escapé, rompiendo por todo, siempre les quiero, y soy su
invariable hija y hermana\ldots{} Querido Pepe: les dirás también que
estoy buena, aunque con la zozobra natural de saber lo que ellos
padecen, y que no me arrepiento de haber tomado el portante, porque soy
feliz, y me importa un rábano la opinión pública\ldots{} A ellos les
deseo conformidad, y les pido que me perdonen el mal rato que les he
dado, y que se vayan haciendo, porque ya digo\ldots{} me importa menos
que un rábano, por lo que toca a la sociedad y a los conocimientos de
casa\ldots{} Les dices también, como cosa tuya, que no me busquen ni den
parte, ni nada, porque ni hecha pedacitos así, vuelvo yo con ese
marmolillo de Ernesto\ldots{} y tú, ya sabes, Pepe querido, que no has
de hacer por averiguar dónde estamos\ldots{} No lo sabrás aunque te
vuelvas mico\ldots{} Si te portas como caballero, yo te escribiré alguna
vez que otra, para que por ti entiendan mis padres que estoy buena y que
les quiere mucho su hija. Adiós, buen amigo. \emph{Éste} te saluda y te
manda expresiones. Tú se las das mías a María Ignacia, y a tu niño dos
besitos, uno de cada uno de nosotros dos\ldots{} Pepe, mira bien lo que
te encargo: que no nos busquen, que no den parte\ldots{} Si así lo
haces, tendrás la confianza de tu leal amiga---\emph{Virginia.»}

No pude yo contar las cruces que después de leída la carta trazó María
Ignacia sobre su rostro y pecho, ni las exclamaciones de pena y asombro,
que fueron su primer comentario a suceso tan inaudito. Al fin hizo estas
observaciones rápidas: Ya dije yo que esa chiquilla no era tan buena
como parecía. Recordarás que era de las dos la más modosita, y, para
mayor absurdo, la menos romántica. Pero yo he creído ver en ella, antes
y después de casada, relámpagos de locura. Mira por dónde sale ahora.
¡Dios mío, qué desgracia, qué vergüenza! Pero ¿cuándo ha sido la fuga?
¿Ayer noche? No lo dice\ldots{} Y el hombracho, ¿quién es, Pepe? ¿Será
de estos mequetrefes sentimentales que engañan a las tontas cantándoles
el \emph{Suspiros hay, mujer, que ahoga el alma en flor}, o será algún
cotorrón maduro de éstos que\ldots? En fin, tú sospecharás, tú tendrás
algún indicio.»

Respondile que nada sé ni sospecho, y que, en vez de dar vanamente al
viento nuestras lamentaciones y conjeturas, debíamos acudir a las dos
familias heridas por aquel escándalo, para ofrecerles el consuelo de
nuestra amistad, como es costumbre, así en los duelos de muerte como en
los de honra. Con perfecta unanimidad de pensamiento tomamos la
resolución de proceder en sentido contrario a los deseos de la
bribonzuela de Virginia, dando conocimiento de la carta a los padres, y
ayudando a la busca, captura y castigo de los fugitivos. Media hora
después de tomado este acuerdo, entrábamos mi mujer y yo en la casa que
podríamos llamar mortuoria, y que encontramos toda revuelta. A don
Serafín le habían sangrado, y doña Encarnación estaba en cama con
furibundos ataques nerviosos. Eufrasia y Cristeta, que atendían a todo y
recibían las visitas de duelo, nos informaron de la tribulación de los
desdichados padres; y como yo pidiese noticias del estado de ánimo de
los Rementería, contestome Eufrasia: «Pues esta mañana estuvo aquí don
Mariano José a ver si sabíamos algo, y al responderle yo que seguíamos a
obscuras, me dijo: `Lo más lamentable es que en España no tengamos
divorcio. ¡Estamos muy atrasados!'»

De acuerdo con María Ignacia, determiné practicar por mi cuenta y riesgo
las primeras diligencias para dar con la prófuga, y me fui derechito a
Gobernación.

\hypertarget{vii}{%
\chapter{VII}\label{vii}}

\emph{13 de Enero de 1854}.---Mal día para negocios que no fueran de
política. En la Puerta del Sol me encontré a dos amigos que salían del
Ministerio: eran Antonio Cánovas del Castillo y Ángel Fernández de los
Ríos. Al primero le conocí el año pasado en casa de su tío, don Serafín
Estébanez Calderón. Es malagueño, cecea un poco; su talento duro y poco
flexible me cautiva precisamente por eso, por la dureza y rigidez. Ya
está uno harto de los ingenios chispeantes, volubles, imaginativos, que
fascinan, y no van ni nos llevan a ninguna parte. Éste no dice más que
la mitad de lo que piensa, y hará, creo yo, el doble de lo que dice. Así
me gustan a mí los hombres. A Fernández de los Ríos le trato desde que
fundó \emph{Las Novedades}, en Diciembre del 50. Quiso que yo escribiera
en su periódico; pero mi pereza y el deseo de conservar la libertad de
mi juicio pudieron más que mis ganas de complacerle. Es buen periodista
y gran plasmante de periódico; pero mi pereza y el deseo de conservar la
libertad de mi juicio pudieron más que mis ganas de complacerle. Es buen
periodista y gran plasmante de periódicos. Su idea dominante es la unión
de España y Portugal\ldots{} ¿Cuándo madurarán esas uvas?

Ambos amigos me dijeron que no intentara ver a Sartorius, porque a nadie
quería recibir; estaba con las manos en alto sosteniendo la nube que se
le viene encima, y lo mismo pesa sobre el Gobierno que sobre las
instituciones. Un rato fui con ellos hacia la Carrera de San Jerónimo,
donde se nos separó Cánovas para entrar en la librería de Monnier, y se
nos juntó Nicolás Rivero, que de ésta salía. Andando, nuestro grupo
llegó a tener ocho personas, entre las que recuerdo a Romero Ortiz y al
poeta Gabriel Tassara. No necesito decir que todos hablaban horrores del
Gobierno, de su arrogancia frente a la opinión, y de lo arisca y
deslenguada que ésta se va poniendo. En las conversaciones particulares
y en los papeles clandestinos, se prodigan a la situación \emph{polaca}
los siguientes piropos: \emph{tahúres políticos, cuadrilla de rateros,
turba de lacayos y rufianes}. La tormenta empezó a levantarse a la
subida de San Luis, y sus primeros rayos cayeron en Diciembre sobre el
Senado, con motivo de los debates y votación famosa del Proyecto de
Ferrocarriles. Derrotado Sartorius, limpió el comedero a todos los
senadores que habían votado en contra, de lo que provino un mayor
estallido de la tempestad, con los truenos y el furioso granizar de la
prensa desmandada. Acudió el Gobierno a poner a cada periódico su
correspondiente mordaza. Chillaron los periodistas por la boca de una
protesta colectiva. Fue también ahogada la protesta, y de aquí vino una
manifestación general, enérgicamente escrita, firmada por hombres de
diversos colores y opuestos cotarros, comprendidas figuras tan grandes
como Quintana y el Duque de Rivas, otras de lucida talla, como González
Bravo, Pastor Díaz y Olózaga, y toda la gente joven de más valía,
Cánovas, Florentino Sanz, Vega Armijo, Ayala\ldots{} En este punto de la
tempestad estamos ahora. En tanto que descargan nuevos rayos y se
ennegrecen las pavorosas nubes, los periódicos amordazados se vengan del
Gobierno y de la Casa Real, callándose todo lo que habían de decir del
parto de Su Majestad. El 5 nació una Princesita, y el mismo 5 se volvió
al Cielo, dicen que para no ver las cosas tremebundas que aquí ocurrirán
pronto. Sé que en Palacio ha sentado muy mal el torvo silencio de la
Prensa: esperaban oír los ampulosos ditirambos que en loor de la
Institución se prodigan a cada triquitraque. Pero esta vez falló la
costumbre: los periodistas se han callado como cartujos; no han escrito
una palabra de \emph{regio vástago}, ni de nada de eso\ldots{} Lo que
ellos dicen: «o estas trompetas suenan para todo, o no suenan para
nada.» Por ahí duele.

Proponiéndome yo que no pasase el día sin iniciar por lo menos mis
diligencias en busca de la dislocada Virginia, abandoné a mis amigos y
me fui al Gobierno Civil, que desempeñaba otra vez el buen Zaragoza; mas
tampoco pude verle. ¡Desgracia como ella! Fue uno de esos días aciagos
en que no hay puerta ni mampara que no se cierre adustamente sobre
nuestras narices. Traté de ver a Chico en el Gobierno Civil; luego
estuve dos veces en su casa, y todo inútil. Evidentemente, el Cielo
protegía con manifiesta parcialidad a los amantes prófugos. Ya me
retiraba, reconociéndome con muy mala mano para la cacería de criminales
de amor, cuando me deparó el Cielo a don José María Mora, director de
\emph{El Heraldo}, hombre muy amable y de extraordinaria corpulencia.
Recordando al punto la gran amistad de aquel cetáceo con los
Rementerías, le paré en medio de la calle; hablamos\ldots{} Poco más que
yo sabía del suceso; pero algo me dijo que era la primera luz que debía
esclarecernos el camino de la verdad. Sus palabras, entre resoplidos,
fueron: «Pero ¿ha visto usted qué trasto de niña? ¡Qué borrón para las
dos familias! Y ello no tiene remedio. ¡Si aquí hubiera divorcio, como
dice don Mariano José\ldots! Pero quia: no hay lañadura para este
puchero roto. Estamos muy atrasados\ldots{} Pormenores no sé, mi amigo.
Naturalmente, no he querido preguntar\ldots{} Me ha dicho el secretario
de \emph{La Previsión} que Ernesto se ha ido a Canillejas\ldots{} Parece
que tiene obra en la casa. Quiere aumentar la altura de todas las
puertas y entradas del edificio, ja, ja\ldots{} Y del gavilán que se ha
llevado a la paloma, nada sé\ldots{} Oí que es pintor.»

Nada más pude sacarle, porque el buen señor, que temía exponer al frío
su gordura sudorosa en tarde tan fría, dio por terminado el plantón, y
se despidió, apretándome mis dos manos con una sola suya\ldots{} «¿Con
que pintor?---pensaba yo, encaminándome a la casa de Socobio para
recoger a mi costilla.---Algo he descubierto; no dirá esa que he perdido
el día.» Visitas fastidiosas que iban sin duda a guluzmear, metiendo el
hocico en el dolor de los padres de Virginia, me impidieron comunicar a
Ignacia mi precioso descubrimiento. Llegó a la puerta nuestro coche, nos
avisaron, partimos, y al bajar la escalera desembuché lo poco que sabía.
En el trayecto de la calle de las Infantas a nuestra casa, Ignacia no
hizo más que burlarse de mí con desenfado y gracejo. «Yo creí que esta
tarde nos traerías a los dos fugitivos, cada uno por una oreja. ¿Y
Sartorius, Zaragoza y Chico no te han dado más que esa luz: que el galán
es pintor? ¿Y lo sabes por el gordo Mora?\ldots{} ¡Pintor! Pues eso yo
también lo supe, a poco de salir tú para el Ministerio. Me lo dijo
Ceferina, una de las criadas de la prófuga.»

En casa, tratando del mismo asunto, mi mujer, con poca seriedad a mi
parecer, me dijo: «Averigua tú ahora qué es lo que pinta ese bandido, y
quizás por el género de pintura saquemos el nombre.

---No creo que sea difícil sacar el nombre por el género, y el género
por referencias que yo pediré a Federico Madrazo, a Carlos Rivera, o a
Jenaro Villaamil\ldots{}

---Pues no tardes, que ello corre prisa.

---¿Y no te dijo Ceferina si es pintor notable?

---Notabilísimo.

---Pues los chicos que en Madrid descuellan en la pintura se pueden
contar. Verás qué pronto doy con ese pillo.

---¿A ti qué te parece?, ¿será pintor de historia, pintor de paisaje, de
asuntos religiosos, o de Mitología?

---Me parece a mí---dije viendo asomos de chacota en la sonrisa de mi
mujer---que es pintor de historia, y que la pinta al fresco.

---Sí, sí---exclamó ella, rompiendo a reír,---y voy a satisfacer tu
curiosidad diciéndote el género\ldots{} Es pintor\ldots{} ¡de puertas!

---¡De puertas! ¡Mujer, tú te chanceas!

---No\ldots{} Pero no vayas a creer que pinta sólo puertas. Pinta
también ventanas\ldots{} En fin, Pepe: hablando seriamente: sabemos el
oficio, el nombre no. Oye otro dato muy importante: es un chico
guapísimo.

---¿Joven?

---No representa más de los veinte años. Decía la Ceferina\ldots{} y
puso los ojos en blanco diciéndolo\ldots{} que nunca creyó que pudiera
existir un mozo tan guapo. Por la descripción que hace del tal, debe de
ser un perfecto modelo de la hermosura de hombre.

---Bueno: ¿y cómo entró en la casa? ¿Le llamaron para que diera una mano
de pintura al armario de la cocina?\ldots{}

---No: fue llamado para componer una cerradura, porque su verdadero
oficio es mecánico.

---No compondría una cerradura sola.

---Fueron dos, tres o más. Eran cerraduras que no querían dejarse abrir.
Parece que lo arregló tan a gusto de Ernestito, que éste le dio el
encargo de nuevas composturas. En la casa había molinillos de café y
aparatos de asador con mecanismo, que no funcionaban. Pues él lo dejó
todo que no había más que pedir, muy a satisfacción de Ernestito y de la
señora. Luego le dijeron que buscase un pintor; querían dar una mano de
blanco a la galería grande. A esto replicó que no había por qué llamar
pintor, pues él era amañado para todo, y también pintaba.

---Eso es verdad. Bien probada está su maña para todo. Bueno; y en eso
empleó algunos días\ldots{}

---Más de cuatro, y más de seis. Observó Ceferina que la señora iba a
verle pintar, y con él pasaba ratos largos de parloteo. Cuando las
criadas llegaban allí, se callaban como muertos, o sólo hablaba la
señora para decir: `Maestro, tiene usted que dar otra mano'. En los
últimos días, la señora le llevó a las habitaciones interiores para que
le barnizara un entredós. No llegó a barnizarlo, y todo se quedó en la
preparación, raspando y afinando el mueble con lija\ldots{}

---¿Y no sabe más Ceferina?

---No sabe más. La fuga fue el lunes por la noche. Salió sola, con un
lío de ropa, y dijo a Manuela, la criada vieja, que no volvería más. La
hermana de la portera la vio por la calle del Baño andando presurosa con
el pintor, cerrajero y alijador\ldots{} Atravesaron la calle del Prado,
y se perdieron de vista en la de León\ldots{}

---Pues hay una pista segura. Cuando se necesitó en la casa un oficial
mecánico para componer las cerraduras, ¿a quién se dio el encargo de
buscarlo?

---A un albañil que fue al arreglo de las chimeneas. Este albañil se ha
ido a la Mancha. No hay rastro de él.

---El caso es raro, extrañísimo por las circunstancias de tiempo y
lugar; pero no nos asombremos de él como de un fenómeno estupendo, no
visto jamás bajo el sol.

---Vamos, Pepe: eres capaz de disculpar la frescura y la indecencia de
esa mujer? Yo concedo a las flaquezas humanas todo lo que se quiera;
comprendo las pasiones repentinas, la ceguera de un momento, de un día;
¡pero fugarse así\ldots{} condenarse a la deshonra para toda la vida, a
la miseria\ldots! No creas: yo tengo en cuenta todo, y, entre otras
circunstancias, lo guapísimo que es el muchacho. Pues figurándomelo como
un perfecto Adonis, todavía no entiendo la pasión de Virginia: ¡Vaya,
que enamoricarse de un bigardo semejante, que quizás no sepa leer ni
escribir\ldots{} apestando a aceite de linaza y todo manchado de
pintura\ldots{} con aquellas manazas!\ldots{} Pero ¿no piensas tú lo
mismo?

---Querida mujer, me permitirás que reserve mi opinión mientras no
conozca el caso por el anverso y el reverso, por la cara que da a la
Sociedad y a las leyes, y por la otra cara, generalmente poco visible,
que da a la Naturaleza y al reino de las almas.

\emph{19 de Enero}.---Concertado tenía yo mi plan de campaña con el
gobernador don José de Zaragoza; pero este digno funcionario presentó
inopinadamente su dimisión por escrúpulos políticos muy respetables, y
como no conozco al nuevo Pilatos, don Javier de Quinto, me entiendo con
Chico, Jefe de la Policía. Presumo que este inmenso gato, buen conocedor
de todos los agujeros donde se ocultan ratones y ratoncillos señalados
por la ley, sabrá coger las vueltas a los ladrones de mujeres solteras o
casadas. Hace tres días le vi en el Gobierno Civil: concertamos una
entrevista en su casa; en ella estuve ayer y hablamos lo que voy a
referir.---Cuénteme, don Pepito, lo que le pasa---me dijo empleando las
formas confianzudas a que cree tener derecho por sus años, por su
autoridad policíaca, y aun por el miedo que inspira,---y yo veré en qué
puedo servirle.»

Expuesto el caso, resultó que ya tenía conocimiento de la \emph{evasión}
por referencias de don Pedro Egaña, íntimo amigo de los Socobios, y que
había mandado buscar ese rastro, sin resultado alguno.---Lo que contesté
al don Pedro se lo repito a usted, señor don Pepito, a saber: que la
política nos ocupa hoy todo el personal, y aun no basta, por lo que nos
es muy difícil atender a los negocios de familia.

---Ya, ya comprendo---le dije,---que con el cisco que se está armando no
tiene usted ojos ni manos bastantes para perseguir y cazar
conspiradores\ldots{}

---Mi opinión es ésta: o suprimir la policía, dejando que haga cada
\emph{quisque} lo que le salga de los riñones, o aumentarla hasta que
tengamos tantos agentes como españoles existen. Esto está perdido. Desde
que cogió San Luis las riendas, se ha desatado el infierno: aquí
conspiran progresistas y moderados, paisanos y militares, las señoras
\emph{del gran mundo} y los cesantes de todos los ramos, que se cuentan
por miles; conspiran los aguadores, los serenos y hasta las amas de
cría. Yo digo a los señores: «a las cabezas, a las cabezas\ldots»

---Y a las cabezas apuntan. Ya van saliendo deportados casi todos los
Generales\ldots{}

---Que es avivar la hoguera en vez de apagarla. Créame usted a mí, don
Pepito, que he visto mucho, y soy, aunque me esté mal el decirlo,
\emph{el testigo presencial} de la Historia de España, de la Historia
que no se escribe ni se lee\ldots{} Pues verá usted: las deportaciones
no sirven más que para poner en fiebre de revolución toda la sangre de
la Península.

---En fin, parece que han salido ya los Conchas, uno para Canarias y
otro para Baleares. Infante y Armero también están de viaje. ¿Y
O'Donnell, a dónde va?

---Debió salir para Tenerife; pero no hemos podido echarle la vista
encima. Se ha escondido, y locos andamos buscándole. Ese irlandés es muy
largo\ldots{} tan largo de cuerpo como de vista. Échele usted galgos.

---Para esa cacería y otras, don Francisco, le sobran a usted agudeza y
olfato. Y espero que podrá dedicar parte de su atención a este asuntillo
que le recomiendo. Fíjese, en que es un caso grave de violación de la fe
conyugal, en que esos loquinarios atentan a lo más sagrado, la familia,
el santo matrimonio\ldots{}

---¡Ay, mi don Pepito de mi alma!---exclamó moviendo la cabeza y
golpeando los brazos del sillón.---Dónde está ya en España la moral, la
familia y todo ese tinglado! Mire para el Cielo a ver si lo divisa por
allá, que lo que es aquí, tiempo hace que volaron las virtudes. Llevo
cuarenta años en esta faena, y cada día veo menos virtudes. A veces me
digo: «Será que esas señoras no andan por los caminos míos.» ¡Pero si yo
vengo y voy por todos los caminos, hasta por las iglesias! Y de palacios
no digamos\ldots{} En fin, que más vale no hablar.

Decía esto el fiero polizonte desfigurando por un instante su rostro
seco y amarillo con una sonrisa que adelgazó más sus delgados labios.
Sentado estaba frente a mí en un dorado sillón, estilo Luis XV\ldots{}
mirábale yo con examen casi impertinente; en él veía una figura del
pasado siglo, rígida, severa y no falta de elegancia. La chafadura que
tiene en la nariz, efecto de la pedrada con que le obsequiaron en su
juventud, le da la expresión de mal genio y de carácter torcido,
atravesado. Pues luego que echó de su boca los amargos conceptos acerca
de la dudosa moral de nuestros días, varió de tono para decirme: «En ese
asunto de la señora escapada con un silbante se hará lo que se pueda.
Considere que si tuviera yo un millón de agentes, no me bastarían para
perseguir los papeles clandestinos, y descubrir quién los escribe, quién
los imprime y los reparte. Son una peste las tales hojas secretas. En
los años que llevo en este oficio, no he visto desvergüenza
mayor\ldots{} Y, como usted sabe, ya no van las injurias sólo contra el
\emph{Gabinete}: van contra la misma Reina, de la que dicen
horrores\ldots{} ¿Cómo demonios se arreglan para que los papeles lleguen
a todas las manos, para que Su Majestad misma se los encuentre en su
tocador? Yo no lo sé\ldots{} Digo, sí lo sé. Es que en Palacio hay manos
traidoras, blancas o sucias, que de todo habrá\ldots{} y el Gobierno no
tiene poder para cortarlas o siquiera echarles un cordel\ldots{} Allí
dentro no puede nada Francisco Chico\ldots{} Yo se lo digo a Sartorius:
`Señor don Luis, mire que en Palacio hay mar de fondo y peces muy
malos\ldots{}'. Él suspira\ldots{} Tampoco puede nada.»

Dijo esto poniéndose en pie, forma cortés de señalar el término a la
visita. Me despidió con esta útil advertencia, que no he de echar en
saco roto: «Y ya sabe, don Pepito: en cuanto adquiera usted alguna
noticia por referencia, por soplo, por anónimo, véngase al instante acá.
No desprecie usted ningún dato, aunque le parezca mentiroso,
inverosímil\ldots»

\hypertarget{viii}{%
\chapter{VIII}\label{viii}}

\emph{24 de Febrero}.---La tempestad que tenemos encima ha lanzado en
Zaragoza chispazos que ponen miedo en los corazones. ¿Qué ha sido?
Continuación de la Historia de España, sublevación militar. Malo es que
empiecen los soldados con estas bromas, porque serán la Historia o el
cuento de nunca acabar. Dice la \emph{Gaceta} que la intentona fue
sofocada al instante, y lo creo, porque en estos duelos puramente
españoles entre la fuerza y la ley, el primer golpe suele ser en vago;
el segundo ya se verá. Refieren lenguas, no sé si buenas o malas, que el
brigadier Hore, impulsor y víctima del movimiento, contaba con más
fuerzas de las que efectivamente arrastró a la sedición, y que los
compañeros comprometidos le volvieron la espalda en el momento crítico.
Es la eterna quiebra y la eterna inmoralidad de estos arriesgados y
obscuros negocios, porque a los que desde el borde de la prevaricación
se vuelven a la disciplina, les premia el Gobierno con ascensos y
honores. En fin, que al pobre Hore le mataron en las calles de
Zaragoza\ldots{} La \emph{polaquería} se pavonea con su victoria, sin
ver el larguísimo rabo que falta por desollar.

Voy creyendo que este Gobierno toma por modelo al de la \emph{Sublime
Puerta}. No ha celebrado su triunfo de Zaragoza con actos de clemencia,
sino a la manera turca, decretando nuevas proscripciones, y metiendo en
las cárceles a cuantos infelices se han dejado coger. Previa declaración
del estado de sitio, la policía echó su red para pescar a los
periodistas de oposición, y a los directores de los diarios de más
ruido. Cayeron Rancés y López Roberts, de \emph{El Diario Español};
Galilea, de \emph{El Tribuno}, y Bustamante, de \emph{Las Novedades}.
Los cuatro fueron inmediatamente empaquetados para Canarias. Eusebio
Asquerino, que estaba enfermo, pasó de la cama al Saladero, y a Bermúdez
de Castro no le valió la procedencia moderada, ni el haber sido Ministro
de Hacienda en el Gabinete Lersundi: al romper el día le sacaron de su
casa, y en silla de postas, acompañado de guardias civiles, fue a tomar
aires al castillo de Santa Catalina de Cádiz\ldots{} Más listos otros,
supieron imitar la viveza escurridiza del sagaz O'Donnell, dándose buena
maña para no estar en sus casas ni en las redacciones cuando se personó
en ellas la policía para ofrecerles cortésmente sus respetos. No han
sido habidos Fernández de los Ríos, ni Montemar, ni Romero Ortiz, ni
Barrantes, de \emph{Las Novedades}; volaron también Coello, de \emph{La
Época}, y Lorenzana, de \emph{El Diario Español}. Pero ninguno de los
pájaros perseguidos ha dado tanta y tan inútil guerra como Cánovas,
contra quien se desplegó todo el ejército policíaco; ¿sabéis por
qué?\ldots{} Porque en sus conferencias del Ateneo sobre los políticos
de la casa de Austria retrató el malagueño a nuestros ministriles en las
figuradas personas de don Rodrigo Calderón y del Conde-Duque,
describiendo tan al vivo y con tan fino matiz de actualidad sus mañas y
picardías, que el público lo celebró como una sátira de las picardías y
mañas presentes\ldots{} Desapareció, como he dicho, Cánovas, burlando a
los ojeadores y sabuesos. Pero no ha salido de Madrid: en Madrid está;
lo sé, y sé también dónde.

He leído a mi mujer estos párrafos, y le han parecido bien. Después nos
hemos puesto a hablar mal del Gobierno, y no porque éste nos haya hecho
ningún daño, sino por la imposibilidad de sustraernos al enconado
pesimismo del medio ambiente. Repetimos todos los horrores que se dicen
de Sartorius y de sus desgraciados compañeros, y luego, por fin de
fiesta, dirigimos nuestros tiros a la calle de las Rejas, palacio de
Cristina, que es, según la fraseología de los papeles clandestinos, el
\emph{antro de la corrupción}, \emph{el inmundo taller de los
chanchullos de ferrocarriles}, y más, mucho más\ldots{} es un serrallo,
es un pandemónium donde se fraguan todos los planes maquiavélicos contra
la Libertad. Observamos luego que el sinnúmero de términos
estrambóticos, a troche y moche difundidos por periódicos y hojas
volantes, traen harta confusión al pueblo, que los oye y los repite
ignorando lo que significan. A este propósito me contó Ignacia que la
servidumbre de nuestra casa estaba el otro día en gran controversia
sobre el significado de la palabra \emph{agio}. Tanto la oyen, que
sienten, ¡pobrecillos!, la necesidad de saber lo que es. Entró mi mujer
en el comedor de criados cuando más acalorada era la disputa, y
Bonifacia, la pincha, pidiole que sacase de dudas a la reunión.
«Señorita, ¿quiere hacer el favor de decirnos qué son \emph{agios}?
Porque dice la Juana que debe ser algo así como \emph{ajos} echados a
perder\ldots» Echose a reír Ignacia, y como Dios le dio a entender
despachó la consulta. Pero vino la más gorda. Tiburcio, el mozo de
cuadra, planteó a la señorita un problema mucho más grave. «Señorita,
¿quiere decirnos lo que es eso de que tanto hablan los papeles, el
\emph{pandemónium?} (y lo pronunció acentuando la última sílaba),
porque, como no sea el \emph{pan de munición} que se da a los soldados,
no sé qué demonches podrá ser.» Mi mujer se moría de risa, y no pudo
explicarles lo que es \emph{pandemónium}, porque ella tampoco lo sabe.

---Bueno, querida mía---dije yo a mi cara mitad, cuando acabamos de
reír.---Estas jocosidades de la plebe también tendrán un hueco en mis
\emph{Memorias}.

Pues, hijo, mal historiador de tu tiempo serías si no lo hicieras. En
nada de lo que ves y oyes hay tanta Historia como en eso que te he
contado de los \emph{agios} y del \emph{pandemónium}. Ya ves: ¡un pueblo
que pide las cabezas de sus gobernantes sin saber de qué se les acusa!

---¡Sí lo sabe, sí lo sabe! El pueblo, que no es solamente la clase
inferior de la sociedad, sino el conjunto de todos los seres que se
llaman españoles, la gran masa nacional, posee la percepción clara de la
conducta de sus mandarines. ¿Cómo adquiere este conocimiento? Ello ha de
ser por fenómenos morbosos que nota en sí misma, estados eruptivos,
congestivos, qué sé yo\ldots{} por algo que le duele y le pica\ldots{}
Este picor doloroso es la conciencia nacional\ldots{} Este picor dice:
«los que me gobiernan, me engañan, me tiranizan y me roban.» La gran
masa todo lo sabe. Poco importa que los menos instruidos desconozcan el
valor de algunas voces. El enfermo, cuando algo le duele, tampoco sabe
designar su dolor con el terminacho científico que le dan los médicos.

---Está muy bien, gran Pepito. Y ahora, ¿por qué no empleas tu
perspicacia en buscar a Virginia, para que sus infelices padres tengan
algún consuelo?\ldots{} Tanto hablar, tanto ir y venir los primeros
días, y después nada.

---Yo no soy policía. Habrás visto que, en estos tiempos, Dios guarda
las espaldas a los que huyen, y protege a los escondidos. Si hay una
nube providencial para O'Donnell y Cánovas, haya también para Virginia y
su pintor\ldots{} \emph{el pintor de su deshonra}. Yo continúo
estudiando el caso, que es singularísimo, y hoy mismo he descubierto un
dato muy importante. Voy a decirlo: esta tarde he visto al \emph{Joven
Anacarsis} guiando un carricoche en la Castellana. En su rostro
epíscopo-infantil vi pintada una tranquilidad seráfica y un evangélico
menosprecio de los juicios de la opinión. Ya veo claro que Virginia, no
aviniéndose a tener por marido a un marmolillo, lo ha tirado al arroyo.

---Pepe, no desbarres\ldots{} ¡Vaya una moral que sacas tú ahora! Cierto
que los Rementerías, hijo y padre, están más frescos que una lechuga.
Anoche dio don Mariano José una gran comida\ldots{}

---A la que asistieron, lo sé, la flor y nata de la \emph{polaquería}:
el imponderable Domenech, Ministro de Hacienda; Saturnino de la Parra, y
Eduardo San Román\ldots{} También se atracó allí, como de costumbre, el
cetáceo Mora, y a los postres, al olor del riquísimo café y de los puros
de a cuarta, acudió el brigadier Rotalde, que ahora pide la bicoca de
ochenta mil duros por las obras del Teatro Real\ldots{} Y me dice el
corazón que se los van a dar. ¡Viva \emph{Polonia!}\ldots{} Volviendo a
Virginia y al \emph{pinturero}, te diré que me alegro de que no
parezcan.

---¡Pepe, Pepito!\ldots{} si no fuera por el aquél de que eres mi
marido, te tiraba un arañazo\ldots{} No me hagas reír.

\emph{Marzo de 1854}.---¡Bomba, bomba!\ldots{} ¡Gran novedad, estupenda
noticia!\ldots{} No, no es cosa de la Revolución\ldots{} Digo,
revolución es; pero no la chica, no la de liberales, o sean
\emph{chorizos} contra \emph{polacos}, sino la grande, la de\ldots{} Ha
llegado otra carta de Virginia.

La trajo el correo interior\ldots{} Aquí la copio, retocándole la
ortografía: «Pepillo, mala persona: ¿con que se pone en movimiento la
policía para buscarnos? Fastídiate, que no nos encontrarán\ldots{}
Porque recibas ésta con franqueo del Interior, no vayas a creer que
estamos en Madrid. Buenos tontos seríamos, y tú más simple que las habas
si lo creyeras. Vivimos muy lejos de esa Babilonia sucia; pero no tan
lejos que no nos llegue el mal olor\ldots{} Ya sabemos que se está
armando una muy gorda. Yo le pido a Dios y a la Virgen que \emph{haiga}
revolución, que \emph{haigan} tiros, y que escabechen a tantos
\emph{lairones}\ldots{} Quiero que por mi manera de escribir comprendas
que me estoy \emph{golviendo} muy ordinaria. Es lo que deseo: jacerme
palurda, y olvidarme de que fui señorita del pan \emph{pringao} y señora
\emph{de poco acá}.

«Para que tú rabies, y hagas rabiar a otros contándolo, te diré que
estoy contenta, fuera de la penita que me da el no saber de mis padres.
Harás el favor de decirles que me acuerdo mucho de ellos, y les deseo
paz y salud. La mía es buena. ¿Quieres que te cuente mi vida? Pues lee.
Dos semanas llevamos albergados en un magnífico garitón, llámalo más
bien pajar, donde no pagamos alquiler. Nos han dado esta espaciosa
vivienda de teja vana y paredes de tablas, con la condición de que
trabajemos. ¿En qué? No te lo digo. \emph{Él} y yo trabajamos, y sin
gran apuro nos ganamos la casa y el sustento\ldots{} Dormimos
tranquilos, nos levantamos antes que el sol, y oímos los canticios de
las aves del Cielo, que nos regocijan el alma. Rendidos nos acostamos a
la entrada de la noche; y como a nadie envidiamos ni nadie nos envidia
ni tenemos cavilaciones, nos coge pronto el sueño\ldots{} Hay aquí un
prado verde por donde yo ando descalza, Pepe, riéndome mucho de los
zapateros. ¡Vaya con el negocio, que harán conmigo! El viento me
despeina y me vuelve a peinar: es un peluquero \emph{à la dernière}, que
no pasa la cuenta como \emph{Monsieur Pinaud}, el de la calle de las
Infantas\ldots{} Más abajo del prado pasa un río, en el cual me meto yo
hasta las rodillas y lavo mi ropa y la de \emph{Él}. Luego la tiendo al
sol, y con este aire bendito, pronto se me seca, y me la traigo a casa
más blanca que la nieve\ldots{} ¡Ay, Pepe!, ¡de qué buena gana te
convidaría a las sopas que hago yo al anochecer en mi cazuela puesta
sobre una trébede!\ldots{} No has comido nunca cosa más rica. Le pongo
de todo lo que encuentro, y encima nuestra alegría, que es la sal, y
nuestro buen diente, que es el picante. Son unas sopas que
\emph{ajuman}. Ya ves qué fina me estoy volviendo. Bueno; pues te lo
diré en francés: cenamos \emph{potage aux finis yerbis}, y luego
alabamos a Dios, acostándonos en nuestra cama grandísima, que también es
de \emph{yerbis}\ldots{} Sabrás que no la cambio por la de la Reina.

»¡Qué gusto tan grande no tener que ocuparse de lo que dirá don
\emph{Efe} y don \emph{Jota}, ni de lo que murmurarán las de \emph{Eme!}
Este vivir libre y sano no lo conoces tú, ni ninguno de los desgraciados
que se pudren en ese presidio, condenados a pensar en el sastre, en la
modista, en lo que traerá el cartero, en lo que dirá el periódico, en si
cae el Gobierno, en las pisadas del aguador y en el precio de la
carne\ldots{} Sólo de pensar que he vivido de ese modo, se me nublan las
alegrías\ldots{} ¡Ay, Pepe!, para que le puedas decir a Madrid todo mi
desprecio, te pongo aquí una larga fila de \emph{emes}\ldots{}

«Con que, mi buen Pepín, haz el favor de poner a un lado la moral, o
\emph{morral}, que gastáis vosotros para disimular tantos crímenes, y
dejarnos aquí en paz, o donde estuviéramos. ¡Cuidado con echarnos la
policía! Nosotros no hemos hecho daño a \emph{naide}; \emph{semos}
libres, y el único que podría perseguirme, que es ese \emph{Caranarsis},
o \emph{Acanársilis}, ya no me acuerdo, no dará ningún paso contra mí,
por la cuenta que le tiene.

»Y ahora, señor \emph{morralizador}, allá van memorias para los que por
mí preguntaren. Al Ernesto, aunque no pregunte, le dirás que estoy muy
contenta desde que le he perdido de vista, y como cosa tuya le das una
palmadica en las mejillas sonrosadas. A mi suegro, director de \emph{La
Previsora}, por mal nombre \emph{El Robo ilustrado}, le darás
expresiones. Paréceme que le tengo delante cuando, después de atracarse
como un buitre en las comidas, se lleva la mano a la boca con finura
para tapar un regüeldo. Con las expresiones le darás un papirotazo en
las narices\ldots{} como cosa tuya, se entiende. Y si quieres que
después del papirotazo te dé don Marianico las gracias, asegúrate la
vida aunque sea por dos cuartos al año\ldots{} A todos los que suelen ir
de comistraje a la maldita casa donde tanto pené, les das mis recuerdos,
y con disimulo les metes pica-pica por el cuello de la camisa, para que
se estén rascando tres días con sus noches. Eso pensé yo hacer con el
Ministro de Hacienda, señor Domenech; pero no me atreví. Me parece que
le estoy viendo, tan pulcro, tan tiesecito, sin juego de la bisagra del
pescuezo. Siempre que tiene que mirar a un lado, ladea todo el
cuerpo\ldots{} Al gordo don José Mora, memorias también, y que deseo que
alguien le dé una patada y que vaya rodando, para que reviente y podamos
ver lo que lleva dentro de aquel barrigón\ldots{} A mi tía Cristeta, que
es una enredadora, de ti para mí, y la que lleva los chismes a Palacio,
le dirás que le deseo una pulmonía. Ella es, para que lo sepas, la que
mete en la cámara de la Reina los papeles clandestinos, y al mismo
tiempo alcahuetea en otras cosas. Es mi tía y no digo más. En señal del
amor que le tengo, te encargo que le levantes las enaguas y le des una
buena solfa en semejante parte\ldots{} Expresiones a la Puerta del Sol,
que yo vea convertida en hoguera donde se achicharre tanto pillo;
expresiones a la Cibeles, llevándole de mi parte un poco de cordilla
para sus leones; memorias al \emph{salooon} del Prado, y le pongo muchas
oes para expresar lo que me he aburrido en él; y memorias a los teatros.
Te vas a cualquiera, y echas una mirada al público, y le dices de mi
parte que estoy contentísima de no verle. Doy gracias a Dios porque me
ha concedido oír el ruido del viento en vez de oír palmadas, y el
\emph{jipío} de las actrices\ldots{}

«Pepito, siento que no conozcas una cosa que yo he descubierto y
disfruto en algunos instantes, después que me tomé lo que era mío: mi
preciosa libertad. ¿No sabes lo que es esto que yo disfruto y tú no?
Pues es la alegría, una onda fresca que sale del fondo del alma y te
embriaga, y te hace más enamorada de lo que amas, y más\ldots{} en fin,
no sé decirlo. Tú lo entenderás, porque, como buen entendedor, ya lo
eres.

«Si yo supiera que tranquilizabas a mis padres y les convencías de que
no deben llorarme, sería completamente dichosa, y te estaría muy
agradecida. Hazlo, por Dios, Pepe; hazlo por tu niño y por tu mujer. A
esos tus seres queridos, mando abrazos y besos. Y ya sabes que, sin
saber dónde, tiene \emph{dos} buenos amigos: te lo dice la que lo fue y
lo es\ldots{} \emph{Virginia.»}

\hypertarget{ix}{%
\chapter{IX}\label{ix}}

\emph{Marzo}.---Mi mujer y yo:

---Ese idilio\ldots{} ¿no se dice \emph{idilio}? será interrumpido,
cuando menos lo piensen, por la Guardia Civil.

---La Guardia Civil, mujer, está ahora muy ocupada con otros idilios.

---Según eso, ¿tú crees que les durará la libertad, y que esa alegría,
de que habla la muy bribona, será eterna? ¿Crees que se pueda vivir en
ese salvajismo, sin que les salgan mil calamidades, la miseria, la
envidia y las malas voluntades de los pueblos, y acaben por hacerse
aborrecibles el uno al otro, y maldecir la hora en que se juntaron
violando\ldots?

---Acaba, mujer; es frase que se dice sola: \emph{violando todas las
leyes divinas y humanas}\ldots{}

---Yo, qué quieres, dudo que tanta dicha sea verdad. ¿Sabes lo que es
esa chica? Una gran embustera. ¡Sabe Dios, sabe Dios cómo estarán!
Llenos de miseria, con más hambre que Dios paciencia, y deseando que la
Guardia Civil les coja y les lleve bajo un techo de abrigo, aunque sea
la cárcel.

---Yo creo lo contrario: que viven pobres y felices, sin ambición, sin
cuidados. En la vida complicada, presa en mil artificios, a que nos ha
traído la civilización, hemos perdido la idea de la verdadera felicidad.

---Podrá existir la felicidad en un mundo en que todos los seres sean
salvajes y buenos; pero ese mundo, ¿dónde está? A las puertas de las
ciudades, el salvajismo no puede existir, y si existe tiene que ser de
corta duración.

---Quizás; no te digo que no. Nos falta saber en qué se ocupan ella y
él, y con qué especie de trabajo se ganan la vida. ¿Son labradores,
poseen algún ganado? Esto no lo dice la carta. Supongo que donde ellos
viven no habrá puertas que pintar, ni cerraduras que componer.

---Habrá otras cosas y otros oficios; vete a saber\ldots{} Ya sabes lo
que él dijo: «soy amañado para todo.» Puede que sea leñador o carbonero;
que recoja hierbas para los boticarios; que pesque anguilas o
sanguijuelas. Di otra cosa: ¿en la carta no habla de viñas?\ldots{}

---No nombra viñas, ni dice que beban vino.

---Lo pregunto por ver si los datos de ella casan con uno que hoy me han
traído\ldots{} dato importante, que puede dar mucha luz\ldots{} Pues
verás: ya te dije que las criadas de Virginia me hablaron de una
lavandera que por aquellos días iba mucho a la casa, madre de la
Casiana, que a nosotros nos sirvió el año anterior. La hemos buscado:
dimos ayer con ella\ldots{} Nos ha dicho que una tarde, entrando en la
galería donde el pintor estaba dale que dale a la brocha, le oyó decir,
como respondiendo a una pregunta que ella le hizo acerca de su
familia\ldots{} de él: «Tengo una hermana casada con un rico de la Villa
del Prado.» Y ella dijo: «Pues ya le mandará a usted buenas uvas.» Por
eso te pregunté si en la carta habla de viñas.

---A juzgar por la carta, el sitio en que están no revela la vecindad de
una hermana rica, ni de nadie que verdaderamente les ampare. La vida
salvaje y mísera de que habla Virginia debe de estar lejos de toda
ciudad, villa o villorrio. Presumo yo que es en la falda de la
Sierra\ldots{} en lugar medio despoblado.

---Por sí o por no, llévale pronto este dato al señor Chico, o al
Gobernador de la provincia, para que pidan informes al Alcalde de allá,
o a cualquier conocido\ldots{} Todo es empezar, Pepe\ldots{} Verás cómo
de una referencia sale otra, y al fin la verdad y el escarmiento de esos
pícaros. Valeria, en cuanto supo el dicho de la lavandera, se fue a ver
a unas amigas, que son de un pueblo próximo a ese de las uvas:
Cadalso\ldots{} ¿Hay un pueblo que se llama Cadalso?\ldots{} Pues las
amigas han quedado en escribir\ldots{} Y ya que hablo de Valeria, Pepe,
tengo que contarte\ldots{} Hoy me he cansado de reñirla. Figúrate: los
padres están chochos con ella. Naturalmente, es la honrada, es además la
única, porque a la otra la tienen por muerta. Y ella, la muy ladina, se
aprovecha\ldots{} Sabrás que le ha entrado el delirio de la casa
elegante, de los muebles de última moda, cortinas a la \emph{Gobelín},
alfombras de moqueta, y reclinatorio y estantitos maqueados\ldots{} sin
contar otras elegancias y refinamientos. No hay mañana que no eche dos o
tres horas a tiendas.

---Historia, hija, Historia de España. Sigue.

---Ya sabe que los padres no le niegan nada. Es la buena, es la honrada,
es la única. Si les ve reacios, allá van cuatro carantoñas, y ya tienes
catequizados a los pobres viejos. Con una mano se limpian la baba que se
les cae, y con la otra sacan y acarician la bolsa, que sólo se abre para
la niña. Ésta les besuquea, y corre a las tiendas a pagar lo que debe y
a traer más, más\ldots{}

---Historia de España\ldots{} ¡y qué Historia! Adelante.

---Ayer me la encontré en casa de los \emph{Hijos de Sobrino}, en
Majaderitos, donde fui a comprar tela para los delantales del niño, y en
poco más de un cuarto de hora hizo Valeria compras de batista superior
para camisas, y de adorno en blanco, por valor de mil y cuatrocientos
reales\ldots{} Después fue a la perfumería de Quiroga, y se dejó una
buena porrada de duros.

---Historia nacional, retrato del pueblo español\ldots{} Sigue\ldots{}
Entre paréntesis: a Valeria le ha sentado bien el matrimonio; se ha
puesto muy linda.

---Es una monada\ldots{} Pues sigo. Como yo, cuando me intereso por una
familia, no reparo en tomarme todas las libertades, también he reñido a
Navascués\ldots{} como lo oyes: ¡ayer le eché una andanada!\ldots{} Al
hombre, un color se le iba y otro se le venía. Pues ¿no es un dolor ver
que esa pobre niña no halle distracciones y alegría más que en las
tiendas?\ldots{} Y todo porque al zángano del marido se le cae encima la
casa, y no sabe vivir fuera del Casino y los cafés, demente con la
dichosa política. ¿Sabes, Pepe, que, a mi parecer, este joven va por mal
camino? ¿Quién le mete a regenerador de la patria? ¡Lucida estaría esta
pobre enferma si sus médicos fueran capitanes y tenientes! Navascués es
de los que creen que, echando a los \emph{polacos}, ataremos aquí los
perros con longaniza\ldots{} Pues en los \emph{Dos Amigos} le tienes
mañana, tarde y noche\ldots{} me lo ha dicho él mismo con una ingenuidad
que le honra\ldots{} allí le tienes \emph{siguiendo paso a paso}, son
sus palabras, \emph{el movimiento revolucionario}, y sacando la cuenta
de los comprometidos, de los que no quieren comprometerse\ldots{} O
mucho me engaño, o este joven nos dará el mejor día un disgusto.

---A mí no. Sigue, hija, sigue: tu capítulo de Historia no tiene
desperdicio.

---Yo le he puesto de vuelta y media\ldots{} «Usted es un simple,
Rogelio, o un ambicioso vulgar; y si no es esto, seguramente será otra
cosa peor. Todo militar que no se encierre en la esclavitud de la
disciplina, es un perjuro\ldots{} A usted le han dado esa espada y le
han puesto ese uniforme para que defienda la ley, no para que se meta
locamente a cambiarla. ¿Quién es usted para cambiar la ley? Eso es
cuenta de otros. Usted no sabe una palabra de leyes, ni ha cogido jamás
un libro, como no sea el de la Táctica. ¿De dónde saca toda esa
palabrería que ahora usa? Quisiera yo poder oír, por un agujerito, las
gansadas que usted y sus amigos hablarán en el café\ldots{} Ya puede
andarse con cuidado. El mejor día le recetan los aires lejanos de
Filipinas, o le encierran en una fortaleza, si no es que el niño se va
del seguro, y entonces, ¡pobre Rogelio!, sus cuatro tiros no hay quien
se los quite.»

---Ese caso no llegará, porque triunfarán los sublevados\ldots{} ahora
toca triunfar; lo asegura el historiador\ldots{} y Navascués tendrá el
ascenso que busca. Si he de decirte lo que siento, Ignacia, los
militares, siguiendo la rutina histórica, no van a cambiar la ley, sino
a restablecerla, a levantarla del suelo en que arrojada fue por la
\emph{polaquería}. Esto debe hacerlo el pueblo, la masa total; pero aquí
nos hemos acostumbrado a que el pueblo delegue esa función en los
militares, y ya no es fácil cambiar de sistema. Lo que te digo es un
hecho, que arranco de las entrañas de la Historia efectiva, muy distinta
de esa otra Historia que sale al mundo cubierta de artificios, como una
vieja que se adoba el rostro, y todo lo lleva postizo, empezando por el
lenguaje. Los militares se sublevan cuando la Nación no puede aguantar
ya más atropellos, inmoralidades y corrupciones, y en estos casos el
brazo militar triunfa, sencillamente porque debe triunfar\ldots{} Y con
esto dimos fin a nuestra charla sabrosa, porque llegó la hora de comer,
que todo llega en este mundo.

\emph{Abril}.---Apenas salgo del fastidioso ataque de reúma que me ha
tenido cerca de un mes condenado a encierro, tristeza y emplastos de
belladona, me decido a vaciar mis pensamientos sobre el papel de estas
\emph{Memorias}, donde me atormentarán menos que amontonados en el
caletre. Allá voy con el material histórico que almacenado tengo aquí; y
empiezo por afirmar que la conspiración continúa su labor profunda, pero
no se la ve, porque se ha metido bajo tierra y\ldots{}

Espérense un poco, que aquí llega, como llovido, un asunto al cual es
forzoso dar la preferencia. ¿No lo dije? Cartita de esa loquinaria, de
esa que ha hecho mangas y de los santos principios, de esa, en fin, que
ahora la gaita de resucitar la edad de oro, funesta para los sastres y
maestros de obra prima\ldots{} Llevo la epístola a mi mujer, que la lee
en voz alta. Dice así:

«Ay, Pepe, déjame que te cuente las amarguras que he pasado! Te
horrorizarás cuando leas esta carta y me tendrás mucha compasión,
¿verdad que sí? Aplacado el sufrimiento mío, puedo contártelo, para que
lo sepa María Ignacia, y lo sepan también mis padres y hermana. Tan
desgraciada he sido, que creí que Dios me castigaba cruelmente; mas
ahora veo que no ha sido castigo, sino prueba, y que de ella sale mi
alma como de un crisol, con lo que ahora está más fuerte, más brillante;
y si no lo crees, entérate de lo que te escribo\ldots{} Pues sabrás,
Pepillo, que hará hoy catorce días, a punto de anochecer, vino del
trabajo mi \emph{Ley} muy alicaído, con la cara arrebatada y quejándose
de un horrible dolor de cabeza. (Pongo este paréntesis, querido Pepe,
para decirte que le llamo \emph{Ley}, porque de algún modo he de
llamarle, que ahora de él tengo que hablar, y me será preciso nombrarle
a menudo; conque \emph{Ley}, ya sabes.) Su piel abrasaba, y transido de
frío daba diente con diente. Le hice acostar y le arropé lo mejor que
pude. Todo se me volvía decirle: «\emph{Ley}, ¿qué tienes?» y él no me
respondía: estaba como aletargado, de la fuerza del dolor y de la
calentura\ldots{} Yo, como puedes suponer, angustiadísima; hazte
cargo\ldots{} \emph{Ley} enfermo; \emph{Ley}, que es mi vida, como si
fuese a perder la suya. ¡Y yo sin tener a quién volverme, ni a quién
pedir socorro; yo sola con él, y sin médico ni botica\ldots{} con las
estrellas encima por únicos testigos de lo que me pasaba!\ldots{}

»En fin, para mis adentros dije: aquí yo con mucho valor, y sobre mí y
sobre mi \emph{Ley}, la voluntad de Dios. Lo primero fue calentar agua:
afortunadamente tenía un poco de azúcar morena, como unas dos libras;
tenía también algo de vino. Pues a darle agua templada con azúcar y unas
gotas de vino; no había otra cosa: el corazón me decía que aquello era
muy bueno\ldots{} La noche, ya puedes figurarte cómo fue. \emph{Ley},
abrasado de calor, a destaparse, apartando con sus manos la paja y la
única manta que tenemos, agujeradita; yo a volverle a tapar, y a darle
calor con mi cuerpo. Te advierto que nuestra habitación es como una
jaula, y que por los costados y techo, las troneras y rendijas dejan
entrada libre a los aires de Dios. Y la noche era ventosa; no quiero
decirte más\ldots{}

»En fin, Pepe, lo que te cuento fue principio de una larga y malísima
enfermedad, que no sé cómo se llama, pero para mí que es algo como
tabardillo; y si en los primeros días pareciome que no iba peor, de
repente le entró una tan grande agravación, que llegué a creerme que me
quedaba sin \emph{Ley}\ldots{} ¡Dios mío, lo que he penado! Ahora que
pasó todo, pienso que Dios no está en contra mía, sino a favor: buena
prueba me ha dado de ello. Yo no tenía recursos, ni a quién llamar en mi
auxilio. Como a distancia de medio cuarto de legua están los vecinos más
cercanos. Son dos viejos, marido y mujer, con un nieto enano, idiota y
casi mudo, pues sólo dice \emph{mu}, como los animales. Corrí a darles
aviso; fueron a verme; lleváronme unas patatas, más vino, hierbas de
malva para cocimiento, hierbas de sanguinaria y pan. Después no
volvieron, y mandaban al mudo a que preguntara\ldots{} ¡Pues fueron ocho
días, Pepe, que me parecieron ocho siglos! Cada tarde creía yo que mi
\emph{Ley} anochecía y no amanecía, y por las mañanas pensaba que no
vería la tarde. No puedes imaginar mi angustia. Siempre he querido a
\emph{Ley}: figúrate si le amé, que por unirme con él tiré al arroyo
familia, sociedad, posición, todo. Luego de unirnos, le quise más, sin
que mi amor flaqueara ni un punto en ninguna ocasión. Pues viéndole con
aquella enfermedad terrible; viendo que se me moría por momentos, sin
que yo pudiera evitarlo, le quería y le adoraba de una manera tan loca,
que yo no sé, Pepe, no sé que haya palabras con que expresártelo. Y
cansada ya de pedir inútilmente a Dios y a la Virgen que no me quitaran
a \emph{Ley}, les pedí con muchísimo fervor que me llevaran a mí también
en el instante en que él muriese\ldots{}

«El día y las dos noches en que llegó al extremo peligro, si muere o no
muere, noches y día que no puedo señalar, porque para mí no hay
almanaque, ni fechas, ni nada de eso, los pasé como puedes figurarte,
abrazada a mi \emph{Ley}, queriendo darle vida con mi aliento, fija la
vista, fijo el oído en su respiración fatigosa, que a cada rato me
parecía con un compás más lento, y yo no cesaba de pensar que una de
aquellas respiraciones sería la última. Cuando no hacía esto, ponía yo
en limpiarle toda mi atención: pensaba que limpiando su cuerpo de la
miseria de la enfermedad había de salvarle\ldots{} Y entre tanto, a lo
que llamaré mi casa, por darle algún nombre, no llegaba más que el mudo,
que desde la puerta decía \emph{mu}, y con él un perro que se colaba
dentro y me revolvía todo. El mudo me veía llorar, y corría con la
noticia de que \emph{Ley} se estaba muriendo. Yo decía: «¿Pero tan mala
soy, Señor, que así me abandonas?» A la Virgen de los Dolores, a quien
siempre he tenido devoción, le rezaba yo con todo el fervor de mi alma
para que me amparase, y me la figuraba con la imaginación, por no tener
delante efigie ni estampa en que fijar mis ojos\ldots{} En la madrugada
del último día, viendo a \emph{Ley} que, después de una gran congoja, se
quedó atontadito y como si durmiera, me puse de rodillas, y a grandes
voces pedí a la Virgen que me socorriese, dejándome la vida de
\emph{Ley}, o llevándose la mía con la suya. Después de amanecer, le
acometió otra congoja tan fuerte que pensé que de ella no volvía\ldots{}
Le di agua con azúcar, que era toda mi farmacia, y se le calmó la
sofocación. Pareciome que respiraba mejor. Dijo algunas palabras, le di
muchos besos, y me reí para ver si le hacía reír.

«El resto de la mañana fue de mayor tranquilidad. A ratos me hablaba,
diciéndome con mimo que no me separase de él; que no le hacía falta
médico, ni medicinas, ni nada más que verme. Viéndome, creía el pobre
que se iría curando\ldots{} Por fin, a la tarde, observándole despejado
y con más animación en los ojos, tuve alguna, muy poca esperanza; pero
yo me empeñaba en aumentarla pidiéndole a la Virgen y al Señor
Crucificado que, después de darme aquel poquito de esperanza, no me la
quitasen. A la noche, \emph{Ley} no tuvo recargo; se despejó mucho, se
le animó el rostro, crecieron mis esperanzas\ldots{} me andaba por toda
el alma una luz divina. Me quedé dormida junto a \emph{Ley}, tan rendida
estaba de tantas noches, y él se durmió también. Yo desperté primero, y
estuve un gran rato mirándole dormir, y escuchándole la respiración, que
ya era sosegada\ldots{} Despertó \emph{Ley}, y echándome los brazos me
dijo: \emph{«Mita}, de ésta no muero\ldots» ¡Ay, qué alegría se me metió
por los oídos y por los ojos, viéndole y oyéndole! Desde aquella
madrugada, ya las esperanzas fueron a más, a más, hasta que he visto a
mi \emph{Ley} salvado\ldots{} y con mi \emph{Ley} salvado, ya soy tan
feliz, Pepe, que no cambio mi choza por todos los palacios del mundo. Y
viendo que la Virgen y el Señor han librado de la muerte a \emph{Ley},
por el afán y dolor grande con que yo se lo pedí, bendigo mi pobreza,
bendigo mi soledad, y no quiero otra vida ni otro mundo.

«Ahora que \emph{Ley} se va fortaleciendo, y sacudió aquel terrible mal,
todo me parece bueno, todo muy bonito; y cuando el viento entra silbando
en mi alcázar por los huecos y rehendijas, se me antoja que viene a
felicitarme por haber arrancado a \emph{Ley} de la muerte, con la ayuda
de Dios, sin más medicina que mi cariño y las agüitas azucaradas; y
presento mi cara a los vientos para que me la besen, y les digo: «Venid,
aires del Cielo, a ver a \emph{Mita} contenta\ldots»

\hypertarget{x}{%
\chapter{X}\label{x}}

Suspendió María Ignacia la lectura, y llevose la mano al pecho, como si
el aliento le faltara. Un ratito estuvimos los dos silenciosos,
mirándonos. Yo fui el primero en vencer la emoción.---¿Qué piensas de
esto?---le dije.---¿Te parece que debemos apurar las averiguaciones del
sitio en que están, para que pueda ir allá la Guardia Civil y traerles
codo con codo?

---Eso no\ldots{} ¡pobrecitos! Sepamos dónde están para mandarles un par
de mantas, ropa, comida\ldots{} Pero ¿no vivirían mejor en un pueblo,
por miserable que fuera?

---Ya ves que no les va tan mal en ese despoblado. Es muy probable que
en un villorrio, asistido \emph{Ley} por curanderos o veterinarios, y
metido en un local fétido, no habría escapado de la muerte, mientras
que, en la choza ventilada, el cariño de \emph{Mita} y las agüitas con
azúcar le han sacado adelante.

---¿Y por qué la llamará \emph{Mita}? ¿Qué quiere decir \emph{Mita}?

---Contracción será de algún nombre cariñoso, inventado por él. Estos
amantes libres, por borrar la última relación con el mundo que
abandonan, suprimen hasta sus nombres de pila.

---Sigue leyendo tú: aún faltan dos carillas\ldots{} Yo no puedo más.
¡Siento una opresión\ldots{} y unas ganas de llorar\ldots!

Aquí va el resto de la carta, que yo leí: «Desde que vi a \emph{Ley}
fuera de peligro de muerte, hasta que se recobró y fortaleció, volviendo
a ser lo que era, han pasado otros ocho días, en los cuales he tenido
que discurrir mucho para sacar adelante a mi amado convaleciente. Pero
como ya estaba yo tranquila y contenta, por nada me afligía, y el afán
de las dificultades lo compensaba el gusto de vencerlas. Era forzoso
alimentar a \emph{Ley} para que recobrara sus perdidas fuerzas y se le
renovara la sangre. Pero carecíamos de todo recurso, y no había más
remedio que buscarlo\ldots{} Yo seguía pidiendo socorro a Dios y a la
Virgen, y éstos, a mi parecer, me decían: «busca y encontrarás.» Porque
no habían de traérmelo los ángeles\ldots{} Acudí primero a los vecinos
de que antes te hablé, y me dieron pan, cebollas y un poco de vino; esto
no me bastaba. Algún alimento más delicado necesitaba mi enfermo.

»En esto, llegó un día que me sonó a domingo; en esta soledad conozco
los días de fiesta por los sones de campanas que el viento me
trae\ldots{} de campanas llamando a misa en pueblecitos que están
distantes. Pero el viento, unos días más que otros, trae los toques de
campana tan al vivo, que parece que las tienes a un tiro de fusil. Yo le
dije a mi \emph{Ley}, después de arroparle bien y darle unas sopas en
vino: «Hoy es domingo, \emph{Ley}: si tú me prometes estarte aquí bien
tapadito, sin que te entren tentaciones de echarte fuera, yo me voy a la
iglesia que campanea, y en ella oiré misa y daré gracias a Dios por
haberte curado. Y como en derredor de esa iglesia ha de haber un pueblo,
después que oiga misa buscaré almas caritativas que me den algo para tu
alimento.» Y \emph{Ley} me dijo: «\emph{Mita}, ve a la iglesia que
campanea y da gracias a Dios por haberme salvado. Después buscarás almas
caritativas que nos socorran. Te prometo no moverme; pero no tardes más
de lo preciso, que estaré muy triste sin ti\ldots» Dejándole tan
conforme me puse en camino. Era un día, Pepe, que\ldots{} me río yo de
lo que llamáis días buenos en ese Madrid pestilente\ldots{} yo no sé
decirte cómo aquel día era. Mucha luz, un sol que consolaba sin calentar
demasiado, y un aire fresco que, sin alborotar, hacía ruiditos mansos en
las encinas\ldots{} Los pajarillos, las maricas y los cuervos, tan
contentos todos, buscando cada cual su remedio\ldots{} Pues, señor,
anduve, anduve, siguiendo la dirección que me indicaban los toques de
campanas, y llegué por fin a un cerro, desde donde divisé un campanario,
y otro más allá\ldots{} pero la torre más cercana distaba todavía como
un cuarto de hora\ldots{} No se me apartaba del pensamiento mi pobre
\emph{Ley}, allá tan solito, y los minutos que tardara en volver a su
lado me parecían siglos. Calculé que si me llegaba hasta el primer
campanario, se me iría toda la mañana; y estando en estos cálculos del
tiempo y la distancia, tuve una inspiración, Pepe\ldots{} tuve la idea
de oír mi misa en el mismo cerro donde me hallaba. Me arrodillé, mirando
al campanario, y rodeada del sol y el viento, con tanto mundo de
campiñas y montes delante de mis ojos, le dije al Señor y a la Virgen
todo lo que se me ocurría\ldots{} que no fue poco\ldots{} y cosas muy
sentidas y de mucha religión se me vinieron al pensamiento, y del
pensamiento a la boca, puedes creérmelo.

»Cuando yo estaba en lo mejor de mi misa, sonaron más las campanas
próximas y otras lejanas, como si hubiera gran festejo y
procesión\ldots{} De rodillas estuve un largo rato, y al concluir mi
misa, pensaba que por allí cerca encontraría el socorro que necesitaba
para \emph{Ley}. Yo había visto dos casitas; las volví a mirar: eran
blancas, y sus chimeneas echaban humo\ldots{} Bien podía ser que en
ellas vivieran almas caritativas\ldots{} No había dado yo cuatro pasos
hacia las casitas, cuando sentí son de cencerros, y vi que por el cerro
subían cabras; tras ellas venían dos hombres y un chiquillo. No creas
que me dio reparo de pedirles limosna. Les conté lo que me pasaba, y que
había dejado a \emph{Ley} acostado, convaleciente de una terrible
enfermedad. Les rogué que, por el amor de Dios, me dieran un poco de
leche, que yo sé trabajar. «\emph{Ley} también sabe---dije,---y en
cuanto se ponga bueno trabajaremos y pagaremos la leche que nos den.» El
más viejo de los pastores, alto y huesudo, con unas barbas muy grandes,
que parecían las del Padre Eterno, se encaró conmigo, y poniendo la cara
como de enfadarse, y echando un vozarrón que atronaba, me dijo: «Alguna
leche le diéremos, mujer; mas no trujo cuenco para llevarla. ¿Llevarla
ha en el pañizuelo?» Yo le contesté que no había traído cuenco porque no
pensé encontrar rebaños; pero que pediría me prestasen un jarro en
aquellas casas de abajo. Y él entonces, echando el vozarrón más fuerte,
y enarbolando el palo como si quisiera pegarme, me dijo: «Arrea
\emph{cacia ti}, mujer, que allá te daré la leche.»

«Hacia casa me vine, y conmigo el viejo parecido al Padre Eterno, y las
cabritas, que eran cuatro, muy saltonas, con las ubres contoneándose
entre las patas. Por el camino hablamos poco; el viejo echaba un
cantorrio entre dientes. Me preguntó cómo me llamo, y le contesté que me
llamo Ana. Nunca declaro mi verdadero nombre. El dijo: «Arrea, moza, que
tengo priesa\ldots{} Voy a bajarme con mis cabras a\ldots» (callo este
lugar, que es un soto junto al río). Pues llegamos a casa; me adelanté
corriendo para ver si \emph{Ley} estaba bien arropadito, y le encontré
lo mismo que le había dejado\ldots{} y tan contento de verme. No
necesité decirle lo que le traía, porque cuando el viejo y sus cabras
entraron en mi guarida, ya tenía yo dispuesto un cazolón bien lavado
para la leche que el buen pastor quisiera darme.

»Sin decirnos nada, se puso el hombre a ordeñar, y yo a tener el cazolón
y a ver cómo salían de los pezones de las ubres los hilos de leche,
alternando uno con otro y cayendo con fuerza dentro de la vasija. A
medida que ésta se iba llenando, los chorritos levantaban espuma. ¡Ay,
Pepe!, lo que entonces sentí, no puedo explicártelo\ldots{} Viendo los
chorritos de leche, y oyendo la musiquita que hacían, aquel rasgueo y
aquel \emph{chirrís-chirrís}, se levantó en mi alma una alegría tan
grande, tan grande, que no podía yo tenerla dentro, y me eché a
llorar\ldots{} Mis lágrimas corrían silenciosas. No había más ruido que
el de los hilos de leche\ldots{} Ordeñada una cabra, luego fue el hombre
con otra\ldots{} «Basta, señor,» le dije yo con toda mi alegría y mi
agradecimiento y mis lágrimas, que no acababan de correr\ldots{} ¡Ay,
Pepe, Pepillo loco!, esta alegría, ni tú ni María Ignacia la habéis
sentido nunca, ni sabéis lo que es\ldots»

Suspendí la lectura viendo que mi mujer, vencida de su grande
sofocación, rompía en llanto, y con su gesto me decía que callase.
Hicimos un descanso, sin cambiar observación alguna, hasta que al fin
María Ignacia, recobrado su aliento, pudo decirme: «¡Qué pena siento,
Pepe, qué vacío tan grande aquí!\ldots{} ¡Pobre \emph{Mita}! Una duda
tengo todavía: después la sabrás\ldots{} También me extraña mucho que en
la miseria de esa choza, donde se carece de todo, haya papel, tintero y
pluma para escribir carta tan larga.

---Espérate un poco. Pasando la vista por el pliego último, me parece
que he visto la palabra \emph{tintero}. Si te parece, acabaré. Ya falta
poco.» Sigo leyendo: «Puse a cocer la leche, y todo el día estuve
dándole a \emph{Ley} racioncitas cortas y frecuentes, marcando el tiempo
con el reloj de mi cuidado. ¡Oh, cómo le gustaba la leche y cómo se
relamía de gusto, pidiéndome más, más! Por horas, por minutos, le veía
yo reponerse\ldots{} Pues al siguiente día, a punto del amanecer, el
viento me trajo son de esquilas. Salí a ver, y era el \emph{Padre
Eterno} que venía con sus cabras a darnos más leche. «No te apures,
Anica---me dijo con su vozarrón, viéndome algo confusa ante tanta
bondad.---Ya me la pagarás cuando puedas\ldots{} y si no puedes, que
pase a la cuenta de las ánimas\ldots» Pues asómbrate, Pepe: volvió el
pastor otro día y otro, y \emph{un porción} de días\ldots{} ya ves cómo
me voy afinando de lenguaje\ldots{} En fin, que \emph{Ley} sale
adelante: pronto volverá al trabajo. Ayer bajé yo a lavar al río\ldots{}
Tan alegre estoy, que a lo mejor me pongo a cantar\ldots{} canciones
mías, cosas que invento. Ni yo misma sé lo que canto, porque es como un
gorjeo\ldots{} Ayer subía yo gorjeando del río, y por el camino, con mi
carga sobre la cabeza, decía yo: «Tengo que escribir esto a Pepe, para
que él y María Ignacia, las personas que más estimo después de mis
padres, sepan lo desgraciada que fui y lo dichosa que soy.» En la puerta
de mi palacio estaba \emph{Ley} sentadito, esperándome. Yo me quité la
carga y senteme a su lado. En aquel momento llegó \emph{Mu}, el enano de
nuestros vecinos: nos traía cebollas, pan y dos lechugas. Como el pobre
chico no dice más que \emph{mu}, y no sé su nombre, \emph{Mu} a secas le
llamo yo. Pues le dije, digo: \emph{«Mu}, te agradeceré que me traigas
el tintero de cuerno y la pluma de tu \emph{güelo}. Papel tengo yo.» En
cuanto se fue \emph{Mu}, le dije a \emph{Ley}: «¿Te parece que escriba
al buen amigo de Madrid todo esto que hemos pasado? Así verán allá que
Dios mira por nosotros.» Y \emph{Ley} me dijo, dice: «Cuéntale todo al
amigo de Madrid, y él verá, si quiere verlo, que Dios mira por
nosotros.»

«Ya he salido de la grandísima tarea de esta carta, y créete que no me
ha costado poco trabajo concluirla, porque el tintero de cuerno venía
muy escaso de tinta, y he tenido que bautizarla, por lo que notarás que
esta letra se parece más al agua que a la tinta. Concluyo
encargándote\ldots{} No, no: espérate un poco, que se me olvidaba una
cosa.

«Lo que te dije en mi anterior de echarle pica pica al gordo Mora, darle
un papirotazo a D. Mariano y unos buenos azotes a mi tía Cristeta, tenlo
por no dicho. No hagas nada de eso, que a todos perdono y todo
resentimiento se ha borrado de mi alma. Perdón general, perdón hasta
para mi tía Cristeta, que fue la que me hizo más daño, porque ya tenía
yo a mis padres convencidos de que no debía casarme con Ernestito,
cuando metió ella sus narices en el negocio; tales cosas dijo a papá y
mamá de las riquezas del niño y de lo feliz que iba yo a ser con tantos
millones, que les embaucó, y ellos a mí, y al fin pasó lo que sabes.
Gracias a que supe descasarme a tiempo, que si no\ldots{} En fin, no más
por hoy.

«Adiós, Pepe: a tu mujer, todos los cariños que se te ocurran; a tu
nene, besos mil, y tú recibe, con los afectos de \emph{Ley}, el de tu
amiga---\emph{Mita-Virginia.»}

En silencio hicimos, cada cual a su modo, los primeros comentarios.
Suspiraba mi mujer, limpiándose el rostro de lágrimas. Yo esperaba oír
sus opiniones antes de manifestar las mías. «Vas a saber---dijo María
Ignacia---la duda que tengo\ldots{} La carta de Virginia me ha
conmovido, me ha levantado en el corazón una pena muy grande, y luego
un\ldots{} no sé cómo llamarlo\ldots{} un tumulto de ideas\ldots{} Veré
si puedo explicarme\ldots{} No te diré yo que estos cuentos de Virginia
sean puro embuste\ldots{} pero sí sospecho que nuestra pobre amiga se
nos ha vuelto poetisa\ldots{} que posee el arte de adornar los hechos, y
de componerlos y retocarlos para que impresionen más a los que han de
leerlos. Esto que ha escrito nos ha hecho llorar. ¿Habría producido el
mismo efecto contado por ella o visto por nosotros? Ésta es mi duda. Tú
me dirás lo que piensas.

---¿Crees tú que Virginia es artista y obra literaria su carta? Algo de
arte hay siempre en todo lo que se escribe, y los hechos, aun referidos
en forma descarnada, se revisten de un extraño resplandor más o menos
vivo, según la sensibilidad de quien los refiere. En la carta de
Virginia resplandece la narradora que no carece de habilidad: adorna un
poquito. Pero bien se ve que es cierto lo que nos cuenta, y en el sello
de verdad está todo el interés y todo el encanto de lo que hemos leído.

---¿Según eso, no crees tú que esa desdichada nos haya salido poetisa, y
quiera trastornarnos la cabeza\ldots{} versificando en prosa, como quien
dice?

---No, mujer\ldots{} No hay en esta carta versificación. El olor de
poesía que nos da en la nariz sale de los hechos, y estos son tales, que
ninguna de \emph{nuestras primeras poetisas} o literatas sería capaz de
inventarlos\ldots{} Ateniéndome a la realidad, yo te pregunto: ¿qué
hacemos ahora? ¿Perseguimos a \emph{Mita} y \emph{Ley}?\ldots{} Creo que
no ha de ser difícil descubrir la guarida, poniéndose a ello con fe y
perseverancia.

---Sí\ldots{} ¿qué duda tiene? Basta de salvajismo. \emph{Mita} y su
hombre merecen mejor suerte y otros medios de vida\ldots{} Pero espérate
un poco, Pepe. ¡Vaya un torbellino que tengo en mi cabeza! Si les
descubrimos, será forzoso sacarles de su estado salvaje y de su
condición libre. Así lo manda la moral. Dejarles en esa independencia,
favorecerles con recursos que les ayuden a campar por sus respetos, será
dar una bofetada a todas las leyes divinas y humanas\ldots{} No habría
más remedio que poner a cada cual en su lugar, separarles\ldots{} No,
no: esto tampoco puede ser\ldots{} Discurre tú por mí, que yo no
puedo\ldots{} ¡Separarles a viva fuerza! Eso nunca. Sería un atentado a
la moral\ldots{} ¿a qué moral? ¿Hay por ventura dos morales?

---Yo no sé cuántas hay, ni cuál es la mejor, en el caso de que haya más
de una. Mientras esto se averigua, no atentemos a la libertad de nadie,
y dejemos a cada pájaro en su nido. ¡La ley!\ldots{} ¡la moral! Créeme a
mí, mujer: si queremos dar con la moral y la ley, busquémoslas en
nuestros corazones.

---¡Una moral por este lado, otra moral por el otro!---dijo María
Ignacia vacilante y confusa.---Y nuestros corazones en medio\ldots{}
¡Pobres corazones!, ¿acertaréis a elegir el mejor camino?\ldots{} En
este torbellino de dudas, ¿sabes lo que pienso ahora? Pienso que el
soplado hablador don Mariano José no es tan tonto como tú crees. Me
suenan en el oído las palabras del asegurador de vidas: «No tenemos
divorcio\ldots{} Estamos muy atrasados.»

\hypertarget{xi}{%
\chapter{XI}\label{xi}}

\emph{Abril de 1854}. Recibo un pliego en sobre con filetes de luto.
«¡Quién se habrá muerto!» digo al abrirlo, no sin ligero temblor, porque
me asustan las defunciones de personas conocidas\ldots{} Resultó que no
era un muerto, sino un vivo que coleaba, un papel clandestino titulado
\emph{El Murciélago}\ldots{} Leo en él furibundas diatribas contra los
\emph{polacos}. Cada párrafo es emponzoñada flecha, o un canto muy duro
disparado contra cabezas altas y medianas. Los anuncios son crueles
epigramas. «El que desee conseguir un destino, diríjase a don Fulano de
Tal. En el Ministerio de Fomento darán razón. La cantidad que se
estipule, se ha de dar anticipadamente. No se admiten corredores.»
Versos no mal construidos ponen en la picota a los Ministros, y en ella
reciben una zurribanda de azotes. A San Luis se le llama el
\emph{condesillo}; \emph{lacayos} a los Ministros; a la Corte,
\emph{centro de liviandades}. En el pie de imprenta se lee: «Editor
responsable, don José Salamanca.---Imprenta del Conde de Vilches.»

Luego vi que todos mis amigos lo habían recibido. El maldito pájaro,
metiéndose por ventanas y puertas, visitaba las moradas de próceres y
magnates. ¿Lo habrán encontrado la Reina en su tocador, y el Rey en su
reclinatorio? Ya se lo preguntaremos a Cristeta Socobio y a doña
Victoria Sarmiento, que me parece a mí que están en el ajo, y se dejan
caer del lado de la conspiración. Éstas dos naturalezas astutas,
ratoniles, criadas en los escondrijos de Palacio, olfatean ya el nuevo
queso\ldots{} No necesito decir que el periódico misterioso tiene en
Madrid un éxito colosal. Su aparición ha sido como un rocío del Cielo
para las almas resecas del odio al \emph{polaquismo}. No hay idea de lo
que a las muchedumbres regocija y entusiasma ver en letras de molde las
opiniones subversivas, que ahogadas nacen en la conversación privada. Se
ensancha el pecho viendo que el periódico dice lo que pensamos y aún
más, con nuevas gracias, y sin pedir perdón por el modo de señalar. Todo
el que posee el primer número de \emph{El Murciélago} se cree dichoso
mortal: lo enseña con precaución; hace constar que lo recibió
directamente, con sobrescrito a su nombre, prueba de lo mucho que le
estiman los incógnitos redactores; coge a los amigos por la solapa y les
conduce al reservado de un café para leerles el papelito; niégase a
transmitir a manos extrañas aquel tesoro; ofrece sacar copias,
\emph{para que corra}, y las saca con extrañas adiciones. Todo el mundo
quiere ser \emph{Murciélago}. He leído en las copias del primer número:
\emph{«La Muñoza y consorte, esa familia rapaz\ldots»} Persignémonos.

\emph{Mayo}.---Ayer puse en conocimiento de don Francisco Chico el único
dato que se ha podido adquirir acerca del gavilán que se llevó a la
paloma de Rementería. «Se sabe que tiene una hermana casada en la Villa
del Prado.» Oída esta referencia por el astuto cazador de criminales, se
rascó una oreja; después la punta de la nariz, estirando levemente los
delgados labios en una sonrisa casi imperceptible. «¿Sabe usted, don
Pepito---me dijo,---que ese dato, con parecer tan poca cosa, podría ser
el primer jalón de un camino seguro? No me pregunte qué pienso, porque
no pienso nada. Me dijo usted hace días que el tal es buen tipo; vamos,
un chicarrón guapo, despejado él\ldots{}

---Oí que es guapo; de su despejo, nada sé\ldots{} Debo advertirle
ahora, mi buen don Francisco, que no tengo interés en que los fugitivos
sean presos, ni menos que les traigan para que se cebe en ellos la
Justicia.

---Vamos, quiere usted que cacemos a la señora sola, para meterla en las
Arrepentidas.

---No, no: nada de Arrepentidas, ni de coger a la señora. Mi deseo es
tan sólo saber quién es él, y dónde está la pareja. Quiero ponerme al
habla con ese irregular matrimonio.

---Pero la Justicia\ldots{}

---De la Justicia, nada. Dejémosla en sus altares, bien guardada entre
papeles.

---¿Y la Moral, don Pepito?

---Dejémosla también, dondequiera que esté metida.

---¡Oh, si aquí tuviéramos divorcio!\ldots{} Pero estamos muy atrasados.

---Progresemos. En este asunto, el progreso consiste en dejar las cosas
como están. ¿No piensa usted lo mismo?

---Por mí\ldots{} figúrese usted\ldots{} Adelante o atrás, todo me
parece igual. Ya estoy curado de espanto\ldots{} Dos caras tengo yo: una
me sirve para mirar al pasado, otra para mirar al porvenir\ldots{} Pero
a veces, señor mío, me equivoco de cara, y cuando me pongo a mirar lo
nuevo, veo lo viejo, y viceversa\ldots{} De modo que ya no sé si
empeorando mejoramos, o si mejorando vamos a peor, a peor\ldots{} En las
revoluciones no creo; en la tradición tampoco\ldots{} He visto
progresistas del 40 besándole el anillo a Bonell y Orbe; he visto
realistas del 24 tirando del coche de Espartero\ldots{} ¿Qué más me
queda que ver? Pues eso, que descubra a un raptor de casada y a la
casada, para dejarles en la libertad de su delito\ldots{} Pero ¿a mí qué
me importa? Se hará como usted quiera, siempre que no se me adelante
alguno que le lleve a él a presidio y a ella a las Magdalenas\ldots{} Yo
le aseguro a usted que trataré de averiguar quién es él, y dónde se
esconde la parejita. Si yo tuviera personal disponible, pienso que en
ocho días saldríamos de dudas. Pero ¿usted sabe cómo estamos con esto de
\emph{El Murciélago}, y la guerra que nos está dando el pájaro maldito?
¡Qué quién lo escribe, que quién lo imprime, que quién pone los sobres,
que quién lo reparte!\ldots{} Averígüelo usted en un Madrid, que cada
día es más grande, más poblado de pillos\ldots{} y con un vecindario que
es todo de encubridores\ldots{}

---Trabajo le mando, don Francisco. Hoy, en Madrid, el que no conspira,
tapa, y el que no tapa, está en acecho de la policía, para dar el aviso:
«¡Qué vienen!\ldots{} ¡a esconder!»

---Es verdad. ¿Y en un pueblo así, quién es el guapo que descubre a un
\emph{Murciélago}?

---Si no se me enfada, don Francisco, diré que no está ya descubierto y
enjaulado porque usted no quiere. Las imprentas de Madrid no son
tantas\ldots{} El periódico tiene que pasar por multitud de manos, pues
no ha de ser un solo hombre el que lo escriba, lo componga y lo
tire\ldots{} Yo policía, le aseguro a usted que el pájaro no se me
escapaba.

---¡Ay, don Pepe, qué mal conoce usted el mundo, y este Madrid, este
pozo Airón de los delincuentes!\ldots{} No, no. A usted policía, le
pasaría como a mí: no cazaría \emph{El Murciélago}.

---Porque no querría cazarlo; porque no querría indisponerme con los
conspiradores de hoy\ldots{} a quienes seguramente tendré que obedecer
mañana.

---No, no, don Pepito, no\ldots{} Bien sé que esto está perdido\ldots{}
Pero no somos\ldots{} no somos tan adulones del que ha de venir.

Al decir esto, su cara de pillo, en la cual no se cuidaba de poner la
máscara del disimulo, contradecía sus medias palabras. «¿Cómo me hace
creer a mí don Francisco Chico---proseguí,---que no sabe dónde está el
general O'Donnell? Pues qué, ¿se puede esconder un hombre tan alto, y
estar cuatro meses oculto sin que asome un pie, una oreja\ldots{} un
codo?

---¡Vaya si puede\ldots{} en un Madrid tan grande!

---Naturalmente, usted qué ha de decirme\ldots{} Y, sin duda, soy yo
algo impertinente al hablarle de este modo.

---Eso no: diga lo que quiera\ldots{}

Y la picardía brillaba ya en su cara, eclipsando a la sagaz reserva del
oficio. «¿Cómo me hará creer el policía astuto que no sabe quién escribe
\emph{El Murciélago}?

---Como saberlo, no\ldots{} como sospechar, sí\ldots{} Todos los días me
traen soplos: no hago caso. Los escritores del pajarraco son dos. El uno
creo que no se me despinta. Me equivocaré mucho si no es ese
malagueño\ldots{} el Cánovas.

---Hombre, no.

---Déjeme acabar: el otro\ldots{} estoy en que es uno de \emph{Las
Novedades}, ese Ríos.

---¿Pues si eso sabe por qué no les mete mano?

---¡Ah! Créame usted que al Cánovas le tengo ganas, muchas ganas; pero
no puedo cogerlo.»

Se pasaba la mano suavemente por la barbilla y quijada inferior, y sus
ojos bajos afectaban un respeto hipócrita. A las expresiones de mi
asombro, contestó al fin con este concepto que me dejó helado: «Señor
marqués de Beramendi, me aseguran que el Antonio Cánovas está escondido
en su casa de usted.» El estupor no me impidió negar desde el primer
momento con una energía que sobrecogió al fiero polizonte. «Bueno,
bueno: no es para incomodarse---dijo mirando al suelo.---Si no está con
usted, estará en casa de alguno de sus hermanos, don Agustín o don
Gregorio. Para el caso es lo mismo. Negué también que Cánovas fuese
huésped oculto de mis hermanos; pero Chico, en quien la suspicacia y la
desconfianza eran una segunda naturaleza, pareció no darme entero
crédito. Levantose del sillón Luis XV, y paseándose delante de mí,
metidas las manos en los bolsillos del pantalón, me dijo: «¡Qué venga
Dios vivo a dirigir la policía, y veremos lo que hace en este Madrid,
donde todo es un juego de compadres!\ldots{} ¿Cómo hemos de cazar a los
conspiradores, si ellos saben esconderse en lugar sagrado, o en
burladeros donde no podemos entrar?\ldots{} Corremos tras de
\emph{Fulanito}: ¿y dónde está Fulanito? Pues en su palacio le tiene, a
mesa y mantel, el propio duque de Berwick. «¡Que cojan a
\emph{Menganito}; que nos le traigan vivo o muerto!» Pues nos echamos en
busca del Menganito, y descubrimos que habita con el propio don José de
Zaragoza\ldots{} ¡guarda que es podenco!\ldots{} Pues verá usted: hemos
andado locos tras de un tal Bartolomé Gracián, militar él, condenado a
muerte, indultado y luego vuelto a condenar\ldots{} la cabeza más
destornillada que echó Dios al mundo\ldots{} Al fin mis agentes le
descubren el rastro. Vamos a echarle mano. ¿En dónde creerá usted que se
guarece ese pillo? Pues entre faldas. En las habitaciones altas de
Palacio le tienen escondido dos señoras que no quiero nombrar.
Naturalmente, allí no podemos\ldots{} Y no es ése sólo el que ha hecho
su burladero en lugares, como quien dice, sagrados. Óigame usted otra:
¿a que no me acierta dónde han ido a celebrar sus aquelarres los
malditos masones, que yo desalojé de la logia de Tepa? Pues a la casa de
una tal Rosenda, frescachona ella y desahogada, que hoy es querida del
señor Toja, uno de los primeros saca-platos y mete-sillas de la Casa
Grande. Llamamos a la puerta de la Rosenda, una, dos veces, y entramos
sin encontrar a nadie. Al día siguiente vino a verme el señor Toja, y
aquí entró, andando como un lorito, y me dijo, dice: «Amigo Chico, no se
meta en vedado si no quiere tener un disgusto.» ¡Pues anda, y que se les
lleve a todos el demonio!\ldots{} Aquí, lo primero de que se cuidan los
que revuelven a España es de buscarse un buen fiador\ldots{} Detrás de
cada revolucionario hay siempre un padrino gordo. ¿Qué hemos de hacer
nosotros, tristes empleados sin libertad, atenidos a un sueldo? ¿Hemos
de ser los únicos que cumplan con su deber, cuando los de arriba no
cumplen? Crea usted que en este tabernáculo, ningún santo está en su
puesto, ni tampoco en el suyo el Santísimo\ldots{} Pues que venga el
tronicio gordo, y a vivir todo el mundo como pueda. ¡Señores
\emph{polacos}, el que tenga aldabones, que se agarre, y el que no, que
se estrelle\ldots{} porque lo que es el terremoto de la Martinica
viene\ldots{} vaya si viene!

Antes de que yo pudiese contestar a esta honda crítica del ser interno
de nuestra patria, don Francisco, parándose ante mí, me sorprendió con
esta peregrina proposición: «Ea, don Pepito, ¿quiere que hagamos un
trato? Si usted no tiene al Cánovas en su casa, de seguro sabe dónde
está\ldots{} Dígamelo\ldots{} yo no he de prenderle, ¡cuidado! Puede
estar tranquilo. Con saber el paradero me conformo\ldots{} Y a cambio de
que usted me diga dónde se esconde el malagueño, me comprometo yo, en el
término de tres días, a descubrir los prados en que viven comiendo
hierbas la hija del señor De Socobio y el pillastre que la robó.»

No me convenía el trato, pues aunque yo supiera el paradero de Cánovas
del Castillo, no había de revelárselo por nada de este mundo. Cien veces
le aseguré no tener la menor noticia del escondite de mi amigo; pero el
muy tunante no me creía: tan metida tiene en el alma la desconfianza.
Entiendo yo que constituyen su alma el escepticismo de todo lo bueno y
la credulidad de cuanto malo hay en el mundo. La profesión de Chico,
ejercida con un sentimiento parecido a la fe, no puede menos de crear
grandes profesores de maldad, que nada ven donde la maldad no existe.
Sin duda es un pájaro de mucha cuenta este don Francisco. Su vuelo
rápido y bajuno se pierde de vista, sin que nadie pueda saber a dónde
van a parar sus pensamientos. Y tanto desconfía, que jamás inspira
confianza. La proyección de su malicia sobre el espíritu del que le
escucha es tal, que, tratándole a menudo, llega uno a sentirse
delincuente.

\emph{Sigue Mayo del 54}.---¡Pataplum! Otro número de \emph{El
Murciélago}. Lo primero que me echo a la cara, al desdoblar el papel, es
esta piadosa frase: «Salamanca colgado del balcón principal de la Casa
de Correos, sería una gran lección de moralidad.» Temblemos, y sigamos
leyendo: «A Salamanca se han unido cuantos Ministros ladrones hubo en
España, y, por último, se le agrega también el duque de Riánsares para
los ruidosos negocios de ferrocarriles.» En otro párrafo se burla del
Gobernador, conde de Quinto, y de sus inútiles esfuerzos en la
persecución de la hoja clandestina; y dice con frescura: «Este
\emph{Murciélago} no podrá ser habido: está en parte más segura de lo
que parece, y entra hasta donde S. E. no podrá entrar siempre que
quiera.» Razón tiene Chico. ¡Ah, los altos burladeros\ldots! Leo esto,
que es muy interesante: «Corren por Madrid, y parece que están próximos
a imprimirse, algunos versos contra la Reina, en los que se habla hasta
de su vida privada\ldots{} Sabemos que estos versos están escritos y
serán publicados por cuenta de los \emph{polacos}, con el objeto de
hacer ver a Su Majestad que la oposición la trata de una manera
violenta.» ¡Hola, hola! Truena después contra los llamados \emph{agios}:
el regalo de 80.000 duros a Rotalde por las obras del Teatro de Oriente;
la concesión a la casa Sangróniz de un servicio de vapores para La
Habana, y el empréstito forzoso de 180 millones\ldots{} Hablando de esta
operación, \emph{El Murciélago} toca el cielo con las negras alas. «¿Y
van siquiera a emplearse con utilidad del país los 180 millones? Una
parte, no pequeña, se invertirá en esos \emph{agios}, que con el nombre
de giros, descuentos, etc., enriquecen a los que comercian con la
fortuna pública\ldots{} Después, 40 millones servirán para pagar el
camino de hierro de Langreo\ldots» Y sigue poniendo como chupa de dómine
a la que llama \emph{familia Muñoz}, hasta declarar que es una familia
que \emph{vende su honra} por dinero. Me parece que al pájaro se le va
un poco la cabeza. Entre los sueltos cortos, leo: «Dícese que el conde
de Quinto ha sido nombrado Gentilhombre. De seguro hace de la llave una
ganzúa.» Pasa luego revista a los militares que defienden al
\emph{polaquismo}, y no deja hueso sano a Blaser, Lara, San Román y
Vistahermosa. Del ejército dice que \emph{calla, avergonzado del inmundo
cuadro de desmoralización que tiene delante}\ldots{} Se atreve, por fin,
con la Reina, contra quien va esta china: «Ya empieza a rodar por la
cabeza de mucha gente la idea de un destronamiento\ldots» ¡Qué nos coja
confesados!

\hypertarget{xii}{%
\chapter{XII}\label{xii}}

\emph{Junio de 1854}.---Sorprendido fui hace pocas noches, a deshora,
por la visita de Rogelio Navascués, esposo de Valeria, al que acompañaba
otro sujeto desconocido, que por el aire me pareció militar. Ambos
vestían de paisano, con afectada traza de señoretes pobres de
provincias, de los que años ha llegaban sin más objeto que ver \emph{La
Pata de Cabra}, y hogaño vienen a ponerse en contacto con la novísima
civilización, llevándose, como señal o muestra de ella, baratijas de
corto precio adquiridas en las tiendas más a la moda. Encarar con ellos
en mi despacho, y ver sus fachas, y darme en la nariz olor de
conspiradores buscando un escondite, fue todo uno. Lo primero que hizo
Navascués fue presentarme a su compañero: «Bartolomé Gracián, Comandante
con grado de Teniente Coronel, uno de los más fervientes enamorados de
la Libertad\ldots{} etc\ldots» Luego se rieron los dos del pergenio que
traían, alabándose de su agudeza para burlar a los corchetes, y acabaron
por poner sus apreciables personas bajo mi amparo, para que yo las
guardase en el sagrario de mi domicilio. La verdad, no me inspiraban
interés ni lástima, y a ello contribuyó la cínica ligereza con que
hablaban de sus trabajos, el menosprecio de sus superiores, y la
confianza en salir victoriosos siempre que lograsen una libertad
relativa con el escondrijo y la engañosa vestimenta. Además, mi suegro,
don Feliciano, con egoísta previsión de hombre acomodado que aborrece
toda molestia, me había dicho que nuestra casa no facilitaría el tapujo
a \emph{patriotas} militares o civiles acosados por la autoridad.

De esto hablábamos, cuando entró Valeria compungida, y con temblorosa
frase y estilo de teatro imploró la hospitalidad, asegurando que sería
por poco tiempo, pues la Revolución había de triunfar, y los perseguidos
serían prontito los perseguidores. Mi mujer, que desde la pieza
inmediata oyó la voz de su amiga, no tuvo más remedio que intervenir
tomando su parte en las demostraciones de piedad que el sexo le impone,
y no necesitó más Valeria para romper en llanto y hacernos una escena
dramática, hiposa y sofocante, de ésas que en la escena nos encocoran y
en nuestra casa mucho más. Sin que se nos ocultara que en la tribulación
de la señora de Navascués había no poco de artificio, Ignacia y yo nos
rendimos al formulismo de la amistad, y los perseguidos fueron
amparados. Mas, no siendo posible tenerles en casa por el rigor cívico
de mi suegro, discurrimos darles asilo seguro en otra casa de mi
familia. Desechada la hospitalidad de mi hermano Agustín, por miedo de
comprometer gravemente su furioso ministerialismo, con Gregorio me
entendí aquella misma noche para traspasarle el embuchado; y tan bien
dispuesta encontré a mi cuñada Segismunda, que no necesité gastar saliva
para que consintiera en ser patrona de conspiradores. Obscuros y sutiles
negocios, de que hablaré en otra ocasión, han enriquecido a Gregorio en
poco tiempo. Segismunda se lanza con ambicioso vuelo a más altas
esferas; quiere brillar y meter ruido, poner en su persona relumbrones
aristocráticos recogidos en medio de la calle, o traídos del invisible
Rastro en que van a parar las efectivas grandezas; y aunque las ideas de
Gregorio y Segismunda son \emph{moderadas}, como es ley de gentes que
improvisan su posición, ambos ven con gusto que se alberguen en su casa
dos caballeros revolucionarios de la clase militar. En estos revueltos
tiempos, el conspirar ha llegado a ser de buen tono. A mi hermano, y
particularmente a mi cuñada, les halaga que, cuando triunfe la
Revolución, se les señale como generosos encubridores de los que hoy son
facciosos y mañana serán héroes. ¡Qué no darían por esconder a un
O'Donnell, a un Messina, o al avisado malagueño Cánovas del Castillo!

Todo quedó arreglado a media noche, y antes de amanecer, los paladines
de la Libertad dieron fondo en el cómodo asilo que con maternal
solicitud les preparó la esposa de mi hermano. Y ya muy entrado el día,
pidiome audiencia en mi casa un sujeto que se anunció como funcionario
de la Seguridad Pública, sin decir su nombre. Picada mi curiosidad, no
tardé en recibirle; y si de la persona no puedo decir que fuera
interesante, lo que el tal echó de su boca en la visita merece cabida
preferente en estas Memorias de mi tiempo. En el hombre vi, como rasgos
culminantes del tipo, un bigote negro cerdoso cortado en forma de
cepillo, cabellera abundante cortada como escobillón, nariz pequeña y
atomatada, bastón de cachiporra, gabán claro de largo uso, y sombrero,
que en toda la visita permaneció en la mano de su dueño. Ostentaba la
pelambre de esta prenda innumerables cicatrices, testimonios de una vida
azarosa, estrujones, apabullos, palos ganados en escaramuzas callejeras.
Quizás, en alguna reunión tumultuosa, sirvió de asiento a persona de
extremada gordura; quizás, antes de cubrir la cabeza de su actual
propietario, fue remate del figurado guardián que se arma en medio de
las huertas para espantar a los gorriones. Pero si mucho el sombrero
decía, más dijo el hombre, y sus manifestaciones encerraron tanta
enseñanza, que aquí las copio, sin más enmienda que la supresión de mis
observaciones y preguntas en el curso del diálogo. Así parece más clara
y compendiosa esta página viva de la Historia Nacional.

«Vuecencia no me conoce, señor don José. Yo soy \emph{Sebo}\ldots{}
quiero decir que así me llaman, y por \emph{Sebo} me conoce todo el
mundo en Madrid, aunque mi nombre es Telesforo del Portillo. El mote
proviene de que nací y me criaron en un taller de extracción de sebo,
calle del Peñón, donde mi padre y toda mi familia tenían la industria de
velas, que allá por el 48 vino a parar en ruina, por causa de la
introducción de la maldita esperma y otras porquerías, sacadas, según
dicen, de las ballenas de mar\ldots{} Desde que yo empecé a discurrir,
más que los oficios de mano me gustaban los de cabeza, todo lo que fuera
cosa de ilustración, o por mejor decir, de literatura. Con otros chicos
representaba comedias, y de noche, en mi casa, copiaba versos de algún
periódico para aprendérmelos de memoria. Llevado de mis aficiones, el
primer pan que gané me lo dieron en la escuela de párvulos de la calle
de Rodas, donde serví la plaza de auxiliar dos años cumplidos. Aunque me
esté mal el alabarme, yo aseguro que no me faltaban disposiciones para
desasnar criaturas. Con la paciencia que Dios me ha dado y cierto don
natural para dominar las almas infantiles, hice verdaderos milagros en
aquel desbravadero de las inteligencias. A muchos borricos domé, y más
de un idiota me debe el dejar de serlo. El maestro, mi jefe, me tenía en
grande estimación; era yo su brazo derecho, y en los últimos meses
llevaba el peso de la escuela. Pero como nadie me agradecía los
servicios que yo prestaba a la Nación, cogiendo de mi cuenta a los
españoles chicos para convertirlos de animales en ciudadanos, y como mi
estipendio era tan corto que apenas pasaba de dos reales y medio al día,
insuficiente para pan y arenque o molleja, me vi precisado a cambiar de
oficio. Por aquel tiempo empezó a salirme familia\ldots{} pues, aunque
yo no estaba casado todavía, la que hoy es mi mujer me había dado ya el
primer hijo, principio de la cáfila de nueve que ya lleva paridos, de
los cuales me viven seis para servir a Dios y a Vuecencia.

\justifying{»Para no cansarle, señor don José, después de mil contradanzas molestando
a medio Madrid en busca de colocación, el señor Beltrán de Lis me metió en este
*pandeldemonio* de la policía, que es, hablando pronto y mal, el oficio más perro
del mundo... y el más deshonrado, el más comprometido, si no se pone uno al
igual de los criminales, y come de ellos y con ellos, para ayuda del gasto de
casa... que es muy grande, señor. Los ricos no tienen idea de las fatigas de un
padre de familia con seis criaturas, mujer, hermana mayor, y otros parientes
que acuden al olor de un triste puchero. Esto no lo sabe el rico, que nos paga
míseramente para que le cuidemos su vida y hacienda, y sobre pagarnos tan mal,
tan mal, que todo mi haber, pongo el caso, no pasa de nueve reales y medio al
día, nos exige que tengamos virtudes... ¡virtudes, señor, virtudes! maniobrando
uno entre todos los vicios, y cuando en su casa no le entran a uno por los
oídos más que clamores de la mujer: ¡que si los chicos están descalzos, y ella
sin camisa, y todos con hambre por la cortedad del alimento! Yo tendré todas
las virtudes conocidas, y algunas más, el día en que me las avaloren por moneda
corriente, que de otro modo no puede ser. Si quieren virtudes baratas o de
gorra, formen un Cuerpo de Policía de anacoretas, clérigos u otra calaña de
gente sin familia ni necesidades. Suprímase la familia, seamos todos sueltos,
tengamos refectorios públicos para matar el hambre, y habrá virtudes. De otro
modo no puede haberlas y aquí estoy yo para decir con el corazón en la mano que
no soy virtuoso. Gazmoñerías hipócritas no entran en mí. Y frente a un
caballero que sabe apreciar las cosas como son, abro primero mi conciencia,
después mi boca, y alargo mi mano para que los pudientes me den el pedazo de
pan que el Gobierno, mi amo, no quiere darme por mi servicio. Yo huelo donde
guisan y allá me voy. Hablo con un caballero, y humildemente le digo: «Señor,
\textit{Sebo} se pone a sus órdenes para todo lo tocante a dejar tranquilos a esos
beneméritos Navascués y Gracián, que \dotfill}

»Gracias, señor, por su ofrecimiento de socorro, que debe hacer efectivo
a toca teja, porque en mi casa se carece de lo más preciso\ldots{} Y
paso a informarle de lo que desea saber. Anoche, cuando entraron en su
casa el Navascués y el Gracián, vestidos de paisano, di conocimiento al
señor Chico, que me ordenó suspender la vigilancia de estos sujetos.
Naturalmente, ¿qué vamos ganando nosotros con extremar las cosas?
¿Apurar la ley para que el día de mañana los perseguidos de hoy nos
limpien el comedero? Españoles somos todos, con derecho a vivir, y el
grano que para nuestro alimento nos tira la Providencia desde el cielo,
lo hemos de coger donde caiga\ldots{} Digo, señor, que si el granito cae
en campo revolucionario, allá nos tiramos a comerlo. Revolución quiere
decir: «Caballeros, apártense un poco, que ahora vamos los de acá.» En
fin, que Juanes y Pedros todos son unos\ldots{} y si el señor no se
incomoda, le diré que mis chicos andan descalzos. \dotfill

»Gracias, señor, por su nuevo ofrecimiento. ¿Quiere saber los
antecedentes criminales de esos dos \emph{peines}? Pues allá van: al
Navascués le conozco poco, pues no ha sido de mi \emph{parroquia}. Le
\emph{tenía} un compañero mío, por quien sé que se pasa la noche
comprometiendo a la oficialidad de \emph{Constitución} y de
\emph{Extremadura}. Al otro, al Gracián, sí le conozco, y más cuenta me
tendría lo contrario, porque los porrazos que de él y por él he recibido
no se pueden contar. Créame, señor: entre todos los españoles locos, el
más rematado es ese Gracián. Si el conspirar no existiese, él lo hubiera
inventado. Desde que estrenó el uniforme anduvo en líos de
pronunciamiento. Por poco pierde la pelleja en Madrid, el 48, y después
en las Peñas de San Pedro. Vive de milagro: le matan y resucita. Es
valiente; pero de esos que no pueden vivir sin faltar a la ley. A
mujeriego no hay quien le gane. Cuando no engaña a dos, a tres engaña.
Las mujeres quieren salvarle, y él no se deja. No hay en la Policía
quien no tenga en alguna parte del cuerpo señal de sus manos. Yo,
\emph{sin ir más lejos}, estuve dos semanas con la cara hinchada,
porque\ldots{} verá Vuecencia: quise cogerle una noche, a su salida de
Palacio, ¡de Palacio, señor!, que allí tenía su albergue. Me dio tan
fuerte golpe que perdí el sentido, y creí que escupía todas las muelas
de este lado. A dos compañeros míos, otra noche, junto a Caballerizas,
les descerrajó un pistoletazo, pasándole a uno el sombrero y quitándole
a otro un pedazo de oreja. Intentaron echarle mano; pero él sacó un
cuchillo de este tamaño, con perdón, y les acometió con tanto coraje,
que si no echan a correr, allí se dejan el mondongo. Asómbrese
Vuecencia: hasta hace poco vivía en los altos de Palacio; parece que es
sobrino carnal de una señora que vistió el hábito de monja en el
convento de Jesús. Don Francisco Chico, cuando le llevamos esta
referencia, nos dijo: «Cepos quedos, muchachos. Tres sitios hay donde no
debéis meteros nunca: \emph{río, rey y religión}.» \dotfill

«Razón tiene mi señor don José: dentro de Palacio hay ideas y personas
para todos los gustos\ldots{} Bien dice don Francisco Chico que el piso
segundo es una república\ldots{} Al tercero suelo subir yo, porque allí
vive un primo mío, que me debe ochenta y dos reales de unos colchones
que mi mujer vendió a la suya, y por cobrárselos a pijotadas, apechugo,
en los primeros de mes, con el sin fin de peldaños de aquellas malditas
escaleras. Por el primo sé muchas cosas, de ésas que se le dicen a uno
para que las calle, y así hago yo\ldots{} oír y callar. Los magnates se
encargan de pregonarlas: ellos, que de presente adulan, por detrás
despellejan. El pobre es el que habla siempre bien de las personas
altas, pues como está mal comido, no tiene aliento más que para honrar y
aclamar. El pobre mal comido dice a todo que sí, porque para el
\emph{sí} no necesitamos tanto aliento como para el \emph{no}\ldots{}
Por esto, yo sostengo, y no se ría don José, yo sostengo que si el
pueblo estuviera bien comido, bien bebido, y asistido en total de sus
necesidades, diría que no, viniendo a ser enteramente revolucionario. Lo
que oíamos cuando éramos niños, seguimos repitiéndolo de grandes.
\emph{¡Viva Isabel!} fue el son con que nos arrullaban en nuestras
cunas, y \emph{¡Viva Isabel!} gritamos hasta la muerte. Es un estribillo
que tiene por causa la mala alimentación. Los hambrientos cogen un decir
y no lo sueltan en toda la vida. Los señores bien cebados son los que
pueden discurrir y hacerse cargo de las cosas públicas, mientras que el
pobre sin sustancia es perezoso del cerebro, y no le entran más ideas
que las que ya entraron, o sea las que recibió como herencia al mismo
tiempo de recibir el patrimonio de su pobreza. Tomando pie de esto,
excelentísimo señor, le suplico que mire por Telesforo del Portillo,
\emph{alias Sebo}, que es buen hombre, aunque en este oficio condenado
no lo parezca; y puesto Vuecencia a proteger, eche una mano a toda la
familia. Verbigracia, el chiquillo mayor de los míos, a su padre sale en
lo agudo y a su madre en lo hacendoso. Sabe leer y escribe con buena
letra. En esta gran casa podría tener colocación, aunque sólo fuera para
llevar y traer recados. Si quiere ponerle librea, mejor, que así se
acostumbrará el niño al empaque tieso y a las posturas nobles, como
quien dice. La niña mayor, aunque me ha salido un poco jorobadita, es
muy dispuesta para todo, y un águila para la costura\ldots{} quiero
decir, que cose con primor y que sus dedos vuelan\ldots{} Bien podría la
señora Marquesa traerla acá, y tenérmela empleada de sol a sol en la
costura de casa tan grande\ldots{} De mi esposa, sólo digo que tiene
manos de ángel para el planchado en fino, y que en la compostura de
encajes da quince y raya a la más pintada\ldots{} Vea el señor Marqués
qué fácilmente puede ayudar y socorrer a este pobre \emph{Sebo}, a este
honrado \emph{Sebo}, que por las callejuelas de su oficio camina en
persecución de las virtudes sin poder encontrarlas, y\ldots»

\hypertarget{xiii}{%
\chapter{XIII}\label{xiii}}

No le dejé concluir: ya el sonsonete de su voz, que había empezado
festiva, volviéndose gradualmente cavernosa y lúgubre, retumbaba en mi
cerebro como el insufrible aleteo del moscardón. Con un socorro
pecuniario correspondí al ofrecimiento de sus servicios, y puse precio
razonable a su filosofía policíaca, forma vaga de humanismo no
disconforme con mis ideas\ldots{} Para él sería yo el verdadero
Gobierno, el Estado efectivo; y pues mi bolsa suplía las escaseces de la
nómina oficial, a mí debía darme \emph{Sebo} el fruto informativo de su
trabajo, entendiéndose que jamás haría yo uso inconveniente de tan
extraña subrogación. Retirose el hombre muy contento, y aquel mismo día
por la tarde medio pruebas del entusiasmo y diligencia con que a mí se
acogía, llevándome nuevos informes del Bartolomé Gracián, los cuales
avivaron hasta lo indecible la curiosidad que aquel extraño conspirador
había despertado en mí. Y el gran \emph{Sebo}, como si su memoria fuese
un canasto repleto de inútiles cosas cuya pesadumbre le estorbaba, vació
en mis oídos estos fragmentos de Historia privada.

Es Gracián el más atrevido y el más afortunado mujeriego de España y sus
dominios. Su extraordinaria guapeza de figura y rostro le dan las
primeras armas; las completa y afina con su increíble atrevimiento y con
su exquisita labia para trastornar el seso a las hembras. Desde que fue
hombre, declaró guerra sin cuartel a dos principios fundamentales: la
moral doméstica y la disciplina militar. Es éste en él un doble apetito,
una doble pasión que le coge toda el alma, sin dejar espacio para más.
No hay situación política que él no intente destruir con las armas, ni
mujer bonita, casada, soltera o viuda, que no quiera conquistar por el
amor. Aseguró \emph{Sebo} que todas absolutamente se le rinden, y
extendiéndose en este sabroso asunto, alargaba el hocico, erizando las
cerdas de su bigote, y entornaba con picardía sus ojos vivarachos.

---Cuidado, \emph{Sebo}---le dije,---que eso del rendimiento de todas no
puede constarle a usted ni a nadie.» Y él, afectando imparcialidad y
moderación: «Pongamos casi todas, señor don José. Hay en la vida de este
hombre un caso que sé por referencia, pues en aquel año, que era el 51,
no estaba yo encargado de él, ni le había echado la vista encima. Andaba
escondido; sobre él pesaba sentencia de muerte; vivía con una moza muy
guapa, que le quería y le cuidaba como si fuera su esposo por la
Iglesia, y de la noche a la mañana\ldots{} No, no es que la abandonara,
señor; no es eso\ldots{} El abandono de la moza bella no tendría nada de
particular. Fue un caso más raro y nunca visto. Otra mujer, a quien él
no conocía más que de oídas, le robó\ldots{} robó al Capitán,
llevándosele a sitios escondidos para tenerle por suyo\ldots{} No se
ría, señor: es como se lo cuento. Se ven raptos de mujeres en el mundo
muchas veces; en el teatro a cada momento. Pero raptos de hombres, así,
cogiéndole en un camino, por manos de policías y cargando con él, como
se carga a una doncella, no se ha visto, creo yo, más que en el caso del
capitán Gracián.

---¿Y la otra, amigo \emph{Sebo}, la que vivía con él?

---Buscándole estará todavía, pienso yo\ldots{}

---¿Y él?\ldots{}

---Un compañero mío que estuvo en el ajo me ha contado que el muy
tunante hizo poca o ninguna resistencia\ldots{} Vamos, que se dejó
robar. O algún antecedente tenía ya de esa señora ladrona de hombres
guapos, o por ser tan listo, le dio en la nariz olor de una barraganía
de provecho.

---¡Vaya, que la mujer que tal hizo era de veras arriscada y jaquetona!
¿Y anduvieron policías en ese gatuperio? En él veo yo la mano experta de
don Francisco Chico.

---El señor coge las cosas al vuelo, y así no tengo yo que decir nada
malo de mi jefe.

---Es un travieso de marca mayor, y sin ningún escrúpulo. Pero la dama
ladrona, sólo por idear el robo de un hombre y afrontar las
consecuencias, tiene sus puntas de heroína. ¿Quién es, \emph{Sebo}?

---Pues una solterona que por lunática fue arrojada del convento de
Jesús; mujer de cuidado, que, según dicen, es un poco boticaria y un
poco droguista. Compone aguas de olor para refrescar el cutis o para
pintar el pelo, y bebedizos para echar del cuerpo los demonios, o
meterlos en él si a mano viene. Entiendo yo que es bruja, señor, y de
aquellas que, por criarse en conventos, saben más que Merlín y su hijo.
¿Me pregunta Vuecencia que qué cargo tiene en Palacio? Pues el de
centinela, para ver quién entra y sale, quien priva y quién no priva, y
contarle todo a la \emph{Madre}. La \emph{Madre de las Llagas} es su
verdadera Reina y el amo a quien sirve\ldots{} Vea el señor si tendrá
poder la \emph{Boticaria}, maniobrando entre dos Reinas, la de acá y la
de Jesús\ldots{}

Al llegar a este punto, se apareció en mi mente la figura que yo creí
desconocida, y de pronto resultaba persona de mi particular
conocimiento. ¡Acabáramos\ldots! En la compleja vida social, sucede a
menudo que oímos referir peregrinos hechos, y dudando de su certeza, los
atribuimos a individuos fantásticos. Nuestro asombro se asemeja al
infantil interés que despiertan los maravillosos cuentos de hadas. Mas
por cualquier circunstancia descubrimos que el héroe, o la heroína, de
tan singular historia es un ser vivo, le conocemos, le tratamos, y
entonces nuestro asombro se concreta, se refuerza con la credulidad, y
es clara percepción de la realidad humana. Lo que juzgábamos absurdo,
sólo por ser impersonal, nos parece verídico en cuanto el caso se
cristaliza en el rostro de una persona conocida.

«Señor \emph{Sebo}---dije yo interrumpiéndole, no sin alegría,---ya
estoy al cabo de la calle. La que usted llama \emph{Boticaria}, no puede
ser otra que doña Domiciana Paredes, hija de un cerero de la calle de
Toledo, y amiga íntima de la que lo es de mi familia, doña Victorina
Sarmiento. Mi mujer y yo la tratamos, la hemos recibido en nuestra casa,
la hemos obsequiado con chocolate, no una sola tarde, sino dos y tres, y
por venir en compañía de doña Victorina la hemos mirado como a persona
de respeto. Ya sabía yo que en Palacio era \emph{embajadora} o
\emph{cónsula} de la \emph{Madre}. La gracia y amenidad de su
conversación nos revelan a una mujer de talento. Ahora, oído el informe
del rapto de Gracián, vemos en ella un entendimiento y un brío de
voluntad que la igualan al mismo Napoleón.

---Eso he dicho yo, señor don José\ldots{} doña Domiciana es un
Napoleón\ldots{} no éste que ahora gobierna en Francia, sino el otro, el
que llamaban Primero y acabó sus días en una ínsula. Porque ha de saber
Vuecencia que la \emph{Boticaria} engañó a doña Victorina, haciéndole
creer que el Capitán robado era su sobrino, y demostrándoselo con
documentos que sacó no sabemos de dónde. Sabe imitar el papel de barbas
antiguo, la tinta vieja, y las obleas y lacres del siglo
anterior\ldots{} Pues arrebatado el galán, le escondió en una casa de
campo, radicante en el camino real de Aragón, un poco más acá del sitio
donde arranca el senderillo de la Fuente del Berro\ldots{} A poco del
secuestro, se ocupó la ladrona de conseguir el indulto de su sobrino,
pues ya he dicho que estaba dos veces condenado a muerte por Consejo de
guerra\ldots{} Ello no fue difícil, con las buenas agarraderas de doña
Victorina\ldots{}

---En las gestiones para ese indulto anduve yo también, amigo
\emph{Sebo}.

---Anduvieron muchos. O'Donnell se resistía. Una cartita de Su Majestad
amansó los rigores del General. Indultado el Capitán, sus primeras
salidas fueron para las conquistas de amor. En tres días hizo estragos
en las afueras de la Puerta de Alcalá, cortejando y enloqueciendo a
varias mujeres: una en el Parador de Muñoz, otra en la huerta de Retena,
dos en las casas de Piernas y en el portazgo del Espíritu Santo\ldots{}
Doña Domiciana tocaba con sus manos el santo cielo. Para evitar
escándalos, discurrió mandar al galán a Puerto Rico con el general Prim.
Ya estaba todo concertado para este viaje; pero la noche en que
Bartolomé debía marchar a Cádiz en silla de postas, desapareció como un
duende, sin que por ninguna parte le pudiéramos encontrar\ldots{} Por
fin, a los tres días se delató él mismo, en casa del señor Toja\ldots{}
calle del Factor. ¿Y cómo se descubrió ese pillo? Pues rompiéndole la
cabeza a un amigo suyo, Nicasio Pulpis\ldots{} Allí se reunían\ldots{}
dicen que a jugar al tresillo: yo entiendo que a verlas venir. Para mí,
Gracián cameló a la Rosenda, que antes fue querida de un tal Castillejo,
también de la cáscara amarga, y conspirador de afición, como esos que
matan \emph{por dar gusto al dedo}\ldots{} Las heridas del Pulpis
desbarataron el tapujo del Gracián. Pero el señor Toja, más ciego que un
topo, seguía defendiéndole, y entre él y la Rosenda nos le quitaron de
las manos. Volvió a parar el hombre a las de la \emph{Boticaria}, que
esta vez en el mismo Real Palacio le aposentó, sin que doña Victorina se
llamase a engaño: con tanta sutileza tramó el enredo aquella sabia
\emph{Napoleona}, o Napoleón de las mujeres\ldots{}

«¿Qué más quiere Vuecencia que le cuente, ilustre señor? Si no se cansa,
le diré que por Gracián enloquecieron y se trastornaron una tal
Eduvigis, esposa de cierto carrerista, y la hija mayor de un
gentilhombre, llamada Inés\ldots{} Fue tal el desatino de ésta, que se
propinó una toma de fósforos en aguardiente, porque el galán había
faltado a una cita que le dio en la iglesia de la Encarnación\ldots{}
Pasaron días, muchos días, en los que perdí el hilo de estas cosas.
Gracián desapareció de Palacio. No hace dos meses que doña Domiciana
volvió un día a su casa con el cuerpo lleno de contusiones, un ojo
hinchado y el morro enteramente torcido. ¿Qué le había pasado?\ldots{}
Por decir extravagancias, hasta se dijo que en el convento de Jesús
anduvieron las monjas a trastazo limpio con una caterva de diablos
rabudos y cornudos que entraron por las chimeneas. Y tan estúpida es la
gente, y tan desleída en la sangre tenemos los españoles la superstición
maldita, que no faltaron personas que creyeran estos disparates\ldots{}
Otras vieron en el rostro de la \emph{Boticaria} la mano del Gracián.
Pero ella se aguantó, y con sus alquimias se puso a la curación de las
mataduras, quedándose en un santiamén como nueva. Digo que es bruja,
señor, porque como santa no lo es\ldots{} Gracián vivió algún tiempo en
Caballerizas, durmiendo allí de día, y largándose por las noches a la
vida de tertulias secretas, en la redacción de algún periódico, o en
zahúrdas donde conspiran hasta los gatos\ldots{} En estas idas y
venidas, ya teníamos armada una trampa para cogerle, juntamente con el
señor de Navascués, cuando se guarecieron los dos en esta casa\ldots{}
Yo pregunto: ¿por seducir mujeres se debe perseguir a un hombre? ¿Y por
pronunciarse, qué? ¿De estas dichosas \emph{pronunciaciones} no han
salido todos los Generales que nos mandan? Pues los pronunciados de ayer
ya llenaron bien el buche, dejen comer a otros, señor. Yo digo que debe
haber turno en las mesas de los ricos,

o que alternen, para que podamos alternar también los pobres. Bien nos
dice la experiencia que cuando los Gobiernos duran mucho, todo el
tráfico se paraliza, la clase menestrala no tiene qué comer, aumentan
los robos, las patronas y pupileras están a la cuarta pregunta, la
mendicidad crece, disminuye la caridad pública, el abasto de la plaza es
malo y carísimo, la carretería se estanca, los taberneros echan más agua
al vino, el pueblo se entristece, bajan las rentas de Tabacos y de
Loterías, nacen más chiquillos, las calles se desaniman, los sastres
perecen, y toda la Nación está como una novia desconsolada, a quien
nadie le dice \emph{por ahí te pudras}.

---Muy bien, muy bien, amigo \emph{Sebo}. Estamos conformes. Los
Gobiernos duraderos originan enormes calamidades. ¿No condenamos la
pereza en las personas? Pues peor es en los pueblos. Progresar quiere
decir moverse, renovarse, mudar de estado, de postura, de ideales, de
ensueños, de vestido, de modas. Hasta los enfermos crónicos y aprensivos
abominan del reposo, cambian de enfermedades, y cada día inventan una
nueva. No basta variar de médico; hay que variar de dolores. «Ya no me
duele aquí, sino aquí.» Progresar es cambiar de amigos, de novias, de
afeites, de juegos, de aires. España es un mendigo que se aburre de
estar siempre pidiendo en la misma esquina. «Vámonos a la de enfrente,
que por ésta no pasa nadie.» España no necesita de la acción
consolidadora del tiempo, porque no tiene nada que consolidar; necesita
de la acción destructora, porque sus grandes necesidades son
destructivas. Las revoluciones, que en otras partes desequilibran la
existencia, aquí la entonan. ¿Por qué? Porque nuestra existencia es en
cierto modo transitoria, algo que no puede definirse bien. Yo la veo
como si el ser nacional estuviera muriendo y naciendo al mismo tiempo.
Ni acaba de morirse ni acaba de nacer. Por eso apetece el movimiento, la
variación de ambiente, de personal, el cambio de hombres públicos, a ver
si éstos son menos sepultureros y más comadrones. ¿Me entiende usted,
amigo \emph{Sebo}?

---La verdad, señor: no lo entiendo muy bien.

---Digo que el ser nacional está en todo este siglo muriendo y naciendo.
Los hombres públicos y cuantos de política se ocupan, incluso los
militares, sepultan y al propio tiempo vivifican\ldots{} La nación
quiere mudanzas y revoluciones, para que el nacer sea fijo y se acabe el
morir.

---Ahora lo entiendo menos, excelentísimo señor.

---Pues sí: húndanse los gobiernos, vengan revoluciones, para que el
país se despabile y aprenda a vivir a la moderna, y salgan hombres de
gran poder, y tengamos más medios de ganar la vida, y se acabe el morir
lento de un pueblo.

---Ahora entiendo, porque\ldots{} como dijo el otro: los pueblos no
mueren.

---Se modifican, se refunden. España no ha encontrado el molde nuevo.
Para dar con él tiene que pasar todavía por difíciles probaturas, y
sufrir mil quebrantos que la harán renegar de sí misma y de los
demás\ldots{} Pero si el señor \emph{Sebo} no tiene interés en que
lleguemos a desentrañar este punto hondísimo de histórica filosofía,
pasemos el rato en el examen de los hechos, alegres o tristes, patéticos
o graciosos, más interesantes que esas peregrinas imitaciones de la
realidad que llamamos novelas. Celebramos ver ensanchado el campo de la
verosimilitud. Nada es mentira, amigo \emph{Sebo}: la verdad se viste
con los arreos de lo fabuloso para cautivarnos más, y cuando ve que la
contemplamos embobados, suelta la risa, se quita el disfraz y nos dice:
«Mentecatos, no soy arte: soy\ldots{} yo.»

---Quiere decir que estas cosas parecen cosas de poetas, y aquí el poeta
no es otro que el mundo de Dios\ldots{} Aún no he contado al señor todos
los enredos del Gracián, ni los apuros de doña Domiciana para tenerle
sujeto.

---No es conveniente, buen \emph{Sebo}, que ahora prosiga usted
relatando estas verdades que parecen mentirosas\ldots{} Con lo referido
basta por hoy, que la acumulación de pasajes interesantes en cualquier
historia produce en el oyente un efecto congestivo\ldots{} Economicemos
el asombro, y sujetemos a medida el examen recreativo de los hechos
humanos\ldots{} Necesito comunicar a mi mujer estos descubrimientos, y
que ella y yo, en íntima conversación, nos solacemos comentándolos. Los
ricos que no tienen nada que hacer, se morirían de tedio si no alegraran
su vida, en que todo está hecho, pasando revista a la vida de los
demás\ldots{} Vea usted en qué consiste la única felicidad de los ricos,
precisamente por ricos, ociosos: son felices mirando y midiendo la
infelicidad ajena\ldots{} Mi mujer ha salido a visitas. Ya se me hacen
largos los minutos que dure su ausencia más del término regular\ldots{}
Paréceme que siento abrir la puerta\ldots{} Es ella\ldots{} Haga el
favor de ver\ldots{}

---Es la señora Marquesa\ldots{}

---Retírese, amigo \emph{Sebo}, y venga pronto, que tengo que
encomendarle un asunto\ldots{}

---¿De mujeres, excelentísimo señor?---dijo Telesforo en voz baja,
dulzona, después de aguardar a que los pasos de María Ignacia se
perdieran en el pasillo.

---De mujer\ldots{} pero no sola\ldots{} que donde hay mujer hay hombre.

\hypertarget{xiv}{%
\chapter{XIV}\label{xiv}}

Estupor, miedo, risa causó en María Ignacia la revelación de las
inauditas aventuras donjuanescas de nuestra venerable amiga Domiciana.
¡Cuán verdadero es que en visita toda persona nos parece juiciosa y de
intachable moral! Conocíamos a la monja \emph{Boticaria} por haberla
recibido en nuestra casa más de una tarde, en compañía de Victorina
Sarmiento, antigua relación de los Emparanes. Nos había cautivado la
conversación amena, el delicado gracejo de la buena señora, y sus
felices ocurrencias expresadas con la dicción más pura, dentro de los
términos más decorosos. Encantados la oíamos todos los de casa, y
ausente, consagrábamos a su recuerdo alabanzas salidas del corazón.
Referíanos episodios del claustro, ridiculeces ingenuas de algunas
monjas, poniendo en sus relatos toda la sal compatible con la piedad y
el respeto de la religión; y si nos hablaba de tipos y escenas
palaciegas, hacíalo con exquisito comedimiento, sin que el menor roce de
su lengua, siempre muy pulcra, empañara nombres ni manchara
reputaciones, aun las más equívocas. ¡Quién nos había de decir que
aquella simpática jamona, todavía fresca, más graciosa de palabra que de
hocico, divierte sus ocios de exclaustrada en cacerías y robos de
hombres guapos! ¡Qué cosas se ven, y cuán caprichosas, en su inmenso
reino, son la flora y la fauna del vivir humano! ¡Qué infinita variedad
de formas, qué extravagancia en algunas, qué sencillez elemental en
otras! Llamamos original a lo que vemos por primera vez, y nuevo a lo
viejo que no conocíamos. Todos los casos morales tienen la misma edad,
como los tipos vegetales. La Naturaleza lo inventó todo de una vez, y ya
no inventa; no hace más que combinar las ocasiones y escenarios en que
nos descubre sus secretos: así llamamos a lo que, después de visto por
millones de ojos en cien generaciones, pasa ante los ojos
nuestros\ldots{}

Tan pronto risueña como asustada, mi cara esposa expresaba de este modo
sus turbadas ideas: «Me da miedo el descubrimiento de acciones tan
contrarias a lo que hemos visto y creído. Con tales sorpresas acabaremos
por dudar de todo. ¡Vaya con doña Domiciana! Necesito que pase algún
tiempo para formar un juicio claro de esa mujer. Lo que es ahora, no
puedo decirte con verdad si se empequeñece o se engrandece a mis ojos.
Es que\ldots{} veamos si acierto\ldots{} es que las debilidades achican,
y los grandes actos de voluntad agigantan. Domiciana es débil, Domiciana
es fuerte. ¿Con cuál de las dos mujeres nos quedamos?

---Yo me quedaría siempre con la fuerte. En Napoleón Bonaparte, la
acción enérgica eclipsa todo lo demás: los errores, las vanidades, las
infamias menudas\ldots{}

---¿Y qué me dices del don Juan ése? ¡Valiente sinvergüenza! Pero ¿el
tipo del seductor de oficio existe todavía? ¿No me dijiste que había
pasado, como los poetas pastoriles y los bandidos generosos?

---Pasan, sí\ldots{} pero vuelven.

---No sé qué me repugna más: si el hombre degradado que hace del amor
criminal una profesión, o las bribonas que se dejan burlar por un
tunante de esa ralea. La que cae en semejante trampa es tonta, o está
moralmente perdida antes de que la pierdan\ldots{} Ahora, Pepe mío, ya
lo estoy viendo, la primera idiota a quien pondrá las paralelas ese
canalla, será su patrona, tu cuñada Segismunda.

---Esa improvisada señorona no es vulnerable en su imaginación, sino en
su vanidad y en su interés. No siente otra poesía que la de los
diamantes, joyas y objetos de valor\ldots{} Gracián, según entiendo, es
pobre, y su arsenal se compone de palabras y artificios amorosos. Más
que por Segismunda, que no tiene un pelo de tonta, temo yo por Valeria,
ligerita de cascos y sin consistencia en nada\ldots{} digo, es
consistente en la pasión por los muebles bonitos y los trajes elegantes.

---¿Qué diferencia ves entre las dos hermanas?

---La diferencia que hay entre una muñeca y una mujer. Valeria es un
juguete; Virginia, una fuerza.

---Ese burlador de profesión, con ser ridículo, y sus víctimas unas
pobres ilusas, me causa miedo. ¿Y has dicho que conspira contra la
Autoridad, contra todos los Gobiernos? En buenas manos está la salvación
de la Patria\ldots{} No le deseo la muerte; pero sí una encerrona larga
en cualquier castillo donde no entren mujeres\ldots{}

---Es un soñador, que no se conforma con la realidad, y busca siempre lo
que está detrás de lo visible\ldots{}

---Y detrás de lo visible, ¿qué se encuentra?\ldots{}

---Se encuentra\ldots{} lo que se busca\ldots{} una imagen que al
encarar con ella nos dice: «No soy lo que quieres\ldots{} Lo que quieres
viene detrás\ldots» Y así sucesivamente hasta lo infinito\ldots{}

---Pues el que persiste en buscar lo que no encuentra, o es un loco, o
necio de solemnidad.

---Es un descontento, un ambicioso, un investigador de almas. Puedes
creerlo: el tal Gracián me interesa y deseo tratarle\ldots{}

---¡Ay, Pepe, Pepito, ya te me estás echando a perder!\ldots{}
¿Volveremos a las andadas\ldots{} a la persecución de lo invisible?

\emph{Sigue Junio}.---Tan a pechos ha tomado sus obligaciones de soplón
el diligente \emph{Sebo}, que ya me fatiga. Los cuentos que un día y
otro me trae, y que enredándose como las cerezas salen de su boca de
caníbal, enriquecen mi conocimiento con preciosos datos de la vida real,
los cuales vienen a mí mezclados con salpicaduras poco gratas de la
saliva del comunicante\ldots{} Creyera yo que sus bigotes cerdosos son
un hisopo automático, que rocía bendiciones mientras su palabra les da
realidad fonética. En fin, el hombre me dice tantas cosas, que ya no
tiene mi memoria cabida para conservarlas: unas se me olvidan; a otras
no doy crédito; las hay que me causan enojo porque aclaran demasiado los
senos recónditos de la vida, y destruyen el sabroso misterio.

Mi mujer ha tomado entre ojos al policía revelador: cree que sus
continuas, minuciosas confidencias alteran mi ecuanimidad; ha llegado a
ver en él como un diablo que viene a posesionarse de mí trayendo la
forma propia de nuestra época, no ya cuernos, rabo y escamas, sino el
cortado bigote de rígidas hebras, como un cepillo, y el bastón de agente
de la Seguridad Pública. Debo, según mi mujer, ponerle bonitamente en la
calle. No soy de esta opinión, porque entre infinitas referencias
menudas, que son como los dichos del vulgo recogidos a espuertas en
medio del arroyo, me ha traído algunas de un valor inapreciable.

«Por este diablo de \emph{Sebo}---dije a María Ignacia,---sé que
O'Donnell está escondido en la Travesía de la Ballesta, número 3.
\emph{Tres} jóvenes que yo conozco, Vega Armijo, Cánovas y Fernández de
los Ríos, le ponen en comunicación con \emph{tres} Generales, que
aparentan servir al Gobierno. ¿Quiénes son? Aún no me lo ha dicho mi
diablo. Lo que sí sabe es que el Regimiento de \emph{Extremadura} y el
segundo Batallón de \emph{Constitución} han picado el anzuelo. Tendremos
guerra en las calles. Ya puedes ir haciendo provisiones para la
encerrona que nos espera\ldots{} ¡Ah!, también está en el ajo nuestro
amigo Echagüe, que manda el \emph{Príncipe}. Cuidado, no te olvides de
almacenar vituallas, como para un largo asedio\ldots{} Pues sí: el
general O'Donnell, hoy perseguido, mañana triunfante, se aloja en un
piso segundo: escalera empinada\ldots{} portal obscuro y mingitorio. En
tan vulgar mansión reside la cabeza de la España política y militar de
mañana. ¿No lo crees?

---Cuéntaselo a papá. Según él, lo que se dice de Mesina y de Dulce es
invención de desocupados. Ni esos caballeros, ni otros que andan en
bocas de la gente, han pensado en volverse contra el Gobierno. El
Ministro de la Guerra, señor Blaser, llamó a Dulce y le metió los dedos
en la boca. Pero Dulce, poniéndose la mano en el pecho, juró que él es
leal, y tal y qué sé yo\ldots{}

---Haz provisiones, mujer mía, y tu papá, que es un inocente, lo
agradecerá mucho. Madrid será, el día menos pensado, mañana quizás, una
plaza sitiada. Se me ocurre que debemos comprar dos buenas cabras, y
habilitar para establo una de las habitaciones más ventiladas\ldots{}
Figúrate que la tremolina dura tres días, cuatro\ldots{} ¿De dónde
sacaremos la leche?\ldots{} Asegurada la subsistencia para toda la
familia, nada me importa que Madrid sea un campo de batalla: vengan
tiros, lucha, sacudimientos; sean allanadas las moradas de los
soberbios; las viejas rutinas caigan; ábrase paso la vida nueva\ldots{}

---¡Jesús, Jesús, ya te veo trastornado!\ldots{} Pero ¿no deliras? ¿Será
menester que compremos las cabritas?

---Sí\ldots{} Para que den leche a nuestro hijo prisionero\ldots{} Al
propio tiempo le proporcionamos un juguete vivo.

\emph{14 de Junio}.---«¿Sabes, mujer mía, lo que ocurre? Encerrémonos
aquí, y hablemos. No se entere nadie de lo que voy a contarte, que es
reservadísimo. Llevaría el cuento a don Félix Jacinto Domenech, y éste a
San Luis, y San Luis a Blaser\ldots{} No, no: esto queda entre nosotros.
¡Y luego pensarás mal del pobre \emph{Sebo}, que es para mí el susurro
de la Historia: hoy habla bajito, y mañana lo dirá todo a
gritos!\ldots{} Pues ayer, ayer estalló la revolución\ldots{} No,
digo\ldots{} quiso estallar; pero tuvo que dejar el estallido para mejor
ocasión. Verás: Cánovas\ldots{} No, no fue Cánovas; déjame que ponga
orden en mis recuerdos\ldots{} Vega Armijo, tan joven y ya
revolucionario, sacó de su escondrijo al general don Leopoldo
O'Donnell\ldots{} Eran las cinco de la mañana cuando el coche arrancó de
la Travesía de la Ballesta\ldots{} ¿A dónde iban? A Canillejas, mujer,
un pueblo agreste, más allá de las Ventas del Espíritu Santo. En esto,
Cánovas\ldots{} No, no: Dulce\ldots{} Debí empezar por decirte que Dulce
sacó muy temprano la Caballería\ldots{} con el fin laudable de
maniobrar\ldots{} y que Echagüe maniobraba también en las afueras de la
Puerta de Alcalá\ldots{} Todos maniobraban, y maniobrando se les fue la
mañana, mientras esperaba O'Donnell en un mesón de Canillejas\ldots{} el
caballo a la puerta, ensillado con montura de teniente general. Todo el
que pasaba por allí pudo verlo; lo vio la Policía\ldots{} Esperó
Leopoldo más tiempo del regular. ¿Y Dulce qué hacía? Y los Regimientos
de \emph{Extremadura} y \emph{Constitución}, ¿dónde estaban? En el
Cuartel de San Francisco\ldots{} Habían prometido salir\ldots{} pero no
se determinaron\ldots{} Querida mujer, ya no necesitas traer provisiones
ni comprar las cabritas. Todo ha fracasado\ldots{} A la fuerza expansiva
de las ideas ha vencido una fuerza mayor, la inercia, la formidable
pesadumbre de las almas que no vuelan ni corren\ldots{} ¿Y O'Donnell?
Pues mohíno volvió de Canillejas a su lugar, o sea la mísera casa de la
Travesía. Me le figuro arrastrando por el suelo su mirada, el largo
cuerpo en curva, Quijote irlandés, lúgubre y desaborido, sin la cómica
elegancia del manchego.»

A mis noticias contestó María Ignacia felicitándose del fracaso del
movimiento. Mala es, según ella, la \emph{polaquería}; pero los
conjurados no traen otro fin que quitar de las manos \emph{polacas} el
ronzal con que sujetan a esta pobre bestia de la Nación\ldots{} El
ronzal cambiará de mano; pero en éstas o las otras manos, continuarán
las mismas mataduras en el pescuezo nacional. No lo dijo así mi mujer.
Expresó la idea, que yo adorno a mi gusto al vaciarla en estas Memorias.
«Ven acá, esposa mía---le dije, movido del prurito español de las
discusiones,---ven acá: ¿qué hablas ahí de ronzales? ¿No hallas
diferencia entre la \emph{polaquería}, que es la política mohosa,
rutinaria, y esta revolución juvenil, que trae espíritu y modos
nuevos?\ldots{} Fíjate en lo que llamamos pléyade\ldots{} en esta trinca
de muchachos leídos, como que todos ellos saben francés, y nos sacan a
cada momento ejemplos mil de los pueblos extranjeros. Conoces a Ríos
Rosas, le has visto y hablado en casa de Bravo Murillo. Es aquel
rondeño, de áspera fisonomía, de palabra premiosa que revela su
austeridad\ldots{} Conoces a Cánovas, el chico de Málaga que discurre
con juicio sereno, y sabe esclarecer las cuestiones que nos parecen más
obscuras\ldots{} Conoces a Gabriel Tassara, poeta y orador, que viene a
ser lo mismo\ldots{} Los tres hacen versitos, o los hicieron cuando iban
a la escuela. La poesía es el germen de la Sabiduría política. De
Cánovas y de Ríos Rosas espero yo que sean humanas alquitaras: sus
privilegiadas cabezas destilarán la sensibilidad andaluza, para obtener
el puro espíritu lógico\ldots{} La lógica no es más que el zumo y la
esencia de la poesía\ldots{} No te rías, mujer\ldots{} Pues en esta
brillante cáfila de jóvenes salidos del estado llano, ponme también a
Fernández de los Ríos, Ortiz de Pinedo, Nicolás Rivero, Martos\ldots{} A
todos les conozco. Abogados son los más, y están bien enterados de cómo
se hacen y se deshacen las leyes. Veo que te ríes de mí, y no
sigo\ldots{} Si he de decirte la verdad, yo tampoco tengo en estas cosas
más que una fe relativa. Los pueblos desgraciados se enamoran de lo
nuevo\ldots{} Y si en esos seres desgraciados están en mayoría los
hambrientos, el entusiasmo por las revoluciones es delirio. Analizando y
desmenuzando la llamada opinión, encontramos este voto atomístico:
«Comemos poco y mal; queremos comer más y mejor.» Por esto los ricos
bien comidos no labramos más que una opinión artificial, de resonancia
hueca. La verdadera opinión, el verdadero \emph{sentimiento público}, es
el hambre.

---Divagas, Pepe, y lo que temo es que recaigas en los trastornos que
llamabas \emph{efusiones}, y que tanto nos dieron que sentir\ldots{}

---La Sociedad divaga, yo no\ldots{} Yo estoy quieto en mi casa, y ella
es la que da vueltas en derredor mío\ldots{} Yo estoy harto y quieto,
viendo venir la siniestra procesión de los estómagos vacíos, viendo
pasar las revoluciones.

\hypertarget{xv}{%
\chapter{XV}\label{xv}}

Con tanta prevención me habla mi mujer del oficioso \emph{Sebo}, y tal
ojeriza le manifiesta, que acabo yo por asustarme de las corrupciones
que el tal me cuenta. Dudo de la veracidad del soplón, y siento que va
infiltrando en mí dosis graduales de espíritu maligno. Y anoche, cuando
se retiraba después de contarme nuevas atrocidades amorosas y políticas
del Gracián, me faltó poco para padecer la más vulgar alucinación de las
comedias de magia. \emph{Sebo} desaparecía por la ventana, o al través
de la pared, despidiendo chispas de su bigote cerdoso\ldots{} Tras sí
dejaba olorcillo de azufre.

Volvió al siguiente día con una carta, que me entregó diciendo estas
palabras infernales, acompañadas de un destello sulfúreo: «Por este
papel, excelentísimo señor, el general O'Donnell le llama a Vuecencia
para conferenciar sobre el movimiento que se prepara.

---¿Qué me cuenta el amigo \emph{Sebo}? ¿Le ha dado a usted la carta el
General, en propia mano?

---No, señor. Entraba yo en el portal al mismo tiempo que el portador de
la carta, el cual es un chico aprendiz de hojalatero. Y como yo conozco
a ese rapaz, y sé que ha llevado papelitos al general Messina, a don
Antonio Cánovas y a otros, deduzco que la carta es de O'Donnell, a quien
ya llaman por ahí el \emph{General libertador}\ldots{} Puso el aprendiz
la carta en manos del portero, y de las manos del portero la cogí yo
para subirla al señor don José.» Mientras esto decía \emph{Sebo},
reconocí en el sobre la escritura de Virginia. Mi estupor fue grande,
más que por la letra del sobrescrito, por la relación que el endemoniado
cuentero estableció entre la pobre \emph{Mita} y el buen don Leopoldo.
¿Qué podía ser esto?\ldots{} A mis dudas y confusión acudió el policía
con nuevas referencias que no esclarecían el asunto: «Ya dije al señor
don José que el número 3 de la Travesía de la Ballesta es por la calle
del Desengaño el número 22. Son dos casas que se comunican por dentro.
En la del Desengaño, tienda, tiene su habitación y taller un maestro
hojalatero llamado José María Albear.

---Basta, \emph{Sebo}. ¿Y usted me asegura que esta carta la trajo el
aprendiz de la hojalatería?

---Entre su mano y mi mano, señor Marqués, sólo estuvo un momento la del
portero\ldots{} Puedo averiguar el nombre del chico; puedo cogerle,
traerle acá\ldots{}

---Dejemos por ahora ese dato\ldots{} De la carta, puedo decir antes de
leerla que tiene miga, ¡vaya si la tiene!\ldots{}

---Para mí, señor Marqués, le ofrecen a Vuecencia una cartera en el
primer Gabinete que formen los revolucionarios.

---Puede que eso sea\ldots---dije yo abriendo la carta y pasando
rápidamente la vista por ella.---En efecto, algo de eso es\ldots{}
Retírese por hoy el buen \emph{Sebo}, que esto es para leído y tratado
despacito por mi mujer y yo. Vuélvase mañana\ldots»

De mala gana se fue el diablo, y yo corrí en busca de María Ignacia, que
estaba lavando al pequeño: «Mira, mira: carta de \emph{Mita}\ldots{} ¿Y
no sabes quién la ha traído? Asómbrate, la ha traído \emph{Sebo}. ¡Para
que hables mal del pobre \emph{Sebo}!\ldots{} Ya tenemos a \emph{Mita} y
Ley bajo nuestra mano. Ya son nuestros, gracias a O'Donnell, a
\emph{Sebo}, al hojalatero\ldots{}

---¿Qué dices, hombre?

---Digo que acabes. Quiero que la leamos juntos.

Momentos después leía María Ignacia: «Pepillo, ya no estamos donde
estábamos. Días ha, pensamos que debíamos mejorar de vida, tener
madriguera más cómoda y comer algo de más sustancia. Como nuestro ajuar
nos permite mudarnos tan fácilmente, levantamos el campo, y nos pusimos
en camino. ¿Hacia dónde? No te lo diré. Sabrás, sí, que tenemos amigos,
y que almas generosas hay dispuestas a protegernos\ldots{} Pues llevando
sobre nosotros lo que poseemos, carga muy llevadera para los dos,
rompimos marcha una mañanita por los campos y laderas de Dios. Donde nos
parecía bien, descansábamos; comíamos de lo nuestro, sin pedir nada a
nadie. ¿Qué nombre daban antiguamente a los pueblos, familias o personas
que andaban de un lado para otro? Es una palabra que ya no se usa en
sociedad: \emph{mónadas} o \emph{nómanas}. Pues bueno: somos caminantes,
no vagabundos, puesto que sabemos a dónde vamos. Nos dirigimos a una
tierra que a mi parecer no es muy bonita, pero sí de más avío para el
trabajo. Seguiremos de salvajes; pero no tan metidos en el reino de la
soledad. No siento más sino que lleguemos cerca de Madrid\ldots{} tan
cerca que me dé en la nariz la tufarada de vuestra civilización\ldots{}
olor de cuadra, olor de pucheros recalentados, olor de boticas\ldots{}
¡qué asco!»

«Hemos pasado por caseríos y pueblos. No te digo sus nombres; ni estoy
tampoco muy segura de saberlos. La Geografía escrita me interesa poco;
la que voy yo estudiando con mis ojos y mis pisadas\ldots{} o patadas,
como quieras\ldots{} me interesa más. Te escribo en la sacristía de la
parroquia de un pueblo, que no es de los más chicos ni de los más feos.
\emph{Vele ahí} que el sacristán es amigo nuestro, y nos cree marido y
mujer. En verdad que lo somos ante Dios y ante nuestras conciencias, y
con eso nos basta. Pero tememos que se nos descubra el engaño, y venga
la maldita ley con su cara de vieja legañosa y nos suelte una sentencia
bárbara. Por esto te escribo, querido Pepe, te escribo para que tú y tu
mujer, la simpática María Ignacia, se interesen por nosotros, y corten
el paso a esa ley entrometida y sin entrañas. Vosotros sois personas
influyentes, y en este país las personas de influjo lo pueden todo. Por
amistad y recomendaciones, en España se hace picadillo de las leyes. Los
ricos, si a más de ricos están un poco arrimados a la política, son los
amos de vidas y haciendas; y ya que su egoísmo hace tantas iniquidades,
haga su caridad alguna vez una obra buena\ldots{} Esto lo aprendí cuando
vivía con mis padres, y aún más en el suplicio de aquel matrimonio;
después lo he visto mejor en los campos, donde a cada triquitraque se ve
una víctima de esos que ahí llamáis \emph{altos funcionarios,
prohombres, eminencias de la Banca, de la Política}, etc., los cuales
vienen a ser o celebridades de figurón, o bandidos, verdaderos truhanes
con traje y ringorrangos de caballeros. Bandidos hay de la Política, que
explotan al Pueblo; bandidos eclesiásticos, que echan bendiciones, y
otras clases de bandolerismo ilustrado, como don Mariano, mi ex suegro,
del cual no puedo decir que \emph{santa gloria} haiga, porque
desgraciadamente no ha reventado todavía. ¡Ay, Pepe, lo que aprende una
cuando se hace salvaje, cuando se mete tierra adentro por el verdadero
país, y ve de cerca sus miserias, y siente el latido de la sangre de la
Nación!

«Verás en esto que te escribo \emph{un porción} de disparates, Pepe;
pero yo te digo: «Hazte salvaje como yo; bájate a lo más hondo de lo que
mi ex suegro llama \emph{capas sociales}, a esta capa de la pobreza que
vive sobre el terruño, y verás las verdades netas. Por
\emph{desageraciones} las tenéis los de allá. Pero yo me río de ti y de
tus sabidurías, sacadas de libros y discursos. Yo soy una ignorante que
ha leído en el libro grande de las cosas tales como son, y ha visto de
cerca la España en cueros, musculosa, cargada de cadenas. Viviendo en
ella y con ella es como nos instruimos. Yo sé más que tú, porque sé lo
que cuesta el pedazo de pan negro que se llevan a la boca, para no
morirse de hambre, cientos de miles de españoles\ldots» En fin, no te
digo más de esto, porque no siendo salvaje como yo, nunca llegarías a
comprender mis barbarismos. Vamos al por qué de esta carta.

»Nos fijaremos en un pueblo que no está lejos de Madrid, y en él
trabajaremos honradamente para ganarnos la vida. Como el \emph{aquél} de
nuestro trabajo nos ha de poner en comunicación con la maldita
\emph{Corte}, me temo que no nos valga el recurso de los falsos nombres
con que nos hemos rebautizado\ldots{} Tememos, querido Pepe, que entre
curas, polizontes, o algún alcaldillo de los que son criados de los
poderosos, nos hagan una mala partida. ¡Vaya, que si nos cogen y nos
llevan a Madrid atados codo con codo! No sé por qué, tanto \emph{Ley}
como yo confiamos en que tú evitarías nuestra persecución. Y ahora
pregunto: ¿estás dispuesto a protegernos, Pepe, asegurándonos contra las
asechanzas de esa condenada justicia? Hemos roto la ley, hemos hecho
mangas y capirotes de los que nos parecían falsos respetos o mentirosas
idolatrías. Para que nos separemos \emph{Ley} y yo, será menester que
antes nos descuarticen\ldots{} Conque ¿nos proteges?, ¿sí o no?

»Y ahora viene la gran dificultad. Para que tú puedas contestar a mi
pregunta, buen Pepillo, he de romper el secreto de nuestra residencia, o
valerme de una tercera persona, y ambas cosas son de mucho peligro,
porque no me fío de nadie, y aun a ti mismo te tengo un poquitín de
miedo. Cavilando en esta dificultad estuve toda la noche, y hoy toda la
mañana, sin que haya podido resolver nada a la hora en que te
escribo\ldots{} Llega el momento de seguir nuestra viajata, y me dan
prisa, mucha prisa para que concluya ésta, y pueda salir para Madrid
llevada por los geniecillos del aire. Rabia, rabia, que no sabes ni
sabrás cómo va mi carta, ni quién la lleva\ldots{} Pues no tengo más
remedio que poner punto aquí. La cuestión batallona de tu respuesta se
queda sin resolver; pero de aquí a mañana espero que se cuaje la idea,
que anda por mi caletre como una papilla. Perdona que materialice estas
cosas del pensamiento. Desde que soy salvaje, tengo más sal en la
mollera, más pesquis. Antes, en mi vida de señorita, no se me cuajaba
ninguna idea. Todas se quedaban en estado parecido a la clara de huevo,
como una baba, como un moco, y en mi vida de señora casada sólo cuajó
una idea, la de descasarme, como lo hice, y de ello no me arrepentiré
nunca. Pues verás cómo ahora el talentazo que me ha dado mi escapatoria
encontrará un bonito ardid para que sepamos si estás decidido o no a
protegernos contra curas y curiales, y contra guindillas, que son la
peor gente del mundo\ldots{} No puedo seguir por hoy. Un día o dos
tardaré en volver a escribirte. Prepárate. ¡Ay, Pepe, ten compasión de
este matrimonio montaraz que con nadie se mete, y se contenta con que le
dejen vivir!\ldots{} Yo le digo a \emph{Ley} que nos vayamos a
Marruecos, donde él tiene un hermano que se ha hecho moro y está en
grande; pero \emph{Ley} no se determina: no desespera de encontrar aquí
un buen acomodo para ganar el pan, y comerlo juntos\ldots{} De ti
depende, Pepillo\ldots{} Hasta mañana\ldots{} A Ignacia y a tu niño, mil
besos de vuestra amiga---\emph{Mita}.

Impacientes quedamos mi mujer y yo, aguardando la anunciada clave para
comunicar con la pareja silvestre, aunque en rigor no la necesitábamos,
porque \emph{Sebo} nos ofreció traer a nuestra presencia al aprendiz de
hojalatero, y someterlo a un interrogatorio, del cual habíamos de sacar
la verdad. Dispuestos estábamos ya a la cacería del chico, cuando llegó
la carta. \emph{Mita} me proponía un medio de los más inocentes para
mostrarle mis buenos propósitos en su favor. ¿Qué tenía yo que hacer?
Pues un papel semejante al de los novios de la categoría más angelical,
que se pasan el día tomando las medidas de la calle en que su amor
reside, y regocijando a los vecinos y transeúntes con los signos de una
telegrafía candorosa. No me imponía \emph{Mita} más que tres paseos de
ida y vuelta en la calle del Desengaño, entre San Basilio y Portaceli,
en día y hora fijos, y había de llevar un pañuelo blanco en la mano
derecha, no enteramente desplegado, sino en forma que fuese bien visible
a distancia. Podía, sí, hacer que me sonaba, como si padeciese un fuerte
romadizo; mas no guardarme el pañuelo en el bolsillo. Esta deambulación
de hombre constipado era la fórmula muda con que yo me comprometía
solemnemente a ser depositario leal del secreto de su residencia.

A mi mujer y a mí nos pareció ridícula esta telegrafía de galán
sensible, trota-calles, y además innecesaria, pues, decididos a proteger
a la pareja volante, teníamos medio más seguro y serio para ponernos al
habla con ella. Mejor que los paseos, pañuelo al aire, para que me viese
el hojalaterillo de la calle del Desengaño, era que nos desengañásemos
pronto y bien, cogiendo al tal chico y trayéndolo a nuestra presencia.
El astuto \emph{Sebo}, a quien di conocimiento, por la tarde, a solas,
de la carta de \emph{Mita}, opinó que eran puras pamplinas los
telégrafos propuestos por la señora libre, y me prometió detener al
chiquillo y traerle a casa. «Es su hermano, señor. Me consta que es
hermano no precisamente de ella, sino de él, \emph{vulgo} cuñado, y se
llama Rodrigo Ansúrez.

---Ansúrez, Ansúrez\ldots{} Le conozco\ldots{} conozco a toda la
familia\ldots{} Familia trashumante\ldots{} castillo de Atienza\ldots{}

\hypertarget{xvi}{%
\chapter{XVI}\label{xvi}}

Mi mujer y yo señalamos para la mañana del siguiente día la captura y
examen del hojalatero; pero el oficioso polizonte, desviviéndose por
servirme, nos trajo el chico aquella misma noche. Le cazó en la calle
del Desengaño cuando salía con un recado de arandelas de latón para la
Cofradía de la Leche y Buen Parto, y después de acompañarle hasta dejar
las piezas en San Luis, le condujo a nuestra casa, en calidad de preso,
sin darle más explicaciones que la oferta de una paliza si alborotaba
por el camino. Llegó a nuestra presencia consternado el pobre rapaz, y
lo menos que pensaba y temía era que le íbamos a condenar a cadena
perpetua. A mi mujer y a mí nos dio lástima de verle tan compungido y
lloroso, como un reo que se dispone a confesar sus tremendos crímenes,
entregándose a la compasión y a la indulgencia de sus jueces. Trabajo
nos costó apartarle de los ojos los puños, y hacerle comprender que no
le haríamos ningún daño.

---Ya sé que te llamas Rodrigo Ansúrez---le dije,---y que eres buen
aprendiz de hojalatería. Tu padre se llama Jerónimo\ldots{} Mírame bien,
y di si recuerdas haberme visto en alguna parte. Sus ojos empañados del
llanto se fijaron en mí; mas no revelaron que me conociera. Cuando en el
castillo de Atienza vi a la familia celtíbera, la rama más pequeña de
aquel frondoso árbol humano era el niño a quien Miedes llamaba
\emph{Ruy}, buscando el son arcaico de los nombres. Hoy representa unos
catorce años, y en su persona resplandece la hermosura y gallardía de
aquella selecta casta de españoles. Yo no me cansaba de mirar en el
rostro de él la impresión del de su hermana; impresión borrosa y triste,
como la del rostro de Jesús que los cristianos vemos en el paño de la
Verónica. Sólo que aquí era semblante de mujer, no impreso en una tela,
sino en otro semblante; y por cierto que no era afeminada la cara de
Rodrigo, sino muy varonil, en su adolescencia vigorosa.

Las primeras declaraciones del hojalatero fueron de cerrada negativa a
cuanto le preguntamos; mas, al fin, Ignacia con dulzura, \emph{Sebo} con
rudeza policíaca, y yo con un tira y afloja entre ambos resortes de
convicción, le trajimos a la verdad. Con decirle y asegurarle que no
queremos descubrir a los salvajes para perseguirles, sino para
socorrerles, acabé de traerle a nuestra confianza, y obtuvimos los
secretos que embuchados tenía. Aquí pongo lo esencial de la revelación:
Leoncio Ansúrez, hermano del declarante, es el \emph{robador}, palabra
textual, de la señora Virginia. Leoncio ha cumplido ya los veintidós
años, y es muy hábil en cerrajería, en composturas de toda clase de
máquinas, y conoce y maneja con admirable pericia las armas de fuego. De
la arrancada inicial, se fueron los amantes a San Sebastián de los
Reyes, donde estuvieron pocos días, y allí, tirando siempre hacia el
Norte por la Mala de Francia, llegaron a la falda de la Sierra. Allí se
establecieron, cambiándose los nombres y figurándose \emph{matrimonio
por la religión}. En diferentes parajes habitaron, acomodando su vivir
errante a las necesidades de cada día, y a las ofertas de trabajo para
satisfacerlas. En un pueblo que llaman Bustarviejo, Virginia lavaba, y
Leoncio se contrató con el Ayuntamiento para matar los animales dañinos
que en invierno bajan de la Sierra, cobrando tantos o cuantos reales por
cada cabeza de alimaña que presentase. Por una loba, treinta y cinco
reales, y veinticinco por el macho; por cada zorra, ocho reales; por
cada gato montés, seis; por un tejón, doce; por un patialbillo, lo
mismo. El turón, la garduña y la jineta se pagaban a siete reales, y el
águila, el milano, el alcotán y el búho valían dos reales. De todo esto
dio relación minuciosa el pobre chico, declarando con cierto orgullo que
había él andado con su hermano en aquellas arriesgadas monterías, y
terminó esta parte del relato diciéndonos que por una culebra que no
tuviera menos de tres cuartas de largo, daban un real.

Disentimientos de Leoncio con el señor Alcalde por la informalidad en el
pago de alimañas muertas, moviéronle a largarse de Bustarviejo,
corriéndose hacia los altos de la Sierra. El declarante, obligado a
volverse a Madrid para seguir en el oficio, no les siguió en esta nueva
etapa de azarosos trabajos. Sabe que Leoncio llevaba una vida muy
aperreada, sacando piedra de las canteras, y que estuvo muy malo, en
gravísimo peligro de muerte. Ignora por qué los salvajes abandonaron la
Sierra viniéndose a tierra llana. Seguramente, alguien les ofreció por
acá mejores medios de vida. El pueblo donde ahora se encuentran es
Coslada, a mano derecha del camino real, más allá de Canillejas. De
Coslada escribió Virginia su última carta con las instrucciones para que
yo le diese en la forma que he referido las seguridades de mi
protección, y Rodrigo tenía órdenes de al transmitir al mismo pueblo lo
que resultara de mi paseo telegráfico, si en efecto yo respondía por tan
ridículo lenguaje. Con esto concluyó la declaración del hojalatero, y
dimos por bien empleada su captura.

Alegres por este feliz resultado, tranquilizamos al hojalatero,
añadiendo a nuestras palabras cariñosas una gratificación en metálico,
que no quería tomar ni a tiros. Para que consintiese en aceptarla, fue
preciso que mi mujer le repitiera una y otra vez que no haríamos ningún
daño a Virginia y Leoncio; antes bien, ellos y toda la familia tendrían
de nosotros cuanto pudieran necesitar. No quise dejarle partir sin que
me diese informes de su parentela. Díjome que su padre vivía tranquilo y
satisfecho en la Villa del Prado, \emph{al frente} de la labor de su
yerno el señor Halconero. De su hermana Lucila diome la estupenda
noticia de que ha engrosado considerablemente, y tiene ya dos chicos. Oí
estas referencias como el estallido de una bomba de poesía que se
deshace en cascos de prosa. Hízome el efecto de una esfera de cristal,
lumínica, que se rompe, se apaga, disolviéndose en un vapor
rastrero\ldots{} ¡Gorda y campesina, con principios de numerosa
prole!\ldots{} Guiada por el tiempo, la Naturaleza nos baja desde las
cumbres de la vida soñadora al llano de la vida ordinaria.

\emph{31 de Junio}.---Indulgente Posterioridad: Antes que yo te lo diga,
comprenderás que, sabido el paradero de \emph{Mita} y \emph{Ley},
determiné correr hacia ellos. Menos vehemente que yo, mi mujer quiso que
lo tomase con calma, pues tiempo había de sobra para ejercer la caridad
con el libre matrimonio. Mas no me convenció, y aquella misma noche
mandé preparar un coche de colleras con buenas mulas, para salir a la
siguiente mañana con la fresca. Viéndome tan decidido, María Ignacia no
insistió, pues harto conoce cuán pernicioso es para mi salud el continuo
encierro dentro de la esfera de un vivir acompasado y sin accidentes.
Bien sabe mi esposa que contenerme en estas expansiones de generosidad
es reducirme a tristeza y desaliento. Convinimos, al fin, en que
llevaría conmigo al criado de más confianza, un hombrachón atenzano,
llamado Zafrilla; y para ir reforzado de un poquitín de autoridad
gubernativa, que bien podía ser necesaria, acordé llevarme también al
gran \emph{Sebo}. Figúrate, ¡oh Posteridad!, el júbilo con que esta idea
fue acogida por el interesado. Pero como tiene, según dijo, servicio en
el barrio de la Plaza de Toros, desde media noche hasta el amanecer, me
rogó que pues yo había de salir por la Puerta de Alcalá, mandase parar
el coche en el Parador de Muñoz, donde se uniría conmigo para
\emph{custodiarme} hasta Coslada.

Salí, pues, tempranito con Zafrilla. Guiaba un excelente cochero, y el
ganado era de lo mejor: tres mulas capaces de llevarme a Pekín, si
necesario fuese. Para que nada faltara, llevábamos provisiones para
sustentarnos durante todo el día, y aun para dejar surtidas las flacas
despensas de la desamparada \emph{Mita}. Nada digno de contarse nos
ocurrió a la salida de Madrid. Pero al llegar al Parador de Muñoz,
serían las seis y media de la mañana, me vi sorprendido por la súbita
emergencia de un interesante capítulo de la Historia de España. Entró
\emph{Sebo} en el coche con risueño semblante, el bigote más cerdoso y
erizado que de costumbre, y entre salivas, me roció estas palabras:
«Señor, ya se armó la trifulca. Ya está O'Donnell en campaña con sin fin
de tropas de Caballería y mucha Infantería.

---O usted sueña, o al tomar la mañana, ha empinado más de lo regular.

---Yo no empino sino cuando tengo disgustos en casa. Pero en todo lo que
es del procomún, guardo la serenidad para hacerme cargo bien de las
cosas, y ver qué postura me conviene\ldots{} Por aquí han pasado, serían
las dos y media, tropas que no sé si son de \emph{Extremadura}. Iban
algo desmandadas. La Caballería, según me han dicho, salió por la Mala
de Francia, y con ella los del \emph{Príncipe}, mandados por Echagüe, y
en el Campo de Guardias hicieron alto. Al frente de la Caballería iba el
Director del Arma, general Dulce.

---Eso no puede ser, \emph{Sebo}. Don Domingo Dulce dio al Gobierno
\emph{polaco} seguridades de lealtad.

---Lealtad es palabra de dos caras: con una mira al Gobierno de la
Reina; con otra, a la Reina del mundo, que es la Libertad sacratísima.

---Revolucionario estáis, amigo \emph{Sebo}.

---Es que no como; es que once reales y medio al día dan poco de sí,
excelentísimo señor, y una de dos: o las revoluciones no sirven para
nada, o sirven para que el español un poco listo ponga unos garbanzos
más en el puchero, y si a mano viene, una pata de gallina\ldots{} Digo y
repito que el general Dulce ha sacado la Caballería, que es como sacar
el Cristo. Vuecencia no podrá negarme que este Dulce no lo es más que
por su apellido, pues tiene un genio más agrio que las uvas verdes, y la
mano, como las de almirez, muy dura. Esto va de veras, señor; esto no es
ya jugar a los soldaditos. Hoy temblará Madrid.

---Pero, a todas éstas, de O'Donnell nada se sabe.

---Se sabe que a las tres de la mañana salió por la Puerta de Bilbao, en
coche que guiaba el propio marquesito de la Vega de Armijo\ldots{} No sé
si se agregó a la Caballería en el Campo de Guardias. Los sublevados a
pie y a caballo, otros que iban en coche, y muchos paisanos, bajaron,
antes de amanecer, del Campo de Guardias a la Fuente Castellana,
siguiendo por los tejares hasta la venta y portazgo del Espíritu Santo.

---Llevan, como nosotros, la dirección de Canillejas o de Alcalá de
Henares.

---Para mí que no pasan de Canillejas, donde aguardarán a las tropas del
Gobierno para darles la batalla.

Cuánto me alegré de este inopinado brote de sucesos graves, no hay para
qué decirlo. Frente a mí tenía una revolución, no de éstas que se
manifiestan en las declamaciones teóricas de libros y discursos, sino
viva, con choque formidable de hombres y caballos, caídas de cuerpos y
de ideas, alzamiento de nuevos principios. El asunto que motivaba mi
viaje por el camino real de Aragón quedaba ya en segundo término, y lo
más interesante para mí era la página histórica que de improviso ante
mis ojos se abría. Mi alma necesita hoy, más que nunca, un poco de
drama. La comedia me aburre ya, y sus blandas emociones no satisfacen el
hambre de mi espíritu. Hasta la política, desde las guerras últimas, ha
venido a ser casera, cominera y familiar, como las comedias del amigo
Bretón. Ya llevamos largos años de una paz tediosa, empapada en la
insulsez administrativa. Por ley física, han de venir ahora
estremecimientos que despierten la vitalidad de la Nación, que hagan
circular su sangre y sacudan sus nervios. Tendremos, pues, enfermedad
saludable, de esas que hacen crisis en el individuo, y promueven el
crecimiento, la adquisición de fuerza nueva.

Pensando esto, deseaba yo que las mulitas de mi coche se convirtieran en
hipogrifos, para que velozmente me transportasen al lugar en que la
página histórica había de ser escrita con empujones, gritos, choque de
armas, sangre, y todo lo demás que es del caso, hasta que caen unos
hombres y otros suben, y las utopías derriban del pedestal a las
verdades para ponerse ellas\ldots{} De estas meditaciones me distraía el
gran \emph{Sebo}, haciéndome notar los grupos de paisanos armados que
por el mismo camino que nosotros iban. Unos llevaban trabucos, otros
escopetas; reían y bromeaban como si fueran a una feria. Vi caras
conocidas; otras que anualmente se ven en la función patriótica del Dos
de Mayo, en las algaradas callejeras, en las ejecuciones de pena de
muerte, en las Vueltas del día de San Antón, y en el Entierro de la
Sardina. A muchos designó Telesforo como gente alborotada y maleante,
\emph{patriotas} de oficio que acuden a donde hay tumulto y bullanga por
la Libertad o la Constitución, aunque ninguno sepa cuál de las que
tenemos está vigente\ldots{} Y también iban entre ellos algunos de quien
\emph{Sebo} se recató, agachándose para no ser visto. «Estos que ahí
van, señor---me dijo,---son los de más cuidado entre la familia
\emph{patriotera}. Si llegan a verme, no escapamos sin que nos tiren una
piedra, a mí, que no a Vuecencia\ldots{} y a uno de los dos nos podían
machacar un ojo. Me tienen tirria porque les he metido mano más de
cuatro veces cuando andaban en el trajín de acariciar lo ajeno. Ahora
van tras de O'Donnell, como irían tras de José María o del moro Muza.
Éstos confunden a la diosa Libertad con el dios\ldots{} ¿No hay un dios
que se llama Caco?

---No era dios, según creo. Para mí era de la Policía.»

\hypertarget{xvii}{%
\chapter{XVII}\label{xvii}}

Estaba yo en mis glorias, y me recreaba previamente en las emociones de
aquel día, como un chiquillo contemplando los zapatos nuevos que van a
ponerle. El polvo que mi coche levantaba rodando por el camino real,
parecíame polvo de batalla, y en los cambiantes que hacían sus ondas
traspasadas por el sol, veía yo las terribles falanges en lucha. El
paisaje que a los dos lados del camino se extiende, no podía ser más
apropiado a guerras y trapisondas. Todo es allí aridez, tierras
desoladas y libres, donde los hombres no tienen nada que hacer, como no
sea lanzarse a desesperados combates, por el gusto de pelear, no por la
conquista de un suelo que tan poco vale.

A la vista de Canillejas, vimos obstruido el camino por un grupo de
gente que vitoreaba a la Libertad y a los generales sublevados. Mandé
parar, y antes que yo pudiese ordenar al cochero y a mis acompañantes
que secundaran los clamores patrióticos, saltó \emph{Sebo} al camino, y
echó al aire su sombrero de copa, gritando hasta desgañitarse. Pasó el
sombrero de mano en mano hasta volver a las de su dueño en estado de
ruina lastimosa, y sin ponérselo, para no desairar su figura, pronunció
\emph{Sebo} un ronco panegírico de la revolución, terminándolo con
frenéticos vivas a mi humilde persona. Entendió la multitud que iba yo
en seguimiento de la columna de O'Donnell, y no fue preciso más para que
me aclamasen como a libertador de la clase civil. Los más próximos al
coche me contaron que las tropas habían hecho un alto en Canillejas para
reconocer y proclamar la autoridad suprema de O'Donnell, el cual se
presentó vestido de paisano, a caballo. Vulgar y breve fue su arenga,
limitándose a las frases de ritual en la literatura de
pronunciamientos\ldots{} «que él no daba aquel paso por vengar agravios
personales, sino por sacar a la Patria de su envilecimiento\ldots{} que
para esto era menester el esfuerzo de todos sus hijos\ldots» y pitos y
flautas\ldots{} Eran los tópicos de siempre, y las inveteradas fórmulas
de requiebro que gastan los políticos delante de la Nación, cuando
encaran con ella para declararle un amor honesto, apasionado y con buen
fin.

Disparado por O'Donnell el ruidoso cohete de la proclama, siguieron las
tropas su camino. Quién decía que llegarían hasta Alcalá, quién que no
pasarían de Torrejón. Entré yo en Canillejas, y al arrimarnos a un
parador para dar pienso a las mulas, y a nuestros cuerpos alguna
reparación, me vi rodeado de multitud de gente de aquel pueblo de
Madrid, que ávidamente me pedía noticias del plan de campaña, y de lo
que hacía o dejaba de hacer el Gobierno. ¿Continuaba la Corte en El
Escorial? ¿Era cierto que Sartorius había salido de Madrid disfrazado de
carbonero, y que se formaba un Ministerio Blaser-Custodia-Domenech?
¿Disponía el Gobierno de tropas leales para batir a los revolucionarios?
¿Se confirmaba que la \emph{Polaquería} no contaba con un solo soldado?
Contesté que nada sabía yo de planes de campaña; y a responder a las
otras preguntas me disponía, cuando \emph{Sebo} me quitó la palabra de
la boca para trazar una relación sucinta de los acontecimientos futuros,
como si ya fuesen pasados y él los hubiese visto. De las fogosas
expansiones de mi acompañante, declarando que había sido de la Policía,
pero que ya renegaba de su ominoso empleo, y ponía su voluntad y la
porra de su bastón al servicio de los caudillos libertadores; de esto y
de mi buen humor resultó que hube de convidar a todos los presentes a
tomar cuantas copas quisieran. En medio del barullo patriótico y
tabernario que allí se armó, yo, silencioso, batallaba en mi espíritu
entre un deber y un deseo. ¿Qué haría yo? ¿Seguir mi camino hacia
Coslada en cumplimiento del plan humanitario, móvil primero de mi viaje,
o abandonar este plan para correr tras el suceso histórico que la suerte
me deparaba? Por fin, pudo más que la obligación la curiosidad, y a ello
contribuyó mi diablo con estas sugestivas razones: «Señor, lo primero es
la Patria, que hoy está en el filo de perderse o salvarse. Vuecencia es,
ante todo, un buen español. ¿Cuándo se le presentará ocasión como ésta
de ver salvar a España, y aun de contribuir, si a mano viene, al
salvamento? Y considere que para visitar a los bárbaros de Coslada, lo
mismo da un día que otro.»

No necesité más para convencerme: mandé enganchar, y salimos hacia
Torrejón. Al mediodía pasábamos el puente llamado de Viveros; poco
después entrábamos en el pueblo a que ha dado fama un hecho militar del
amigo Narváez. Rindiendo culto a la verdad histórica, debo decir que
nuestra entrada fue triunfal, entre aplausos, vocerío y disparos al
aire. Creían que llevábamos la dimisión y caída del Gobierno, la subida
de O'Donnell, quizás la cabeza de Sartorius. No me valió decir que nada
de esto llevábamos, porque el maldito \emph{Sebo}, con sus gárrulos
discursos, hacía entender a la gente que no íbamos a Torrejón por pura
curiosidad. También allí vi defraudada mi esperanza de alcanzar la
columna de O'Donnell. Poco antes de mi llegada había salido el caudillo
para Alcalá con el grueso de la tropa sublevada, dejando en Torrejón el
regimiento del \emph{Príncipe} con Echagüe y una sección de Caballería.
Tuve intención de verle: quería yo charlar con mi amigo de aquel
aparatoso alzamiento; mas, antes de llegar al caserón donde se alojaba,
me vi envuelto por una nube, que de otro modo no puedo llamar a la
turbamulta de personas que me rodearon, caras de Madrid, conocidas,
algunas de amigos.

La primera intimación fue que nos reuniéramos todos a comer. Díjeles que
yo les convidaba, pues, a más del repuesto de provisiones de boca, traía
exquisitos vinos: comeríamos y beberíamos a la salud de los
libertadores. Interpretando con agudeza y prontitud mis deseos, corrió
\emph{Sebo} al parador y mandó disponer comistraje abundante, de lo que
hubiese, que con lo llevado por nosotros formaría un banquete
espléndido. Y mientras bajo la dirección de Telesforo funcionaban las
cocinas, recorría yo el pueblo de un lado para otro, viéndome abrazado
por cuantas personas encontraba. En estas expansiones populares, el
abrazo entre desconocidos es el signo externo del cordial regocijo, de
la esperanza que toda insurrección despierta en el sufrido pueblo
español, mal gobernado siempre. Las revoluciones tienen entre nosotros
el carácter de salutación al nuevo tiempo, y establecen entre los
ciudadanos la confianza, la fraternidad y el generoso cambio de
demostraciones cariñosas. Yo me vi en Torrejón festejado por la
multitud. No sólo me abrazaban los de Madrid, sino los del pueblo, éstos
con mayor efusión. A mi paso avanzaban también las mujeres, alzados los
brazos, y soltaban con chillona voz el grito de ¡Viva España! Algunas
viejas me besuquearon, y los chicos gritaban: ¡Viva Madrid! ¡Vivan los
hombres de corazón! Se les había metido en la cabeza que yo llevaba una
misión política, y no siéndome fácil sacarles de aquel error, pues no
había razón que les convenciera, dejeme llevar de la ola popular. Cerca
del caseretón que me pareció Ayuntamiento, se vino hacia mí un señor que
con cierta solemnidad se presentó a sí mismo, diciendo: «Simón Carriedo,
Alcalde de Torrejón de Ardoz.» «Por muchos años,» contesté yo dejándome
abrazar, y él prosiguió: «Está Vuecencia en uno de los pueblos más
liberales de España. Aquí aborrecemos la tiranía, y queremos un Gobierno
que mire por la libertad y por la ilustración. ¡Viva Isabel II! ¡Mueran
los \emph{polacos}!\ldots{}

---Bien, señor, muy bien. Pero yo\ldots{}

---No se nos achique Vuecencia, ni crea que aquí no conocemos a los
hombres de valer. Torrejón sabe que tiene en su recinto al que es cabeza
civil de la revolución bendita\ldots{} Señores: ¡Viva el marqués de
Beramendi!

---¡Vivaaa\ldots!

---Pero, señores---dije balcuciente, de pura modestia,---yo les aseguro
que no toco pito\ldots{}

---Adelante\ldots{} Aquí no valen tapujos. Torrejón es un pueblo muy
liberal, un pueblo ilustrado\ldots{} El Ayuntamiento, señor Marqués, se
reúne esta noche para proclamar con la debida solemnidad el
pronunciamiento. Torrejón se pronuncia, Torrejón destituye a Sartorius,
y no reconoce más autoridad que la de los libertadores. ¡Viva Isabel II!

Pues adelante. Ya no me opuse a ninguna demostración; ya me creí
obligado a tomar la iniciativa en los abrazos, y estrechaba efusivamente
contra mi pecho a todo el que cogía por delante. Y mientras esto
ocurría, noté que en todas las ventanas y ventanuchos aparecían trapos
de colores, colchas y pañuelos; sábanas, donde no había otra cosa, y
hasta bayetas amarillas, en representación del tono gualda de nuestra
bandera. El pueblo se engalanaba para festejar el cambio venturoso, la
nueva dirección hacia los vagos horizontes del progreso y el bienestar.
Todas las mujeres estaban en la calle: algunas alzaban en brazos sus
chiquillos mamones, como alzarían un estandarte, o cualquier signo para
guiar a las multitudes, y los muchachos sacaban cuantos objetos pudieran
servir de instrumentos de ruido para imitar el de tambores, remedando
con boca y narices el piafar de caballos y el estridor de cornetas. En
medio de esta algazara, vino \emph{Sebo} a decir que la comida estaba
pronta. Convidé al Alcalde, que aceptó, con la recíproca de prometerle
yo cenar en su casa. Arrastrado por aquel vértigo y metido en él, llegué
a creerme que soy, en efecto, la cabeza civil de la revolución; y en el
bullicio del parador, rodeado de diversa gente, tan dispuesta al buen
comer y mejor beber como al derroche de palabrería patriótica, mi
alucinación casi llegó al pleno convencimiento. Los discursos que en el
curso de la comilona pronunció \emph{Sebo}, arranques oratorios con
toques de esa sinceridad bufonesca que tanto agrada en los días de
exaltación popular, me contagiaron, lanzándome a improvisaciones locas.
Ni recuerdo bien lo que dije, ni hago traer aquellos disparates desde
las neblinas de mi memoria a la claridad de estas páginas.

Entre los comensales había dos cadetes de Caballería que desde el primer
instante de nuestro conocimiento me encantaron por su juvenil desenfado,
por su ingenio vivísimo, que así se manifestaba en la charla voluble
como en los desahogos de la maledicencia política. No he olvidado sus
nombres: Pastorfido se llamaba el uno; el otro Narciso Serra, y ambos
hablaban por los codos, empezando en prosa y acabando en verso. Serra,
principalmente, se disparaba en redondillas sin darse cuenta de ello.
Cuando salíamos del parador para ir al alojamiento del brigadier
Echagüe, me dijo con la naturalidad de la prosa corriente:

\small
\newlength\mlenc
\settowidth\mlenc{Sueño pensando en mi suerte.}
\begin{center}
\parbox{\mlenc}{\textit{\quad ¿Dormir yo? No tengo gana.               \\
                        Sueño pensando en mi suerte.                   \\
                        El descanso de la muerte                       \\
                        quédese para mañana.}}                         \\
\end{center}
\normalsize

Antes de visitar a Echague, acordeme de mi casa, de mi mujer y mi niño,
y sentí ardientes deseos de volverme a Madrid. Mas ya era tarde para
emprender el regreso, y además la página histórica, ofreciéndose a mi
mente con extraordinaria luz, debilitó mi querencia del hogar y de la
familia. Resolví quedarme, y para que María Ignacia no estuviese con
cuidado, mandé a mi criado Zafrilla que alquilase un buen caballo y a
Madrid partiera sin demora. Hecho esto, fue sosegada mi permanencia en
Torrejón toda la noche, que hube de pasar de claro en claro, por los
obsequios del Alcalde, por el patriotismo bullanguero del vecindario y
el continuo movimiento de tropas, y por los divertidos coloquios con
Serra y Pastorfido, que ni dormían ni dejaban dormir a nadie.

A Echagüe le encontré caviloso, inquieto, como hombre que sabe pesar la
grave responsabilidad contraída. La importancia militar y política de
los personajes sublevados era garantía de un éxito feliz; pero siempre
quedaba el temor de inesperadas contingencias, de ésa no vista piedra
que hace volcar el carro, del olvidado detalle que destruye en un
momento la lenta obra de la previsión. Díjome que los Generales contaban
con el concurso de todas las fuerzas que tenemos en Alcalá, y con los
medios que ofrece aquel depósito para poder armar buen número de
paisanos. Nada sabía de los planes del Gobierno, que de fijo habría
dispuesto que la Corte regresase a Madrid, para disponer de las tropas
de guarnición en el Real Sitio. Con poca o ninguna Caballería contaba el
Gobierno; en cambio, no le faltaba la suficiente Artillería para dar un
mal rato a los rebeldes. ¿El plan de O'Donnell era caer sobre Madrid en
son de ataque, o esperar a pie firme al ejército llamado leal? No lo
sabía, o quizás sabiéndolo no estimaba prudente decírmelo. Ya de
madrugada, supe por mis amigos Serra y Pastorfido que Echagüe tenía
orden de ponerse en camino a la mañana siguiente, tomando la dirección
de Coslada. A Coslada iría yo también, haciendo de la página histórica y
de la novelesca una sola página.

Dormí un par de horas, y más habría dormido si no me despertara con
grandes voces mi amigo el coronel Milans del Bosch, que acababa de
llegar de Madrid, de paso para Alcalá, con una misión del Gobierno.
Hombre expansivo, de corazón fácilmente inflamable por toda idea
generosa, pródigo de palabra, en las resoluciones más impetuoso que
tenaz, ha ilustrado su nombre en las armas, junto a Prim, y en sociedad
es de los que saben ganar numerosos amigos. Mientras yo me vestía,
tomaba el desayuno que le subió \emph{Sebo}. Hablamos de la revolución,
que él miraba con simpatía como liberal y patriota, y lamentaba que la
disciplina no le permitiera secundarla. Tal fuerza expansiva tenía en su
alma la sinceridad, que no me fue difícil obtener alguna indiscreción
referente al mensaje que llevaba. Oyéndole charlar con espontaneidad
impropia de un diplomático, vine a sacar en limpio que la Reina
concedería su perdón a O'Donnell y a los demás Generales,
reconociéndoles sus grados y honores, siempre que volviesen a Madrid y
entregaran a Dulce para someterle a un Consejo de guerra. Me pareció que
era gran tontería proponer a un sublevado español vilipendio semejante,
y que la misión del parlamentario había de ser inútil. También Milans
así lo creía. En él advertí desconfianza de su papel diplomático, y
ganitas de ponerse al lado de los libertadores y en el puesto de mayor
peligro. Es de los que no quieren lucha sin gloria, ni ven la gloria
donde no hay mil probabilidades de perder la vida.

Bajamos a la plaza, y cuando le despedía junto a la portezuela del
coche, me veo venir a Andrés Borrego rodeado de un grupo de patriotas y
periodistas. Habían llegado por la noche, y después de un descanso breve
continuaban camino de Alcalá. Poco pude hablar con aquel buen amigo tan
experto en cosas políticas y revolucionarias. Díjome que el Gobierno
había perdido la chaveta, y con sus desatinos daría el triunfo a la
insurrección. No se le ocurría más que ordenar prisiones de gente de
viso, entre ellas los banqueros Collado y Sevillano; suspender toda la
prensa independiente, y amenazar con comerse los niños crudos. «Pero lo
más ridículo que estos pobres \emph{polacos} han podido idear---me dijo
Borrego en los últimos apretones de manos,---es la revista militar que
han dispuesto para hoy en el Prado, con asistencia de Su Majestad, para
que las tropas la vitoreen y le digan que por ella derramarán su sangre.
Ya sabe usted, mi querido Beramendi, que estas paradas son un recurso
teatral que nada resuelve. En ninguna revolución ha faltado este
prologuito de las grandes catástrofes, ceremonia militar, desfile de
soldados melancólicos. Los vivas de ordenanza, el estruendo de clarines
y tambores, suenan a melopea desmayada y quejumbrosa, a marcha fúnebre.

\hypertarget{xviii}{%
\chapter{XVIII}\label{xviii}}

Sueltos, en parejas o en alegres bandadas, pasaron también por Torrejón,
el día de San Pedro, multitud de pájaros, la inquieta juventud de estos
tiempos, revolucionaria y masónica, vanguardia del pensamiento y
zapadora de la acción. El \emph{polaquismo}, con sus increíbles
desafueros, ha fomentado el entusiasmo, la impaciencia temeraria y
generosa de la juventud militante. Mal haya el Gobierno que desprecia
estas manifestaciones de la vida nacional en que andan poetas y
escritores sin juicio. Resultará que lo tienen de sobra cuando son
olvidados o perseguidos. Y por fin, de los que hacen coplas o chistes es
el reino de la opinión\ldots{} A muchos de los que en Torrejón
aparecieron conocía yo de trato, a otros de nombre y fisonomía. Hablé
con Cristino Martos, que en todas las funciones de la palabra es orador,
como es poeta Serra siempre que abre la boca. El sentimiento
revolucionario se desborda en él con las formas gramaticales más graves
y rítmicas. Lleva en sí el espíritu girondino: su verbosidad sentenciosa
resulta noble y clásica, y por esto mismo no es de los que conmueven a
la plebe. Yo le digo que, hablando siempre en nombre del pueblo, resulta
el más aristocrático de los tribunos. Ortiz de Pinedo, Cisneros, Somoza,
Abascal, y otros que vi pasar aquel día, me contaron que de Madrid
venían contra los sublevados los generales Blaser y Lara con la
Infantería que había quedado en Madrid y la que regresó de El Escorial,
con la Caballería de \emph{Villaviciosa}, algunos tercios de Guardia
Civil y no pocas piezas de artillería\ldots{}

De mi casa me trajo Zafrilla noticias que me permitieron aguardar con
tranquilidad los acontecimientos que Clío nos preparaba. Esta bondadosa
divinidad cuida siempre de evitar el aburrimiento a los pueblos que se
envanecen de tenerla por relatora de sus grandezas. Y apenas entró en
funciones la buena musa en aquellos campos, tuve que tomar nota de un
hecho singular: la transformación del gran \emph{Sebo}. No viéndole por
parte alguna en toda la mañana, mandé a Zafrilla que le buscase, y al
fin me le trajo en tal manera cambiado, que al pronto no le conocí.
Habíase afeitado el cerdoso bigote, operación que debió de inutilizar
las navajas barberiles. Se había cortado el pelo al rape, haciéndose un
tipo de cura montaraz, que se completaba con ropas negras y raídas, faja
mugrienta, obscura, y gorra de pelo de conejo. «Señor Marqués---me dijo
con voz que revelaba más miedo que vergüenza,---he tenido que
disfrazarme porque\ldots{} desde anoche andan por aquí más de cuatro y
más de cinco policías, algunos de mi propia sección y de mi propio
barrio\ldots{} Traen proclamas \emph{leales} para repartirlas a los
soldados, y con las proclamas, sin fin de mentiras que van echando en
todos los oídos para que la gente se desanime\ldots{} Me han visto, me
han preguntado si ando con la Revolución, y les he respondido que estoy
donde estoy. No debe uno comprometerse antes de tiempo\ldots{} No debe
uno dar su brazo a torcer\ldots{} A la Revolución pertenezco yo en
cuerpo y alma, y de ella espero la recompensa de mis buenos servicios.
Pero mientras se decide si la Revolución vive o muere, ¿qué necesidad
tengo de dar la cara? Póngome esta postiza para que mis compañeros no
puedan decir que han visto a \emph{Sebo} entre los sublevados. Si vienen
mal dadas, serán capaces de fusilarme o de meterme en un
presidio\ldots{} Y sería lástima, excelentísimo señor, porque, aquí
donde Vuecencia me ve\ldots{} no creo haber nacido para el oficio vil de
corchete\ldots{} Modestia a un lado, señor, Telesforo se siente jefe
político\ldots»

Asentí a cuanto decía, regocijándome de su infantil ambición, no
enteramente injustificada, pues gobernadores he visto salidos del más
bajo montón burocrático, o de obscuros aprendizajes políticos\ldots{}
Sigo contando. En la tarde del 29, gran número de paisanos mal armados o
por armar entraron en Torrejón, presentándose al brigadier Echagüe. Al
anochecer supimos que en la madrugada próxima saldría de Alcalá la
división libertadora, engrosada con las fuerzas de Infantería y
Caballería que guarnecían aquella ciudad, con el contingente de la
Escuela Militar, con los setecientos quintos armados de tercerolas, y el
núcleo de paisanos, que iba aumentando por el camino. Dieron a la hueste
adventicia el nombre de Voluntarios de Madrid. Madrugamos para salir al
encuentro de esta variada y pintoresca tropa. Salió todo el vecindario:
la carretera parecía un campo ferial en movimiento. Nunca vi gente más
alegre: creyérase que esperaban lluvia de monedas de oro y plata, o
presenciar gloriosos combates caballerescos, con intervención del
apóstol Santiago. ¡Desgraciado pueblo, que no esperando nada de la paz,
porque en este escepticismo lo mantienen sus gobernantes, lo espera todo
de la guerra civil.

Cuando las avanzadas del ejército libertador aparecieron, destacándose
del horizonte obscuro en las primeras claridades del alba, rompió la
multitud en exclamaciones de júbilo. El toque de clarines de Caballería
y el grave paso de ésta encendían en todos los corazones un sentimiento
de admiración, de piedad y ternura, que no es fácil definir. En los
sentimientos que despierta tan sublime música, se confunden y hermanan
la grandeza heroica y el fervor religioso. Las pausas en el toque, aquel
silencio del metal sonoro que deja oír las patadas rítmicas de los
caballos, es de una solemnidad que induce a la efusión, al llanto
mismo\ldots{} Entró la Caballería en Torrejón, después la Infantería y
Voluntarios, luego el Estado Mayor General escoltado por una sección de
coraceros\ldots{} Por falta de viento, la nube de polvo rastreaba, no
subiendo más arriba de las barrigas de los caballos y de los pies de los
jinetes. Lamañana entró alegre, luminosa, esparciendo su luz rosada por
los campos estériles y por las abigarradas multitudes de militares y
campesinos. Dijérase que traía la fecundidad al suelo, y a todos los
corazones la esperanza.

En el ejército encontré multitud de amigos. Pero apenas pude hablar con
algunos, pues el descanso en Torrejón fue brevísimo. Salieron las tropas
en dos divisiones; la una, mandaba por Dulce, siguió por el camino real
con órdenes de llegar hasta Canillejas para hacer un reconocimiento; la
otra, con O'Donnell al frente, tomó la dirección de Vicálvaro. A la
impedimenta de ésta me agregué yo, gozoso con la idea de pasar por
Coslada. En esto me equivoqué, porque el paso fue por un sitio distante
media legua del pueblo en que los salvajes residían. Salieron a vernos
hombres, chiquillos, mujeres. Miré las caras de éstas, buscando entre
ellas la de Virginia; pero no la vi. O no estaba, o desfigurada
totalmente por la barbarie, no pude reconocerla.

En la parada que allí hicimos, se adelantó \emph{Sebo} para refrescar
con el aguardiente que vendían unos cantineros, y al volver me dijo:
«Vea, señor, a los dos oficiales que desde aquellas tapias le están
mirando\ldots{} Ahora le saludan con la mano. Son Navascués y Gracián.
Fui hacia ellos y ellos vinieron hacia mí, partiendo el camino, y
afectuosamente nos saludamos. El General les había encargado de la
instrucción y mando de las compañías de Voluntarios, tarea no floja,
pues eran gente revoltosa, temeraria, más fácil al heroísmo que a la
disciplina. Volviéndose de improviso hacia \emph{Sebo}, Gracián le
agarró de una oreja, diciéndole: «Ahora me vas a pagar todas las que me
has hecho, perillán. Pensabas que yo no te conocería con esa facha de
saltatumbas\ldots{} Ya no te suelto; te doy la filiación, quieras o no
quieras; te pongo en las manos una carabina, y como no seas valiente, te
fusilo por la espalda\ldots» «Señor---contestó \emph{Sebo} con mal
disimulado pánico,---no se empeñe en hacerme héroe, porque no lo soy. No
he nacido para eso\ldots Si quiere emplearme en el ejército libertador,
como es mi gusto, deme un cargo administrativo, sanitario o, si a mano
viene, de municiones de boca, que algo entiendo de esto\ldots» Y
Gracián: «Si no tienes ánimos para cargar el chopo, te haré mi capellán
castrense.

---Señor, no soy cura.

---Lo pareces, y es lo mismo\ldots{} No te me escapas ya. De los malos
ratos que me has hecho pasar dándome caza, voy a vengarme ahora,
tunante.» Y sin esperar a más razones de \emph{Sebo} ni mías, llamó a un
sargento y le entregó el nuevo \emph{voluntario}, con esta suave
recomendación: «Ea, cogedme a este gandul, que es un cura mal
disfrazado\ldots{} Ponédmele en el servicio de cartuchos, hasta que
llegue la ocasión de auxiliar a los moribundos\ldots{} Cuidado con él; y
si quiere escaparse, dadle veinticinco palos como primera providencia.
Desapareció \emph{Sebo} dolorido y rezongando, y siguió su marcha la
división por el camino polvoroso. Picaba el sol; los ánimos ardían.

Apenas entraron en Vicálvaro las tropas sublevadas, corrió la voz de que
estaban a la vista las del Gobierno. Expectación, toques de mando,
movimiento. Era una falsa alarma, que se repitió media hora más tarde,
cuando los soldados requerían sus alojamientos y olfateaban las
humeantes cocinas. Por fin, cerca de las tres, ya fue indudable que
venía el Ministro de la Guerra, general Blaser, con Lara, Capitán
General, y casi toda la guarnición de Madrid. Antes de que viéramos las
avanzadas, una bala de cañón, que casi tocó a las tapias del pueblo, fue
como el primer grito de guerra. Nube de polvo lejana anunció la
Caballería del Ejército que el convencionalismo histórico llamaba
\emph{leal}. Pronto vimos que la Artillería enemiga escogía posición
excelente en lo alto de un cerro, detrás de un arroyo. Entendiendo poco
de estrategia, pareciome que Blaser no pecaba de tonto. Lo mismo
pensaron los de acá, según después supe. Pero ya no podían rehuir el
combate en el terreno escogido por los de Madrid. Vi que avanzó el
batallón de la Escuela Militar, como en reconocimiento, y sobre ellos
vinieron con furia los caballos de Villaviciosa. La batalla estaba
empeñada. Híceme cargo del plan de ambos caudillos. El de allá ganaría
si desalojaba de la posición de Vicálvaro a los que bien puedo llamar
\emph{nuestros}. Ganarían los sublevados si conseguían tomar de frente
los cañones de Blaser.

Reconozco mi falta absoluta de espíritu bélico, y no me avergüenzo de
confesar que me siento incapaz de todo heroísmo en el terreno de las
armas. Como además no gusto de acudir a donde sé que mi persona no hace
ninguna falta, determiné situarme en lugar seguro, aunque en él no
pudiese ver en todo su desarrollo el que ha de ser histórico suceso. Me
interesan, sí, en grado sumo las consecuencias políticas o sociales de
este duelo marcial; pero las peripecias y lances del mismo no
despertaban en mí ninguna emoción, como no fuera la de piedad y lástima
por los que habían de morir. Desde las tapias más lejanas del pueblo,
por el Este, procuraba yo ver y enterarme, recorriendo con ávidos ojos
el campo de batalla. Entre nubes de polvo y humo vi las filas de
caballos, las filas de hombres, grupos contra grupos, y\ldots{} ¿lo diré
como lo siento? Yo no deseaba sino que acabaran pronto. ¿Diré también
que toda aquella porfía me pareció estúpida? Pues lo digo. Y al fin,
entre mis confusiones y mi hastío de tanta barbarie, surgía la pregunta
no contestada: «¿Y todo esto, para qué?» No sé qué habría yo pensado si
me viera ante un San Quintín; pero ante aquel combate, en cierto modo
casero, entre \emph{cuatro gatos}, como suele decirse, lucha por el
gobierno de un país siempre desgobernado, mi pensamiento no podía
elevarse a las alturas de la Historia trágica. Nada, nada: que acabaran
pronto, y se fueran a sus casas.

Dos horas corrieron, y no se veía ventaja en ninguna de las dos partes.
Se tiroteaban, se acuchillaban, y las ondulaciones de las masas
combatientes no determinaban ganancia ni pérdida de los trozos de suelo
en que reñían. En la salida del pueblo, por el camino de San Fernando,
donde busqué mi refugio, había multitud de viejas, que allí se guarecían
de las balas. Alguna de ellas me dijo que a la villa le venía bien
aquella guerra, porque la tropa siempre deja dinero, y otra se lamentó
de las muertes que \emph{habería}, no sin atenuar su pena con esta
consideración filosófica: «También hay que ver que es güena la guerra
cevil, porque en ella fenece toda la granujería de los pueblos.
Perdidos, vagos, ladrones: en tiempo de paz no hay quien vos mate. Salta
la guerra, y a la guerra os vais como las moscas a la miel. Sois
valientes, metéis el pecho de veras. Ahí morís todos, pestilencia.» Y un
vejete medio alelado y paralítico tomó así la palabra: «Esto que vedéis
no es guerra mesmamente y de por sí, sino rigolución\ldots{} Y quien diz
rigolución, diz dinero en Vicálvaro: la rigolución trai derribo de casas
viejas, de conventos y santularios; rompición de calles, de lo que viene
obra mucha de casas nuevas, y vender acá más yeso del que ora vendemos.
Ya vedéis la paradez del yeso. Pus como ganen los libres, tendréis en
Madril obra de casas, y aquí el quintal de yeso por las nubes.» No podía
yo enterarme bien de otras cosas que el vejete decía, porque en el sitio
donde estábamos se habían refugiado todos los perros del pueblo,
asustados de la batalla, y allí coreaban con sus ladridos el militar
estruendo.

También los mendigos de ambos sexos tenían allí su resguardo, y entre
éstos un ciego que, según contó, estuvo en los famosos sitios de
Zaragoza. «Aquéllas eran guerras por honra, señor---dijo revolviendo sus
ojos muertos,---y no estas comedias con tiros, por el mangoneo, y por
ver quién pone o quién no pone un par de principios después de los
garbanzos. Bien claro está que no hay cosa de por medio. España se va
volviendo muy comelona; los ricos traen cocineros de Francia\ldots{}
¡Comer bien, vivir bien! ¿Lujo grande, \emph{monises} pocos? Pues
revolución, para que el dinero que hoy está en tus manos venga a las
mías\ldots{} Yo he sido en Madrid cocinero de fondas y de alguna casa.
¡Veinticinco años cocinero, señor! Me arruiné por poner un
establecimiento en que daba de comer a lo que llaman \emph{precios
reducidos}. La maldita baratura y el no entender el negocio me dejaron
por puertas, y para acabar de arreglarlo, mis pobres ojos, del calor de
las hornillas día y noche, se quemaron, se quedaron ciegos\ldots{} No
veo las cosas, y veo las causas de estas marimorenas\ldots{} Todo es
cuestión de \emph{principios}\ldots{} de poner dos
\emph{principios}\ldots{} Yo ponía tres por dos pesetas, y ya se sabe lo
que me pasó. Las \emph{clases} no pueden comer dos principios sin hacer
una revolución cada pocos años para que los buenos sueldos pasen de unas
manos a otras manos\ldots{} Tenemos en Madrid el furor del buen vivir,
que viene de Francia, como las modas de sombreros\ldots{} tenemos el
furor de los \emph{dos principios}, de los chalecos de felpa y de
cachemir, de los pantalones \emph{patencur}, de las butacas con muelles,
de las alfombras de moqueta, de los jabones de Benjuí o de
\emph{terciopelo}, y de los féretros metálicos, que también en esto hay
furor\ldots{} Muchos furores y poco dinero, señor. ¿Poco dinero? Pues ya
se sabe: revolución al canto\ldots{} estas revoluciones de dentro de
casa\ldots{} el fogón, la despensa, el guardarropa, los
\emph{principios}\ldots{}

\hypertarget{xix}{%
\chapter{XIX}\label{xix}}

Con estas conversaciones, me distraía de la acción campal y no paraba
mientes en sus peripecias. Al recogido lugar donde yo estaba venían
noticias de que iban ganando los libertadores. Los zambombazos de la
Artillería eran menos frecuentes; hasta me parecieron más lejanos. No
fue menester, como en los tiempos de Josué, que se detuviera el sol en
su carrera para dar lugar a que los combatientes decidieran cuál se
llevaba la victoria\ldots{} El día, como de Junio, era largo, tan largo,
que no acababa nunca, y la victoria no parecía. Liberticidas y
libertadores se peleaban, sin darse ni quitarse posiciones, ni extremar
sus ataques. Creyérase que todo era una comedia marcial, representada
entre compadres con menos saña que ruido\ldots{} La página histórica me
resultaba poco interesante. Sin duda, su interés estaba en otro lugar y
ocasión. La verdadera página histórica con gravedad y trascendencia
vendría después, larga secuela de un hecho militar pequeño y de poca
sangre. Antes que empezase a declinar el día, cansado de su propia
largura, sentimos que menguaban los tiros. Al extremo del pueblo donde
yo estaba llegaron grupos de paisanos y soldados, sedientos, el polvo
pegado al sudor. Nos decían que llevaban ventaja; pero no traían en sus
rostros ni en sus palabras el júbilo de la victoria. Entraban en las
casas atropelladamente, buscando agua con que aplacar la sed. Al paso de
los hombres por los corrales, huían despavoridas las gallinas, que ya
requerían los palos de sus albergues. Con más agua que vino se
refrescaban los combatientes; algunos hablaban con poco miramiento de
los Generales libertadores, que no les habían mandado tomar a pecho
descubierto las piezas de artillería. Éstas se retiraban, según dijeron.
Blaser y su ejército leal se volvían a Madrid, donde seguramente darían
un parte proclamándose vencedores.

Mi criado y el cochero, a quienes di permiso para que se corrieran un
poco hacia las eras del pueblo, donde podrían ver algo de la función,
amparándose de las casas más próximas, volvieron a contarme lo que
habían visto, y de ello no copio más que esta referencia histórica: «¿No
sabe, señor? A don Telesforo \emph{Sebo} le han traído entre cuatro,
digo, entre dos, cogido por los brazos. Viene todo magullado de la
carrera de baqueta que le dieron antes de empezar la acción, y de los
pisotones de tropa y patadas de caballos que luego sufrió el pobre en lo
más recio de un ataque. Heridas de arma no tiene, sino cardenales y
mataduras que dan compasión. En nuestro alojamiento le han metido: allí
le están curando unas mujeres, y él echando de su boca maldiciones
contra los \emph{Blases} de allá y los \emph{Dulces} de acá.» Quise ver
y consolar al desdichado \emph{Sebo}; mas no pude hacerlo tan pronto
como quería, pues desde el punto en que recibí la noticia hasta mi
alojamiento era dificultoso el tránsito, por la muchedumbre de tropa y
paisanos que invadían las calles. En medio del tumulto tuve una grata
sorpresa: vi un rostro conocido, de persona que como yo trataba de
abrirse paso. Era Rodrigo Ansúrez, el aprendiz de hojalatero, que había
sido como el anunciador de las disposiciones del Destino, determinantes
de mi viaje a Vicálvaro\ldots{} Le cogí por un brazo. Díjome que su
maestro, el señor Albear, le había dado permiso para seguir los pasos
del general O'Donnell, el cual salió de la Travesía de la Ballesta en la
madrugada del 28. Con otro chico y el oficial de la hojalatería, ambos
de ideas muy liberales, se había ido Rodriguín a Torrejón y a Alcalá,
después a Vicálvaro. Había visto todo y podía contarlo. No disparó tiros
porque no le dieron carabina ni escopeta; pero ayudó lo que pudo,
llevando cartuchos para el fuego y agua para la sed. Agua y pólvora eran
lo más preciso.

«Oye una cosa, Rodrigo. Sin noticia ni dato alguno en que fundarme, sólo
por corazonadas, pienso que tu hermano Leoncio está en Vicálvaro.
¿Acierto? ¿Le has visto tú?

---Sí, señor\ldots{} le vi un momento y hablamos. Estaba con otros
paisanos que iban en el batallón de la Escuela Militar. Mi hermano, que
es gran tirador, llevaba una carabina muy maja, que no sé de dónde pudo
sacarla\ldots{} Le perdí de vista\ldots{} Como hay Dios, que Leoncio ha
matado a muchos \emph{Blases}.

---Haz por encontrarle, Rodriguillo. Deseo conocerle, preguntarle por su
mujer. ¿Tú qué crees? ¿Leoncio se habrá vuelto a Coslada?\ldots{} Si
estuvo en Alcalá, y de Alcalá se vino aquí con las tropas, ¿le habrá
seguido en esos trotes su mujer?

---Para mí que le ha seguido, señor, pues ella es también muy trotona.
No se cansa de correr, y como se quieren tanto, juntos están siempre.
Adonde va él va ella, y al revés, que es lo que se dice
\emph{viceversa}.

---Siempre juntos. Y si Leoncio ha estado en medio del fuego, ¿ella
también\ldots?

---Como en el fuego mismo no estaría \emph{Mita}; pero cerca sí, señor,
que valiente lo es hasta dejárselo de sobra\ldots{} Se habrá puesto
donde pudiera verle con su carabina maja tirando tiros y acertando
siempre, porque, créame el señor, no hay puntería como la de Leoncio.

---Pues si están en Vicálvaro, hemos de revolver el pueblo hasta
encontrarles, lo que no es fácil con tanto barullo de tropa y de
paisanos forasteros. Rodriguillo, cuenta con una recompensa magnífica,
un traje nuevo, o su importe si prefieres el dinero a la ropa; un reloj
si te gusta más que el traje; en fin, lo que quieras, si encuentras a
\emph{Mita} y \emph{Ley}. Ponte en movimiento ahora mismo; no descanses,
chiquillo. Mejor que ropa o reloj querrás\ldots{} no sé qué\ldots{}
algún antojo tuyo\ldots{} Dímelo.

---¿Para qué quiero yo reloj, si no me importa nada saber la hora? Y de
trajes, con lo que tengo me basta\ldots{} Más me gustaría otra cosa,
señor\ldots{}

---Pide por esa boca, hijo, y no seas corto de genio.

---Pues, señor, lo que quiero es un violín.

---¿Eres músico?

---Tengo afición. Cuando estuve en la fábrica de cuerdas, mi principal,
que es de la orquesta del Circo, me dio lecciones. Aprendí pronto. Yo me
habría pasado toda la noche rasca que te rasca; pero los vecinos se
quejaban del ruido que hacía, porque el violín que tengo canta como un
grillo, y en los graves parece un rabel de los que tocan los chicos en
Navidad.

---Pues nada, cuenta con un violín bueno, de aprendizaje. No será un
Stradivarius: ése lo tendrás cuando sepas, cuando seas un gran
concertista\ldots{} Pero no nos entretengamos. Ya estás echando a
correr. Queda hecho el trato\ldots{} Tráeme a \emph{Mita} y \emph{Ley} o
dime dónde están, y \emph{tú rascarás}.

Salió el chico presuroso a su encargo, y yo entré en mi alojamiento, la
casa de un yesero, con almacén, dos corrales, y arriba estancias
vivideras; mas era tal el cúmulo de gente militar y civil en todos los
patios y aposentos, que allí no podía uno revolverse, ni aun pensar en
comida y descanso. Lo mejor que hacer podía el que tuviera libertad, era
huir de Vicálvaro; pero la obligación que me impuse de buscar a los
salvajes me retenía en el pueblo, esperando el resultado de las
investigaciones del hojalatero violinista. Mientras éste llegaba, bajé a
consolar a \emph{Sebo}, que asistido de dos viejas piadosas, curanderas,
yacía sobre un montón de sacos de yeso, vacíos, entre sacos llenos. El
polvillo blanco, flotante en la cavidad del almacén, se fijaba en el
rostro y manos del magullado policía, dándole aspecto de cadáver o de
figurón con jalbegue. Sus muecas de dolor y sus plañideras voces sonaban
a bromas lúgubres de Carnaval. ¡Pobre \emph{Sebo!} Más que de los
dolores de sus mataduras, quejábase de la crueldad de Bartolomé Gracián,
que había dado permiso a sus tropas para zarandearle y jugar con él a la
pelota, dejando correr la fábula de que era cura disfrazado. Y no sentía
tanto el molimiento y los cardenales como el grave daño inferido a su
dignidad. Menos le dolerían sus huesos si se los hubieran roto, y sus
carnes si se las hubieran hecho picadillo, que le dolía el alma, del
escarnio sufrido y de la vergüenza de haberse visto en tan ruin vapuleo.
Y era lo peor que por ningún camino podría llegar a vengarse del don
Bartolomé, quien, al parecer, estaba en gran predicamento con Dulce y
con Ros de Olano. En mí confiaba para su delicada traslación a Madrid
manteniendo el disfraz, y ocultándose en mi casa hasta que se decidiera
si los perros se ponían el collar revolucionario o el absolutista. Y yo
tendría la caridad de sustentar a la familia mientras durase la
encerrona y llegara el nuevo destino. Éste correría de mi cuenta si
triunfaba O'Donnell, lo mismo que si ganaba Sartorius, que para uno y
otro perro tengo yo buenas aldabas. «Después de servir a Vuecencia con
tanta lealtad---me dijo haciendo pucheros y besuqueándome la mano,---no
tendrá Vuecencia entrañas si abandona a su fiel servidor.»

Mientras hablaba yo con \emph{Sebo} y miraba por su asistencia, metieron
en el almacén unos ocho heridos, algunos graves, y aquella atmósfera de
hospital en que respirábamos yeso se me hizo insufrible. Salí al
portalón, donde había corros de militares, y hablando con ellos adquirí
la certeza de que la batallita no les había satisfecho, por su equívoco
resultado. Ni en ellos cabía vanagloria, ni en Blaser tampoco. Verdad
que los de acá, como sublevados, podían contentarse con medio triunfo, o
con la modesta gloria de un combate a la defensiva, sin perder terreno,
mientras que los otros, como Gobierno constituido, quedaban muy
desairados con la media victoria. Su retirada hacia Madrid, sin
desorganizar y dispersar a los Generales sediciosos, era un verdadero
desastre.

Así me lo dijo Borrego, hombre de gran pesquis, añadiendo que la
situación está moralmente derrotada. Borrego, Martos, Ortiz de Pinedo, y
otros madrileños que habían venido de mirones, andaban a la busca de
comestibles, a cada hora más escasos. Los estómagos empezaban a renegar
del patriotismo; llevaban muy a mal que las cabezas, antes de prevenir
lo tocante al sostén de los cuerpos, se lanzaran a trastornar la
política y la sociedad. Compartí con los buenos amigos lo poco que a mí
me quedaba; allegamos algo más, todo ello fiambre, reseco y con sabor a
yeso, no sé si real o imaginario, y luego nos fuimos a la casa próxima,
donde moraban Ros de Olano y Echagüe. A éste no le vimos: había sido
llamado por O'Donnell. Ros comía tranquilamente con Buceta, el teniente
coronel Zalamero y el paisano don Ceferino España, sentados los cuatro
en derredor de una silla, por no haber mesa disponible. ¿Qué comían?
Lonjas de carne de cabra rebozadas con yeso, y almendras de Alcalá, que
parecían pedazos de escayola. Con su habitual gracejo, Ros nos dijo: «El
primer acto no ha sido malo. Como primero, no podía tener efectos
grandes, ¿verdad? Es una exposición hecha con arte sobrio. Sería necedad
acalorarse demasiado pronto.

---¿Y dónde pasará el segundo acto, mi General?

---¡Ah!, no lo sé\ldots{} yo no hago la Historia\ldots{} La que ven
ustedes de mi letra la escribo al dictado. Me dictan; yo escribo\ldots{}

---¿Se puede saber a dónde va mañana el ejército libertador?

---Si dejáramos los caballos de Dulce entregados a su instinto, creo yo
que nos llevarían a la querencia de sus cuarteles.

---¡A Madrid, al bulto!

---Puede que sea más práctico esperar a que el bulto se mueva para saber
lo que tiene dentro.

De las medias palabras del general Ros, colegimos que se esperaba un
movimiento en Madrid. El Gobierno, con toda su tropa disponible, tendría
que atender a sofocarlo: ésta era la ocasión de caer sobre la Villa y
Corte. Entre tanto que el formidable tumulto estallaba, los libertadores
pasearían militarmente por la zona circundante, \emph{pronunciando} a
los pueblos y reclutando patriotas. El fogoso Martos, que todo lo veía
conforme a los anhelos de su impaciente corazón de sectario, dijo, con
la solemnidad que ponía siempre en su limpia palabra, que las piedras de
Madrid se levantarían para contestar con barricadas a las insolencias
del \emph{polaquismo}. Las lenguas habían sido ya bastante elocuentes.
Lo que aún restaba por decir, lo dirían los guijarros\ldots{} y la
pólvora.

La idea y la esperanza de un alzamiento general en los Madriles eran
unánimes. Oficiales y paisanos que se acercaron a la mesa del General,
las expresaron ruidosamente, unos como chispazos de su inflamado
patriotismo, otros como consuelo de la deslucida función bélica de
aquella tarde. Pasando de corrillo en corillo y de casa en casa, oí,
entre mil comentarios del suceso y de sus consecuencias, una versión que
me pareció la más discreta y juiciosa, como salida del entendimiento de
Andrés Borrego, uno de nuestros más expertos catadores de
acontecimientos, y de los móviles y personas que los determinan. La
jornada de esta tarde---nos dijo a Pinedo y a mí,---desairada como
acción de guerra, es una obra maestra de sagacidad política y de
cuquería revolucionaria. Lanzar a las tropas al ataque con todo el brío
que ellas saben desplegar, habría sido dar a este movimiento, desde sus
primeros vagidos, un carácter rencoroso, sanguinario, como el de las
luchas por principios irreconciliables. Y aquí no hay ni puede haber a
la postre lucha de esa naturaleza. No hay cosas más conciliables que dos
porciones de nuestro ejército regular, la una frente a la otra. ¡Cómo
que en el fondo de estos movimientos y de estos choques entre
pronunciados y leales, hay siempre un compadrazgo disimulado con las
apariencias de antagonismo! Compadres son todos; no se tiran a
destruirse, y ninguno de ellos quiere echar sobre su contrario la
tristeza de una grave derrota. Con perfecta \emph{bonhommie} atacó
Blaser a Dulce, y éste y O'Donnell te devolvieron su cortés tiroteo.
¡Oh!, este irlandés sabe mucho, y no sólo es un buen guerrero, sino un
excelente estratega del corazón humano. En el curso de la acción he
visto yo los efectos de su malicia sajona, y de su admirable delicadeza
para efectuar el tacto de codos sin que nadie lo advierta, ni dejen de
enterarse los codos del contrario. Viendo la acción sin ver al caudillo,
yo le oía decir: «Compadre Blaser, no nos comprometamos derramando más
sangre de la que manda la etiqueta. El \emph{polaquismo} es cosa
perdida. En el reloj del Destino ha sonado mi hora, y yo y los que están
conmigo hemos de coger la sartén por el mango\ldots{} Amigos seremos
todos, aunque ahora el buen parecer pida que nos figuremos rivales. No
tardaremos en abrazarnos\ldots{} Nadie tema venganzas. Yo miraré por
unos y otros\ldots{} Somos el Ejército de un país sin fuerza de opinión,
de un país que un día nos pide Orden, otro día Libertad\ldots{} y lo que
nos pide\ldots{} tenemos que dárselo.»

\hypertarget{xx}{%
\chapter{XX}\label{xx}}

Hastiado al fin de las prolijas versiones de un mismo asunto, salí con
Zafrilla en busca de Rodrigo Ansúrez, que aún no había parecido por
nuestro alojamiento. Recorrimos varias calles, la plaza, parte del
camino real. En la plaza vi largas filas de caballos comiendo el pienso
que con solicitud fraternal les servían sus jinetes. Los nobles
animales, que habían trabajado todo el día pisando terrones y cardos
borriqueros, o algún cuerpo herido, muerto quizás, respirando tufo de
pólvora y enardeciéndose al incitante toque de clarines, mascaban
tranquilos su cebada, ajenos a la gloria militar. Observé que todos los
perros del pueblo, que durante la batalla se habían alejado del campo de
guerra expresando su terror con aullidos, se congregaban en derredor de
los bridones, mirándolos con respeto y envidia. Algunos, después de
mostrar su afecto a los generosos brutos de la guerra, salían a ladrar
por las calles próximas, como queriendo espantar a otros animales
enemigos que veían en forma vaporosa, o avisar la llegada de escuadrones
fantásticos, sólo vistos por ellos. Luego volvían junto a los caballos
nuestros, efectivos, y les hacían la tertulia, sentándose en postura de
esfinge en medio de los grupos de soldados y corceles, o escarbando
graciosamente la paja que a éstos se les caía.

Divagué por entre estos renglones de la página histórica, más
interesantes, a mi ver, que los renglones belicosos, y al volver de una
esquina me encontré al hojalatero, que corrió hacia mí gozoso,
diciéndome: «Señor, dos horas hace ya que le busco. Estuve en la yesería
tres veces\ldots»

---¿Y qué hay, chiquillo? ¿Dónde está mi gente?

---¡Ay, me parece que me quedo sin violín!\ldots{} no por culpa mía,
pues he revuelto todo este poblacho\ldots{} y\ldots{}

---¿No parecen?

---Se me ha secado la boca de tanto preguntar\ldots{} Por fin, señor, en
la puerta de la Iglesia me encontré a un yesero de Coslada: vivía pared
por medio de la casa donde se albergaban Leoncio y \emph{Mita}; le
llaman el tío \emph{Meas}\ldots{} Me dijo que con mi hermano había
venido su mujer, la cual, todo el tiempo que duró el tiroteo, estuvo en
casa de unas viejas que apodan las \emph{Cangrejas}, porque tienen en
San Fernando el negocio de mandar cangrejos a Madrid.

---Bueno, y te dijo\ldots{}

---Que, en cuanto se acabaron los tiros, \emph{Mita} fue en busca de
Leoncio y se le llevó\ldots{} Él no quería; pero ella\ldots{} ¡vaya, que
gasta un genio!\ldots{} Es la que manda.

---Se fueron, pues\ldots{} ¿a dónde?

---A San Fernando\ldots{} con las tías ésas, que, como le digo, allí
tienen gran casa\ldots{} Algunos días está toda llena de cangrejitos del
Jarama\ldots{}

---¡Todo por Dios\ldots! Pues no es cosa de que ahora vayamos a San
Fernando. Yo tengo que volverme a casa: hace tres días que no veo a mi
mujer y a mi hijo.

---Y no es lo peor que se hayan ido tan lejos, señor\ldots{} Van más
allá. Mañana, según me dijo el \emph{tío Meas}, pasan a Mejorada, donde
se establecerán, porque el negocio de Coslada parece ser que se les
torció.

---A mí sí que se me han torcido todos mis planes. Pero ya volveré a
ponerme en camino: iremos a Mejorada\ldots{}

---Yo también\ldots{} Si ahora no me gano el violín, ¿lo ganaré después?

\emph{---Tú rascarás}\ldots{} Tendrás un buen instrumento para
estudio\ldots{} Aplícate, y si eres realmente artista, yo te
protegeré\ldots{}

Desde aquel momento prevaleció en mi espíritu con poderosa fijeza la
idea de partir inmediatamente para Madrid. Di a Zafrilla las órdenes
necesarias para emprender la marcha. ¿Llevaríamos a \emph{Sebo}?
¿Llevaríamos a Rodriguillo Ansúrez? Este compañero de viaje no nos
traería ninguna dificultad; el otro tal vez sí. Resolvimos disfrazarle
de cura, y al efecto encargué a Zafrilla que buscara, o comprase si era
menester, las ropas necesarias para la mutación del travieso policía en
venerable sacerdote. Andando en estos preparativos, encontramos a
Navascués, el cual nos dijo que a la Corte se volvía con su inseparable
Gracián. Comprendí que la misión de estos pájaros en Madrid no era otra
que levantar al paisanaje y encender la lucha de barricadas. No se
necesitaba mucho para esto, y oyendo hablar al impetuoso Martos, se
adquiría el convencimiento de que Madrid sería un volcán en todo el día
próximo. Las dificultades que tuvimos para conseguir la ropa clerical de
\emph{Sebo} las resolvió fácilmente Bartolomé Gracián, que estaba en
buenas relaciones con el ama de un cura, frescachona, la cual facilitó
sotana y balandrán viejos, y un sombrero de teja, raído, tan largo como
un ataúd. No tenía \emph{Sebo} magulladuras en el rostro; los chichones
de la cabeza se tapaban con el sombrero, y el cuerpo, bien bizmado,
quedaba bajo la sotana holgadísima, pues el difunto era mayor.

Sobre las once nos dispusimos a salir. Gracián y Navascués, vestidos de
paisano, confiaban en su audacia para entrar en la Corte sin infundir
sospechas. Pensaban detenerse en Vallecas, donde tenía Gracián una amiga
diligente que a él y a Navascués proporcionaría, en caso de necesidad,
traje, burros y mercancía de panaderos para colarse sin ningún riesgo en
la capital. Martos, Pinedo, Borrego y otros se las arreglarían
fácilmente para el regreso. Llegada la hora de partir, y abreviadas las
despedidas, metí en el coche al maltrecho Telesforo, convertido en
clérigo campestre; subieron al pescante Zafrilla y el hojalatero, y
deseando yo disfrutar a pie de la plácida noche y de la conversación
grata de Navascués y Gracián, me fui andando con ellos detrás del
vehículo, a regular distancia, por el camino de Vallecas. Innumerables
perros salieron a decirnos adiós, ladrando, y enseñándonos los dientes;
se retiraban rezongando; volvían con más furiosos ladridos; acudían
otros de casas lejanas. Navascués, que se preciaba de entender el léxico
perruno, les dirigió la palabra amenazándoles con su garrote:
«Caballeros, que no somos gitanos ni frailes mendicantes. Retírense en
buen orden, y váyanse a cuidar las casas del lugar.» Gracián, también
entendido en el lenguaje canino, dijo que todo el alboroto que hacen los
perros al ver pasar coches y trajinantes, significa que desean saber el
punto a que éstos se dirigen. Créense investidos de facultades para el
reconocimiento de pasaportes y para la vigilancia de caminos, y sus
ladridos, que no son más que el cumplimiento de un deber, cesan cuando
se responde a la interrogación que expresan. «Señores perros---les dijo
Bartolomé mostrándoles el Palo, después una pistola,---sepan que vamos a
Vallecas\ldots{} ¡a Vallecas! No sé decirlo de otro modo: ya tenían
ustedes tiempo de aprender castellano\ldots{} Conque ya saben\ldots{} a
Vallecas vamos. Si no se dan por satisfechos, lo diré con la estaca; y
si la estaca no hablara con bastante claridad, les pegaré un tiro.» Oído
esto, los perros se fueron retirando por escalones. De hocico al pueblo,
todavía rezongaban con mugido displicente.

Diré que el tal Gracián me encantaba por su donosura, por la fatuidad de
buen gusto con que se había atribuido un papel constantemente activo en
la comedia o drama de la vida. No fue menester estímulo de mi parte para
que su confianza se abriera y su verbosidad se desbordara. Es de estos
que entregan todo su interior, sin reservar ninguna porción de lo malo
ni de lo bueno, tan ineptos para la hipocresía como para la modestia. La
conversación que yo entablé sobre el tema de la guerra y de la política,
fue derivando, por los giros que le daba el sensualismo de Gracián,
hasta llegar al tema de mujeres. La vida de aquel libertino era
manantial inagotable de asuntos para tal conversación. Oyéndole contar
alguna de sus aventuras, acometida con harta impavidez y cierta
convicción profesional, vi reproducida en él la figura del burlador de
antaño, a un tiempo heroico y cínico. La degeneración del tipo es
evidente, como lo es la de las víctimas, más fáciles hoy a la seducción.
Sobre este punto me permití opinar que en la moral no tenemos progreso.
Hay, sí, más pudor de lenguaje, y escrúpulos de palabra que antes no se
conocían; pero, con todo esto, los baluartes que hoy guardan la virtud
femenina son de estructura más endeble.

Consiste la presente relajación, según indicó Gracián, en que apenas hay
ya quien crea en el Infierno, y las mujeres que aún profesan este dogma
terrible, lo han reformado en su pensamiento, estableciendo un Infierno
sin infinito y con salidas al mundo de los vivos. Mi parecer es que la
sociedad actual, con la facilidad de relaciones entre los sexos y la
mayor licencia en las costumbres, permite a los galanes triunfos
baratos, que no exigen ni grande agudeza, ni arranques de valor
temerario. De aquí resulta que el ejemplar, el tipo de burlador más
común en estos tiempos, es de un prosaísmo evidente, agravado por los
toques de sensiblería fúnebre y de languidez mocosa. Hay también tipos
de varonil desvergüenza que sostienen la tradición mejor que los galanes
lánguidos, y en los pueblos tenemos el tenorio cerril, que no deja mal
puesto el pabellón de la galantería ilegal. A propósito de esto, hizo
Gracián una observación que sintetiza su gracioso cinismo. Dijo que los
tenorios rústicos prestan un gran servicio a la sociedad contemporánea,
porque ellos contribuyen en gran parte a la producción de amas de cría y
al fomento de niños de madres pudientes. El mal y el bien andan
enlazados en el mundo, y a cada instante vemos que algún trozo del
edificio de las virtudes sociales se caería si no estuviera apuntalado
por un vicio. ¡Qué sería de la infancia rica sin tanto menoscabo y
deshonor de muchachas pobres! Y si las criaturas ganan al cambiar el
esquilmado pecho de sus madres por el exuberante de las nodrizas,
también éstas salen gananciosas, porque se desasnan, se civilizan, y al
concluir llevan al pueblo sus ahorros, y encuentran un labrador honrado
que se casa con ellas\ldots{}

Apurando el tema con sofistería inagotable, llegó a sostener Gracián que
las aventuras ilegales de amor son manantial de poesía. El mundo se
volverá enteramente prosaico y la vida humana totalmente estúpida, si no
le prestan su encanto la turbación de matrimonios y el desconcierto de
los hogares, donde toda monotonía y toda insulsez tienen su asiento. Y a
tales ventajas deben agregarse las de preparar a la sociedad para las
revoluciones, que vienen a ser como una limpia general y mudanza de
aires, ambas cosas muy necesarias en la vida de los pueblos. Los amores
ilegítimos desatan lazos, aflojan vínculos. La volubilidad y el capricho
de la mujer extiende por toda la sociedad un cierto espíritu de rebeldía
que es el principal elemento de las alteraciones políticas. Sin darse
cuenta de ello, los hombres, sean burlados o por burlar, se ven
arrastrados a este remolino, que acaba por conmover los cimientos del
Estado. Las mujeres sienten, los hombres ejecutan. El pecado turba las
conciencias, y éstas tratan de aplacarse buscando la alteración de la
ley, por la cual es pecado lo que no debiera serlo. El deseo de alterar
la ley trae las agitaciones públicas, que son tentativas, ensayos,
cambios de postura; y aunque pasado el trastorno vuelven las cosas a su
estado natural, y la ley sigue imperando y jorobándonos a todos, ello es
que se quebranta con tantas sacudidas, como se resquebraja el terreno en
que se suceden los terremotos.

Esta singular teoría de que los pecados mujeriles abren camino a las
revoluciones, y de que éstas resultan siempre fecundas, no podía ser
admitida sino como una forma de humorismo para pasar el rato, como quien
dice. Pero él a sus paradojas se aferraba; con ellas se metió en el
terreno político, y explicome su fervor revolucionario como un estado
fisiológico contra el cual su voluntad nada puede. La regularidad y
permanencia de las instituciones se representa en su ánimo como una
enfermedad. La paz pública es como una parálisis. Él se subleva por
instinto de conservación o de salud, sintiendo en sí una parte de la
dolencia que a toda la Nación afecta. Es un miembro, un pedazo de carne
y nervios, partícipe del general dolor. Romper la disciplina es lo mismo
que medicinarse, o por lo menos hacer ejercicio, con el fin de buscar la
salud en la actividad muscular y en la fluidez sanguínea. La Ordenanza,
la Constitución vigente y sus predecesoras, son síntomas terribles de
una lesión honda que ha de traer la muerte. En todas las leyes
establecidas hemos de ver formas del dolor, de la congestión, de la
fiebre. Sus efectos en la vida equivalen a tumores, úlceras, sarna,
postemas, calambres y demás lacerias, contra las cuales hay que aplicar,
no sólo el movimiento, sino el fuego y las sangrías.

Todo esto, y lo que omito, lo decía Gracián dando suelta a la caudalosa
vena de su ingenio. Sus acompañantes reíamos a veces, o dábamos nuestro
asentimiento por el regocijo que nos causaba. Es, en verdad, un
admirable hablador. Posee la elocuencia de los disparates, y el arte de
entretener al oyente con graciosos absurdos, expresados en el tono de
una profunda convicción\ldots{} El camino se nos hizo corto con estas
charlas, y por mi parte no sentía cansancio cuando divisamos las
primeras casas de Vallecas. Los perros de esta villa salieron a
recibirnos en cuanto nos olfatearon, y con ladridos regañones nos
preguntaban de dónde veníamos y qué demonios íbamos a buscar allí. De
Vicálvaro---dijo Gracián,---y no seáis impertinentes. De Vicálvaro, y no
alborotar: llevo cargadas las pistolas. Tras de los perros vinieron
hombres, preguntándonos por la acción de aquella tarde, y si O'Donnell
iba ya sobre Madrid. Respondió Gracián con informes totalmente opuestos
a la verdad, sacados de su cabeza, y anunció que en Vallecas se había de
librar el próximo día la más tremenda de las batallas. En esto, nos
llevó a una de las casas que están a la entrada del pueblo, un poco
apartadas del camino, y antes de que llegáramos a ella vimos luz en las
habitaciones y oímos musiquilla de murga, violines mal rascados,
clarinete y un trombón. Pronto supe que se habían celebrado los días de
la dueña de la casa, y que llegábamos a los últimos pataleos y ronquidos
de la bulliciosa fiesta. Entramos, y lo primero que me eché a la cara
fue una mujerona de buen ver, alta de pechos, la tez morena, los ojos
fulgurantes, de una categoría mixta, pues si por el continente y la
finura del rostro parecía señora noble, su habla y modales denunciaban
la mujer del pueblo. Era una transición o producto híbrido de esta
sociedad infiel al principio de castas. Recibionos afablemente, y a
Gracián con mayor cariño y confianza. Luego me invitó a descansar,
ofreciéndome un dormitorio para esperar el día. No accedí, pues deseaba
continuar mi viaje sin demora. La señora puso término a la fiesta;
despidió a los pocos convidados que allí quedaban; dio licencia a los
músicos, y a mí las buenas noches, agarrando por un brazo a Gracián, el
cual es dueño del corazón de aquella tarasca, según me dijeron los
músicos momentos después, añadiendo que a la señora se la conoce por
\emph{La Panadera} y que es viuda y rica. Para más pormenores, Gracián
la engaña con una sobrina de ella, pobre, habitante en el mismo pueblo,
y a las dos con una casada residente en Canillas. En la casa de \emph{La
Panadera} se quedó Navascués, inseparable amigo del otro peine, y su
\emph{mono de imitación}, con tan mala sombra, que cuantas aventuras
intenta son bajas parodias de las del maestro.

En la calle ya (más propio será decir en el campo), dispuesto a
proseguir el viaje, se me acercaron los pobres murguistas suplicándome
que les trajese a Madrid en mi coche, pues se hallaban rendidos de
tantas tocatas y caminatas: habían tenido boda en Mejorada, bautizo en
Loeches, y en Perales festividad del patrono San Pedro Apóstol. Ya no
podían con sus almas, ni con sus instrumentos. Accedí gustoso a
transportarles, y corrieron al parador, donde tenían sus livianos bultos
de ropa, y paquetes o envoltorios pesados, pues casi todo su trabajo
musical lo cobraban en especie. Mi protegido el hojalatero pidió a uno
de ellos que le dejase su violín, mientras iban a recoger la
impedimenta. Accedió el murguista. Cogió el chico la carraca, se la echó
al pescuezo, tendió el arco sobre las tripas vibrantes, y allí fue el
sacar sonidos largos, dulces, elocuentes, que rasgaban el silencio en la
noche plácida\ldots{}

\hypertarget{xxi}{%
\chapter{XXI}\label{xxi}}

Estábamos en un terreno polvoroso que no sé si era camino, plaza, o
ejido. Sentado yo en un trozo de construcción de adobes, que lo mismo
podía ser resto de un edificio que principio de él, a mi espalda veía
las chozas que se arman en las eras para guardar la mies en gavillas;
frente a mí, casas mezquinas agrupadas, como si quisieran formar calles;
a mi derecha, la de \emph{La Panadera}, grande y con letreros, en que se
distinguían las palabras \emph{Salvados}, \emph{Harinas}\ldots{} Ningún
árbol vivo alcanzaban a ver mis ojos; había, sí, frente a mí uno muerto,
tronco y ramas en completa desnudez esquelética. Los tejados y el árbol
se destacaban con trazo vigoroso sobre un cielo limpio, sin ninguna nube
en su concavidad majestuosa, alumbrado por una luna menguante, tuerta,
con un solo carrillo y un ojo solo, bastante luminosa para que
palidecieran las estrellas, quedando las de primera magnitud muy
rebajadas de categoría. Frente a mí, de espaldas a mí, sentado en una
piedra, estaba el hojalatero encorvado sobre su violín, pasándole el
arco, ahora con suavidad, ahora con brío\ldots{} Cuando rozaba en la
prima, el arco apuntaba al cielo con su contera, y a la tierra cuando
rozaba en la cuarta. Tocó Rodrigo aires del \emph{Pirata}, de
\emph{Beatrice di Tenda}, de \emph{Maria di Rudenz}, de otras óperas en
boga. Sin duda por el estado de mi espíritu, más que por la destreza del
violinista, la emoción que sentí fue muy honda, de esas que remueven lo
más quieto y despiertan lo más dormido del alma. Y alguna parte tendría
en esta emoción el mérito del artista: cuanto más yo le oía, más me
admiraba la perfecta afinación, el juego elocuente del arco, su fuerza,
su delicadeza, según los pasajes y diseños que atacaba. Llegué a
sentirme encantado de aquella música, deseando que durase todo el resto
de la noche, y que ésta fuese muy larga. Tocaba el muchacho con devoción
y fe, poniendo la mitad de su alma en los dedos de su mano izquierda, y
en la derecha la otra mitad. Quería serme grato, y mostrarme su afecto
en el lenguaje que mejor conocía\ldots{} Con la palabra no habría podido
expresar ni aún mínima parte de lo que sentía, gratitud, esperanza. De
mí esperaba medios para ser un artista eminente, de universal renombre.

En lo más solemne de la serenata, cuando yo me hallaba en pleno éxtasis,
oí que las mulas del coche, situado como a veinte pasos de distancia,
detrás de mí, redoblaban las manifestaciones de su inquietud, pateando
con más fuerza y sacudiendo las colleras, que arrojaban al aire la
tintinabulación de sus cascabeles, como un espolvoreo de notas
metálicas. Este ruidillo no turbaba la dulce melopea del violín, sino,
antes bien, la exornaba con un comentario gracioso, de cómica
elegancia\ldots{} Volví mis ojos hacia el coche, y vi que por la
portezuela asomaba la cabeza de \emph{Sebo}, como un mascarón lívido,
que lo mismo podía ser de clerizonte que de rufián. La bella música le
atraía, le embelesaba, como a mí. Aprobaba con un movimiento expresivo
de la cabeza, y luego lanzó esta frase, rasgo de poeta y de crítico:
«Anda, hijo, no sabía yo que fueras tan buen profesor\ldots{} Toca,
toca: las estrellas te oyen.»

Cortaron bruscamente los músicos la bella serenata, presentándose con
ruidosa premura cargados de sus paquetes. Calló el violín maravilloso, y
los viajeros se ocuparon de colocar sus bultos en el pescante o dentro
del coche. Subimos: Ansúrez pidió al dueño del instrumento nuevo permiso
para seguir tocando por el camino, y obtenido lo que deseaba, se
encaramó en el pescante. Entramos los demás, acomodándonos en aquella
estrechez como pudimos, y las impacientes mulas no aguardaron la
intimación del cochero para emprender la marcha. Por el camino, el
hojalatero, sentado al borde del pescante junto a Zafrilla, con una
pierna colgando, tocaba todo lo que sabía, himnos patrióticos, mazurcas
y valses, tiernas melodías de Bellini y Donizetti. En el curso del viaje
hasta las inmediaciones de Madrid, no dejé de sentirme embelesado con la
música, adormeciéndome en un vago ensueño. Las notas patéticas del
violín flotaban sobre el pesado ruido del coche, como una cabellera
dorada y vagarosa que el viento agita sin desprenderla del cráneo en que
se arraiga. La cabellera se daba al viento como una idealidad que vuela,
sin abandonar la realidad que la sustenta y la produce. Las propias
mulas parecían adaptar su paso al ritmo de las tocatas\ldots{} Yo me
adormecí\ldots{} Todo era música\ldots{} música también el son continuo
de los cascabeles y los ronquidos de \emph{Sebo}.

\emph{6 de Julio}.---Dos días con parte de sus noches tardé en contar a
María Ignacia lo que había visto en mi excursión, desviada de su
primordial objeto por el Acaso, más poderoso que mi voluntad. No estaba
conforme mi costilla con el quiebro que di a mis planes, y sentía que no
hubiese persistido en la busca y captura de \emph{Mita} y \emph{Ley}.
Propuse nueva salida; pero Ignacia no aprobaba la repetición del viaje,
sin duda por notar que del primero había vuelto yo muy melancólico, con
tendencias a dormirme o amodorrarme encima de una sola idea. ¿Se
reproducían en mí las tristezas o saudades que años atrás alteraron
gravemente mi salud? ¿Volveré a sentir mi pensamiento balanceándose
sobre aquella línea, sobre aquel lindero que separa la razón de la
sinrazón?\ldots{} Mi mujer me interroga con cierta prolijidad, al modo
facultativo, que me pone en cuidado. Yo, sondeando cuidadosamente mi
interior, le respondo que lo que ahora siento es\ldots{} ganas de
vomitar toda la historia contemporánea que tengo en el cuerpo, y que se
me ha indigestado formando un bolo: necesito expulsar este bolo. María
Ignacia se ríe; yo me explico mejor diciéndole que mis ilusiones de ver
a España en camino de su grandeza y bienestar han caído y son llevadas
del viento. No espero nada; no creo en nada\ldots{} Me hastía el
recuerdo de la batalleja que vi en Vicálvaro. Me figuro a los niños de
Clío jugando con soldaditos de plomo\ldots{} En cuanto a las ambiciones
que han movido esta trifulca las considero semejantes por su altura
moral a las ambiciones de mi amigo \emph{Sebo}\ldots{} La página
histórica tras la cual corrí, resúltame ahora como pliego de aleluyas o
romance de ciego. ¿Será que mi mente ha caído en la dolencia de
remontarse y picar muy alto, o que los hechos y los hombres son por sí
sobradamente rastreros y miserables?

A cuantas noticias vienen a mí de sucesos ocurridos en Madrid, o en el
camino que llevan los que se llamaron libertadores, les doy carpetazo.
¡Quién pudiera disponer del olvido, como de un pozo en el cual se
arrojara todo lo que no se quiere saber! Olvidar las cosas ingratas en
el mismo punto en que suceden, sería la mejor reparación de las
sofoquinas a que diariamente está sujeta nuestra alma. Pero el maldito
tiempo no permite al olvido andar solo, y hemos de conformarnos con la
insufrible lentitud del presente, y su resistencia a convertirse en
pasado\ldots{}

¿Me pide la Posteridad referencias históricas? Pues allá va una que
juzgo en extremo interesante. Sabed que el gran \emph{Sebo} se aposenta
en mi casa, confundido con mi servidumbre, conservando su figurada
estampa de clérigo. Por las noches, con el aditamento de anteojos verdes
y de un raído traje, sale y visita sin recelo a su familia. El
sostenimiento de ésta corre ahora de mi cuenta, y ello ha de ser hasta
que Clío nos depare la total ruina del \emph{polaquismo} y el triunfo de
los de Vicálvaro. Yo le rezo devotamente a Santa Clío, pidiéndole que
apresure este negocio, porque pesa sobre mí como un mundo el hambriento
familión de mi huésped.

\emph{10 de Julio}.---Sabed, ¡oh generaciones venideras!, que los
sublevados, ni victoriosos ni vencidos en Vicálvaro, tomaron el camino
de Aranjuez. Tratan de despertar a su paso a la Nación dormida. Diríase
que la Nación abre los ojos, se despereza, vuélvese del otro lado y
recobra la plácida quietud del sueño. En Madrid, el Gobierno echa
furibundas roncas subido a la \emph{Gaceta}, y continúa alimentándose
con niños crudos, que le dan malas digestiones. A los sublevados da el
nombre de \emph{traidores} y otros no menos infamantes, y en sendos
decretos exonera y pone en la picota a Dulce, O'Donnell, Messina, Ros de
Olano, y a los \emph{ilusos} que van con ellos\ldots{} Noto en el pueblo
de Madrid cierta depresión de la fiebre revolucionaria. En los cafés
sigue la gente despotricando contra Sartorius, y denominando simplemente
\emph{ladrones, turba de lacayos y rufianes}, a los personajes más
empingorotados de la situación. Todo esto ha venido a representárseme
como vocinglería de gitanos. La flojedad del acto militar que lleva el
nombre de Vicálvaro ha producido el enfriamiento de la temperatura
política. Las revoluciones, como las tiranías, acaban en ociosas
algaradas cuando no son robustecidas por la fuerza.

Más importancia que estas manifestaciones de la vida pública tenía en mi
ánimo lo que a contar voy, con permiso de la señora Posteridad. Pues
sepan que compré al hojalatero un violín excelente para estudio, y que
el pobre chico no halló mejor manera de mostrarme su gratitud que
ofrecerse a darnos en casa cuantos conciertos quisiéramos oír. A mi
mujer le encantaba la música, y le hacía gracia el fervoroso entusiasmo
con que Rodrigo tocaba en nuestra presencia. Afectado yo de tristezas
grises, me sentía en situación semejante a la de Felipe V, buscando su
consuelo en el arte de Farinelli. Para distraerme con más eficacia, el
buen chico estudiaba cada noche nuevas piezas, y de esta variedad
resultaba para Ignacia y para mí mayor deleite. Tanto ha llegado a
interesarnos este incipiente artista, que hemos decidido ponerle un buen
maestro, el mejor que hoy tenemos en Madrid. Bajo la férula del anciano
don Juan Díez, adquirirá seguramente la perfección de estilo que ha de
ser el mejor adorno de sus prodigiosas facultades\ldots{} «Y a todas
éstas, ni por el hojalatero violinista, ni por otro conducto, nos llegan
noticias de \emph{Mita} y \emph{Ley}. ¿Cómo no escribe la salvaje? ¿Será
menester que salgamos por segunda vez en su busca? A repetir la suerte
me inclino yo; pero mi mujer no me deja: quiere retenerme; confía en que
el reposo y las emociones dulces han de serme más provechosas que el
traqueteo de un viaje y las caminatas en pos de lo desconocido.

\emph{15 de Julio}.---Despierto una mañana con la idea de que\ldots{}
Vamos, creo haber descubierto el verdadero sentido y fundamento de estas
mis nuevas murrias, parecidas, si no iguales, a las de antaño. Fue muy
consoladora para mí la convicción de que mi dolencia no es más que
\emph{Ansia de belleza}. Parece que no, y ello es que todo enfermo
siente algo parecido al alivio cuando escucha de boca de su médico el
nombre del dolor o molestia que sufre. En las alteraciones nerviosas
principalmente, cualquier denominación técnica suele hacer veces de
calmante. \emph{¡Ansia de belleza}, que por el reverso es el desdén y
hastío de las vulgares cosas que me rodean! Anhelo lo grande y hermoso,
la poesía de los hechos humanos, así del orden privado como del público.
El recuerdo de una batalla \emph{de aficionados} en campo casero, me
lleva al ardiente afán de presenciar un Austerlitz, o algo semejante; y
para que se me quite el mal gusto de boca que me dejan estas peleas por
un puñado de garbanzos, miro hacia las ambiciones de un César, de un
Cromwell, de un Bonaparte.

Desdeño las tintas medias, la clase media, el justo medio y hasta la
moral media, ese punto de transacción o componenda entre lo bueno y lo
malo. No me gusta nada que sea medio; me seduce más lo entero. Váyase
mucho con Dios el buen sentido, y tráiganme la sinrazón, el desenfreno
de la inteligencia y de la voluntad. ¡Bonito se va poniendo el mundo
desde que nos ha entrado esta bárbara invasión de lo práctico, desde que
los hombres de pro se consagraron a desterrar la exageración, y a
recortar y reducir a estrechas medidas los alientos humanos! Hemos
vuelto del revés la fabulilla del asno vestido con piel de león, y
ponemos todo nuestro empeño en que los leones se vistan de
borricos\ldots{} Hablo de esto con mi mujer, y ella me exhorta
blandamente a tomar las cosas como son, a combatir el \emph{Ansia de
belleza}, aspiración insana, y a conformarme con la única realidad
accesible a nuestros deseos, el gracioso engaño de la fealdad pintadita
y retocada, que nos dice: «soy bella.» Creámosla y admirémosla sin
discutirla.

\emph{16 de Julio}.---¿Qué pasa en Madrid? Oigo ruido, pisadas de un
pueblo que ha roto la silenciosa quietud en que vivía, y se agita
buscando armas y posiciones para combatir. Perdóneme mi dulce amiga la
Posteridad: con esto de mis murrias, que a nadie interesan, he olvidado
contar las pequeñeces del vivir público, que usurpan un puesto en las
filas históricas. Allá voy. Los Generales que a sí propios se denominan
\emph{libertadores}, y que el Gobierno llama \emph{facciosos}, se fueron
al Real Sitio de Aranjuez, y de allí enfilaron las planicies manchegas,
adelante siempre, reclutando mozos, requisando caballerías, y
requiriendo amorosamente cuantos fondos guardaban las administraciones
subalternas de los pueblos\ldots{} Tras ellos han ido Blaser y
Vistahermosa, despacito, persiguiéndoles sin querer alcanzarles, a la
distancia que marca el compadrazgo fraternal, norma constante de toda
esta gente.

Me cuenta el gran \emph{Sebo} que en Madrid quedó un Comité
revolucionario, del cual son alma Cánovas del Castillo, Fernández de los
Ríos, y no sé si Tassara o Vega Armijo. Ello es que los dos primeros
cogieron muy calladitos el camino de la Mancha hasta dar con O'Donnell,
y charlaron con él largo y tendido, diciendo que Madrid no se levanta y
los \emph{polacos} no se rinden, porque las promesas de los
\emph{libertadores}, harto vagas, hablan poco a la inteligencia del
país, nada a su corazón. No se hacen las revoluciones por las ideas
puras, sino por los sentimientos, revestidos del ropaje de las ideas.
Los \emph{libertadores} ofrecen cosas muy buenas, de esas que forman el
tejido artificioso de todo programa político y revolucionario.
Veámoslas: \emph{Pureza del régimen representativo, Mejora de la
legislación electoral y de imprenta, Rebaja de los impuestos}. ¿Te
parece poco, infeliz Nación; te parece vano, retórica de quincalla, de
la de a dos cuartos la pieza? Pues allá va otra cosa: \emph{¡Moralidad!}
Esto sí que es bonito. \emph{¡Moralidad!} Vamos a tener en el Gobierno
esa preciosa virtud. Y por si es poco, ahí va también otra joya
incomparable: \emph{¡Descentralización!} ¿Qué tal? Descentralización y
todo, y para completar tanta ventura, también os damos \emph{Economías}.
No queremos pecar de cortos en el ofrecer. Economizaremos, moralizaremos
y descentralizaremos\ldots{} ¿Qué?, ¿no nos creen?

En efecto: el pueblo no da valor ninguno a tales pamplinas, y alza los
hombros viendo a unos pasar hacia la Mancha, viendo al Gobierno inmóvil
en su inmoralidad, en su despilfarro y en su centralismo. Cánovas y
Fernández de los Ríos, bien pulsada la opinión en Madrid, ven clara la
vacuidad de ese programa; corren a la Mancha, y en los polvorosos
caminos encuentran a O'Donnell. Paréceme que les oigo: «Mi General, dé
por abortada su revolucioncita si no cambia esas monsergas por otras, o
no les añade un tópico resonante, de esos que hablan, más que al
entendimiento, a la fantasía, o si se quiere, a la vanidad del pueblo
español; algo que sea o que parezca ser garantía de las libertades
públicas, y aparato político de pura figuración externa, y de ruido y
colorines\ldots» Paréceme que veo al irlandés rebelde al convencimiento.
No cede; se aferra con terquedad al plan primero de su revolución,
exenta de toda concomitancia con las muchedumbres; revolución cómoda,
casera, cambio de nombres y de personas nada más\ldots{} Es como un
calzado viejo, holgadito, con el cual andará el hombre por casa sin
ninguna molestia. Nada de calzado nuevo, que aprieta y chilla\ldots{}
Pero tanto le dicen sus amigos, y tanto machacan, que al fin llevan a su
ánimo la convicción. No concede que sea bueno lo que le proponen; pero
reconoce que, de no admitirlo, él y sus compañeros y su ejército corren
a una triste desbandada y al amargo destierro\ldots{} No había más
remedio que ceder. O'Donnell cede; los de Madrid redactan un nuevo
programa, en el cual, después de estampar las consabidas monsergas de
\emph{Moralidad}, \emph{Descentralización}, etc., añaden otras
sugestivas monsergas. En el programa debieron poner esta frase:
«Caballeros, se nos había olvidado lo principal, lo más importante.
Perdonad el error, que en este pueblo de Manzanares subsanamos,
escribiendo en nuestra bandera el mágico lema de \emph{Milicia
Nacional.»}

\hypertarget{xxii}{%
\chapter{XXII}\label{xxii}}

\emph{17 de Julio}.---Volvieron a Madrid los mensajeros con el reformado
papelito, y apenas lo dieron a conocer, se sintió en esta villa como una
trepidación del suelo, y lo mismo fue publicarlo que volverse loco todo
el vecindario\ldots{} Las dos palabras añadidas tuvieron el efecto
explosivo que hacía falta, y que en vano se pidió a los otros términos
del programa. \emph{Milicia Nacional} es una bomba cargada de pólvora.
Hablar de \emph{Moralidad}, de \emph{Descentralización} y
\emph{Economías} era cargar la bomba con miga de pan. Para mayor
fascinación del público, el Manifiesto declara que la popular
institución se planteará \emph{sobre sólidas bases}. ¿Qué tal?
\emph{Milicia} ya es mucho; \emph{sólidas bases}, ¡ah!, ya son miel
sobre hojuelas\ldots{}

¿Pero qué escucho? Ahí es nada. ¡Qué se sublevan o pronuncian Barcelona
y Valladolid! Y en esta Corte de las Españas parece que todos se vuelven
epilépticos. Salgo a dar una vuelta, y noto en las caras de los
transeúntes un júbilo extraño; en los cuerpos, síntomas claros del mal
de San Vito. La gente se agrupa sin darse cuenta de ello. En cuanto dos
secretean, agréganse cuantos van pasando. Donde hay tres personas, antes
que pasen cinco minutos hay treinta. En la Puerta del Sol se estacionan
los grupos, mirando al Principal. Es la expectación, la ansiedad pública
ante el rostro ceñudo del Destino. ¿Qué pasará, qué resoluciones
expresan o anuncian los ojos inmóviles y la torva seriedad de la
esfinge? De tanto mirar al Principal, llegamos a ver en las ventanas y
rejas facciones que algo dicen\ldots{} que algo callan.

Sigo mi paseo; entro en la librería de Monier, encuentro amigos que me
llevan a divagar por la Carrera de San Jerónimo y calle del Príncipe, de
grupo en grupo. El tránsito es difícil\ldots{} ¿Qué pasa? Ahí es nada lo
del ojo\ldots{} que ha caído Sartorius. Estatua de barro, se ha deshecho
en pedazos mil al estrellarse contra el suelo. Refiere el batacazo un
exaltado progresista, que acumula sobre la cabeza del Conde los epítetos
más infamantes. No puedo contenerme: salgo a la defensa del favorito que
ha dejado de serlo. Mi defensa es tomada a chacota, y da margen a mayor
impiedad y a burlas más crueles. El que con más dulzura le trata llámale
\emph{Monipodio}. Entre unos y otros, verbalmente, le escarnecen y le
escupen. Luego le arrastran por las calles, y no encuentran muladar
bastante inmundo en que arrojarle.

En otro grupo contaban que a Sartorius se le ha despedido como a un
criado. Llegó a Palacio y no le dejaron pasar a las habitaciones reales.
Esto no me parece verosímil. Sea como fuere, ello es que no hay
Gobierno, que la \emph{infame pandilla polaca} tiene ya su merecido. No
falta un furibundo sectario que, al oír lo que se cuenta de la desgracia
del Conde, exclama en dramático tono: «Esto no es cuestión de política,
sino de vergüenza. Ya podemos sacar a nuestras mujeres a la calle. ¡Viva
España decente!»

Hablando con gente diversa, pude advertir el radiante júbilo de los
corazones ante este hecho negativo: \emph{No hay Gobierno}. El no haber
Gobierno viene a ser como un descanso, como la sedación de un largo
suplicio doloroso; parece como la vuelta a la normalidad de la
existencia, o el renacer a la edad de oro cantada por los poetas. En la
Puerta del Sol, los grupos estacionados frente al Principal esperan ver
salir de él algo extraordinario y magnífico: un genio pródigo que salude
al pueblo arrojándole puñados de centenes, o panecillos, o credenciales.
Veo miles de caras de cesantes que con ninguna clase de rostros pueden
confundirse. Sus trajes de buen corte y muy ajados ya, sus sombreros sin
lustre, proclaman la penuria de innumerables familias decentes. Al fin
ha sonreído la esperanza para muchos que desde el 48 viven condenados al
estudio de las matemáticas, a calcular las probabilidades de cambio de
situación, y en tanto mantienen a la familia con el olor de las ollas
ministeriales.

Camino de mi casa, me encontré a \emph{Sebo} en la calle del Arenal.
Díjome con sigilo que se armaría el tumulto grande a la salida de los
Toros. «No olvide Vuecencia que hoy es lunes. La plaza está llena de
gente; allí están todos los aficionados a la tauromaquia y a la
\emph{politicomaquia}\ldots{} Otra cosa, señor: sepa que formará
Ministerio el general Córdova\ldots{} Dejando aparte la amistad de
Vuecencia con don Fernando, yo diré que éste no es hombre para el
remedio de la tremenda enfermedad de España\ldots{} Caerá, caerá
también, y si hoy decimos «¡pobre Sartorius!» mañana diremos «¡pobre
Córdova!\ldots» Parece que anda en tratos con algunos señores del
Progreso para ver de ponerles el collar de ministro; pero ellos no
quieren ponerse más collar que el de su \emph{dogma}. ¿Me entiende,
señor? Dicen que su \emph{dogma} o nada\ldots{} Pues yo, con permiso de
Vuecencia, digo que no me conviene ser colocado hasta que vengan los que
han de ser estables, O'Donnell y Dulce, un Gobierno tranquilo. Es ése mi
\emph{dogma}, señor. Entiendo yo que esta palabra significa la cosa más
necesaria del mundo, verbigracia, \emph{comer con tranquilidad}.

\emph{Sigue Julio}.---El 17 por la noche, cenando, supimos que la salida
de los toros había sido tumultuosa. El himno de Riego resonó en las
puertas de la plaza, y creciendo, creciendo en intensidad, al llegar el
coro a la Puerta del Sol era como si todo Madrid cantase. Supimos
también que Córdova ha catequizado a Ríos Rosas para que le acompañe en
el Ministerio. Ante la insurrección popular, que me parece ha de ser de
cuidado, ¿quién podrá vaticinar si estos nuevos gobernantes lograrán la
reconciliación del Pueblo y el Trono? Mal avenidos los deja el
\emph{polaquismo}. De sobremesa, llegan algunos amigos que sostienen la
tertulia normal de casa, los perdurables reaccionarios señor Sureda y
don Roa, y otros, que en noche de acontecimientos vienen a traer y a
recibir impresiones: mi hermano Agustín, \emph{polaco}; don Nicolás
Hurtado, \emph{bravo-murillista}; San Román, narvaísta; Bruno Carrasco,
progresista, y Aransis, indefinido, con cierta inclinación a los
partidos bullangueros. No se entablaron las disputas agrias que amenizan
nuestra tertulia las más de las noches. Fue aquélla, noche de
expectación, de conjeturas, de profecías, de apuestas. Vaticinó mi
suegro la \emph{disolución social}, y Carrasco apostó a que antes de
ocho días tendremos a Espartero en Madrid. La repentina emergencia de
sucesos históricos graves no podía menos de traer igual aglomeración
congestiva de acontecimientos privados. Llegó \emph{Sebo} a decirme que
había visto a Gracián en la calle de Rodas, con \emph{La Panadera} de
Vallecas y un tal Ramos, que es el gran reclutador de populacho para
motines; presentose después Valeria, lloriqueando, con el cuento de que
su maridito, el angelical Navascués, no había parecido por el domicilio
conyugal en cuatro días con sus noches, y por fin se coló en casa el
hojalatero, trayendo la novedad interesantísima de que \emph{Mita} y
\emph{Ley} están en Madrid. Sentí en mi cabeza el torbellino de la
brusca entrada de tantos huéspedes mentales en la cavidad del cerebro.
Aunque algunos me interesaban poco, la súbita irrupción de tantas ideas
me hizo estremecer. Se introducían todas a un tiempo, montando unas
sobre otras.

No era posible que yo me privase de salir a la calle, para contemplar
una página histórica, que sin duda habría de ser más bella que la de
Vicálvaro. Los temores de mi mujer se acallaron viéndome acompañado de
Aransis y Bruno, de otros amigos de la casa, y de \emph{Sebo} y Rodrigo.
Formábamos una caravana bastante fuerte para despreciar el peligro.
Además, dimos formal palabra de no intentar ver de cerca ninguna
barricada, caso que las hubiere, y de ponernos a discreta distancia de
las aglomeraciones de gente\ldots{} Con ver un poquito bastaría: quizás,
y sin quizás, ciertas páginas históricas, como las obras de pintura,
pierden bastante miradas desde un punto de vista cercano\ldots{} Mi
primer pensamiento, bajando la escalera, fue para preguntar al
violinista la residencia de su hermano y de \emph{Mita}. «Ayer, señor,
se aposentaban en el mesón de la calle Angosta de San Bernardo, donde
paran las tartanas de Loeches y Mejorada; pero fui a verles hoy, y me
dijeron que se han ido a otra parte, no saben a donde. Yo lo averiguaré
esta noche, pues de seguro lo sabe mi cuñado Halconero, que también está
en Madrid con su familia.»

Nada respondí al buen Ansúrez. Sentía yo que el Destino se mostraba
propicio a mis deseos: no era menester que mi voluntad solicitara sus
favores\ldots{} Entre tanto, embargaba mi atención el espectáculo de
público regocijo que ofrecía Madrid, con luces en todas sus ventanas y
balcones, y hasta en los últimos agujeros de los más altos desvanes.
Ningún vecino había dejado de sacar al exterior el farol o candil, las
elegantes bujías o el velón lujoso, que era como sacar al rostro las
esperanzas y los gozos del alma. En ninguna festividad oficial, ni en
coronación de Rey, ni en parto de Reina o bautizo de Príncipe, vi más
espontánea y sincera manifestación del júbilo de un pueblo. Iban por la
calle en grupos bulliciosos los vecinos, hombres y mujeres, niños y
ancianos, y con ingenuo fervor gritaban: «¡Viva la Libertad, muera
Cristina, abajo los ladrones!» poniendo en sus acentos, más que la idea
política, un sentimiento familiar con ecos de exaltación religiosa.
Dábanse unos a otros parabienes expresivos, y personas que no se
conocían se abrazaban; otros que jamás se vieron se preguntaban
recíprocamente por la familia y se deseaban mil bienandanzas. Nunca vi
cosa igual.

Para poder llegar a la Puerta del Sol, tuvimos que dar un largo rodeo
desde mi casa, subiendo a la plazuela de Santo Domingo y bajando por
Preciados. Esta calle no estaba tan obstruida por el gentío como la del
Arenal, donde las multitudes se obstinaban en que repicara San Ginés,
una de las pocas iglesias que a celebrar se resistían con el volteo de
sus campanas la felicidad popular. Al fin repicó San Ginés, repicaron
las Descalzas Reales, y no hubo campana ni esquila que no uniera su voz
al cántico de tantas alegrías. Hacia el Postigo de San Martín vimos a un
hombre gordo que, plantado en medio de la calle, convidaba a los
transeúntes a tomar café o copas en el café de la Estrella. El lo
pagaría todo. Más abajo, un tabernero invitaba bizarramente al público a
entrar en el establecimiento, y hacer todo el consumo de vino que
requerían las venturosas circunstancias. Penetrando a fuerza de
empujones y codazos en la Puerta del Sol, vimos que las turbas se
arremolinaban. En el inmenso oleaje se marcó una corriente que marchaba
hacia la calle de la Montera. Sobre las cabezas movibles se destacaba un
hombre a caballo; el hombre enarbolaba un trapo a guisa de bandera;
otras banderas de colorines le seguían, y al compás de la marcha
cantaban las voces el Himno de Riego, y pedían que vivieran los buenos y
que murieran los malos. Pronto pudimos enterarnos de que aquel enorme
destacamento de la masa plebeya se encaminaba al Saladero, con el noble
fin de poner en libertad a los presos políticos que en aquella inmunda
cárcel penaban por sus ideas o sus escritos.

Lo mas curioso, o si se quiere lo más instructivo, de aquella noche
memorable, fue que todos los policías y corchetes desaparecieron, como
si se los tragara la tierra. No se veía en las calles ninguna autoridad.
Dentro de la Casa de Correos había, sin duda, alguna representación de
ella, que no se manifestaba al exterior más que por la claridad de las
habitaciones bajas, y por las figuras quietas que al través de las rejas
se vislumbraban. El pueblo miraba sin cesar a las rejas, y después de
mucho mirar, se entablaron diálogos familiares entre los de fuera y los
de dentro. Ved la muestra:

«Abrid la puerta. Entraremos y nos daréis armas.

---No puede ser. No somos enemigos de la Libertad. Ya veis que no os
hacemos fuego.

---Abrid, y seamos hermanos.

---Ni vosotros entraréis, ni nosotros saldremos. Todo seguirá como ahora
está.

---¿Hasta cuándo? Nosotros estaremos aquí hasta que nos den armas.

---No necesitáis armas. Nosotros no haremos fuego contra el
pueblo\ldots{} Abriremos un instante las puertas para recoger a nuestros
centinelas\ldots{} Pero habéis de prometernos y jurarnos que, al ver
abrir la puerta, no empujaréis para colaros.

---Lo prometemos. Abrid, y que entren vuestros centinelas.»

Se hizo como aquí lo digo. Abrieron los de dentro; entraron los dos
centinelas, el que hacía la guardia en la esquina de la calle de
Carretas, y el de la puerta principal. El pueblo, que aún permanecía en
los rosados nimbos de su entusiasmo inicial, todavía generoso, cumplió
su palabra, y no aprovechó la fugaz abertura para empujar y colarse
adentro. Después se acentuaron en la inquieta masa las ganas de
proveerse de armas, cuya posesión conceptuaba como un derecho, y se
entabló nuevo diálogo más vivo y apremiante que el anterior. Nada podía
resultar de estos dimes y diretes al través de una puerta: el pueblo no
podía pasar más tiempo sin armas, y las armas estaban dentro de la Casa
de Correos. Para cogerlas era menester franquear la entrada del
edificio, y esto se haría batiéndola con ariete a estilo romano: el
ariete fue una viga que los más decididos sacaron del derribo de la Casa
de Beneficencia. Puesto el madero en hombros de una docena de bigardos
en fila, lo movían horizontalmente a compás, descargando furibundos
golpes en la puerta. Esta respondía con hondo gemido; pero no se abría.
A los golpes agregaron el fuego, prendido con maderas de la casilla del
retén de policía que al anochecer destruyó el pueblo en los Caños de
Peral. Combinados el incendio y los porrazos, cedió al fin la dura
puerta de Gobernación y entró el paisanaje, encontrando a la Guardia
Civil y soldados en correcta formación descansando sobre las armas. Sin
violencia alguna pasaron los fusiles y espadas de las manos militares a
las plebeyas. Trámite más sencillo de revolución no se ha visto nunca.

Según me contó el gran \emph{Sebo}, que anduvo en este lance, los
paisanos invadieron las habitaciones altas de Gobernación, respetando
los objetos de valor, escribanías de plata, fajos de papeles, y legajos
atados con balduque, cuidándose tan sólo de encender cuantas bujías y
velones en las dependencias había para arrimar luces a las ventanas. Se
buscaba el efecto decorativo de iluminar la Casa de Correos, único
edificio que hasta entonces había permanecido a obscuras. La muchedumbre
que llenaba la Puerta del Sol, recibía con aplausos y gritos de júbilo
las sucesivas apariciones de luminarias en los altos huecos\ldots{}
Pueril era esta forma de venganza popular. No era menos inocente el
gustazo que se dieron todos de sentarse en la poltrona que había ocupado
San Luis. A bofetadas se disputaban los paisanos el honor de sentarse en
ella, forma de vindicación que a muchos les pareció bastante. ¿Qué más
quería el pueblo que convertir la silla del tirano en mueble nacional
para uso de todos los españoles?

Era media noche. El pueblo armado, libre, dueño de Madrid, evolucionaba
lentamente desde el período de las alegrías ingenuas hacia el de las
vindicaciones terribles. En días, en horas, pasa este soberano de niño a
hombre, y sus derechos, que empiezan siendo juguetes, se convierten en
armas.

\hypertarget{xxiii}{%
\chapter{XXIII}\label{xxiii}}

Cuando \emph{Sebo} volvió a reunirse a mí en la calle de Cofreros (núm.
6, paragüería de Larrea), donde teníamos nuestro cuartel general, ya nos
faltaban algunas figuras del grupo: unos por cansancio, otros
arrebatados por el oleaje, desaparecieron de nuestra vista. Sólo
quedábamos Aransis, el hojalatero y yo. Sin movernos de aquel rincón, en
el cual teníamos retirada segura por la calle de Peregrinos, supimos que
en el Ayuntamiento y Gobierno Civil había penetrado también el pueblo, y
que en uno y otro local se habían constituido juntas, de ésas que nacen
con espontánea fecundidad de los días y más aún en las noches calurosas
de revolución. Que alguna de estas juntas intentó parlamentar con
Palacio, se dijo por allí; que en Palacio no quisieron recibirla, me lo
imaginé yo; que en Palacio estaban a la sazón los nuevos Ministros por
no poder estar en ninguna de las dependencias del Estado, lo suponíamos.
Luego, al saber la verdad, vimos que habíamos acertado, y que componemos
la Historia sin saberlo\ldots{} Lo extraño es que antes de oír el rumor
de que la plebe invadía y quemaba la vivienda de Cristina, tuve yo
adivinación o presciencia de aquel suceso. No es difícil hoy que el
pensamiento de cualquier ciudadano se anticipe a los hechos históricos,
porque estos se anuncian por inequívocos efluvios en el ambiente que
respiramos. Los odios más frenéticos del pueblo español en estos días
recaen sobre dos cabezas: la de San Luis y la de María Cristina. Uno y
otra han desatado sobre España todos los males. San Luis es el insolente
capitán de esta cuadrilla de \emph{ladrones públicos}; Cristina, la
mujer \emph{rapaz}, \emph{avarienta}, \emph{insaciable}, que con diestra
mano escamotea los tesoros de la Nación. Así lo cree la gente.

Apenas se vieron las multitudes dueñas de la capital, sin autoridad que
las contuviera, corrió la noticia de que ardía el palacio de la esposa
de Muñoz. Se dijo antes de que fuera cierto, y todo el mundo lo creyó
antes de oírlo decir. Cuando llegó a nosotros el primer rumor, ya íbamos
hacia allá, seguros de encontrar incendio. Nada sabíamos aún, y nos daba
en la nariz olor de madera quemada. En la plazuela de las Descalzas, un
amigo de \emph{Sebo} con quien tropezamos, nos dijo que, atacado por las
turbas el palacio de la calle de las Rejas, la Reina Madre escapó por
las caballerizas, vestida de hombre. No lo creí: ya sabe un ciudadano
listo distinguir las mentiras absurdas de las verosímiles. Sin que nadie
me lo asegurara, pensé que doña Cristina, desde los primeros síntomas de
alboroto, se había trasladado al Palacio Real, llevándose por delante
cuantos papeles pudieran comprometerla.

He perdido la memoria de las vueltas que tuvimos que dar para poder
asomamos a la plazuela de Ministerios por la calle de Torija. Desde
lejos vimos el resplandor del incendio. Frente a doña María de Molina
habían hecho una hoguera, en la cual hombres y mujeres de mala facha
arrojaban lo que iban sacando del palacio: muebles, cuadros, cortinas.
No sé qué habría sido de la Reina Madre si hubieran podido cogerla como
cogían un sillón. Oí decir que fue respetada la servidumbre; oí también
que hicieron pedazos todo lo que por su pesadez no podían transportar a
la hoguera. ¡Benigna es ciertamente la barbarie de un pueblo que venga
sus agravios en muebles, porcelanas y objetos insensibles!

La curiosidad pudo en mí más que el sentimiento del peligro, y descendí
a la plazuela con mis acompañantes, abriendo paso a empujones. La
hoguera desplegaba al viento sus llamas rojizas. En lo más animado de la
quemazón se hallaba el pueblo, pasando ya casi de lo siniestro a lo
divertido, cuando vino a cortar bruscamente la gresca un destacamento de
Cazadores mandado por un paisano. A la luz vivísima del incendio le
conocí: era Gándara, el conspirador de 1848, el militar valiente que
adora la Libertad, y sabe capitanear igualmente soldados y pueblo. Yo le
supuse afecto a O'Donnell y comprometido en la Revolución. Me sorprendió
verle mandando tropas. ¿Por qué las mandaba? Según la versión corriente
aquella noche, habiendo sabido Gándara que el pueblo trataba de
incendiar la casa de su entrañable amigo don José Salamanca, corrió a
Palacio en busca del general Córdova, flamante Presidente del Consejo, y
allá tuvieron los dos unas palabritas harto vivas, por si el nuevo
Gobierno se cruzaba o no de brazos delante de las turbas de bandidos.
Acabó Córdova por decirle que a sus órdenes ponía dos o tres compañías
de Cazadores de Baza; que con ellas fuese al instante a contener los
brutales atentados, empezando por lo más próximo, que era el asalto y
destrucción del domicilio de la Reina Madre. No necesitó más Gándara
para ponerse al frente de los Cazadores. Como insignias llevaba la faja
y sombrero que le dio un caballerizo. Sus bromas, en casos de esta
naturaleza, solían ser pesadas, y testigo fui de ello, pues en cuantito
que embocó por la calle de Bailén a la plazuela de Ministerios, gritó
con estentórea voz: «¡Cazadores, por mitades! ¡Preparen, apunten\ldots!»

¡Fuego! A la primera descarga cayeron muchos. La dispersión fue
instantánea y rapidísima. Corrí de los primeros, pues maldita la gracia
que me hacía ser fusilado estúpidamente\ldots{} Encontreme solo junto a
la escalerilla de la calle del Reloj, donde se me reunió Rodriguillo
Ansúrez\ldots{} Preguntéle por Aransis y \emph{Sebo}, y no supo
contestarme. Díjome que en derredor de la hoguera había más de una
docena de muertos. En la puerta del Senado, vi a un pobre que allí quedó
seco, y a pocos pasos de él una mujer, que yacía boca abajo\ldots{} La
segunda descarga ya nos cogió en salvo; pero poco faltó para que nos
arrollara la multitud que por la calle del Reloj emprendía la retirada.
La ardiente curiosidad me picó de nuevo y asomé las narices a la
plazuela. Vi a los Cazadores acometiendo a bayonetazos a los infelices
que buscaron refugio en el mismo palacio asaltado. Dentro de éste
sonaron tiros. Los de Baza mataban sin piedad a cuantos andaban todavía
en el trajín de acarrear muebles para la hoguera. Los atizadores de
ésta, o habían desaparecido o pataleaban en el suelo. Clamaban los
heridos entre tizones: aquí y allí zapatos, ropas, gorras y sombreros
abandonados; algún trabuco, charcos de sangre, despojos del palacio y de
sus asaltadores\ldots{}

No quise ver más ni exponerme a nuevos peligros, y por la travesía del
Reloj y la calle de Fomento tratamos de ganar la Cuesta de Santo
Domingo. Fuimos a parar a la rinconada próxima al pórtico del venerable
convento de monjas, y allí, por extraño encadenamiento de ideas, me
acordé de un poema burlesco, graciosísimo, que Narciso Serra nos había
recitado con prodigiosa memoria en el banquete de Torrejón. Era una
historia disparatada, parodia de las leyendas en boga: un galán
hambriento que rondaba el convento de dominicas buscando la manera de
escalarlo para verse con la \emph{Sor} de sus pensamientos; un jorobado
sacristán que le llevaba bartolillos y empanadas, y dentro de una de
éstas la esperada cartita\ldots{} La misteriosa volubilidad del
pensamiento humano se me patentizó entonces, pues, hallándome frente a
una situación trágica, mi memoria dio un salto hacia las cosas de
burlas, reconstruyendo sin perder sílaba la primera quintilla del
chistoso poema, y, en voz alta la repetí:

\small
\newlength\mlend
\settowidth\mlend{van todos el verbo MINGO...}
\begin{center}
\parbox{\mlend}{\textit{\quad En un obscuro lugar                       \\
                que junto a Santo Domingo                               \\
                húmedo se suele hallar,                                 \\
                porque en él a conjugar                                 \\
                van todos el verbo} \normalfont\small{\textsc{mingo…}}} \\
\end{center}
\normalsize

Nada tenía que ver esto con lo que a la sazón me rodeaba; pero los
caprichos de nuestra mente no están sujetos a ninguna ley conocida. Mi
memoria continuó jugando con la quintilla, y dándosela y quitándosela a
la palabra\ldots{} Luego vi que venían tropas por la calle de Fomento.
Era un retén que debía cerrar el paso a la plazuela de Ministerios. Los
soldados ocuparon la bocacalle, y el oficial que los mandaba pasó a la
rinconada con objeto, al parecer, de conjugar el verbo mencionado en la
quintilla. Le vi de vuelta y, reconociéndole, le salí al paso. Era
Navascués. Hablamos rápidamente, encareciendo él la rabia que sentía de
verse obligado a hacer fuego contra el pueblo, preguntándole yo dónde
estaba Gracián, y qué papel hacía en la diabólica función de aquella
noche. Con desconsolado acento me dijo que su amigo había logrado
evadirse del servicio de tropa regular, y andaba organizando los grupos
más levantiscos de la plebe allá por la calle de Toledo y plaza Mayor. Y
yo: «Mucho me temo que Bartolomé perezca de mala manera en este
torbellino de las venganzas populares.» Y él: «Gracián no muere. Está
donde le llaman su destino y su vocación. Algo terrible y grande hará si
le ayudan\ldots{} Como no hay Gobierno, el pueblo es el amo esta noche.»
Y yo: «Si no hay Gobierno a media noche, a la madrugada lo habrá\ldots{}
El héroe de esta noche y de mañana no será Gracián, sino Joaquín de la
Gándara.» Y él: «Gándara es héroe popular, por más que ahora nos haya
traído a castigar a los incendiarios. Por caudillo del pueblo le tuve yo
siempre. Con él me batí el 48.

---Pues si es héroe popular, ¿por qué ha mandado fusilar al
pueblo?\ldots{} Gándara es hoy el héroe del Orden, no de la Libertad.

---No, señor: de la Libertad. El Orden que el defiende es el Orden del
Desorden.

---Es decir, la autoridad de la revolución. ¿Cree usted, Rogelio, que el
rebaño puede ser pastor?

---Si es rebaño de ovejas, no; si lo es de hombres, sí.

---Siempre hay un hombre que pastoree. Ya no es Sartorius el pastor: ¿lo
será Gracián?

---Lo es, lo será\ldots{} Retírese, querido Pepe\ldots{} Abur.

¡Pobrecillo!, era un eco de la fraseología de Gracián; su pensamiento
volaba con las ideas del otro pájaro\ldots{} Seguimos Rodriguillo y yo,
no recuerdo por qué calles, en busca de emociones nuevas, viendo y
admirando el aspecto de Madrid a tales horas, en noche de plena
emancipación del pueblo. ¡Infeliz pueblo! Por una noche, por algunas
horas no más, le permiten los dioses el uso práctico de su soberanía, de
esa realeza ideal que sólo existe en las vanas retóricas de algún
tratadista vesánico. Y en su candidez, en la inexperiencia de su
soberanía, es el pueblo como un niño al que entregan un juguete de
mecanismo delicado y sutil. No sabe de qué suerte lo ha de poner en
movimiento, ni con qué frenos pararlo, ni con qué llaves darle
cuerda\ldots{} acaba por romper el juguete y abominar de él\ldots{}

Pues bien: aunque por ninguna parte se notaban indicios de autoridad, no
vimos desafueros más graves que los que ordinariamente turban la paz del
vecindario; no advertimos más que una alegría desatinada, burlas
ruidosas de las autoridades ausentes, desvanecidas como el humo de los
incendios. En la Red de San Luis vimos en el cielo el resplandor de las
hogueras, y a nosotros llegaba un alarido sordo de embriaguez
revolucionaria, mezcla de vivas y mueras\ldots{} A lo largo de la calle
de Caballero de Gracia oímos tiros, y mayor escándalo de voces airadas o
triunfales. Hermoso me pareció el tinte rojo del cielo; solemne el ruido
lejano de combatientes, incendiarios o lo que fueren. Descartando el
juicio de los hechos, y ateniéndome sólo a la estética, la noche
ruidosa, iluminada por las hogueras, me arrebataba de admiración, de un
júbilo de artista. Veía, por fin, una página histórica, interesante,
dramática, producida en el tiempo, sin estudio, por espontáneo brote en
el cerebro y en la voluntad de millares de hombres, que el día anterior
ignoraban que iban a ser histriones de una teatralidad tan bella. No
tenían más inspiración que sus odios, verdadera razón de Estado para los
ciudadanos que no habían gobernado nunca, y entonces con actos bárbaros
gobernaban a su modo, realizando algo parecido a la justicia, si no era
la justicia misma en todo su esplendor.

Mañana, pensaba yo, se juzgarán estos hechos como atentados a la
propiedad, como profanación de la ley o arrebatos de salvaje cólera. ¡Y
las culpas de esta brutal plebe, nadie las atenuará con el recuerdo de
las horribles violaciones de toda ley moral y cristiana que se contienen
en el gobierno regular de las sociedades; nadie verá la inmensa barbarie
que encierra el régimen burocrático, expoliador del ciudadano y
martirizador de pobres y ricos; nadie se acordará del sinnúmero de
verdugos que constituyen la familia oficial, y cuya única misión es
oprimir, vejar, expoliar y apurar la paciencia, la sangre y el bolsillo
de tantos miles de españoles que sufren y callan!\ldots{} Nadie se
fijará en el crimen lento, hipócrita, metodizado, de la acción
gobernante, mientras que salta a la vista el crimen desnudo,
instantáneo, de unas gavillas de insensatos que asaltan, queman, matan,
sin respetar haciendas ni vidas. Nadie ve las víctimas obscuras que
inmoló la ambición de los poderosos, ni los atropellos que se suceden en
el seno recatado de una paz artificiosa, sostenida por la fuerza bruta
dominante, y todos se horrorizan de que la fuerza oprimida y dominada se
sacuda un día y, aprovechando un descuido del domador, tome venganza en
horas breves de los ultrajes y castigos de siglos largos\ldots{} Y bien
mirado esto, delante del sacro altar de Clío, ante el cual no cabe
falsear la verdad; bien miradas estas vindicaciones instantáneas frente
a las demasías que las motivaron, todo se reduce a una bella variedad de
formas de justicia dentro del canon de Naturaleza. Tenemos la justicia
espiritual, que nos habla, nos oprime y nos mata con el lenguaje del
derecho. Tenemos la justicia animal, que nos aterra con manotazos y
rugidos. De la intercadencia histórica de una y otra justicia resulta
una armonía mágica, que es de grande enseñanza para los pueblos.

Puestos todos a violar, no creo que deban cargarse a la cuenta de la
plebe las más escandalosas violaciones. El favoritismo en altas esferas
no hace menos estragos que la desatada barbarie en las bajas. No es el
pueblo quien da forma de embudo a las leyes, ni quien envenena las aguas
del poder en su propio manantial. Su ignorancia no es el único mal;
otros males hay, de que son responsables los que leen de corrido, los
que escriben con buena sintaxis, y los que hablan con sonora elocuencia.
Así están las leyes, arrinconadas como trastos viejos cuando perjudican
a los que las han hecho. Así huele tan mal el libro de la
Constitución\ldots{}

\small
\newlength\mlene
\settowidth\mlene{van todos el verbo MINGO...}
\begin{center}
\parbox{\mlene}{\textit{porque en él a conjugar                          \\
                van todos el verbo} \normalfont\small{\textsc{mingo.}}}  \\
\end{center}
\normalsize

\hypertarget{xxiv}{%
\chapter{XXIV}\label{xxiv}}

Vi las hogueras en que ardían los muebles de Salamanca, calle de
Cedaceros; vi las quemazones en la casa de San Luis, calle del Prado,
esquina al León; vi otros juegos de pirotecnia en diferentes calles
donde vivían hombres aborrecidos. De dos a tres de la madrugada, la
tropa mandada por Gándara iba calmando el furor de quemas. En Cedaceros
y Carrera de San Jerónimo cayeron ciudadanos que andaban por allí de
mirones, mientras que los incendiarios escapaban con veloz carrera. En
otros puntos de Madrid hubo tiroteo y lucha cuerpo a cuerpo entre
paisanos y tropa, y por todas partes se iba revelando la autoridad, como
si saliera de un eclipse o despertara de un pesado sueño. Hasta en el
paso de la gente que iba en retirada, se conocía que no estábamos ya
huérfanos de gobernantes: aún no se veía la mano dura; pero su acción la
sentíamos todos.

El carnaval revolucionario con chafarrinones de sangre y fuego se
acababa pronto. Los Dioses, envidiosos del Hombre, lo reducían a breves
horas. En éstas, los bromazos no llegaron al trágico desenfreno de las
revoluciones más señaladas en la Historia. Casi todos los muertos eran
de la clase humilde. El carnaval de la turba emancipada ofreció la
tremenda ironía de que, vistiéndose de jueces, las máscaras resultaron
víctimas. Todo el furor que al pueblo enardecía en las primeras horas de
la noche, quedó reducido a un soez pataleo delante de las casas en que
habían vivido los tiranuelos, a gritar con aullidos patrióticos, y a
quemar sillas y mesas inocentes, cuadros y cortinajes. No arrastraron a
nadie, no quitaron de en medio a los que con voces roncas llamaban
\emph{rateros} y \emph{truhanes}. Pagaron el pato los objetos de
carpintería y tapicería, venganza popular harto benigna\ldots{} Pensaba
yo que la destrucción de muebles de lujo es un hecho favorable a los
progresos de la industria y a la renovación de formas suntuarias. Los
ebanistas y decoradores de casas ricas estaban de enhorabuena, así como
los que inventan nuevos estilos de sillas y sofás. El fuego perjudicaba
poco a los Salamancas y Sartorius, y beneficiaba providencialmente a los
fabricantes.

Retireme a casa cuando amanecía. Triste era el aspecto de las calles
donde hubo fogatas. Por ellas desfilaban presurosos los transeúntes como
gato escaldado. Ceniza y tizones quedaban, restos humeantes, en los
cuales revolvían merodeadores rapaces. Cadáveres vi en la calle de
Cedaceros y en la del Baño; los heridos se retiraban por su pie si
podían, o eran auxiliados por gentes caritativas, que nunca faltan. En
la Puerta del Sol vi bastante tropa y Guardia Civil; las puertas del
Principal, cerradas a piedra y barro. De lejanas calles venía rumor de
algarada. Ya teníamos otra vez Gobierno, ya teníamos autoridad. Entre
los grupos se deslizaban, todavía medrosos, algunos policías
vergonzantes.

En casa me esperaba \emph{Sebo}, que, al disgregarse de nuestro grupo en
la calle de Torija, creyó que yo me había retirado a descansar.
Intranquilo estaba, y mucho más mi mujer, que me abrazó como si yo
volviese de un largo y peligroso viaje. «No me ha pasado nada. No verás
en mí ni un rasguño. Todo lo he presenciado, y me he divertido
muchísimo\ldots{} No te rías, mujer. El espectáculo ha sido de
incomparable belleza, y de esos que rara vez podemos disfrutar. ¡El
pueblo ejerciendo de soberano por unas cuantas horas, y entreteniéndose
en jugar con los flecos y garambainas de su manto!\ldots{} Luego, al
andar, se pisa el manto, se cae de bruces\ldots{} En fin, que estos
carnavales son forzosamente muy breves, y el pueblo, que por divina
licencia los celebra, se divierte poco, como no entienda por diversión
el ser fusilado en lo más entretenido de la fiesta.»

Me acosté, pero no dormí. Ardía mi mente del recuerdo de los
espectáculos de la noche pasada. Era una visión que no quería borrarse,
y que me atormentaba, engrandeciendo el interés de sus incidentes, y
revistiéndose a cada instante de mayor hermosura. No me canso de
decirlo: una de las cosas más bellas que yo había contemplado en mi vida
era la acción libre del pueblo durante algunas horas, el albedrío
nacional desenfrenado y \emph{en pelo}, manifestándose como es;
paréntesis de realidad abierto en el tedioso sistema de ficciones que
constituyen nuestra vida social y política. Buen alivio daba el tal
espectáculo al \emph{Ansia de belleza} que me afligía, dolencia del
alma; pero no acababa de satisfacerme: yo quería más, más pataleos y
manotazos de la plebe restituida a su libertad; quería gozar más de las
ideas elementales, como fueron antes de toda organización, y ver el
Gobierno y la Justicia reproducidos en la desnudez y simplicidad de su
estado primitivo\ldots{}

Desde mi lecho, cerrados los ojos figurando que dormía, creía yo sentir
lejanos alaridos de la multitud. Llegué a pensar que habían levantado
barricadas en la vecina calle del Arenal o en la plaza de Santo Domingo.
Mi mujer me desengañó, incitándome al descanso, y diciéndome que no
había tales barricadas, y que Madrid tomaba su chocolate tranquilo
debajo del poder de Fernando Córdova y de Ríos Rosas. Yo aparenté
creerlo; pero seguían mis oídos dándome la sensación de bullicio popular
cercano. Órdenes severas había dado mi señor suegro para que ni a
\emph{Sebo} ni a Rodrigo se les permitiese llegar a mi presencia.
Querían encerrarme, aislarme de la vía pública, que era mi encanto, como
escenario de la más bella función teatral\ldots{} Fui bastante astuto
para disimular mi deseo de echarme a la calle\ldots{} me levanté\ldots{}
supe asentir a las insustanciales opiniones de don Feliciano,
maldiciendo los excesos demagógicos\ldots{} Las once serían cuando
aproveché un descuido de mi mujer para escabullirme de la casa
lindamente. Nadie me vio salir. En el portal me esperaba el hojalatero,
y juntos, sin hablarle yo más que por señas, dimos la vuelta a la
esquina\ldots{} y \emph{desaparecimos} por la calle de la Sartén.

---¿Sabe el señor dónde están Leoncio y su mujer?---me dijo Ansúrez en
cuanto nos vimos en paraje seguro.---Pues en una tienda de la plazuela
de San Miguel, donde se vende al por mayor toda la cangrejería que viene
a Madrid.

---¿Y tu hermana Lucila?

---En la calle de Toledo\ldots{} no sé el número\ldots{} Entrando por la
plaza Mayor, a mano derecha, pasadas dos o tres casas.

---Bien, bien. Vamos hacia allá. ¿Hay guerra en las calles?

---Muchísima guerra y tiros muchos, lo que los músicos llamamos \emph{à
piacere}, disparando cada cual como se le antoja, y en \emph{allegro
vivace} que quiere decir: vivo, vivo\ldots{}

---¿Esta mañana, después que me dejaste en casa, ocurrió algo? Yo he
sentido tiros.

---¡Anda! Un fuego tremendo en la plaza de Santo Domingo, mucho pueblo
contra la tropa que guarda Palacio y el Teatro Real, donde dicen que
están escondidos los \emph{ministros ladrones}, y el jefe de los rateros
y guindillas, D. Francisco Chico\ldots{}

---Bien decía yo que había trifulca\ldots{} ¡Y mi mujer queriendo
convencerme de que Madrid es la antesala del Limbo!

---Pues hoy hemos tenido función bonita\ldots{} Mucho pueblo\ldots{}
¡qué bulla, qué entusiasmo!, y en medio del paisanaje un militar a
caballo, con su ordenanza, también a caballo. Era el Coronel de
Farnesio\ldots{} En fin, señor, que el Coronel y el pueblo hablaron a
gritos: primero en la plaza Mayor, después en el Principal de la Puerta
del Sol; pero yo no entendí lo que decían, porque no pude acercarme. Ni
sé cómo se llamaba el Coronel\ldots{} ni qué le pedían, ni qué
contestaba\ldots{} Sólo sé que después fueron todos a la plaza de Santo
Domingo, donde andaban a tiros tropas y paisanos\ldots{} Yo fui
también\ldots{} Por el camino, cadáveres, sangre\ldots{} las casas
cerradas todas, mucho miedo\ldots{} ¿Qué pasó? No lo sé\ldots{} El
Coronel mandó cesar el fuego, y después, por la vuelta que llevaba, me
pareció que iba a Palacio, y gritaba la gente: «¡Viva\ldots{}
\emph{tal!»} No me acuerdo del nombre.

---Por algo que hoy ha dicho mi suegro en casa, entiendo ahora que ese
que viste es el coronel Garrigó\ldots{} Con O'Donnell estaba en
Vicálvaro. Se portó como un héroe. Fue herido, cayó prisionero frente a
los cañones de Blaser\ldots{} Al volver a Madrid las tropas del
Gobierno, procedía fusilar a Garrigó\ldots{} Pero ¿qué había de hacer la
Reina más que indultarle? En estas guerras mansas entre dos partes del
Ejército, no puede haber ensañamiento. Al fin han de entenderse y ser
todos amigos. Ni han fusilado al Coronel de Farnesio, ni habrían
fusilado a O'Donnell, ni a Messina, ni a Dulce, ni a Ros de Olano si les
hubieran cogido\ldots{} ¿Te vas enterando? Garrigó es liberal, y además
muy valiente, tan buen patriota como buen soldado. El pueblo le quiere;
la tropa también. De lo que me cuentas y de lo que oí a mi señor suegro,
saco esta conjetura, mejor dicho, dos conjeturas: o Garrigó ha sido
mandado por Córdova a parlamentar con los insurrectos para ver de
encontrar un término de avenencia y paces, o ha dado este paso por su
propio impulso generoso, deseando facilitar a la Reina lo que la Reina
estará deseando a estas horas: reconciliarse con la Nación\ldots{} A ver
si recuerdas, hijo. ¿A lo que Garrigó decía contestaba la multitud con
aclamaciones?

---A veces, sí; a veces, no\ldots{} Cuando entró ese señor en el
Principal, el pueblo pidió que saliese al balcón\ldots{} pero no salía.

---No salía, y las turbas impacientes\ldots{}

---Gritaban: «¡Qué desarmen a la Guardia Civil.» Esto sí lo entendí
bien\ldots{} Y yo también lo grité, sin más razón que oírlo a los demás.

---Perfectamente. Y al fin salió Garrigó al balcón.

---Sí, señor\ldots{} y parecía que queriendo decir una cosa, no podía
decirla\ldots{} o que no sabía cómo contestar a los de fuera y a los de
dentro.

---¿Los de fuera pedían el desarme de la Guardia Civil?

---Y gritábamos: «¡Mueran los asesinos!» Al Coronel no le entendí más
que unas pocas palabras hablando de la Reina.

---Del bondadoso corazón de la Reina\ldots{} Y el pueblo, que es un
buenazo, se ablandaba con esto.

---Se ablandaba un poquito, y después, otra vez con que se desarmara a
la Guardia Civil\ldots{}

---Naturalmente\ldots{} Puesto que la Reina es tan bondadosa, que mande
a la Guardia Civil parar el fuego\ldots{} En fin, que Garrigó fue a la
plaza de Santo Domingo a ordenar que cesaran las hostilidades. Y las
masas tras él\ldots{} ¿no es eso?

---Tras él yo también, gritando, hasta quedarme ronco, todo lo que oía
gritar a los demás\ldots{} Y como digo, pararon los tiros, y el señor
Garrigó siguió luego hacia Palacio\ldots{} Puede que haya llevado a la
Reina unas palabritas del paisanaje\ldots{}

---De seguro habrá dicho a Su Majestad algo del bondadoso corazón del
pueblo, y corazones frente a corazones, tendremos sensiblería, pucheros,
abrazos, y \emph{tutti contenti}. Esto podrá ser; pero no será; ni
quiero yo estas blanduras ni estos abrazos, que son la pérfida
componenda, el engaño recíproco, para vivir siempre en un régimen de
mentiras. Nada de paces ni arreglitos, sino lucha, y que veamos
practicadas las ideas elementales de Justicia y de Gobierno. ¿Me
entiendes tú? A las sociedades les conviene volver de vez en cuando al
salvajismo\ldots{} como ventilación, como saneamiento\ldots{} Vivimos
una vida de artificios, que a la larga nos van cubriendo de polvo y
telarañas\ldots{} Conviene barrer, ventilar, fumigar\ldots{}

Recordando ahora, un poco lejos ya de aquel día y de aquellos sucesos,
lo que entonces pensaba yo y decía, obligado me veo a reconocer que no
me encontraba, el 18 de Julio, en la completa serenidad de juicio que
normalmente disfruto. Las escenas trágicas de la noche del 17, el fulgor
de las hogueras, mis \emph{ansias de belleza}, el \emph{desarreglo
estético}, digámoslo así, que se inició en mí desde el regreso de
Vicálvaro, habían turbado mi espíritu. Al escaparme de mi casa,
escabulléndome con Rodrigo Ansúrez por calles poco frecuentadas, me
sentí romántico: la leyenda, la poesía política, si así puede llamarse,
me seducía y me rondaba. Érame grato no llevar la compañía de
\emph{Sebo}, cuyo prosaísmo me apestaba, y sí la del hojalatero artista,
de ingenua sencillez en su trato. Pensaba yo que el muchacho podría
extasiarme tocando en el violín la \emph{Tabla de los Derechos del
hombre}, o el \emph{Contrato social}\ldots{} Camino de la plaza de
Oriente, rodeando calles y travesías, díjele que le nombraba mi
escudero, y que, no gustando yo de cosas comunes ni de términos
trillados, le llamaba \emph{Ruy}, nombre conciso y de sabor arcaico.

No nos fue posible llegarnos a la plaza de Oriente, defendida por todos
sus approches como campamento atrincherado. Rodeando seguimos, y por la
calle de la Escalinata y de Mesón de Paños nos subimos a la calle Mayor,
donde la enorme aglomeración de gente dificultaba el tránsito; quisimos
pasar a la plaza de San Miguel, y la ola humana que tratamos de
atravesar nos arrastró hacia Platerías. No sé las veces que fui y vine a
lo largo de la calle, no por mi voluntad, sino por traslación de la masa
viviente a la cual pertenecíamos mi escudero y yo. Tan pronto me veía en
la embocadura de la Puerta del Sol como en el arco de Boteros. En la
plaza Mayor sonaba horroroso fuego. La Guardia Civil trataba de arrojar
de ella a los amotinados.

La confianza se establece pronto entre las moléculas que componen la
muchedumbre, por más que estas moléculas cambien de sitio a cada
instante. Caras que yo veía próximas me sonreían, y bocas de aquellas
desconocidas caras me dijeron: «En Palacio no se dan a partido\ldots{}
No nos entendemos\ldots{} Dicen que pitos, y luego son flautas\ldots» En
la onda fluctuábamos cuando aparecieron tropas por la Puerta del Sol,
con un General al frente. Oí que era Mata y Alós. Con dificultad se
abría camino. Antes de que llegase a la calle de Coloreros, apareció por
ésta el popular Garrigó con escolta de infantería y caballería. Se
encararon uno y otro caudillo; hablaron\ldots{} Yo estaba lejos, nada
pude oír. Mata siguió hacía los Consejos: sin duda iba a Palacio; el
otro entró en la plaza. Observamos que cesaba el fuego. ¿Se entenderían
al fin? ¿Traía Garrigó instrucciones de Palacio y nuevos arranques del
bondadoso corazón de la Reina? Sin duda no traía nada decisivo, porque
el fuego se reanudó\ldots{} Fue que la Guardia Civil, por orden del
valiente Coronel de Farnesio, puso culatas arriba. El pueblo entonces se
arrojó obre los guardias para quitarles las armas. Pero los guardias no
son gente que a dos tirones se deja desarmar\ldots{} Otra vez el fuego,
y algunos paisanos mandados al otro mundo.

Yo no presencié estos incidentes. Oí los tiros, y vi los heridos y
muertos un rato después. Terminó la reyerta retirándose Garrigó con la
Guardia Civil, y penetrando impetuosamente en la plaza por sus
diferentes boquetes, el mar, el pueblo\ldots{} En aquel mar iba yo con
mi leal escudero, cuya vista de lince descubrió al instante rostros
conocidos, y me dijo: «Vea, señor: allí, entre aquellos que manotean,
está el valiente\ldots{} mírelo\ldots{} Don Bartolomé Gracián, en mangas
de camisa, sin sombrero\ldots{} Arma no lleva, si no es que la esconde.

\hypertarget{xxv}{%
\chapter{XXV}\label{xxv}}

Corrí tras del hombre que en aquella ocasión a mis ojos tomaba
proporciones de figura heroica, tribuno y caudillo de la plebe; pero las
oscilaciones del gentío le alejaban de mí cuando ya creía tenerle al
alcance de mi mano. Yo gritaba: «¡Gracián, Gracián!» Se perdía mi voz en
el bramido estentóreo del viento y la mar, que esto era el pueblo,
océano revuelto y aires desencadenados\ldots{} Por fin, pude cogerle en
el arco de la calle de Atocha, y hablamos brevemente, pues no había
lugar de largas conversaciones. «¿Quiere usted armas?---me dijo.---¿Se
batirá usted con nosotros y por nosotros?

---Los hombres que se lanzan con tanto valor y entereza a una lucha
desigual contra la burocracia y el militarismo, tienen todas mis
simpatías. Pero yo no soy de armas tomar; no sirvo para esto\ldots{}
Vengo de curioso\ldots{}

---De cronista quizás.

---Algo también de cronista. Quiero ver el atleta desnudo, inerme,
luchando con su hermano, el otro atleta, vestido de todas armas, pueblo
contra ejército, que es dos formas de pueblo la una frente a la otra.
Entiendo, querido Gracián, que no hay ni puede haber en el siglo que
corremos espectáculo más hermoso que este pugilato entre dos hijos de
una misma madre: el hijo soldado, el hijo paisano\ldots{} Dos
gladiadores y una sola espada.

---Me parece que el gladiador desnudo llevará la ventaja. Ahora tenemos
que ajustarle las cuentas a este asesino de Gándara, que sube de Atocha
con Artillería de montaña, Ingenieros, Guardia Civil de a caballo, y la
bendición del Patriarca de las Indias\ldots{} Allá vamos. En la plaza de
Antón Martín se verá quién es más guapo, si él o yo\ldots{} Tengo antojo
de merendarme a ese Gándara con toda su fachenda\ldots{} Ya sabe él que
estoy aquí; ya sabe que a estos pobres borregos los hago yo leones, y
que con ellos y conmigo no se juega\ldots{} No digo que no sea
valiente\ldots{} le conozco; nos conocemos: juntos peleamos el
48\ldots{} Viniendo contra mí, crea usted que no viene tranquilo\ldots{}
En fin, ya nos veremos luego, y le contaré a usted nuestras hazañas para
que las escriba.»

Siguió por la calle de Atocha, precedido y seguido de una turba de
aspecto feroz, armada con variedad de instrumentos mortíferos, obediente
al jefe, a él sujeta por una disciplina improvisada, mezcla de respeto,
de entusiasmo a estilo militar, y de terror a usanza de bandidos.
Pareciome el valor de Gracián como un producto de la arrogancia
histriónica y farandulera. Era valiente por el aplauso, y acometía y
realizaba sus hazañas para que le viera el público. Su heroísmo era
orgullo con guirindolas y cascabeles; se había imaginado el tipo del
héroe popular, y como gran artista, encarnaba admirablemente el papel
que para sí mismo había compuesto.

Volvimos mi escudero y yo a la Plaza, donde tuvimos poco tiempo de
tranquilidad. Por la calle Mayor aparecieron tropas con intento de
ocupar la Plaza, y el paisanaje corrió a cortarles el paso por los
portales de Bringas y por los tres ingresos que dan a Platerías. El
tiroteo arreció en pocos minutos. Adquirieron ventaja los paisanos por
la ocupación previa de las casas. Mientras unos, parapetados en los
porches, abrasaban a los soldados, otros, desde los altos pisos y
desvanes, les dañaban horrorosamente con auxilio del vecindario: mujeres
y chicos arrojaban sobre la cabeza del gladiador armado, tiestos, tejas,
agua caliente y otras materias. El gladiador desnudo se defendía con
todos los proyectiles que pudieran suplir la corta eficacia de su
armamento. No creyéndonos seguros de un balazo en ninguna de las
galerías de la Plaza, emprendimos nuestra retirada por la calle de
Botoneras. Vimos un portal que se abría para dar paso a un hombre;
descubrí en éste a un antiguo conocimiento mío, Sotero Trujillo, el
esposo de la pobre Antoñita, que acabó sus tristes días el 48, en un
segundo piso de la cercana Plaza: agraciome el tal con un fino saludo;
invitome a guarecerme en su domicilio, desde donde podía ver la función
sin riesgo; acepté, subimos.

Escalones arriba, me contó Sotero que había contraído segundas nupcias
con una viuda, la cual con suave dominación le había curado de su
vagancia y borracheras; que su mujer era sastra de curas y ganaba buen
dinero, y él, por influencias de ella, había conseguido un empleíto en
la expendeduría de las Bulas\ldots{} Tiempo hubo de que me contara esto
y algo más, por ser la escalera larguísima\ldots{} Llegamos por fin al
término de la ascensión\ldots{} Sotero me presentó a su cara mitad, que
es fea, gorda, tuerta; no tiene pescuezo, el seno casi se toca con la
barbilla, y los hombros se dejan acariciar por los pendientes de
filigrana que cuelgan de sus orejas. La sala en que nos recibieron, y
que estaba llena de viejas de la vecindad espantadas de los tiros, era
taller de sastrería eclesiástica, y no se veían allí más que sotanas y
manteos en corte o en hilván, roquetes y sobrepellices, y algún modelo
de bonete colocado sobre una cabeza de sacerdote de cartón. En la pared
vi retratos de diferentes Papas, Vírgenes del Sagrario, de la Cinta, de
la O, de la Fuencisla, de la Valvanera, y el escudo de la Santa Cruzada
junto a un cuadro de mesa revuelta, que fue la especialidad de Sotero en
sus floridos tiempos de dibujante.

Con remilgos de finura me hizo los honores de su casa la esposa de
Trujillo, no sin decirme que ella es de la familia de los Samaniegos,
oriundos de Mena, muy señores míos, y hablamos de aquella cruel guerra
en las calles, que no había de traer más que desolación, hambre,
irreligiosidad y, por fin, ateísmo\ldots{} No pudimos extendernos en
este coloquio, porque Sotero nos invitó a salir al tejado por un
buhardillón, para ver desde lo alto la tremenda lucha entre los dos
hermanos, según Gracián: el gladiador vestido y el gladiador desnudo.
Yo, que no deseaba otra cosa, acepté la invitación de Sotero, sin hacer
caso de los arrumacos de susto con que quiso retenernos la señora;
subimos por empinadas escaleras, salimos a un ventanón de donde se veía
toda la Plaza\ldots{} El espectáculo era desde arriba muy interesante,
no exento de peligro, pues bien podían algunas balas subir más alto que
la intención de los que disparaban. Los paisanos defendían las entradas
de Boteros y Amargura, el callejón del Infierno y los portales de
Bringas; desde los techos de la Casa-Panadería y casas próximas hacían
fuego contra la calle Mayor\ldots{}

Como el estado singular de mi espíritu ante la revolución visible solía
distraer mi atención, apartándola de los objetos de mayor importancia
para fijarla en los accesorios e insignificantes, me entretuve un
momento, y aun dos momentos, en mirar los gatos que en aquellos
irregulares y viejísimos tejados tienen su habitual residencia. Andaban
los animales de un lado para otro, paseando su turbación, y excitados
por el fuego\ldots{} Contagiados por el ejemplo de los hombres, unos a
otros se desafiaban con furiosos mayidos, y no lejos de mí, en un
tejadillo que vierte a la calle Imperial, dos de atigrada piel vinieron
a las uñas, y se sacudían y arañaban de firme como encarnizados
enemigos. Probablemente se peleaban por dar gusto a la garra, y
desconocían el motivo y fin de sus querellas. Observé asimismo que no se
veían gorriones ni palomas por aquellos aires. Los tiros ahuyentaban a
todos los pájaros que merodean en la zona urbana. Las golondrinas, menos
asustadizas de la pólvora, no se habían perdido de vista, y volaban,
trazando grandes círculos, en tomo a la mole de San Isidro o sobre el
copete de San Justo.

De estas observaciones me apartó Ruy, llamándome a que mirara lo que en
la Plaza ocurría: «Señor, mire hacia abajo. ¿Ve aquellos dos hombres que
cargan un herido, uno le coge por los pies, otro por los sobacos?

---Sí les veo\ldots{} y paréceme que no es herido, sino muerto el que
traen\ldots{} Lo tiran en el suelo, como si fuera un saco, al pie del
caballo de bronce\ldots{}

---Muerto parece. De los dos que lo han traído y que ahora vuelven hacia
los portales, fíjese en el que va delante, que lleva un gorro
colorado\ldots{} Es mi hermano Leoncio.»

Desde las alturas no pude ver del hombre que yo había conocido por
\emph{Ley}, y ahora es Leoncio, más que la gallarda estatura, el andar
resuelto, y el encarnado turbante o pañuelo que ceñía su cabeza.

«Si esto se acaba y podemos andar por el suelo sin peligro---dije a mi
escudero,---hemos de buscarle y echar un párrafo con él.»

El tiroteo era ya menos vivo. Los defensores de Boteros se retiraron al
centro de la Plaza. Vimos uniformes que avanzaban. Parapetados tras el
pedestal de Felipe III, aún defendían la Plaza los más tenaces. Heridos
había muchos; muertos, no pocos. Un hombre de aspecto agitanado yacía
junto a un farol, el cráneo deshecho, el calañés a media vara de
distancia, y arrimados a la Casa-Panadería, tres hombres tumbados, que
más parecían borrachos que muertos: eran cadáveres de héroes bebidos,
que habían peleado enardeciendo su patriotismo con el aguardiente.
Mujeres vimos recogiendo heridos y metiéndoles en una tienda de vaciador
y en la zapatería de Arnáiz\ldots{} La ventaja de la tropa se
manifestaba bien a las claras y crecía por momentos. Al fin el pueblo se
retiró a la calle de Toledo, y los soldados ocuparon la Plaza.

Admirable punto de defensa era para el gladiador desnudo el arranque
estrecho de la calle de Toledo, entre gruesos porches que le servían de
amparo. Allí y en la calle de Botoneras se entabló de nuevo el combate,
que no fue de larga duración, porque al gladiador armado le llegó
refuerzo por la Concepción Jerónima. El paisanaje se dispersó,
filtrándose por los edificios. En tanto, la tropa no podía estacionarse,
por no ser suficiente para guarnecer y fortificar todos los lugares
estratégicos. Su misión era despejar la extensa línea entre Palacio y
Atocha para impedir que en los puntos principales de ella se fortificase
la insurrección, y contener a ésta en los barrios del Sur, impidiéndole
la comunicación fácil con las zonas del Barquillo y Maravillas, donde
también había, por lo que después me contaron, tiroteo gordo. Despejada
la plaza Mayor, la tropa siguió hacia la de San Miguel y calles de
Milaneses y Santiago. Otras secciones recorrían la línea desde la plaza
del Progreso hasta San Francisco.

En tanto, la más tremenda lucha de aquel día se empeñaba en la plazuela
de Antón Martín, primero; después, en la del Ángel, entre el bravo
Gándara y el paisanaje dirigido por el temerario Gracián y otros tales,
no menos arriscados y feroces. Desde la atalaya en que nos habíamos
subido, oíamos el estruendo de fusilería y cañones, y veíamos la
humareda que el viento empujaba hacia el Oeste, arremolinándola en torno
a la torre de Santa Cruz. Observando esto, dijimos que la torre se ponía
mantilla. Si el humo nos daba idea de un terrible combate, no era menos
pavoroso el efecto de los tiros. Creyérase que todo aquel núcleo de
casas, entre la Trinidad y la Imprenta Nacional, entre Santa Cruz y las
Niñas de Loreto, se resquebrajaba, y que a pedazos caían paredes y
techumbres. La pelea se iba corriendo hacia el Este. Ya el humo no
parecía tan amigo de la torre de Santa Cruz, y acariciaba la de San
Sebastián. La artillería tronaba por la calle de Atocha\ldots{}

A todas éstas, el reloj de la Casa-Panadería, que, por encima de la
rabia y el delirio de los hombres, seguía fiel a su obligación, nos dijo
que eran las cuatro; nos recordó que no habíamos almorzado ni comido, y
el hambre nos advirtió que debíamos dar algún lastre a nuestros pobres
cuerpos suspendidos en el aire. Discurriendo estábamos el modo y ocasión
de comer algo, cuando el amigo Sotero subió a invitarnos en nombre de su
señora. Aceptamos con gratitud, y la propia sastra nos sirvió unas malas
sopas que sabían a sebo; una fritanga de mollejas, queso, vino y pan de
picos, duro de cuatro días. Con ser tan malo el comistraje, nos supo a
gloria, y reparamos las fuerzas del gran sofoco de estar todo el día
sobre tejas, mirando a los hombres matarse de tejas abajo. Muy
agradecidos a las amabilidades de Sotero y su esposa, abandonamos
nuestra torre-atalaya; descendimos, y en la calle Imperial nos echamos a
la cara un montón de muertos que había arrimado a la pared, en la
rinconada del Fiel Contraste. No se llevó flojo susto mi buen Ruy,
porque, viendo entre los cadáveres uno con trapos rojos en la cabeza,
liados al modo de turbante, creyó por un momento que era Leoncio; mas,
examinado de cerca el pobre difunto, nos tranquilizamos, y para mayor
seguridad los miramos todos, pues en lo más bajo del montón vi asomar
unos pies que me parecieron los del gran \emph{Sebo}. Tampoco estaba
\emph{Sebo} entre aquellos mártires políticos, cosa natural en quien
siempre tuvo por vocación lo contrario del sacrificio por una bandería
pequeña o por una idea grande.

«Ya parecerá---dije a mi escudero,---debajo de alguna mesa, o embutido
dentro de un armario donde los masones guarden los trastos y chirimbolos
de sus ritos\ldots{} Y entre tanto, Ruysillo, hazme el favor de guiarme
hacia donde yo pueda ver y saludar a los queridísimos salvajes
\emph{Mita} y \emph{Ley.»}

Aprovechando el despejo de las calles de Toledo y Latoneros, Ruy me
llevó a la de Cuchilleros, diciéndome: «Antes de llegarnos a la
\emph{cangrejería}, donde me parece que no encontraremos a nadie,
entremos en el establecimiento del señor Erasmo\ldots» Siguiéndole,
miraba yo los rótulos de las estrechas tiendas y pobrísimas industrias
de aquel rincón de Madrid. Vi taller de \emph{estañero}, con muestrario
de jeringas; vi tienda de \emph{albayalde} y \emph{ocre}; vi
\emph{albardero} y \emph{jalmero}, \emph{cestero},
\emph{jaulero}\ldots{} Por fin, dijo Ruy: «aquí es;» y por la entornada
puerta nos colamos en el local angosto de una tienda que tiene por
muestra: \emph{Obleas, lacre y fósforos}.

\hypertarget{xxvi}{%
\chapter{XXVI}\label{xxvi}}

Vi un taller parecido a los laboratorios de nigromantes o brujos que
aparecen en las comedias de magia, calderos y vasos de extraña forma,
hornillas, telarañas, y una pátina de polvo y mugre sobre paredes y
techo; el suelo de tierra, apelmazado y endurecido por las pisadas.
Suspenso el trabajo, sin fuego los hornos, volcados los calderos, todo
revelaba pobreza y el mísero rendimiento de las industrias que viven un
día sí y otro no, conforme a la desigual demanda de consumidores. Verdad
que aquellas modestísimas artes se relacionan con otras artes o
granjerías de alguna importancia: las obleas son hermanas de las
hostias; el lacre tiene algo que ver con los barnices, y los fósforos
con la pirotecnia. Por esto había surtido de carretillas de pólvora para
jugar los chicos; pasta para pegar cristalería y porcelanas rotas, y
diversas materias malolientes, en frascos y pucheros, solidificadas al
enfriarse; cola, pez, trementina, ingredientes tintóreos y mixturas de
todos los diablos\ldots{} Me detuve a contemplar aquella miseria, y a
considerar los esfuerzos que representa, titánicos, pero ineficaces para
obtener un pedazo de pan. ¡Lo que luchan y se afanan estas clases
inferiores de la industria para sostener una existencia mezquina sin
esperanzas de mejora! Y los infelices que en aquel taller echan
diariamente el quilo, estarían seguramente en las calles haciendo fuego
contra el poder establecido, y presentando su pecho a las balas y a las
bayonetas del Ejército.

A mis preguntas sobre este particular, contestó Ruy que era dueño del
titulado establecimiento un pobre hombre, que había gastado su vida en
aquellos trajines. Un hijo le quedaba, de los tres que tuvo, y ambos
eran tan furibundos patriotas como cuitados menestrales, que empleaban
toda su fuerza física y moral en la conquista de unas sopas, y éstas,
¡ay!, no se lograban todos los días. Representaban allí el
\emph{patriotismo} dos estampas: una, de Espartero a caballo; de Martín
Zurbano la otra, en el acto de ponerle la venda para fusilarle. Yo no
las había visto: estaban adheridas con engrudo a la pared, y del humo y
la mugre apenas se conocían las figuras. Díjome Ruy que ambos, hijo y
padre, tienen la monomanía de las revueltas, y son los primeros en
echarse a la calle en días de motín, y que apetecen los puestos de mayor
peligro. ¿Y por qué lucha esta gente? Por ésta o la otra Constitución
que no conocen, por derechos vagos que no entienden, o por idolatría
fetichista de hombres y principios, cuyas ventajas en la práctica no han
de disfrutar jamás. De fijo que si esta revolución triunfa y tenemos
Milicia Nacional \emph{sobre sólidas bases}, como dice el programa de
Manzanares, estos dos hombres, Erasmo Gamoneda y su hijo Tiburcio, serán
los primeros que se gasten cuanto tienen para endilgarse el uniforme y
salir a pintarla militarmente en procesiones y paradas. Y con esto se
quedarán muy satisfechos, sin reparar que siguen y seguirán tan pobres
como antes, y que irán al sepulcro sin que conozcan ni aun parte mínima
del bienestar posible dentro de los humanos. ¡Inocentes y generosos
hombres! De veras les admiro.

---Señor---me dijo Ruy,---espérese aquí un poquito, mientras yo subo a
ver si está Virginia\ldots{} Ella no bajará, ni le mandará subir a usted
sin saber quién la visita, porque no se le pasa el miedo de la policía
ni aun con estas trifulcas. Siempre está con cuidado. Se viene acá,
porque los Gamónedas, primos de mi padre, son gente de toda confianza
que en ningún caso la venderían.»

Desapareció Ruy por una escalera empinadísima, angosta como de dos
tercias, con los peldaños de fábrica, gastados. Empezaba en un pasillo,
al fondo del taller, y no se le veía el fin\ldots{} Yo no quitaba mis
ojos de los peldaños más altos, últimos para el que subiera, primeros
para el que bajase, y no tardé en ver unos pies de mujer, una falda
azul\ldots{} Pies y falda se pararon cuando sólo estaba visible menos de
la mitad inferior del cuerpo. Yo me agaché para ver algo más\ldots{} La
media figura seguía inmóvil. «¿Será \emph{Mita?»} me decía yo. Salí de
dudas cuando ella, doblándose por la cintura, me mostró su cabeza
ladeada al nivel del techo del taller\ldots{} Con una exclamación de
júbilo avancé hacia la escalera, y ella gritó: «Pepe, Pepillo, pero
¿eres tú de veras?» Bajó dos escalones y me alargó su mano para darme
apoyo y guía en aquella subida gimnástica\ldots{} Sin soltarme de la
mano, me llevó a un aposento de bajo techo, pobrísimo, lleno de
estrafalarios objetos, herramientas y cacharros. En el fondo obscuro,
una mujer de mediana edad, sentada cerca de un anafre, cuidaba de los
pucheros puestos a la lumbre. Era la dueña de la casa\ldots{} Después de
presentarme, Virginia me acercó una silla de paja desfondada, y en otra
se sentó ella. «Me parece mentira que nos vemos, que me ves tú---fue lo
primero que ella dijo en cuanto nos sentamos.---¿Sabes, Pepe, que por
primera vez, en mi vida de salvajismo, siento\ldots{} no sé cómo lo
diga\ldots{} vamos, que me da vergüenza de que me veas en esta facha?»

La tranquilicé, mirándola bien y apreciando con rápido examen toda su
persona, de pies a cabeza. Entiendo que está más bella de salvaje que lo
estuvo de señorita y señora, y que los efectos del sol y el aire superan
a cuantos cosméticos inventa la industria del tocador. No obstante, se
advierte en su rostro la fatiga del trabajo duro, que acabará por
deteriorar su belleza si no le depara Dios un vivir reposado. Entre la
Virginia de Madrid y la \emph{Mita} de los bosques, entre la damisela
frívola y la dríada correntona, ha puesto la Naturaleza sus mayores
distancias elementales. Es ya otra mujer: figura, modales, expresión, y
hasta la voz, han cambiado; conserva la gracia y el ingenio. Hablando
con ella largo rato, pude advertir que sobre sus facultades brilla hoy
un sol nuevo que todo lo ilumina, la razón, antes apenas perceptible,
como un resplandor de aurora entre brumas. Noto que se han desmejorado
extraordinariamente las manos, antes blancas, finísimas, de perfecta
forma, hoy ásperas, coloradotas; los dedos, que fueron los más bellos
instrumentos de la holgazanería, son hoy duros, acerados, con lóbulos
que marcan la deformación de los huesos, por causa de la ruda faena de
lavar ropa en agua muy fría\ldots{} En su vida silvestre ha sabido
\emph{Mita} conservar la limpieza y corrección de su dentadura; su
peinado no es del estilo de pueblo, con moñitos y picaporte, ni tampoco
el que en Madrid se usa, sino más bien un estilo propio suyo, sencillo y
airoso; el calzado muy tosco, y bastante usadito, disimula la pequeñez y
buena forma de sus pies. En su ropa, de todo hay: remiendos, agregados,
telas que lucieron en Madrid, y otras que proceden del mercado de
Bustarviejo, así como el corte del cuerpo denuncia los figurines de
Miraflores de la Sierra.

Las primeras expresiones de \emph{Mita} fueron para sus padres, dulce
recuerdo acompañado de la indispensable ofrenda de lágrimas. Como yo
hablase de posible reconciliación, con el solo objeto de sondar su
ánimo, me dijo: «Pensar en eso es locura, Pepe. Para volver a llamarme
hija, mis padres me pedirán que deshaga yo todo lo hecho desde mi fuga,
y que me ponga el capisayo de un arrepentimiento que me parece tan
absurdo como si el sol saliera por Poniente. ¡Arrepentirme yo de lo
único bueno que he sabido hacer en mi vida! Esto no lo verá mi
familia\ldots{} Por encima de mi familia está \emph{Ley} y el amor que
le tengo. Los padres son padres, y una les quiere porque a ellos debe la
vida; pero sobre todos los amores está el del hombre que será padre de
los hijos que una tenga\ldots{} ¿No lo ha establecido así el mismo
Dios?\ldots{} El amor entre hombre y mujer ha de mirar más a lo que ha
de venir que a lo que pasó. ¿Me das en esto la razón?

---Te la doy, hija\ldots{} Pero es lástima que por algún medio no puedas
consolar a tus padres de la tristeza en que viven.

---Pues busca tú ese medio, Pepe; búscalo con ayuda de los Cuatro
Evangelistas, de los Siete Sabios de Grecia y de las Nueve Musas; porque
yo he pensado mucho en ello, y no veo de dónde puede venir ese consuelo,
que deseo más que nadie. Que maten al que se llamó mi marido por la
Iglesia, o que reformen todo ese catafalco de la Religión y la
Sociedad\ldots{} A ver, a ver\ldots{} vengan esos guapos reformadores y
consoladores. Yo, dispuesta estoy a todo\ldots{} a todo lo que quieran,
de \emph{Ley} para abajo\ldots{} porque lo que es sin \emph{Ley}, siendo
\emph{Ley} menos que el mundo entero, que no me hablen a mí de
arreglitos\ldots»

Por giro natural, la conversación fue a parar a su hermana. «Sácame de
dudas, hombre---me dijo.---¿Es dichosa Valeria? ¿Está contenta de su
marido? En Valeria pienso cuando me sobra algún rato del tiempo que
tengo que consagrar a mis cosas\ldots{} y no sé por qué se me figura que
mi hermana no es feliz\ldots{} Siempre tuve a Rogelio por un tarambana;
sería milagro que, por la sola virtud de las bendiciones de un cura, se
volviera listo y bueno el que de soltero no inventó la pólvora, ni supo
hacer nada con sentido\ldots{} No, no me digas que mi hermana es
dichosa, porque no lo creeré\ldots{} Sospecho que se aburre, que se
distrae recibiendo y pagando visitas, y asistiendo a todos los
teatros\ldots{} A no ser que le dé por matar el fastidio en las
iglesias, comiéndose los santos\ldots{} Si es así, de veras la
compadezco.»

Respondile que yo también dudo de la felicidad matrimonial de Valeria, y
que ésta no tiene el mal gusto de distraer sus ocios en devociones
insustanciales, entre clérigos y beatas. La señora de Navascués,
desatendida por su esposo, busca en los trapos elegantes y en los
muebles de lujo y novedad el regocijo de su alma. Mimada por sus padres,
Valeria es protectora de los que se dedican a la importación de telas
suntuarias y de tapicería y ornamento de casas nobles. No hace muchos
días me dijo: «¿Cuándo volverá Rogelio? No quiero estar sola: le aguardo
y le tiemblo\ldots{} todo me lo estropea. ¡Es tan bruto!\ldots{} En
cuanto entra en casa, se tumba en los sillones de la sala, forrados de
terciopelo, y allí echa sus siestas\ldots{} como si mis sillones fueran
camastros de campaña. Por más que le riño, no hace caso. Las colchas de
seda, que cubren las camas durante el día, y otras cosas que son de puro
adorno, no le merecen ningún respeto. A lo mejor se quita las espuelas y
las pone en el platillo de ágata que tengo en la chimenea de mi
gabinete; las alfombras las trata como si fuesen esteras; entra con las
botas llenas de barro y todo me lo deja perdido\ldots{} La mantelería
fina la he retirado del uso diario, porque\ldots{} parece que lo hace
adrede\ldots{} siempre que come en casa, derrama el vino y hace mil
porquerías\ldots{} En fin, que es muy bestia\ldots{} no se hace cargo de
mis afanes para tener la casa tan bien adornada y tan decentita.»

Todo esto le conté a Virginia para que se enterara del estado
psicológico de su hermana. Me oyó con interés, mostrando sorpresa,
disgusto\ldots{} después se distrajo, haciendo menos caso de mí que de
su propia inquietud y sobresalto por la tardanza de \emph{Ley}. «¿Qué te
pasa, hija?\ldots{} ¿Esperas a tu hombre?\ldots{} ¿Temes por él?

---Siempre temo, Pepe, y no puedo estar tranquila---dijo Mita mirando a
la calle por un ventanucho no mayor que su cabeza.---Tengo una fe ciega
en que Dios ha de guardar a \emph{Ley} y librármele de todo daño; pero
la fe es una cosa y el temor es otra. No estoy tranquila, y las horas
que pasa en esas calles, disparando sus armas, se me han hecho
siglos\ldots{}

---Si tienes fe, no temas\ldots{} Yo le he visto, serían las
cuatro\ldots{} Estaba en la plaza Mayor recogiendo heridos\ldots{}

Rodrigo corroboró este informe; pero \emph{Mita}, sin acabar de
tranquilizarse, mandó al hojalatero que se diese una vuelta por la plaza
Mayor y calle Imperial. Luego seguimos hablando de Valeria y de
Navascués, y contesté como pude al sin fin de preguntas que me hizo
acerca de ellos y de los Rementerías, hijo y padre, sin ocultar el
desprecio que estos le merecen. De los míos también hablamos. Tanto se
interesó Virginia por María Ignacia y por mi niño, que hube de referirle
las gracias de éste, y hacer cuenta de los dientes que le han salido.

«Sácanos de una duda, Virginia. Ni mi mujer ni yo hemos podido
desentrañar el significado de tu nombre salvaje. ¿Qué quiere decir
\emph{Mita}?

---Tonto, el amor tiene lengua de niño para abreviar los nombres. Al
declaramos libres, quisimos olvidamos hasta de cómo nos
llamábamos\ldots{} Él me decía \emph{Mujercita}\ldots{} y quitando
letras y letras, vino a parar en \emph{Mita}\ldots{} Yo, sin saber cómo,
convertí el Leoncio en \emph{Ley}\ldots{} Los salvajes, ya lo sabes,
cuando no tienen otra cosa que comer, se comen las sílabas\ldots{}

En esto llegó Rodrigo diciendo que detrás de él venían su tío Gamoneda y
su primo Tiburcio\ldots{} A Leoncio nada le pasaba. Entró un momento en
casa de su hermana Lucila\ldots{} Pronto llegaría. Tranquilizadas
Virginia y la otra mujer, activaron la comida que hacían en el anafre, y
pusieron la mesa. Entraron el Erasmo y su hijo, satisfechos, alabándose
de su ardimiento, y de haber causado a la tropa el mayor daño posible.
Por la noche tendríamos barricadas. Hijo y padre eran hombres de talla
menos que mediana, desmedrados, paliduchos. El tizne de sus rostros y el
desgaire de su ropa derrotada amedrentaría a cualquiera que se les
encontrase de noche en un camino solitario. Arrimaron sus escopetas a la
pared y vinieron a saludarme, a punto que \emph{Mita} les decía mi
nombre y mi antiguo conocimiento con su familia. Hablamos de la
revolución, y de lo que vendría o debía venir. «Para mí---dijo el
fabricante de obleas y lacre,---la partida está ganada. La Reina no
tendrá más remedio que llamar a la gobernación a los hombres del
Progreso, y lo primero que \emph{pongan} será la Milicia
Nacional\ldots{} Con Milicia no puede haber \emph{polaquismo}, ni
pillería, ni chanchullos. Ya estaremos al tanto para llevar al gobierno
por buen camino\ldots{} y todo marchará como Dios manda, y habrá pan
para las clases\ldots{} ¡Abajo el monopolio!» De estas y otras frases
que luego echó de su boca tiznada, colegí su inocente optimismo. Pensaba
que con el establecimiento de la Milicia Nacional se venderían más
obleas, más lacre y más fósforos.

Alguien subía la escalera silbando, canturriando. Con decir que desalada
corrió \emph{Mita} a su encuentro, se dice que era Leoncio el que subía.

\hypertarget{xxvii}{%
\chapter{XXVII}\label{xxvii}}

\emph{No sé cuántos de Julio}.---Leoncio era: su alegre rostro, su
gallarda soltura me cautivaron desde que entrar le vi, reconociendo en
él un hermoso ejemplar de la raza de Ansúrez. Su varonil belleza
respiraba salud, fuerza, y un perfecto equilibrio de los dos elementos
que nos componen, el animal y el hombre. «Éste es \emph{Ley}---dijo
\emph{Mita} haciendo las presentaciones con una sencillez encantadora,
su mano en la mano de él.---\emph{Ley}, aquí tienes a nuestro amigo
Pepe, que, aunque nada me ha dicho todavía, nos protegerá, ¡vaya si nos
protegerá!\ldots{} en la cara se lo conozco. Es un buenazo, y como
nosotros, tiene ideas libres.» Con cierto embarazo me saludó Leoncio. Yo
le animé con mi afabilidad sincera, y él se arrancó a decirme: «Don
José, puede creerme que, antes de conocerle, yo le quería, por lo que mi
\emph{Mita} me contaba de usted\ldots{} Muchas tardes, muchas noches
hemos hablado de usted largamente, calculando lo que el señor haría por
nosotros si nos viéramos entre las garras de la curia, por el aquel de
casarnos por nosotros mismos.

---Sí haré, sí haré---dije yo con efusión de simpatía. Y él prosiguió
repitiendo el último concepto: «Casarnos por nosotros mismos, y echarnos
las bendiciones\ldots{} Perdone el señor don José que hable tan a lo
bruto, y que no sepa decir los verdaderos nombres de cada cosa\ldots{}
Poca instrucción tuvo un servidor\ldots{} y luego, como hemos vivido
\emph{Mita} y un servidor tan a lo salvaje, se nos iba marchando de la
memoria todo el vocablo fino\ldots{} No gastábamos más que las palabras
precisas para entendernos\ldots{} y cada día\ldots{} nos entendíamos con
menos palabras.

Le hice sentar a mi lado. \emph{Mita} no se sentó, porque la reclamaba
el trajín de la próxima comida. Iba y venía, moviéndose graciosamente,
desde la mesilla en que ponían los platos, sin mantel, y el rincón en
que Leoncio y yo estábamos. Sintiéndome poseído de inmensa piedad hacia
los que ya miraba como amigos de mi predilección, casados a contrafuero,
burladores de toda ley, les aseguré que yo les protegería contra viento
y marea. Prometí yo lo que quizás no podría cumplir, sin desconocer las
dificultades del asunto. Pero en España todo se puede, aquí donde lo
provisional es eterno, donde lo ilegal se legaliza, y no hay montes que
no se muevan con el influjo personal y las recomendaciones\ldots{}
Hablando luego de las enconadas luchas de aquel día, Leoncio me contó
que tenía muchas ganas de andar a tiros con los del gobierno, y sólo
para desahogar su apetito y \emph{dar gusto al dedo} había venido a
Madrid con \emph{Mita}. Se alabó de ser un buen tirador, y de conocer a
la perfección el mecanismo de las armas de fuego. «Vea usted esta
pistola---me dijo mostrando la que con la escopeta había dejado al
entrar.---Es un arma de nuevo sistema, y casi desconocida en Madrid. La
inventó un norteamericano, un \emph{Mister Colt}\ldots{} se llama
\emph{pistola giratoria}\ldots{} también la llaman
\emph{revólver}\ldots{} El armero ése del 10 de la calle Mayor la tiene
de venta. Pero aquí, como no saben manejarla, los pocos que la han
comprado, la descomponen a los primeros tiros. Ésta me la dio un amigo
como cosa que no servía para nada. Yo la examiné, y en el taller de
Rosendo, en esta misma calle, la puse como nueva\ldots{} Vea usted, se
carga de una vez para seis tiros\ldots{}

Cuidado, Leoncio, no se escape una bala y me deje en el sitio\ldots{}

---Está descargada: no tema usted. Pues hoy la estrené en los portales
de Bringas con un resultado magnífico. Arrimado a un pilarote de
aquéllos, me harté de apuntar a mi gusto: no perdía ni un tiro\ldots{}
Crea usted que con la rabia que les tengo a los que mangonean en la
Nación, se me afinaba la puntería. Yo me hacía cuenta de que por cada
bala que yo mandaba, había en la otra banda un enemigo nuestro que caía
patas arriba. \emph{«Pim}\ldots{} ésta para el suegro de
\emph{Mita}\ldots{} \emph{Pim}: ésta para el cura que casó a
\emph{Mita}\ldots{} ésta para su tía Cristeta\ldots{} \emph{Pim},
\emph{pim}: éstas para los padrinos de su boda\ldots{} ésta para los que
duden que \emph{Mita} es mi mujer\ldots{}

---Y con todo ese furor, amigo mío, y eso de mandar las balas con
sobrescrito como si fueran cartas, lo que ha hecho usted es matar a unos
cuantos soldados inocentes\ldots{}

---No sé, no sé a quién he matado: También pudieron ellos matarme a
mí\ldots{} Yo tiro contra los \emph{del Gobierno}, y caiga el que caiga.
Esto son las guerras\ldots{} Y si por matar yo a muchos de allá, viene
un gobierno que ponga las cosas en su punto, permitiendo que los mal
casados se descasen, y que todo se ordene como es debido, y los pobres
puedan respirar, unas cuantas vidas nada significan.»

Pusiéronse a comer, no sin invitarme cada uno por sí, y en coro. Por
causa de las porquerías que metí en el buche en la casa de Sotero, no
tenía yo ni asomos de apetito. Si lo tuviera, seguramente habría
participado de la pitanza de aquella pobre gente. No cabían ellos en la
mesa angosta, y Rodrigo y su hermano comían de pie, cogiendo los dos del
plato de \emph{Mita} lo que llevaban a las bocas con la cuchara o el
tenedor de peltre. «Aunque tuvieras ganas---me dijo \emph{Mita},---no
podrías comer en tanta pobreza. Este guisado que nos sabe tan rico, a ti
te repugnará, como las herramientas con que comemos.» Y al decir esto,
volviéndose en la silla, y mostrándome la feísima cuchara, el movimiento
de sus hombros y de la cabeza desdecía del salvajismo pobre, pues fue
movimiento de gran señora. Yo me excusé. Bebí el vino con sabor a pez
que me ofreció Leoncio, y comí unas almendras con que graciosamente me
obsequió \emph{Mita}. Esto y pasta de higos era el único postre.
Comiendo con voracidad, Erasmo Gamoneda explanaba teorías de gobierno, y
profetizaba los próximos acontecimientos políticos. «Si ganamos, vendrá
un gobierno de hombres del pueblo, pudientes, y lo primero que hagan
será decirnos a todos que nos armemos. Milicia Nacional al canto, y ¡ay
del gobernante que no ande derecho! Ladrocinio y agios no consentimos.
Mucho ojo, caballeros, que aquí está el pueblo armado para vigilar, para
deciros si vais mal o vais bien. Y que tendremos de todo: Infantería,
Caballería, Artillería\ldots{} Yo seré de Artillería, como la otra vez,
por lo que sé de polvorista y de bombista\ldots{} pues de todo habrá.
Ingenieros también, y Ambulancias, y hasta nuestro poquito de Clero
castrense\ldots{} Conque, mucho ojo, caballeros.»

Tiburcio Gamoneda, que se había fogueado en diferentes puntos de Madrid,
nos contó que si terribles fueron los combates de las plazas del Ángel y
Antón Martín, donde Bartolomé Gracián con sus valientes \emph{le quitó
los moños al fantasioso} de Gándara, también se habían \emph{machacado
las liendres} paisanaje y tropa, allá en el tras de la Universidad,
plazuela de los Mostenses y calle del Álamo\ldots{} Pues en la parroquia
de Santiago y en toda la \emph{caída de calles} que bajan a la plaza de
Oriente, la tremolina fue superior, con una de tiros que daba gloria
oírlo, más de cuatro muertos en las calles, y uno en un balcón, con
medio cuerpo fuera\ldots{} Entraron dos vecinos, el estañero del número
inmediato, fabricante de jeringas y otros objetos, y un cacharrero de
Puerta Cerrada, tratante y expendedor de sanguijuelas. Venían ya
preparados para las faenas de la noche, con escopeta en mano y pistola
en cinto. Dijeron que las tropas \emph{liberticidas} no se atrevían a
salir de sus posiciones. Por la noche, Madrid se cubriría de barricadas.
Un día más de constancia y valor, y la \emph{masa patriótica}, dueña del
campo ya, podrá escoger gobernantes a su gusto. Ya se decía que la Reina
quería entenderse con el pueblo, y formar un Ministerio de plebeyos
ilustrados; pero que no la dejaban. Los \emph{verdugos de la Libertad} y
\emph{los secuaces de la Reacción} tenían a Su Majestad con las manos
atadas, como quien dice, y armaban mil enredos para quitarle la buena
voluntad.

Sentí lástima de aquella pobre gente, y también admiración muy viva,
pues desde la hondura de su vida miserable se lanzaban impávidos a la
conquista de una España nueva. Cuanto tenían, las vidas inclusive, lo
sacrificaban por aquel ideal de pura soñación, y por un programa de
Gobierno que no habrían podido puntualizar, si fueran llamados a
realizarlo. Y después de pasarse largos días y noches en tan peligrosas
andanzas, volvería cada cual a sus obligaciones. El uno seguiría
fabricando obleas y lacre; el otro, jeringas, y el tercero vendiendo
sanguijuelas, para ganar un triste cocido y vivir estrechamente entre
afanes y miserias. Todo lo soñaban, menos llegar a ser ricos, o al
menos, vivir con desahogo. ¡A luchar y a pelearse por un principio
fantástico, vagaroso, como las formas de hombres y animales que se
dibujan en las nubes! ¡Y luego volver al trabajo, a las privaciones, a
la insignificancia! ¿Cómo no admirarles si, en medio de su ruda
ignorancia, advierto en ellos una elevación moral que en mí propio y en
los de mi clase no veo, no puedo ver, por más que la busco?

Esto, con más concisa palabra, dije a \emph{Mita}, mientras los hombres,
en grupo aparte, hablaban de su plan defensivo. «Yo no entiendo de esas
cosas---respondió la salvaje;---pero quiero que peleen\ldots{} y no
importa que muera alguno, con tal que no sea \emph{Ley}; que haya mucha
trapisonda y se estremezca todo eso que llaman \emph{el Trono y el
Altar}, para que resulte vencedora la Libertad. Sí, Pepe: que me traigan
Libertad\ldots{} y gobiernos muy libres\ldots{} ¿No vendrá también,
entre las libertades nuevas, el libre matrimonio, y el descasarse, que
es, como quien dice, divorcio?»

Con una sonrisa le contesté, por no atreverme a manifestarle de palabra
mi escepticismo acerca de los progresos de nuestro país en la
legislación matrimoñesca. Luego, viéndola descorazonada, le dije: «Sí,
que peleen y se destrocen, a ver si la sangre nos trae mucha Libertad,
pero mucha; ideas nuevas, prácticas de otros países. Quién sabe,
\emph{Mita}: no pierdas la esperanza. Esta revolución será tan tremenda,
que el Altar y el Trono quedarán necesitados de una mano de carpintería
que los componga y los deje como nuevos. Confiemos en la Providencia;
esperemos que después de estas guerras no se diga, como otras veces, que
todo queda lo mismo que estaba.

---¡Ay, Pepillo!, en la manera de decirlo te conozco que no tienes
fe\ldots{} ¿Tú piensas que todo quedará lo mismo?

---No, hija, esperemos\ldots{} Vendrá libertad, libertad a chorros, a
torrentes\ldots{}

No quise seguir tratando este punto, por no empañar con mi escepticismo
las ilusiones de aquella Libertad áurea con que \emph{Mita} soñaba, y
que, según ella, debía organizar en forma nueva el mundo de los
enamorados. Leoncio, llegándose a nosotros, dijo que si la caída del
\emph{polaquismo} y el triunfo de los patriotas traía mucha Libertad,
vivirían ellos tranquilamente en un pueblo; pero que si la tal Libertad
no venía, grande y con alma, poniendo patas arriba todo lo existente, se
irían a Marruecos, y tomarían trazas y habla de moros, para vivir
tranquilos. En Marruecos tiene él un hermano que se ha hecho al vivir
berberisco, y ya no le conocería por español ni la madre que le
parió\ldots{} Esto y algo más que dijo del hermano marroquí, me movió a
preguntarle por su hermana Lucila, que a dos pasos de donde estábamos
vivía. Sin dar reposo a la lengua, Leoncio limpiaba sus armas y se
proveía de pistones para una larga función de guerra, ayudándole
solícita la que me atrevo a llamar su mujer, y su hermanillo, aunque más
entendido en violines que en pistolas. Lo primero que contaron de Lucila
fue que es dichosa en su matrimonio con el señor Halconero: ha sabido
adaptarse a la vida campesina, y no se halla bien fuera de su casa y
tierras. Vino a Madrid acompañando a su marido, por diligencias de éste,
y esperan que amaine la revolución para meterse en la tartana y volverse
al pueblo. «Es tan guapa mi cuñada---dijo \emph{Mita} con extremos de
admiración,---que cuando la veo me quedo embobada, y no sé quitar de
ella mis ojos. Pienso que escultores y pintores debieran tenerla siempre
delante para sacar del rostro de ella toda la belleza de sus cuadros y
estatuas, porque de seguro no ha criado Dios modelo más perfecto de
hermosura de mujer\ldots{} modelo de que se podrían sacar los retratos
de Diosas y Vírgenes para los museos y las iglesias.

Como hablara yo de su gordura, recordando lo que Rodrigo aseguró, los
dos salvajes se echaron a reír, de lo que no se abroncó poco el pequeño.
«Está en el punto preciso de las buenas formas de mujer---dijo
\emph{Mita}:---ni un punto más ni un punto menos que la medida y peso
justos.

---Su cuerpo---indicó Leoncio,---es como su cara: la perfección, señor
don José. Cuantos la ven lo dicen. ¿Sabe por qué ha dicho este tontaina
que Lucila está de libras? Porque él tiene en su cabeza un tipo de mujer
enteramente espiritado, y a toda la que no sea un palo vestido, la llama
gorda.

---Ya ves, Pepe---dijo \emph{Mita} riendo:---él es como una espátula, y
tiene una novia que allá se va en corpulencia con el arco del violín.
Perdidamente enamorado está de semejante lombriz\ldots{} Mira, mira qué
colorado se pone cuando hablamos de sus amores. La verdad, Pepe, no es
fea la muchacha.

---Sería bonita si echara unas pocas de carnes\ldots{} Pero a Rodrigo
así le gusta, porque está enamorado de lo magro. Magro es lo que toca en
el violín; magro todo lo que piensa.

---Su \emph{bello ideal}, como se dice, es un alambre, y cuando ve comer
a su novia pierde la ilusión. ¿Verdad, Rodrigo? Quieres que todo sea
música; que las vidas sean, como las notas musicales, almas sin cuerpo.

Los tres nos reíamos de mi escudero, que ruboroso se defendía torpemente
con monosílabos. Reconoció al fin que se había equivocado al decir que
su hermana está gorda. Fue aberración de sus sentidos, incapaces de
apreciar la verdad material y la verdad numérica, por natural desvarío
de gran artista. Y no sólo había desvariado en lo de la gordura, sino en
lo de la fecundidad, pues hablando los cuatro aquella tarde, se deshizo
el engaño de que Lucila era madre de numerosa prole. No tiene más que un
niño, precioso como un ángel, y no hay indicios hasta hoy de que aumente
la descendencia de Halconero. «Estos benditos músicos---dijo \emph{Mita}
con agudeza y donaire,---no aprecian el número, y de una cosa hacen
siempre dos. No hay aria que no tenga su \emph{ritornello}. Así este
tonto vio un niño; luego vio otro\ldots{} y aun creyó que iba a venir un
tercer niño.

---Es verdad que me equivoco en el número, y que duplico los objetos, y
que, por gustarme lo flaco, paréceme gordo lo que no lo es---dijo el
violinista burlándose de sí mismo.---Esto será porque mi maestro don
Juan Díaz me dice a cada instante: «Afina, hijo, afina\ldots{} no
rasgues el sonido. Hay que ahilar, ahilar\ldots{} busca el hilo del
sonido\ldots{} el hilo delgado, delgadísimo.» Y cuando me vuelvo yo
tarumba para que el sonido me salga delgado y puro, el maestro me dice:
«Repite, hijo: otra vez\ldots{} otra.»

\hypertarget{xxviii}{%
\chapter{XXVIII}\label{xxviii}}

Detonación cercana nos hizo estremecer. Recogió Leoncio todos sus
bártulos de guerra para lanzarse a la calle. \emph{Mita} quiso detenerle
un rato más, y no lográndolo, dijo que ella también saldría\ldots{}
Salimos todos. ¿Qué habíamos de hacer allí? En la calle vimos sin fin de
paisanos que subían a la plaza por la Escalerilla. No tomó Leoncio
aquella dirección, sino la de Latoneros, donde le aguardaban los amigos
a cuyo lado combatía siempre. Entramos en una tienda de cedazos,
ratoneras, cucharas de palo y molinillos para el chocolate. Celebrose
allí una especie de consejo de guerra o conferencia entre caudillos. No
me enteré bien de lo que discutían; sólo al fin pude comprender que
todos los \emph{patriotas} allí presentes, que eran más de siete y más
de ocho, declaraban que no se batirían a las órdenes de Gracián, a quien
tacharon de orgulloso y déspota, execrando su vanidad y su afán de
lucirse él solo y de tomar para sí las glorias de los demás. No llegaron
a mi oído razonamientos más detallados ni pormenores precisos de la
conducta del héroe popular, porque \emph{Mita} y el hojalatero me
hablaban de cosas diferentes a las cuales no podía negar mi atención.
Cuando de la tienda salimos, anochecía ya. En la calle obscura veíanse,
como sombras fugaces, los \emph{patriotas} que acudían a sus puestos.
Despidiose Leoncio de su mujer, con la orden de que se fuese a la casa
de Lucila y le esperase allí a media noche, o cuando tuvieran fin las
refriegas que se preparaban. Le vi partir sereno, y acompañé a
\emph{Mita} hasta los soportales de la calle de Toledo, frente a la
Imperial, notando que a pesar de su fe ciega en la invulnerabilidad de
\emph{Ley}, la salvaje no gozaba de tranquilidad. Es difícil que la fe y
las balas se pongan de acuerdo, y a lo mejor la divinidad que protege a
ciertos hombres escogidos sufre lamentables distracciones. Sobre esto me
dijo algo la pobre \emph{Mita} cuando íbamos hacia los porches,
añadiendo que si Dios se volviese atrás de lo dicho y dejase morir a
\emph{Ley}, ella se iría para el otro mundo sin perder momento.

Ni \emph{Mita} me invitó a subir a la habitación del señor Halconero, ni
habría yo subido aunque me invitara. El síntoma más penoso de mi
\emph{Ansia de belleza} era que la atracción y el miedo del ideal se
juntaban en un punto. Después me dijo Ruy que el señor Halconero es muy
celoso, y aunque Lucila no le da motivo de escama, no gusta de que en su
casa entren hombres, ni menos señoritos. Yo no entraría, ni tenía para
qué. Entendía que mi dolencia, más punzante y angustiosa en aquel triste
anochecer, requería como eficaz remedio el largo pasear y el tender mi
espíritu por diferentes calles. Así lo hicimos, metiéndonos por la calle
del Grafal y la Cava Baja hasta Puerta de Moros, regresando luego por la
Costanilla de San Pedro y calle del Nuncio. Animación y bulla vimos por
todas partes; ventanas y balcones con luminarias en todos los sitios
dominados por el pueblo; obscuridad siniestra en los parajes ocupados
por tropa. El acaso nos trajo de nuevo al punto de partida. Desde Puerta
Cerrada habíamos querido salir a Platerías por la plazuela de San
Miguel, para irnos a casa por las Hileras y Santa Catalina de los
Donados; pero no hallamos camino franco. Tenebrosas estaban aquellas
barriadas, y a cada momento daban los soldados el \emph{quién vive}.

Cuando llegamos a los portales de la calle de Toledo, ya habían echado
los insurrectos el fundamento de la barricada, la cual avanzaba en
ángulo para hacer frente con una de sus caras a la calle Imperial, y con
otra a la de Toledo. Mi admiración de aquellos inocentes vecinos subió
de punto viéndoles trabajar como hormigas en el parapeto que había de
protegerles contra las iras del poder público, y no sólo sacrificaban su
vida y su tiempo por un ideal político que entendían como la escritura
chinesca, sino que también ponían en ello el ajuar pobre de sus casas.
Era de ver la diligencia con que hombres y mujeres, y también
chiquillos, acarreaban de las casas trastos y trebejos para echarlos en
el montón, y luego ponían encima los colchones, privándose de dormir en
blando con tal de ofrecer cómodo abrigo a los defensores del pueblo. ¿En
dónde y cuándo se ha visto mayor abnegación, ni entusiasmo más
candoroso? El señor Erasmo Gamoneda, que como artillero con pujos de
ingeniero dirigía la barricada, me dijo que los españoles sacrifican
colchones, esteras, y aun sofás de paja de Vitoria, en \emph{aras del
patriotismo}. Cuando les pareció que la barricada tenía bastante altura
y que la escarpa de ella ofrecía resistencia eficaz a las balas
enemigas, se ocuparon en decorar la fortificación. Eran pueblo, que es
como decir niños, y el poder imaginativo les arrastraba a la juguetería.
En el extremo de la derecha, tocando al portal último, pusieron un
retrato de Espartero clavado en la pared; al otro extremo, unas banderas
en pabellón, donadas por un vecino ebanista, y que habían hecho su papel
en el adorno de la calle cuando entró doña María Cristina para casarse
con Fernando VII, y en el vértice del ángulo, un lienzo con el retrato
de la Virgen de la Paloma, desclavado del bastidor y muy estropeadito.
Después de servir de imagen titular en una tienda de la calle de
Latoneros en el pasado siglo, estuvo largos años en un portal, con
ofrenda de velas y aceite, parando al fin Nuestra Señora en patrona y
capitana de la plebe amotinada. Desde el palo en que pusieron la Virgen
hasta los dos extremos de la barricada, tendieron cuerdas con banderolas
y pingajos de diferentes colorines; moñas de toros, y el indispensable
cartel de \emph{Pena de muerte al ladrón}.

Dentro de la barricada, en las dos bandas de soportales, tenía la calle
aspecto de feria. Paseamos por un lado y otro, viendo las hileras de
tiendas, que de día me parecían cavernas forradas de bayetas y paños. En
noche de revolución estaban cerradas, o a media puerta, para entrar y
salir la gente armada. Las mujeres y los niños se refugiaban en los
entresuelos, tan lóbregos de noche como de día. Cerradas las tiendas, se
destacan los rótulos: aquí nombres muy acreditados en el comercio, como
el famoso \emph{Tío Rico}, celebridad choricera; allí denominaciones
simbólicas al uso, como \emph{La Perla}, \emph{El Jazmín}, que anuncian
ropa de niños, camisería y género de punto. De todo hay en aquellas
grutas, donde guarda su hacienda un pueblo de afanosas hormigas. Desde
la puerta de una de estas tiendas señaló mi escudero al piso segundo de
la casa de enfrente, dirigiendo mi atención a una ventana más iluminada
que las demás de la calle, y me dijo: «Vea, señor: aquella ventana de
tanta luz que parece un retablo, es de las habitaciones donde viven mi
hermana Lucila y su marido don José Halconero.» «Liberal de veras será
tu cuñado---observé yo,---cuando tan espléndida luminaria pone en su
casa.» Y él: «Liberal y patriota fue, según dicen, en tiempo de aquel
que llamaron \emph{Héroe de las Cabezas}, verbigracia, Riego; y él mismo
cuenta que en una batalla que dieron los Milicianos en el Arco de
Triunfo, el día tantos del mes de Julio de un año de aquel tiempo, se
batió como un león, y sacó dos heridas en la cabeza que por poco le
cuestan la vida. Hoy sigue liberal y partidario del Progreso; pero ya no
le queda más que el compás, y todo lo que dice es: «¡Ah, en mi
tiempo!\ldots{} ¡Oh, aquellos eran hombres!\ldots» También mi hermana
Lucila es \emph{patriota}, al modo de mujer, clamando por que triunfe el
Adelanto; pero dice que no vendrá civilización grande si no tenemos
antes Diluvio\ldots{} el Diluvio, señor.

---Tiene razón Lucila. En este inmenso secano, no puede haber buena
cosecha sin lluvias abundantes.

Sin hablar más de esto, pasamos al otro lado: nos llamaba Erasmo
Gamoneda con fuertes voces, y en cuanto nos tuvo al habla díjonos que
habíamos de tomar las armas o largarnos de la Plaza; éramos un estorbo y
un cuidado, y no estaban allí los valientes para custodiar a los
petimetres que venían de mirones. Antes de que yo le manifestara mi
ineptitud para el heroísmo, y aun para el manejo de las armas de fuego,
salió Ruy diciendo que bien podía yo permanecer en la Plaza sin
necesidad de cargar el chopo, pues venía como historiador, y a los
historiadores se les respeta en los campamentos por el bien que traen
narrando las hazañas. Al oír esto Gamoneda, cambió de tono, y con gesto
y expresiones corteses demostró la admiración que siente por los que
consagran su ingenio a reproducir y encomiar las glorias de la patria.
«Bien, muy bien, señor mío---me dijo estrechándome la mano.---Ya
leeremos todo lo que Vuecencia escriba de este sufrido pueblo\ldots{} y
si sale esa Historia por entregas, a cuartillo de real, los pobres
podremos comprarla\ldots{}

---Yo---declaró un ciudadano, que a nuestro grupo se acercó, fusil al
hombro---me quitaré el pan de la boca para tener en casa esa Historia y
leérmela de corrido\ldots{} Pero tenga cuidado de que no se le olvide
nada.

---¡Ah!---indiqué yo,---de eso trato, de no perder el menor detalle de
estas grandezas.

---Nos dará luego la Segunda Parte---dijo Gamoneda,---que traerá todo el
relato del establecimiento de la Milicia Nacional\ldots{} Yo, como dije
al señor don José, seré de Artillería, por lo que entiendo de materias
para disparos y explosiones\ldots»

Invitáronme a visitar el local que allí tenían, en la cabecera de la
barricada, un almacén que fue de paños, ya desalquilado y vacío.
Ocupáronlo por ley de guerra los \emph{patriotas}, y en él pusieron su
almacén de provisiones de guerra y boca, con botijos de agua fresca, dos
catres para los heridos, depósito de armas, y líos de esteras viejas
para refuerzo de la barricada. Entramos y nos ofrecieron como asiento
unas cubas vacías. Mientras Gamoneda daba órdenes a unos tipos con
morriones de miliciano del año 22, y que debían de ser ayudantes o cosa
parecida, el otro ciudadano me hacía los honores del \emph{Cuartel
general}, y de paso daba unos toques políticos y sociales: «Para mí,
señor, la Revolución no debe cuidarse sólo de traer más Libertad. Venga,
sí, toda la Libertad del mundo; pero venga también la mejora de las
clases\ldots{} porque, lo que yo digo, ¿qué adelanta el pueblo con ser
muy libre, si no come? Los gobernantes nuevos han de mirar mucho por el
trabajo y por la industria. Hay que proteger al trabajador, y echar
leyes que abaraten el comestible y den mayor precio a las cosas de
fabricación. Yo, señor, soy fabricante de zorros para quitar el polvo a
los muebles. Mi establecimiento está en la rinconada del Almendro, donde
el señor tiene su casa, y puede visitar los \emph{talleres} cuando
guste. La muestra dice: \emph{Hermosilla}. \emph{Zorros y plumeros}.
Hermosilla es un servidor, para lo que guste mandar\ldots{} Mis zorros
son especiales\ldots{} Castiga usted con ellos los muebles de tapicería
sin deteriorarlos, porque empleo material escogido de orillo. Ahora me
dedico también al plumero, que fabrico para los muebles finos de
Francia, que están de moda. Es un plumero tan suave, que se come todo el
polvo y hasta el polvillo, y es de larga duración si cae en manos de
criadas que lo manejen con suavidad, acariciando, señor, acariciando,
sin dar golpes\ldots»

Con todo lo que dijo estaba yo conforme, y así se lo manifesté al bueno
de Hermosilla con franca urbanidad, añadiendo que la Revolución no sería
eficaz si no nos traía un gran desarrollo de las artes industriales.
Luego me quedé solo con Ruy, el cual me llevó por los espacios
interiores de aquel local vacío, que iba a parar a la calle de
Cuchilleros. «Esta puerta---me dijo mostrándome una claveteada y con
fuertes cerrojos,---da a la escalera por donde se sube al piso en que
vive mi hermana Lucila, y por la misma se puede salir a cuchilleros;
pero está cerrada y por aquí no podemos salir. Vámonos por donde
entramos, señor, y veamos de procurarnos un sitio de descanso, ya que no
pueda el señor ir a su casa y yo acompañarle, como es mi deber.» La idea
de volverme al seguro de mi casa me halagó un instante; pero me sentía
perezoso, o más bien, una fuerza de adhesión casi irresistible,
pegajosa, en aquellos lugares me retuvo. Cierto que mi mujer estaría sin
sosiego; pero ya se tranquilizaría viéndome entrar a media noche o a la
madrugada\ldots{} Con esta idea fuimos a la Plaza, y después de vagar
por ella nos refugiamos en el quicio de una cerrada puerta, junto a la
calle de la Sal\ldots{} Fatigado yo y anhelando la quietud, me senté en
el suelo; Ruy se puso a mi lado. Parecíamos dos mendigos: no nos faltaba
más que alargar una mano y soltar la quejumbrosa plegaria para solicitar
la caridad del transeúnte. El vivo resplandor de mi espíritu en aquella
hora triste de la noche, que no parecía sino que cien hogueras ardían en
él, es de tal importancia en estas memorias, que necesito tomarme tiempo
y descanso para referirlo como Dios manda. Mañana, amiguita Posteridad,
seré otra vez contigo.

\hypertarget{xxix}{%
\chapter{XXIX}\label{xxix}}

\emph{¿Julio todavía?\ldots{} No sé en qué día vivo}.---Sigo contando, y
describiré brevemente las batallas que andaban dentro de mí. Mi alma era
toda tristeza, considerando cuán poco soy y cuán poco valgo. ¡Entre
aquellos hombres inocentes y rudos que perciben un ideal y corren ciegos
tras él menospreciando sus propias vidas, y yo, existencia infecunda,
inmóvil pieza de un mecanismo que anda sólo a medias y a tropezones, qué
colosal diferencia! Ellos me parecían materia viva, aunque tosca; yo,
materia inerte, ociosamente refinada. Ellos marchan; yo permanezco
apegado al suelo como un vegetal. Ellos son elemento activo; yo,
formación petrificada del egoísmo y de la pereza. Para consolarme de la
envidia que me punza el corazón, pienso en la barbarie de ellos; comparo
su grosería con mi finura, y su ignorancia con las varias erudiciones de
segunda mano que me adornan. Pero esto no me vale, y en lo mejor de mis
comparaciones, les veo agigantarse, mientras yo, de tanto empequeñecer,
llego a ser del tamaño de un cañamón. Ellos trabajan rudamente todo el
año para vivir con estrechez, y yo vivo de riquezas que no he labrado, y
de rentas que no sé cómo han venido a mí. Y viviendo en la inactividad,
amenizando mis ocios con el recreo de ver pasar hombres y cosas, ellos
se lanzan a la hechura de los acontecimientos, a impulsar la vida
general, y a desenmohecer los ejes del carro de la Historia. Ellos dan
su hacienda corta y su vida, no por el beneficio y mejora de sí mismos,
y de la clase a que pertenecen, sino por la mejora de toda la sociedad.
Si algo bueno resultare de esta revolución, no será para ellos, que
seguirán tan pobres, obscurecidos y bárbaros como antes, mientras
recogen el fruto de la mudanza política los camastrones que han
cultivado y adquirido la agilidad oratoria, o los áureos gandules como
yo.

No me conformo con esta inferioridad a que me condena mi propio juicio,
y evoco toda mi voluntad para ver si en ella encuentro fuerza bastante
con que acometer algo que a tales hombres me iguale. ¿Qué puedo hacer?
¿Coger un arma y lanzarme a la pelea junto a esos admirables ciudadanos?
No, porque ya sé lo que ha de pasarme si me meto a revolucionario de
acción. Me faltará ardimiento, la fiera impavidez ante el peligro. Me
figuro que intento ponerme a disparar tiros en una barricada, y antes de
empezar me sentiré invadido de un sentimiento humanitario, incompatible
con el heroísmo bélico. Vamos, que si suelto el tiro con buena o mala
puntería, y tengo la desgracia de matar a un pobre soldado, he de
afligirme como si a mi propio hermano matara. No, no: he de buscar un
heroísmo que no sea el militar. Pues ¿qué, entonces? ¿Recoger y asistir
a los heridos, exponiendo mi cuerpo a las balas, como si éstas fueran
motitas de algodón? ¿Predicar de casa en casa y de pueblo en pueblo las
doctrinas salvadoras, y no cejar en ello, desafiando persecuciones,
cárceles, presidios y la muerte misma? Estos y otros medios de elevación
moral iban pasando por mi mente, sin que me decidiera por ninguno, pues
aunque todos me parecieran buenos, yo ambicionaba el mejor, el
insuperable. Había de ser algo que yo fuese a buscar a los más eminentes
espacios de la bondad humana\ldots{}

No sé a dónde fue a parar mi desconcertada mente. Sí sé que mis nervios
cayeron en una sedación honda. ¿Yo dormía o velaba? Cualquiera lo
averigua\ldots{} ¿Sentí los ronquidos de Ruy, o es que éste tocaba el
violín? «Ruy---le dije sobresaltado,---eso que tocas ¿es el aria del
\emph{Rapto en el Serrallo}, del amigo Mozart, o un motivo de tu
invención?

---¿Qué motivo ni qué carneros?\ldots{} Despierte, señor, y vea que no
tengo violín, que estamos pasando la noche arrimados a una puerta, en la
plaza Mayor.

---¿Crees tú que yo he dormido?

---¡Anda! Pues no ha soñado poco\ldots{}

---¿Sabes una cosa? Es muy agradable dormir al raso en estas noches de
verano. En la calle, sueña uno cosas más bellas que en casa\ldots{} Y
dime, ¿has oído tú al reloj dar las horas?

---Le oí, señor; pero todas las horas las daba equivocadas. Dio dos
veces la una; dio las once después de las doce, y repitió las dos para
que parecieran las cuatro.

---¿De modo que con ese reloj no sabemos a qué hora vivimos? Así es
mejor. No hay cosa más cargante que saber la hora, y sentir el tiempo
marchar siempre hacia adelante. Yo he pensado que estábamos a prima
noche, y que cuando \emph{Mita} subió a casa del señor Halconero,
subíamos con ella.

---Me parece, señor, que no es verdad que subiéramos\ldots{} Lo que hay
es que usted y yo deseábamos subir; pero no fue más que deseo\ldots{}
quiero decir, que el deseo subió, y nosotros nos quedamos en la calle
viendo hacer la barricada\ldots{}

---Más bonita es Lucila que la barricada, pienso yo\ldots{} y también te
digo que en los ojos de tu hermana están todas las revoluciones.

---Mi hermana es tan bella, que yo mismo, al mirarla, me quedo pasmado.
Creo a veces que Lucila no es mujer, sino diosa, una diosa con disfraz,
que tiene el capricho de pasar temporadas entre nosotros los
humanos\ldots{}

---Ciertamente: Lucila no es de este mundo, sino criatura
celestial\ldots{} Dios la encarnó en una raza escogida\ldots{} porque
has de saber, Ruy, que vosotros, los Ansúrez, sois celtíberos, la raza
primaria. Tu padre es el perfecto tipo de la nobleza española, y tu
hermana, el ideal símbolo de nuestra querida patria\ldots{} Y el hijo de
Lucila es como un príncipe que lleva en sí todos los caracteres de la
realeza: cuando crezca, verás en él la más bella persona, y la más
gallarda, la más generosa. No digo yo que reine; pero sí que debe reinar
y que idealmente reinará\ldots{} A propósito, ¿qué nombre le han puesto
a ese niño?

---José\ldots{} el nombre de su padre.

---Y mío\ldots{} Has de notar que todos los españoles nos llamamos José.
Casi, casi, llamarse José es como no llamarse nada, y tu sobrinito ha de
tener otro nombre, que no conocemos; un nombre que le ha puesto Lucila,
y que sólo ella sabe\ldots{} Porque no dudes que ese niño ha sido
engendrado por el Dios celtíbero, o por el mismísimo genio de la patria.

---Poco a poco, señor Marqués\ldots{} Mire lo que dice. No está bien que
una persona como usted vitupere a mi hermana, señora honrada, más
honrada que el sol, y aunque esposa de un viejo, es tan fiel, tan fiel y
tan pura, que ninguna otra mujer la puede superar.

---¡Si lo sé, hijo: si la tengo por dechado y compendio de todas las
virtudes! Pero lo uno no quita lo otro, querido Ruy.

---Todos cuantos conocen a mi hermana se hacen lenguas de su recato y
honestidad, y mi cuñado Halconero es la persona más envidiada que hay en
el mundo. La gente dice en coro: «Vaya una mujer que se ha llevado este
tío.» Su buen comportamiento, digo yo, es lección que debieran
aprenderse de memoria las demás mujeres.

---Lo sé, lo sé. Pero eso no quita\ldots{} Pudo ser con ella el Dios
celtíbero o el genio de la raza española, conservando sin menoscabo su
virtud y, si me apuran, su virginidad\ldots{}

---Señor, señor, tanto como eso no se puede decir\ldots{} Cállese, por
Dios, o creeré que delira\ldots{} Si no estuviéramos a obscuras, vería
usted que, oyéndole esos despropósitos, me he puesto muy colorado.

---Tú podrás ponerte como un cangrejo, si ése es tu gusto. Yo, sin
cambiar de color, expreso una idea elevada, teológica\ldots{} y en el
terreno de la fe la sostengo. Claro que no podrá demostrarse; pero la
demostración contraria, ¿quién será el guapo que hacerla pueda?\ldots{}

---El señor conoce a Lucila: no es necesario que sea teológica para ser
hermosa y buena como los ángeles.

---Cierto: esto lo sé por espontáneo conocimiento, inspiración si así
quieres llamarlo, porque he tratado poco a tu hermana. Sólo dos veces la
he visto, y en ninguna de esas ocasiones he tenido el honor de hablar
con ella.

---Pues si es así, no conoce el señor lo más bello de mi hermana Lucila,
que es el acento, el metal de voz.

---Sin oírlo, lo conozco, Ruy, por percepción intuitiva. En la voz de
esa mujer cantan todos los ángeles y serafines.

---Así es\ldots{} No han oído los hombres música que a la voz de mi
hermana pueda compararse\ldots{} No puedo hacer comprender al señor cómo
es aquella voz\ldots{} Si hubiera traído mi violín, algo podría decirle
acerca de esto.

---Pues que no se te olvide traerlo, siempre que salgamos a divagar de
noche por las calles solitarias\ldots{} ¿Sabes, Ruy, lo que estoy
reparando? Que alumbra la luna con luz tan clara como si tuviéramos en
el cielo tres o cuatro lunas.

---No es claridad de luna lo que vemos, sino del mismo sol. Señor, es de
día.

\hypertarget{xxx}{%
\chapter{XXX}\label{xxx}}

\emph{Julio\ldots{} todavía Julio}.---La primera embestida de esta
dolencia tan vaga como cruel, a la que he dado diferentes nombres sin
acertar, creo yo, con el verdadero, empezó a fines del 49 y no terminó
hasta mi viaje a Roma el 51. Sufrí entonces desórdenes extraños de la
inteligencia y aberraciones sensorias muy peregrinas; pero nunca llegué,
como en este segundo ataque, a confundir la luna con el sol, y la noche
con el día. Tampoco di entonces en la extravagancia de desayunarme con
buñuelos y aguardiente, como hice en la ocasión que refiero. Fue Ruy
quien me incitó a tomar tales porquerías, que en mi estómago, la verdad
sea dicha, cayeron como veneno, poniéndome de cabeza y voluntad más
perdido y desatinado de lo que estaba. En lo que sí coincidían mi
primero y mi segundo ataque era en el olvido de mi cara familia, en el
amor ardiente al pueblo y en la insana ambición de realizar yo una o más
acciones heroicas, siempre dentro de lo popular; es decir, que mi
quijotismo tenía el carácter de amparo de los humildes por estado y
nacimiento\ldots{}

Hoy, mejorado de tan raras turbaciones, no puedo traer fácilmente a mi
memoria mis acciones de aquel día\ldots{} el día de los buñuelos y el
aguardiente\ldots{} porque desde que embuché aquella ponzoña se me
inició la inconsciencia, y ésta fue aumentando, según me dijo Ruy, hasta
que llegué a ser como autómata que iba y venía, y maquinalmente
funcionaba moviendo los muelles y resortes de mi organismo, sin apreciar
la causa impulsora ni darme cuenta de sus efectos. En mi cerebro ha
quedado el estruendo de los tiros que oí, tiros próximos, lejanos
lejanísimos, y después he sabido por Ruy que eran el lenguaje de la
batalla empeñada en las calles, y me ha informado de las defensas que
hubo en estas o las otras posiciones: barricada en la calle de la
Montera, en Antón Martín, en Santo Domingo, en los Mostenses\ldots{} qué
sé yo\ldots{} Todo Madrid debió de ser barricada\ldots{} Mas si el
estampido de la fusilería y cañonazos era recogido por mi cerebro, nada
del lenguaje humano que en aquella mañana oí persiste en mi memoria. Los
que hablaron conmigo, háganse cuenta de que hablaron con una estatua.

Pero lo más peregrino, entre los muchos fenómenos de inconsciencia de
aquel indefinido lapso de tiempo que duró mi turbación, fue que yo me
encontré armado sin saber quién había puesto en mi mano un fusil. Y nada
me ha causado tanto pasmo y terror como el decirme Ruy con ingenua
convicción que yo había hecho fuego\ldots{} Veinte veces me lo aseguró y
aún no le daba yo crédito. Yo disparé mis armas; alguien me las cargaba,
y vuelta otra vez a disparar\ldots{} Añadió mi escudero que durante una
hora o más me batí en la barricada, y que por mi arrojo y mi desprecio
de la muerte parecía un insensato. ¡Válgame Dios con mi heroísmo sin
saberlo! ¿Por desgracia mía, algún cristiano fue víctima de mi
despiadado, inconsciente furor? Las referencias de Ruy y las de Tiburcio
Gamoneda convienen en que fue Leoncio quien me dio las armas y quien las
cargaba. ¡Y mis locas acciones, trabajo de un maniquí de perfeccionado
mecanismo, hiciéronme pasar por valiente a los ojos de los tiradores de
verdad!\ldots{}

Si nada quedó en mi memoria de los disparos que hice, según cuentan, con
marcial coraje, nada recuerdo tampoco de haber comido. Ruy me asegura
que sí. ¡Vaya por Dios! Comí pan, aceitunas negras, un pedazo de cecina,
medio arenque, y apuré un vaso de vino\ldots{} Lo más singular y
maravilloso es que mi escudero jura por la salvación de su alma que lo
comí con apetito\ldots{} Mi memoria no recobró su poder hasta después de
anochecido, y la primera prueba del renacer de la preciosa facultad fue
verdaderamente muy desagradable. Hallábame yo en la calle de Botoneras:
me complacía en observar cómo iba recobrando el sentido del lugar y el
tiempo, y para comprobarlo reconocía la casa de Sotero, y apreciaba la
entrante noche\ldots{} En esto vi que un hombre a mí se llegaba: era
Gracián. Su presencia me hizo temblar. Aunque ni su rostro ni su actitud
indicaban hostilidad, despertó en mí un horrible miedo. Con palabra
balbuciente contesté a sus preguntas, que no sé si se referían a mi
salud o a mi valentía. Dudé si eran manifestaciones de amistad o de
burlas; y deseando perderle de vista, porque su mirada me causaba pavor,
antipatía, consternación, antes que él se apartase me escabullí yo
rozando con la pared de las casas\ldots{} Intenté dar el pretexto de que
alguien me llamaba; pero no sé si lo di\ldots{}

Entrada la noche, y sosegado ya del miedo que me causó Gracián, tuve
mayor prueba del restablecimiento de mi memoria. Fue para mí sorpresa y
confusión grandes verme cómo estaba vestido. Hasta entonces no había
caído yo en la cuenta de que llevaba un chaquetón holgado y vetusto, en
vez de la levita que saqué de mi casa, y de que en lugar de mi sombrero
llevaba una gorra de cuartel. Pantalones y chaleco eran los mismos de mi
anterior vestimenta, y conservaba el reloj y dinero; pero mis botas de
caña, por arte de magia, se habían convertido en zapatos de orillo,
blandos y feos. Lo peor de todo fue que mi escudero no supo darme razón
de aquel cambio de ropa. ¿Me había mudado en mis horas de máquina
inconsciente, o me transformaron los demonios? Ruy no lo sabía, lo que
me probaba que él también había tenido eclipses de la memoria y del
conocimiento. La mejor prueba de que mi cerebro recobraba su normalidad,
la tuve oyendo cuanto se decía de los acontecimientos de carácter
público, y asimilándome aquellas referencias, apuntes que pronto habían
de ser páginas históricas. La lucha en las calles se había suspendido.
En la tregua fraternizaban pueblo y tropa. ¿Qué Gobierno había? Lo
ignorábamos. Sabíamos que una Junta magna tomaba sobre sí la obra de
pacificación\ldots{} Entre los nombres que oí, se estamparon en mi mente
con vigor y claridad los de Sevillano, Vega Armijo, don Joaquín Aguirre,
Fernández de los Ríos y el general San Miguel. Ya estaban en
negociaciones la Junta y Palacio\ldots{} ya se vislumbraba la paz; el
triunfo del Pueblo era evidente. Se contaban maravillas del arrojo y
constancia de los \emph{patriotas} en las barricadas de la calle de la
Montera, en la confluencia de las calles de San Miguel y Caballero de
Gracia, en las Cuatro Calles, plaza de las Cortes\ldots{} Las tropas que
Córdova tenía en la zona de Palacio no habían podido comunicarse con las
que ocupaban el Prado y Recoletos\ldots{} Entre todas las barricadas, la
más ineficaz había sido la nuestra, calle de Toledo, y conceptuándola
disparate estratégico, Gracián había mandado abandonarla. Esta noticia
me llenó de confusión. ¿Dónde me había batido yo? ¿Dónde tuvieron su
teatro mis estupendas hazañas?\ldots{} Mas ¿cómo habíamos de dilucidar
este obscuro punto, si Clío, que todo lo sabe, ignoraba en qué lugar se
habían separado de mi cuerpo mis botas y mi levita?

Muertos vi en gran número en la calle Imperial, Atocha y entrada de la
de Carretas. Heridos transportamos al Fiel Contraste; y hallándome en
esta operación, tuve el sentimiento de acompañar en sus últimas al bueno
de Erasmo Gamoneda. El pobrecito me pidió agua: se la di de un botijo
que pasaba de mano en mano y de boca en boca; bebió con ansia. Parecía
sentir alivio del escozor de sus heridas, que eran tremendas: un agujero
en la clavícula derecha; en el vacío del mismo lado, otro que le pasaba
de parte a parte; la cabeza rota, una mano casi deshecha. Mirándome
agradecido, me dijo con sencillez y satisfacción tranquila, como si se
alabara de terminar felizmente una partida de obleas: «Hemos ganado.
Bien, bien\ldots{} Milicia Nacional: bien\ldots{} Yo artillero\ldots» Y
repitiendo el \emph{bien, bien, yo artillero}, estiró piernas y brazos,
y abriendo la boca en todo su grandor, entregó el alma.

\hypertarget{xxxi}{%
\chapter{XXXI}\label{xxxi}}

Este y otros espectáculos tristes deprimieron horrorosamente mi ánimo.
Iniciado el despejo de mi entendimiento, ganaba terreno por instantes la
querencia de mi familia y el gusto de la vida normal. Pero no volvería
yo a mi casa sin resolver dos problemas de importancia: recobrar mi
ropa, y saber la suerte y paradero de \emph{Mita} y \emph{Ley}. Más
fácil era, según Ruy, lo segundo que lo primero, pues sólo Dios podía
encontrar una levita y un sombrero en aquel \emph{maremágnum} de pobreza
y confusión. En el Rastro quizás parecerían, y quién sabe si veríamos
ambas piezas, dos días después, en la hinchada persona de algún
funcionario de la flamante situación popular. No hallando a \emph{Mita}
y \emph{Ley} en la casa de los Gamonedas, desierta y abandonada, fuimos
a la \emph{cangrejería} de la plazuela de San Miguel, donde nos dijeron
que cansada \emph{Mita} de esperar a Leoncio, y medio muerta de
ansiedad, andaba en busca de él, de barricada en barricada. ¡Vaya por
Dios! Afanosos nos lanzamos mi escudero y yo a la misma caminata, y en
ella se nos pasó gran parte de la noche, sin encontrar a los salvajes.
Lo peor fue que con tanto ajetreo me sentí nuevamente amagado de mis
desórdenes cerebrales y nerviosos: yo estaba fatigadísimo. Para contener
el mal que me rondaba, y dar algún descanso a mis pobres huesos, me metí
en el desalojado almacén de paños de la calle de Toledo, ahora
convertido en cuartel general de la plebe, depósito de armas y algo que
con optimismo burlón llamábamos víveres\ldots{} Entramos. Vimos diversa
gente; hombres fatigados que no podían moverse; otros que perezosos
recogían objetos diversos para devolverlos a los hogares: botijos,
sillas, colchones. En un rincón había heridos graves, rodeados de sus
familias, que no sabían si dejarles morir allí o llevárselos a casa.
Mujeres vi en actitud estoica, mujeres desesperadas\ldots{} Mi cansancio
físico no me permitía ya ni aun ser piadoso\ldots{} Me interné por
aquellas obscuras y destartaladas estancias, y fui a parar a la más
interior, que cae sobre Cuchilleros. No podía yo con mi cuerpo ni con mi
alma; en un montón de esteras que me brindaba las blanduras de un diván,
me dejé caer, y estirándome todo lo que daban de sí brazos y piernas,
sin llegar a las medidas del camastro, me dormí profundamente.

El tiempo que duró mi sueño no puedo precisarlo\ldots{} Desperté con una
idea triste, una desfavorable opinión de mí mismo: yo era inferior, muy
inferior a toda la caterva popular entre la cual había vivido tantas
horas; yo no podía compararme a ellos, pues mis hazañas eran
fantásticas, quizás burlescas, y ellos sabían luchar y morir por un
ideal tanto más grande cuanto más nebuloso. Volvería yo a mi clase o
jerarquía social, materializada y egoísta, sin haber hecho nada fuera de
lo común, sin encontrar medio de ennoblecer mi alma con un acto hermoso
de piedad, o de justicia, o de moral grandeza\ldots{} Esta idea me
mortificaba, y también la sed: revolví mis ojos por la estancia, que
alumbraban candiles moribundos; vi a Ruy, dormido a mi lado como un
tronco; en el opuesto rincón, un hombracho, envuelto en manta gris, era
también tronco durmiente. Creyendo ver junto a éste un cántaro de agua,
me levanté para cogerlo, y no había dado dos pasos cuando entró en la
estancia un hombre, que al punto reconocí\ldots{} ¡Ay, qué miedo!: era
Bartolomé Gracián. No esperó a que yo le hablase, y reconociéndome al
punto, y llegándose a mí jovial, me dijo: «Hola, Beramendi: no creí
encontrarle aquí\ldots{} ¿Salía usted?\ldots» «No---le respondí
temblando;---iba en busca de aquel cántaro: tengo sed.» Por disimular mi
miedo, me dirigí a donde estaba el cántaro, y volviendo junto al héroe
de la plebe, con un gesto le ofrecí agua\ldots{} Me sentía mudo.

«Gracias, que aproveche. Pero ¿qué, se asusta de mí?\ldots{} Al
contrario, diviértase con lo que voy a decirle, amigo Beramendi. Es
usted de los míos\ldots{} Ha terminado la partida militar, y ahora
empiezan las amorosas partidas\ldots{} Yo soy así: salgo de una locura,
y emprendo locura mayor\ldots{} De ésta saldré tan bien como de la
otra\ldots{} ¿Qué le pasa a usted, que no dice nada? Beba si tiene
sed\ldots{} Y si quiere presenciar un grande atrevimiento de amor, más
interesante y más dramático que todos los que le conté caminando de
Vicálvaro a Vallecas, espéreme aquí un momento.»

Al decir esto, ponía la mano en los hierros de una puerta. Siguiendo con
mis ojos su mano, reconocí la puerta de la escalera que por arriba
conduce a las habitaciones donde moraba Lucila con su marido, por abajo
a la calle de Cuchilleros. Interrogado por mis ojos, que debieron de
echar lumbre, Gracián me dijo: «En el piso segundo está una mujer a
quien yo amé tres años ha; me quiso ella con locura\ldots{} El Destino
nos separó. No he vuelto a verla desde entonces. Casó Lucila con un
viejo campesino\ldots{} Ayer supe que está en Madrid y en esta
casa\ldots{} No soy quien soy si no la saco esta noche del domicilio
conyugal para llevármela al mío. Después de estos terribles combates,
¿qué puede apetecer el soldado más que descansar en un éxtasis de amor?
Marte nos irrita; Venus nos consuela\ldots{} Parece que usted no me
cree, y aun que se burla de mí. ¿Quiere que hagamos una apuesta? Lucila
no me ha visto desde aquella separación de comedia; Lucila, esta tarde,
no sabía que yo existo\ldots{} A media noche le escribí una carta, de
las que yo pongo, con el alma, toda la fascinación del mundo en pocas
letras\ldots{} Con Servanda se la mandé. ¿Sabe usted quién es Servanda?
Una mujer muy sutil que \emph{celestinea} maravillosamente\ldots{} Sé
que la carta está en su poder. Lucila no me contestó, ni hace falta su
respuesta\ldots{} ¿Qué creerá usted que puse yo en mi carta?\ldots{}
Cuatro palabras de fascinación, y pocas más diciéndole que\ldots{} Todo
ello es sencillísimo, mi querido Marqués\ldots{} diciéndole que en
cuanto deje a su marido bien dormido, pero bien dormido, salga al
pasillo de su casa\ldots{} Allí me espera\ldots{} Entro yo\ldots{}

---Pero usted no entrará---le dije poniendo mi mano sobre la suya, que
tocaba la puerta.

---Tengo la llave de aquí\ldots{} y la de arriba también\ldots{} véalas.
Servanda me las ha dado.

---Pero Lucila no se prestará, no, a ese ardid impropio de un
caballero\ldots{} Lucila es honrada.

---Yo subo; yo entro\ldots{}

---Y a encontrarle saldrá don José Halconero, armado hasta los dientes.

---Ríase usted de Halconeros y de dientes armados. Es usted un candidote
que no conoce el mundo misterioso de la infidelidad conyugal, ni los
impulsos de una mujer que, enamorada ciegamente una vez, no deja en
ningún caso de acudir al reclamo de su ilusión.

---No acudirá, no acudirá, Gracián---afirmé yo, libre ya mi alma de todo
miedo.

---Hagamos la grande apuesta. Usted aquí se queda en expectación de mi
aventura. Si al cuarto de hora no me ve pasar por aquí con Lucila,
pierdo\ldots{} Estipulemos lo que se ha de perder o ganar, según falle o
no la empresa.

---Yo no apuesto, señor Gracián---respondí sintiéndome todo
entereza.---Yo no hago más que decir a usted que no subirá\ldots{}
porque no debe subir, porque yo no debo permitirlo; más claro, porque yo
le prohíbo que suba.

---No contaba yo con este guardián caballeresco---dijo Gracián echándose
atrás;---pero va usted a ver cómo me sacudo yo a los caballeros de la
\emph{Guarda y Vela.»}

Al movimiento que hizo para echarme sus crispadas manos al pescuezo, me
anticipé yo levantando con poderoso impulso y coraje el cántaro mediado
de agua; y ello fue tan rápido, que al tiempo que sus dedos me tocaban,
se estrelló el cántaro en su cabeza, y los cascos y el agua envolvieron
su rostro, le cegaron\ldots{} En el mismo instante oí una voz que
gritaba: «¡Mátele, señor, mátele!» Y el hombracho aquel que dormía se
llegó a mí y puso en mi mano una pistola\ldots{} Antes que Gracián,
rehecho del golpe y mojadura, volviera sobre mí con furiosa exclamación
de cólera, la bala se le metió en el cráneo, y de golpe toda su
arrogancia y toda su maldad cayeron en los profundos abismos.

Segundos duró lo que cuento. El hombre que me había dado el arma, me
cogió del brazo, y sin dejarme ni tan siquiera mirar a la víctima, me
llevó fuera diciéndome: «Señor, no le tenga lástima\ldots{} Vámonos de
aquí. ¿No me conoce? Soy Hermosilla, el fabricante de zorros y
plumeros\ldots{} Almendro, 14\ldots{} Ha quitado el señor de en medio la
mayor calamidad del mundo. ¡Vive Dios que ha sido grande hazaña!\ldots{}
Ese tunante me \emph{perdió} a mi hija mayor, la Rafaelita; después a mi
segunda, la \emph{Generosa}. ¡Qué dolor! Las dos andan por esas
calles\ldots»

~

¿Dónde estaba yo en la mañanita del 20, con Hermosilla? En una
sombrerería de la Concepción Jerónima, buscando una prenda decente,
cobertera de mi cráneo, para poder entrar en mi casa con el decoro
propio de la clase a que pertenezco. Mi diligente escudero, a quien
había mandado por cigarros, vino desolado a decirme: «Ahí están,
señor\ldots{} Míreles\ldots{} \emph{Mita} y mi hermano Leoncio\ldots{}
Se van, se van de Madrid.»

Salí, y en la misma puerta de la tienda me vi cogido de las manos por
\emph{Mita}, que, con premioso acento de despedida, me dijo: «Nos vamos,
Pepe\ldots{} adiós. Ya hay Gobierno; otra vez hay leyes: ya no podemos
seguir en este pueblo maldito.

---¿Y \emph{Ley}?

---Mírale allí, metiendo nuestros baulitos en la tartana\ldots{} «Nos
vamos al campo, al sol\ldots{} ¡Salvajes otra vez, hasta que Dios
quiera!\ldots»

Corrí a donde estaba el coche, y apenas tuve tiempo de despedir a mis
buenos amigos con toda la efusión de mi cariñosa amistad y sinceros
ofrecimientos de protección. El coche partió. ¡Cuándo volveríamos a
vernos! Díjome Ruy que se iban a la Villa del Prado, donde vivirían al
amparo de Lucila.

«¿Y tu hermana, y el bendito señor de Halconero, a quien estimo mucho
sin tener el honor de conocerle?

---Pues han salido hace un cuarto de hora en otra tartana que va
delante.»

Se iban a la paz y a las alegrías del campo, y aquí quedaba Madrid con
su corte, su política y el eterno rodar de los artificios, que se
suceden mudándose, y se mudan para ser siempre los mismos\ldots{} Y yo a
mi casa: ya era irresistible mi deseo de ver a mi mujer y a mi
hijo\ldots{} No me faltaba más que buscar una levita semejante, si no
igual, a la que perdí, pues no me resignaba, no, al deplorable efecto de
mi aparición con la facha de \emph{jamancio crúo}. Y me faltaba también
discurrir la ingeniosa mentira con que debía justificar mi ausencia de
casa en las turbulentas noches y días de la Revolución. Pensando en ello
estaba, y ocupado además en la diligencia de buscar la levosa, cuando vi
pasar por la calle de Toledo abajo al general San Miguel, a caballo, con
abigarrado séquito de patriotas y militares, también a caballo. Vestía
don Evaristo de paisano, con fajín, y a su paso le saludaba la multitud
con aclamaciones de respeto y júbilo. Era el pacificador, la
personificación del feliz consorcio de Pueblo y Ejército. A poco de
verle pasar, una ideíta que yo buscaba entró gozosa en mi mente. «A casa
mandaré a Ruy---me dije---para que prepare la vuelta del prófugo con un
lindo embuste. Dirá que me cogió el general San Miguel el día 18 para
que le ayudara en sus trabajos de pacificación\ldots{} que no pude
zafarme del compromiso, ni de la encerrona en patrióticas
asambleas\ldots{} No, no: esto no lo creerán\ldots{} Tengo que inventar
otra cosa, fabricar mi novela en históricos moldes\ldots{} Diré que
Córdova me llamó a Palacio; que luego se me encargó una misión muy
delicada cerca de la Junta que se reunía en casa del señor Sevillano;
que fui detenido por un grupo de revolucionarios ardientes; que me
encerraron en la Posada del Peine\ldots{} en el palacio del
Nuncio\ldots{} en las casas de Porras\ldots{} averígüelo Vargas\ldots{}
guardándome prisionero con exquisitas consideraciones y esmerado trato
de aposento y boca\ldots»

Esto contaría yo \emph{mutatis mutandis}, y una vez salvado el decoro de
mi presentación, a mi mujer le contaría la verdad escueta, sin omisión
ni aditamento, historiador sincero y leal de una de las páginas más
interesantes y dolorosas de mi pobre existencia\ldots{} Así lo hice. No
se cuidaba mi mujer más que de llevarme al reposo y a la franca sedación
de mi mal, y lo consiguió con su dulzura. A los trágicos y cómicos
lances que le referí, y a mis variados cuentos y descripciones, puse un
juicio sintético que aquí reproduzco como término de esta parte de mis
Memorias. Ved aquí el juicio y la fría opinión, una vez pasado el hervor
revolucionario y entibiadas las pasiones que del corazón de los demás
pasaban al mío: Todo es pequeño, en conjunto. Relativa grandeza o
mediana talla veo en la obra del pueblo sacrificándose por renovar el
ambiente político de los señoretes y cacicones que vivimos en alta
esfera. Menguados son los políticos, y no muy grandes los militares que
han movido este zipizape. Pobre y casera es esta revolución, que no
mudará más que los externos chirimbolos de la existencia, y sólo pondrá
la mano en el figurón nacional, en el cartón de su rostro, en sus
afeites y postizos, sin atreverse a tocar ni con un dedo la figura real
que el maniquí representa y suple a los ojos de la ciega muchedumbre. De
mezquina talla es asimismo mi hazaña, la rápida muerte que di a Gracián,
en defensa de la paz obscura de una mujer\ldots{} única paz que en lo
humano existe\ldots{} Todo es pequeño, todo; sólo son grandes
\emph{Mita} y \emph{Ley}.

Mi mujer no me deja continuar mis Memorias, y por culpa de su cariñosa
prohibición, en el tintero se queda la trágica muerte de Chico, y la
entrada de Espartero, explosión grande del entusiasmo inocente y de la
candidez revolucionaria. Otros contarán estos hechos, que yo no
presencié, porque mi esposa me aísla de lo que llamaremos \emph{emoción
pública}\ldots{} Desde mi doméstico retiro, atendiendo a mi salud, que
lentamente recobro, y privado de la compañía de \emph{Ruy} y de
\emph{Sebo} (que ahora goza un lucido empleo en el Gobierno Civil), sigo
con la imaginación los varios acontecimientos, y ya sean dramáticos, ya
de risa, les pongo por comentario un grito que me sale del corazón.
Siempre que mi mujer me da cuenta de algo que merece lugar en la
Historia, yo digo: «¡Viva \emph{Mita!}\ldots{} ¡Viva \emph{Ley!»}

\flushright{Santander, Septiembre 1903.—Madrid, Marzo 1904.}

~

\bigskip
\bigskip
\begin{center}
\textsc{fin de la revolución de julio}
\end{center}

\end{document}
